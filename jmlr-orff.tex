\documentclass[twoside,11pt]{article}

% Any additional packages needed should be included after jmlr2e.
% Note that jmlr2e.sty includes epsfig, amssymb, natbib and graphicx,
% and defines many common macros, such as 'proof' and 'example'.
%
% It also sets the bibliographystyle to plainnat; for more information on
% natbib citation styles, see the natbib documentation, a copy of which
% is archived at http://www.jmlr.org/format/natbib.pdf
\usepackage[abbrvbib]{jmlr2e}
\usepackage{pythontex}
\usepackage{acro}
\usepackage{ragged2e}
\usepackage{cmap} % copy-paste pdf
\usepackage{microtype}
\usepackage{dirtytalk}
\usepackage{pdflscape}
\usepackage{afterpage}
\usepackage{fancyvrb}

\VerbatimFootnotes

\usepackage[export]{adjustbox}
\usepackage{tikz}
\usetikzlibrary{cd}
\usetikzlibrary{matrix}
\usetikzlibrary{calc}
\makeatletter
\tikzset{%
  column sep/.code=\def\pgfmatrixcolumnsep{\pgf@matrix@xscale*(#1)},
  row sep/.code   =\def\pgfmatrixrowsep{\pgf@matrix@yscale*(#1)},
  matrix xscale/.code=%
    \pgfmathsetmacro\pgf@matrix@xscale{\pgf@matrix@xscale*(#1)},
  matrix yscale/.code=%
    \pgfmathsetmacro\pgf@matrix@yscale{\pgf@matrix@yscale*(#1)},
  matrix scale/.style={/tikz/matrix xscale={#1},/tikz/matrix yscale={#1}}}
\def\pgf@matrix@xscale{1}
\def\pgf@matrix@yscale{1}
\makeatother

\usepackage[american]{babel}

\usepackage{braket}
\usepackage{colonequals}
\usepackage{thmtools}
\usepackage{nameref, cleveref}
\usepackage{mathtools}
\usepackage[algo2e,ruled,linesnumbered]{algorithm2e}
\usepackage{breqn}
\usepackage{commands}

\makeatletter
\let\cref@old@eq@setnumberOld\eq@setnumber
\def\eq@setnumber{%
\cref@old@eq@setnumberOld%
\cref@constructprefix{equation}{\cref@result}%
\protected@xdef\cref@currentlabel{%
[equation][\arabic{equation}][\cref@result]\p@equation\eq@number}}
\makeatother

\usepackage{multicol}
\usepackage{multirow}
\usepackage{booktabs}
\usepackage{tabularx}
\usepackage{ltablex}
\usepackage{threeparttablex}
\setlength{\extrarowheight}{3pt} % Increase table row height
\newcommand{\tableheadline}[1]{\multicolumn{1}{c}{\spacedlowsmallcaps{#1}}}
\newcommand{\myfloatalign}{\centering} % To be used with each float for alignment
\newcommand{\theHalgorithm}{\arabic{algorithm}}
\usepackage{caption}

\crefname{algocfline}{Algorithm}{Algorithms}
\Crefname{algocfline}{Algorithm}{Algorithms}
\crefname{algocf}{Algorithm}{Algorithm}
\Crefname{algocf}{Algorithm}{Algorithms}
\crefname{table}{Table}{Tables}
\Crefname{table}{Table}{Tables}
\crefname{figure}{Figure}{Figures}
\Crefname{figure}{Figure}{Figures}
\crefname{chapter}{Chapter}{Chapters}
\Crefname{chapter}{Chapter}{Chapters}
\crefname{section}{Section}{Sections}
\Crefname{section}{Section}{Sections}
\crefname{subsection}{Subsection}{Subsections}
\Crefname{subsection}{Subsection}{Subsections}
\crefname{theorem}{Theorem}{Theorems}
\Crefname{theorem}{Theorem}{Theorems}
\crefname{proposition}{Proposition}{Propositions}
\Crefname{proposition}{Proposition}{Propositions}
\crefname{lemma}{Lemma}{Lemmas}
\Crefname{lemma}{Lemma}{Lemmas}
\crefname{corollary}{Corollary}{Corollaries}
\Crefname{corollary}{corollary}{Corollaries}
\crefname{equation}{Equation}{Equations}
\Crefname{equation}{Equation}{Equations}
\crefname{remark}{Remark}{Remarks}
\Crefname{remark}{Remark}{Remarks}
\crefname{example}{Example}{Examples}
\Crefname{example}{Example}{Examples}
\creflabelformat{equation}{#2\textup{#1}#3}

% Heading arguments are {volume}{year}{pages}{submitted}{published}
%     {author-full-names}

\jmlrheading{1}{2000}{1--48}{4/00}{10/00}{brault00a}
    {Romain Brault and Florence d'Alch\'e-Buc}

% Short headings should be running head and authors last names

\ShortHeadings{RFFs for OVKs}{Brault and d'Alch\'e-Buc}
\firstpageno{1}

\sloppy

\begin{document}

\title{Random Fourier Features for Operator-Valued Kernels}

\author{\name{}Brault Romain
       \email~romain.brault@telecom-paristech.fr \\
       \addr~LTCI\\
       T\'el\'ecom ParisTech\\
       Paris, 46 rue Barrault, France \\
       Universit\'e Paris-Saclay \\
       \AND%
       \name{}Florence d'Alch\'e-Buc
       \email~florence.dalche@telecom-paristech.fr \\
       \addr~LTCI\\
       T\'el\'ecom ParisTech\\
       Paris, 46 rue Barrault, France \\
       Universit\'e Paris-Saclay}

\editor{Francis Bach}

\maketitle

\begin{abstract}%   <- trailing '%' for backward compatibility of .sty file
    Many problems in Machine Learning can be cast into
    vector-valued functions approximation. Operator-Valued Kernels
    \emph{\acl{OVK}s} and vector-valued Reproducing Kernel Hilbert Spaces
    provide a theoretical and practical framework to address that issue,
    extending nicely the well-known setting of scalar-valued kernels.
    However large scale applications are usually not affordable with these
    tools that require an important computational power along with a large
    memory capacity. In this paper, we propose and study scalable methods
    to perform regression with \emph{\acl{OVK}s}. To achieve this goal, we
    extend Random Fourier Features, an approximation technique originally
    introduced for scalar-valued kernels, to \emph{\acl{OVK}s}. The idea is
    to take advantage of an approximated operator-valued feature map in
    order to come up with a linear model in a finite-dimensional space.
\end{abstract}

\begin{keywords}
    Random Fourier Feature, Operator-Valued Kernel
\end{keywords}

%%%%%%%%%%%%%%%%%%%%%%%%%%%%%%%%%%%%%%%%%%%%%%%%%%%%%%%%%%%%%%%%%%%%%%%%%%%%%%%

\section{Introduction}
This paper is dedicated to the definition of a general and flexible approach to
learn vector-valued functions together with an efficient implementation of the
learning algorithms. To achieve this goal, we study shallow architectures,
namely the product of a (nonlinear) operator-valued feature
$\widetilde{\Phi}(x)$ and a parameter vector $\theta$ such that
$\widetilde{f}(x) = {\widetilde{\Phi}(x)}^* \theta$, and combine two appealing
methodologies: Operator-Valued Kernel Regression and Random Fourier Features.
\paragraph{}
Operator-Valued Kernels \citep{Micchelli2005,Carmeli2010,Kadri_aistat10,
Brouard2011,Alvarez2012} extend the classic scalar-valued kernels to functions
with values in some \emph{output} Hilbert space. As in the scalar case,
\acfp{OVK} are used to build Reproducing Kernel Hilbert Spaces (\acs{RKHS}) in
which representer theorems apply as for ridge regression or other appropriate
loss functional. In these cases, learning a model in the \acs{RKHS} boils down
to learning a function of the form $f(x)=\sum_{i=1}^N K(x,x_i)\alpha_i$ where
$x_1, \ldots, x_N$ are the training input data and each $\alpha_i, i=1, \ldots,
N$ is a vector of the output space $\mathcal{Y}$, and each $K(x,x_i)$ is an
operator on $\mathcal{Y}$.
\paragraph{}
However, \acsp{OVK} suffer from the same drawbacks as classic
(sca\-lar-va\-lued) kernel machines: they scale poorly to large datasets
because they are exceedingly demanding in terms of memory and computations. We
propose to approximate OVKs by extending a methodology called \acfp{RFF}
\citep{Rahimi2007, Le2013, Yang2015, sriper2015, Bach2015, sutherland2015,
rudi2016generalization} so far developed to speed up scalar-valued kernel
machines. The \acs{RFF} approach linearizes a shift-invariant kernel model by
generating explicitly an approximated feature map $\tilde{\phi}$. \acsp{RFF}
has been shown to be efficient on large datasets \citep{rudi2016generalization}
and has been further improved by efficient matrix computations such as
\citep[``FastFood'']{Le2013} and \citep[``SORF'']{felix2016orthogonal}, which
are considered as the best large scale implementations of kernel methods, along
with Nystr\"om approaches proposed in \citet{drineas2005nystrom}. Moreover
thanks to \acsp{RFF}, kernel methods have been proved to be competitive with
deep architectures \citep{lu2014scale, dai2014scalable, yang2015deep}.

\subsection{Outline and contributions}
The paper is structured as follow. In \cref{sec:background} we recall briefly
how to obtain \acp{RFF} for scalar-valued kernels and list the state of the
art implementation of \acp{RFF} for large scale kernel learning. Then we define
properly \aclp{OVK}, give some important theorems and properties used
throughout this paper before given a non exhaustive list of problem tackled
with \acsp{OVK}.
\paragraph{}
Then we move on to our contributions. In \cref{sec:ORFF_construction} we
propose an \acs{RFF} construction from $\mathcal{Y}$-Mercer shift invariant
\acs{OVK} that we call \acf{ORFF}. Then we study the structure of a random
feature corresponding to an \acs{OVK} (without having to specify the target
kernel). Eventually we use the framework used to construct \acsp{ORFF} to study 
the regularization properties of \acsp{OVK} in terms of \acl{FT}.
\paragraph{}
In \cref{sec:consistency_of_the_ORFF_estimator} we assess theoretically the
quality of our \acs{ORFF}: we show that the stochastic \acs{ORFF} estimator
converges with high probability toward the target kernel and derive convergence
rates. We also give a bound on the variance of the approximated \acs{OVK}
constructed from the corresponding \acs{ORFF}.
\paragraph{}
In \cref{sec:learning_with_operator-valued_random-fourier_features} we focus on
Ridge regression with \acsp{OVK}. First we study the relationship between
finding a minimizer in the \acs{vv-RKHS} induce by a given \acs{OVK} and the
feature induced by the corresponding \acs{ORFF}. Then we define a gradient
based algorithm to tackle Ridge regression with \acs{ORFF}, show how to
obtain an efficient implementation and study its complexity.
\paragraph{}
Eventually we end this paper by some numerical experiments in
\cref{sec:num_exp} on toy and real datasets before giving a general conlusion
in  \cref{sec:conclusion}.
%%%%%%%%%%%%%%%%%%%%%%%%%%%%%%%%%%%%%%%%%%%%%%%%%%%%%%%%%%%%%%%%%%%%%%%%%%%%%%%

\section{Background}
\label{sec:background}
%In this section we summarize briefly important notions used throughout this
%document. It is mainly based on books and lecture notes of
%\citet{kurdila2006convex,cotaescu2016elements}.

%\subsection{Notations}
%\label{sec:notations}
%We note $\mathbb{K}$ any Abelian field and call its elements scalars.
%$\mathbb{R}$ is the Abelian field of real numbers and $\mathbb{C}$ is the
%Abelian field of complex numbers. The unit pure imaginary number
%$\sqrt{-1}\in\mathbb{C}$ is denoted $\iu$ and the Euler constant
%$\exp(1)\in\mathbb{R}$ is denoted $\ec$.  $\mathbb{N}$ represents the set of
%natural numbers and $\mathbb{N}_n$, $n\in\mathbb{N}$ the set of natural numbers
%smaller or equal to $n$. For any space $\mathcal{S}$, $\mathcal{S}^d$,
%$d\in\mathbb{N}$ represents the Cartesian product space $\mathcal{S}^d =
%\mathcal{S}\times\cdots\times\mathcal{S}$. For any two algebraic structures
%$\mathcal{S}$ and $\mathcal{S}'$ we write $\mathcal{S}\cong\mathcal{S}'$ if
%there exist an isomorphism between these two structures. If $a+\iu b = x \in
%\mathbb{C}$ then $\conj{x}=a-\iu b\in\mathbb{C}$ denotes the complex conjugate.
%By extension if $x\in\mathbb{R}$, $\conj{x}=x\in\mathbb{R}$.

%\subsubsection{Topology and continuity}
%In order to define a proper notion of continuity, we focus on topological
%spaces. A topological space is a pair of sets $(\mathcal{X},\mathcal{T}_x)$
%where $\mathcal{X}$ describes the points considered, and $\mathcal{T}_x$
%describes the possible neighbourhoods. The standard axioms of topology suppose
%that $\mathcal{T}_x\subseteq{\mathcal{P}(\mathcal{X})}$ is a collection of
%subsets of $\mathcal{X}$ such that the empty set and $\mathcal{X}$ itself
%belongs to $\mathcal{T}_x$, any (finite or infinite) union of members of
%$\mathcal{T}_x$ still belongs to $\mathcal{T}_x$ and the intersection of any
%finite number of members of $\mathcal{T}_x$ still belongs to $\mathcal{T}_x$.
%The elements of $\mathcal{T}_x$ are called open sets and the collection
%$\mathcal{T}_x$ is a topology on $\mathcal{X}$. If
%$(\mathcal{X},\mathcal{T}_x)$ and $(\mathcal{Y},\mathcal{T}_y)$ are topological
%spaces, a function $f$ is said to be continuous if for every open set
%$\mathcal{V}\in \mathcal{T}_y$, the inverse image $f^{-1}(\mathcal{V}) = \Set{
%x \in \mathcal{X} | f ( x ) \in \mathcal{V} }$ is an open subset of
%$\mathcal{T}_x$. Since the notion of continuity depends on open sets, it
%depends on the topology of the spaces $\mathcal{X}$ and $\mathcal{Y}$.
%\paragraph{}
%If $(\mathcal{X},\mathcal{T}_{x})$ is a topological space and $x$ is a point in
%$\mathcal{X}$, a neighbourhood of $x$ is a subset $\mathcal{V}$ of
%$\mathcal{X}$ that includes an \emph{open} set $\mathcal{U}$ containing $x$. A
%topological space $\mathcal{X}$ is said to be Hausdorff (T2) when all distinct
%points in $\mathcal{X}$ are pairwise neighbourhood-separable. \acs{ie}~if there
%exists a neighbourhood $\mathcal{U}$ of $x$ and a neighbourhood $\mathcal{V}$
%of $y$ such that $\mathcal{U}$ and $\mathcal{V}$ are disjoint. It implies the
%uniqueness of limits of sequences and existence of nets used throughout this
%thesis. Therefore in the whole document we always assume that a topological
%space $\mathcal{X}$ is Haussdorff.
%\paragraph{}
%A topological space is said to be second countable if it has a countable base.
%Every second-countable space is separable. The reverse implications do not
%hold). A space is metrisable if and only if it is second countable.
%\paragraph{}
%A topological space is said to be separable if there exists a sequence
%$(x_n)_{n\in\mathbb{N}^*}$ of elements of $\mathcal{X}$ such that every
%nonempty open subsets of the space contains at least one element of the
%sequence. Separability plays an important role in numerical analysis because
%many theorems have only constructive proofs for separable spaces. Such
%constructive proofs can be turned into algorithms which is the primary goal of
%this work. In this document we also assume that any topological space is
%separable if there is no specific mention of the contrary. Moreover we recall
%that a Hilbert space is separable if and only if it has a countable orthonormal
%basis (Hence separable Hilbert spaces are second countable). Hence an operator
%between two separable Hilbert spaces can be written as an infinite dimensional
%matrix. In some cases we also introduce \emph{Polish spaces} which are
%separable topological spaces $\mathcal{X}$ that have at least one metric $d$
%such that $(\mathcal{X}, d)$ is complete. Then $d$ induces the topology
%$\mathcal{T}_x$ of $\mathcal{X}$. As metrisable spaces, Polish spaces are
%always second countable. Moreover every second countable locally compact
%Hausdorff space is a Polish space and every separable Banach space is a Polish
%space.
%\paragraph{}
%If $\mathcal{X}$ and $\mathcal{Y}$ are two topological spaces, we denote by
%$\mathcal{F}(\mathcal{X};\mathcal{Y})$ the topological vector space of
%functions $f:\mathcal{X}\to\mathcal{Y}$ and
%$\mathcal{C}(\mathcal{X};\mathcal{Y}) \subset
%\mathcal{F}(\mathcal{X};\mathcal{Y})$ the subspace of continuous functions,
%endowed with the product topology (topology of pointwise convergence).
%\subsubsection{Measure theory}
%A $\sigma$-algebra on $\mathcal{X}$ is a set
%$\mathcal{M}\subseteq\mathcal{P}(\mathcal{X})$ of subsets of $\mathcal{X}$,
%containing the empty set, which is closed under taking complements and
%countable unions. A pair $(\mathcal{X},\mathcal{M})$ where $\mathcal{X}$ is a
%set and $\mathcal{M}$ is a $\sigma$-algebra is called a measure space. The
%Borel $\sigma$-algebra $\mathcal{B}(\mathcal{X})$ is a $\sigma$-algebra
%generated by the open sets of $\mathcal{X}$. A measure on a measurable space
%$(\mathcal{X},\mathcal{B}(\mathcal{X}))$ is a map $\mu: \mathcal{B}(X) \to
%\overline{\mathbb{R}}_+$ which is zero on the empty set and countably additive,
%\acs{ie}~for any subset $(\mathcal{Z}_n)_{n\in\mathbb{N}}$ is a sequence of
%pairwise disjoint measurable sets,
%\begin{dmath*}
    %\mu\left(\bigcup_{n\in\mathbb{N}}\mathcal{Z}_n\right) =
    %\sum_{n\in\mathbb{N}}\mu(\mathcal{Z}_n).
%\end{dmath*}
%We note $\mathcal{N}(m, \sigma)$ the Gaussian distribution with
%mean $m\in\mathbb{R}$ and variance $\sigma^2\in\mathbb{R}$. $\mathcal{U}(a, b)$
%is the uniform distribution with support $(a, b)$ and $\mathcal{S}(m, \sigma)$
%is the hyperbolic secant distribution with mean $m$ and variance $\sigma^2$.

%\subsubsection{Vector spaces, linear operators and matrices}
%Given any vector space $\mathcal{H}$ over an Abelian field $\mathbb{K}$, the
%(continuous) dual space $\mathcal{H}^\adjoint$
%is defined as the set of all \emph{continuous} linear functionals $x^*:
%\mathcal{H} \to \mathbb{K}$.
%\paragraph{}
%Let $\mathcal{H}_1$ and $\mathcal{H}_2$ be two vector spaces.  We call operator
%any linear function from $\mathcal{H}_1$ to $\mathcal{H}_2$.
%We set $\mathcal{L}(\mathcal{H}_1;\mathcal{H}_2)$ to be the space of
%\emph{continuous} (linear) operators from $\mathcal{H}_1$ to $\mathcal{H}_2$.
%The vector space $\mathcal{H}_1$ is called the domain, noted $\Dom$ and
%$\mathcal{H}_2$ the codomain. We use the shortcut notation
%$\mathcal{L}(\mathcal{H})=\mathcal{L}(\mathcal{H}; \mathcal{H})$.  The
%transpose (or dual) of an operator $W:~\mathcal{H}_1\to\mathcal{H}_2$ is
%defined as $W^\transpose :\mathcal{H}_2^\adjoint \to \mathcal{H}_1^\adjoint$
%such that $W^\transpose :x^\adjoint\mapsto x^\adjoint(W)$.
%\paragraph{}
%Let $\mathcal{H}_1$ and $\mathcal{H}_2$ be two Hilbert spaces. The adjoint of
%an operator $W:\mathcal{H}_1\to\mathcal{H}_2$ is the unique mapping
%$W^\adjoint:\mathcal{H}_2\to\mathcal{H}_1$ such that $\inner{W^\adjoint x,
%z}_{\mathcal{H}_1}=\inner{x, Wz}_{\mathcal{H}_2}$ for all
%$x\in\Dom(W^\adjoint)$, $z\in\Dom(W)$. Its existence is guaranteed by Riesz's
%representation theorem. An operator
%$W:\Dom(W)\subseteq\mathcal{H}\to\mathcal{H}$ is said to be symmetric when
%$W^\adjoint=W$, and self-adjoint when $W$ is bounded, symmetric,
%$\Dom(W^\adjoint) = \Dom(W)$ and $\Dom(W)$ is dense in $\mathcal{H}$. If $W$ is
%bounded, symmetric and $\Dom(W)=\mathcal{H}$ then $W$ is self-adjoint.
%Interestingly if $\mathcal{H}_1$ and $\mathcal{H}_2$ are normed vector spaces,
%they can be viewed as topological vector spaces, and the notion of continuity
%coincides with that of boundedness. We recall that the induced norm of a linear
%operator is given by
%\begin{dmath*}
    %\norm{W}_{\mathcal{H}_1,\mathcal{H}_2} = \sup_{x\neq 0}
    %\frac{\norm{W x}_{\mathcal{H}_2}}{\norm{x}_{\mathcal{H}_1}}.
%\end{dmath*}
%If $W\in\mathcal{L}(\mathcal{H}_1, \mathcal{H}_2)$
%\begin{dmath*}
    %\Ker W=\Set{x\in\Dom(W) | W x = 0}
%\end{dmath*}
%denotes the kernel (nullspace), which is a vector subspace of the domain and
%\begin{dmath*}
    %\Ima W = \Set{y\in\mathcal{H}_2 | y =Wx,\enskip x \in \Dom(W)}
%\end{dmath*}
%the image (range) which is a vector subspace of the codomain $\mathcal{H}_2$.
%\paragraph{}
%If $\mathcal{H}$ is an Hilbert space on a field $\mathbb{K}$ we denote its
%scalar product by $\inner{\cdot,\cdot}_{\mathcal{H}}$ and its norm by
%$\norm{\cdot}_{\mathcal{H}}$. When the base field of $\mathcal{H}$ is
%$\mathbb{R}$, $\inner{\cdot,\cdot}_{\mathcal{H}}$ is a \emph{bilinear} form.
%When the base field of $\mathcal{H}$ is $\mathbb{C}$,
%$\inner{\cdot,\cdot}_{\mathcal{H}}$ is a \emph{sesquilinear} form.
%Let $\mathcal{H}$ be a \emph{separable} Hilbert space and let
%$(e_i)_{i\in\mathbb{N}^*}$ be a basis of $\mathcal{H}$. We call
%$(e_i^\adjoint)_{i\in\mathbb{N}^*}$ the dual basis of $\mathcal{H}$, the basis
%of $\mathcal{H}^\adjoint$ such that for all $i$, $j\in\mathbb{N}^*$,
%$e_i^\adjoint(e_j)=\inner{e_i, e_j}_{\mathcal{H}}=\delta_{ij}$. In the whole
%document we consider that $\mathcal{H}^\adjoint$ is always equipped with the
%dual basis of $\mathcal{H}$.  For a vector $x\in\mathcal{H}$ with a basis
%$(e_i)_{i\in\mathbb{N}^*}$ we write $x_i=e_i^\adjoint(x)$. For a linear
%operator $W:\mathcal{H}_1\to\mathcal{H}_2$ where $\mathcal{H}_1$ and
%$\mathcal{H}_2$ are Hilbert spaces with respective basis
%$(e_i)_{i\in\mathbb{N}^*}$ and $(e'_j)_{j\in\mathbb{N}^*}$, we note $W_i=We_i$
%and $W_{ij}=e_j^\adjoint(We_i)$. Eventually given two separable Hilbert spaces
%$\mathcal{H}_1$ and $\mathcal{H}_2$, an operator
%$W:\mathcal{H}_1\to\mathcal{H}_2$, $(e_i)_{i\in\mathbb{N}^*}$ a basis of
%$\mathcal{H}_1$ and $(e'_i)_{i\in\mathbb{N}^*}$ a basis of $\mathcal{H}_2$ we
%have
%\begin{dmath*}
    %(W^\transpose)_{ij}=e_j^{\adjoint\adjoint}W^\transpose
    %{e'}_i^\adjoint\hiderel{=} e_j^{\adjoint\adjoint}{e'}_i^\adjoint W
    %\hiderel{=} {e'}_i^\adjoint W e_j \hiderel{=} W_{ji}.
%\end{dmath*}
%\paragraph{}
%We call matrix $M$ of size $(m,n)\in\mathbb{N}^2$ on an Abelian field
%$\mathbb{K}$ a collection of elements $M=(m_{ij})_{1\le i\le m, 1\le j \le n}$,
%$m_{ij}\in\mathbb{K}$. We note $\mathcal{M}_{m,n}(\mathbb{K})$ the vector space
%of all matrices. If $\mathcal{H}_1$ and $\mathcal{H}_2$ are two separable
%Hilbert spaces on an Abelian field $\mathbb{K}$, any linear operator
%$L\in\mathcal{L}(\mathcal{H}_1;\mathcal{H}_2)$ can be viewed as a (potentially
%infinite) matrix. Let $n=\dim(\mathcal{H}_1)$, $m=\dim(\mathcal{H}_2)$ and let
%$B=(e_i)_{i=1}^{n}$ and $C=(e'_i)_{i=1}^{m}$ be the respective bases of
%$\mathcal{H}_1$ and $\mathcal{H}_2$. We note $\text{mat}_{B, C}:
%\mathcal{L}(\mathcal{H}_1;\mathcal{H}_2) \to \mathcal{M}_{m,n}(\mathbb{K})$
%such that $M=\text{mat}_{B, C}(L)=({e'}_j^\adjoint L e_i)_{1\le i\le n, 1\le j
%\le m}\in\mathcal{M}_{m,n}(\mathbb{K})$. Let
%$M_1\in\mathcal{M}_{m,n}(\mathbb{K})$ and
%$M_2\in\mathcal{M}_{n,l}(\mathbb{K})$. The product between two matrices is
%written $M_1M_2\in\mathcal{M}_{m,l}(\mathbb{K})$ and obey $(M_1M_2)_{ij} =
%\sum_{k=1}^n M_{ik}M_{kj}$. Given two linear operator
%$L_1\in\mathcal{L}(\mathcal{H}_1;\mathcal{H}_2)$ and
%$L2\in\mathcal{L}(\mathcal{H}_2;\mathcal{H}_3)$ we have
%$L_1L_2\in\mathcal{L}(\mathcal{H}_1;\mathcal{H}_3)$ and i
%\begin{dmath*}
    %\text{mat}_{B, D}(L_1L_2)=\text{mat}_{B, C}(L_1)\text{mat}_{C, D}(L_2).
%\end{dmath*}
%The operator $\text{mat}_{B, C}$ is a vector space isomorphism allowing us to
%identify $\mathcal{L}(\mathcal{H}_1;\mathcal{H}_2)$ with
%$\mathcal{M}_{mn}(\mathbb{K})$ where $n=\dim(\mathcal{H}_1)$ and
%$m=\dim(\mathcal{H}_2)$.
Notations used throughout this paper are summarized in
\cref{table:notations1}.
\begin{table}
    \centering
    \caption{Mathematical symbols and their signification (part 1).
    \label{table:notations1}}
    \begin{tabularx}{\textwidth}{cX}
        \toprule
            Symbol & \multicolumn{1}{c}{Meaning} \\
        \cmidrule{1-2}
        \endhead
            %$\colonequals$ & Equal by definition. \\
            %$\mathbb{N}$ & The semi-group of natural numbers. \\
            %$\mathbb{K}$ & Any non-discrete Abelian field endowed with an
            %absolute value. Elements of $\mathbb{K}$ are called scalars. \\
            %$\mathbb{R}$ & The Abelian field of real numbers. \\
            %$\mathbb{C}$ & The Abelian field of complex numbers. \\
            %$\mathbb{U}$ & The circle group of complex numbers with unit
            %module. \\
            %$\iu \in\mathbb{C}$ & Unit pure imaginary number
            %$\iu^2\colonequals-1$.  \\
            %$\ec \in\mathbb{R}$ & Euler constant. \\
            $e \in \mathcal{X}$ &  The neutral element of the group
            $\mathcal{X}$. \\
            $\delta_{ij}$ & Kronecker delta function. $\delta_{ij}=0$ if $i
            \neq j$, $1$ otherwise. \\
            %$\inner{\cdot,\cdot}_2$ & Euclidean inner product. \\
            %$\norm{\cdot}_2$ & Euclidean norm. \\
            %$\mathcal{X}$ & Input space. \\
            $\dual{\mathcal{X}}$ & The Pontryagin dual of $\mathcal{X}$ when
            $\mathcal{X}$ is a \acs{LCA} group. \\
            %$\mathcal{Y}$ & Output space (Hilbert space). \\
            %$\mathcal{H}$ & Feature space (Hilbert space).  \\
            $\inner{\cdot,\cdot}_{\mathcal{Y}}$ & The canonical inner
            product of the Hilbert space $\mathcal{Y}$. \\
            $\norm{\cdot}_{\mathcal{Y}}$ & The canonical norm induced by the
            inner product of the Hilbert space $\mathcal{Y}$. \\
            $\mathcal{F}(\mathcal{X};\mathcal{Y})$ & Topological vector space
            of functions from $\mathcal{X}$ to $\mathcal{Y}$. \\
            $\mathcal{C}(\mathcal{X};\mathcal{Y})$ & The topological vector
            subspace of $\mathcal{F}$ of continuous functions from
            $\mathcal{X}$ to $\mathcal{Y}$. \\
            $\mathcal{L}(\mathcal{H};\mathcal{Y})$ & The space bounded linear
            operator from a Hilbert space $\mathcal{H}$ to a
            Hilbert space $\mathcal{Y}$. \\
            $\norm{\cdot}_{\mathcal{Y},\mathcal{Y}'}$ & The operator norm
            $\norm{\Gamma}_{\mathcal{Y}, \mathcal{Y'}} =
            \sup_{\norm{y}_{\mathcal{Y}}=1}\norm{\Gamma y}_{\mathcal{Y}'}$ for
            all $\Gamma\in\mathcal{L}(\mathcal{Y},\mathcal{Y'})$ \\
            $\mathcal{M}_{m,n}(\mathbb{K})$ & The space of matrices of size
            $(m,n)$. \\
            $\mathcal{L}(\mathcal{Y})$ & The space of bounded linear operator
            from a Hilbert space $\mathcal{Y}$ to itself. \\
            $\mathcal{L}_{+}(\mathcal{Y})$ & The space of non-negative bounded
            linear operator from a Hilbert space $\mathcal{H}$ to itself. \\
            $\mathcal{B}(\mathcal{X})$ & Borel $\sigma$-algebra on a
            topological space $\mathcal{X}$. \\
            %$\mu(\mathcal{X})$ & A scalar positive measure of $\mathcal{X}$. \\
            $\Leb(\mathcal{X})$ & The Lebesgue measure of $\mathcal{X}$. \\
            $\Haar(\mathcal{X})$ & A Haar measure of $\mathcal{X}$. \\
            $\probability_{\mu, \rho}(\mathcal{X})$ & A probability measure of
            $\mathcal{X}$ whose Radon-Nikodym derivative (density) with respect
            to the measure $\mu$ is $\rho$. \\
            $\FT{\cdot}$ & The \acl{FT} operator. \\
            %$\IFT{\cdot}$ & The \acl{IFT} operator. \\
            %$\esssup$ & The essential supremum. \\
            %$L^p(\mathcal{X}, \mu)$ & The Banach space of
            %$\abs{\cdot}^p$-integrable function from
            %$(\mathcal{X},\mathcal{B}(\mathcal{X}), \mu)$ to $\mathbb{C}$ for
            %$p\in\mathbb{R}_+$. \\
            $L^p(\mathcal{X}, \mu;\mathcal{Y})$ & The Banach space of
            $\norm{\cdot}_{\mathcal{Y}}^p$ (Bochner)-integrable function from
            $(\mathcal{X},\mathcal{B}(\mathcal{X}), \mu)$ to $\mathcal{Y}$ for
            $p\in\mathbb{R}_+$. $L^p(\mathcal{X},\mu,\mathbb{R}) \colonequals
            L^p(\mathcal{X},\mu)$ and
            $L^p(\mathcal{X},\mu,\mathbb{R})=L^p(\mathcal{X}, \mu)$. \\
            $\Vect_{j=1}^D x_i$ & The direct sum of $D\in\mathbb{N}$ vectors
            $x_i$'s in the Hilbert spaces $\mathcal{H}_i$. By definition
            $\inner{\Vect_{j=1}^D x_j, \Vect_{j=1}^D z_j} = \sum_{j=1}^D
            \inner{x_j, z_j}_{\mathcal{H}_i}$. \\
            $\norm{\cdot}_p$ & The $L^p(\mathcal{X}, \mu, \mathcal{Y})$ norm.
            $\norm{f}_p^p\colonequals \int_{\mathcal{X}}
            \norm{f(x)}_{\mathcal{Y}}^p d\mu(x)$.  When
            $\mathcal{X}=\mathbb{N}^*$, $\mathcal{Y}\subseteq \mathbb{R}$ and
            $\mu$ is the counting measure and $p=2$ it coincide with the
            Euclidean norm $\norm{\cdot}_2$ for finite dimensional vectors. \\
            $\norm{\cdot}_{\infty}$ & The uniform norm $\norm{f}_{\infty}=
            \esssup \set{\norm{f(x)}_{\mathcal{Y}} |
            x\in\mathcal{X}}=\lim_{p\to\infty}\norm{f}_p$. \\
            %${}^\transpose$ & The transpose operator of a linear operator. \\
            %${}^\adjoint$ & The adjoint operator of a linear operator. \\
            $\abs{\Gamma}$ & The absolute value of the linear operator
            $\Gamma\in\mathcal{L}(\mathcal{Y})$, \acs{ie}
            $\abs{\Gamma}^2=\Gamma^{\adjoint}\Gamma$. \\
            $\Tr\left[\Gamma\right]$ & The trace of a linear operator
            $\Gamma\in\mathcal{L}(\mathcal{Y})$. \\
            %$\sigma(\Gamma)$ & The spectrum of the bounded linear operator
            %$Gamma\in\mathcal{L}(\mathcal{Y})$ where $\mathcal{Y}$ is a Hilbert
            %space, \acs{ie}~$\sigma(\Gamma)=\Set{\lambda\in\mathbb{C} |
            %\nexists s, s(\lambda e - \Gamma) = e}$. \\
            %$\lambda_i(\Gamma)$ & The $i$-th eigenvalue of
            %$\Gamma\in\mathcal{L}(\mathcal{Y})$, ranked by increasing modulus,
            %where $\mathcal{Y}$ is a \emph{separable} Hilbert space and
            %$i\in\mathbb{N}^*$. \\
            %$\rho(\Gamma)$ & The spectral radius of the linear operator
            %$\Gamma$ \acs{ie}~$\rho(\Gamma)=\sup\set{\abs{\lambda} | \lambda
            %\in \sigma(\Gamma)}$. \\
            $\norm{\cdot}_{\sigma, p}$ & The Schatten $p$-norm,
            $\norm{\Gamma}_{\sigma,
            p}^p=\Tr\left[\abs{\Gamma}^p\right]$ for
            $\Gamma\in\mathcal{L}(\mathcal{Y})$, where $\mathcal{Y}$ is a
            Hilbert space. Note that $\norm{\Gamma}_{\sigma,\infty} =
            \rho(\Gamma) \le \norm{\Gamma}_{\mathcal{Y},\mathcal{Y}}$.  \\
            $\succcurlyeq$ & \say{Greater than} in the Loewner partial order of
            operators. $\Gamma_1 \succcurlyeq \Gamma_2$ if $\sigma(\Gamma_1 -
            \Gamma_2) \subseteq \mathbb{R}_+$. \\
            %$\bar{\mathbb{R}}$ & The one point compacification of the real
            %line $\mathbb{R} \cup \Set{\infty}$. \\
            $\cong$ & Given two sets $\mathcal{X}$ and $\mathcal{Y}$,
            $\mathcal{X} \cong \mathcal{Y}$ if there exists an isomorphism
            $\phi:\mathcal{X}\to\mathcal{Y}$. \\
        \bottomrule
    \end{tabularx}
\end{table}
%\begin{table}[t]
    %\centering
    %\caption{Mathematical symbols and their signification (part 2).
    %\label{table:notations2}}
    %\begin{tabularx}{\textwidth}{cX}
        %\toprule
            %Symbol & \multicolumn{1}{c}{Meaning} \\
        %\cmidrule{1-2}
        %\endhead
        %\bottomrule
    %\end{tabularx}
%\end{table}
%\subsection{Introduction to kernel methods}\label{subsec:kernels}
%\subsubsection{Kernels and Reproducing Kernel Hilbert Spaces}
%The idea of kernel methods
%\citep{Aronszajn1950,KIMELDORF1971,boser1992training,
%Berlinet2003,Shawe-TaylorBook} is to work in a subset of the set of all
%functions, namely a \acf{RKHS}, associated to a well chosen positive
%semi-definite and symmetric function (\emph{a kernel}).
%\paragraph{}
%\begin{definition}[Positive-definite kernels]
    %Let $\mathcal{X}$ be a locally compact second countable topological space.
    %A kernel $k:\mathcal{X} \times \mathcal{X} \to \mathbb{R}$ is said to be
    %\acf{PSD} if for any $(x_1, \ldots, x_N) \in \mathcal{X}^N$, the (Gram)
    %matrix
    %\begin{dmath*}
        %\mathbf{K} =
        %\begin{pmatrix}
            %k(x_i,x_j)
        %\end{pmatrix}_{i = 1, j = 1}^{i = N, j = N} \hiderel{\in}
        %\mathcal{M}_{N, N}(\mathbb{R})
    %\end{dmath*}
    %is \acf{SPSD}\footnote{Note that for historical reasons valid kernels are
    %called \say{positive-definite kernels}, although for any sequences of
    %points the corresponding Gram matrix needs only to be (symmetric) Positive
    %Semi-Definite \citep{fukumizu2008elements}.}.
%\end{definition}
%The following proposition gives sufficient conditions to obtain a \acs{SPSD}
%matrix.
%\begin{proposition}[\acs{SPSD} matrix]
    %$K$ is SPSD if it is symmetric and one of the following conditions holds:
    %\begin{itemize}
        %\item The eigenvalues of $\mathbf{K}$ are non-negative
        %\item for any column vector $c= (c_1, \ldots, c_N)^\transpose \in
        %\mathcal{M}_{N, 1}(\mathbb{R})$,
        %\begin{dmath*}
            %c^\transpose\mathbf{K}c=\sum_{i,j=1}^N c_ic_jK(x_i,x_i)
            %\hiderel{\geq} 0
        %\end{dmath*}
    %\end{itemize}
%\end{proposition}
%One of the most important property of \acs{PD} kernels \citep{Mohri2012} is
%that a \acs{PD} kernel defines a unique \acs{RKHS}. Note that the converse is
%also true.
%\begin{theorem}[\citet{Aronszajn1950}]
    %Suppose $k$ is a symmetric, positive-definite kernel on a set
    %$\mathcal{X}$. Then there is a unique Hilbert space of functions
    %$\mathcal{H}$ on $\mathcal{X}$ for which $k$ is a reproducing kernel,
    %\acs{ie}
    %\begin{dgroup}
        %\begin{dmath}\label{eq:reproducing-prop}
            %\forall x \in \mathcal{X}, k(\cdot, x) \hiderel{\in} \mathcal{H}
        %\end{dmath}
        %\begin{dmath}
            %\forall h \in \mathcal{H}, \forall x \hiderel{\in} \mathcal{X},
            %h(x) \hiderel{=} \inner{h,k(\cdot,x)}_{\mathcal{H}}.
        %\end{dmath}
    %\end{dgroup}
    %$\mathcal{H}$ is called a reproducing kernel Hilbert space (\acl{RKHS})
    %associated to $k$, and will be denoted, $\mathcal{H}_k$.
%\end{theorem}
%Another way to use Aronszajn's results is to state the feature map property for
%the \acs{PD} kernels.
%\begin{proposition}[Feature map]
     %Suppose $k$ is a symmetric, positive-definite kernel on a set
     %$\mathcal{X}$. Then, there exists a Hilbert space $\mathcal{H}$ and a
     %mapping $\phi$ from $\mathcal{X}$ to $\mathcal{H}$ such that:
    %\begin{dmath*}
        %\forall x, x' \hiderel{\in} \mathcal{X},~
        %k(x,x')\hiderel{=}\inner{\phi(x),\phi(x')}_{\mathcal{H}}.
    %\end{dmath*}
    %The mapping $\phi$ is called a \emph{feature map} and $\mathcal{H}$, a
    %feature space.
%\end{proposition}
%\begin{remark}
    %Aronszajn's theorem tells us that there always exists at least one feature
    %map called \emph{canonical feature map}, with associated feature space
    %called the \acl{RKHS} $\mathcal{H}=\mathcal{H}_k$ (associated with $k$),
    %\begin{dmath*}
        %\phi(x)= k(\cdot, x)
    %\end{dmath*}
    %There exists several pairs of feature maps and features spaces for a given
    %kernel $k$.
%\end{remark}
%\subsubsection{Learning in Reproducing Kernel Hilbert Spaces}
%\label{subsubsec:learning_in_rkhs}
%We place ourself in the context of supervised statistical learning
%\citep{vapnik1998statistical}, where the goal is to minimize a quantity called
%the true risk over a class of function which in our case is a \acs{RKHS}. We
%suppose that we are given a local loss function
%$L:\mathcal{X}\times\mathcal{H}_k\times \mathcal{Y} \to \mathbb{R}_+$ and that
%our training samples $(x_i, y_i)$ are \ac{iid} random variables following the
%probability distribution $\probability(X, Y)$. Then we define the true risk as
%\begin{dmath*}
    %\mathfrak{R}(f)=\expectation_{X, Y} L(X, f, Y)\condition{$X,
    %Y\sim\probability(X, Y)$}.
%\end{dmath*}
%However since the probability distribution $\probability(X, Y)$ is usually
%unknown we define and minimize its empirical counterpart called the empirical
%risk.
%\begin{dmath*}
    %\label{eq:empirical_risk}
    %\mathfrak{R}_{emp}(f, \seq{s}) = \frac{1}{N}\sum_{i=1}^NL(x_i, f, y_i).
%\end{dmath*}
%In this context, a fair question is how to pick-up functions in a space
%$\mathcal{H}_k$ with inifinitely many elements, that minimize the empirical
%risk (\cref{eq:empirical_risk}), in polynomial time? The answer comes from the
%regularization and interpolation theory. To limit the size of the space in
%which we search for the function minimizing the empirical risk we add a
%regularization term to the empirical risk.
%\begin{dmath*}
    %\mathfrak{R}_{\lambda}(f, \seq{s}) = \mathfrak{R}_{\text{emp}}(f, \seq{s})
    %+ \frac{\lambda}{2} \norm{f}_{\mathcal{H}_k}^2
    %= \frac{1}{N} \sum_{i=1}^N L\left(x_i, f,
    %y_i\right) + \frac{\lambda}{2}\norm{f}_{\mathcal{H}_k}^2
%\end{dmath*}
%and we minimize $\mathfrak{R}_{\lambda}$ with respect to $f$ for a given
%$\seq{s}$ instead of $\mathfrak{R}_{\text{emp}}$. Then the representer theorem
%(also called minimal norm interpolation theorem) states the following.
%\begin{theorem}[Representer theorem, \citet{Wahba90}]
    %If $f_{\seq{s}}$ is a solution of
    %\begin{dmath*}
        %\argmin_{f\in\mathcal{H}_k} \mathfrak{R}_{\lambda}(f, \seq{s}),
    %\end{dmath*}
    %where $\lambda > 0$ then $f_{\seq{s}}=\sum_{i=1}^N k(\cdot, x_i) \alpha_i$.
%\end{theorem}
%We note the vector $\boldsymbol{\alpha} = (\alpha_i)_{i=1}^N$ and the matrix
%$\mathbf{K}=(k(x_i, x_k))_{i, k = 1}^N$. Because of the representer theorem,
%stating that a solution of the empirical risk minimization is a linear
%combination of kernel evaluations weighted by a vector $\boldsymbol{\alpha}$,
%with mild abuse of notation we identify the function $f\in\mathcal{H}_k$ with
%the vector $\boldsymbol{\alpha}\in\mathbb{R}^N$.  Thus we rewrite the loss
%$L(x, f, y)$ as $L(x, \boldsymbol{\alpha}, y)$. Then we can rewrite
%\begin{dmath*}
    %\mathfrak{R}_{\lambda}(\boldsymbol{\alpha}, \seq{s}) =
    %\frac{1}{N}
    %\sum_{i=1}^NL(x_i, \boldsymbol{\alpha}, y_i) +
    %\frac{\lambda}{2} \inner{\boldsymbol{\alpha},
    %\mathbf{K}\boldsymbol{\alpha}}_2,
%\end{dmath*}
%and $f(x_i) = (\mathbf{K}\boldsymbol{\alpha})_i$ for any $x_i$ in the
%training set. For instance if we choose $L(x, f,
%y)=\frac{1}{2}\abs{f(x)-y}^2$ to be the least square loss, then
%\begin{dmath*}
    %L(x_i, \boldsymbol{\alpha}, y_i) =
    %\frac{1}{2}\abs{(\mathbf{K}\boldsymbol{\alpha})_i-y_i}^2.
%\end{dmath*}
%In this case $L$ is convex in $\boldsymbol{\alpha}$, thus it is possible to
%derive a polynomial time (in $N$) algorithm minimizing $\mathfrak{R}_{\lambda}$
%for the least square loss, which is called \emph{kernel Ridge regression}:
%\begin{dmath}
    %\label{eq:ridge_regression}
    %\mathfrak{R}_{\lambda}(\boldsymbol{\alpha}, \seq{s}) =
    %\frac{1}{2N}\norm{\mathbf{K}\boldsymbol{\alpha} - (y_i)_{i=1}^N}_2^2 +
    %\frac{\lambda}{2} \inner{\boldsymbol{\alpha},
    %\mathbf{K}\boldsymbol{\alpha}}_2.
%\end{dmath}
%As a result of the representer theorem we see that we search a minimizer over
%$\boldsymbol{\alpha}\in\mathbb{R}^N$ instead of $f\in\mathcal{H}_k$. By strict
%convexity and coercivity of $\mathfrak{R}_{\lambda}$, and because $\mathbf{K} +
%\lambda I_N$ is invertible\footnote{Note that although $\mathbf{K} + \lambda
%I_N$ is always invertible if $\lambda>0$, choosing a too small value of
%$\lambda$ can leads to an ill-conditioned system if the eigenvalues of
%$\mathbf{K}+\lambda I_N$ are too small.} for any $\lambda > 0$, the unique
%solution is $\alpha_{\seq{s}} = \argmin_{\boldsymbol{\alpha}\in\mathbb{R}^N}
%\mathfrak{R}_{\lambda}(\boldsymbol{\alpha},\seq{s}) = (\mathbf{K}/N + \lambda
%I_N)^{-1}(y_i)_{i=1}^N$. This is an $O_t\left(N^3\right)$ algorithm.
%\paragraph{}
%Another way of describing positive-definite kernels and \acs{RKHS} consists in
%defining a \emph{feature map} $\phi:\mathcal{X}\to\mathcal{H}$ where
%$\mathcal{H}$ is a Hilbert space.  Then any function in $\mathcal{H}_k$ can be
%written $f(x)=\inner{\phi(x), \theta}_{\mathcal{H}}$ In a nutshell the function
%$\phi$ is called feature map because it \say{extracts characteristic elements
%from a vector}. Usually a feature map takes a vector in an input space with low
%dimension and maps it to a potentially infinite dimensional Hilbert space. Put
%it differently, any function in $\mathcal{H}_k$ is the composition of linear
%functional $\theta^\adjoint$ with a non linear feature map $\phi$. Thus if the
%feature map $\phi$ is fixed (which is equivalent to fixing the kernel), it is
%possible to \say{learn} with a linear class of functions $\theta\in\mathcal{H}$
%(see \cref{fig:feature_map}).
%\begin{figure}
    %%\centering\resizebox{\textwidth}{!}{%
    %%\begin{tikzpicture}
        %%\node[inner sep=0pt] (input) at (0,0)
            %%{\includegraphics[width=.35\textwidth]{./gfx/input.eps}};
        %%\node[inner sep=0pt] (feature) at (5,-6)
            %%{\includegraphics[width=.35\textwidth]{./gfx/feature.eps}};
        %%\draw[->,thick] (input.east) -- (feature.west)
            %%node[midway,fill=white] {$\phi:\mathcal{X} \to \mathcal{H}$};
    %%\end{tikzpicture}}
    %\centering
    %\begin{tabular}{c}
        %\includegraphics[valign=m, width=.5\textheight]{./gfx/input.eps} \\
        %$\xdownarrow{1cm} \phi: \enskip \mathcal{X} = \mathbb{R}^2 \to
        %\mathcal{H} = \mathbb{R}^3$ \\
        %\includegraphics[valign=m, width=.5\textheight]{./gfx/feature.eps}
    %\end{tabular}
    %\caption[A scalar-valued feature map]{We map the two circles in
    %$\mathbb{R}^2$ to $\mathbb{R}^3$. In $\mathbb{R}^3$ it is now possible to
    %separate the circles with a linear functional: a plane. We used the feature
    %map \\ $\phi(x) = 0.82 \begin{pmatrix} \cos(1.76 x_1 + 2.24 x_2 + 2.75) \\
    %\cos(0.40 x_1 + 1.87 x_2 + 5.6) \\ \cos(0.98 x_1 - 0.98 x_2 + 6.05)
    %\end{pmatrix}$. \\
    %Here $\phi:\mathbb{R}^2\to\mathbb{R}^3$ has been chosen
    %as a realization of an \acs{RFF} map (see \cref{eq:rff2}). A \say{cleaner}
    %feature map adapted to this problem could have been \\
    %$\phi(x)=\begin{pmatrix} x_1 \\ x_2 \\ x_1^2 + x_2^2 \end{pmatrix}$.
    %\label{fig:feature_map}}
%\end{figure}
%If we note
%\begin{dmath*}
    %\boldsymbol{\phi} =
    %\begin{pmatrix}
        %\phi(x_1) & \dots & \phi(x_N)
    %\end{pmatrix}
%\end{dmath*}
%the \say{matrix} where each column represents the feature map evaluated at the
%point $x_i$ with $1 \le i \le N$, the regularized risk minimization with the
%least square loss reads
%\begin{dmath*}
    %\mathfrak{R}_{\lambda}(\theta, \seq{s}) =
    %\frac{1}{2N}\norm{\boldsymbol{\phi}^\transpose \theta - (y_i)_{i=1}^N
    %}_{2}^2 + \frac{\lambda}{2}\norm{\theta}_2^2.
%\end{dmath*}
%and the unique solution is $\theta_{\seq{s}} =
%\left(\boldsymbol{\phi}\boldsymbol{\phi}^\transpose/N + \lambda
%I_{\mathcal{H}}\right)^{-1}\boldsymbol{\phi}$. This is an
%\begin{dmath*}
    %O_t\left( \dim(\mathcal{H})^2(N + \dim{\mathcal{H}}) \right).
%\end{dmath*}
%time complexity algorithm.  This algorithm seems more appealing than its kernel
%counterpart when many data are given since once the space $\mathcal{H}$ has
%been fixed, the algorithm is linear in the number of training points. However
%many questions remains. First although it is possible to design a feature map
%\emph{ex nihilo}, can we design systematically a feature map from a kernel? For
%some kernels (\acs{eg} the Gaussian kernel) it is well known that the Hilbert
%space corresponding to it has dimension $\dim(\mathcal{H}) = \infty$. Is it
%possible to find an approximation of the kernel such that $\dim(\mathcal{H}) <
%\infty$? If such a construction is possible and we know that $N$ training data
%are available, is it possible to have a sufficiently good approximation%
%\footnote{When $\dim(\mathcal{H}) \ge N$ then is it is better to use the kernel
%algorithm than the feature algorithm. This is called the kernel trick.} with
%$\dim(\mathcal{H}) \ll N$?
%\subsection{Towards large scale learning with kernels}
%Motivated by large scale applications, different methodologies have been
%proposed to approximate kernels and feature maps. This subsection briefly
%reminds the main approaches based on  Random Fourier Features and Nystr\"om
%techniques. Notice that another line of research concerns online learning
%method such as \acs{NORMA} developed in \cite{kivinen2004online}, later
%extended to the operator-valued kernel case by \citet{audiffren2013online}.  We
%start with the seminal work of \citet{Rahimi2007} who show that given a
%continuous shift-invariant kernel ($\forall x, z, t \in \mathcal{X}$, $k(x + t,
%z + t) = k(x, z)$), it is possible to obtain a feature map called \acs{RFF}
%that approximate the given kernel.
%\subsubsection{Random Fourier Feature maps}
\subsection{Random Fourier Feature maps}
\aclp{RFF} methodology introduced  by \citet{Rahimi2007} provides a
way to scale up kernel methods when kernels are Mercer and
\emph{translation-invariant}.  We view the input space $\mathcal{X}$ as a group
endowed with the addition. Extensions to other group laws such as
\citet{li2010random} are described in \cref{subsubsec:skewedchi2} within the
general framework of operator-valued kernels.
\paragraph{}
Denote $k: \mathbb{R}^d \times \mathbb{R}^d \to \mathbb{R}$ a positive
definite kernel \citep{Aronszajn1950} on $\mathbb{R}^d$. A kernel $k$ is said
to be \emph{shift-invariant} or \emph{translation-invariant} for the addition
if for for all $(x,z,t) \in \left(\mathbb{R}^d\right)^3$ we have $k(x+t,z+t) =
k(x,z)$.  Then, we define $k_0: \mathbb{R}^d \to \mathbb{R}$ the function such
that $k(x,z)= k_0(x-z)$. $k_0$ is called the \emph{signature} of kernel $k$. If
$k_0$ is a continuous function we call the kernel \say{Mercer}. Then, Bochner's
theorem \citep{folland1994course} is the theoretical result that leads to the
Random Fourier Features.
\begin{theorem}[Bochner's theorem]\label{th:bochner-scalar}
    Any continuous positive-definite function (\acs{eg} a Mercer kernel) is the
    \acl{FT} of a bounded non-negative Borel measure.
\end{theorem}
It implies that any positive-definite, continuous and shift-invariant kernel
$k$, have a continuous and positive-definite signature $k_0$, which is the
\acl{FT} $\mathcal{F}$ of a non-negative measure $\mu$. We therefore have the
$k(x,z)=k_0(x-z) \hiderel{=} \int_{\mathbb{R}^d} \exp(-\iu \inner{\omega,x -
z}) d\mu(\omega) \hiderel{=}\FT{k_0}(\omega)$.  Moreover $\mu = \IFT{k_0}$.
Without loss of generality, we assume that $\mu$ is a probability measure,
\acs{ie} $\int_{\mathbb{R}^d} d\mu(\omega)=1$ by renormalizing the kernel since
$\int_{\mathbb{R}^d}d\mu(\omega)= \int_{\mathbb{R}^d}\exp(-\iu \inner{\omega,
0})d\mu(\omega)\hiderel{=}k_0(0)$.  and we can write the kernel as an
expectation over a probability measure $\mu$.  For all $x$, $z\in\mathbb{R}^d$
\begin{dmath*}
    k_0(x-z) = \expectation_{\omega\sim\mu}\left[\exp(-\iu \inner{\omega,x -
    z})\right].
\end{dmath*}
Eventually, if $k$ is real valued we only write the real part, $k(x,z) =
\expectation_{\omega\sim\mu}[\cos \inner{\omega,x - z}] \hiderel{=}
\expectation_{\omega\sim\mu}[ \cos \inner{\omega,z} \cos \inner{\omega,x} +
\sin \inner{\omega,z} \sin \inner{\omega,x}]$.  Let $\Vect_{j=1}^D x_j$ denote
the $Dd$-length column vector obtained by stacking vectors $x_j \in
\mathbb{R}^d$.  The feature map $\widetilde{\phi}: \mathbb{R}^d \rightarrow
\mathbb{R}^{2D}$ defined as
\begin{dmath}
    \label{eq:rff}
    \widetilde{\phi}(x)=\frac{1}{\sqrt{D}}\Vect_{j=1}^D
    \begin{pmatrix}
        \cos{\inner{x,\omega_j}} \\
        \sin{\inner{x,\omega_j}}
    \end{pmatrix}\condition{$\omega_j \hiderel{\sim} \IFT{k_0}$ \acs{iid}}
\end{dmath}
is called a \emph{Random Fourier Feature} (map). Each $\omega_{j}, j=1, \ldots,
D$ is independently and identically sampled from the inverse Fourier transform
$\mu$ of $k_0$. This Random Fourier Feature map provides the following
Monte-Carlo estimator of the kernel: $\widetilde{k}(x, z) =
\widetilde{\phi}(x)^* \widetilde{\phi}(z)$. Using trigonometric identities,
\citet{Rahimi2007} showed that the same feature map can also be written
\begin{dmath}
    \label{eq:rff2}
    \tilde{\phi}(x)=\sqrt{\frac{2}{D}}\Vect_{j=1}^D
    \begin{pmatrix}
        \cos(\inner{x,\omega_j} + b_j)
    \end{pmatrix},
\end{dmath}
where $\omega_j \hiderel{\sim} \IFT{k_0}$, $b_j \sim \mathcal{U}(0, 2\pi)$
\acs{iid}.  The feature map defined by \cref{eq:rff} and \cref{eq:rff2} have
been compared in \citet{sutherland2015} where they give the condition under
wich \cref{eq:rff} has lower variance than \cref{eq:rff2}. For instance for the
Gaussian kernel, \cref{eq:rff} has always lower variance. In practice,
\cref{eq:rff2} is easier to program. In this paper we focus on random Fourier
feature of the form given in \cref{eq:rff}.
\paragraph{}
The dimension $D$ governs the precision of this approximation, whose uniform
convergence towards the target kernel can be found in \citet{Rahimi2007} and in
more recent papers with some refinements proposed in \citet{sutherland2015} and
\citet{sriper2015}.  Finally, it is important to notice that Random Fourier
Feature approach \emph{only} requires two steps before the application of a
learning algorithm: (1) define the inverse Fourier transform of the given
shift-invariant kernel, (2) compute the randomized feature map using the
spectral distribution $\mu$.  \citet{Rahimi2007} show that for the Gaussian
kernel $k_0(x-z) = \exp(-\gamma \norm{x - z}_2^2)$, the spectral distribution
$\mu$ is a Gaussian distribution. For the Laplacian kernel $k_0(x-z) =
\exp(-\gamma \norm{x - z}_1)$, the spectral distribution is a Cauchy
distribution.

\subsubsection{Extensions of the RFF method}
\paragraph{}
The seminal idea of \citet{Rahimi2007} has opened a large literature on random
features. Nowadays, many classes of kernels other than translation invariant
are now proved to have an efficient random feature representation.
\citet{kar2012random} proposed random feature maps for dot product kernels
(rotation invariant) and \citet{hamid2014compact} improved the rate of
convergence of the approximation error for such kernels by noticing that
feature maps for dot product kernels are usually low rank and may not utilize
the capacity of the projected feature  space  efficiently. \Citet{pham2013fast}
proposed fast random feature maps for polynomial kernels.
\paragraph{}
\Citet{li2010random} generalized the original \acs{RFF} of \citet{Rahimi2007}.
Instead of computing feature maps for shift-in\-va\-riant kernels on the
additive group $(\mathbb{R}^d, +)$, they used the generalized Fourier transform
on any locally compact abelian group to derive random features on the
multiplicative group $(\mathbb{R}^d, *)$. In the same spirit
\citet{yang2014random} noticed that an theorem equivalent to Bochner's theorem
exists on the semi-group $(\mathbb{R}_{>0}^d, +)$. From this they derived
\say{Random Laplace} features and used them to approximate kernels adapted to
learn on histograms.
\paragraph{}
To speed-up the convergence rate of the random features approximation,
\citet{yang2014quasi} proposed to sample the random variable from a quasi
Monte-Carlo sequence instead of \acs{iid}~random variables. \Citet{Le2013}
proposed the \say{Fastfood} algorithm to reduce the complexity of computing a
\acs{RFF} --using structured matrices and a fast Walsh-Hadarmard transform--
from $O_t(Dd)$ to $O_t(D\log(d))$. More recently \citet{felix2016orthogonal}
proposed also an algorithm \say{SORF} to compute Gaussian \acs{RFF} in
$O_t(D\log(d))$ but with better convergence rates than \say{Fastfood}
\citep{Le2013}.  \Citet{mukuta2016kernel} proposed a data dependent feature
map (comparable to the Nystro\"m method) by estimating the distribution of the
input data, and then finding the eigenfunction decomposition of Mercer's
integral operator associated to the kernel.
\paragraph{}
In the context of large scale learning and deep learning, \citet{lu2014scale}
showed that \acsp{RFF} can achieve performances comparable to deep-learning
methods by combining multiple kernel learning and composition of kernels along
with a scalable parallel implementation. \Citet{dai2014scalable} and
\citet{xie2015scale} combined \acsp{RFF} and stochastic gradient descent to
define an online learning algorithm called \say{Doubly stochastic gradient
descent} adapted to large scale learning. \Citet{yang2015deep} proposed and
studied the idea of replacing the last fully interconnected layer of a deep
convolutional neural network \citep{lecun1995convolutional} by the
\say{Fastfood} implementation of \acsp{RFF}.
\paragraph{}
Eventually \citet{Yang2015} introduced the algorithm \say{\`A la Carte}, based
on \say{Fastfood} which is able to learn the spectral distribution

%%%%%%%%%%%%%%%%%%%%%%%%%%%%%%%%%%%%%%%%%%%%%%%%%%%%%%%%%%%%%%%%%%%%%%%%%%%%%%%

\subsection{On Operator-Valued Kernels}
\label{sec:background_on_operator-valued_kernels} We now introduce the theory
of \acf{vv-RKHS} that provides a flexible framework to study and learn
vector-valued functions. The fundations of the general theory of scalar kernels
is mostly due to \citet{Aronszajn1950}  and provides a unifying point of view
for the study of an important class of Hilbert spaces of real or complex valued
functions. It has been first applied in the theory of partial differential
equation. The theory of \acfp{OVK} which extends the scalar-valued kernel was
first developped by \citet{Pedrick57} in his Ph.~D Thesis. Since then it has
been successfully applied to machine learning by many authors. In particular we
introduce the notion of \aclp{OVK} following the propositions of
\citet{Micchelli2005,carmeli2006vector,Carmeli2010}.
\subsection{Definitions and properties}
\label{subsec:def_properties} In machine learning the goal is often to find a
function $f$ belonging to a class of functions
$\mathcal{F}(\mathcal{X};\mathcal{Y})$ that minimizes a criterion called the
true risk. The class of functions we consider are functions living in a Hilbert
space $\mathcal{H}\subset\mathcal{F}(\mathcal{X};\mathcal{Y})$. The
completeness allows to consider sequences of functions $f_n \in\mathcal{H}$
where the limit $f_n\to f$ is in $\mathcal{H}$. Moreover the existence of an
inner product gives rise to a norm and also makes $\mathcal{H}$ a metric space.
\paragraph{}
Among all these functions $f\in\mathcal{H}$, we consider a subset of functions
$f\in\mathcal{H}_K\subset\mathcal{H}$ such that the evaluation map
$\text{ev}_x:f\mapsto f(x)$ is bounded for all $x$. \acs{ie} such
that~$\norm{\text{ev}_x}_{\mathcal{H}_K}\le C_x\in\mathbb{R}$ for all $x$. For
scalar valued kernels the evaluation map is a linear functional. Thus by Riesz's
representation theorem there is an isomorphism between evaluating a function at
a point and an inner product: $f(x)=\text{ev}_x f = \inner{K_x, f}_K$. From
this we deduce the reproducing property $K(x,z)=\inner{K_x, K_z}_K$ which is
the cornerstone of many proofs in machine learning and functional analysis.
When dealing with vector-valued functions, the evaluation map $\text{ev}_x$ is
no longer a linear functional, since it is vector-valued. However, inspired by
the theory of scalar valued kernel, many authors showed that if the evaluation
map of functions with values in a Hilbert space $\mathcal{Y}$ is bounded, a
similar reproducing property can be obtained; namely $\inner{y',
K(x,z)y}=\inner{K_x y', K_z y}_K$ for all $y$, $y'\in\mathcal{Y}$. This
motivates the following definition of a \acf{vv-RKHS}.
\begin{definition}[\acl{vv-RKHS}~\citep{carmeli2006vector,Micchelli2005}]
    Let $\mathcal{Y}$ be a (real or complex) Hilbert space. A \acl{vv-RKHS} on
    a locally compact second countable topological space $\mathcal{X}$ is a
    Hilbert space $\mathcal{H}$ such that
    \begin{enumerate}
        \item the elements of $\mathcal{H}$ are functions from $\mathcal{X}$ to
        $\mathcal{Y}$ (\acs{ie}~$\mathcal{H} \subset \mathcal{F}(\mathcal{X},
        \mathcal{Y})$);
        \item for all $x\in\mathcal{X}$, there exists a positive constant $C_x$
        such that for all $f\in\mathcal{H}$ $\norm{f(x)}_{\mathcal{Y}}\le
        C_x\norm{f}_{\mathcal{H}}$.
    \end{enumerate}
\end{definition}
Throughout this section we show that a \ac{vv-RKHS} defines a unique
po\-si\-ti\-ve-de\-fi\-ni\-te function called \acf{OVK} and conversely an
\ac{OVK} uniquely defines a \ac{vv-RKHS}. The bijection between \acsp{OVK} and
\acsp{vv-RKHS} has been first proved by~\citet{Senkene73} in 1973. In this
introduction to \acsp{OVK} we follow the definitions and most recent proofs
of~\citet{Carmeli2010}.
%\begin{definition}[Positive-definite \acl{OVK} acting on a complex Hilbert
%space]
    %\label{def:reproducing_kernel}
    %Given $\mathcal{X}$ a locally compact second countable topological space
    %and  $\mathcal{Y}$ a complex Hilbert Space, a map
    %$K:\mathcal{X}\times\mathcal{X}\to\mathcal{L}(\mathcal{Y})$ is called an
    %positive-definite \acl{OVK} kernel if
    %\begin{dmath}
        %\sum_{i,j=1}^N\inner{K(x_i,x_j)y_j,y_i}_{\mathcal{Y}}\ge 0,
    %\end{dmath}
    %for all $N\in\mathbb{N}$, for all sequences of points $(x_i)_{i=1}^N$ in
    %$\mathcal{X}^N$ and all sequences of points $(y_i)_{i=1}^N$ in
    %$\mathcal{Y}^N$. \label{def:ovk}
%\end{definition}
%If $\mathcal{Y}$ is a real Hilbert space, a positive-definite \acl{OVK} is
%always self-adjoint, \acs{ie}~$K(x,z)=K(z,x)^\adjoint$. This gives rise to the
%following definition of positive-definite \acl{OVK} acting on a real Hilbert
%space.
\begin{definition}[Positive-definite \acl{OVK}]
    \label{def:reproducing_kernel_real} Given $\mathcal{X}$ a locally compact
    second countable topological space and $\mathcal{Y}$ a real Hilbert Space,
    a map $K:\mathcal{X}\times\mathcal{X}\to\mathcal{L}(\mathcal{Y})$ is called
    a positive-definite \acl{OVK} kernel if $K(x,z)=K(z,x)^\adjoint$ and
    \begin{dmath}
        \sum_{i,j=1}^N\inner{K(x_i,x_j)y_j,y_i}_{\mathcal{Y}}\ge 0,
    \end{dmath}
    for all $N\in\mathbb{N}$, for all sequences of points $(x_i)_{i=1}^N$ in
    $\mathcal{X}^N$, and all sequences of points  $(y_i)_{i=1}^N$ in
    $\mathcal{Y}^N$. \label{def:ovk_real}
\end{definition}
As in the scalar case any \acl{vv-RKHS} defines a unique positive-definite
\acl{OVK} and conversely a positive-definite \acl{OVK} defines a unique
\acl{vv-RKHS}.
\begin{proposition}[\citep{carmeli2006vector}]
    \label{pr:unique_rkhs} Given a \acl{vv-RKHS} there is a unique
    positive-definite \acl{OVK}
    $K:\mathcal{X}\times\mathcal{X}\to\mathcal{L}(\mathcal{Y})$.
\end{proposition}
iven $x\in\mathcal{X}$,
$K_x:\mathcal{Y}\to\mathcal{F}(\mathcal{X};\mathcal{Y})$ denotes the linear
operator whose action on a vector $y$ is the function
$K_xy\in\mathcal{F}(\mathcal{X};\mathcal{Y})$ defined for all $z\in\mathcal{X}$
by $K_x=\text{ev}_x^\adjoint$. As a consequence we have that
\begin{dmath}
    \label{eq:trivial_feature_op}
    K(x,z)y\hiderel{=}\text{ev}_x\text{ev}_z^\adjoint y\hiderel{=}K_x^\adjoint
    K_zy\hiderel{=}(K_zy)(x).
\end{dmath}
Some direct consequences follow from the definition.
\begin{enumerate}
    \item The kernel reproduces the value of a function $f\in\mathcal{H}$ at a
    point $x\in\mathcal{X}$ since for all $y\in\mathcal{Y}$ and
    $x\in\mathcal{X}$, $\text{ev}_x^\adjoint y=K_xy=K(\cdot,x)y$ such that
    $\label{eq:reproducing_prop} \inner{f(x),y}_{\mathcal{Y}}
    \hiderel{=}\inner{f,K(\cdot, x)y}_{\mathcal{H}}
    \hiderel{=}\inner{K_x^*f,y}_{\mathcal{Y}}$.
    \item For all $x\in\mathcal{X}$ and all $f\in\mathcal{H}$,
    $\norm{f(x)}_{\mathcal{Y}}\le
    \sqrt{\norm{K(x,x)}_{\mathcal{Y},\mathcal{Y}}}\norm{f}_{\mathcal{H}}$. This
    comes from the fact that $\norm{K_x}_{\mathcal{Y} ,\mathcal{H}} =
    \norm{K_x^\adjoint}_{\mathcal{H}, \mathcal{Y} }=
    \sqrt{\norm{K(x,x)}_{\mathcal{Y}, \mathcal{Y}}}$ and the operator norm is
    sub-multiplicative.
\end{enumerate}
Additionally given a positive-definite \acl{OVK}, it defines a unique
\ac{vv-RKHS}.
\begin{proposition}[\citep{carmeli2006vector}]
    Given a positive-definite \acl{OVK}
    $K:\mathcal{X}\times\mathcal{X}\to\mathcal{L}(\mathcal{Y})$, there is a
    unique \acl{vv-RKHS} $\mathcal{H}$ on $\mathcal{X}$ with reproducing kernel
    $K$.
\end{proposition}
Since an positive-definite \acl{OVK} defines a unique \acf{vv-RKHS} and
conversely a \ac{vv-RKHS} defines a unique \acl{OVK}, we denote the Hilbert
space $\mathcal{H}$ endowed with the scalar product $\inner{\cdot,\cdot}$
respectively $\mathcal{H}_K$ and $\inner{\cdot,\cdot}_K$. From now we refer to
positive-definite \aclp{OVK} or reproducing \aclp{OVK} as \aclp{OVK}. As a
consequence, given $K$ an
\acl{OVK}, define $K_x=K(\cdot,x)$ we have
\begin{dgroup}
    \begin{dmath}
        \label{eq:kernel_operator_product}
        K(x,z)=K^\adjoint_x K_z \enskip\forall x,z\hiderel{\in}\mathcal{X}
    \end{dmath},
    \begin{dmath}
        \label{eq:span_RKHS}
        \mathcal{H}_K=\lspan\Set{K_x y | \forall
        x\hiderel{\in}\mathcal{X},\enskip\forall y\hiderel{\in}\mathcal{Y} }.
    \end{dmath}
\end{dgroup}
Where $\lspan$ is the closed span of a given set. Another way to describe
functions of $\mathcal{H}_K$ consists in using a suitable feature map.
\begin{proposition}[Feature Operator~\citep{Carmeli2010}]
    \label{pr:feature_operator} Let $\mathcal{H}$ be any Hilbert space and
    $\Phi:\mathcal{X}\to\mathcal{L}(\mathcal{Y};\mathcal{H})$, with $\Phi_x :=
    \Phi(x)$. Then the operator $W : \mathcal{H} \to \mathcal{F}(\mathcal{X};
    \mathcal{Y})$ defined for all $g \in\mathcal{H}$, and for all
    $x\in\mathcal{X}$ by $(W g)(x) = \Phi_x^\adjoint g$ is a partial isometry
    from $\mathcal{H}$ onto the \ac{vv-RKHS} $\mathcal{H}_K$ with reproducing
    kernel $K(x,z)=\Phi^\adjoint_x\Phi_z, \enskip \forall x,
    z\hiderel{\in}\mathcal{X}$.  $W^\adjoint W$ is the orthogonal projection
    onto $(\Ker W)^\perp = \lspan\Set{\Phi_x y | \forall
    x\hiderel{\in}\mathcal{X},\enskip\forall y\hiderel{\in}\mathcal{Y}}$.  Then
    $\label{eq:norm_relation_w} \norm{f}_K=\inf\Set{\norm{g}_{\mathcal{H}} |
    \forall g \in\mathcal{H},\enskip Wg=f}$.
\end{proposition}
In this work we mainly focus on the kernel functions inducing a \acs{vv-RKHS}
of continuous functions. Such kernel are named $\mathcal{Y}$-Mercer kernels and
generalize Mercer kernels.
\begin{definition}[$\mathcal{Y}$-Mercer kernel \citet{Carmeli2010}]
    A positive definite \acs{OVK}
    $K:\mathcal{X}\times\mathcal{X}\to\mathcal{L}(\mathcal{Y})$ is called
    $\mathcal{Y}$-Mercer if the associated \acs{vv-RKHS} $\mathcal{H}_K$ is a
    subspace of the space of continuous functions
    $\mathcal{C}(\mathcal{X};\mathcal{Y})$.
\end{definition}
\subsection{Examples of \aclp{OVK}}
\label{subsec:ovk-ex}
In this subsection we list some \acfp{OVK} that have been used successfully in
the litterature. We do not recall the proof that the following kernels are well
defined are refer the interrested reader to the respective authors original
work.
\paragraph{}
\acsp{OVK} have been first introduced in Machine Learning to solve multi-task
regression problems. Multi-task regression is encountered in many fields such
as structured classification when classes belong to a hierarchy for instance.
Instead of solving independently $p$ single output regression task, one would
like to take advantage of the relationships between output variables when
learning and making a decision.
\begin{proposition}[Decomposable kernel \citep{Micheli2013}]
    \label{dec-kernel}
    Let $\Gamma$ be a non-negative operator of $\mathcal{L}_+(\mathcal{Y})$.
    $K$ is said to be a \emph{decomposable}
    kernel\footnote{Some authors also refer to as \emph{separable} kernels.} if for
    all $(x,z) \in \mathcal{X}^2$, $K(x,z) \colonequals k(x,z)\Gamma$, where
    $k$ is a \emph{scalar} kernel.
\end{proposition}
When $\mathcal{Y}=\mathbb{R}^p$, the operator $\Gamma$ can be represented by a
matrix which can be interpreted as encoding the relationships between the
outputs coordinates.  If a graph coding for the proximity between tasks is
known, then it is shown in~\citet{Evgeniou2005,Baldassare2010,Alvarez2012} that
$\Gamma$ can be chosen equal to the pseudo inverse $L^{\dagger}$ of the graph
Laplacian such that the norm in $\mathcal{H}_K$ is a graph-regularizing penalty
for the outputs (tasks).  When no prior knowledge is available, $\Gamma$ can be
set to the empirical covariance of the output training data or learned with one
of the algorithms proposed in the literature~\citep{Dinuzzo2011, Sindhwani2013,
Lim2015}. Another interesting property of the decomposable kernel is its
universality (a kernel which may approximate an arbitrary continuous target
function uniformly on any compact subset of the input space). A reproducing
kernel $K$ is said \emph{universal} if the associated \ac{vv-RKHS}
$\mathcal{H}_K$ is \emph{dense} in the space of continuous functions
$\mathcal{C}(\mathcal{X},\mathcal{Y})$.  The conditions for a kernel to be
universal have been discussed in~\citet{caponnetto2008,Carmeli2010}. In
particular they show that a decomposable kernel is universal provided that the
scalar kernel $k$ is universal and the operator $\Gamma$ is injective.  Given
$(e_k)_{k=1}^p$ a basis of $\mathcal{Y}$, we recall here how the matrix
$\Gamma$ act as a regularizer between the components of the outputs $f_k =
\inner{f(\cdot), e_k}_{\mathcal{Y}}$ of a function $f\in\mathcal{H}_K$.  We
prove a generalized version of \cref{pr:kernel_reg} to any \acl{OVK} in
\cref{subsec:regularization_property}.
\begin{proposition}[Kernels and Regularizers~\citep{Alvarez2012}]
    \label{pr:kernel_reg}
    Let $K(x,z) \colonequals k(x,z)\Gamma$ for all $x$, $z\in\mathcal{X}$ be a
    decomposable kernel where $\Gamma$ is a matrix of size $p\times p$. Then
    for all $f\in\mathcal{H}_K$, $\norm{f}_K = \sum_{i,j=1}^p
    \left(\Gamma^\dagger\right)_{ij}\inner{f_i,f_j}_k$ where $f_i=\inner{f,e_i}$
    (resp $f_j=\inner{f,e_j}$), denotes the $i$-th (resp $j$-th) component of
    $f$.
\end{proposition}
\paragraph{}
Curl-free and divergence-free kernels provide an interesting application of
operator-valued kernels~\citep{Macedo2008, Baldassare2012, Micheli2013} to
\emph{vector field} learning, for which input and output spaces have the same
dimensions ($d=p$). Applications cover shape deformation
analysis \citep{Micheli2013} and magnetic fields
approximations \citep{Wahlstrom2013}. These kernels discussed in
\citep{Fuselier2006} allow encoding input-dependent similarities between
vector-fields. An illustration of a synthetic $2D$ curl-free and divergence
free fields are given respectively in \cref{fig:curl-field} and
\cref{fig:div-field}. To obain the curl-free field we took the gradient of
a mixture of five two dimensional Gaussians (since the gradient of a potential
is always curl-free). We generated the divergence-free field by taking the
orthogonal of the curl-free field.
\begin{figure}
    \begin{minipage}{.5\textwidth}
        \centering
        \includegraphics[trim=1.8cm 1cm 2cm 1cm,width=.8\textwidth,clip=true]{./gfx/curl_field.eps}
        \captionof{figure}{Synthetic $2D$ curl-free field \label{fig:curl-field}.}
    \end{minipage}%
    \begin{minipage}{0.5\textwidth}
        \centering
        \includegraphics[trim=1.8cm 1cm 2cm 1cm,width=.8\textwidth,clip=true]{./gfx/div_field.eps}
        \captionof{figure}{Synthetic $2D$ divergence-free field
        \label{fig:div-field}.}
    \end{minipage}
\end{figure}
\begin{proposition}[Curl-free and Div-free kernel \citep{Macedo2008}]
    \label{curl-div-free}
    Assume $\mathcal{X}=(\mathbb{R}^d, +)$ and $\mathcal{Y}=\mathbb{R}^p$ with
    $d=p$. The \emph{divergence-free} kernel is defined as
    $K^{div}(x,z)=K^{div}_0(\delta) \hiderel{=} (\nabla\nabla^\transpose  -
    \Delta I) k_0(\delta)$ and the \emph{curl-free} kernel as $K^{curl}(x,z)
    \hiderel{=} K_0^{curl}(\delta) =-\nabla\nabla^\transpose k_0(\delta)$,
    where $\nabla$ is the gradient operator
    %\footnote{See
    %\cref{subsec:gradient_methods} for a formal definition of the operator
    %$\nabla$.}
    , $\nabla\nabla^\transpose $ is the Hessian operator and $\Delta$ is the
    Laplacian operator.
\end{proposition}
\subsection{Shift-Invariant \acs{OVK} on \acs{LCA} groups}
The main subjects of interest of the present paper are shift-invariant
\acl{OVK}. When referring to a shift-invariant \ac{OVK}
$K:\mathcal{X}\times\mathcal{X}\to\mathcal{L}(\mathcal{Y})$ we assume that
$\mathcal{X}$ is a locally compact second countable topological group with
identity $e$.
\begin{definition}[Shift-invariant \ac{OVK}]
    A reproducing \acl{OVK}
    $K:\mathcal{X}\times\mathcal{X}\to\mathcal{L}(\mathcal{Y})$ is called
    shift-invariant if for all $x$, $z$, $t\in\mathcal{X}$, $K(x\groupop t,
    z\groupop t) = K(x, z)$.
\end{definition}
A shift-invariant kernel can be characterized by a function of one variable
$K_e$ called the signature of $K$. Here $e$ denotes the neutral element of the
\ac{LCA} group $\mathcal{X}$ endowed with the binary group operation
$\groupop$.
\begin{proposition}[Kernel signature~\citep{Carmeli2010}]
    \label{pr:kernel_signature} Let
    $K:\mathcal{X}\times\mathcal{X}\to\mathcal{L}(\mathcal{Y})$ be a
    reproducing kernel. The following conditions are equivalents.
    \begin{enumerate}
        \item \label{pr:kernel_signature_1} $K$ is a positive-definite
        shift-invariant \acl{OVK}.
        \item \label{pr:kernel_signature_2} There
        is a positive-definite function
        $K_e:\mathcal{X}\to\mathcal{L}(\mathcal{Y})$ such that
        $K(x,z)=K_e(\inv{z}\groupop x)$.
    \end{enumerate}
    If one of the above conditions is satisfied, then
    $\norm{K(x,x)}_{\mathcal{Y}}=\norm{K_e(e)}_{\mathcal{Y}}$, $\forall
    x\hiderel{\in}\mathcal{X}$.
\end{proposition}
The notation $K_e$ for the function of completely positive type associated with
the reproducing kernel $K$ is consistent with the definition given by
\cref{eq:trivial_feature_op} since for all $x\in\mathcal{X}$ and all
$y\in\mathcal{Y}$, $(K_ey)(x)=K_e(x)y$.  We recall that if $K$ is a
$\mathcal{Y}$-Mercer kernel, there is a function $K_e$ such that for all $x$
and $z\in\mathcal{X}$, $K(x, z)=K_e(x\groupop z^{-1})$. Then an \acs{OVK} $K$
is $\mathcal{Y}$-Mercer if and only if for all $y\in\mathcal{Y}$,
$K_e(\cdot)y\in\mathcal{C}(\mathcal{X};\mathcal{Y})$. In other words a
$\mathcal{Y}$-Mercer kernel is nothing but a functions whose signature is
continuous and positive definite \citep{Carmeli2010}, which fulfil the
conditions required for the \say{operator-valued} Bochner theorem to apply.
Note that if $K$ is a shift invariant $\mathcal{Y}$-Mercer kernel, then
$\mathcal{H}_K$ contains continuous \emph{bounded} functions
\citep{Carmeli2010}.
\subsection{Some applications of Operator-valued kernels}
We give here a non exhaustive list of works concerning \aclp{OVK}.  A good
review of \aclp{OVK} has been conducted in \citet{Alvarez2012}. For a
theoretical introduction to \acsp{OVK} the interested reader can refer to the
papers \citet{carmeli2006vector, caponnetto2008, Carmeli2010}. Generalization
bounds for \acs{OVK} have been studied in \citet{Sindhwani2013,
kadri2015operator,sangnier2016joint, maurer2016vector}.  Operator-valued Kernel
Regression has first been studied in the context of Ridge Regression and
Multi-task learning by \citet{Micchelli2005}.  Multi-task regression
\citep{micchelli2004kernels}  and structured multi-class classification
\citep{Dinuzzo2011,minh2013unifying,mroueh2012multiclass} are undoubtedly the
first target applications for working in \acl{vv-RKHS}.  \aclp{OVK} have been
shown useful to provide a general framework for structured output prediction
\citep{Brouard2011,Brouard2016_jmlr} with a link to Output Kernel Regression
\citep{Kadri_icml2013}. Beyond structured classification, other various
applications such as link prediction, drug activity prediction or recently
metabolite identification \citep{brouard2016fast} and  image colorization
\citep{ha2010image} have been developed.
\paragraph{}
\citet{Macedo2008, Baldassare2012} showed the interest of spectral algorithms
in Ridge regression and introduced vector field learning as a new multiple
output task in Machine Learning community. \citet{Wahlstrom2013} applied vector
field learning with \acs{OVK}-based Gaussian processes to the reconstruction of
magnetic fields (which are curl-free).  The works of
\citet{Kadri_aistat10,kadri2015operator} have been the precursors of
regression with functional values, opening a new avenue of applications.
Appropriate algorithms devoted to on-line learning have been also derived  by
\citet{audiffren2013online}.  Kernel learning was addressed at least in two
ways: first with using Multiple Kernel Learning in \citet{Kadri_nips2012} and
second, using various penalties, smooth ones in \citet{Dinuzzo2011,
ciliberto2015} for decomposable kernels and non smooth ones in
\citet{lim2015operator} using proximal methods in the case of decomposable and
transformable kernels.  Dynamical modeling was tackled in the context of
multivariate time series modelling in
\citet{lim2013okvar,Sindhwani2013,lim2015operator} and as a generalization of
Recursive Least Square Algorithm in \citet{amblard2015operator}.
\citet{sangnier2016joint} recently explored the minimization of a pinball loss
under regularizing constraints induced by a well chosen decomposable kernel in
order to handle joint quantile regression.

%%%%%%%%%%%%%%%%%%%%%%%%%%%%%%%%%%%%%%%%%%%%%%%%%%%%%%%%%%%%%%%%%%%%%%%%%%%%%%%

\section{Main contribution: Operator Random Fourier Features}
\label{sec:ORFF_construction}
We present in this section a construction methodology devoted to
shift-invariant $\mathcal{Y}$-Mercer operator-valued kernels defined on any
\acf{LCA} group, noted ($\mathcal{X}, \groupop)$, for some operation noted
$\groupop$. This allows us to use the general context of Pontryagin duality for
\acl{FT} of functions on \acs{LCA} groups. Building upon a generalization of
the celebrated Bochner's theorem for operator-valued measures, an
operator-valued kernel is seen as the \emph{\acl{FT}} of an operator-valued
positive measure. From that result, we extend the principle of \acs{RFF} for
scalar-valued kernels and derive a general methodology to build \acf{ORFF} when
operator-valued kernels are shift-invariant according to the chosen group
operation. Elements of this paper have been developped
in~\citet{brault2016random}.
\paragraph{}
We present a construction of feature maps called \acf{ORFF}, such that $f:
x\mapsto \tildePhi{\omega}(x)^\adjoint \theta$ is a continuous function that
maps an arbitrary \acs{LCA} group $\mathcal{X}$ as input space to an arbitrary
output Hilbert space $\mathcal{Y}$. First we define a functional \emph{Fourier
feature map}, and then propose a Monte-Carlo sampling from this feature map to
construct an approximation of a shift-invariant $\mathcal{Y}$-Mercer kernel.
Then, we prove the convergence of the kernel approximation
$\tilde{K}(x,z)=\tildePhi{\omega}(x)^\adjoint \tildePhi{\omega}(z)$ with high
probability on \emph{compact} subsets of the \acs{LCA} $\mathcal{X}$.
Eventually we conclude with some numerical experiments.
\subsection{Theoretical study}
\label{sec:theoretical_study}
The following proposition of~\citet{Zhang2012, neeb1998operator} extends
Bochner's theorem to any shift-invariant $\mathcal{Y}$-Mercer kernel. We give a
short intoduction to \acs{LCA} groups and abstract harmonic analysis in
\cref{sec:abstract_harmonic}. For the sake of simplicity, the reader can take
$\overline{\pairing{x,\omega}}=\overline{\exp(\iu\inner{x,
\omega}_2})=\exp(-\iu\inner{x,\omega}_2)$, when $x\in\mathcal{X}=(\mathbb{R}^d,
+)$.
\begin{proposition}[Operator-valued Bochner's theorem~\citep{Zhang2012,
neeb1998operator}]
    \label{pr:operator_valued_bochner}
    If a function $K$ from $\mathcal{X} \times \mathcal{X}$ to $\mathcal{Y}$ is
    a shift-invariant $\mathcal{Y}$-Mercer kernel on $\mathcal{X}$, then there
    exists a unique positive projection-valued measure $\dual{Q}:
    \mathcal{B}(\mathcal{X}) \to
    \mathcal{L}_+(\mathcal{Y})$ such that for all $x$, $z \in \mathcal{X}$,
    \begin{dmath}
        \label{eq:bochner-gen}
        K(x, z) = \int_{\dual{\mathcal{X}}} \conj{\pairing{x \groupop \inv{z},
        \omega}} d\dual{Q}(\omega),
    \end{dmath}
    where $\dual{Q}$ belongs to the set of all the projection-valued measures
    of bounded variation on the $\sigma$-algebra of Borel subsets of
    $\dual{\mathcal{X}}$. Conversely, from any positive operator-valued measure
    $\dual{Q}$, a shift-invariant kernel $K$ can be defined by
    \cref{eq:bochner-gen}.
\end{proposition}
Although this theorem is central to the spectral decomposition of
shift-invariant $\mathcal{Y}$-Mercer \acs{OVK}, the following results proved
by~\citet{Carmeli2010} provides insights about this decomposition that are more
relevant in practice. It first gives the necessary conditions to build
shift-invariant $\mathcal{Y}$-Mercer kernel with a pair $(A, \dual{\mu})$ where
$A$ is an operator-valued function on $\dual{\mathcal{X}}$ and $\dual{\mu}$ is
a real-valued positive measure on $\dual{\mathcal{X}}$. Note that obviously
such a pair is not unique and the choice of this paper may have an impact on
theoretical properties as well as practical computations.  Secondly it also
states that any \acs{OVK} have such a spectral decomposition when $\mathcal{Y}$
is finite dimensional or $\mathcal{X}$.
\begin{proposition}[\citet{Carmeli2010}]
    \label{pr:mercer_kernel_bochner}
    Let $\dual{\mu}$ be a positive measure on
    $\mathcal{B}(\mathcal{\dual{\mathcal{X}}})$ and $A: \dual{\mathcal{X}}\to
    \mathcal{L}(\mathcal{Y})$ such that $\inner{A(\cdot)y,y'}\in
    L^1(\mathcal{X},\dual{\mu})$ for all $y,y'\in\mathcal{Y}$ and
    $A(\omega)\succcurlyeq 0$ for $\dual{\mu}$-almost all
    $\omega\in\dual{\mathcal{X}}$. Then, for all $\delta \in \mathcal{X}$,
    \begin{dmath}
        \label{eq:AK0}
        K_e(\delta)=\int_{\dual{\mathcal{X}}} \conj{\pairing{\delta, \omega}}
        A(\omega) d\dual{\mu}(\omega)
    \end{dmath}
    is the kernel signature of a shift-invariant $\mathcal{Y}$-Mercer kernel
    $K$ such that $K(x,z)=K_e(x \groupop \inv{z})$. The \acs{vv-RKHS}
    $\mathcal{H}_K$ is embed in
    $L^2(\dual{\mathcal{X}},\dual{\mu};\mathcal{Y}')$ by means of the feature
    operator
    \begin{dmath}
        \label{eq:feature_operator}
        (Wg)(x)=\int_{\mathcal{\dual{X}}} \conj{\pairing{x,\omega}} B(\omega)
        g(\omega) d\dual{\mu}(\omega),
    \end{dmath}
    Where $B(\omega)B(\omega)^\adjoint=A(\omega)$ and both integrals converge
    in the weak sense. If $\mathcal{Y}$ is finite dimensional or $\mathcal{X}$
    is compact, any shift-invariant kernel is of the above form for some pair
    $(A, \dual{\mu})$.
\end{proposition}
\paragraph{}
When $p=1$ one can always assume $A$ is reduced to the scalar $1$, $\dual{\mu}$
is still a bounded positive measure and we retrieve the Bochner theorem applied
to the scalar case (\cref{th:bochner-scalar}).
\paragraph{}
\Cref{pr:mercer_kernel_bochner} shows that a pair $(A,\dual{\mu})$ entirely
characterizes an \acs{OVK}. Namely a given measure $\dual{\mu}$ and a function
$A$ such that $\inner{y', A(.)y}\in L^1(\mathcal{X},\dual{\mu})$ for all $y$,
$y'\in\mathcal{Y}$ and $A(\omega)\succcurlyeq 0$ for $\dual{\mu}$-almost all
$\omega$, give rise to an \acs{OVK}. Since $(A,\dual{\mu})$ determine a unique
kernel we can write
$\mathcal{H}_{(A,\dual{\mu})}{\scriptstyle\implies}\mathcal{H}_K$ where $K$ is
defined as in \cref{eq:AK0}. However the converse is not true: Given a
$\mathcal{Y}$-Mercer shift invariant \acl{OVK}, there exist infinitely many
pairs $(A,\dual{\mu})$ that characterize an \acs{OVK}.
\paragraph{}
The main difference between \cref{eq:bochner-gen} and
\cref{eq:AK0} is that the first one characterizes an \acs{OVK}
by a unique \acf{POVM}, while the second one shows that the \acs{POVM} that
uniquely characterizes a $\mathcal{Y}$-Mercer \acs{OVK} has an operator-valued
density with respect to a \emph{scalar} measure $\dual{\mu}$; and that this
operator-valued density is not unique.
\paragraph{}
Finally \cref{pr:mercer_kernel_bochner} does not provide any
\emph{constructive} way to obtain the pair $(A,\dual{\mu})$ that characterizes
an \acs{OVK}. The following \cref{subsec:sufficient_conditions} is based on
another proposition of~\citeauthor{carmeli2006vector} and shows that if the
kernel signature $K_e(\delta)$ of an $\acs{OVK}$ is in $L^1$ then it is
possible to construct \emph{explicitly} a pair $(C,\dual{\Haar})$ from it.
Additionally, we show that we can always extract a scalar-valued
\emph{probability} density function from $C$ such that we obtain a pair
$(A,\probability_{\dual{\mu},\rho})$ where $\probability_{\dual{\mu},\rho}$ is
a \emph{probability} distribution absolutely continuous with respect to
$\dual{\mu}$ and with associated \ac{pdf}~$\rho$. Thus for all
$\mathcal{Z}\subset\mathcal{B}(\dual{\mathcal{X}})$,
$\probability_{\dual{\mu},\rho}(\mathcal{Z})=\int_{\mathcal{Z}}
\rho(\omega)d\dual{\mu}(\omega)$.  When the reference measure $\dual{\mu}$ is
the Lebesgue measure, we note
$\probability_{\dual{\mu},\rho}=\probability_\rho$. For any function
$f:\mathcal{X}\times\dual{\mathcal{X}}\times\mathcal{Y}\to\mathbb{R}$, we also
use the notation $\expectation_{\dual{\Haar}, \rho}\left[f(x, \omega, y)\right]
=\expectation_{\omega\sim\probability_{\dual{\Haar},\rho}}\left[f(x, \omega,
y)\right] \hiderel{=}\int_{\dual{\mathcal{X}}} f(x, \omega, y)
d\probability_{\dual{\Haar}, \rho}(\omega) =\int_{\dual{\mathcal{X}}} f(x,
\omega, y)\rho(\omega) d\dual{\Haar}(\omega)$.  where the two last equalities
hold by the \say{law of the unconscious statistician} (change of variable
formula) and the fact that $\probability_{\dual{\Haar}, \rho}$ has density
$\rho$.

\subsubsection{Sufficient conditions of existence}
\label{subsec:sufficient_conditions}
While \cref{pr:mercer_kernel_bochner} gives some insights on how to build an
approximation of a $\mathcal{Y}$-Mercer kernel, we need a theorem that provides
an explicit construction of the pair $(A, \probability_{\dual{\mu},\rho})$ from
the kernel signature $K_e$. Proposition 14 in~\citet{Carmeli2010} gives the
solution, and also provides a sufficient condition for
\cref{pr:mercer_kernel_bochner} to apply.
\begin{proposition}[\citet{Carmeli2010}]
\label{pr:inverse_ovk_Fourier_decomposition}
    Let $K$ be a shift-invariant $\mathcal{Y}$-Mercer kernel of signature
    $K_e$.  Suppose that for all $z \in \mathcal{X}$ and for all $y$, $y'
    \in\mathcal{Y}$, the function $\inner{K_e(.)y,y'}_{\mathcal{Y}}\in
    L^1(\mathcal{X},\Haar)$ where $\mathcal{X}$ is endowed with the group law
    $\groupop$. Denote $C: \dual{X} \to \mathcal{L}(Y)$, the function defined
    for all $\omega \in \dual{\mathcal{X}}$ that satisfies for all $y$, $y'$ in
    $\mathcal{Y}$:
    \begin{dmath}\label{eq:CK0}
        \inner{y',C(\omega)y}_{\mathcal{Y}}= \int_{\mathcal{X}}
        \pairing{\delta, \omega}\inner{y',
        K_e(\delta)y}_{\mathcal{Y}}d\Haar(\delta) \hiderel{=} \IFT{\inner{y',
        K_e(\cdot)y}}_{\mathcal{Y}}(\omega).
    \end{dmath}
    Then
    \begin{enumerate}
        \item $C(\omega)$ is a bounded non-negative operator for all $\omega
        \in \dual{\mathcal{X}}$,
        \item $\inner{y, C(\cdot)y'}_{\mathcal{Y}}\in
        L^1\left(\dual{\mathcal{X}},\dual{\Haar}\right)$ for all
        $y,y'\in\mathcal{X}$,
        \item for all $\delta\in\mathcal{X}$ and for all $y$, $y'$ in
        $\mathcal{Y}$, $\inner{y', K_e(\delta)y}_{\mathcal{Y}}=
        \int_{\dual{\mathcal{X}}}\conj{\pairing{\delta,\omega}}\inner{y',
        C(\omega)y}_{\mathcal{Y}}d\dual{\Haar}(\omega) \hiderel{=}
        \FT{\inner{y', C(\cdot)y}_{\mathcal{Y}}}(\delta)$.
    \end{enumerate}
\end{proposition}
We found some confusion in the literature whether a kernel is the
\acl{FT} or \acl{IFT} of a measure. However \cref{lm:C_characterization}
clarifies the relation between the \acl{FT} and \acl{IFT} for a translation
invariant \acl{OVK}. Notice that in the real scalar case the \acl{FT} and
\acl{IFT} of a shift-invariant kernel are the same, while the difference is
significant for \acs{OVK}.  The following lemma is a direct consequence of the
definition of $C(\omega)$ as the \acl{FT} of the adjoint of $K_e$ and also
helps to simplify the definition of \acs{ORFF}.
\begin{lemma}
    \label{lm:C_characterization}
    Let $K_e$ be the signature of a shift-invariant $\mathcal{Y}$-Mercer kernel
    such that for all $y$, $y'\in\mathcal{Y}$, $\inner{y',
    K_e(\cdot)y}_{\mathcal{Y}}\in L^1(\mathcal{X},\Haar)$ and let $\inner{y',
    C(\cdot)y}_{\mathcal{Y}}=\IFT{\inner{y', K_e(\cdot)y}_{\mathcal{Y}}}$.
    Then
    \begin{enumerate}
        \item \label{lm:C_characterization_1} $C(\omega)$ is self-adjoint and
        $C$ is even.
        \item \label{lm:C_characterization_2} $\IFT{\inner{y',
        K_e(\cdot)y}_{\mathcal{Y}}} = \FT{\inner{y',
        K_e(\cdot)y}_{\mathcal{Y}}}$.
        \item \label{lm:C_characterization_3} $K_e(\delta)$ is self-adjoint and
        $K_e$ is even.
    \end{enumerate}
\end{lemma}
While \cref{pr:inverse_ovk_Fourier_decomposition} gives an explicit form of the
operator $C(\omega)$ defined as the \acl{FT} of the kernel $K$, it is not
really convenient to work with the Haar measure $\dual{\Haar}$ on
$\mathcal{B}(\dual{\mathcal{X}})$. However it is easily possible to turn
$\dual{\Haar}$ into a probability measure to allow efficient integration over
an infinite domain.
\paragraph{}
The following proposition allows to build a spectral decomposition of a
shift-invariant $\mathcal{Y}$-Mercer kernel on a \acs{LCA} group $\mathcal{X}$
endowed with the group law $\groupop$ with respect to a scalar probability
measure, by extracting a scalar probability density function from $C$.
\begin{proposition}[Shift-invariant $\mathcal{Y}$-Mercer kernel spectral
decomposition]
\label{pr:spectral}
    Let $K_e$ be the signature of a shift-invariant $\mathcal{Y}$-Mercer
    kernel. If for all $y$, $y' \in\mathcal{Y}$, $\inner{K_e(.)y,y'}\in
    L^1(\mathcal{X},\Haar)$ then there exists a positive probability measure
    $\probability_{\dual{\Haar},\rho}$ and an operator-valued function $A$ an
    such that for all $y,$ $y'\in\mathcal{Y}$,
    \begin{dmath}
        \label{eq:expectation_spec} \inner{y', K_e(\delta)y}
        =\expectation_{\dual{\Haar},\rho}\left[\conj{\pairing{\delta,
        \omega}}\inner{y', A(\omega)y}\right],
    \end{dmath}
    with $\inner{y', A(\omega)y}\rho(\omega) = \FT{\inner{y',
    K_e(\cdot)y}}(\omega)$.  Moreover
    \begin{enumerate}
        \item for all $y,$ $y'\in\mathcal{Y}$, $\inner{A(.)y,y'}\in
        L^1(\dual{\mathcal{X}}, \probability_{\dual{\Haar},\rho})$,
        \item $A(\omega)$ is non-negative for
        $\probability_{\dual{\Haar},\rho}$-almost all
        $\omega\in\dual{\mathcal{X}}$,
        \item $A(\cdot)$ and $\rho(\cdot)$ are even functions.
    \end{enumerate}
\end{proposition}
\subsection{Examples of spectral decomposition}
\label{subsec:dec_examples}
In this section we give examples of spectral decomposition for various
$\mathcal{Y}$-Mercer kernels, based on \cref{pr:spectral}.

\subsubsection{Gaussian decomposable kernel}
\label{par:gaussian_dec} Recall that a decomposable $\mathbb{R}^p$-Mercer
introduced in \cref{dec-kernel} has the form $K(x,z)=k(x,z)\Gamma$, where
$k(x,z)$ is a scalar Mercer kernel and $\Gamma\in\mathcal{L}(\mathbb{R}^p)$ is
a non-negative operator. Let us focus on
$K^{dec,gauss}_e(\cdot)=k_e^{gauss}(\cdot)\Gamma$, the Gaussian decomposable
kernel where $K_e^{dec, gauss}$ and $k_e^{gauss}$ are respectively the
signature of $K$ and $k$ on the additive group $\mathcal{X}=(\mathbb{R}^d,+)$
-- $\acs{ie}~\delta=x-z$ and $e=0$. The well known Gaussian kernel is defined
for all $\delta\in\mathbb{R}^d$ as follows
$k^{\text{gauss}}_0(\delta)\hiderel{=}\exp\left(
-\sigma^{-2}\norm{\delta}^2_2\right)/2$ where $\sigma \in \mathbb{R}_{>0}$ is
an hyperparameter corresponding to the bandwith of the kernel. The
--Pontryagin-- dual group of $\mathcal{X}=(\mathbb{R}^d,+)$ is
$\dual{\mathcal{X}}\cong(\mathbb{R}^d,+)$ with the pairing
$\pairing{\delta,\omega}=\exp\left(\iu\inner{\delta,\omega}\right)$ where
$\delta$ and $\omega\in\mathbb{R}^d$. In this case the Haar measures on
$\mathcal{X}$ and $\dual{\mathcal{X}}$ are in both cases the Lebesgue measure.
However in order to have the property that $\IFT{\FT{f}}=f$ and
$\IFT{f}=\mathcal{R}\FT{f}$ one must normalize both measures by
$\sqrt{2\pi}^{-d}$, \acs{ie}~for all
$\mathcal{Z}\in\mathcal{B}\left(\mathbb{R}^d\right)$,
$\sqrt{2\pi}^{d}\Haar(\mathcal{Z}) = \Leb(\mathcal{Z})$ and
$\sqrt{2\pi}^{d}\dual{\Haar}(\mathcal{Z}) = \Leb(\mathcal{Z})$.  Then the
\acl{FT} on $(\mathbb{R}^d,+)$ is
\begin{dmath*}
    \FT{f}(\omega)
    =\int_{\mathbb{R}^d} \exp\left(-\iu\inner{\delta, \omega}\right) f(\delta)
    d\Haar(\delta)
    \hiderel{=}\int_{\mathbb{R}^d} \exp\left(-\iu\inner{\delta, \omega}\right)
    f(\delta) \frac{d\Leb(\delta)}{\sqrt{2\pi}^d}.
\end{dmath*}
Since $k^{\text{gauss}}_0\in L^1$ and $\Gamma$ is bounded, it is possible to
apply \cref{pr:spectral}, and obtain for all $y$ and $y'\in\mathcal{Y}$,
\begin{dmath*}
    \inner*{y',C^{dec,gauss}(\omega)y}
    =\FT{\inner*{y',K^{dec,gauss}_0(\cdot)y}}(\omega)
    \hiderel{=}\FT{k_0^{gauss}}(\omega)\inner*{y', \Gamma y}.
\end{dmath*}
Thus
\begin{dmath*}
    C^{dec,gauss}(\omega)
    =\int_{\mathbb{R}^d}\exp\left(-\iu\inner{\omega, \delta} -
    \frac{\norm{\delta}^2_2}{2\sigma^2}\right)
    \frac{d\Leb(\delta)}{\sqrt{2\pi}^d} \Gamma.
\end{dmath*}
Hence
\begin{dmath*}
    C^{dec,gauss}(\omega)
    =\underbrace{\frac{1}{\sqrt{2\pi\frac{1}{\sigma^2}}^d}\exp\left(
    -\frac{\sigma^2}{2}\norm{\omega}^2_2\right)\sqrt{2\pi}^d}_{\rho(\cdot)
    =\mathcal{N}(0,\sigma^{-2}I_d)\sqrt{2\pi}^d}\underbrace{\Gamma}_{A(\cdot)
    =\Gamma}.
\end{dmath*}
Therefore the canonical decomposition of $C^{dec,gauss}$ is
$A^{dec,gauss}(\omega)=\Gamma$ and
$\rho^{dec,gauss}=\mathcal{N}(0,\sigma^{-2}I_d)\sqrt{2\pi}^d$, where
$\mathcal{N}$ is the Gaussian probability distribution. Note that this
decomposition is done with respect to the \emph{normalized} Lebesgue measure
$\dual{\Haar}$, meaning that for all
$\mathcal{Z}\in\mathcal{B}(\dual{\mathcal{X}})$,
\begin{dmath*}
    \probability_{\dual{\Haar}, \mathcal{N}(0, \sigma^{-2} I_d)
    \sqrt{2\pi}^d}(\mathcal{Z})
    =\int_{\mathcal{Z}} \mathcal{N}(0, \sigma^{-2} I_d) \sqrt{2\pi}^d
    d\dual{\Haar}(\omega)
    =\int_{\dual{\mathcal{X}}}\mathcal{N}(0,\sigma^{-2}I_d)d\Leb(\omega)
    \hiderel{=}\probability_{\mathcal{N}(0,\sigma^{-2}I_d)}(\mathcal{Z}).
\end{dmath*}
Thus, the same decomposition with respect to the usual --non-normalized--
Lebesgue measure $\Leb$ yields
\begin{dgroup}
    \begin{dmath}
        A^{dec,gauss}(\cdot)=\Gamma
    \end{dmath}
    \begin{dmath}
        \rho^{dec,gauss}=\mathcal{N}(0,\sigma^{-2}I_d).
    \end{dmath}
\end{dgroup}
\subsubsection{Skewed-$\chi^2$ decomposable kernel}
\label{subsubsec:skewedchi2} The skewed-$\chi^2$ scalar kernel
\citep{li2010random}, useful for image processing, is defined on the \acs{LCA}
group $\mathcal{X}=(-c_k;+\infty)_{k=1}^d$, with $c_k\in\mathbb{R}_{>0}$ and
endowed with the group operation $\odot$. Let $(e_k)_{k=1}^d$ be the standard
basis of $\mathcal{X}$. The operator $\odot:
\mathcal{X}\times\mathcal{X}\to\mathcal{X}$ is defined by $x\odot z =
\left((x_k + c_k)(z_k + c_k) - c_k\right)_{k=1}^d$.  The identity element $e$
is $\left(1-c_k\right)_{k=1}^d$ since $(1-c) \odot x = x$. Thus the inverse
element $x^{-1}$ is $((x_k+c_k)^{-1} - c_k)_{k=1}^d$. The skewed-$\chi^2$
scalar kernel reads
\begin{dmath}
    k^{skewed}_{1-c}(\delta)
    =\prod_{k=1}^d\frac{2}{\sqrt{\delta_k+c_k}+\sqrt{\frac{1}{\delta_k+c_k}}}.
\end{dmath}
The dual of $\mathcal{X}$ is $\dual{\mathcal{X}}\cong\mathbb{R}^d$ with the
pairing $\pairing{\delta,\omega}
=\prod_{k=1}^d\exp\left(\iu\log(\delta_k+c_k)\omega_k\right)$.  The Haar
measure are defined for all $\mathcal{Z}\in\mathcal{B}((-c;+\infty)^d)$ and all
$\dual{\mathcal{Z}}\in\mathcal{B}(\mathbb{R}^d)$ by
$\sqrt{2\pi}^d\Haar(\mathcal{Z})
=\int_{\mathcal{Z}}\prod_{k=1}^d\frac{1}{z_k+c_k}d\Leb(z)$ and
$\sqrt{2\pi}^d\dual{\Haar}(\dual{\mathcal{Z}})=\Leb(\dual{\mathcal{Z}})$. Thus
the \acl{FT} is
\begin{dmath*}
    \FT{f}(\omega)
    =\int_{(-c;+\infty)^d}\prod_{k=1}^d\frac{\exp\left(-\iu
    \log(\delta_k+c_k)\omega_k\right)}{\delta_k +
    c_k}f(\delta)\frac{d\Leb(\delta)}{\sqrt{2\pi}^d}.
\end{dmath*}
Then, applying Fubini's theorem over product space, and the fact that each
dimension is independent
\begin{dmath*}
    \FT{k_0^{skewed}}(\omega)
    =\prod_{k=1}^d\int_{-c_k}^{+\infty}\frac{2\exp\left(-\iu
    \log(\delta_k+c_k)\omega_k\right)}{(\delta_k +
    c_k)\left(\sqrt{\delta_k + c_k} + \sqrt{\frac{1}{\delta_k + c_k}}\right)}
    \frac{d\Leb(\delta_k)}{\sqrt{2\pi}^d}.
\end{dmath*}
Making the change of variable $t_k=(\delta_k+c_k)^{-1}$ yields
\begin{dmath*}
    \FT{k_0^{skewed}}(\omega)
    = \prod_{k=1}^d\int_{-\infty}^{+\infty} \frac{2 \exp\left(-\iu t_k
    \omega_k\right)}{\exp\left(\frac{1}{2} t_k \right) + \exp\left(-\frac{1}{2}
    t_k \right)} \frac{d\Leb(t_k)}{\sqrt{2\pi}^d}
    \hiderel{=}\sqrt{2\pi}^d\prod_{k=1}^d\sech(\pi\omega_k).
\end{dmath*}
Since $k^{\text{skewed}}_{1-c}\in L^1$ and $\Gamma$ is bounded, it is possible
to apply \cref{pr:spectral}, and obtain
\begin{dmath*}
    C^{dec,skewed}(\omega)
    =\FT{k_{1-c}^{skewed}}(\omega)\Gamma
    \hiderel{=}\underbrace{\sqrt{2\pi}^d\prod_{k=1}^d\sech(\pi
    \omega_k)}_{\rho(\cdot) = \mathcal{S}(0, 2^{-1})^d\sqrt{2
    \pi}^d}\underbrace{\Gamma}_{A(\cdot)}.
\end{dmath*}
Hence the decomposition with respect to the usual --non-normalized-- Lebesgue
measure $\Leb$ yields
\begin{dgroup}
    \begin{dmath}
        A^{dec,skewed}(\cdot)=\Gamma
    \end{dmath}
    \begin{dmath}
        \rho^{dec,skewed}=\mathcal{S}\left(0,2^{-1}\right)^d.
    \end{dmath}
\end{dgroup}
\subsubsection{Curl-free Gaussian kernel}
The curl-free Gaussian kernel is defined as
$K^{curl,gauss}_0=-\nabla\nabla^\transpose k_0^{gauss}$. Here
$\mathcal{X}=(\mathbb{R}^d, +)$ so the setting is the same than
\cref{par:gaussian_dec}.
\begin{dmath*}
    C^{curl,gauss}(\omega)_{ij}=
    \FT{K^{curl,gauss}_{1-c}(\cdot)_{ij}}(\omega)
    %=\FT{-\frac{d^2}{d\delta_id\delta_j}k^{gauss}_0}(\omega)
    %=-(\iu\omega_i)(\iu\omega_j)\FT{k_0^{gauss}}(\omega)
    %=\omega_i\omega_j\FT{k_0^{gauss}}(\omega)
    =\sqrt{2\pi\frac{1}{\sigma^2}}^d\exp\left(
    -\frac{\sigma^2}{2}\norm{\omega}^2_2\right)\sqrt{2\pi}^d\omega_i\omega_j.
\end{dmath*}
Hence
\begin{dmath*}
    C^{curl,gauss}(\omega)
    =\underbrace{\frac{1}{\sqrt{2\pi\frac{1}{\sigma^2}}^d}\exp\left(
    -\frac{\sigma^2}{2} \norm{\omega}^2_2\right) \sqrt{2\pi}^d}_{\mu(\cdot)
    =\mathcal{N}(0 ,\sigma^{-2} I_d)\sqrt{2 \pi}^d}
    \underbrace{\omega\omega^\transpose
    }_{A(\omega)=\omega\omega^\transpose }.
\end{dmath*}
Here a canonical decomposition is
$A^{curl,gauss}(\omega)=\omega\omega^\transpose $ for all
$\omega\in\mathbb{R}^d$ and
$\mu^{curl,gauss}=\mathcal{N}(0,\sigma^{-2}I_d)\sqrt{2\pi}^d$ with respect to
the normalized Lebesgue measure $d\omega$. Again the decomposition with respect
to the usual --non-normalized-- Lebesgue measure is for all
$\omega\in\mathbb{R}^d$
\begin{dgroup}
    \begin{dmath}
        A^{curl,gauss}(\omega)=\omega\omega^\transpose
    \end{dmath}
    \begin{dmath}
        \mu^{curl,gauss}=\mathcal{N}(0,\sigma^{-2}I_d).
    \end{dmath}
\end{dgroup}
\subsubsection{Divergence-free kernel}
The divergence-free Gaussian kernel is defined as
$K^{div,gauss}_0=(\nabla\nabla^\transpose -\Delta)k_0^{gauss}$ on the group
$\mathcal{X}=(\mathbb{R}^d, +)$. The setting is the same than
\cref{par:gaussian_dec}. Hence
\begin{dmath*}
    C^{div,gauss}(\omega)_{ij}
    = \FT{K^{div,gauss}_0(\cdot)_{ij}}(\omega)
    %= \FT{\frac{\partial^2}{\partial \delta_i \partial \delta_j}k^{gauss}_0 -
    %\delta_{i=j} \sum_{k=1}^d \frac{\partial^2}{\partial \delta_k
    %\partial\delta_k} k^{gauss}_0}(\omega)
    %= \left(-(\iu \omega_i)(\iu \omega_j) - \delta_{i=j}\sum_{k=1}^d(\iu
    %\omega_k)^2\right) \FT{k_0^{gauss}}
    \hiderel{=}\left(\delta_{i=j}\sum_{k=1}^d \omega_k^2 - \omega_i
    \omega_j\right) \FT{k_0^{gauss}}(\omega).
\end{dmath*}
Hence
\begin{dmath*}
    C^{div,gauss}(\omega)
    = \underbrace{\frac{1}{\sqrt{2\pi\frac{1}{\sigma^2}}^d}\exp\left(
    -\frac{\sigma^2}{2} \norm{\omega}^2_2\right) \sqrt{2\pi}^d}_{\rho(\cdot) =
    \mathcal{N}(0, \sigma^{-2} I_d)\sqrt{2 \pi}^d}\underbrace{\left(I_d
    \norm{\omega}_2^2 - \omega \omega^\transpose
    \right)}_{A(\omega)=I_d\norm{\omega}_2^2-\omega\omega^\transpose }.
\end{dmath*}
Thus the canonical decomposition with respect to the normalized Lebesgue
measure is $A^{div,gauss}(\omega)=I_d\norm{\omega}_2^2-\omega\omega^\transpose
$ and the measure
    $\rho^{div,gauss}=\mathcal{N}(0,\sigma^{-2}I_d)\sqrt{2\pi}^d$.
The canonical decomposition with respect to the usual Lebesgue measure is
\begin{dgroup}
    \begin{dmath}
        A^{div,gauss}(\omega)=I_d\norm{\omega}_2^2-\omega\omega^\transpose
    \end{dmath}
    \begin{dmath}
        \rho^{div,gauss}=\mathcal{N}(0,\sigma^{-2}I_d).
    \end{dmath}
\end{dgroup}

\subsection{Operator-valued Random Fourier Features (ORFF)}
\label{sec:building_ORFF}
\subsubsection{Building Operator-valued Random Fourier Features}
As shown in \cref{pr:spectral} it is
always possible to find a pair $(A, \probability_{\dual{\Haar},\rho})$ from a
shift invariant $\mathcal{Y}$-Mercer \acl{OVK} $K_e$ such that
$\probability_{\dual{\Haar},\rho}$ is a probability measure, \acs{ie}
$\int_{\dual{\mathcal{X}}} \rho d\dual{\Haar}=1$ where $\rho$ is the density of
$\probability_{\dual{\Haar},\rho}$ and
$K_e(\delta)=\expectation_{\dual{\Haar},
\rho}{\conj{\pairing{\delta,\omega}}A(\omega)}$. In order to obtain an
approximation of $K$ from a decomposition $(A,
\probability_{\dual{\Haar},\rho})$ we turn our attention to a Monte-Carlo
estimation of the expectation in \cref{eq:expectation_spec} characterizing a
$\mathcal{Y}$-Mercer shift-invariant \acl{OVK}. % \paragraph{}
%  \probability_{\dual{\Haar},\rho}
\begin{proposition}
    \label{cr:ORFF-kernel} Let $K(x,z)$ be a shift-invariant
    $\mathcal{Y}$-Mercer kernel with signature $K_e$ such that for all $y$,
    $y'\in\mathcal{Y}$, $\inner{y', K_e(\cdot)y}\in L^1(\mathcal{X},\Haar)$.
    Then one can find a pair $(A, \probability_{\dual{\Haar},\rho})$ that
    satisfies \cref{pr:spectral}. \acs{ie} for
    $\probability_{\dual{\Haar},\rho}$-almost all $\omega$, and all $y,
    y'\in\mathcal{Y}$, $\inner{y, A(\omega)y'}\rho(\omega)=\FT{\inner{y',
    K_e(\cdot)y}}(\omega)$.  If $(\omega_j)_{j=1}^D$ be a sequence of
    $D\in\mathbb{N}^*$ \acs{iid}~random variables following the law
    $\probability_{\dual{\Haar},\rho}$ then the operator-valued function
    $\tilde{K}$ defined for $(x,z) \in \mathcal{X}\times \mathcal{X}$ as
    \begin{dmath*}
        \tilde{K}(x,z)= \frac{1}{D} \sum_{j=1}^D
        \conj{\pairing{x\groupop\inv{z},\omega_j}} A(\omega_j)
    \end{dmath*}
    is an approximation of $K$. \acs{ie} it satisfies for all $x$,
    $z\in\mathcal{X}$, $\tilde{K}(x, z) \converges{\acs{asurely}}{D\to\infty}
    K(x, z)$ in the weak operator topology, where $K$ is a $\mathcal{Y}$-Mercer
    \acs{OVK}.
\end{proposition}
Now, for efficient computations as motivated in the introduction, we are
interested in finding an approximated \emph{feature map} instead of a kernel
approximation. Indeed, an approximated feature map will allow to build linear
models in regression tasks. The following proposition deals with the feature
map construction.
\begin{proposition}
    \label{cr:ORFF-map-kernel} Assume the same conditions as
    \cref{cr:ORFF-kernel}. Moreover, if one can define $B: \dual{\mathcal{X}}
    \to \mathcal{L}(\mathcal{Y}',\mathcal{Y})$ such that for
    %such that for all
    %$y\in\mathcal{Y}$ and all $y' \in\mathcal{Y}'$,
    %%$\inner{y, B(\cdot)y'} \in
    %L^2(\dual{\mathcal{X}}, \probability_{\dual{\Haar},\rho})$,  and for
    $\probability_{\dual{\Haar},\rho}$-almost all $\omega$, and all $y,
    y'\in\mathcal{Y}$, $\inner{y, B(\omega)B(\omega)^\adjoint y'}\rho(\omega)=
    \inner{y, A(\omega)y'}\rho(\omega) \hiderel{=} \FT{\inner{y,
    K_e(\cdot)y'}}(\omega)$, then the function
    $\tildePhi{\omega}:\dual{\mathcal{X}} \to \mathcal{L}(\mathcal{Y},
    \Vect_{j=1}^D \mathcal{Y'})$ defined for all $y \in \mathcal{Y}$ as
    follows:
    \begin{dmath*}
        \tildePhi{\omega}(x)y
        = \frac{1}{\sqrt{D}}\Vect_{j=1}^D\pairing{x,
        \omega_j}B(\omega_j)^\adjoint y \condition{$\omega_j \sim
        \probability_{\dual{\Haar},\rho}$ \ac{iid},}
    \end{dmath*}
    is an approximated feature map\footnote{\acs{ie}~it satisfies
    for all $x,z\in\mathcal{X}$, $\tildePhi{\omega}(x)^\adjoint
    \tildePhi{\omega}(z)\converges{\acs{asurely}}{D\to\infty}K(x,z)$ in the
    weak operator topology, where $K$ is a $\mathcal{Y}$-Mercer \acs{OVK}.} for
    the kernel $K$.
\end{proposition}
\begin{remark}
    We find a decomposition such that
    $A(\omega_j)=B(\omega_j)B(\omega_j)^\adjoint $ for all $j\in\mathbb{N}^*_D$
    either by exhibiting a closed-form or using a numerical decomposition.
    Such a decomposition always exists since $A(\omega)$ is positive
    semi-definite for all $\omega\in\dual{\mathcal{X}}$.
\end{remark}
Notice that an \acs{ORFF} map as defined in \cref{cr:ORFF-map-kernel} is also
the Monte-Carlo sampling of the corresponding functional Fourier feature map
$\Phi_x: \mathcal{Y} \to L^2(\dual{\mathcal{X}}, \probability_{\dual{\Haar},
\rho}; \mathcal{Y}' )$ as defined in \cref{pr:fourier_feature_map}.  Indeed,
for all $y\in\mathcal{Y}$ and all $x\in\mathcal{X}$, $\tildePhi{\omega}(x)y =
\Vect_{j=1}^D (\Phi_x y)(\omega_j) \condition{$\omega_j \sim
\probability_{\dual{\Haar}, \rho}$ \acs{iid}}$ \Cref{cr:ORFF-map-kernel} allows
us to define \cref{alg:ORFF_construction} for constructing \acs{ORFF} from an
operator valued kernel.
\SetKwInOut{Input}{Input}
\SetKwInOut{Output}{Output}
\begin{center}
    \begin{algorithm2e}[t!]\label{alg:ORFF_construction}
        \SetAlgoLined
        \Input{$K(x, z)=K_e(\delta)$ a shift-invariant $\mathcal{Y}$-Mercer
        kernel on $(\mathcal{X}, \groupop)$ such that $\forall
        y,y'\in\mathcal{Y},$ $\inner{y', K_e(\cdot)y}\in L^1(\mathbb{R}^d,
        \Haar)$ and $D$ the number of features.}
        \Output{A random feature $\tildePhi{\omega}(x)$ such that
        $\tildePhi{\omega}(x)^\adjoint \tildePhi{\omega}(z) \approx K(x,z)$}
        \BlankLine
        Define the pairing $\pairing{x, \omega}$ from the \acs{LCA} group
        $(\mathcal{X}, \groupop)$\;
        Find a decomposition $(A,\probability_{\dual{\Haar},\rho})$ and $B$
        such that $B(\omega)B(\omega)^\adjoint \rho(\omega)=
        A(\omega)\rho(\omega)\hiderel{=}\IFT{K_e}(\omega)$\;
        \nl Draw $D$ \acs{iid} realizations $(\omega_j)_{j=1}^D$ from the
        probability distribution $\probability_{\dual{\Haar},\rho}$\;
        \nl \Return
        $\begin{cases}
            \tildePhi{\omega}(x) \in \mathcal{L}(\mathcal{Y}, \tildeH{\omega})
            &: y \mapsto \frac{1}{\sqrt{D}}\Vect_{j=1}^D\pairing{x,
            \omega_j}B(\omega_j)^\adjoint y \\
            \tildePhi{\omega}(x)^\adjoint \in\mathcal{L}(\tildeH{\omega},
            \mathcal{Y}) &: \theta \mapsto \frac{1}{\sqrt{D}} \sum_{j=1}^D
            \pairing{x, \omega_j}B(\omega_j)\theta_j
        \end{cases}$\;
        \caption{Construction of \acs{ORFF} from \acs{OVK}}
    \end{algorithm2e}
\end{center}
\paragraph{}
We give a numerical illustration of different $\tilde{K}$ built from different
\acs{iid} realization $(\omega_j)_{j=1}^D$,
$\omega_j\sim\probability_{\dual{\Haar},\rho}$. In \cref{fig:not_Mercer}, we
represent the approximation of  a reference function (black line) defined as
$(y_1, y_2)^\transpose = f(x_i)=\sum_{j=1}^{250}\mathbf{K}_{ij}u_j$ where
$u_j\sim\mathcal{N}(0,I_2)$ and $K$ is a Gaussian decomposable kernel.  We took
$\Gamma=.5 I_2 +.5 1_2$ such that the outputs $y_1$ and $y_2$ share some
similarities.  We generated $250$ points equally separated on the segment
$(-1;1)$. The Gram matrix is then $\mathbf{K}_{ij}=\exp\left(-\frac{(x_i -
x_j)^2}{2(0.1)^2}\right)\Gamma \condition{for $i$, $j\in\mathbb{N}^*_{250}$.}$
We took $\Gamma=.5 I_2 +.5 1_2$ such that the outputs $y_1$ and $y_2$ share
some similarities.  We generated $250$ points equally separated on the segment
$(-1;1)$.
\begin{pycode}[not_mercer]
sys.path.append('./src/')
import not_mercer

not_mercer.main()
\end{pycode}
\begin{figure}[t]
    \pyc{print(r'\centering\resizebox{.85\textwidth}{!}{\input{./not_Mercer.pgf}}')}
    \caption[Approximation of a function in a vv-RKHS using different
    realizations of Operator Random Fourier Feature]{Approximation of a
    function in a VV-RKHS using different realizations of Operator Random
    Fourier Feature.
    Top row and bottom row correspond to two different realizations of
    $\tildeK{\omega}$, which are \emph{different} \acl{OVK}. However when $D$
    tends to infinity, the different realizations of $\tildeK{\omega}$ yield
    the same \acs{OVK}.}
    \label{fig:not_Mercer}
\end{figure}
Then we computed an approximate kernel matrix $\tilde{\mathbf{K}}\approx
\mathbf{K}$ for $25$ increasing values of $D$ ranging from $1$ to $10^4$. The
two graphs in \cref{fig:not_Mercer} on the top row shows that the more the
number of features increases the closer the model
$\widetilde{f}(x_i)=\sum_{j=1}^{250}\tilde{\mathbf{K}}_{ij}u_j$ is to $f$.
The bottom row shows the same experiment but for a different realization of
$\tilde{\mathbf{K}}$. When $D$ is small the curves of the bottom and top rows
are very dissimilar --and sine wave like-- while they both converge to $f$ when
$D$ increase.  We introduce a \emph{functional} feature map, we call
\emph{Fourier Feature map}, defined by the following proposition as a direct
consequence of \cref{pr:mercer_kernel_bochner}.
\begin{proposition}[Functional Fourier feature map]
    \label{pr:fourier_feature_map} Let $\mathcal{Y}$ and $\mathcal{Y}'$ be two
    Hilbert spaces. If there exists an operator-valued function
    $B:\dual{\mathcal{X}}\to\mathcal{L}(\mathcal{Y},\mathcal{Y}')$ such that
    for all $y$, $y'\in\mathcal{Y}$, $\inner{y, B(\omega)B(\omega)^\adjoint
    y'}_{\mathcal{Y}} =\inner{y', A(\omega)y}_{\mathcal{Y}}$
    $\dual{\mu}$-almost everywhere and $\inner{y', A(\cdot)y}\in
    L^1(\dual{\mathcal{X}},\dual{\mu})$ then the operator $\Phi_x$ defined for
    all $y$ in $\mathcal{Y}$ by $(\Phi_x
    y)(\omega)=\pairing{x,\omega}B(\omega)^\adjoint y$, is \emph{a feature
    map}\footnote{\acs{ie}~it satisfies for all $x$, $z \in \mathcal{X}$,
    $\Phi_x^\adjoint \Phi_z=K(x,z)$ where $K$ is a $\mathcal{Y}$-Mercer
    \acs{OVK}.} of some shift-invariant $\mathcal{Y}$-Mercer kernel $K$.
\end{proposition}
With this notation we have $\Phi: \mathcal{X} \to \mathcal{L}(\mathcal{Y};
L^2(\dual{\mathcal{X}}, \dual{\mu}; \mathcal{Y}'))$ such that $\Phi_x\in
\mathcal{L}(\mathcal{Y}; L^2(\dual{\mathcal{X}}, \dual{\mu}; \mathcal{Y}'))$
where $\Phi_x\colonequals\Phi(x)$.
\begin{figure}[t]
    \centering
    \resizebox{\textwidth}{!}{%
    \input{./gfx/feature_relationship.tikz}}
    \caption[Relationships between feature-maps.]{Relationships between
    feature-maps. For any realization of
    $\omega_j\sim\probability_{\dual{\Haar},\rho}$ \ac{iid},
    $\tildeH{\omega} = \Vect_{j=1}^D \mathcal{Y}'$.}
    \label{fig:rel_features}
\end{figure}
Notice that an \acs{ORFF} map as defined in \cref{cr:ORFF-map-kernel} is also
the Monte-Carlo sampling of the corresponding functional Fourier feature map
$\Phi_x: \mathcal{Y} \to L^2(\dual{\mathcal{X}}, \probability_{\dual{\Haar},
\rho}; \mathcal{Y}' )$ as defined in \cref{pr:fourier_feature_map}.  Indeed,
for all $y\in\mathcal{Y}$ and all $x\in\mathcal{X}$,
\begin{dmath*}
    \tildePhi{\omega}(x)y = \Vect_{j=1}^D (\Phi_x y)(\omega_j)
    \condition{$\omega_j \sim \probability_{\dual{\Haar}, \rho}$ \acs{iid}}
\end{dmath*}

\subsection{From Operator Random Fourier Feature maps to OVKs}
It is also interesting to notice that we can go the other way and define from
the general form of an \acl{ORFF}, an operator-valued kernel.
\begin{proposition}[Operator Random Fourier Feature map]
    \label{pr:ORFF-map} Let $\mathcal{Y}$ and $\mathcal{Y}'$ be two Hilbert
    spaces. If one defines an operator-valued function on the dual of a LCA
    group $\mathcal{X}$, $B: \dual{\mathcal{X}} \to
    \mathcal{L}(\mathcal{Y},\mathcal{Y}')$, and a probability measure
    $\probability_{\dual{\Haar},\rho}$ on $\mathcal{B}(\dual{\mathcal{X}})$,
    such that for all $y\in\mathcal{Y}$ and all $y'\in\mathcal{Y}'$, $\inner{y,
    B(\cdot)y'} \in L^2(\dual{\mathcal{X}}, \probability_{\dual{\Haar},\rho})$,
    then the operator-valued function $\tildePhi{\omega}: \mathcal{X}
    \hiderel{\to} \mathcal{L}\left(\mathcal{Y},
    \vect_{j=1}^D\mathcal{Y}'\right)$ defined for all $x \in \mathcal{X}$ and
    for all $y\in\mathcal{Y}$ by
    \begin{dmath}
        \label{eq:phitilde}
        \tildePhi{\omega}(x)y
        = \frac{1}{\sqrt{D}}\Vect_{j=1}^D\pairing{x,
        \omega_j}B(\omega_j)^\adjoint y\condition{$\omega_j \sim
        \probability_{\dual{\Haar, \rho}}$, \acs{iid}, }
    \end{dmath}
    is an approximated feature map of some $\mathcal{Y}$-Mercer operator-valued
    kernel\footnote{\acs{ie}~it satisfies $\tildePhi{\omega}(x)^\adjoint
    \tildePhi{\omega}(z)\converges{\acs{asurely}}{D\to\infty}K(x,z)$ in the
    weak operator topology, where $K$ is a $\mathcal{Y}$-Mercer \acs{OVK}}.
\end{proposition}
The difference between \cref{pr:ORFF-map} and \cref{cr:ORFF-map-kernel} is that
in \cref{pr:ORFF-map} we do not assume that $A(\omega)$ and
$\probability_{\dual{Haar}, \rho}$ have been obtained from \cref{pr:spectral}.
We conclude by showing that any realization of an approximate feature map gives
a proper operator valued kernel. Hence we can always view $\tilde{K}(x,
z)=\tildePhi{\omega}(x)^\adjoint \tildePhi{\omega}(z)$ ---where
$\tildePhi{\omega}$ is defined as in \cref{cr:ORFF-kernel} (construction from
an \acs{OVK}) or \cref{pr:ORFF-map}--- as a $\mathcal{Y}$-Mercer and thus apply
all the classic results of the \acl{OVK} theory on $\tilde{K}$.
\begin{proposition}
    \label{pr:orff_defines_kernel} Let $\seq{\omega}\in\dual{\mathcal{X}}^D$.
    If for all $y$, $y'\in\mathcal{Y}$ $\inner{y',
    \tildeK{\omega}_e\left(x\groupop z^{-1}\right)y}_{\mathcal{Y}}
    =\inner{\tildePhi{\omega}(x)y', \tildePhi{\omega}(z)y}_{\tildeH{\omega}}
    =\inner*{y', \frac{1}{D}\sum_{j=1}^D \conj{\pairing{x\groupop
    z^{-1},\omega_j}}B(\omega_j)B(\omega_j)^*y}_{\mathcal{Y}}$, for all $x$,
    $z\in\mathcal{X}$, then $\tildeK{\omega}$ is a shift-invariant
    $\mathcal{Y}$-Mercer \acl{OVK}.
\end{proposition}
Note that the above theorem does not consider the $\omega_j$'s as random
variables and therefore does not shows the convergence of the kernel
$\widetilde{K}$ to some target kernel $K$. However is shows that any
realization of $\widetilde{K}$ when $\omega_j$'s  are random variables yields
a valid $\mathcal{Y}$-Mercer operator-valued kernel.
Note that the above theorem does not considers the $\omega_j$'s as random
variables and therefore does not shows the convergence of the kernel
$\widetilde{K}$ to some target kernel $K$. However is shows that any
realization of $\widetilde{K}$ when $\omega_j$'s  are random variables yields
a valid $\mathcal{Y}$-Mercer operator-valued kernel.  Indeed, as a result of
\cref{pr:orff_defines_kernel}, in the same way we defined an \acs{ORFF}, we can
define an approximate feature operator $\tildeW{\omega}$ which maps
$\tildeH{\omega}$ onto $\mathcal{H}_{\tildeK{\omega}}$, where
$\tildeK{\omega}(x,z)= \tildePhi{\omega}(x)^\adjoint\tildePhi{\omega}(z)$, for
all $x$, $z\in\mathcal{X}$.
\begin{definition}[Random Fourier feature operator]
    Let $\seq{\omega}=(\omega_j)_{j=1}^D\in\dual{\mathcal{X}}^D$ and let
    $\tildeK{\omega}_e=\frac{1}{D}\sum_{j=1}^D
    \conj{\pairing{\cdot,\omega_j}}B(\omega_j)B(\omega_j)^\adjoint$.  We call
    random Fourier feature operator the linear application
    $\tildeW{\omega}:\tildeH{\omega}\to \mathcal{H}_{\tildeK{\omega}}$ defined
    as
    \begin{dmath*}
        \left(\tildeW{\omega} \theta\right)(x)
        \colonequals \tildePhi{\omega}(x)^\adjoint \theta
        =\frac{1}{\sqrt{D}}\sum_{j=1}^D
        \conj{\pairing{x,\omega_j}}B(\omega_j)\theta_j
    \end{dmath*}
    where $\theta=\Vect_{j=1}^D\theta_j \in\tildeH{\omega}$.  Then from
    \cref{pr:feature_operator}, $\left(\Ker \tildeW{\omega}\right)^\perp =
    \lspan\Set{\tildePhi{\omega}(x)y | \forall x\in\mathcal{X},\enskip \forall
    y\in\mathcal{Y}} \hiderel{\subseteq} \tildeH{\omega}$.
\end{definition}
The random Fourier feature operator is useful to show the relations between the
random Fourier feature map with the functional feature map defined in
\cref{pr:fourier_feature_map}. The relationship between the generic feature map
(defined for all \acl{OVK}) the functional feature map (defining a
shift-invariant $\mathcal{Y}$-Mercer \acl{OVK}) and the random Fourier feature
map is presented in \cref{fig:rel_features}.
\begin{proposition}
    \label{pr:phitilde_phi_rel} For any $g\in
    \mathcal{H}=L^2(\mathcal{\dual{X}}, \probability_{\dual{\Haar},\rho};
    \mathcal{Y}')$, let $\theta \colonequals \frac{1}{\sqrt{D}}\Vect_{j=1}^D
    g(\omega_j), \enskip \omega_j \sim \probability_{\dual{\Haar},\rho}
    \enskip\text{\ac{iid}}$ Then
    \begin{enumerate}
        \item \label{pr:cv_feature_map_1} $\left(\tildeW{\omega}
        \theta\right)(x)=\tildePhi{\omega}(x)^\adjoint \theta
        \converges{\acs{asurely}}{D\to\infty} \Phi_x^\adjoint g=(Wg)(x)$,
        \item \label{pr:cv_feature_map_2} $\norm{\theta}_{\tildeH{\omega}}^2
        \converges{\acs{asurely}}{D\to\infty} \norm{g}_{\mathcal{H}}^2$,
    \end{enumerate}
\end{proposition}
We write $\tildePhi{\omega}(x)^\adjoint \tildePhi{\omega}(x)\approx K(x,z)$
when $\tildePhi{\omega}(x)^\adjoint \tildePhi{\omega}(x)
\converges{\acs{asurely}}{} K(x,z)$ in the weak operator topology when $D$
tends to infinity. With mild abuse of notation we say that
$\tildePhi{\omega}(x)$ is an approximate feature map of the functional feature
map $\Phi_x$ \acs{ie}~$\tildePhi{\omega}(x)\approx \Phi_x$, when for all $y'$,
$y\in\mathcal{Y}$,
\begin{dmath*}
    \inner{y, K(x,z)y'}_{\mathcal{Y}}=\inner{\Phi_x y, \Phi_z
    y'}_{L^2(\dual{\mathcal{X}}, \probability_{\dual{\Haar}, \rho};
    \mathcal{Y'})} \hiderel{\approx} \inner{\tildePhi{\omega}(x)y,
    \tildePhi{\omega}(x)y'}_{\tildeH{\omega}}\colonequals \inner{y,
    \tilde{K}(x,z)y'}_{\mathcal{Y}}
\end{dmath*}
where $\Phi_x$ is defined in the sense of \cref{pr:fourier_feature_map}.

\subsection{Examples of Operator Random Fourier Feature maps}
\label{subsec:examples_ORFF} We now give two examples of operator-valued random
Fourier feature map. First we introduce the general form of an approximated
feature map for a matrix-valued kernel on the additive group
$(\mathbb{R}^d,+)$.
\begin{example}[Matrix-valued kernel on the additive group]
    \label{ex:additive_group} In the following let $K(x,z)=K_0(x-z)$ be a
    $\mathcal{Y}$-Mercer matrix-valued kernel on $\mathcal{X}=\mathbb{R}^d$,
    invariant \acs{wrt}~the group operation $+$. Then the function
    $\tildePhi{\omega}$ defined as follow is an \acl{ORFF} of $K_{0}$.
    For all $y\in\mathcal{Y}$,
    \begin{dmath*}
        \tildePhi{\omega}(x)y=\frac{1}{\sqrt{D}}\Vect_{j=1}^D
        \begin{pmatrix}
            \cos{\inner{x,\omega_j}_2}B(\omega_j)^\adjoint y \\
            \sin{\inner{x,\omega_j}_2}B(\omega_j)^\adjoint y
        \end{pmatrix}
        \condition{$\omega_j \sim \probability_{\dual{\Haar},\rho}$
        \acs{iid}.}
    \end{dmath*}
\end{example}
In particular we deduce the following features maps for the kernels proposed in
\cref{subsec:dec_examples}.
\begin{itemize}
    \item For the decomposable Gaussian kernel
    $K_0^{dec,gauss}(\delta)=k_0^{gauss}(\delta)\Gamma$ for all
    $\delta\in\mathbb{R}^d$, let $BB^\adjoint=\Gamma$. A bounded --and
    unbounded-- \acs{ORFF} map is
    \begin{dmath*}
        \tildePhi{\omega}(x)y
        =\frac{1}{\sqrt{D}} \Vect_{j=1}^D
        \begin{pmatrix}
            \cos{\inner{x,\omega_j}_2} B^\adjoint y \\
            \sin{\inner{x,\omega_j}_2}B^\adjoint y
        \end{pmatrix}
        \hiderel{=}(\tildephi{\omega}(x)\otimes B^\adjoint)y,
    \end{dmath*}
    where $\omega_j \hiderel{\sim}
    \probability_{\mathcal{N}(0,\sigma^{-2}I_d)}$ \ac{iid}~and
    $\tildephi{\omega}(x)=\frac{1}{\sqrt{D}}\Vect_{j=1}^D
    \begin{pmatrix}
        \cos{\inner{x,\omega_j}_2} \\
        \sin{\inner{x,\omega_j}_2}
    \end{pmatrix}$ is a scalar \acs{RFF}
    map~\citep{Rahimi2007}.
    \item For the curl-free Gaussian kernel,
    $K_0^{curl,gauss}=-\nabla\nabla^\transpose k_0^{gauss}$ an unbounded
    \acs{ORFF} map is
    \begin{dmath}
        \label{eq:unbounded_curl_free_orff}
        \tildePhi{\omega}(x)y=\frac{1}{\sqrt{D}}\Vect_{j=1}^D
        \begin{pmatrix}g
            \cos{\inner{x,\omega_j}_2}\omega_j^\transpose y \\
            \sin{\inner{x,\omega_j}_2}\omega_j^\transpose y
        \end{pmatrix},
    \end{dmath}
    $\omega_j \hiderel{\sim} \probability_{\mathcal{N}(0,\sigma^{-2}I_d)}$
    \ac{iid}~and a bounded \acs{ORFF} map is
    \begin{dmath*}
        \tildePhi{\omega}(x) y=\frac{1}{\sqrt{D}}\Vect_{j=1}^D
        \begin{pmatrix}
            \cos{\inner{x,\omega_j}_2}\frac{\omega_j^\transpose
            }{\norm{\omega_j}} y \\
            \sin{\inner{x,\omega_j}_2}\frac{\omega_j^\transpose
            }{\norm{\omega_j}} y
        \end{pmatrix}
        \condition{$\omega_j \hiderel{\sim} \probability_{\rho}$ \ac{iid}.}
    \end{dmath*}
    where $\rho(\omega) = \frac{\sigma^2\norm{\omega}^2}{d} \mathcal{N}(0,
    \sigma^{-2} I_d)(\omega)$ for all $\omega\in\mathbb{R}^d$.
    \item For the divergence-free Gaussian kernel
    $K_0^{div,gauss}(x,z)=(\nabla\nabla^\transpose -\Delta I_d)
    k_0^{gauss}(x,z)$ an unbounded \acs{ORFF} map is
    \begin{dmath}
        \label{eq:unbounded_div_free_orff}
        \tildePhi{\omega}(x) y
        =\frac{1}{\sqrt{D}}\Vect_{j=1}^D
        \begin{pmatrix}
            \cos{\inner{x,\omega_j}_2}B(\omega_j)^\transpose y \\
            \sin{\inner{x,\omega_j}_2}B(\omega_j)^\transpose y
        \end{pmatrix}
    \end{dmath}
    where $\omega_j \hiderel{\sim} \probability_{\rho}$ \ac{iid}~and
    $B(\omega)=\left(\norm{\omega}I_d-\omega\omega^\transpose \right)$ and
    $\rho=\mathcal{N}(0,\sigma^{-2}I_d)$ for all $\omega\in\mathbb{R}^d$. A
    bounded \acs{ORFF} map is
    \begin{dmath*}
        \tildePhi{\omega}(x) y=\frac{1}{\sqrt{D}}\Vect_{j=1}^Dg
        \begin{pmatrix}
            \cos{\inner{x,\omega_j}_2}B(\omega_j)^\transpose y \\
            \sin{\inner{x,\omega_j}_2}B(\omega_j)^\transpose y
            \end{pmatrix}
            \condition{$\omega_j \hiderel{\sim} \probability_{\rho}$ \ac{iid},}
    \end{dmath*}
    where $B(\omega) = \left(I_d - \frac{\omega\omega^\transpose
    }{\norm{\omega}^2}\right)$ and $\rho(\omega) =
    \frac{\sigma^2\norm{\omega}^2}{d}\mathcal{N}(0,\sigma^{-2}I_d)$ for allg
    $\omega\in\mathbb{R}^d$.
\end{itemize}
The second example extends scalar-valued Random Fourier Features on the skewed
multiplicative group --described in \cref{subsec:character} and
\cref{subsubsec:skewedchi2}-- to the operator-valued case.
\begin{example}[Matrix-valued kernel on the skewed multiplicative group]
    In the following, $K(x,z)=K_{1-c}(x\odot z^{-1})$ is a $\mathcal{Y}$-Mercer
    matrix-valued kernel on $\mathcal{X}=(-c;+\infty)^d$ invariant
    \acs{wrt}~the group operation\footnote{The group operation $\odot$ is
    defined in \cref{subsubsec:skewedchi2}.} $\odot$. Then the function
    $\tildePhi{\omega}$ defined as follow is an \acl{ORFF} of $K_{1-c}$. For
    all $y\in\mathcal{Y}$,
    \begin{dmath*}
        \tildePhi{\omega}(x) y=\frac{1}{\sqrt{D}}\Vect_{j=1}^D
        \begin{pmatrix}
            \cos{\inner{\log(x+c),\omega_j}_2}B(\omega_j)^\adjoint y \\
            \sin{\inner{\log(x+c),\omega_j}_2}B(\omega_j)^\adjoint y
        \end{pmatrix}\condition{$\omega_j \sim
        \probability_{\dual{\Haar},\rho}$ \ac{iid}.}
    \end{dmath*}
\end{example}
\subsection{Regularization property}
\label{subsec:regularization_property}
We have shown so far that it is always possible to construct a feature map that
allows to approximate a shift-invariant $\mathcal{Y}$-Mercer kernel. However we
could also propose a construction of such map by studying the regularization
induced with respect to the \acl{FT} of a target function $f\in \mathcal{H}_K$.
In other words, what is the norm in $L^2(\dual{\mathcal{X}}, \dual{\Haar};
\mathcal{Y}')$ induced by $\norm{\cdot}_K$?
\begin{proposition}
    \label{pr:fourier_reg_ovk}
    Let $K$ be a shift-invariant $\mathcal{Y}$-Mercer Kernel such that for all
    $y$, $y'$ in $\mathcal{Y}$, $\inner{y', K_e(\cdot)y}_{\mathcal{Y}}\in
    L^1(\mathcal{X}, \Haar)$. Then for all $f\in\mathcal{H}_K$
    \begin{dmath}
        \norm{f}^2_K = \displaystyle\int_{\dual{\mathcal{X}}}
        \frac{\inner*{\FT{f}(\omega), A\left(\omega\right)^\dagger
        \FT{f}(\omega)}_{\mathcal{Y}}}{\rho(\omega)} d\dual{\Haar}(\omega).
        \label{eq:reg_L2}
    \end{dmath}
    where $\inner{y', A(\omega)y}\rho(\omega)\colonequals\FT{\inner{y',
    K_e(\cdot)y}}(\omega)$.  \label{pr:regularization}
\end{proposition}
Note that if $K(x,z)=k(x,z)$ is a scalar kernel then for all $\omega$ in
$\dual{\mathcal{X}}$, $A(\omega)=1$. Therefore we recover the well known result
for kernels that is for any $f\in\mathcal{H}_k$ we have $\norm{f}_k =
\int_{\dual{\mathcal{X}}} \FT{k_e}(\omega)^{-1} \FT{f}(\omega)^2
d\dual{\Haar}(\omega)$~\citep{Yang2012, vertregularization,
smola1998connection}. Eventually from this last equation we also recover
\cref{pr:kernel_reg} for decomposable kernels. If
$A(\omega)=\Gamma\in\mathcal{L}_+(\mathbb{R}^p)$, $\norm{f}_K = \sum_{i,j=1}^p
\left(\Gamma^\dagger\right)_{ij}\inner{f_i,f_j}_k$ We also note that the
regularization property in $\mathcal{H}_K$ does not depends (as expected) on
the decomposition of $A(\omega)$ into $B(\omega)B(\omega)^\adjoint $.
Therefore the decomposition should be chosen such that it optimizes the
computation cost. For instance if $A(\omega)\in\mathcal{L}(\mathbb{R}^p)$ has
rank $r$, one could find an operator $B(\omega)\in\mathcal{L}(\mathbb{R}^p,
\mathbb{R}^r)$ such that $A(\omega)=B(\omega)B(\omega)^\adjoint$. Moreover, in
light of \cref{pr:regularization} the regularization property of the kernel
with respect to the \acl{FT}, it is also possible to define an approximate
feature map of an \acl{OVK} from its regularization properties in the
\acs{vv-RKHS} as proposed in \cref{alg:ORFF2_construction}.
\SetKwInOut{Input}{Input}
\SetKwInOut{Output}{Output}
\begin{center}
    \begin{algorithm2e}[t!]\label{alg:ORFF2_construction}
        \SetAlgoLined
        \Input{%
        \begin{itemize}
            \item The pairing $\pairing{x, \omega}$ of the \acs{LCA} group
            $(\mathcal{X}, \groupop)$.
            \item A probability measure $\probability_{\dual{\Haar},\rho}$ with
            density $\rho$ \acs{wrt}~the haar measure $\dual{\Haar}$ on
            $\dual{\mathcal{X}}$.
            \item An operator-valued function
            $B:\dual{\mathcal{X}}\to\mathcal{L}(\mathcal{Y},\mathcal{Y}')$ such
            that for all $y$ $y'\in\mathcal{Y}$, $\inner{y',
            B(\cdot)B(\cdot)^\adjoint y}\in
            L^1(\dual{\mathcal{X}},\probability_{\dual{\Haar},\rho})$.
            \item $D$ the number of features.
        \end{itemize}}
        \Output{A random feature $\tildePhi{\omega}(x)$ such that
        $\tildePhi{\omega}(x)^\adjoint \tildePhi{\omega}(z) \approx K(x,z)$.}
        \BlankLine
        Draw $D$ random vectors $(\omega_j)_{j=1}^D$ \ac{iid}~from the
        probability law $\probability_{\dual{\Haar},\rho}$\;
        \Return $
        \begin{cases}
            \tildePhi{\omega}(x) \in\mathcal{L}(\mathcal{Y}, \tildeH{\omega})
            &: y \mapsto \frac{1}{\sqrt{D}}\Vect_{j=1}^D\pairing{x,
            \omega_j}B(\omega_j)^\adjoint y \\
            \tildePhi{\omega}(x)^\adjoint \in\mathcal{L}(\tildeH{\omega},
            \mathcal{Y}) &: \theta \mapsto \frac{1}{\sqrt{D}} \sum_{j=1}^D
            \pairing{x, \omega_j}B(\omega_j)\theta_j
        \end{cases}$\;
        \caption{Construction of \acs{ORFF}}
    \end{algorithm2e}
\end{center}

%%%%%%%%%%%%%%%%%%%%%%%%%%%%%%%%%%%%%%%%%%%%%%%%%%%%%%%%%%%%%%%%%%%%%%%%%%%%%%%

\section{Main contribution: convergence with high probability of the
\acpdfstring{ORFF} estimator}
\label{sec:consistency_of_the_ORFF_estimator}
We are now interested in a non-asymptotic analysis of the \ac{ORFF}
approximation of shift-invariant $\mathcal{Y}$-Mercer kernels on \acs{LCA}
group $\mathcal{X}$ endowed with the operation group $\groupop$ where
$\mathcal{X}$ is a Banach space (The more general case where $\mathcal{X}$ is
a Polish space is discussed in the appendix \cref{subsec:concentration_proof}).
For a given $D$, we study how close is the
approximation $\tilde{K}(x,z)=\tildePhi{1:D}(x)^*\tildePhi{1:D}(z)$ to the
target kernel $K(x,z)$ for any $x,z$ in $\mathcal{X}$.
\paragraph{}
If $A\in\mathcal{L}_+(\mathcal{Y})$ we denote
$\norm{A}_{\mathcal{Y},\mathcal{Y}}$ its operator norm (the induced norm). For
$x$ and $z$ in some non-empty compact $\mathcal{C} \subset \mathbb{R}^d$, we
consider: $F(x \groupop \inv{z}) =\tilde{K}(x,z)-K(x,z)$ and study how the
uniform norm $\norm{\tilde{K}-K}_{\mathcal{C}\times\mathcal{C}} \colonequals
\sup_{(x,z)\in\mathcal{C}\times\mathcal{C}}
\norm{\tilde{K}(x,z)-K(x,z)}_{\mathcal{Y},\mathcal{Y}}$ behaves according to
$D$. All along this document we denote $\delta=x\groupop z^{-1}$ for all $x$
and $z\in\mathcal{X}$. \Cref{fig:approximation_error} empirically shows
convergence of three different \acs{OVK} approximations for $x,z$ sampled from
the compact $[-1,1]^4$ and using an increasing number of sample points $D$. The
log-log plot shows that all three kernels have the same convergence rate, up to
a multiplicative factor.
%\paragraph{}
%In order to bound the error with high probability, we turn to concentration
%inequalities devoted to random matrices~\citep{Boucheron}. The concentration
%phenomenon can be summarized in the following sentence of
%\citet{ledoux2005concentration}. \say{A random variable that depends (in a
%smooth way) on the influence of many random variables (but not too much on any
%of them) is essentially constant}.
\paragraph{}
A typical application is the study of the deviation of the empirical mean of
\acl{iid} random variables to their expectation. This means that given an error
$\epsilon$ between the kernel approximation $\tildeK{\omega}$ and the true
kernel $K$, if we are given enough samples to construct $\tildeK{\omega}$, the
probability of measuring an error greater than $\epsilon$ is essentially zero
(it drops at an exponential rate with respect to the number of samples $D$). To
measure the error between the kernel approximation and the true kernel at a
given point many metrics are possible. \acs{eg} any matrix norm such as the
Hilbert-Schmidt norm, trace norm, the operator norm or Schatten norms. In this
work we focus on measuring the error in terms of operator norm. For all $x$,
$z\in\mathcal{X}$ we look for a bound on
\begin{dmath*}
    \probability_{\rho} \Set{(\omega_j)_{j=1}^D | \norm{\tildeK{\omega}(x, z) -
    K(x, z)}_{\mathcal{Y}, \mathcal{Y}} \ge \epsilon }
    =
    \probability_{\rho} \Set{(\omega_j)_{j=1}^D | \sup_{0\neq y\in\mathcal{Y}}
    \frac{\norm{(\tildeK{\omega}(x, z) - K(x,
    z))y}_{\mathcal{Y}}}{\norm{y}_{\mathcal{Y}}} \ge \epsilon}
\end{dmath*}
In other words, given any vector $y\in\mathcal{Y}$ we study how the residual
operator $\tildeK{\omega} - K$ is able to send $y$ to zero. We believe that
this way of measuring the \say{error} to be more intuitive. Moreover, on
contrary to an error measure with the Hilbert-Schmidt norm, the operator norm
error does not grows linearly with the dimension of the output space as the
Hilbert-Schmidt norm does. On the other hand the Hilbert-schmidt norm makes the
studied random variables Hilbert space valued, for which it is much easier to
derive concentration inequalities \citep{smale2007learning, pinelis1994optimum,
naor2012banach}. Note that in the scalar case ($A(\omega)= 1$) the
Hilbert-Schmidt norm error and the operator norm are the same and measure the
deviation between $\tildeK{\omega}$ and $K$ as the absolute value of their
difference.
\paragraph{}
A raw concentration inequality of the kernel estimator gives the error on one
point. If one is interesting in bounding the maximum error over $N$ points,
applying a union bound on all the point would yield a bound that grows linearly
with $N$. This would suggest that when the number of points increase, even if
all of them are concentrated in a small subset of $\mathcal{X}$, we should draw
increasingly more features to have an error below $\epsilon$ with high
probability. However if we restrict ourselves to study the error on a compact
subset of $\mathcal{X}$ (and in practice data points lies often in a closed
bounded subset of $\mathbb{R}^d$), we can cover this compact subset by a finite
number of closed balls and apply the concentration inequality and the union
bound only on the center of each ball. Then if the function
$\norm{\tildeK{\omega}_e-K_e}$ is smooth enough on each ball (\acs{ie}
Lipschitz) we can guarantee with high probability that the error between the
centers of the balls will not be too high. Eventually we obtain a bound in the
worst case scenario on all the points in a subset $\mathcal{C}$ of
$\mathcal{X}$. This bound depends on the covering number
$\mathcal{N}(\mathcal{C}, r)$ of $\mathcal{X}$ with ball of radius $r$. When
$\mathcal{X}$ is a Banach space, the covering number is proportional to the
diameter of the diameter of $\mathcal{C}\subseteq\mathcal{X}$.
\paragraph{}
Prior to the presentation of general results, we briefly recall the uniform
convergence of \acs{RFF} approximation for a scalar shift invariant kernel on
the additive \acs{LCA} group $\mathbb{R}^d$ and introduce a direct corollary
about decomposable shift-invariant \acs{OVK} on the \acs{LCA} group
$(\mathbb{R}^d, +)$.
\begin{figure}[t]
    \centering
    \resizebox{.85\textwidth}{!}{\input{./gfx/approximation.pgf}}
    \caption[\acs{ORFF} reconstruction error]{Error reconstructing the target
    operator-valued kernel $K$ with \acs{ORFF}
    approximation $\tilde{K}$ for the decomposable, curl-free and
    divergence-free kernel.}
    \label{fig:approximation_error}
\end{figure}
\subsection{Random Fourier Features in the scalar case and decomposable OVK}
\citet{Rahimi2007} proved the uniform convergence of \acf{RFF} approximation
for a scalar shift-invariant kernel on the \acs{LCA} group $\mathbb{R}^d$
endowed with the group operation $\groupop=+$. In the case of the
shift-invariant decomposable \acs{OVK}, an upper bound on the error can be
obtained as a direct consequence of the result in the scalar case obtained
by~\citet{Rahimi2007} and other authors~\citep{sutherland2015, sriper2015}.
\begin{theorem}[Uniform error bound for \ac{RFF},~\citet{Rahimi2007}]
    \label{rff-scalar-bound}
    Let $\mathcal{C}$ be a compact of subset of $\mathbb{R}^d$ of diameter
    $\abs{\mathcal{C}}$. Let $k$ be a shift invariant kernel, differentiable
    with a bounded second derivative and $\probability_{\rho}$ its normalized
    \acl{IFT} such that it defines a probability measure. Let
    $\widetilde{k}=\sum_{j=1}^D\cos{\inner{\cdot, \omega_j}} \hiderel{\approx}
    k(x,z) \enskip\text{and}\enskip \sigma^2\hiderel{=}\expectation_{\rho}
    \norm{\omega}^2_2$.  Then we have
    \begin{dmath*}
        \probability_{\rho}\Set{(\omega_j)_{j=1}^D |
        \norm{\tilde{k}-k}_{\mathcal {C}\times\mathcal{C}}\ge \epsilon } \le
        2^8\left( \frac{\sigma \abs{\mathcal{C}}}{\epsilon} \right)^2\exp\left(
        -\frac{\epsilon^2D}{4(d+2)} \right)
    \end{dmath*}
\end{theorem}
From \cref{rff-scalar-bound}, we can deduce the following corollary about the
uniform convergence of the \acs{ORFF} approximation of the decomposable kernel.
We recall that for a given pair $x$, $z$ in $\mathcal{C}$, $\tilde{K}(x,z)=
\tildePhi{\omega}(x)^* \tildePhi{\omega}(z)=\Gamma\tilde{k}(x,z)$ and
$K_0(x-z)=\Gamma \expectation_{\dual{\Haar},\rho}[\tilde{k}(x,z)]$.
\begin{corollary}[Uniform error bound for decomposable \acs{ORFF}]
    \label{c:dec-bound}
    Let $\mathcal{C}$ be a compact of subset of $\mathbb{R}^d$ of diameter
    $\abs{\mathcal{C}}$. Let $K$ be a decomposable kernel built from a positive
    operator self-adjoint $\Gamma$, and $k$ a shift invariant kernel with
    bounded second derivative such that
    $\widetilde{K}=\sum_{j=1}^D\cos{\inner{\cdot, \omega_j}}\Gamma
    \hiderel{\approx} K \enskip\text{and}\enskip
    \sigma^2\hiderel{=}\expectation_{\rho} \norm{\omega}^2_2$.  Then
    \begin{dmath*}
        \probability_{%
        \rho}\Set{(\omega_j)_{j=1}^D|\norm{\widetilde{K}-K}_{\mathcal{C} \times
        \mathcal{C}}\ge \epsilon } \le 2^8\left( \frac{\sigma
        \norm{\Gamma}_{\mathcal{Y},\mathcal{Y}} \abs{\mathcal{C}}}{\epsilon}
        \right)^2\exp\left( -\frac{\epsilon^2D}{4\norm{\Gamma}_2^2(d+2)} \right)
    \end{dmath*}
\end{corollary}
Note that a similar corollary could have been obtained for the recent
result of~\citet{sutherland2015} who refined the bound proposed by Rahimi and
Recht by using a Bernstein concentration inequality instead of the Hoeffding
inequality. More recently~\citet{sriper2015} showed an optimal bound for
\acl{RFF}. The improvement of~\citet{sriper2015} is mainly in the constant
factors where the bound does not depend linearly on the diameter
$\abs{\mathcal{C}}$ of $\mathcal{C}$ but exhibit a logarithmic dependency
$\log\left(\abs{\mathcal{C}}\right)$, hence requiring significantly less random
features to reach a desired uniform error with high probability. Moreover,
\citet{sutherland2015} also considered a bound on the expected max error
$\expectation_{\dual{\Haar}, \rho} \norm{\widetilde{K}-K}_{\infty}$, which is
obtained using Dudley's entropy integral~\citep{dudley1967sizes, Boucheron} as
a bound on the supremum of an empirical process by the covering number of the
indexing set. This useful theorem is also part of the proof of
\citet{sriper2015}.
\subsection{Uniform convergence of \acpdfstring{ORFF} approximation on
\acpdfstring{LCA} groups}
%In this analysis, we assume that $\mathcal{Y}$ is finite dimensional, in
%\cref{remark:infinite_dimension}, we discuss how the proof could be extended to
%infinite dimensional output Hilbert spaces. We propose a bound for \acl{ORFF}
%approximation in the general case. It relies on two main ideas:
%\begin{enumerate}
    %\item a matrix-Bernstein concentration inequality for random matrices need
    %to be used instead of concentration inequality for scalar random variables,

    %\item a general theorem, valid for random matrices with bounded norms (such
    %as decomposable kernel \acs{ORFF} approximation) as well as un\-bound\-ed
    %norms (such as the \acs{ORFF} approximation we proposed for curl and
    %divergence-free kernels, for which the norm behave as subexponential random
    %variables).
%\end{enumerate}
Before introducing the new theorem, we give the definition of the Orlicz norm
which gives a proxy-bound on the norm of subexponential random variables.
\begin{definition}[Orlicz norm~\citep{van1996weak}]
    \label{def:orlicz}
    Let $\psi:\mathbb{R}_+\to\mathbb{R}_+$ be a non-decreasing convex function
    with $\psi(0)=0$. For a random variable $X$ on a measured space
    $(\Omega,\mathcal{T} (\Omega),\mu)$, the quantity $\norm{X}_{\psi}
    \hiderel{=} \inf \Set{C > 0  | \expectation_{\mu}[\psi\left( \abs{X}/C
    \right)]\le 1}$.  is called the Orlicz norm of $X$.
\end{definition}
Here, the function $\psi$ is chosen as $\psi(u)=\psi_{\alpha}(u)$ where
$\psi_{\alpha}(u) \colonequals e^{u^{\alpha}}-1$. When $\alpha=1$, a random
variable with finite Orlicz norm is called a \emph{subexponential variable}
because its tails decrease at an exponential rate. Let $X$ be a self-adjoint
random operator. Given a scalar-valued measure $\mu$, we call \emph{variance}
of an operator $X$ the quantity $\variance_{\mu}[X]=\expectation_
{\mu}[X-\expectation_{\mu}[X]]^2$. 
%With this convention if $X$ is a $p\times
%p$ Hermitian matrix,
%\begin{dmath*}
    %\variance_{\mu}[X]_{\ell m}=\sum_{r=1}^p\covariance{X_{\ell r}, X_{rm}}.
%\end{dmath*}
Among the possible concentration inequalities adapted to random operators
\citep{tropp2015introduction, minsker2011some, ledoux2013probability,
pinelis1994optimum, koltchinskii2013remark}, we focus on the results of
\citet{tropp2015introduction, minsker2011some}, for their robustness to high or
potentially infinite dimension of the output space $\mathcal{Y}$. To guarantee
a good scaling with the dimension of $\mathcal{Y}$ we introduce the notion of
intrinsic dimension (or effective rank) of an operator.
\begin{definition}
    \label{def:intdim}
    Let $A$ be a trace class operator acting on a Hilbert space $\mathcal{Y}$.
    We call intrinsic dimension the quantity: $\intdim(A)
    =\norm{A}_{\mathcal{Y}, \mathcal{Y}}^{-1} \Tr\left[A\right]$.
\end{definition}
Indeed the bound proposed in our first publication at \acs{ACML}
\citep{brault2016random} based on \citet{koltchinskii2013remark} depends on $p$
while the present bound depends on the intrinsic dimension of the variance of
$A(\omega)$ which is always smaller than $p$ when the operator $A(\omega)$ is
Hilbert-Schmidt ($p\le\infty$).
\begin{corollary}
    \label{corr:unbounded_consistency}
    Let $K:\mathcal{X}\times\mathcal{X}\to\mathcal{L}(\mathcal{Y})$ be a
    shift-invariant $\mathcal{Y}$-Mercer kernel, where $\mathcal{Y}$ is a
    finite dimensional Hilbert space of dimension $p$ and $\mathcal{X}$ a
    finite dimensional Banach space of dimension $d$. Moreover, let
    $\mathcal{C}$ be a closed ball of $\mathcal{X}$ centred at the origin of
    diameter $\abs{\mathcal{C}}$,
    $A:\dual{\mathcal{X}}\to\mathcal{L}(\mathcal{Y})$ and
    $\probability_{\dual{\Haar},\rho}$ a pair such that
    \begin{dmath*}
        \tilde{K}_e = \sum_{j=1}^D \cos{\pairing{\cdot,\omega_j}}A(\omega_j)
        \hiderel{\approx}
        K_e\condition{$\omega_j\sim\probability_{\dual{\Haar}, \rho}$
        \acs{iid}.}.
    \end{dmath*}
    Let $\mathcal{D}_{\mathcal{C}}=\mathcal{C}\groupop\mathcal{C}^{-1}$ and
    $V(\delta) \succcurlyeq \variance_{\dual{\Haar},\rho} \tilde{K}_e(\delta)$
    for all $\delta\in\mathcal{D}_{\mathcal{C}}$ and $H_\omega$ be the
    Lipschitz constant of the function $h: x\mapsto
    \pairing{x,\omega}$. If the three following constants exist
    \begin{dmath*}
        m \ge \int_{\dual{\mathcal{X}}} H_\omega
        \norm{A(\omega)}_{\mathcal{Y},\mathcal{Y}} d\probability_{\dual{\Haar},
        \rho}(\omega) \hiderel{<} \infty
    \end{dmath*}
    and
    \begin{dmath*}
        u \ge 4\left(\norm{\norm{A(\omega)}_{\mathcal{Y},\mathcal{Y}}}_{\psi_1}
        + \sup_{\delta\in\mathcal{D}_{\mathcal{C}}}
        \norm{K_e(\delta)}_{\mathcal{Y},\mathcal{Y}}\right) \hiderel{<} \infty
    \end{dmath*}
    and
    \begin{dmath*}
        v \ge \sup_{\delta\in\mathcal{D}_{\mathcal{C}}} D
        \norm{V(\delta)}_{\mathcal{Y}, \mathcal{Y}} \hiderel{<} \infty.
    \end{dmath*}
    Define $p_{int}\ge \sup_{\delta\in\mathcal{D}_{\mathcal{C}}}
    \intdim(V(\delta))$, then for all $0 < \epsilon \le m \abs{C}$,
    \begin{dmath*}
        \probability_{\dual{\Haar,\rho}}\Set{(\omega_j)_{j=1}^D |
        \norm{\tilde{K}-K}_{\mathcal{C}\times\mathcal{C}} \ge \epsilon}
        \le 8\sqrt{2} \left( \frac{m\abs{\mathcal{C}}}{\epsilon}
        \right)
        {\left(p_{int}r_{v/D}(\epsilon)\right)}^{\frac{1}{d + 1}}
        \begin{cases}
            \exp\left(-D\frac{\epsilon^2}{8
            v(d+1)\left(1 + \frac{1}{p}\right)}
            \right) \condition{$\epsilon \le
            \frac{v}{u}\frac{1+1/p}{K(v,
            p)}$} \\
            \exp\left(-D\frac{\epsilon}{8u(d+1)K(v,
            p)}\right)\condition{otherwise,}
        \end{cases}
    \end{dmath*}
    where $K(v, p)=\log\left(16 \sqrt{2}
    p\right)+\log\left(\frac{u^2}{v}\right) $ and $r_{v/D}(\epsilon)=1 +
    \frac{3}{\epsilon^2\log^2(1 + D \epsilon / v)}$.
\end{corollary}
We give a comprehensive full proof of the theorem in
\cref{subsec:concentration_proof}. It follows the usual scheme derived
in~\citet{Rahimi2007} and~\citet{sutherland2015} and involves Bernstein
concentration inequality for unbounded symmetric matrices
(\cref{th:Bernstein3}).

\subsection{Dealing with infinite dimensional operators}
\label{remark:infinite_dimension}
We studied the concentration of \acsp{ORFF} under the assumption that
$\mathcal{Y}$ is finite dimensional. Indeed a $d$ term characterizing the
dimension of the input space $\mathcal{X}$ appears in the bound proposed in
\cref{corr:unbounded_consistency}, and when $d$ tends to infinity, the
exponential part goes to zero so that the probability is bounded by a
constant greater than one. Unfortunately, considering unbounded random
operators \citet{minsker2011some} doesn't give any tighter solution.
\paragraph{}
In our first bound presented at \acs{ACML}, we presented a bound based on a
matrix concentration inequality for unbounded random variable. Compared to this
previous bound, \cref{corr:unbounded_consistency} does not depend on the
dimensionality $p$ of the output space $\mathcal{Y}$ but on the intrinsic
dimension of the operator $A(\omega)$. However to remove the dependency in $p$
in the exponential part, we must turn our attention to operator concentration
inequalities for bounded random variable. To the best of our knowledge we are
not aware of concentration inequalities working for \say{unbounded} operator-
valued random variables. Following the same proof than
\cref{corr:unbounded_consistency} we obtain
\cref{corr:bounded_infinite_dim_consistency}.
\begin{corollary}
    \label{corr:bounded_infinite_dim_consistency}
    Let $K:\mathcal{X}\times\mathcal{X}\to\mathcal{L}(\mathcal{Y})$ be a
    shift-invariant $\mathcal{Y}$-Mercer kernel, where $\mathcal{Y}$ is a
    Hilbert space and $\mathcal{X}$ a finite dimensional Banach space of
    dimension $D$. Moreover, let $\mathcal{C}$ be a closed ball of
    $\mathcal{X}$ centered at the origin of diameter $\abs{\mathcal{C}}$,
    subset of $\mathcal{X}$, $A:\dual{\mathcal{X}}\to\mathcal{L}(\mathcal{Y})$
    and $\probability_{\dual{\Haar},\rho}$ a pair such that
    \begin{dmath*}
        \tilde{K}_e = \sum_{j=1}^D \cos{\pairing{\cdot,\omega_j}}A (\omega_j)
        \hiderel{\approx}
        K_e\condition{$\omega_j\sim\probability_{\dual{\Haar}, \rho}$
        \acs{iid}.}
    \end{dmath*}
    where $A(\omega_j)$ is a Hilbert-Schmidt operator for all $j \in
    \mathbb{N}^*_D$. Let $\mathcal{D}_{\mathcal{C}}=\mathcal{C} \groupop
    \mathcal{C}^{-1}$ and $V (\delta) \succcurlyeq\variance_{\dual{\Haar},\rho}
    \tilde{K}_e (\delta)$ for all $\delta\in\mathcal{D}_{\mathcal{C}}$ and
    $H_\omega$ be the Lipschitz constant of the function $h: x\mapsto
    \pairing{x,\omega}$. If the three following constants exists
    \begin{dmath*}
        m \ge\int_{\dual{\mathcal{X}}} H_{\omega}
        \norm{A (\omega)}_{\mathcal{Y},\mathcal{Y}}
        d\probability_{\dual{\Haar}, \rho}(\omega) \hiderel{<} \infty{}
    \end{dmath*}
    and
    \begin{dmath*}
        u \ge\esssup_{\omega\in\dual{\mathcal{X}}}
        \norm{A (\omega)}_{\mathcal{Y}, \mathcal{Y}} +
        \sup_{\delta\in\mathcal{D}_{\mathcal{C}}}
        \norm{K_e (\delta)}_{\mathcal{Y}, \mathcal{Y}} \hiderel{<} \infty{}
    \end{dmath*}
    and
    \begin{dmath*}
        v \ge\sup_{\delta\in\mathcal{D}_{\mathcal{C}}} D
        \norm{V (\delta)}_{\mathcal{Y}, \mathcal{Y}} \hiderel{<} \infty.
    \end{dmath*}
    define $p_{int} \ge \sup_{\delta\in\mathcal{D}_{\mathcal{C}}}
    \intdim\left(V(\delta)\right)$ then for all $\sqrt{\frac{v}{D}} +
    \frac{u}{3D} < \epsilon < m\abs{\mathcal{C}}$,
    \begin{dmath*}
        \probability_{\dual{\Haar,\rho}}\Set{(\omega_j)_{j=1}^D |
        \sup_{\delta\in\mathcal{D}_{\mathcal{C}}}
        \norm{F (\delta)}_{\mathcal{Y}, \mathcal{Y}} \ge\epsilon} \le~8\sqrt{2}
        \left(\frac{m\abs{\mathcal{C}}}{\epsilon}\right) p_{int}^{\frac{1}{d +
        1}} \exp\left(-D\psi_{v,d,u} (\epsilon) \right)
    \end{dmath*}
    where $\psi_{v,d,u}(\epsilon)=\frac{\epsilon^2}{2(d+1)(v + u
    \epsilon / 3)}$.
\end{corollary}
Again a full comprehensive proof is given in \cref{subsec:concentration_proof}
of the appendix. Notice that in this result, The dimension
$p=\dim{\mathcal{Y}}$ does not appear. Only the intrinsic dimension of the
variance of the estimator. Moreover when $d$ is large, the term
$p_{int}^{\frac{1}{d + 1}}$ goes to one, so that the impact of the intrinsic
dimension on the bound vanish when the dimension of the input space is large.
subsection{Variance of the \acpdfstring{ORFF} approximation}
We now provide a bound on the norm of the variance of $\tilde{K}$, required to
apply \cref{corr:unbounded_consistency,corr:bounded_infinite_dim_consistency}.
This is an extension of the proof of \citet{sutherland2015} to the
operator-valued case, and we recover their results in the scalar case when
$A(\omega)=1$. An illustration of the bound is provided in
\cref{fig:approximation_error_var} for the decomposable and the curl-free
\acs{OVK}.
\begin{proposition}[Bounding the \emph{variance} of $\tilde{K}_e$]
    \label{pr:variance_bound}
    Let $K$ be a shift invariant $\mathcal{Y}$-Mercer kernel on a second
    countable \ac{LCA} topological space $\mathcal{X}$. Let
    $A:\dual{\mathcal{X}}\to\mathcal{L}(\mathcal{Y})$ and
    $\probability_{\dual{\Haar},\rho}$ a pair such that $\tilde{K}_e =
    \sum_{j=1}^D \cos{\pairing{\cdot,\omega_j}}A (\omega_j) \hiderel{\approx}
    K_e$, $\omega_j\sim\probability_{\dual{\Haar}, \rho}$ \acs{iid} Then,
    \begin{dmath*}
        \variance_{\dual{\Haar}, \rho} \left[ \tilde{K}_e(\delta) \right]
        \preccurlyeq \frac{1}{2D} \left( \left( K_e(2\delta) + K_e(e) \right)
        \expectation_{\dual{\Haar}, \rho}\left[ A(\omega) \right] -
        2 K_e(\delta)^2 + \variance_{\dual{\Haar}, \rho}\left[
        A(\omega) \right]\right)
    \end{dmath*}
\end{proposition}
\begin{figure}[t]
    \begin{minipage}[c]{.46\linewidth}
        \centering\resizebox{\linewidth}{!}{%
        \input{./gfx/variance_dec.tikz}}
    \end{minipage}
    \begin{minipage}[c]{.54\linewidth}
        \centering\resizebox{\linewidth}{!}{%
        \input{./gfx/variance_curl.tikz}}
    \end{minipage}
    \caption[ORFF variance bound]{Comparison between an empirical bound on the
    norm of the variance of the decomposable (left) and  curl-free (right) ORFF
    obtained and the theoretical bound proposed in \cref{pr:variance_bound}
    versus $D$. \label{fig:approximation_error_var}}
\end{figure}
\subsection{Application on decomposable, curl-free and divergence-free
\acpdfstring{OVK}}
First, the two following examples discuss the form of $H_\omega$ for the
additive group and the skewed-multiplicative group. Here we view
$\mathcal{X}=\mathbb{R}^d$ as a Banach space endowed with the Euclidean norm.
Thus the Lipschitz constant $H_{\omega}$ is bounded by the supremum of the norm
of the gradient of $h_{\omega}$.
\begin{example}[Additive group]
    On the additive group, $h_\omega(\delta)=\inner{\omega, \delta}$. Hence
    $H_\omega=\norm{\omega}_2$.
\end{example}
\begin{example}[Skewed-multiplicative group]
    On the skewed multiplicative group, $h_\omega(\delta)=\inner{\omega,
    \log(\delta+c)}$. Therefore $\sup_{\delta\in\mathcal{C}}\norm{\nabla
    h_\omega(\delta)}_2 = \sup_{\delta\in\mathcal{C}}\norm{\omega/(\delta +
    c)}_2$.  Eventually $\mathcal{C}$ is compact subset of $\mathcal{X}$ and
    finite dimensional thus $\mathcal{C}$ is closed and bounded. Thus
    $H_\omega=\norm{\omega}_2/(\min_{\delta\in\mathcal{C}} \norm{\delta}_2+c)$.
\end{example}
Now we compute upper bounds on the norm of the variance and Orlicz norm of the
three \acsp{ORFF} we took as examples.
\subsubsection{Decomposable kernel}
notice that in the case of the Gaussian decomposable kernel, \acs{ie}
$A(\omega)=A$, $e=0$, $K_0(\delta)= Ak_0(\delta)$, $k_0(\delta) \geq 0$ and
$k_0(\delta)=1$, then we have
\begin{equation*}
    D\norm{\variance_\mu \left[ \tilde{K}_0(\delta)
    \right]}_{\mathcal{Y},\mathcal{Y}}\leq
    (1+k_0(2\delta))\norm{A}_{\mathcal{Y},\mathcal{Y}}/2 + k_0(\delta)^2.
\end{equation*}
\subsubsection{Curl-free and divergence-free kernels:}
recall that in this case $p=d$. For the (Gaussian) curl-free kernel,
$A(\omega)=\omega\omega^*$ where $\omega\in\mathbb{R}^d\sim\mathcal{N}(0,
\sigma^{-2}I_d)$ thus $\expectation_\mu [A(\omega)] = I_d/\sigma^2$ and
$\variance_{\mu}[A(\omega)]=(d+1)I_d/\sigma^4$. Hence,
\begin{equation*}
    D\norm{\variance_\mu \left[ \tilde{K}_0(\delta)
    \right]}_{\mathcal{Y},\mathcal{Y}} \leq
    \frac{1}{2}\norm{\frac{1}{\sigma^2}K_0(2\delta)-2
    K_0(\delta)^2}_{\mathcal{Y},\mathcal{Y}} + \frac{(d+1)}{\sigma^4}.
\end{equation*}
This bound is illustrated by \cref{fig:approximation_error} B, for a given
datapoint. Eventually for the Gaussian divergence-free kernel,
$A(\omega)=I\norm{\omega}_2^2-\omega\omega^*$, thus $\expectation_\mu
[A(\omega)] = I_d(d-1)/\sigma^2$ and $
\variance_{\mu}[A(\omega)]=d(4d-3)I_d/\sigma^4$. Hence,
\begin{equation*}
    D\norm{\variance_\mu \left[ \tilde{K}_0(\delta)
    \right]}_{\mathcal{Y},\mathcal{Y}} \leq
    \frac{1}{2}\norm{\frac{(d-1)}{\sigma^2}K_0(2\delta)-2
    K_0(\delta)^2}_{\mathcal{Y}, \mathcal{Y}}+ \frac{d(4d-3)}{\sigma^4}.
\end{equation*}
To conclude, we ensure that the random variable $\norm{A(\omega)}_{\mathcal{Y},
\mathcal{Y}}$ has a finite Orlicz norm with $\psi=\psi_1$ in these three cases.
\subsubsection{Computing the Orlicz norm}
for a random variable with strictly monotonic moment generating function (MGF),
one can characterize its inverse $\psi_1$ Orlicz norm by taking the functional
inverse of the MGF evaluated at 2 (see \cref{lm:orlicz_mgf} of the
appendix). In other words
$\norm{X}_{\psi_1}^{-1}=\MGF(x)^{-1}_X(2)$. For the Gaussian curl-free and
divergence-free kernel,
\begin{dmath*}
    \norm{A^{div}(\omega)}_{\mathcal{Y},\mathcal{Y}} =
    \norm{A^{curl}(\omega)}_{\mathcal{Y},\mathcal{Y}} \hiderel{=}
    \norm{\omega}_{2}^2,
\end{dmath*}
where $\omega\sim\mathcal{N}(0,I_d/\sigma^2)$, hence $\norm{A(\omega)}_2\sim
\Gamma(p/2,2/\sigma^2)$. The MGF of this gamma distribution is
$\MGF(x)(t)=(1-2t/\sigma^2)^{-(p/2)}$. Eventually
\begin{equation*}
    \norm{\norm{A^{div}(\omega)}_{\mathcal{Y},\mathcal{Y}}}_{\psi_1}^{-1} =
    \norm{\norm{A^{curl}(\omega)}_{\mathcal{Y},\mathcal{Y}}}_{\psi_1}^{-1} =
    \frac{\sigma^2}{2}\left(1-4^{-\frac{1}{p}}\right).
\end{equation*}

%%%%%%%%%%%%%%%%%%%%%%%%%%%%%%%%%%%%%%%%%%%%%%%%%%%%%%%%%%%%%%%%%%%%%%%%%%%%%%%

\section{Learning with \acs{ORFF}}
\label{sec:learning_with_operator-valued_random-fourier_features} Before
focusing on learning function with an ORFF model, we briefly review the context
of supervised learning in \acs{vv-RKHS}.  model.
\subsection{Supervised learning within \acs{vv-RKHS}}
Let $\seq{s} = (x_i,y_i)_{i=1}^N\in\left(\mathcal{X}\times\mathcal{Y}\right)^N$
be a sequence of training samples. Given a local loss function $L:
\mathcal{X}\times\mathcal{F}\times\mathcal{Y}\to \overline{\mathbb{R}}$ such
that $L$ is proper, convex and lower semi-continuous in $\mathcal{F}$, we are
interested in finding a \emph{vector-valued function}
$f_{\seq{s}}:\mathcal{X}\to\mathcal{Y}$, that lives in a \acs{vv-RKHS} and
minimize a tradeoff between a data fitting term $L$ and a regularization term
to prevent from overfitting. Namely finding $f_{\seq{s}}\in\mathcal{H}_K$ such
that
\begin{dmath}
    f_{\seq{s}} = \argmin_{f\in\mathcal{H}_K}
    \frac{1}{N}\displaystyle\sum_{i=1}^NL(x_i, f, y_i) +
    \frac{\lambda}{2}\norm{f}^2_{K}
    \label{eq:learning_rkhs}
\end{dmath}
where $\lambda\in\mathbb{R}_+$ is a (Tychonov) regularization hyperparameter.
We call the quantity
\begin{dmath*}
    \mathcal{R}_{\lambda}(f,\seq{s})=\frac{1}{N}\displaystyle\sum_{i=1}^NL(x_i,
    f, y_i)+\frac{\lambda}{2}\norm{f}_K^2 \condition{$\forall
    f\in\mathcal{H}_K$, $\forall
    \seq{s}\in\left(\mathcal{X}\times\mathcal{Y}\right)^N$.}
\end{dmath*}
the (Tychonov) regularized risk of the model $f\in\mathcal{H}_K$ according the
local loss $L$. A common choice for $L$ is the squared error loss $L:(x,
f, y) \mapsto \norm{f(x)-y}_{\mathcal{Y}}^2$ which yields the vector-valued
ridge regression problem.
%We introduce a corollary from Mazur and Schauder
%proposed in 1936 (see~\citet{kurdila2006convex, gorniewicz1999topological})
%showing that \cref{eq:learning_rkhs} --and \cref{eq:learning_rkhs_gen}--
%attains a unique mimimizer.
%\begin{theorem}[Mazur-Schauder]
    %\label{cor:unique_minimizer}
    %Let $\mathcal{H}$ be a Hilbert space and $J:\mathcal{H}\to
    %\overline{\mathbb{R}}$ be a proper, convex, lower semi-continuous and
    %coercive function. Then $J$ is bounded from below and attains a minimizer.
    %Moreover if $J$ is strictly convex the minimizer is unique.
%\end{theorem}
%This is easily verified for Ridge regression. Define
%\begin{dmath}
    %\label{eq:ridge}
    %\mathfrak{R}_\lambda(f, \seq{s})=\frac{1}{N}\sum_{i=1}^N\norm{f(x_i) -
    %y_i}_{\mathcal{Y}}^2+ \frac{\lambda}{2}\norm{f}_K^2,
%\end{dmath}
%where $f\in\mathcal{H}_K$ and $\lambda\in\mathbb{R}_{>0}$.
%$\mathfrak{R}_\lambda$ is continuous\footnote{Reminder, if $f\in\mathcal{H}_k,
%\text{ev}_x : f\mapsto f(x)$ is continuous, see \cref{pr:unique_rkhs}.} and
%strictly convex.  Additionally $\mathfrak{R}_\lambda$ is coercive since
%$\norm{f}_K$ is coercive, $\lambda\in\mathbb{R}_{>0}$, and all the summands of
%$\mathfrak{R}_\lambda$ are positive.  Hence for all positive $\lambda$,
%$f_{\seq{s}} = \argmin_{f\in\mathcal{H}_K}\mathfrak{R}_\lambda(f, \seq{s})$
%exists, is unique and attained.
%\begin{remark}[\citet{kadri2015operator}]
    %\label{rk:rkhs_bound} We consider the optimization problem proposed in
    %\cref{eq:ridge} where $L:(x_i, f, y_i) \mapsto
    %\norm{f(x_i)-y_i}_{\mathcal{Y}}^2$. If given a training sample $\seq{s}$,
    %we have
    %\begin{dmath*}
        %\frac{1}{N}\sum_{i=1}^N\norm{y_i}_{\mathcal{Y}}^2 \le \sigma_y^2,
    %\end{dmath*}
    %then $\lambda\norm{f_{\seq{s}}}_K\le 2\sigma_y^2$. Indeed, since
    %$\mathcal{H}_K$ is a Hilbert space, $0\in\mathcal{H}_K$, thus
    %\begin{dmath*}
        %\frac{\lambda}{2}\norm{f_{\seq{s}}}^2_{K} \le
        %\frac{1}{N}\displaystyle\sum_{i=1}^NL(x_i, f_{\seq{s}}, y_i) +
        %\frac{\lambda}{2}\norm{f_{\seq{s}}}^2_{K} \le
        %\frac{1}{N}\displaystyle\sum_{i=1}^NL(x_i, 0, y_i) \hiderel{\le}
        %\sigma_y^2 \condition{by optimality of $f_{\seq{s}}$.}
    %\end{dmath*}
    %Since for all $x\in\mathcal{X}$, $\norm{f(x)}_{\mathcal{Y}}\le
    %\sqrt{\norm{K(x, x)}_{\mathcal{Y},\mathcal{Y}}}\norm{f}_{K}$, the maximum
    %value that the solution $\norm{f_{\seq{s}}(x)}_{\mathcal{Y}}$ of
    %\cref{eq:ridge} can reach is $\sigma_y\sqrt{\frac{2\norm{K(x,
    %x)}_{\mathcal{Y}, \mathcal{Y}}}{\lambda}}$. Thus when solving a Ridge
    %regression problem, given a shift-invariant kernel $K_e$, one should choose
    %\begin{dmath*}
        %0 \hiderel{<} \lambda \hiderel{\le}
        %2\norm{K_e(e)}_{\mathcal{Y}, \mathcal{Y}}\frac{\sigma_y^2}{C^2}.
    %\end{dmath*}
    %with $C\in\mathbb{R}_{>0}$ to have a chance to fit all the $y_i$ with norm
    %$\norm{y_i}_{\mathcal{Y}} \le C$ in the train set.
%\end{remark}

\subsubsection{Representer theorem and Feature equivalence}
Regression in \acl{vv-RKHS} has been well studied~\citep{Alvarez2012,
Argyriou_jmlr09,
Minh_icml13,minh2016unifying,sangnier2016joint,kadri2015operator,Micchelli2005,
Brouard2016_jmlr}, and a cornerstone of learning in \acs{vv-RKHS} is the
representer theorem\footnote{Sometimes referred to as minimal norm
interpolation theorem.}, which allows to replace the search of a minimizer in a
infinite dimensional \acs{vv-RKHS} by a finite number of parameters
$(u_i)_{i=1}^N$, $u_i\in\mathcal{Y}$.
\paragraph{}
In the following we suppose we are given a cost function
$c:\mathcal{Y}\times\mathcal{Y}\to\overline{\mathbb{R}}$, such that $c(f(x),y)$
returns the error of the prediction $f(x)$ \acs{wrt}~the ground truth $y$. A
loss function of a model $f$ with respect to an example
$(x,y)\in\mathcal{X}\times\mathcal{Y}$ can be naturally defined from a cost
function as $L(x,f,y)=c(f(x),y)$. Conceptually the function $c$ evaluates the
quality of the prediction versus its ground truth $y\in\mathcal{Y}$ while the
loss function $L$ evaluates the quality of the model $f$ at a training point
$(x,y)\in\mathcal{X}\times\mathcal{Y}$.
\begin{theorem}[Representer theorem]
    \label{th:representer}
    Let $K$ be a $\mathcal{Y}$-Mercer \acl{OVK} and $\mathcal{H}_K$ its
    corresponding $\mathcal{Y}$-Reproducing Kernel Hilbert space.  Let
    $c:\mathcal{Y}\times\mathcal{Y}\to\overline{\mathbb{R}}$ be a cost function
    such that $L(x, f, y)=c(Vf(x), y)$ is a proper convex lower semi-continuous
    function in $f$ for all $x\in\mathcal{X}$ and all $y\in\mathcal{Y}$.
    Eventually let $\lambda\in\mathbb{R}_{>0}$ be the Tychonov regularization
    hyperparameters The solution $f_{\seq{s}}\in\mathcal{H}_K$ of the
    regularized optimization problem
    \begin{dmath}
        \label{eq:argmin_rkhs}
        f_{\seq{s}} = \argmin_{f\in\mathcal{H}_K}
        \frac{1}{N}\displaystyle\sum_{i=1}^N c(f(x_i), y_i) +
        \frac{\lambda}{2}\norm{f}^2_{K}
        \label{eq:learning_rkhs_gen}
    \end{dmath}
    has the form $f_{\seq{s}}=\sum_{j=1}^{N}K(\cdot,x_j)u_{\seq{s},j}$ where
    $u_{\seq{s},j}\in\mathcal{Y}$ and
    \begin{dmath}
        \label{eq:argmin_u} u_{\seq{s}} =
        \argmin_{u\in\Vect_{i=1}^{N}\mathcal{Y}}\frac{1}{N}
        \displaystyle\sum_{i=1}^N c\left(\sum_{k=1}^{N}K(x_i,x_j)u_j,
        y_i\right) + \frac{\lambda}{2}\sum_{k=1}^{N}u_i^\adjoint
        K(x_i,x_k)u_k.
    \end{dmath}
\end{theorem}
The first representer theorem was introduced by~\citet{Wahba90} in the case
where $\mathcal{Y}=\mathbb{R}$. The extension to an arbitrary Hilbert space
$\mathcal{Y}$ has been proved by many authors in different
forms~\citep{Brouard2011,kadri2015operator,Micchelli2005}. The idea behind the
representer theorem is that even though we minimize over the whole space
$\mathcal{H}_K$, when $\lambda>0$, the solution of \cref{eq:learning_rkhs_gen}
falls inevitably into the set $\mathcal{H}_{K,
\seq{s}}=\Set{\sum_{j=1}^{N}K_{x_j}u_j| \forall (u_i)_{i=1}^{N}
\in\mathcal{Y}^{N}}$.  Therefore the result can be expressed as a finite linear
combination of basis functions of the form $K(\cdot,x_k)$. Notice that we can
perform the kernel expansion of
$f_{\seq{s}}=\sum_{j=1}^{N}K(\cdot,x_j)u_{\seq{s},j}$ even though $\lambda=0$.
However $f_{\seq{s}}$ is no longer the solution of \cref{eq:learning_rkhs_gen}
over the whole space $\mathcal{H}_K$ but a projection on the subspace
$\mathcal{H}_{K, \seq{s}}$. 
%While this is in general not a problem for
%practical applications, it might raise issues for further theoretical
%investigations. In particular, it makes it difficult to perform theoretical
%comparison of the \say{exact} solution of \cref{eq:learning_rkhs_gen} with
%respect to the \acs{ORFF} approximation solution given in
%\cref{th:orff_representer}.  
The representer theorem show that minimizing a
functional in a \acs{vv-RKHS} yields a solution which depends on all the points
in the training set. Assuming that for all $x_i$ and $x\in\mathcal{X}$ and for
all $u_i\in\mathcal{Y}$ it takes time $O(P)$ to compute $K(x_i, x)u_i$, making
a prediction using the representer theorem takes $O(NP)$. Obviously If
$\mathcal{Y}=\mathbb{R}^p$, Then $P=O(p^2)$ thus making a prediction cost
$O(Np^2)$ operations.
%% on commence la partie approchée
\subsection{Learning with Operator Random Fourier Feature maps}
Instead of  learning a model $f$ that depends on all the points of the training
set, we would like to learn a parametric model of the form
$\tildef{\omega}(x) = \tildePhi{\omega}(x)^\adjoint \theta$, where $\theta$
lives in some space $\tildeH{\omega}$. We are interested in
finding a parameter vector $\theta_{\seq{s}}$ such that
\begin{dmath}
    \label{eq:argmin_applied} \theta_{\seq{s}}=
    \argmin_{\theta\in\tildeH{\omega}} \mathfrak{R}_{\lambda}(\theta, \seq{s})
    \hiderel{=}\argmin_{\theta\in \tildeH{\omega}}
    \frac{1}{N}\sum_{i=1}^Nc\left(\tildePhi{\omega}(x_i)^\adjoint \theta,
    y_i\right) + \frac{\lambda}{2}\norm{\theta}^2_{\tildeH{\omega}}
\end{dmath}
The following theorem states that when $\lambda > 0$ then learning with a
feature map is equivalent to learn with a kernel. Moreover if
$f_{\seq{s}}\in\mathcal{H}_K$ is a solution of \cref{eq:argmin_rkhs} and
$\theta_{\seq{s}}\in\mathcal{H}$ is the solution of
\cref{eq:argmin_RKHS_rand}, then $f_{\seq{s}}=\Phi(\cdot)^\adjoint
\theta_{\seq{s}}$. This equivalence could have been obtained by means of
Lagrange duality. However in this proof we do not use such tool: we only
focus on the representer theorem and the fact that there exists a partial
isometry $W$ between the \acs{vv-RKHS} and a feature space $\mathcal{H}$. We
show that if $\theta_{\seq{s}}$ is a solution of $\cref{eq:argmin_applied}$,
then theta belongs to $(\Ker W)^\bot$, thus there is an isometry between
$\theta_{\seq{s}}\in\tilde{\mathcal{H}}$ and $\mathcal{H}_{\widetilde{K}}$:
namely $W$.
\begin{theorem}[Feature equivalence]
    \label{th:orff_representer} Let $\tildeK{\omega}$ be an \acl{OVK} such that
    for all $x$, $z\in\mathcal{X}$, $\tildePhi{\omega}(x)^\adjoint
    \tildePhi{\omega}(z) = \widetilde{K}(x,z)$ where $\widetilde{K}$ is a
    $\mathcal{Y}$-Mercer \acs{OVK} and $\mathcal{H}_{\tildeK{\omega}}$ its
    corresponding $\mathcal{Y}$-Reproducing kernel Hilbert space.  Let
    $c:\mathcal{Y}\times\mathcal{Y}\to\overline{\mathbb{R}}$ be a cost function
    such that $L\left(x, \widetilde{f}, y\right)=c\left(\widetilde{f}(x),
    y\right)$ is a proper convex lower semi-continuous function in
    $\widetilde{f}\in\mathcal{H}_{\tildeK{\omega}}$ for all $x\in\mathcal{X}$
    and all $y\in\mathcal{Y}$.  Eventually let $\lambda\in\mathbb{R}_{>0}
    \mathbb{R}_+$ be the Tychonov regularization hyperparameter. The solution
    $f_{\seq{s}}\in\mathcal{H}_{\tildeK{\omega}}$ of the regularized
    optimization problem
    \begin{dmath}
        \label{eq:argmin_RKHS_rand} \widetilde{f}_{\seq{s}} =
        \argmin_{\widetilde{f}\in\mathcal{H}_{\tildeK{\omega}}}
        \frac{1}{N}\displaystyle\sum_{i=1}^N c\left(\widetilde{f}(x_i),
        y_i\right) +
        \frac{\lambda}{2}\norm{\widetilde{f}}^2_{\tildeK{\omega}}
    \end{dmath}
    has the form $\widetilde{f}_{\seq{s}} = \tildePhi{\omega}(\cdot)^\adjoint
    \theta_{\seq{s}}$, where $\theta_{\seq{s}} \in (\Ker
    \tildeW{\omega})^{\perp}$ and
    \begin{dmath}
        \label{eq:argmin_featurespace}
        \theta_{\seq{s}}=\argmin_{\theta\in \tildeH{\omega}}
        \frac{1}{N}\sum_{i=1}^Nc\left(\tildePhi{\omega}(x_i)^\adjoint \theta,
        y_i\right) + \frac{\lambda}{2}\norm{\theta}^2_{\tildeH{\omega}}
        \label{eq:arming_RKHS_rand_feat}
    \end{dmath}
\end{theorem}
In the aforementioned theorem, we use the notation $\widetilde{K}$ and
$\tildePhi{\omega}$ because our main subject of interest is the \acs{ORFF} map.
However this theorem works for \emph{any} feature maps $\Phi(x)\in
\mathcal{L}(\mathcal{Y}, \mathcal{H})$ even when $\mathcal{H}$ is infinite
dimensional.\footnote{If $\Phi(x): \mathcal{L}(\mathcal{Y}, \mathcal{H})$ and
$\dim(\mathcal{H})=\infty$, the decomposition $\mathcal{H}=(\Ker W) \oplus
(\Ker W)^\perp$ holds since $\mathcal{H}$ is a Hilbert space and $W$ is a
bounded operator.}.  This shows that when $\lambda>0$ the solution of
\cref{eq:argmin_u} with the approximated kernel $K(x,z) \approx
\tildeK{\omega}(x,z) = \tildePhi{\omega}(x)^\adjoint\tildePhi{\omega}(z)$ is
the same than the solution of \cref{eq:argmin_featurespace} up to an isometric
isomorphm (see \cref{subsubsec:proof_feature_equiv}). Namely, if $u_{\seq{s}}$
is the solution of \cref{eq:argmin_u}, $\theta_{\seq{s}}$ is the solution of
\cref{eq:argmin_featurespace} and $\lambda>0$ we have
\begin{dmath*}
    \theta_{\seq{s}} = \sum_{i=1}^{N} \tildePhi{\omega}(x_i) (u_{\seq{s}})_i
    \hiderel{\in} (\Ker W)^{\perp} \hiderel{\subseteq} \tildeH{\omega}.
\end{dmath*}
If $\lambda_K=0$ we can still find a solution $u_{\seq{s}}$ of
\cref{eq:argmin_u}. By construction of the kernel expansion, we have
$u_{\seq{s}}\in(\Ker W)^\bot$. However looking at the proof of
\cref{th:orff_representer} we see that $\theta_{\seq{s}}$ might \emph{not}
belong to $(\Ker W)^\bot$. We can compute a residual vector $r_{\seq{s}} =
\sum_{i=1}^{N} \tildePhi{\omega}(x_i) (u_{\seq{s}})_i - \theta_{\seq{s}}$.
Since $\sum_{j=1}^N \tildePhi{\omega}(x_j)\in(\Ker W)^\bot$ by construction, if
$r_{\seq{s}}=0$, it means that $\lambda_K$ is large enough for both representer
theorem and \acs{ORFF} representer theorem to apply. If $r_{\seq{s}}\neq 0$ but
$\tildePhi{\omega}(\cdot)^\adjoint r_{\seq{s}} = 0$ it means that both
$\theta_{\seq{s}}$ and $\sum_{j=1}^{N} \tildePhi{\omega}(x_j) u_{\seq{s}}$ are
in $(\Ker W)^\bot$, thus the representer theorem fails to find the \say{true}
solution over the whole space $\mathcal{H}_{\widetilde{K}}$ but returns a
projection onto $\mathcal{H}_{\tildeK{\omega},\seq{s}}$ of the solution. If
$r_{\seq{s}} \neq 0$ and $\tildePhi{\omega}(\cdot)^\adjoint r_{\seq{s}} \neq 0$
means that $\theta_{\seq{s}}$ is \emph{not} in $(\Ker W)^\bot$, thus the
feature equivalence theorem fails to apply. Since $r_{\seq{s}} = \sum_{i=1}^N
\tildePhi{\omega}(x_i)(u_{\seq{s}})_i - \theta_{\seq{s}}^\perp -
\theta_{\seq{s}}^\parallel$ and $\sum_{i=1}^N
\tildePhi{\omega}(x_i)(u_{\seq{s}})_i$ is in $(\Ker W)^\perp$, with mild abuse
of notation we write $r_{\seq{s}}=\theta^\parallel$. This remark is illustrated
in \cref{fig:representer}.
\paragraph{}
In \cref{fig:representer}, we generated the data from a since wave to which we
add some Gaussian noise. We learned a Gaussian kernel based \ac{RFF} model
(blue curve) and a kernel model (yellow curve) where the kernel is obtained
from the \acs{RFF} map. The left column represents the fit of the model to the
points for four different valued of $\lambda$ (top to bottom: $10^{-2}$,
$10^{-5}$, $10e^{10}$, $0$). The middle column shows if the \acs{RFF} solution
$\theta_{\seq{s}}$ is in $(\Ker \tilde{W})^\perp$.  This is true for all values
of $\lambda$. The right column shows that even though $\theta_{\seq{s}}$ is in
$(\Ker \tilde{W})^\perp$, when $\lambda\to0$ learning with \acs{RFF} is
different from learning with the kernel constructed from the \acs{RFF} maps
since the coefficients of $\theta^{\parallel}$ are all different from $0$.
%\paragraph{}
%\cref{fig:representer2} is the same setting than \cref{fig:representer} except
%that we decreased the scale parameter $\sigma$ of the Gaussian kernel to make
%it overfit, and emphasize that when $\lambda=0$, $\theta_{\seq{s}}$ might not
%belong to $(\Ker \tilde{W})^\perp$, as represented on the middle column.

\begin{pycode}[representer]
sys.path.append('./src/')
import representer

err = representer.main()
\end{pycode}

%\begin{pycode}[representer2]
%sys.path.append('./src/')
%import representer2

%err = representer2.main()
%\end{pycode}

\afterpage{%
\begin{landscape}
    \begin{figure}
        \pyc{print(r'\centering\resizebox{1.5\textwidth}{!}{%% Creator: Matplotlib, PGF backend
%%
%% To include the figure in your LaTeX document, write
%%   \input{<filename>.pgf}
%%
%% Make sure the required packages are loaded in your preamble
%%   \usepackage{pgf}
%%
%% Figures using additional raster images can only be included by \input if
%% they are in the same directory as the main LaTeX file. For loading figures
%% from other directories you can use the `import` package
%%   \usepackage{import}
%% and then include the figures with
%%   \import{<path to file>}{<filename>.pgf}
%%
%% Matplotlib used the following preamble
%%   \usepackage{fontspec}
%%   \setmainfont{DejaVu Serif}
%%   \setsansfont{DejaVu Sans}
%%   \setmonofont{DejaVu Sans Mono}
%%
\begingroup%
\makeatletter%
\begin{pgfpicture}%
\pgfpathrectangle{\pgfpointorigin}{\pgfqpoint{12.991635in}{5.344870in}}%
\pgfusepath{use as bounding box, clip}%
\begin{pgfscope}%
\pgfsetbuttcap%
\pgfsetmiterjoin%
\definecolor{currentfill}{rgb}{1.000000,1.000000,1.000000}%
\pgfsetfillcolor{currentfill}%
\pgfsetlinewidth{0.000000pt}%
\definecolor{currentstroke}{rgb}{1.000000,1.000000,1.000000}%
\pgfsetstrokecolor{currentstroke}%
\pgfsetdash{}{0pt}%
\pgfpathmoveto{\pgfqpoint{0.000000in}{0.000000in}}%
\pgfpathlineto{\pgfqpoint{12.991635in}{0.000000in}}%
\pgfpathlineto{\pgfqpoint{12.991635in}{5.344870in}}%
\pgfpathlineto{\pgfqpoint{0.000000in}{5.344870in}}%
\pgfpathclose%
\pgfusepath{fill}%
\end{pgfscope}%
\begin{pgfscope}%
\pgfsetbuttcap%
\pgfsetmiterjoin%
\definecolor{currentfill}{rgb}{1.000000,1.000000,1.000000}%
\pgfsetfillcolor{currentfill}%
\pgfsetlinewidth{0.000000pt}%
\definecolor{currentstroke}{rgb}{0.000000,0.000000,0.000000}%
\pgfsetstrokecolor{currentstroke}%
\pgfsetstrokeopacity{0.000000}%
\pgfsetdash{}{0pt}%
\pgfpathmoveto{\pgfqpoint{0.456635in}{4.237239in}}%
\pgfpathlineto{\pgfqpoint{4.833106in}{4.237239in}}%
\pgfpathlineto{\pgfqpoint{4.833106in}{5.209870in}}%
\pgfpathlineto{\pgfqpoint{0.456635in}{5.209870in}}%
\pgfpathclose%
\pgfusepath{fill}%
\end{pgfscope}%
\begin{pgfscope}%
\pgfpathrectangle{\pgfqpoint{0.456635in}{4.237239in}}{\pgfqpoint{4.376471in}{0.972632in}}%
\pgfusepath{clip}%
\pgfsetbuttcap%
\pgfsetroundjoin%
\definecolor{currentfill}{rgb}{1.000000,0.000000,0.000000}%
\pgfsetfillcolor{currentfill}%
\pgfsetlinewidth{2.007500pt}%
\definecolor{currentstroke}{rgb}{1.000000,0.000000,0.000000}%
\pgfsetstrokecolor{currentstroke}%
\pgfsetdash{}{0pt}%
\pgfpathmoveto{\pgfqpoint{2.601594in}{4.391106in}}%
\pgfpathlineto{\pgfqpoint{2.684927in}{4.391106in}}%
\pgfpathmoveto{\pgfqpoint{2.643260in}{4.349440in}}%
\pgfpathlineto{\pgfqpoint{2.643260in}{4.432773in}}%
\pgfusepath{stroke,fill}%
\end{pgfscope}%
\begin{pgfscope}%
\pgfpathrectangle{\pgfqpoint{0.456635in}{4.237239in}}{\pgfqpoint{4.376471in}{0.972632in}}%
\pgfusepath{clip}%
\pgfsetbuttcap%
\pgfsetroundjoin%
\definecolor{currentfill}{rgb}{1.000000,0.000000,0.000000}%
\pgfsetfillcolor{currentfill}%
\pgfsetlinewidth{2.007500pt}%
\definecolor{currentstroke}{rgb}{1.000000,0.000000,0.000000}%
\pgfsetstrokecolor{currentstroke}%
\pgfsetdash{}{0pt}%
\pgfpathmoveto{\pgfqpoint{4.618881in}{5.050208in}}%
\pgfpathlineto{\pgfqpoint{4.702215in}{5.050208in}}%
\pgfpathmoveto{\pgfqpoint{4.660548in}{5.008541in}}%
\pgfpathlineto{\pgfqpoint{4.660548in}{5.091875in}}%
\pgfusepath{stroke,fill}%
\end{pgfscope}%
\begin{pgfscope}%
\pgfpathrectangle{\pgfqpoint{0.456635in}{4.237239in}}{\pgfqpoint{4.376471in}{0.972632in}}%
\pgfusepath{clip}%
\pgfsetbuttcap%
\pgfsetroundjoin%
\definecolor{currentfill}{rgb}{1.000000,0.000000,0.000000}%
\pgfsetfillcolor{currentfill}%
\pgfsetlinewidth{2.007500pt}%
\definecolor{currentstroke}{rgb}{1.000000,0.000000,0.000000}%
\pgfsetstrokecolor{currentstroke}%
\pgfsetdash{}{0pt}%
\pgfpathmoveto{\pgfqpoint{3.853103in}{4.426269in}}%
\pgfpathlineto{\pgfqpoint{3.936436in}{4.426269in}}%
\pgfpathmoveto{\pgfqpoint{3.894769in}{4.384603in}}%
\pgfpathlineto{\pgfqpoint{3.894769in}{4.467936in}}%
\pgfusepath{stroke,fill}%
\end{pgfscope}%
\begin{pgfscope}%
\pgfpathrectangle{\pgfqpoint{0.456635in}{4.237239in}}{\pgfqpoint{4.376471in}{0.972632in}}%
\pgfusepath{clip}%
\pgfsetbuttcap%
\pgfsetroundjoin%
\definecolor{currentfill}{rgb}{1.000000,0.000000,0.000000}%
\pgfsetfillcolor{currentfill}%
\pgfsetlinewidth{2.007500pt}%
\definecolor{currentstroke}{rgb}{1.000000,0.000000,0.000000}%
\pgfsetstrokecolor{currentstroke}%
\pgfsetdash{}{0pt}%
\pgfpathmoveto{\pgfqpoint{3.386272in}{4.774649in}}%
\pgfpathlineto{\pgfqpoint{3.469605in}{4.774649in}}%
\pgfpathmoveto{\pgfqpoint{3.427938in}{4.732982in}}%
\pgfpathlineto{\pgfqpoint{3.427938in}{4.816316in}}%
\pgfusepath{stroke,fill}%
\end{pgfscope}%
\begin{pgfscope}%
\pgfpathrectangle{\pgfqpoint{0.456635in}{4.237239in}}{\pgfqpoint{4.376471in}{0.972632in}}%
\pgfusepath{clip}%
\pgfsetbuttcap%
\pgfsetroundjoin%
\definecolor{currentfill}{rgb}{1.000000,0.000000,0.000000}%
\pgfsetfillcolor{currentfill}%
\pgfsetlinewidth{2.007500pt}%
\definecolor{currentstroke}{rgb}{1.000000,0.000000,0.000000}%
\pgfsetstrokecolor{currentstroke}%
\pgfsetdash{}{0pt}%
\pgfpathmoveto{\pgfqpoint{1.836512in}{4.619003in}}%
\pgfpathlineto{\pgfqpoint{1.919845in}{4.619003in}}%
\pgfpathmoveto{\pgfqpoint{1.878178in}{4.577336in}}%
\pgfpathlineto{\pgfqpoint{1.878178in}{4.660669in}}%
\pgfusepath{stroke,fill}%
\end{pgfscope}%
\begin{pgfscope}%
\pgfpathrectangle{\pgfqpoint{0.456635in}{4.237239in}}{\pgfqpoint{4.376471in}{0.972632in}}%
\pgfusepath{clip}%
\pgfsetbuttcap%
\pgfsetroundjoin%
\definecolor{currentfill}{rgb}{1.000000,0.000000,0.000000}%
\pgfsetfillcolor{currentfill}%
\pgfsetlinewidth{2.007500pt}%
\definecolor{currentstroke}{rgb}{1.000000,0.000000,0.000000}%
\pgfsetstrokecolor{currentstroke}%
\pgfsetdash{}{0pt}%
\pgfpathmoveto{\pgfqpoint{1.836427in}{4.867576in}}%
\pgfpathlineto{\pgfqpoint{1.919760in}{4.867576in}}%
\pgfpathmoveto{\pgfqpoint{1.878094in}{4.825909in}}%
\pgfpathlineto{\pgfqpoint{1.878094in}{4.909243in}}%
\pgfusepath{stroke,fill}%
\end{pgfscope}%
\begin{pgfscope}%
\pgfpathrectangle{\pgfqpoint{0.456635in}{4.237239in}}{\pgfqpoint{4.376471in}{0.972632in}}%
\pgfusepath{clip}%
\pgfsetbuttcap%
\pgfsetroundjoin%
\definecolor{currentfill}{rgb}{1.000000,0.000000,0.000000}%
\pgfsetfillcolor{currentfill}%
\pgfsetlinewidth{2.007500pt}%
\definecolor{currentstroke}{rgb}{1.000000,0.000000,0.000000}%
\pgfsetstrokecolor{currentstroke}%
\pgfsetdash{}{0pt}%
\pgfpathmoveto{\pgfqpoint{1.493624in}{4.805046in}}%
\pgfpathlineto{\pgfqpoint{1.576957in}{4.805046in}}%
\pgfpathmoveto{\pgfqpoint{1.535290in}{4.763379in}}%
\pgfpathlineto{\pgfqpoint{1.535290in}{4.846712in}}%
\pgfusepath{stroke,fill}%
\end{pgfscope}%
\begin{pgfscope}%
\pgfpathrectangle{\pgfqpoint{0.456635in}{4.237239in}}{\pgfqpoint{4.376471in}{0.972632in}}%
\pgfusepath{clip}%
\pgfsetbuttcap%
\pgfsetroundjoin%
\definecolor{currentfill}{rgb}{1.000000,0.000000,0.000000}%
\pgfsetfillcolor{currentfill}%
\pgfsetlinewidth{2.007500pt}%
\definecolor{currentstroke}{rgb}{1.000000,0.000000,0.000000}%
\pgfsetstrokecolor{currentstroke}%
\pgfsetdash{}{0pt}%
\pgfpathmoveto{\pgfqpoint{4.322898in}{4.886956in}}%
\pgfpathlineto{\pgfqpoint{4.406232in}{4.886956in}}%
\pgfpathmoveto{\pgfqpoint{4.364565in}{4.845290in}}%
\pgfpathlineto{\pgfqpoint{4.364565in}{4.928623in}}%
\pgfusepath{stroke,fill}%
\end{pgfscope}%
\begin{pgfscope}%
\pgfpathrectangle{\pgfqpoint{0.456635in}{4.237239in}}{\pgfqpoint{4.376471in}{0.972632in}}%
\pgfusepath{clip}%
\pgfsetbuttcap%
\pgfsetroundjoin%
\definecolor{currentfill}{rgb}{1.000000,0.000000,0.000000}%
\pgfsetfillcolor{currentfill}%
\pgfsetlinewidth{2.007500pt}%
\definecolor{currentstroke}{rgb}{1.000000,0.000000,0.000000}%
\pgfsetstrokecolor{currentstroke}%
\pgfsetdash{}{0pt}%
\pgfpathmoveto{\pgfqpoint{3.394872in}{4.884213in}}%
\pgfpathlineto{\pgfqpoint{3.478206in}{4.884213in}}%
\pgfpathmoveto{\pgfqpoint{3.436539in}{4.842546in}}%
\pgfpathlineto{\pgfqpoint{3.436539in}{4.925879in}}%
\pgfusepath{stroke,fill}%
\end{pgfscope}%
\begin{pgfscope}%
\pgfpathrectangle{\pgfqpoint{0.456635in}{4.237239in}}{\pgfqpoint{4.376471in}{0.972632in}}%
\pgfusepath{clip}%
\pgfsetbuttcap%
\pgfsetroundjoin%
\definecolor{currentfill}{rgb}{1.000000,0.000000,0.000000}%
\pgfsetfillcolor{currentfill}%
\pgfsetlinewidth{2.007500pt}%
\definecolor{currentstroke}{rgb}{1.000000,0.000000,0.000000}%
\pgfsetstrokecolor{currentstroke}%
\pgfsetdash{}{0pt}%
\pgfpathmoveto{\pgfqpoint{3.769350in}{4.375818in}}%
\pgfpathlineto{\pgfqpoint{3.852683in}{4.375818in}}%
\pgfpathmoveto{\pgfqpoint{3.811016in}{4.334151in}}%
\pgfpathlineto{\pgfqpoint{3.811016in}{4.417485in}}%
\pgfusepath{stroke,fill}%
\end{pgfscope}%
\begin{pgfscope}%
\pgfpathrectangle{\pgfqpoint{0.456635in}{4.237239in}}{\pgfqpoint{4.376471in}{0.972632in}}%
\pgfusepath{clip}%
\pgfsetbuttcap%
\pgfsetroundjoin%
\definecolor{currentfill}{rgb}{1.000000,0.000000,0.000000}%
\pgfsetfillcolor{currentfill}%
\pgfsetlinewidth{2.007500pt}%
\definecolor{currentstroke}{rgb}{1.000000,0.000000,0.000000}%
\pgfsetstrokecolor{currentstroke}%
\pgfsetdash{}{0pt}%
\pgfpathmoveto{\pgfqpoint{1.362333in}{5.028707in}}%
\pgfpathlineto{\pgfqpoint{1.445666in}{5.028707in}}%
\pgfpathmoveto{\pgfqpoint{1.403999in}{4.987041in}}%
\pgfpathlineto{\pgfqpoint{1.403999in}{5.070374in}}%
\pgfusepath{stroke,fill}%
\end{pgfscope}%
\begin{pgfscope}%
\pgfpathrectangle{\pgfqpoint{0.456635in}{4.237239in}}{\pgfqpoint{4.376471in}{0.972632in}}%
\pgfusepath{clip}%
\pgfsetbuttcap%
\pgfsetroundjoin%
\definecolor{currentfill}{rgb}{1.000000,0.000000,0.000000}%
\pgfsetfillcolor{currentfill}%
\pgfsetlinewidth{2.007500pt}%
\definecolor{currentstroke}{rgb}{1.000000,0.000000,0.000000}%
\pgfsetstrokecolor{currentstroke}%
\pgfsetdash{}{0pt}%
\pgfpathmoveto{\pgfqpoint{4.686088in}{4.766769in}}%
\pgfpathlineto{\pgfqpoint{4.769422in}{4.766769in}}%
\pgfpathmoveto{\pgfqpoint{4.727755in}{4.725102in}}%
\pgfpathlineto{\pgfqpoint{4.727755in}{4.808435in}}%
\pgfusepath{stroke,fill}%
\end{pgfscope}%
\begin{pgfscope}%
\pgfpathrectangle{\pgfqpoint{0.456635in}{4.237239in}}{\pgfqpoint{4.376471in}{0.972632in}}%
\pgfusepath{clip}%
\pgfsetbuttcap%
\pgfsetroundjoin%
\definecolor{currentfill}{rgb}{1.000000,0.000000,0.000000}%
\pgfsetfillcolor{currentfill}%
\pgfsetlinewidth{2.007500pt}%
\definecolor{currentstroke}{rgb}{1.000000,0.000000,0.000000}%
\pgfsetstrokecolor{currentstroke}%
\pgfsetdash{}{0pt}%
\pgfpathmoveto{\pgfqpoint{4.204791in}{4.719021in}}%
\pgfpathlineto{\pgfqpoint{4.288125in}{4.719021in}}%
\pgfpathmoveto{\pgfqpoint{4.246458in}{4.677354in}}%
\pgfpathlineto{\pgfqpoint{4.246458in}{4.760687in}}%
\pgfusepath{stroke,fill}%
\end{pgfscope}%
\begin{pgfscope}%
\pgfpathrectangle{\pgfqpoint{0.456635in}{4.237239in}}{\pgfqpoint{4.376471in}{0.972632in}}%
\pgfusepath{clip}%
\pgfsetbuttcap%
\pgfsetroundjoin%
\definecolor{currentfill}{rgb}{1.000000,0.000000,0.000000}%
\pgfsetfillcolor{currentfill}%
\pgfsetlinewidth{2.007500pt}%
\definecolor{currentstroke}{rgb}{1.000000,0.000000,0.000000}%
\pgfsetstrokecolor{currentstroke}%
\pgfsetdash{}{0pt}%
\pgfpathmoveto{\pgfqpoint{2.033699in}{4.858492in}}%
\pgfpathlineto{\pgfqpoint{2.117033in}{4.858492in}}%
\pgfpathmoveto{\pgfqpoint{2.075366in}{4.816826in}}%
\pgfpathlineto{\pgfqpoint{2.075366in}{4.900159in}}%
\pgfusepath{stroke,fill}%
\end{pgfscope}%
\begin{pgfscope}%
\pgfpathrectangle{\pgfqpoint{0.456635in}{4.237239in}}{\pgfqpoint{4.376471in}{0.972632in}}%
\pgfusepath{clip}%
\pgfsetbuttcap%
\pgfsetroundjoin%
\definecolor{currentfill}{rgb}{1.000000,0.000000,0.000000}%
\pgfsetfillcolor{currentfill}%
\pgfsetlinewidth{2.007500pt}%
\definecolor{currentstroke}{rgb}{1.000000,0.000000,0.000000}%
\pgfsetstrokecolor{currentstroke}%
\pgfsetdash{}{0pt}%
\pgfpathmoveto{\pgfqpoint{1.926864in}{4.824645in}}%
\pgfpathlineto{\pgfqpoint{2.010197in}{4.824645in}}%
\pgfpathmoveto{\pgfqpoint{1.968531in}{4.782978in}}%
\pgfpathlineto{\pgfqpoint{1.968531in}{4.866311in}}%
\pgfusepath{stroke,fill}%
\end{pgfscope}%
\begin{pgfscope}%
\pgfpathrectangle{\pgfqpoint{0.456635in}{4.237239in}}{\pgfqpoint{4.376471in}{0.972632in}}%
\pgfusepath{clip}%
\pgfsetbuttcap%
\pgfsetroundjoin%
\definecolor{currentfill}{rgb}{1.000000,0.000000,0.000000}%
\pgfsetfillcolor{currentfill}%
\pgfsetlinewidth{2.007500pt}%
\definecolor{currentstroke}{rgb}{1.000000,0.000000,0.000000}%
\pgfsetstrokecolor{currentstroke}%
\pgfsetdash{}{0pt}%
\pgfpathmoveto{\pgfqpoint{1.932394in}{4.771951in}}%
\pgfpathlineto{\pgfqpoint{2.015728in}{4.771951in}}%
\pgfpathmoveto{\pgfqpoint{1.974061in}{4.730284in}}%
\pgfpathlineto{\pgfqpoint{1.974061in}{4.813618in}}%
\pgfusepath{stroke,fill}%
\end{pgfscope}%
\begin{pgfscope}%
\pgfpathrectangle{\pgfqpoint{0.456635in}{4.237239in}}{\pgfqpoint{4.376471in}{0.972632in}}%
\pgfusepath{clip}%
\pgfsetbuttcap%
\pgfsetroundjoin%
\definecolor{currentfill}{rgb}{1.000000,0.000000,0.000000}%
\pgfsetfillcolor{currentfill}%
\pgfsetlinewidth{2.007500pt}%
\definecolor{currentstroke}{rgb}{1.000000,0.000000,0.000000}%
\pgfsetstrokecolor{currentstroke}%
\pgfsetdash{}{0pt}%
\pgfpathmoveto{\pgfqpoint{2.355469in}{4.793955in}}%
\pgfpathlineto{\pgfqpoint{2.438802in}{4.793955in}}%
\pgfpathmoveto{\pgfqpoint{2.397135in}{4.752289in}}%
\pgfpathlineto{\pgfqpoint{2.397135in}{4.835622in}}%
\pgfusepath{stroke,fill}%
\end{pgfscope}%
\begin{pgfscope}%
\pgfpathrectangle{\pgfqpoint{0.456635in}{4.237239in}}{\pgfqpoint{4.376471in}{0.972632in}}%
\pgfusepath{clip}%
\pgfsetbuttcap%
\pgfsetroundjoin%
\definecolor{currentfill}{rgb}{1.000000,0.000000,0.000000}%
\pgfsetfillcolor{currentfill}%
\pgfsetlinewidth{2.007500pt}%
\definecolor{currentstroke}{rgb}{1.000000,0.000000,0.000000}%
\pgfsetstrokecolor{currentstroke}%
\pgfsetdash{}{0pt}%
\pgfpathmoveto{\pgfqpoint{3.127528in}{4.668742in}}%
\pgfpathlineto{\pgfqpoint{3.210861in}{4.668742in}}%
\pgfpathmoveto{\pgfqpoint{3.169194in}{4.627075in}}%
\pgfpathlineto{\pgfqpoint{3.169194in}{4.710409in}}%
\pgfusepath{stroke,fill}%
\end{pgfscope}%
\begin{pgfscope}%
\pgfpathrectangle{\pgfqpoint{0.456635in}{4.237239in}}{\pgfqpoint{4.376471in}{0.972632in}}%
\pgfusepath{clip}%
\pgfsetbuttcap%
\pgfsetroundjoin%
\definecolor{currentfill}{rgb}{1.000000,0.000000,0.000000}%
\pgfsetfillcolor{currentfill}%
\pgfsetlinewidth{2.007500pt}%
\definecolor{currentstroke}{rgb}{1.000000,0.000000,0.000000}%
\pgfsetstrokecolor{currentstroke}%
\pgfsetdash{}{0pt}%
\pgfpathmoveto{\pgfqpoint{2.802578in}{4.261493in}}%
\pgfpathlineto{\pgfqpoint{2.885912in}{4.261493in}}%
\pgfpathmoveto{\pgfqpoint{2.844245in}{4.219826in}}%
\pgfpathlineto{\pgfqpoint{2.844245in}{4.303160in}}%
\pgfusepath{stroke,fill}%
\end{pgfscope}%
\begin{pgfscope}%
\pgfpathrectangle{\pgfqpoint{0.456635in}{4.237239in}}{\pgfqpoint{4.376471in}{0.972632in}}%
\pgfusepath{clip}%
\pgfsetbuttcap%
\pgfsetroundjoin%
\definecolor{currentfill}{rgb}{1.000000,0.000000,0.000000}%
\pgfsetfillcolor{currentfill}%
\pgfsetlinewidth{2.007500pt}%
\definecolor{currentstroke}{rgb}{1.000000,0.000000,0.000000}%
\pgfsetstrokecolor{currentstroke}%
\pgfsetdash{}{0pt}%
\pgfpathmoveto{\pgfqpoint{2.309907in}{4.773812in}}%
\pgfpathlineto{\pgfqpoint{2.393241in}{4.773812in}}%
\pgfpathmoveto{\pgfqpoint{2.351574in}{4.732145in}}%
\pgfpathlineto{\pgfqpoint{2.351574in}{4.815478in}}%
\pgfusepath{stroke,fill}%
\end{pgfscope}%
\begin{pgfscope}%
\pgfpathrectangle{\pgfqpoint{0.456635in}{4.237239in}}{\pgfqpoint{4.376471in}{0.972632in}}%
\pgfusepath{clip}%
\pgfsetbuttcap%
\pgfsetroundjoin%
\definecolor{currentfill}{rgb}{1.000000,0.000000,0.000000}%
\pgfsetfillcolor{currentfill}%
\pgfsetlinewidth{2.007500pt}%
\definecolor{currentstroke}{rgb}{1.000000,0.000000,0.000000}%
\pgfsetstrokecolor{currentstroke}%
\pgfsetdash{}{0pt}%
\pgfpathmoveto{\pgfqpoint{3.432468in}{4.734883in}}%
\pgfpathlineto{\pgfqpoint{3.515801in}{4.734883in}}%
\pgfpathmoveto{\pgfqpoint{3.474134in}{4.693216in}}%
\pgfpathlineto{\pgfqpoint{3.474134in}{4.776550in}}%
\pgfusepath{stroke,fill}%
\end{pgfscope}%
\begin{pgfscope}%
\pgfpathrectangle{\pgfqpoint{0.456635in}{4.237239in}}{\pgfqpoint{4.376471in}{0.972632in}}%
\pgfusepath{clip}%
\pgfsetbuttcap%
\pgfsetroundjoin%
\definecolor{currentfill}{rgb}{1.000000,0.000000,0.000000}%
\pgfsetfillcolor{currentfill}%
\pgfsetlinewidth{2.007500pt}%
\definecolor{currentstroke}{rgb}{1.000000,0.000000,0.000000}%
\pgfsetstrokecolor{currentstroke}%
\pgfsetdash{}{0pt}%
\pgfpathmoveto{\pgfqpoint{1.778655in}{4.757697in}}%
\pgfpathlineto{\pgfqpoint{1.861989in}{4.757697in}}%
\pgfpathmoveto{\pgfqpoint{1.820322in}{4.716030in}}%
\pgfpathlineto{\pgfqpoint{1.820322in}{4.799363in}}%
\pgfusepath{stroke,fill}%
\end{pgfscope}%
\begin{pgfscope}%
\pgfpathrectangle{\pgfqpoint{0.456635in}{4.237239in}}{\pgfqpoint{4.376471in}{0.972632in}}%
\pgfusepath{clip}%
\pgfsetbuttcap%
\pgfsetroundjoin%
\definecolor{currentfill}{rgb}{1.000000,0.000000,0.000000}%
\pgfsetfillcolor{currentfill}%
\pgfsetlinewidth{2.007500pt}%
\definecolor{currentstroke}{rgb}{1.000000,0.000000,0.000000}%
\pgfsetstrokecolor{currentstroke}%
\pgfsetdash{}{0pt}%
\pgfpathmoveto{\pgfqpoint{2.313113in}{4.879012in}}%
\pgfpathlineto{\pgfqpoint{2.396446in}{4.879012in}}%
\pgfpathmoveto{\pgfqpoint{2.354779in}{4.837345in}}%
\pgfpathlineto{\pgfqpoint{2.354779in}{4.920678in}}%
\pgfusepath{stroke,fill}%
\end{pgfscope}%
\begin{pgfscope}%
\pgfpathrectangle{\pgfqpoint{0.456635in}{4.237239in}}{\pgfqpoint{4.376471in}{0.972632in}}%
\pgfusepath{clip}%
\pgfsetbuttcap%
\pgfsetroundjoin%
\definecolor{currentfill}{rgb}{1.000000,0.000000,0.000000}%
\pgfsetfillcolor{currentfill}%
\pgfsetlinewidth{2.007500pt}%
\definecolor{currentstroke}{rgb}{1.000000,0.000000,0.000000}%
\pgfsetstrokecolor{currentstroke}%
\pgfsetdash{}{0pt}%
\pgfpathmoveto{\pgfqpoint{2.572960in}{4.360745in}}%
\pgfpathlineto{\pgfqpoint{2.656294in}{4.360745in}}%
\pgfpathmoveto{\pgfqpoint{2.614627in}{4.319078in}}%
\pgfpathlineto{\pgfqpoint{2.614627in}{4.402411in}}%
\pgfusepath{stroke,fill}%
\end{pgfscope}%
\begin{pgfscope}%
\pgfpathrectangle{\pgfqpoint{0.456635in}{4.237239in}}{\pgfqpoint{4.376471in}{0.972632in}}%
\pgfusepath{clip}%
\pgfsetbuttcap%
\pgfsetroundjoin%
\definecolor{currentfill}{rgb}{1.000000,0.000000,0.000000}%
\pgfsetfillcolor{currentfill}%
\pgfsetlinewidth{2.007500pt}%
\definecolor{currentstroke}{rgb}{1.000000,0.000000,0.000000}%
\pgfsetstrokecolor{currentstroke}%
\pgfsetdash{}{0pt}%
\pgfpathmoveto{\pgfqpoint{2.887044in}{4.481944in}}%
\pgfpathlineto{\pgfqpoint{2.970378in}{4.481944in}}%
\pgfpathmoveto{\pgfqpoint{2.928711in}{4.440277in}}%
\pgfpathlineto{\pgfqpoint{2.928711in}{4.523610in}}%
\pgfusepath{stroke,fill}%
\end{pgfscope}%
\begin{pgfscope}%
\pgfpathrectangle{\pgfqpoint{0.456635in}{4.237239in}}{\pgfqpoint{4.376471in}{0.972632in}}%
\pgfusepath{clip}%
\pgfsetbuttcap%
\pgfsetroundjoin%
\definecolor{currentfill}{rgb}{1.000000,0.000000,0.000000}%
\pgfsetfillcolor{currentfill}%
\pgfsetlinewidth{2.007500pt}%
\definecolor{currentstroke}{rgb}{1.000000,0.000000,0.000000}%
\pgfsetstrokecolor{currentstroke}%
\pgfsetdash{}{0pt}%
\pgfpathmoveto{\pgfqpoint{4.039302in}{4.583803in}}%
\pgfpathlineto{\pgfqpoint{4.122636in}{4.583803in}}%
\pgfpathmoveto{\pgfqpoint{4.080969in}{4.542136in}}%
\pgfpathlineto{\pgfqpoint{4.080969in}{4.625470in}}%
\pgfusepath{stroke,fill}%
\end{pgfscope}%
\begin{pgfscope}%
\pgfpathrectangle{\pgfqpoint{0.456635in}{4.237239in}}{\pgfqpoint{4.376471in}{0.972632in}}%
\pgfusepath{clip}%
\pgfsetbuttcap%
\pgfsetroundjoin%
\definecolor{currentfill}{rgb}{1.000000,0.000000,0.000000}%
\pgfsetfillcolor{currentfill}%
\pgfsetlinewidth{2.007500pt}%
\definecolor{currentstroke}{rgb}{1.000000,0.000000,0.000000}%
\pgfsetstrokecolor{currentstroke}%
\pgfsetdash{}{0pt}%
\pgfpathmoveto{\pgfqpoint{1.989356in}{4.734953in}}%
\pgfpathlineto{\pgfqpoint{2.072689in}{4.734953in}}%
\pgfpathmoveto{\pgfqpoint{2.031023in}{4.693286in}}%
\pgfpathlineto{\pgfqpoint{2.031023in}{4.776619in}}%
\pgfusepath{stroke,fill}%
\end{pgfscope}%
\begin{pgfscope}%
\pgfpathrectangle{\pgfqpoint{0.456635in}{4.237239in}}{\pgfqpoint{4.376471in}{0.972632in}}%
\pgfusepath{clip}%
\pgfsetbuttcap%
\pgfsetroundjoin%
\definecolor{currentfill}{rgb}{1.000000,0.000000,0.000000}%
\pgfsetfillcolor{currentfill}%
\pgfsetlinewidth{2.007500pt}%
\definecolor{currentstroke}{rgb}{1.000000,0.000000,0.000000}%
\pgfsetstrokecolor{currentstroke}%
\pgfsetdash{}{0pt}%
\pgfpathmoveto{\pgfqpoint{3.090688in}{4.721187in}}%
\pgfpathlineto{\pgfqpoint{3.174022in}{4.721187in}}%
\pgfpathmoveto{\pgfqpoint{3.132355in}{4.679521in}}%
\pgfpathlineto{\pgfqpoint{3.132355in}{4.762854in}}%
\pgfusepath{stroke,fill}%
\end{pgfscope}%
\begin{pgfscope}%
\pgfpathrectangle{\pgfqpoint{0.456635in}{4.237239in}}{\pgfqpoint{4.376471in}{0.972632in}}%
\pgfusepath{clip}%
\pgfsetbuttcap%
\pgfsetroundjoin%
\definecolor{currentfill}{rgb}{1.000000,0.000000,0.000000}%
\pgfsetfillcolor{currentfill}%
\pgfsetlinewidth{2.007500pt}%
\definecolor{currentstroke}{rgb}{1.000000,0.000000,0.000000}%
\pgfsetstrokecolor{currentstroke}%
\pgfsetdash{}{0pt}%
\pgfpathmoveto{\pgfqpoint{3.364411in}{4.915156in}}%
\pgfpathlineto{\pgfqpoint{3.447744in}{4.915156in}}%
\pgfpathmoveto{\pgfqpoint{3.406077in}{4.873490in}}%
\pgfpathlineto{\pgfqpoint{3.406077in}{4.956823in}}%
\pgfusepath{stroke,fill}%
\end{pgfscope}%
\begin{pgfscope}%
\pgfpathrectangle{\pgfqpoint{0.456635in}{4.237239in}}{\pgfqpoint{4.376471in}{0.972632in}}%
\pgfusepath{clip}%
\pgfsetbuttcap%
\pgfsetroundjoin%
\definecolor{currentfill}{rgb}{1.000000,0.000000,0.000000}%
\pgfsetfillcolor{currentfill}%
\pgfsetlinewidth{2.007500pt}%
\definecolor{currentstroke}{rgb}{1.000000,0.000000,0.000000}%
\pgfsetstrokecolor{currentstroke}%
\pgfsetdash{}{0pt}%
\pgfpathmoveto{\pgfqpoint{1.452894in}{4.959589in}}%
\pgfpathlineto{\pgfqpoint{1.536227in}{4.959589in}}%
\pgfpathmoveto{\pgfqpoint{1.494560in}{4.917922in}}%
\pgfpathlineto{\pgfqpoint{1.494560in}{5.001256in}}%
\pgfusepath{stroke,fill}%
\end{pgfscope}%
\begin{pgfscope}%
\pgfpathrectangle{\pgfqpoint{0.456635in}{4.237239in}}{\pgfqpoint{4.376471in}{0.972632in}}%
\pgfusepath{clip}%
\pgfsetbuttcap%
\pgfsetroundjoin%
\definecolor{currentfill}{rgb}{0.000000,0.000000,0.000000}%
\pgfsetfillcolor{currentfill}%
\pgfsetlinewidth{1.003750pt}%
\definecolor{currentstroke}{rgb}{0.000000,0.000000,0.000000}%
\pgfsetstrokecolor{currentstroke}%
\pgfsetdash{}{0pt}%
\pgfsys@defobject{currentmarker}{\pgfqpoint{-0.020833in}{-0.020833in}}{\pgfqpoint{0.020833in}{0.020833in}}{%
\pgfpathmoveto{\pgfqpoint{0.000000in}{-0.020833in}}%
\pgfpathcurveto{\pgfqpoint{0.005525in}{-0.020833in}}{\pgfqpoint{0.010825in}{-0.018638in}}{\pgfqpoint{0.014731in}{-0.014731in}}%
\pgfpathcurveto{\pgfqpoint{0.018638in}{-0.010825in}}{\pgfqpoint{0.020833in}{-0.005525in}}{\pgfqpoint{0.020833in}{0.000000in}}%
\pgfpathcurveto{\pgfqpoint{0.020833in}{0.005525in}}{\pgfqpoint{0.018638in}{0.010825in}}{\pgfqpoint{0.014731in}{0.014731in}}%
\pgfpathcurveto{\pgfqpoint{0.010825in}{0.018638in}}{\pgfqpoint{0.005525in}{0.020833in}}{\pgfqpoint{0.000000in}{0.020833in}}%
\pgfpathcurveto{\pgfqpoint{-0.005525in}{0.020833in}}{\pgfqpoint{-0.010825in}{0.018638in}}{\pgfqpoint{-0.014731in}{0.014731in}}%
\pgfpathcurveto{\pgfqpoint{-0.018638in}{0.010825in}}{\pgfqpoint{-0.020833in}{0.005525in}}{\pgfqpoint{-0.020833in}{0.000000in}}%
\pgfpathcurveto{\pgfqpoint{-0.020833in}{-0.005525in}}{\pgfqpoint{-0.018638in}{-0.010825in}}{\pgfqpoint{-0.014731in}{-0.014731in}}%
\pgfpathcurveto{\pgfqpoint{-0.010825in}{-0.018638in}}{\pgfqpoint{-0.005525in}{-0.020833in}}{\pgfqpoint{0.000000in}{-0.020833in}}%
\pgfpathclose%
\pgfusepath{stroke,fill}%
}%
\begin{pgfscope}%
\pgfsys@transformshift{1.331929in}{5.032449in}%
\pgfsys@useobject{currentmarker}{}%
\end{pgfscope}%
\begin{pgfscope}%
\pgfsys@transformshift{1.349523in}{5.060426in}%
\pgfsys@useobject{currentmarker}{}%
\end{pgfscope}%
\begin{pgfscope}%
\pgfsys@transformshift{1.367117in}{5.098743in}%
\pgfsys@useobject{currentmarker}{}%
\end{pgfscope}%
\begin{pgfscope}%
\pgfsys@transformshift{1.384711in}{5.136781in}%
\pgfsys@useobject{currentmarker}{}%
\end{pgfscope}%
\begin{pgfscope}%
\pgfsys@transformshift{1.402305in}{4.961685in}%
\pgfsys@useobject{currentmarker}{}%
\end{pgfscope}%
\begin{pgfscope}%
\pgfsys@transformshift{1.419899in}{4.963442in}%
\pgfsys@useobject{currentmarker}{}%
\end{pgfscope}%
\begin{pgfscope}%
\pgfsys@transformshift{1.437493in}{4.842116in}%
\pgfsys@useobject{currentmarker}{}%
\end{pgfscope}%
\begin{pgfscope}%
\pgfsys@transformshift{1.455086in}{4.805102in}%
\pgfsys@useobject{currentmarker}{}%
\end{pgfscope}%
\begin{pgfscope}%
\pgfsys@transformshift{1.472680in}{4.981054in}%
\pgfsys@useobject{currentmarker}{}%
\end{pgfscope}%
\begin{pgfscope}%
\pgfsys@transformshift{1.490274in}{5.009122in}%
\pgfsys@useobject{currentmarker}{}%
\end{pgfscope}%
\begin{pgfscope}%
\pgfsys@transformshift{1.507868in}{4.838142in}%
\pgfsys@useobject{currentmarker}{}%
\end{pgfscope}%
\begin{pgfscope}%
\pgfsys@transformshift{1.525462in}{4.921769in}%
\pgfsys@useobject{currentmarker}{}%
\end{pgfscope}%
\begin{pgfscope}%
\pgfsys@transformshift{1.543056in}{4.832509in}%
\pgfsys@useobject{currentmarker}{}%
\end{pgfscope}%
\begin{pgfscope}%
\pgfsys@transformshift{1.560649in}{4.707571in}%
\pgfsys@useobject{currentmarker}{}%
\end{pgfscope}%
\begin{pgfscope}%
\pgfsys@transformshift{1.578243in}{4.788057in}%
\pgfsys@useobject{currentmarker}{}%
\end{pgfscope}%
\begin{pgfscope}%
\pgfsys@transformshift{1.595837in}{4.887373in}%
\pgfsys@useobject{currentmarker}{}%
\end{pgfscope}%
\begin{pgfscope}%
\pgfsys@transformshift{1.613431in}{4.709743in}%
\pgfsys@useobject{currentmarker}{}%
\end{pgfscope}%
\begin{pgfscope}%
\pgfsys@transformshift{1.631025in}{4.855561in}%
\pgfsys@useobject{currentmarker}{}%
\end{pgfscope}%
\begin{pgfscope}%
\pgfsys@transformshift{1.648619in}{4.417217in}%
\pgfsys@useobject{currentmarker}{}%
\end{pgfscope}%
\begin{pgfscope}%
\pgfsys@transformshift{1.666213in}{4.753521in}%
\pgfsys@useobject{currentmarker}{}%
\end{pgfscope}%
\begin{pgfscope}%
\pgfsys@transformshift{1.683806in}{4.668751in}%
\pgfsys@useobject{currentmarker}{}%
\end{pgfscope}%
\begin{pgfscope}%
\pgfsys@transformshift{1.701400in}{4.621427in}%
\pgfsys@useobject{currentmarker}{}%
\end{pgfscope}%
\begin{pgfscope}%
\pgfsys@transformshift{1.718994in}{4.654933in}%
\pgfsys@useobject{currentmarker}{}%
\end{pgfscope}%
\begin{pgfscope}%
\pgfsys@transformshift{1.736588in}{4.440302in}%
\pgfsys@useobject{currentmarker}{}%
\end{pgfscope}%
\begin{pgfscope}%
\pgfsys@transformshift{1.754182in}{4.617554in}%
\pgfsys@useobject{currentmarker}{}%
\end{pgfscope}%
\begin{pgfscope}%
\pgfsys@transformshift{1.771776in}{4.676183in}%
\pgfsys@useobject{currentmarker}{}%
\end{pgfscope}%
\begin{pgfscope}%
\pgfsys@transformshift{1.789370in}{4.791919in}%
\pgfsys@useobject{currentmarker}{}%
\end{pgfscope}%
\begin{pgfscope}%
\pgfsys@transformshift{1.806963in}{4.593766in}%
\pgfsys@useobject{currentmarker}{}%
\end{pgfscope}%
\begin{pgfscope}%
\pgfsys@transformshift{1.824557in}{4.570259in}%
\pgfsys@useobject{currentmarker}{}%
\end{pgfscope}%
\begin{pgfscope}%
\pgfsys@transformshift{1.842151in}{4.608920in}%
\pgfsys@useobject{currentmarker}{}%
\end{pgfscope}%
\begin{pgfscope}%
\pgfsys@transformshift{1.859745in}{4.761637in}%
\pgfsys@useobject{currentmarker}{}%
\end{pgfscope}%
\begin{pgfscope}%
\pgfsys@transformshift{1.877339in}{4.712738in}%
\pgfsys@useobject{currentmarker}{}%
\end{pgfscope}%
\begin{pgfscope}%
\pgfsys@transformshift{1.894933in}{4.637534in}%
\pgfsys@useobject{currentmarker}{}%
\end{pgfscope}%
\begin{pgfscope}%
\pgfsys@transformshift{1.912526in}{4.756049in}%
\pgfsys@useobject{currentmarker}{}%
\end{pgfscope}%
\begin{pgfscope}%
\pgfsys@transformshift{1.930120in}{4.727602in}%
\pgfsys@useobject{currentmarker}{}%
\end{pgfscope}%
\begin{pgfscope}%
\pgfsys@transformshift{1.947714in}{4.830310in}%
\pgfsys@useobject{currentmarker}{}%
\end{pgfscope}%
\begin{pgfscope}%
\pgfsys@transformshift{1.965308in}{4.675933in}%
\pgfsys@useobject{currentmarker}{}%
\end{pgfscope}%
\begin{pgfscope}%
\pgfsys@transformshift{1.982902in}{4.729039in}%
\pgfsys@useobject{currentmarker}{}%
\end{pgfscope}%
\begin{pgfscope}%
\pgfsys@transformshift{2.000496in}{4.737763in}%
\pgfsys@useobject{currentmarker}{}%
\end{pgfscope}%
\begin{pgfscope}%
\pgfsys@transformshift{2.018090in}{4.644339in}%
\pgfsys@useobject{currentmarker}{}%
\end{pgfscope}%
\begin{pgfscope}%
\pgfsys@transformshift{2.035683in}{4.837414in}%
\pgfsys@useobject{currentmarker}{}%
\end{pgfscope}%
\begin{pgfscope}%
\pgfsys@transformshift{2.053277in}{4.848128in}%
\pgfsys@useobject{currentmarker}{}%
\end{pgfscope}%
\begin{pgfscope}%
\pgfsys@transformshift{2.070871in}{4.835743in}%
\pgfsys@useobject{currentmarker}{}%
\end{pgfscope}%
\begin{pgfscope}%
\pgfsys@transformshift{2.088465in}{4.824096in}%
\pgfsys@useobject{currentmarker}{}%
\end{pgfscope}%
\begin{pgfscope}%
\pgfsys@transformshift{2.106059in}{4.716023in}%
\pgfsys@useobject{currentmarker}{}%
\end{pgfscope}%
\begin{pgfscope}%
\pgfsys@transformshift{2.123653in}{4.827119in}%
\pgfsys@useobject{currentmarker}{}%
\end{pgfscope}%
\begin{pgfscope}%
\pgfsys@transformshift{2.141247in}{4.843936in}%
\pgfsys@useobject{currentmarker}{}%
\end{pgfscope}%
\begin{pgfscope}%
\pgfsys@transformshift{2.158840in}{4.804768in}%
\pgfsys@useobject{currentmarker}{}%
\end{pgfscope}%
\begin{pgfscope}%
\pgfsys@transformshift{2.176434in}{4.875457in}%
\pgfsys@useobject{currentmarker}{}%
\end{pgfscope}%
\begin{pgfscope}%
\pgfsys@transformshift{2.194028in}{4.936734in}%
\pgfsys@useobject{currentmarker}{}%
\end{pgfscope}%
\begin{pgfscope}%
\pgfsys@transformshift{2.211622in}{5.089067in}%
\pgfsys@useobject{currentmarker}{}%
\end{pgfscope}%
\begin{pgfscope}%
\pgfsys@transformshift{2.229216in}{4.915931in}%
\pgfsys@useobject{currentmarker}{}%
\end{pgfscope}%
\begin{pgfscope}%
\pgfsys@transformshift{2.246810in}{4.922677in}%
\pgfsys@useobject{currentmarker}{}%
\end{pgfscope}%
\begin{pgfscope}%
\pgfsys@transformshift{2.264404in}{4.885421in}%
\pgfsys@useobject{currentmarker}{}%
\end{pgfscope}%
\begin{pgfscope}%
\pgfsys@transformshift{2.281997in}{4.692981in}%
\pgfsys@useobject{currentmarker}{}%
\end{pgfscope}%
\begin{pgfscope}%
\pgfsys@transformshift{2.299591in}{4.877173in}%
\pgfsys@useobject{currentmarker}{}%
\end{pgfscope}%
\begin{pgfscope}%
\pgfsys@transformshift{2.317185in}{4.876532in}%
\pgfsys@useobject{currentmarker}{}%
\end{pgfscope}%
\begin{pgfscope}%
\pgfsys@transformshift{2.334779in}{5.108721in}%
\pgfsys@useobject{currentmarker}{}%
\end{pgfscope}%
\begin{pgfscope}%
\pgfsys@transformshift{2.352373in}{4.826625in}%
\pgfsys@useobject{currentmarker}{}%
\end{pgfscope}%
\begin{pgfscope}%
\pgfsys@transformshift{2.369967in}{4.861953in}%
\pgfsys@useobject{currentmarker}{}%
\end{pgfscope}%
\begin{pgfscope}%
\pgfsys@transformshift{2.387560in}{4.811617in}%
\pgfsys@useobject{currentmarker}{}%
\end{pgfscope}%
\begin{pgfscope}%
\pgfsys@transformshift{2.405154in}{4.679035in}%
\pgfsys@useobject{currentmarker}{}%
\end{pgfscope}%
\begin{pgfscope}%
\pgfsys@transformshift{2.422748in}{4.894255in}%
\pgfsys@useobject{currentmarker}{}%
\end{pgfscope}%
\begin{pgfscope}%
\pgfsys@transformshift{2.440342in}{4.834562in}%
\pgfsys@useobject{currentmarker}{}%
\end{pgfscope}%
\begin{pgfscope}%
\pgfsys@transformshift{2.457936in}{4.817487in}%
\pgfsys@useobject{currentmarker}{}%
\end{pgfscope}%
\begin{pgfscope}%
\pgfsys@transformshift{2.475530in}{4.623408in}%
\pgfsys@useobject{currentmarker}{}%
\end{pgfscope}%
\begin{pgfscope}%
\pgfsys@transformshift{2.493124in}{4.835296in}%
\pgfsys@useobject{currentmarker}{}%
\end{pgfscope}%
\begin{pgfscope}%
\pgfsys@transformshift{2.510717in}{4.528394in}%
\pgfsys@useobject{currentmarker}{}%
\end{pgfscope}%
\begin{pgfscope}%
\pgfsys@transformshift{2.528311in}{4.706961in}%
\pgfsys@useobject{currentmarker}{}%
\end{pgfscope}%
\begin{pgfscope}%
\pgfsys@transformshift{2.545905in}{4.846544in}%
\pgfsys@useobject{currentmarker}{}%
\end{pgfscope}%
\begin{pgfscope}%
\pgfsys@transformshift{2.563499in}{4.501609in}%
\pgfsys@useobject{currentmarker}{}%
\end{pgfscope}%
\begin{pgfscope}%
\pgfsys@transformshift{2.581093in}{4.522383in}%
\pgfsys@useobject{currentmarker}{}%
\end{pgfscope}%
\begin{pgfscope}%
\pgfsys@transformshift{2.598687in}{4.568288in}%
\pgfsys@useobject{currentmarker}{}%
\end{pgfscope}%
\begin{pgfscope}%
\pgfsys@transformshift{2.616281in}{4.486451in}%
\pgfsys@useobject{currentmarker}{}%
\end{pgfscope}%
\begin{pgfscope}%
\pgfsys@transformshift{2.633874in}{4.360647in}%
\pgfsys@useobject{currentmarker}{}%
\end{pgfscope}%
\begin{pgfscope}%
\pgfsys@transformshift{2.651468in}{4.506193in}%
\pgfsys@useobject{currentmarker}{}%
\end{pgfscope}%
\begin{pgfscope}%
\pgfsys@transformshift{2.669062in}{4.374478in}%
\pgfsys@useobject{currentmarker}{}%
\end{pgfscope}%
\begin{pgfscope}%
\pgfsys@transformshift{2.686656in}{4.514468in}%
\pgfsys@useobject{currentmarker}{}%
\end{pgfscope}%
\begin{pgfscope}%
\pgfsys@transformshift{2.704250in}{4.359371in}%
\pgfsys@useobject{currentmarker}{}%
\end{pgfscope}%
\begin{pgfscope}%
\pgfsys@transformshift{2.721844in}{4.597375in}%
\pgfsys@useobject{currentmarker}{}%
\end{pgfscope}%
\begin{pgfscope}%
\pgfsys@transformshift{2.739438in}{4.350692in}%
\pgfsys@useobject{currentmarker}{}%
\end{pgfscope}%
\begin{pgfscope}%
\pgfsys@transformshift{2.757031in}{4.389098in}%
\pgfsys@useobject{currentmarker}{}%
\end{pgfscope}%
\begin{pgfscope}%
\pgfsys@transformshift{2.774625in}{4.497873in}%
\pgfsys@useobject{currentmarker}{}%
\end{pgfscope}%
\begin{pgfscope}%
\pgfsys@transformshift{2.792219in}{4.286558in}%
\pgfsys@useobject{currentmarker}{}%
\end{pgfscope}%
\begin{pgfscope}%
\pgfsys@transformshift{2.809813in}{4.432241in}%
\pgfsys@useobject{currentmarker}{}%
\end{pgfscope}%
\begin{pgfscope}%
\pgfsys@transformshift{2.827407in}{4.541689in}%
\pgfsys@useobject{currentmarker}{}%
\end{pgfscope}%
\begin{pgfscope}%
\pgfsys@transformshift{2.845001in}{4.248564in}%
\pgfsys@useobject{currentmarker}{}%
\end{pgfscope}%
\begin{pgfscope}%
\pgfsys@transformshift{2.862594in}{4.434386in}%
\pgfsys@useobject{currentmarker}{}%
\end{pgfscope}%
\begin{pgfscope}%
\pgfsys@transformshift{2.880188in}{4.448289in}%
\pgfsys@useobject{currentmarker}{}%
\end{pgfscope}%
\begin{pgfscope}%
\pgfsys@transformshift{2.897782in}{4.509403in}%
\pgfsys@useobject{currentmarker}{}%
\end{pgfscope}%
\begin{pgfscope}%
\pgfsys@transformshift{2.915376in}{4.314963in}%
\pgfsys@useobject{currentmarker}{}%
\end{pgfscope}%
\begin{pgfscope}%
\pgfsys@transformshift{2.932970in}{4.318347in}%
\pgfsys@useobject{currentmarker}{}%
\end{pgfscope}%
\begin{pgfscope}%
\pgfsys@transformshift{2.950564in}{4.518480in}%
\pgfsys@useobject{currentmarker}{}%
\end{pgfscope}%
\begin{pgfscope}%
\pgfsys@transformshift{2.968158in}{4.510638in}%
\pgfsys@useobject{currentmarker}{}%
\end{pgfscope}%
\begin{pgfscope}%
\pgfsys@transformshift{2.985751in}{4.522203in}%
\pgfsys@useobject{currentmarker}{}%
\end{pgfscope}%
\begin{pgfscope}%
\pgfsys@transformshift{3.003345in}{4.549355in}%
\pgfsys@useobject{currentmarker}{}%
\end{pgfscope}%
\begin{pgfscope}%
\pgfsys@transformshift{3.020939in}{4.463763in}%
\pgfsys@useobject{currentmarker}{}%
\end{pgfscope}%
\begin{pgfscope}%
\pgfsys@transformshift{3.038533in}{4.575384in}%
\pgfsys@useobject{currentmarker}{}%
\end{pgfscope}%
\begin{pgfscope}%
\pgfsys@transformshift{3.056127in}{4.601329in}%
\pgfsys@useobject{currentmarker}{}%
\end{pgfscope}%
\begin{pgfscope}%
\pgfsys@transformshift{3.073721in}{4.519432in}%
\pgfsys@useobject{currentmarker}{}%
\end{pgfscope}%
\begin{pgfscope}%
\pgfsys@transformshift{3.091315in}{4.801196in}%
\pgfsys@useobject{currentmarker}{}%
\end{pgfscope}%
\begin{pgfscope}%
\pgfsys@transformshift{3.108908in}{4.680506in}%
\pgfsys@useobject{currentmarker}{}%
\end{pgfscope}%
\begin{pgfscope}%
\pgfsys@transformshift{3.126502in}{4.531912in}%
\pgfsys@useobject{currentmarker}{}%
\end{pgfscope}%
\begin{pgfscope}%
\pgfsys@transformshift{3.144096in}{4.738813in}%
\pgfsys@useobject{currentmarker}{}%
\end{pgfscope}%
\begin{pgfscope}%
\pgfsys@transformshift{3.161690in}{4.592604in}%
\pgfsys@useobject{currentmarker}{}%
\end{pgfscope}%
\begin{pgfscope}%
\pgfsys@transformshift{3.179284in}{4.789348in}%
\pgfsys@useobject{currentmarker}{}%
\end{pgfscope}%
\begin{pgfscope}%
\pgfsys@transformshift{3.196878in}{4.844245in}%
\pgfsys@useobject{currentmarker}{}%
\end{pgfscope}%
\begin{pgfscope}%
\pgfsys@transformshift{3.214472in}{4.659809in}%
\pgfsys@useobject{currentmarker}{}%
\end{pgfscope}%
\begin{pgfscope}%
\pgfsys@transformshift{3.232065in}{4.855341in}%
\pgfsys@useobject{currentmarker}{}%
\end{pgfscope}%
\begin{pgfscope}%
\pgfsys@transformshift{3.249659in}{4.812876in}%
\pgfsys@useobject{currentmarker}{}%
\end{pgfscope}%
\begin{pgfscope}%
\pgfsys@transformshift{3.267253in}{4.866073in}%
\pgfsys@useobject{currentmarker}{}%
\end{pgfscope}%
\begin{pgfscope}%
\pgfsys@transformshift{3.284847in}{4.984994in}%
\pgfsys@useobject{currentmarker}{}%
\end{pgfscope}%
\begin{pgfscope}%
\pgfsys@transformshift{3.302441in}{4.776202in}%
\pgfsys@useobject{currentmarker}{}%
\end{pgfscope}%
\begin{pgfscope}%
\pgfsys@transformshift{3.320035in}{4.731079in}%
\pgfsys@useobject{currentmarker}{}%
\end{pgfscope}%
\begin{pgfscope}%
\pgfsys@transformshift{3.337628in}{4.721788in}%
\pgfsys@useobject{currentmarker}{}%
\end{pgfscope}%
\begin{pgfscope}%
\pgfsys@transformshift{3.355222in}{4.731770in}%
\pgfsys@useobject{currentmarker}{}%
\end{pgfscope}%
\begin{pgfscope}%
\pgfsys@transformshift{3.372816in}{4.807165in}%
\pgfsys@useobject{currentmarker}{}%
\end{pgfscope}%
\begin{pgfscope}%
\pgfsys@transformshift{3.390410in}{4.848137in}%
\pgfsys@useobject{currentmarker}{}%
\end{pgfscope}%
\begin{pgfscope}%
\pgfsys@transformshift{3.408004in}{4.838276in}%
\pgfsys@useobject{currentmarker}{}%
\end{pgfscope}%
\begin{pgfscope}%
\pgfsys@transformshift{3.425598in}{4.888840in}%
\pgfsys@useobject{currentmarker}{}%
\end{pgfscope}%
\begin{pgfscope}%
\pgfsys@transformshift{3.443192in}{4.799331in}%
\pgfsys@useobject{currentmarker}{}%
\end{pgfscope}%
\begin{pgfscope}%
\pgfsys@transformshift{3.460785in}{4.936537in}%
\pgfsys@useobject{currentmarker}{}%
\end{pgfscope}%
\begin{pgfscope}%
\pgfsys@transformshift{3.478379in}{4.752098in}%
\pgfsys@useobject{currentmarker}{}%
\end{pgfscope}%
\begin{pgfscope}%
\pgfsys@transformshift{3.495973in}{5.042653in}%
\pgfsys@useobject{currentmarker}{}%
\end{pgfscope}%
\begin{pgfscope}%
\pgfsys@transformshift{3.513567in}{4.817237in}%
\pgfsys@useobject{currentmarker}{}%
\end{pgfscope}%
\begin{pgfscope}%
\pgfsys@transformshift{3.531161in}{4.652590in}%
\pgfsys@useobject{currentmarker}{}%
\end{pgfscope}%
\begin{pgfscope}%
\pgfsys@transformshift{3.548755in}{4.615485in}%
\pgfsys@useobject{currentmarker}{}%
\end{pgfscope}%
\begin{pgfscope}%
\pgfsys@transformshift{3.566349in}{4.756555in}%
\pgfsys@useobject{currentmarker}{}%
\end{pgfscope}%
\begin{pgfscope}%
\pgfsys@transformshift{3.583942in}{4.668051in}%
\pgfsys@useobject{currentmarker}{}%
\end{pgfscope}%
\begin{pgfscope}%
\pgfsys@transformshift{3.601536in}{4.745571in}%
\pgfsys@useobject{currentmarker}{}%
\end{pgfscope}%
\begin{pgfscope}%
\pgfsys@transformshift{3.619130in}{4.703440in}%
\pgfsys@useobject{currentmarker}{}%
\end{pgfscope}%
\begin{pgfscope}%
\pgfsys@transformshift{3.636724in}{4.630304in}%
\pgfsys@useobject{currentmarker}{}%
\end{pgfscope}%
\begin{pgfscope}%
\pgfsys@transformshift{3.654318in}{4.534209in}%
\pgfsys@useobject{currentmarker}{}%
\end{pgfscope}%
\begin{pgfscope}%
\pgfsys@transformshift{3.671912in}{4.449180in}%
\pgfsys@useobject{currentmarker}{}%
\end{pgfscope}%
\begin{pgfscope}%
\pgfsys@transformshift{3.689506in}{4.540614in}%
\pgfsys@useobject{currentmarker}{}%
\end{pgfscope}%
\begin{pgfscope}%
\pgfsys@transformshift{3.707099in}{4.656551in}%
\pgfsys@useobject{currentmarker}{}%
\end{pgfscope}%
\begin{pgfscope}%
\pgfsys@transformshift{3.724693in}{4.576334in}%
\pgfsys@useobject{currentmarker}{}%
\end{pgfscope}%
\begin{pgfscope}%
\pgfsys@transformshift{3.742287in}{4.414398in}%
\pgfsys@useobject{currentmarker}{}%
\end{pgfscope}%
\begin{pgfscope}%
\pgfsys@transformshift{3.759881in}{4.545407in}%
\pgfsys@useobject{currentmarker}{}%
\end{pgfscope}%
\begin{pgfscope}%
\pgfsys@transformshift{3.777475in}{4.555593in}%
\pgfsys@useobject{currentmarker}{}%
\end{pgfscope}%
\begin{pgfscope}%
\pgfsys@transformshift{3.795069in}{4.417287in}%
\pgfsys@useobject{currentmarker}{}%
\end{pgfscope}%
\begin{pgfscope}%
\pgfsys@transformshift{3.812662in}{4.514411in}%
\pgfsys@useobject{currentmarker}{}%
\end{pgfscope}%
\begin{pgfscope}%
\pgfsys@transformshift{3.830256in}{4.498569in}%
\pgfsys@useobject{currentmarker}{}%
\end{pgfscope}%
\begin{pgfscope}%
\pgfsys@transformshift{3.847850in}{4.372637in}%
\pgfsys@useobject{currentmarker}{}%
\end{pgfscope}%
\begin{pgfscope}%
\pgfsys@transformshift{3.865444in}{4.522463in}%
\pgfsys@useobject{currentmarker}{}%
\end{pgfscope}%
\begin{pgfscope}%
\pgfsys@transformshift{3.883038in}{4.542874in}%
\pgfsys@useobject{currentmarker}{}%
\end{pgfscope}%
\begin{pgfscope}%
\pgfsys@transformshift{3.900632in}{4.597739in}%
\pgfsys@useobject{currentmarker}{}%
\end{pgfscope}%
\begin{pgfscope}%
\pgfsys@transformshift{3.918226in}{4.598852in}%
\pgfsys@useobject{currentmarker}{}%
\end{pgfscope}%
\begin{pgfscope}%
\pgfsys@transformshift{3.935819in}{4.358705in}%
\pgfsys@useobject{currentmarker}{}%
\end{pgfscope}%
\begin{pgfscope}%
\pgfsys@transformshift{3.953413in}{4.411568in}%
\pgfsys@useobject{currentmarker}{}%
\end{pgfscope}%
\begin{pgfscope}%
\pgfsys@transformshift{3.971007in}{4.569122in}%
\pgfsys@useobject{currentmarker}{}%
\end{pgfscope}%
\begin{pgfscope}%
\pgfsys@transformshift{3.988601in}{4.581346in}%
\pgfsys@useobject{currentmarker}{}%
\end{pgfscope}%
\begin{pgfscope}%
\pgfsys@transformshift{4.006195in}{4.595735in}%
\pgfsys@useobject{currentmarker}{}%
\end{pgfscope}%
\begin{pgfscope}%
\pgfsys@transformshift{4.023789in}{4.949966in}%
\pgfsys@useobject{currentmarker}{}%
\end{pgfscope}%
\begin{pgfscope}%
\pgfsys@transformshift{4.041383in}{4.635227in}%
\pgfsys@useobject{currentmarker}{}%
\end{pgfscope}%
\begin{pgfscope}%
\pgfsys@transformshift{4.058976in}{4.711757in}%
\pgfsys@useobject{currentmarker}{}%
\end{pgfscope}%
\begin{pgfscope}%
\pgfsys@transformshift{4.076570in}{4.714090in}%
\pgfsys@useobject{currentmarker}{}%
\end{pgfscope}%
\begin{pgfscope}%
\pgfsys@transformshift{4.094164in}{4.705403in}%
\pgfsys@useobject{currentmarker}{}%
\end{pgfscope}%
\begin{pgfscope}%
\pgfsys@transformshift{4.111758in}{4.630507in}%
\pgfsys@useobject{currentmarker}{}%
\end{pgfscope}%
\begin{pgfscope}%
\pgfsys@transformshift{4.129352in}{4.763272in}%
\pgfsys@useobject{currentmarker}{}%
\end{pgfscope}%
\begin{pgfscope}%
\pgfsys@transformshift{4.146946in}{4.632698in}%
\pgfsys@useobject{currentmarker}{}%
\end{pgfscope}%
\begin{pgfscope}%
\pgfsys@transformshift{4.164540in}{4.712118in}%
\pgfsys@useobject{currentmarker}{}%
\end{pgfscope}%
\begin{pgfscope}%
\pgfsys@transformshift{4.182133in}{4.712341in}%
\pgfsys@useobject{currentmarker}{}%
\end{pgfscope}%
\begin{pgfscope}%
\pgfsys@transformshift{4.199727in}{4.795302in}%
\pgfsys@useobject{currentmarker}{}%
\end{pgfscope}%
\begin{pgfscope}%
\pgfsys@transformshift{4.217321in}{5.046888in}%
\pgfsys@useobject{currentmarker}{}%
\end{pgfscope}%
\begin{pgfscope}%
\pgfsys@transformshift{4.234915in}{4.648241in}%
\pgfsys@useobject{currentmarker}{}%
\end{pgfscope}%
\begin{pgfscope}%
\pgfsys@transformshift{4.252509in}{4.931479in}%
\pgfsys@useobject{currentmarker}{}%
\end{pgfscope}%
\begin{pgfscope}%
\pgfsys@transformshift{4.270103in}{4.722368in}%
\pgfsys@useobject{currentmarker}{}%
\end{pgfscope}%
\begin{pgfscope}%
\pgfsys@transformshift{4.287696in}{4.860858in}%
\pgfsys@useobject{currentmarker}{}%
\end{pgfscope}%
\begin{pgfscope}%
\pgfsys@transformshift{4.305290in}{5.040836in}%
\pgfsys@useobject{currentmarker}{}%
\end{pgfscope}%
\begin{pgfscope}%
\pgfsys@transformshift{4.322884in}{4.957619in}%
\pgfsys@useobject{currentmarker}{}%
\end{pgfscope}%
\begin{pgfscope}%
\pgfsys@transformshift{4.340478in}{4.861123in}%
\pgfsys@useobject{currentmarker}{}%
\end{pgfscope}%
\begin{pgfscope}%
\pgfsys@transformshift{4.358072in}{4.915530in}%
\pgfsys@useobject{currentmarker}{}%
\end{pgfscope}%
\begin{pgfscope}%
\pgfsys@transformshift{4.375666in}{5.072899in}%
\pgfsys@useobject{currentmarker}{}%
\end{pgfscope}%
\begin{pgfscope}%
\pgfsys@transformshift{4.393260in}{4.944346in}%
\pgfsys@useobject{currentmarker}{}%
\end{pgfscope}%
\begin{pgfscope}%
\pgfsys@transformshift{4.410853in}{5.052747in}%
\pgfsys@useobject{currentmarker}{}%
\end{pgfscope}%
\begin{pgfscope}%
\pgfsys@transformshift{4.428447in}{5.046014in}%
\pgfsys@useobject{currentmarker}{}%
\end{pgfscope}%
\begin{pgfscope}%
\pgfsys@transformshift{4.446041in}{4.984025in}%
\pgfsys@useobject{currentmarker}{}%
\end{pgfscope}%
\begin{pgfscope}%
\pgfsys@transformshift{4.463635in}{5.273945in}%
\pgfsys@useobject{currentmarker}{}%
\end{pgfscope}%
\begin{pgfscope}%
\pgfsys@transformshift{4.481229in}{5.125680in}%
\pgfsys@useobject{currentmarker}{}%
\end{pgfscope}%
\begin{pgfscope}%
\pgfsys@transformshift{4.498823in}{4.859056in}%
\pgfsys@useobject{currentmarker}{}%
\end{pgfscope}%
\begin{pgfscope}%
\pgfsys@transformshift{4.516417in}{5.084000in}%
\pgfsys@useobject{currentmarker}{}%
\end{pgfscope}%
\begin{pgfscope}%
\pgfsys@transformshift{4.534010in}{4.997086in}%
\pgfsys@useobject{currentmarker}{}%
\end{pgfscope}%
\begin{pgfscope}%
\pgfsys@transformshift{4.551604in}{5.147756in}%
\pgfsys@useobject{currentmarker}{}%
\end{pgfscope}%
\begin{pgfscope}%
\pgfsys@transformshift{4.569198in}{4.976678in}%
\pgfsys@useobject{currentmarker}{}%
\end{pgfscope}%
\begin{pgfscope}%
\pgfsys@transformshift{4.586792in}{5.039369in}%
\pgfsys@useobject{currentmarker}{}%
\end{pgfscope}%
\begin{pgfscope}%
\pgfsys@transformshift{4.604386in}{5.094749in}%
\pgfsys@useobject{currentmarker}{}%
\end{pgfscope}%
\begin{pgfscope}%
\pgfsys@transformshift{4.621980in}{5.122608in}%
\pgfsys@useobject{currentmarker}{}%
\end{pgfscope}%
\begin{pgfscope}%
\pgfsys@transformshift{4.639573in}{4.903467in}%
\pgfsys@useobject{currentmarker}{}%
\end{pgfscope}%
\begin{pgfscope}%
\pgfsys@transformshift{4.657167in}{4.980417in}%
\pgfsys@useobject{currentmarker}{}%
\end{pgfscope}%
\begin{pgfscope}%
\pgfsys@transformshift{4.674761in}{4.954648in}%
\pgfsys@useobject{currentmarker}{}%
\end{pgfscope}%
\begin{pgfscope}%
\pgfsys@transformshift{4.692355in}{4.924456in}%
\pgfsys@useobject{currentmarker}{}%
\end{pgfscope}%
\begin{pgfscope}%
\pgfsys@transformshift{4.709949in}{5.157016in}%
\pgfsys@useobject{currentmarker}{}%
\end{pgfscope}%
\begin{pgfscope}%
\pgfsys@transformshift{4.727543in}{5.006498in}%
\pgfsys@useobject{currentmarker}{}%
\end{pgfscope}%
\begin{pgfscope}%
\pgfsys@transformshift{4.745137in}{4.825064in}%
\pgfsys@useobject{currentmarker}{}%
\end{pgfscope}%
\begin{pgfscope}%
\pgfsys@transformshift{4.762730in}{5.033381in}%
\pgfsys@useobject{currentmarker}{}%
\end{pgfscope}%
\begin{pgfscope}%
\pgfsys@transformshift{4.780324in}{5.143404in}%
\pgfsys@useobject{currentmarker}{}%
\end{pgfscope}%
\begin{pgfscope}%
\pgfsys@transformshift{4.797918in}{5.021645in}%
\pgfsys@useobject{currentmarker}{}%
\end{pgfscope}%
\begin{pgfscope}%
\pgfsys@transformshift{4.815512in}{4.752577in}%
\pgfsys@useobject{currentmarker}{}%
\end{pgfscope}%
\begin{pgfscope}%
\pgfsys@transformshift{4.833106in}{4.847942in}%
\pgfsys@useobject{currentmarker}{}%
\end{pgfscope}%
\end{pgfscope}%
\begin{pgfscope}%
\pgfsetbuttcap%
\pgfsetroundjoin%
\definecolor{currentfill}{rgb}{0.000000,0.000000,0.000000}%
\pgfsetfillcolor{currentfill}%
\pgfsetlinewidth{0.803000pt}%
\definecolor{currentstroke}{rgb}{0.000000,0.000000,0.000000}%
\pgfsetstrokecolor{currentstroke}%
\pgfsetdash{}{0pt}%
\pgfsys@defobject{currentmarker}{\pgfqpoint{0.000000in}{-0.048611in}}{\pgfqpoint{0.000000in}{0.000000in}}{%
\pgfpathmoveto{\pgfqpoint{0.000000in}{0.000000in}}%
\pgfpathlineto{\pgfqpoint{0.000000in}{-0.048611in}}%
\pgfusepath{stroke,fill}%
}%
\begin{pgfscope}%
\pgfsys@transformshift{0.456635in}{4.237239in}%
\pgfsys@useobject{currentmarker}{}%
\end{pgfscope}%
\end{pgfscope}%
\begin{pgfscope}%
\pgfsetbuttcap%
\pgfsetroundjoin%
\definecolor{currentfill}{rgb}{0.000000,0.000000,0.000000}%
\pgfsetfillcolor{currentfill}%
\pgfsetlinewidth{0.803000pt}%
\definecolor{currentstroke}{rgb}{0.000000,0.000000,0.000000}%
\pgfsetstrokecolor{currentstroke}%
\pgfsetdash{}{0pt}%
\pgfsys@defobject{currentmarker}{\pgfqpoint{0.000000in}{-0.048611in}}{\pgfqpoint{0.000000in}{0.000000in}}{%
\pgfpathmoveto{\pgfqpoint{0.000000in}{0.000000in}}%
\pgfpathlineto{\pgfqpoint{0.000000in}{-0.048611in}}%
\pgfusepath{stroke,fill}%
}%
\begin{pgfscope}%
\pgfsys@transformshift{1.331929in}{4.237239in}%
\pgfsys@useobject{currentmarker}{}%
\end{pgfscope}%
\end{pgfscope}%
\begin{pgfscope}%
\pgfsetbuttcap%
\pgfsetroundjoin%
\definecolor{currentfill}{rgb}{0.000000,0.000000,0.000000}%
\pgfsetfillcolor{currentfill}%
\pgfsetlinewidth{0.803000pt}%
\definecolor{currentstroke}{rgb}{0.000000,0.000000,0.000000}%
\pgfsetstrokecolor{currentstroke}%
\pgfsetdash{}{0pt}%
\pgfsys@defobject{currentmarker}{\pgfqpoint{0.000000in}{-0.048611in}}{\pgfqpoint{0.000000in}{0.000000in}}{%
\pgfpathmoveto{\pgfqpoint{0.000000in}{0.000000in}}%
\pgfpathlineto{\pgfqpoint{0.000000in}{-0.048611in}}%
\pgfusepath{stroke,fill}%
}%
\begin{pgfscope}%
\pgfsys@transformshift{2.207224in}{4.237239in}%
\pgfsys@useobject{currentmarker}{}%
\end{pgfscope}%
\end{pgfscope}%
\begin{pgfscope}%
\pgfsetbuttcap%
\pgfsetroundjoin%
\definecolor{currentfill}{rgb}{0.000000,0.000000,0.000000}%
\pgfsetfillcolor{currentfill}%
\pgfsetlinewidth{0.803000pt}%
\definecolor{currentstroke}{rgb}{0.000000,0.000000,0.000000}%
\pgfsetstrokecolor{currentstroke}%
\pgfsetdash{}{0pt}%
\pgfsys@defobject{currentmarker}{\pgfqpoint{0.000000in}{-0.048611in}}{\pgfqpoint{0.000000in}{0.000000in}}{%
\pgfpathmoveto{\pgfqpoint{0.000000in}{0.000000in}}%
\pgfpathlineto{\pgfqpoint{0.000000in}{-0.048611in}}%
\pgfusepath{stroke,fill}%
}%
\begin{pgfscope}%
\pgfsys@transformshift{3.082518in}{4.237239in}%
\pgfsys@useobject{currentmarker}{}%
\end{pgfscope}%
\end{pgfscope}%
\begin{pgfscope}%
\pgfsetbuttcap%
\pgfsetroundjoin%
\definecolor{currentfill}{rgb}{0.000000,0.000000,0.000000}%
\pgfsetfillcolor{currentfill}%
\pgfsetlinewidth{0.803000pt}%
\definecolor{currentstroke}{rgb}{0.000000,0.000000,0.000000}%
\pgfsetstrokecolor{currentstroke}%
\pgfsetdash{}{0pt}%
\pgfsys@defobject{currentmarker}{\pgfqpoint{0.000000in}{-0.048611in}}{\pgfqpoint{0.000000in}{0.000000in}}{%
\pgfpathmoveto{\pgfqpoint{0.000000in}{0.000000in}}%
\pgfpathlineto{\pgfqpoint{0.000000in}{-0.048611in}}%
\pgfusepath{stroke,fill}%
}%
\begin{pgfscope}%
\pgfsys@transformshift{3.957812in}{4.237239in}%
\pgfsys@useobject{currentmarker}{}%
\end{pgfscope}%
\end{pgfscope}%
\begin{pgfscope}%
\pgfsetbuttcap%
\pgfsetroundjoin%
\definecolor{currentfill}{rgb}{0.000000,0.000000,0.000000}%
\pgfsetfillcolor{currentfill}%
\pgfsetlinewidth{0.803000pt}%
\definecolor{currentstroke}{rgb}{0.000000,0.000000,0.000000}%
\pgfsetstrokecolor{currentstroke}%
\pgfsetdash{}{0pt}%
\pgfsys@defobject{currentmarker}{\pgfqpoint{0.000000in}{-0.048611in}}{\pgfqpoint{0.000000in}{0.000000in}}{%
\pgfpathmoveto{\pgfqpoint{0.000000in}{0.000000in}}%
\pgfpathlineto{\pgfqpoint{0.000000in}{-0.048611in}}%
\pgfusepath{stroke,fill}%
}%
\begin{pgfscope}%
\pgfsys@transformshift{4.833106in}{4.237239in}%
\pgfsys@useobject{currentmarker}{}%
\end{pgfscope}%
\end{pgfscope}%
\begin{pgfscope}%
\pgfsetbuttcap%
\pgfsetroundjoin%
\definecolor{currentfill}{rgb}{0.000000,0.000000,0.000000}%
\pgfsetfillcolor{currentfill}%
\pgfsetlinewidth{0.803000pt}%
\definecolor{currentstroke}{rgb}{0.000000,0.000000,0.000000}%
\pgfsetstrokecolor{currentstroke}%
\pgfsetdash{}{0pt}%
\pgfsys@defobject{currentmarker}{\pgfqpoint{-0.048611in}{0.000000in}}{\pgfqpoint{0.000000in}{0.000000in}}{%
\pgfpathmoveto{\pgfqpoint{0.000000in}{0.000000in}}%
\pgfpathlineto{\pgfqpoint{-0.048611in}{0.000000in}}%
\pgfusepath{stroke,fill}%
}%
\begin{pgfscope}%
\pgfsys@transformshift{0.456635in}{4.601976in}%
\pgfsys@useobject{currentmarker}{}%
\end{pgfscope}%
\end{pgfscope}%
\begin{pgfscope}%
\pgftext[x=0.289968in,y=4.549214in,left,base]{\rmfamily\fontsize{10.000000}{12.000000}\selectfont \(\displaystyle 0\)}%
\end{pgfscope}%
\begin{pgfscope}%
\pgfsetbuttcap%
\pgfsetroundjoin%
\definecolor{currentfill}{rgb}{0.000000,0.000000,0.000000}%
\pgfsetfillcolor{currentfill}%
\pgfsetlinewidth{0.803000pt}%
\definecolor{currentstroke}{rgb}{0.000000,0.000000,0.000000}%
\pgfsetstrokecolor{currentstroke}%
\pgfsetdash{}{0pt}%
\pgfsys@defobject{currentmarker}{\pgfqpoint{-0.048611in}{0.000000in}}{\pgfqpoint{0.000000in}{0.000000in}}{%
\pgfpathmoveto{\pgfqpoint{0.000000in}{0.000000in}}%
\pgfpathlineto{\pgfqpoint{-0.048611in}{0.000000in}}%
\pgfusepath{stroke,fill}%
}%
\begin{pgfscope}%
\pgfsys@transformshift{0.456635in}{5.007239in}%
\pgfsys@useobject{currentmarker}{}%
\end{pgfscope}%
\end{pgfscope}%
\begin{pgfscope}%
\pgftext[x=0.289968in,y=4.954477in,left,base]{\rmfamily\fontsize{10.000000}{12.000000}\selectfont \(\displaystyle 2\)}%
\end{pgfscope}%
\begin{pgfscope}%
\pgftext[x=0.234413in,y=4.723554in,,bottom,rotate=90.000000]{\rmfamily\fontsize{10.000000}{12.000000}\selectfont y}%
\end{pgfscope}%
\begin{pgfscope}%
\pgfpathrectangle{\pgfqpoint{0.456635in}{4.237239in}}{\pgfqpoint{4.376471in}{0.972632in}}%
\pgfusepath{clip}%
\pgfsetrectcap%
\pgfsetroundjoin%
\pgfsetlinewidth{1.505625pt}%
\definecolor{currentstroke}{rgb}{0.121569,0.466667,0.705882}%
\pgfsetstrokecolor{currentstroke}%
\pgfsetdash{}{0pt}%
\pgfpathmoveto{\pgfqpoint{1.331929in}{5.118209in}}%
\pgfpathlineto{\pgfqpoint{1.419899in}{4.991787in}}%
\pgfpathlineto{\pgfqpoint{1.455086in}{4.948653in}}%
\pgfpathlineto{\pgfqpoint{1.472680in}{4.929953in}}%
\pgfpathlineto{\pgfqpoint{1.490274in}{4.913294in}}%
\pgfpathlineto{\pgfqpoint{1.507868in}{4.898647in}}%
\pgfpathlineto{\pgfqpoint{1.525462in}{4.885883in}}%
\pgfpathlineto{\pgfqpoint{1.543056in}{4.874796in}}%
\pgfpathlineto{\pgfqpoint{1.578243in}{4.856521in}}%
\pgfpathlineto{\pgfqpoint{1.631025in}{4.833978in}}%
\pgfpathlineto{\pgfqpoint{1.683806in}{4.810625in}}%
\pgfpathlineto{\pgfqpoint{1.736588in}{4.783693in}}%
\pgfpathlineto{\pgfqpoint{1.789370in}{4.757012in}}%
\pgfpathlineto{\pgfqpoint{1.806963in}{4.749580in}}%
\pgfpathlineto{\pgfqpoint{1.824557in}{4.743464in}}%
\pgfpathlineto{\pgfqpoint{1.842151in}{4.738996in}}%
\pgfpathlineto{\pgfqpoint{1.859745in}{4.736461in}}%
\pgfpathlineto{\pgfqpoint{1.877339in}{4.736078in}}%
\pgfpathlineto{\pgfqpoint{1.894933in}{4.737988in}}%
\pgfpathlineto{\pgfqpoint{1.912526in}{4.742239in}}%
\pgfpathlineto{\pgfqpoint{1.930120in}{4.748784in}}%
\pgfpathlineto{\pgfqpoint{1.947714in}{4.757478in}}%
\pgfpathlineto{\pgfqpoint{1.965308in}{4.768085in}}%
\pgfpathlineto{\pgfqpoint{2.000496in}{4.793703in}}%
\pgfpathlineto{\pgfqpoint{2.088465in}{4.863274in}}%
\pgfpathlineto{\pgfqpoint{2.106059in}{4.874605in}}%
\pgfpathlineto{\pgfqpoint{2.123653in}{4.884219in}}%
\pgfpathlineto{\pgfqpoint{2.141247in}{4.891868in}}%
\pgfpathlineto{\pgfqpoint{2.158840in}{4.897375in}}%
\pgfpathlineto{\pgfqpoint{2.176434in}{4.900637in}}%
\pgfpathlineto{\pgfqpoint{2.194028in}{4.901613in}}%
\pgfpathlineto{\pgfqpoint{2.211622in}{4.900320in}}%
\pgfpathlineto{\pgfqpoint{2.229216in}{4.896819in}}%
\pgfpathlineto{\pgfqpoint{2.246810in}{4.891201in}}%
\pgfpathlineto{\pgfqpoint{2.264404in}{4.883576in}}%
\pgfpathlineto{\pgfqpoint{2.281997in}{4.874055in}}%
\pgfpathlineto{\pgfqpoint{2.299591in}{4.862743in}}%
\pgfpathlineto{\pgfqpoint{2.317185in}{4.849725in}}%
\pgfpathlineto{\pgfqpoint{2.334779in}{4.835063in}}%
\pgfpathlineto{\pgfqpoint{2.352373in}{4.818793in}}%
\pgfpathlineto{\pgfqpoint{2.369967in}{4.800925in}}%
\pgfpathlineto{\pgfqpoint{2.405154in}{4.760353in}}%
\pgfpathlineto{\pgfqpoint{2.440342in}{4.713225in}}%
\pgfpathlineto{\pgfqpoint{2.475530in}{4.659648in}}%
\pgfpathlineto{\pgfqpoint{2.510717in}{4.600384in}}%
\pgfpathlineto{\pgfqpoint{2.563499in}{4.504928in}}%
\pgfpathlineto{\pgfqpoint{2.598687in}{4.441395in}}%
\pgfpathlineto{\pgfqpoint{2.633874in}{4.382583in}}%
\pgfpathlineto{\pgfqpoint{2.651468in}{4.356290in}}%
\pgfpathlineto{\pgfqpoint{2.669062in}{4.332755in}}%
\pgfpathlineto{\pgfqpoint{2.686656in}{4.312435in}}%
\pgfpathlineto{\pgfqpoint{2.704250in}{4.295717in}}%
\pgfpathlineto{\pgfqpoint{2.721844in}{4.282903in}}%
\pgfpathlineto{\pgfqpoint{2.739438in}{4.274197in}}%
\pgfpathlineto{\pgfqpoint{2.757031in}{4.269699in}}%
\pgfpathlineto{\pgfqpoint{2.774625in}{4.269401in}}%
\pgfpathlineto{\pgfqpoint{2.792219in}{4.273189in}}%
\pgfpathlineto{\pgfqpoint{2.809813in}{4.280853in}}%
\pgfpathlineto{\pgfqpoint{2.827407in}{4.292101in}}%
\pgfpathlineto{\pgfqpoint{2.845001in}{4.306571in}}%
\pgfpathlineto{\pgfqpoint{2.862594in}{4.323853in}}%
\pgfpathlineto{\pgfqpoint{2.880188in}{4.343511in}}%
\pgfpathlineto{\pgfqpoint{2.915376in}{4.388189in}}%
\pgfpathlineto{\pgfqpoint{2.950564in}{4.437322in}}%
\pgfpathlineto{\pgfqpoint{3.108908in}{4.664766in}}%
\pgfpathlineto{\pgfqpoint{3.161690in}{4.736762in}}%
\pgfpathlineto{\pgfqpoint{3.196878in}{4.781608in}}%
\pgfpathlineto{\pgfqpoint{3.232065in}{4.821480in}}%
\pgfpathlineto{\pgfqpoint{3.249659in}{4.838660in}}%
\pgfpathlineto{\pgfqpoint{3.267253in}{4.853479in}}%
\pgfpathlineto{\pgfqpoint{3.284847in}{4.865541in}}%
\pgfpathlineto{\pgfqpoint{3.302441in}{4.874481in}}%
\pgfpathlineto{\pgfqpoint{3.320035in}{4.879975in}}%
\pgfpathlineto{\pgfqpoint{3.337628in}{4.881767in}}%
\pgfpathlineto{\pgfqpoint{3.355222in}{4.879682in}}%
\pgfpathlineto{\pgfqpoint{3.372816in}{4.873643in}}%
\pgfpathlineto{\pgfqpoint{3.390410in}{4.863678in}}%
\pgfpathlineto{\pgfqpoint{3.408004in}{4.849921in}}%
\pgfpathlineto{\pgfqpoint{3.425598in}{4.832616in}}%
\pgfpathlineto{\pgfqpoint{3.443192in}{4.812107in}}%
\pgfpathlineto{\pgfqpoint{3.460785in}{4.788823in}}%
\pgfpathlineto{\pgfqpoint{3.495973in}{4.735988in}}%
\pgfpathlineto{\pgfqpoint{3.583942in}{4.593841in}}%
\pgfpathlineto{\pgfqpoint{3.619130in}{4.543633in}}%
\pgfpathlineto{\pgfqpoint{3.636724in}{4.521374in}}%
\pgfpathlineto{\pgfqpoint{3.654318in}{4.501237in}}%
\pgfpathlineto{\pgfqpoint{3.671912in}{4.483290in}}%
\pgfpathlineto{\pgfqpoint{3.689506in}{4.467531in}}%
\pgfpathlineto{\pgfqpoint{3.707099in}{4.453897in}}%
\pgfpathlineto{\pgfqpoint{3.724693in}{4.442283in}}%
\pgfpathlineto{\pgfqpoint{3.742287in}{4.432557in}}%
\pgfpathlineto{\pgfqpoint{3.759881in}{4.424579in}}%
\pgfpathlineto{\pgfqpoint{3.777475in}{4.418212in}}%
\pgfpathlineto{\pgfqpoint{3.795069in}{4.413341in}}%
\pgfpathlineto{\pgfqpoint{3.812662in}{4.409877in}}%
\pgfpathlineto{\pgfqpoint{3.830256in}{4.407772in}}%
\pgfpathlineto{\pgfqpoint{3.847850in}{4.407015in}}%
\pgfpathlineto{\pgfqpoint{3.865444in}{4.407630in}}%
\pgfpathlineto{\pgfqpoint{3.883038in}{4.409679in}}%
\pgfpathlineto{\pgfqpoint{3.900632in}{4.413245in}}%
\pgfpathlineto{\pgfqpoint{3.918226in}{4.418423in}}%
\pgfpathlineto{\pgfqpoint{3.935819in}{4.425311in}}%
\pgfpathlineto{\pgfqpoint{3.953413in}{4.433991in}}%
\pgfpathlineto{\pgfqpoint{3.971007in}{4.444521in}}%
\pgfpathlineto{\pgfqpoint{3.988601in}{4.456923in}}%
\pgfpathlineto{\pgfqpoint{4.006195in}{4.471172in}}%
\pgfpathlineto{\pgfqpoint{4.023789in}{4.487193in}}%
\pgfpathlineto{\pgfqpoint{4.058976in}{4.523994in}}%
\pgfpathlineto{\pgfqpoint{4.094164in}{4.565755in}}%
\pgfpathlineto{\pgfqpoint{4.164540in}{4.655740in}}%
\pgfpathlineto{\pgfqpoint{4.217321in}{4.721342in}}%
\pgfpathlineto{\pgfqpoint{4.270103in}{4.781966in}}%
\pgfpathlineto{\pgfqpoint{4.463635in}{4.992832in}}%
\pgfpathlineto{\pgfqpoint{4.481229in}{5.010083in}}%
\pgfpathlineto{\pgfqpoint{4.498823in}{5.025587in}}%
\pgfpathlineto{\pgfqpoint{4.516417in}{5.038739in}}%
\pgfpathlineto{\pgfqpoint{4.534010in}{5.048900in}}%
\pgfpathlineto{\pgfqpoint{4.551604in}{5.055420in}}%
\pgfpathlineto{\pgfqpoint{4.569198in}{5.057670in}}%
\pgfpathlineto{\pgfqpoint{4.586792in}{5.055069in}}%
\pgfpathlineto{\pgfqpoint{4.604386in}{5.047118in}}%
\pgfpathlineto{\pgfqpoint{4.621980in}{5.033420in}}%
\pgfpathlineto{\pgfqpoint{4.639573in}{5.013707in}}%
\pgfpathlineto{\pgfqpoint{4.657167in}{4.987859in}}%
\pgfpathlineto{\pgfqpoint{4.674761in}{4.955908in}}%
\pgfpathlineto{\pgfqpoint{4.692355in}{4.918053in}}%
\pgfpathlineto{\pgfqpoint{4.709949in}{4.874651in}}%
\pgfpathlineto{\pgfqpoint{4.727543in}{4.826212in}}%
\pgfpathlineto{\pgfqpoint{4.745137in}{4.773386in}}%
\pgfpathlineto{\pgfqpoint{4.780324in}{4.657726in}}%
\pgfpathlineto{\pgfqpoint{4.833106in}{4.472961in}}%
\pgfpathlineto{\pgfqpoint{4.833106in}{4.472961in}}%
\pgfusepath{stroke}%
\end{pgfscope}%
\begin{pgfscope}%
\pgfpathrectangle{\pgfqpoint{0.456635in}{4.237239in}}{\pgfqpoint{4.376471in}{0.972632in}}%
\pgfusepath{clip}%
\pgfsetbuttcap%
\pgfsetroundjoin%
\pgfsetlinewidth{1.505625pt}%
\definecolor{currentstroke}{rgb}{1.000000,0.498039,0.054902}%
\pgfsetstrokecolor{currentstroke}%
\pgfsetdash{{9.600000pt}{2.400000pt}{1.500000pt}{2.400000pt}}{0.000000pt}%
\pgfpathmoveto{\pgfqpoint{1.331929in}{5.118209in}}%
\pgfpathlineto{\pgfqpoint{1.419899in}{4.991787in}}%
\pgfpathlineto{\pgfqpoint{1.455086in}{4.948653in}}%
\pgfpathlineto{\pgfqpoint{1.472680in}{4.929953in}}%
\pgfpathlineto{\pgfqpoint{1.490274in}{4.913294in}}%
\pgfpathlineto{\pgfqpoint{1.507868in}{4.898647in}}%
\pgfpathlineto{\pgfqpoint{1.525462in}{4.885883in}}%
\pgfpathlineto{\pgfqpoint{1.543056in}{4.874796in}}%
\pgfpathlineto{\pgfqpoint{1.578243in}{4.856521in}}%
\pgfpathlineto{\pgfqpoint{1.631025in}{4.833978in}}%
\pgfpathlineto{\pgfqpoint{1.683806in}{4.810625in}}%
\pgfpathlineto{\pgfqpoint{1.736588in}{4.783693in}}%
\pgfpathlineto{\pgfqpoint{1.789370in}{4.757012in}}%
\pgfpathlineto{\pgfqpoint{1.806963in}{4.749580in}}%
\pgfpathlineto{\pgfqpoint{1.824557in}{4.743464in}}%
\pgfpathlineto{\pgfqpoint{1.842151in}{4.738996in}}%
\pgfpathlineto{\pgfqpoint{1.859745in}{4.736461in}}%
\pgfpathlineto{\pgfqpoint{1.877339in}{4.736078in}}%
\pgfpathlineto{\pgfqpoint{1.894933in}{4.737988in}}%
\pgfpathlineto{\pgfqpoint{1.912526in}{4.742239in}}%
\pgfpathlineto{\pgfqpoint{1.930120in}{4.748784in}}%
\pgfpathlineto{\pgfqpoint{1.947714in}{4.757478in}}%
\pgfpathlineto{\pgfqpoint{1.965308in}{4.768085in}}%
\pgfpathlineto{\pgfqpoint{2.000496in}{4.793703in}}%
\pgfpathlineto{\pgfqpoint{2.088465in}{4.863274in}}%
\pgfpathlineto{\pgfqpoint{2.106059in}{4.874605in}}%
\pgfpathlineto{\pgfqpoint{2.123653in}{4.884219in}}%
\pgfpathlineto{\pgfqpoint{2.141247in}{4.891868in}}%
\pgfpathlineto{\pgfqpoint{2.158840in}{4.897375in}}%
\pgfpathlineto{\pgfqpoint{2.176434in}{4.900637in}}%
\pgfpathlineto{\pgfqpoint{2.194028in}{4.901613in}}%
\pgfpathlineto{\pgfqpoint{2.211622in}{4.900320in}}%
\pgfpathlineto{\pgfqpoint{2.229216in}{4.896819in}}%
\pgfpathlineto{\pgfqpoint{2.246810in}{4.891201in}}%
\pgfpathlineto{\pgfqpoint{2.264404in}{4.883576in}}%
\pgfpathlineto{\pgfqpoint{2.281997in}{4.874055in}}%
\pgfpathlineto{\pgfqpoint{2.299591in}{4.862743in}}%
\pgfpathlineto{\pgfqpoint{2.317185in}{4.849725in}}%
\pgfpathlineto{\pgfqpoint{2.334779in}{4.835063in}}%
\pgfpathlineto{\pgfqpoint{2.352373in}{4.818793in}}%
\pgfpathlineto{\pgfqpoint{2.369967in}{4.800925in}}%
\pgfpathlineto{\pgfqpoint{2.405154in}{4.760353in}}%
\pgfpathlineto{\pgfqpoint{2.440342in}{4.713225in}}%
\pgfpathlineto{\pgfqpoint{2.475530in}{4.659648in}}%
\pgfpathlineto{\pgfqpoint{2.510717in}{4.600384in}}%
\pgfpathlineto{\pgfqpoint{2.563499in}{4.504928in}}%
\pgfpathlineto{\pgfqpoint{2.598687in}{4.441395in}}%
\pgfpathlineto{\pgfqpoint{2.633874in}{4.382583in}}%
\pgfpathlineto{\pgfqpoint{2.651468in}{4.356290in}}%
\pgfpathlineto{\pgfqpoint{2.669062in}{4.332755in}}%
\pgfpathlineto{\pgfqpoint{2.686656in}{4.312435in}}%
\pgfpathlineto{\pgfqpoint{2.704250in}{4.295717in}}%
\pgfpathlineto{\pgfqpoint{2.721844in}{4.282903in}}%
\pgfpathlineto{\pgfqpoint{2.739438in}{4.274197in}}%
\pgfpathlineto{\pgfqpoint{2.757031in}{4.269699in}}%
\pgfpathlineto{\pgfqpoint{2.774625in}{4.269401in}}%
\pgfpathlineto{\pgfqpoint{2.792219in}{4.273189in}}%
\pgfpathlineto{\pgfqpoint{2.809813in}{4.280853in}}%
\pgfpathlineto{\pgfqpoint{2.827407in}{4.292101in}}%
\pgfpathlineto{\pgfqpoint{2.845001in}{4.306571in}}%
\pgfpathlineto{\pgfqpoint{2.862594in}{4.323853in}}%
\pgfpathlineto{\pgfqpoint{2.880188in}{4.343511in}}%
\pgfpathlineto{\pgfqpoint{2.915376in}{4.388189in}}%
\pgfpathlineto{\pgfqpoint{2.950564in}{4.437322in}}%
\pgfpathlineto{\pgfqpoint{3.108908in}{4.664766in}}%
\pgfpathlineto{\pgfqpoint{3.161690in}{4.736762in}}%
\pgfpathlineto{\pgfqpoint{3.196878in}{4.781608in}}%
\pgfpathlineto{\pgfqpoint{3.232065in}{4.821480in}}%
\pgfpathlineto{\pgfqpoint{3.249659in}{4.838660in}}%
\pgfpathlineto{\pgfqpoint{3.267253in}{4.853479in}}%
\pgfpathlineto{\pgfqpoint{3.284847in}{4.865541in}}%
\pgfpathlineto{\pgfqpoint{3.302441in}{4.874481in}}%
\pgfpathlineto{\pgfqpoint{3.320035in}{4.879975in}}%
\pgfpathlineto{\pgfqpoint{3.337628in}{4.881767in}}%
\pgfpathlineto{\pgfqpoint{3.355222in}{4.879682in}}%
\pgfpathlineto{\pgfqpoint{3.372816in}{4.873643in}}%
\pgfpathlineto{\pgfqpoint{3.390410in}{4.863678in}}%
\pgfpathlineto{\pgfqpoint{3.408004in}{4.849921in}}%
\pgfpathlineto{\pgfqpoint{3.425598in}{4.832616in}}%
\pgfpathlineto{\pgfqpoint{3.443192in}{4.812107in}}%
\pgfpathlineto{\pgfqpoint{3.460785in}{4.788823in}}%
\pgfpathlineto{\pgfqpoint{3.495973in}{4.735988in}}%
\pgfpathlineto{\pgfqpoint{3.583942in}{4.593841in}}%
\pgfpathlineto{\pgfqpoint{3.619130in}{4.543633in}}%
\pgfpathlineto{\pgfqpoint{3.636724in}{4.521374in}}%
\pgfpathlineto{\pgfqpoint{3.654318in}{4.501237in}}%
\pgfpathlineto{\pgfqpoint{3.671912in}{4.483290in}}%
\pgfpathlineto{\pgfqpoint{3.689506in}{4.467531in}}%
\pgfpathlineto{\pgfqpoint{3.707099in}{4.453897in}}%
\pgfpathlineto{\pgfqpoint{3.724693in}{4.442283in}}%
\pgfpathlineto{\pgfqpoint{3.742287in}{4.432557in}}%
\pgfpathlineto{\pgfqpoint{3.759881in}{4.424579in}}%
\pgfpathlineto{\pgfqpoint{3.777475in}{4.418212in}}%
\pgfpathlineto{\pgfqpoint{3.795069in}{4.413341in}}%
\pgfpathlineto{\pgfqpoint{3.812662in}{4.409877in}}%
\pgfpathlineto{\pgfqpoint{3.830256in}{4.407772in}}%
\pgfpathlineto{\pgfqpoint{3.847850in}{4.407015in}}%
\pgfpathlineto{\pgfqpoint{3.865444in}{4.407630in}}%
\pgfpathlineto{\pgfqpoint{3.883038in}{4.409679in}}%
\pgfpathlineto{\pgfqpoint{3.900632in}{4.413245in}}%
\pgfpathlineto{\pgfqpoint{3.918226in}{4.418423in}}%
\pgfpathlineto{\pgfqpoint{3.935819in}{4.425311in}}%
\pgfpathlineto{\pgfqpoint{3.953413in}{4.433991in}}%
\pgfpathlineto{\pgfqpoint{3.971007in}{4.444521in}}%
\pgfpathlineto{\pgfqpoint{3.988601in}{4.456923in}}%
\pgfpathlineto{\pgfqpoint{4.006195in}{4.471172in}}%
\pgfpathlineto{\pgfqpoint{4.023789in}{4.487193in}}%
\pgfpathlineto{\pgfqpoint{4.058976in}{4.523994in}}%
\pgfpathlineto{\pgfqpoint{4.094164in}{4.565755in}}%
\pgfpathlineto{\pgfqpoint{4.164540in}{4.655740in}}%
\pgfpathlineto{\pgfqpoint{4.217321in}{4.721342in}}%
\pgfpathlineto{\pgfqpoint{4.270103in}{4.781966in}}%
\pgfpathlineto{\pgfqpoint{4.463635in}{4.992832in}}%
\pgfpathlineto{\pgfqpoint{4.481229in}{5.010083in}}%
\pgfpathlineto{\pgfqpoint{4.498823in}{5.025587in}}%
\pgfpathlineto{\pgfqpoint{4.516417in}{5.038739in}}%
\pgfpathlineto{\pgfqpoint{4.534010in}{5.048900in}}%
\pgfpathlineto{\pgfqpoint{4.551604in}{5.055420in}}%
\pgfpathlineto{\pgfqpoint{4.569198in}{5.057670in}}%
\pgfpathlineto{\pgfqpoint{4.586792in}{5.055069in}}%
\pgfpathlineto{\pgfqpoint{4.604386in}{5.047118in}}%
\pgfpathlineto{\pgfqpoint{4.621980in}{5.033420in}}%
\pgfpathlineto{\pgfqpoint{4.639573in}{5.013707in}}%
\pgfpathlineto{\pgfqpoint{4.657167in}{4.987859in}}%
\pgfpathlineto{\pgfqpoint{4.674761in}{4.955908in}}%
\pgfpathlineto{\pgfqpoint{4.692355in}{4.918053in}}%
\pgfpathlineto{\pgfqpoint{4.709949in}{4.874651in}}%
\pgfpathlineto{\pgfqpoint{4.727543in}{4.826212in}}%
\pgfpathlineto{\pgfqpoint{4.745137in}{4.773386in}}%
\pgfpathlineto{\pgfqpoint{4.780324in}{4.657726in}}%
\pgfpathlineto{\pgfqpoint{4.833106in}{4.472961in}}%
\pgfpathlineto{\pgfqpoint{4.833106in}{4.472961in}}%
\pgfusepath{stroke}%
\end{pgfscope}%
\begin{pgfscope}%
\pgfsetrectcap%
\pgfsetmiterjoin%
\pgfsetlinewidth{0.803000pt}%
\definecolor{currentstroke}{rgb}{0.000000,0.000000,0.000000}%
\pgfsetstrokecolor{currentstroke}%
\pgfsetdash{}{0pt}%
\pgfpathmoveto{\pgfqpoint{0.456635in}{4.237239in}}%
\pgfpathlineto{\pgfqpoint{0.456635in}{5.209870in}}%
\pgfusepath{stroke}%
\end{pgfscope}%
\begin{pgfscope}%
\pgfsetrectcap%
\pgfsetmiterjoin%
\pgfsetlinewidth{0.803000pt}%
\definecolor{currentstroke}{rgb}{0.000000,0.000000,0.000000}%
\pgfsetstrokecolor{currentstroke}%
\pgfsetdash{}{0pt}%
\pgfpathmoveto{\pgfqpoint{4.833106in}{4.237239in}}%
\pgfpathlineto{\pgfqpoint{4.833106in}{5.209870in}}%
\pgfusepath{stroke}%
\end{pgfscope}%
\begin{pgfscope}%
\pgfsetrectcap%
\pgfsetmiterjoin%
\pgfsetlinewidth{0.803000pt}%
\definecolor{currentstroke}{rgb}{0.000000,0.000000,0.000000}%
\pgfsetstrokecolor{currentstroke}%
\pgfsetdash{}{0pt}%
\pgfpathmoveto{\pgfqpoint{0.456635in}{4.237239in}}%
\pgfpathlineto{\pgfqpoint{4.833106in}{4.237239in}}%
\pgfusepath{stroke}%
\end{pgfscope}%
\begin{pgfscope}%
\pgfsetrectcap%
\pgfsetmiterjoin%
\pgfsetlinewidth{0.803000pt}%
\definecolor{currentstroke}{rgb}{0.000000,0.000000,0.000000}%
\pgfsetstrokecolor{currentstroke}%
\pgfsetdash{}{0pt}%
\pgfpathmoveto{\pgfqpoint{0.456635in}{5.209870in}}%
\pgfpathlineto{\pgfqpoint{4.833106in}{5.209870in}}%
\pgfusepath{stroke}%
\end{pgfscope}%
\begin{pgfscope}%
\pgfsetbuttcap%
\pgfsetmiterjoin%
\definecolor{currentfill}{rgb}{1.000000,1.000000,1.000000}%
\pgfsetfillcolor{currentfill}%
\pgfsetfillopacity{0.800000}%
\pgfsetlinewidth{1.003750pt}%
\definecolor{currentstroke}{rgb}{0.800000,0.800000,0.800000}%
\pgfsetstrokecolor{currentstroke}%
\pgfsetstrokeopacity{0.800000}%
\pgfsetdash{}{0pt}%
\pgfpathmoveto{\pgfqpoint{0.553858in}{4.306683in}}%
\pgfpathlineto{\pgfqpoint{1.337183in}{4.306683in}}%
\pgfpathquadraticcurveto{\pgfqpoint{1.364960in}{4.306683in}}{\pgfqpoint{1.364960in}{4.334461in}}%
\pgfpathlineto{\pgfqpoint{1.364960in}{5.138009in}}%
\pgfpathquadraticcurveto{\pgfqpoint{1.364960in}{5.165786in}}{\pgfqpoint{1.337183in}{5.165786in}}%
\pgfpathlineto{\pgfqpoint{0.553858in}{5.165786in}}%
\pgfpathquadraticcurveto{\pgfqpoint{0.526080in}{5.165786in}}{\pgfqpoint{0.526080in}{5.138009in}}%
\pgfpathlineto{\pgfqpoint{0.526080in}{4.334461in}}%
\pgfpathquadraticcurveto{\pgfqpoint{0.526080in}{4.306683in}}{\pgfqpoint{0.553858in}{4.306683in}}%
\pgfpathclose%
\pgfusepath{stroke,fill}%
\end{pgfscope}%
\begin{pgfscope}%
\pgfsetrectcap%
\pgfsetroundjoin%
\pgfsetlinewidth{1.505625pt}%
\definecolor{currentstroke}{rgb}{0.121569,0.466667,0.705882}%
\pgfsetstrokecolor{currentstroke}%
\pgfsetdash{}{0pt}%
\pgfpathmoveto{\pgfqpoint{0.581635in}{5.052315in}}%
\pgfpathlineto{\pgfqpoint{0.859413in}{5.052315in}}%
\pgfusepath{stroke}%
\end{pgfscope}%
\begin{pgfscope}%
\pgftext[x=0.970524in,y=5.003704in,left,base]{\rmfamily\fontsize{10.000000}{12.000000}\selectfont \(\displaystyle \widetilde{\Phi}^* \theta\)}%
\end{pgfscope}%
\begin{pgfscope}%
\pgfsetbuttcap%
\pgfsetroundjoin%
\pgfsetlinewidth{1.505625pt}%
\definecolor{currentstroke}{rgb}{1.000000,0.498039,0.054902}%
\pgfsetstrokecolor{currentstroke}%
\pgfsetdash{{9.600000pt}{2.400000pt}{1.500000pt}{2.400000pt}}{0.000000pt}%
\pgfpathmoveto{\pgfqpoint{0.581635in}{4.847454in}}%
\pgfpathlineto{\pgfqpoint{0.859413in}{4.847454in}}%
\pgfusepath{stroke}%
\end{pgfscope}%
\begin{pgfscope}%
\pgftext[x=0.970524in,y=4.798843in,left,base]{\rmfamily\fontsize{10.000000}{12.000000}\selectfont \(\displaystyle \widetilde{K}u\)}%
\end{pgfscope}%
\begin{pgfscope}%
\pgfsetbuttcap%
\pgfsetroundjoin%
\definecolor{currentfill}{rgb}{1.000000,0.000000,0.000000}%
\pgfsetfillcolor{currentfill}%
\pgfsetlinewidth{2.007500pt}%
\definecolor{currentstroke}{rgb}{1.000000,0.000000,0.000000}%
\pgfsetstrokecolor{currentstroke}%
\pgfsetdash{}{0pt}%
\pgfpathmoveto{\pgfqpoint{0.678857in}{4.631444in}}%
\pgfpathlineto{\pgfqpoint{0.762191in}{4.631444in}}%
\pgfpathmoveto{\pgfqpoint{0.720524in}{4.589777in}}%
\pgfpathlineto{\pgfqpoint{0.720524in}{4.673111in}}%
\pgfusepath{stroke,fill}%
\end{pgfscope}%
\begin{pgfscope}%
\pgftext[x=0.970524in,y=4.594986in,left,base]{\rmfamily\fontsize{10.000000}{12.000000}\selectfont train}%
\end{pgfscope}%
\begin{pgfscope}%
\pgfsetbuttcap%
\pgfsetroundjoin%
\definecolor{currentfill}{rgb}{0.000000,0.000000,0.000000}%
\pgfsetfillcolor{currentfill}%
\pgfsetlinewidth{1.003750pt}%
\definecolor{currentstroke}{rgb}{0.000000,0.000000,0.000000}%
\pgfsetstrokecolor{currentstroke}%
\pgfsetdash{}{0pt}%
\pgfsys@defobject{currentmarker}{\pgfqpoint{-0.020833in}{-0.020833in}}{\pgfqpoint{0.020833in}{0.020833in}}{%
\pgfpathmoveto{\pgfqpoint{0.000000in}{-0.020833in}}%
\pgfpathcurveto{\pgfqpoint{0.005525in}{-0.020833in}}{\pgfqpoint{0.010825in}{-0.018638in}}{\pgfqpoint{0.014731in}{-0.014731in}}%
\pgfpathcurveto{\pgfqpoint{0.018638in}{-0.010825in}}{\pgfqpoint{0.020833in}{-0.005525in}}{\pgfqpoint{0.020833in}{0.000000in}}%
\pgfpathcurveto{\pgfqpoint{0.020833in}{0.005525in}}{\pgfqpoint{0.018638in}{0.010825in}}{\pgfqpoint{0.014731in}{0.014731in}}%
\pgfpathcurveto{\pgfqpoint{0.010825in}{0.018638in}}{\pgfqpoint{0.005525in}{0.020833in}}{\pgfqpoint{0.000000in}{0.020833in}}%
\pgfpathcurveto{\pgfqpoint{-0.005525in}{0.020833in}}{\pgfqpoint{-0.010825in}{0.018638in}}{\pgfqpoint{-0.014731in}{0.014731in}}%
\pgfpathcurveto{\pgfqpoint{-0.018638in}{0.010825in}}{\pgfqpoint{-0.020833in}{0.005525in}}{\pgfqpoint{-0.020833in}{0.000000in}}%
\pgfpathcurveto{\pgfqpoint{-0.020833in}{-0.005525in}}{\pgfqpoint{-0.018638in}{-0.010825in}}{\pgfqpoint{-0.014731in}{-0.014731in}}%
\pgfpathcurveto{\pgfqpoint{-0.010825in}{-0.018638in}}{\pgfqpoint{-0.005525in}{-0.020833in}}{\pgfqpoint{0.000000in}{-0.020833in}}%
\pgfpathclose%
\pgfusepath{stroke,fill}%
}%
\begin{pgfscope}%
\pgfsys@transformshift{0.720524in}{4.427587in}%
\pgfsys@useobject{currentmarker}{}%
\end{pgfscope}%
\end{pgfscope}%
\begin{pgfscope}%
\pgftext[x=0.970524in,y=4.391128in,left,base]{\rmfamily\fontsize{10.000000}{12.000000}\selectfont test}%
\end{pgfscope}%
\begin{pgfscope}%
\pgfsetbuttcap%
\pgfsetmiterjoin%
\definecolor{currentfill}{rgb}{1.000000,1.000000,1.000000}%
\pgfsetfillcolor{currentfill}%
\pgfsetlinewidth{0.000000pt}%
\definecolor{currentstroke}{rgb}{0.000000,0.000000,0.000000}%
\pgfsetstrokecolor{currentstroke}%
\pgfsetstrokeopacity{0.000000}%
\pgfsetdash{}{0pt}%
\pgfpathmoveto{\pgfqpoint{5.562518in}{4.237239in}}%
\pgfpathlineto{\pgfqpoint{9.938988in}{4.237239in}}%
\pgfpathlineto{\pgfqpoint{9.938988in}{5.209870in}}%
\pgfpathlineto{\pgfqpoint{5.562518in}{5.209870in}}%
\pgfpathclose%
\pgfusepath{fill}%
\end{pgfscope}%
\begin{pgfscope}%
\pgfpathrectangle{\pgfqpoint{5.562518in}{4.237239in}}{\pgfqpoint{4.376471in}{0.972632in}}%
\pgfusepath{clip}%
\pgfsetbuttcap%
\pgfsetroundjoin%
\definecolor{currentfill}{rgb}{1.000000,0.000000,0.000000}%
\pgfsetfillcolor{currentfill}%
\pgfsetlinewidth{2.007500pt}%
\definecolor{currentstroke}{rgb}{1.000000,0.000000,0.000000}%
\pgfsetstrokecolor{currentstroke}%
\pgfsetdash{}{0pt}%
\pgfpathmoveto{\pgfqpoint{7.707476in}{4.391106in}}%
\pgfpathlineto{\pgfqpoint{7.790809in}{4.391106in}}%
\pgfpathmoveto{\pgfqpoint{7.749143in}{4.349440in}}%
\pgfpathlineto{\pgfqpoint{7.749143in}{4.432773in}}%
\pgfusepath{stroke,fill}%
\end{pgfscope}%
\begin{pgfscope}%
\pgfpathrectangle{\pgfqpoint{5.562518in}{4.237239in}}{\pgfqpoint{4.376471in}{0.972632in}}%
\pgfusepath{clip}%
\pgfsetbuttcap%
\pgfsetroundjoin%
\definecolor{currentfill}{rgb}{1.000000,0.000000,0.000000}%
\pgfsetfillcolor{currentfill}%
\pgfsetlinewidth{2.007500pt}%
\definecolor{currentstroke}{rgb}{1.000000,0.000000,0.000000}%
\pgfsetstrokecolor{currentstroke}%
\pgfsetdash{}{0pt}%
\pgfpathmoveto{\pgfqpoint{9.724764in}{5.050208in}}%
\pgfpathlineto{\pgfqpoint{9.808097in}{5.050208in}}%
\pgfpathmoveto{\pgfqpoint{9.766430in}{5.008541in}}%
\pgfpathlineto{\pgfqpoint{9.766430in}{5.091875in}}%
\pgfusepath{stroke,fill}%
\end{pgfscope}%
\begin{pgfscope}%
\pgfpathrectangle{\pgfqpoint{5.562518in}{4.237239in}}{\pgfqpoint{4.376471in}{0.972632in}}%
\pgfusepath{clip}%
\pgfsetbuttcap%
\pgfsetroundjoin%
\definecolor{currentfill}{rgb}{1.000000,0.000000,0.000000}%
\pgfsetfillcolor{currentfill}%
\pgfsetlinewidth{2.007500pt}%
\definecolor{currentstroke}{rgb}{1.000000,0.000000,0.000000}%
\pgfsetstrokecolor{currentstroke}%
\pgfsetdash{}{0pt}%
\pgfpathmoveto{\pgfqpoint{8.958985in}{4.426269in}}%
\pgfpathlineto{\pgfqpoint{9.042318in}{4.426269in}}%
\pgfpathmoveto{\pgfqpoint{9.000652in}{4.384603in}}%
\pgfpathlineto{\pgfqpoint{9.000652in}{4.467936in}}%
\pgfusepath{stroke,fill}%
\end{pgfscope}%
\begin{pgfscope}%
\pgfpathrectangle{\pgfqpoint{5.562518in}{4.237239in}}{\pgfqpoint{4.376471in}{0.972632in}}%
\pgfusepath{clip}%
\pgfsetbuttcap%
\pgfsetroundjoin%
\definecolor{currentfill}{rgb}{1.000000,0.000000,0.000000}%
\pgfsetfillcolor{currentfill}%
\pgfsetlinewidth{2.007500pt}%
\definecolor{currentstroke}{rgb}{1.000000,0.000000,0.000000}%
\pgfsetstrokecolor{currentstroke}%
\pgfsetdash{}{0pt}%
\pgfpathmoveto{\pgfqpoint{8.492154in}{4.774649in}}%
\pgfpathlineto{\pgfqpoint{8.575487in}{4.774649in}}%
\pgfpathmoveto{\pgfqpoint{8.533821in}{4.732982in}}%
\pgfpathlineto{\pgfqpoint{8.533821in}{4.816316in}}%
\pgfusepath{stroke,fill}%
\end{pgfscope}%
\begin{pgfscope}%
\pgfpathrectangle{\pgfqpoint{5.562518in}{4.237239in}}{\pgfqpoint{4.376471in}{0.972632in}}%
\pgfusepath{clip}%
\pgfsetbuttcap%
\pgfsetroundjoin%
\definecolor{currentfill}{rgb}{1.000000,0.000000,0.000000}%
\pgfsetfillcolor{currentfill}%
\pgfsetlinewidth{2.007500pt}%
\definecolor{currentstroke}{rgb}{1.000000,0.000000,0.000000}%
\pgfsetstrokecolor{currentstroke}%
\pgfsetdash{}{0pt}%
\pgfpathmoveto{\pgfqpoint{6.942394in}{4.619003in}}%
\pgfpathlineto{\pgfqpoint{7.025727in}{4.619003in}}%
\pgfpathmoveto{\pgfqpoint{6.984061in}{4.577336in}}%
\pgfpathlineto{\pgfqpoint{6.984061in}{4.660669in}}%
\pgfusepath{stroke,fill}%
\end{pgfscope}%
\begin{pgfscope}%
\pgfpathrectangle{\pgfqpoint{5.562518in}{4.237239in}}{\pgfqpoint{4.376471in}{0.972632in}}%
\pgfusepath{clip}%
\pgfsetbuttcap%
\pgfsetroundjoin%
\definecolor{currentfill}{rgb}{1.000000,0.000000,0.000000}%
\pgfsetfillcolor{currentfill}%
\pgfsetlinewidth{2.007500pt}%
\definecolor{currentstroke}{rgb}{1.000000,0.000000,0.000000}%
\pgfsetstrokecolor{currentstroke}%
\pgfsetdash{}{0pt}%
\pgfpathmoveto{\pgfqpoint{6.942309in}{4.867576in}}%
\pgfpathlineto{\pgfqpoint{7.025643in}{4.867576in}}%
\pgfpathmoveto{\pgfqpoint{6.983976in}{4.825909in}}%
\pgfpathlineto{\pgfqpoint{6.983976in}{4.909243in}}%
\pgfusepath{stroke,fill}%
\end{pgfscope}%
\begin{pgfscope}%
\pgfpathrectangle{\pgfqpoint{5.562518in}{4.237239in}}{\pgfqpoint{4.376471in}{0.972632in}}%
\pgfusepath{clip}%
\pgfsetbuttcap%
\pgfsetroundjoin%
\definecolor{currentfill}{rgb}{1.000000,0.000000,0.000000}%
\pgfsetfillcolor{currentfill}%
\pgfsetlinewidth{2.007500pt}%
\definecolor{currentstroke}{rgb}{1.000000,0.000000,0.000000}%
\pgfsetstrokecolor{currentstroke}%
\pgfsetdash{}{0pt}%
\pgfpathmoveto{\pgfqpoint{6.599506in}{4.805046in}}%
\pgfpathlineto{\pgfqpoint{6.682839in}{4.805046in}}%
\pgfpathmoveto{\pgfqpoint{6.641173in}{4.763379in}}%
\pgfpathlineto{\pgfqpoint{6.641173in}{4.846712in}}%
\pgfusepath{stroke,fill}%
\end{pgfscope}%
\begin{pgfscope}%
\pgfpathrectangle{\pgfqpoint{5.562518in}{4.237239in}}{\pgfqpoint{4.376471in}{0.972632in}}%
\pgfusepath{clip}%
\pgfsetbuttcap%
\pgfsetroundjoin%
\definecolor{currentfill}{rgb}{1.000000,0.000000,0.000000}%
\pgfsetfillcolor{currentfill}%
\pgfsetlinewidth{2.007500pt}%
\definecolor{currentstroke}{rgb}{1.000000,0.000000,0.000000}%
\pgfsetstrokecolor{currentstroke}%
\pgfsetdash{}{0pt}%
\pgfpathmoveto{\pgfqpoint{9.428781in}{4.886956in}}%
\pgfpathlineto{\pgfqpoint{9.512114in}{4.886956in}}%
\pgfpathmoveto{\pgfqpoint{9.470447in}{4.845290in}}%
\pgfpathlineto{\pgfqpoint{9.470447in}{4.928623in}}%
\pgfusepath{stroke,fill}%
\end{pgfscope}%
\begin{pgfscope}%
\pgfpathrectangle{\pgfqpoint{5.562518in}{4.237239in}}{\pgfqpoint{4.376471in}{0.972632in}}%
\pgfusepath{clip}%
\pgfsetbuttcap%
\pgfsetroundjoin%
\definecolor{currentfill}{rgb}{1.000000,0.000000,0.000000}%
\pgfsetfillcolor{currentfill}%
\pgfsetlinewidth{2.007500pt}%
\definecolor{currentstroke}{rgb}{1.000000,0.000000,0.000000}%
\pgfsetstrokecolor{currentstroke}%
\pgfsetdash{}{0pt}%
\pgfpathmoveto{\pgfqpoint{8.500755in}{4.884213in}}%
\pgfpathlineto{\pgfqpoint{8.584088in}{4.884213in}}%
\pgfpathmoveto{\pgfqpoint{8.542421in}{4.842546in}}%
\pgfpathlineto{\pgfqpoint{8.542421in}{4.925879in}}%
\pgfusepath{stroke,fill}%
\end{pgfscope}%
\begin{pgfscope}%
\pgfpathrectangle{\pgfqpoint{5.562518in}{4.237239in}}{\pgfqpoint{4.376471in}{0.972632in}}%
\pgfusepath{clip}%
\pgfsetbuttcap%
\pgfsetroundjoin%
\definecolor{currentfill}{rgb}{1.000000,0.000000,0.000000}%
\pgfsetfillcolor{currentfill}%
\pgfsetlinewidth{2.007500pt}%
\definecolor{currentstroke}{rgb}{1.000000,0.000000,0.000000}%
\pgfsetstrokecolor{currentstroke}%
\pgfsetdash{}{0pt}%
\pgfpathmoveto{\pgfqpoint{8.875232in}{4.375818in}}%
\pgfpathlineto{\pgfqpoint{8.958565in}{4.375818in}}%
\pgfpathmoveto{\pgfqpoint{8.916899in}{4.334151in}}%
\pgfpathlineto{\pgfqpoint{8.916899in}{4.417485in}}%
\pgfusepath{stroke,fill}%
\end{pgfscope}%
\begin{pgfscope}%
\pgfpathrectangle{\pgfqpoint{5.562518in}{4.237239in}}{\pgfqpoint{4.376471in}{0.972632in}}%
\pgfusepath{clip}%
\pgfsetbuttcap%
\pgfsetroundjoin%
\definecolor{currentfill}{rgb}{1.000000,0.000000,0.000000}%
\pgfsetfillcolor{currentfill}%
\pgfsetlinewidth{2.007500pt}%
\definecolor{currentstroke}{rgb}{1.000000,0.000000,0.000000}%
\pgfsetstrokecolor{currentstroke}%
\pgfsetdash{}{0pt}%
\pgfpathmoveto{\pgfqpoint{6.468215in}{5.028707in}}%
\pgfpathlineto{\pgfqpoint{6.551548in}{5.028707in}}%
\pgfpathmoveto{\pgfqpoint{6.509882in}{4.987041in}}%
\pgfpathlineto{\pgfqpoint{6.509882in}{5.070374in}}%
\pgfusepath{stroke,fill}%
\end{pgfscope}%
\begin{pgfscope}%
\pgfpathrectangle{\pgfqpoint{5.562518in}{4.237239in}}{\pgfqpoint{4.376471in}{0.972632in}}%
\pgfusepath{clip}%
\pgfsetbuttcap%
\pgfsetroundjoin%
\definecolor{currentfill}{rgb}{1.000000,0.000000,0.000000}%
\pgfsetfillcolor{currentfill}%
\pgfsetlinewidth{2.007500pt}%
\definecolor{currentstroke}{rgb}{1.000000,0.000000,0.000000}%
\pgfsetstrokecolor{currentstroke}%
\pgfsetdash{}{0pt}%
\pgfpathmoveto{\pgfqpoint{9.791971in}{4.766769in}}%
\pgfpathlineto{\pgfqpoint{9.875304in}{4.766769in}}%
\pgfpathmoveto{\pgfqpoint{9.833637in}{4.725102in}}%
\pgfpathlineto{\pgfqpoint{9.833637in}{4.808435in}}%
\pgfusepath{stroke,fill}%
\end{pgfscope}%
\begin{pgfscope}%
\pgfpathrectangle{\pgfqpoint{5.562518in}{4.237239in}}{\pgfqpoint{4.376471in}{0.972632in}}%
\pgfusepath{clip}%
\pgfsetbuttcap%
\pgfsetroundjoin%
\definecolor{currentfill}{rgb}{1.000000,0.000000,0.000000}%
\pgfsetfillcolor{currentfill}%
\pgfsetlinewidth{2.007500pt}%
\definecolor{currentstroke}{rgb}{1.000000,0.000000,0.000000}%
\pgfsetstrokecolor{currentstroke}%
\pgfsetdash{}{0pt}%
\pgfpathmoveto{\pgfqpoint{9.310674in}{4.719021in}}%
\pgfpathlineto{\pgfqpoint{9.394007in}{4.719021in}}%
\pgfpathmoveto{\pgfqpoint{9.352340in}{4.677354in}}%
\pgfpathlineto{\pgfqpoint{9.352340in}{4.760687in}}%
\pgfusepath{stroke,fill}%
\end{pgfscope}%
\begin{pgfscope}%
\pgfpathrectangle{\pgfqpoint{5.562518in}{4.237239in}}{\pgfqpoint{4.376471in}{0.972632in}}%
\pgfusepath{clip}%
\pgfsetbuttcap%
\pgfsetroundjoin%
\definecolor{currentfill}{rgb}{1.000000,0.000000,0.000000}%
\pgfsetfillcolor{currentfill}%
\pgfsetlinewidth{2.007500pt}%
\definecolor{currentstroke}{rgb}{1.000000,0.000000,0.000000}%
\pgfsetstrokecolor{currentstroke}%
\pgfsetdash{}{0pt}%
\pgfpathmoveto{\pgfqpoint{7.139582in}{4.858492in}}%
\pgfpathlineto{\pgfqpoint{7.222915in}{4.858492in}}%
\pgfpathmoveto{\pgfqpoint{7.181248in}{4.816826in}}%
\pgfpathlineto{\pgfqpoint{7.181248in}{4.900159in}}%
\pgfusepath{stroke,fill}%
\end{pgfscope}%
\begin{pgfscope}%
\pgfpathrectangle{\pgfqpoint{5.562518in}{4.237239in}}{\pgfqpoint{4.376471in}{0.972632in}}%
\pgfusepath{clip}%
\pgfsetbuttcap%
\pgfsetroundjoin%
\definecolor{currentfill}{rgb}{1.000000,0.000000,0.000000}%
\pgfsetfillcolor{currentfill}%
\pgfsetlinewidth{2.007500pt}%
\definecolor{currentstroke}{rgb}{1.000000,0.000000,0.000000}%
\pgfsetstrokecolor{currentstroke}%
\pgfsetdash{}{0pt}%
\pgfpathmoveto{\pgfqpoint{7.032746in}{4.824645in}}%
\pgfpathlineto{\pgfqpoint{7.116080in}{4.824645in}}%
\pgfpathmoveto{\pgfqpoint{7.074413in}{4.782978in}}%
\pgfpathlineto{\pgfqpoint{7.074413in}{4.866311in}}%
\pgfusepath{stroke,fill}%
\end{pgfscope}%
\begin{pgfscope}%
\pgfpathrectangle{\pgfqpoint{5.562518in}{4.237239in}}{\pgfqpoint{4.376471in}{0.972632in}}%
\pgfusepath{clip}%
\pgfsetbuttcap%
\pgfsetroundjoin%
\definecolor{currentfill}{rgb}{1.000000,0.000000,0.000000}%
\pgfsetfillcolor{currentfill}%
\pgfsetlinewidth{2.007500pt}%
\definecolor{currentstroke}{rgb}{1.000000,0.000000,0.000000}%
\pgfsetstrokecolor{currentstroke}%
\pgfsetdash{}{0pt}%
\pgfpathmoveto{\pgfqpoint{7.038277in}{4.771951in}}%
\pgfpathlineto{\pgfqpoint{7.121610in}{4.771951in}}%
\pgfpathmoveto{\pgfqpoint{7.079943in}{4.730284in}}%
\pgfpathlineto{\pgfqpoint{7.079943in}{4.813618in}}%
\pgfusepath{stroke,fill}%
\end{pgfscope}%
\begin{pgfscope}%
\pgfpathrectangle{\pgfqpoint{5.562518in}{4.237239in}}{\pgfqpoint{4.376471in}{0.972632in}}%
\pgfusepath{clip}%
\pgfsetbuttcap%
\pgfsetroundjoin%
\definecolor{currentfill}{rgb}{1.000000,0.000000,0.000000}%
\pgfsetfillcolor{currentfill}%
\pgfsetlinewidth{2.007500pt}%
\definecolor{currentstroke}{rgb}{1.000000,0.000000,0.000000}%
\pgfsetstrokecolor{currentstroke}%
\pgfsetdash{}{0pt}%
\pgfpathmoveto{\pgfqpoint{7.461351in}{4.793955in}}%
\pgfpathlineto{\pgfqpoint{7.544684in}{4.793955in}}%
\pgfpathmoveto{\pgfqpoint{7.503018in}{4.752289in}}%
\pgfpathlineto{\pgfqpoint{7.503018in}{4.835622in}}%
\pgfusepath{stroke,fill}%
\end{pgfscope}%
\begin{pgfscope}%
\pgfpathrectangle{\pgfqpoint{5.562518in}{4.237239in}}{\pgfqpoint{4.376471in}{0.972632in}}%
\pgfusepath{clip}%
\pgfsetbuttcap%
\pgfsetroundjoin%
\definecolor{currentfill}{rgb}{1.000000,0.000000,0.000000}%
\pgfsetfillcolor{currentfill}%
\pgfsetlinewidth{2.007500pt}%
\definecolor{currentstroke}{rgb}{1.000000,0.000000,0.000000}%
\pgfsetstrokecolor{currentstroke}%
\pgfsetdash{}{0pt}%
\pgfpathmoveto{\pgfqpoint{8.233410in}{4.668742in}}%
\pgfpathlineto{\pgfqpoint{8.316743in}{4.668742in}}%
\pgfpathmoveto{\pgfqpoint{8.275077in}{4.627075in}}%
\pgfpathlineto{\pgfqpoint{8.275077in}{4.710409in}}%
\pgfusepath{stroke,fill}%
\end{pgfscope}%
\begin{pgfscope}%
\pgfpathrectangle{\pgfqpoint{5.562518in}{4.237239in}}{\pgfqpoint{4.376471in}{0.972632in}}%
\pgfusepath{clip}%
\pgfsetbuttcap%
\pgfsetroundjoin%
\definecolor{currentfill}{rgb}{1.000000,0.000000,0.000000}%
\pgfsetfillcolor{currentfill}%
\pgfsetlinewidth{2.007500pt}%
\definecolor{currentstroke}{rgb}{1.000000,0.000000,0.000000}%
\pgfsetstrokecolor{currentstroke}%
\pgfsetdash{}{0pt}%
\pgfpathmoveto{\pgfqpoint{7.908461in}{4.261493in}}%
\pgfpathlineto{\pgfqpoint{7.991794in}{4.261493in}}%
\pgfpathmoveto{\pgfqpoint{7.950127in}{4.219826in}}%
\pgfpathlineto{\pgfqpoint{7.950127in}{4.303160in}}%
\pgfusepath{stroke,fill}%
\end{pgfscope}%
\begin{pgfscope}%
\pgfpathrectangle{\pgfqpoint{5.562518in}{4.237239in}}{\pgfqpoint{4.376471in}{0.972632in}}%
\pgfusepath{clip}%
\pgfsetbuttcap%
\pgfsetroundjoin%
\definecolor{currentfill}{rgb}{1.000000,0.000000,0.000000}%
\pgfsetfillcolor{currentfill}%
\pgfsetlinewidth{2.007500pt}%
\definecolor{currentstroke}{rgb}{1.000000,0.000000,0.000000}%
\pgfsetstrokecolor{currentstroke}%
\pgfsetdash{}{0pt}%
\pgfpathmoveto{\pgfqpoint{7.415790in}{4.773812in}}%
\pgfpathlineto{\pgfqpoint{7.499123in}{4.773812in}}%
\pgfpathmoveto{\pgfqpoint{7.457456in}{4.732145in}}%
\pgfpathlineto{\pgfqpoint{7.457456in}{4.815478in}}%
\pgfusepath{stroke,fill}%
\end{pgfscope}%
\begin{pgfscope}%
\pgfpathrectangle{\pgfqpoint{5.562518in}{4.237239in}}{\pgfqpoint{4.376471in}{0.972632in}}%
\pgfusepath{clip}%
\pgfsetbuttcap%
\pgfsetroundjoin%
\definecolor{currentfill}{rgb}{1.000000,0.000000,0.000000}%
\pgfsetfillcolor{currentfill}%
\pgfsetlinewidth{2.007500pt}%
\definecolor{currentstroke}{rgb}{1.000000,0.000000,0.000000}%
\pgfsetstrokecolor{currentstroke}%
\pgfsetdash{}{0pt}%
\pgfpathmoveto{\pgfqpoint{8.538350in}{4.734883in}}%
\pgfpathlineto{\pgfqpoint{8.621683in}{4.734883in}}%
\pgfpathmoveto{\pgfqpoint{8.580017in}{4.693216in}}%
\pgfpathlineto{\pgfqpoint{8.580017in}{4.776550in}}%
\pgfusepath{stroke,fill}%
\end{pgfscope}%
\begin{pgfscope}%
\pgfpathrectangle{\pgfqpoint{5.562518in}{4.237239in}}{\pgfqpoint{4.376471in}{0.972632in}}%
\pgfusepath{clip}%
\pgfsetbuttcap%
\pgfsetroundjoin%
\definecolor{currentfill}{rgb}{1.000000,0.000000,0.000000}%
\pgfsetfillcolor{currentfill}%
\pgfsetlinewidth{2.007500pt}%
\definecolor{currentstroke}{rgb}{1.000000,0.000000,0.000000}%
\pgfsetstrokecolor{currentstroke}%
\pgfsetdash{}{0pt}%
\pgfpathmoveto{\pgfqpoint{6.884538in}{4.757697in}}%
\pgfpathlineto{\pgfqpoint{6.967871in}{4.757697in}}%
\pgfpathmoveto{\pgfqpoint{6.926204in}{4.716030in}}%
\pgfpathlineto{\pgfqpoint{6.926204in}{4.799363in}}%
\pgfusepath{stroke,fill}%
\end{pgfscope}%
\begin{pgfscope}%
\pgfpathrectangle{\pgfqpoint{5.562518in}{4.237239in}}{\pgfqpoint{4.376471in}{0.972632in}}%
\pgfusepath{clip}%
\pgfsetbuttcap%
\pgfsetroundjoin%
\definecolor{currentfill}{rgb}{1.000000,0.000000,0.000000}%
\pgfsetfillcolor{currentfill}%
\pgfsetlinewidth{2.007500pt}%
\definecolor{currentstroke}{rgb}{1.000000,0.000000,0.000000}%
\pgfsetstrokecolor{currentstroke}%
\pgfsetdash{}{0pt}%
\pgfpathmoveto{\pgfqpoint{7.418995in}{4.879012in}}%
\pgfpathlineto{\pgfqpoint{7.502328in}{4.879012in}}%
\pgfpathmoveto{\pgfqpoint{7.460662in}{4.837345in}}%
\pgfpathlineto{\pgfqpoint{7.460662in}{4.920678in}}%
\pgfusepath{stroke,fill}%
\end{pgfscope}%
\begin{pgfscope}%
\pgfpathrectangle{\pgfqpoint{5.562518in}{4.237239in}}{\pgfqpoint{4.376471in}{0.972632in}}%
\pgfusepath{clip}%
\pgfsetbuttcap%
\pgfsetroundjoin%
\definecolor{currentfill}{rgb}{1.000000,0.000000,0.000000}%
\pgfsetfillcolor{currentfill}%
\pgfsetlinewidth{2.007500pt}%
\definecolor{currentstroke}{rgb}{1.000000,0.000000,0.000000}%
\pgfsetstrokecolor{currentstroke}%
\pgfsetdash{}{0pt}%
\pgfpathmoveto{\pgfqpoint{7.678843in}{4.360745in}}%
\pgfpathlineto{\pgfqpoint{7.762176in}{4.360745in}}%
\pgfpathmoveto{\pgfqpoint{7.720509in}{4.319078in}}%
\pgfpathlineto{\pgfqpoint{7.720509in}{4.402411in}}%
\pgfusepath{stroke,fill}%
\end{pgfscope}%
\begin{pgfscope}%
\pgfpathrectangle{\pgfqpoint{5.562518in}{4.237239in}}{\pgfqpoint{4.376471in}{0.972632in}}%
\pgfusepath{clip}%
\pgfsetbuttcap%
\pgfsetroundjoin%
\definecolor{currentfill}{rgb}{1.000000,0.000000,0.000000}%
\pgfsetfillcolor{currentfill}%
\pgfsetlinewidth{2.007500pt}%
\definecolor{currentstroke}{rgb}{1.000000,0.000000,0.000000}%
\pgfsetstrokecolor{currentstroke}%
\pgfsetdash{}{0pt}%
\pgfpathmoveto{\pgfqpoint{7.992927in}{4.481944in}}%
\pgfpathlineto{\pgfqpoint{8.076260in}{4.481944in}}%
\pgfpathmoveto{\pgfqpoint{8.034593in}{4.440277in}}%
\pgfpathlineto{\pgfqpoint{8.034593in}{4.523610in}}%
\pgfusepath{stroke,fill}%
\end{pgfscope}%
\begin{pgfscope}%
\pgfpathrectangle{\pgfqpoint{5.562518in}{4.237239in}}{\pgfqpoint{4.376471in}{0.972632in}}%
\pgfusepath{clip}%
\pgfsetbuttcap%
\pgfsetroundjoin%
\definecolor{currentfill}{rgb}{1.000000,0.000000,0.000000}%
\pgfsetfillcolor{currentfill}%
\pgfsetlinewidth{2.007500pt}%
\definecolor{currentstroke}{rgb}{1.000000,0.000000,0.000000}%
\pgfsetstrokecolor{currentstroke}%
\pgfsetdash{}{0pt}%
\pgfpathmoveto{\pgfqpoint{9.145185in}{4.583803in}}%
\pgfpathlineto{\pgfqpoint{9.228518in}{4.583803in}}%
\pgfpathmoveto{\pgfqpoint{9.186851in}{4.542136in}}%
\pgfpathlineto{\pgfqpoint{9.186851in}{4.625470in}}%
\pgfusepath{stroke,fill}%
\end{pgfscope}%
\begin{pgfscope}%
\pgfpathrectangle{\pgfqpoint{5.562518in}{4.237239in}}{\pgfqpoint{4.376471in}{0.972632in}}%
\pgfusepath{clip}%
\pgfsetbuttcap%
\pgfsetroundjoin%
\definecolor{currentfill}{rgb}{1.000000,0.000000,0.000000}%
\pgfsetfillcolor{currentfill}%
\pgfsetlinewidth{2.007500pt}%
\definecolor{currentstroke}{rgb}{1.000000,0.000000,0.000000}%
\pgfsetstrokecolor{currentstroke}%
\pgfsetdash{}{0pt}%
\pgfpathmoveto{\pgfqpoint{7.095238in}{4.734953in}}%
\pgfpathlineto{\pgfqpoint{7.178572in}{4.734953in}}%
\pgfpathmoveto{\pgfqpoint{7.136905in}{4.693286in}}%
\pgfpathlineto{\pgfqpoint{7.136905in}{4.776619in}}%
\pgfusepath{stroke,fill}%
\end{pgfscope}%
\begin{pgfscope}%
\pgfpathrectangle{\pgfqpoint{5.562518in}{4.237239in}}{\pgfqpoint{4.376471in}{0.972632in}}%
\pgfusepath{clip}%
\pgfsetbuttcap%
\pgfsetroundjoin%
\definecolor{currentfill}{rgb}{1.000000,0.000000,0.000000}%
\pgfsetfillcolor{currentfill}%
\pgfsetlinewidth{2.007500pt}%
\definecolor{currentstroke}{rgb}{1.000000,0.000000,0.000000}%
\pgfsetstrokecolor{currentstroke}%
\pgfsetdash{}{0pt}%
\pgfpathmoveto{\pgfqpoint{8.196571in}{4.721187in}}%
\pgfpathlineto{\pgfqpoint{8.279904in}{4.721187in}}%
\pgfpathmoveto{\pgfqpoint{8.238237in}{4.679521in}}%
\pgfpathlineto{\pgfqpoint{8.238237in}{4.762854in}}%
\pgfusepath{stroke,fill}%
\end{pgfscope}%
\begin{pgfscope}%
\pgfpathrectangle{\pgfqpoint{5.562518in}{4.237239in}}{\pgfqpoint{4.376471in}{0.972632in}}%
\pgfusepath{clip}%
\pgfsetbuttcap%
\pgfsetroundjoin%
\definecolor{currentfill}{rgb}{1.000000,0.000000,0.000000}%
\pgfsetfillcolor{currentfill}%
\pgfsetlinewidth{2.007500pt}%
\definecolor{currentstroke}{rgb}{1.000000,0.000000,0.000000}%
\pgfsetstrokecolor{currentstroke}%
\pgfsetdash{}{0pt}%
\pgfpathmoveto{\pgfqpoint{8.470293in}{4.915156in}}%
\pgfpathlineto{\pgfqpoint{8.553626in}{4.915156in}}%
\pgfpathmoveto{\pgfqpoint{8.511960in}{4.873490in}}%
\pgfpathlineto{\pgfqpoint{8.511960in}{4.956823in}}%
\pgfusepath{stroke,fill}%
\end{pgfscope}%
\begin{pgfscope}%
\pgfpathrectangle{\pgfqpoint{5.562518in}{4.237239in}}{\pgfqpoint{4.376471in}{0.972632in}}%
\pgfusepath{clip}%
\pgfsetbuttcap%
\pgfsetroundjoin%
\definecolor{currentfill}{rgb}{1.000000,0.000000,0.000000}%
\pgfsetfillcolor{currentfill}%
\pgfsetlinewidth{2.007500pt}%
\definecolor{currentstroke}{rgb}{1.000000,0.000000,0.000000}%
\pgfsetstrokecolor{currentstroke}%
\pgfsetdash{}{0pt}%
\pgfpathmoveto{\pgfqpoint{6.558776in}{4.959589in}}%
\pgfpathlineto{\pgfqpoint{6.642110in}{4.959589in}}%
\pgfpathmoveto{\pgfqpoint{6.600443in}{4.917922in}}%
\pgfpathlineto{\pgfqpoint{6.600443in}{5.001256in}}%
\pgfusepath{stroke,fill}%
\end{pgfscope}%
\begin{pgfscope}%
\pgfpathrectangle{\pgfqpoint{5.562518in}{4.237239in}}{\pgfqpoint{4.376471in}{0.972632in}}%
\pgfusepath{clip}%
\pgfsetbuttcap%
\pgfsetroundjoin%
\definecolor{currentfill}{rgb}{0.000000,0.000000,0.000000}%
\pgfsetfillcolor{currentfill}%
\pgfsetlinewidth{1.003750pt}%
\definecolor{currentstroke}{rgb}{0.000000,0.000000,0.000000}%
\pgfsetstrokecolor{currentstroke}%
\pgfsetdash{}{0pt}%
\pgfsys@defobject{currentmarker}{\pgfqpoint{-0.020833in}{-0.020833in}}{\pgfqpoint{0.020833in}{0.020833in}}{%
\pgfpathmoveto{\pgfqpoint{0.000000in}{-0.020833in}}%
\pgfpathcurveto{\pgfqpoint{0.005525in}{-0.020833in}}{\pgfqpoint{0.010825in}{-0.018638in}}{\pgfqpoint{0.014731in}{-0.014731in}}%
\pgfpathcurveto{\pgfqpoint{0.018638in}{-0.010825in}}{\pgfqpoint{0.020833in}{-0.005525in}}{\pgfqpoint{0.020833in}{0.000000in}}%
\pgfpathcurveto{\pgfqpoint{0.020833in}{0.005525in}}{\pgfqpoint{0.018638in}{0.010825in}}{\pgfqpoint{0.014731in}{0.014731in}}%
\pgfpathcurveto{\pgfqpoint{0.010825in}{0.018638in}}{\pgfqpoint{0.005525in}{0.020833in}}{\pgfqpoint{0.000000in}{0.020833in}}%
\pgfpathcurveto{\pgfqpoint{-0.005525in}{0.020833in}}{\pgfqpoint{-0.010825in}{0.018638in}}{\pgfqpoint{-0.014731in}{0.014731in}}%
\pgfpathcurveto{\pgfqpoint{-0.018638in}{0.010825in}}{\pgfqpoint{-0.020833in}{0.005525in}}{\pgfqpoint{-0.020833in}{0.000000in}}%
\pgfpathcurveto{\pgfqpoint{-0.020833in}{-0.005525in}}{\pgfqpoint{-0.018638in}{-0.010825in}}{\pgfqpoint{-0.014731in}{-0.014731in}}%
\pgfpathcurveto{\pgfqpoint{-0.010825in}{-0.018638in}}{\pgfqpoint{-0.005525in}{-0.020833in}}{\pgfqpoint{0.000000in}{-0.020833in}}%
\pgfpathclose%
\pgfusepath{stroke,fill}%
}%
\begin{pgfscope}%
\pgfsys@transformshift{6.437812in}{5.032449in}%
\pgfsys@useobject{currentmarker}{}%
\end{pgfscope}%
\begin{pgfscope}%
\pgfsys@transformshift{6.455406in}{5.060426in}%
\pgfsys@useobject{currentmarker}{}%
\end{pgfscope}%
\begin{pgfscope}%
\pgfsys@transformshift{6.472999in}{5.098743in}%
\pgfsys@useobject{currentmarker}{}%
\end{pgfscope}%
\begin{pgfscope}%
\pgfsys@transformshift{6.490593in}{5.136781in}%
\pgfsys@useobject{currentmarker}{}%
\end{pgfscope}%
\begin{pgfscope}%
\pgfsys@transformshift{6.508187in}{4.961685in}%
\pgfsys@useobject{currentmarker}{}%
\end{pgfscope}%
\begin{pgfscope}%
\pgfsys@transformshift{6.525781in}{4.963442in}%
\pgfsys@useobject{currentmarker}{}%
\end{pgfscope}%
\begin{pgfscope}%
\pgfsys@transformshift{6.543375in}{4.842116in}%
\pgfsys@useobject{currentmarker}{}%
\end{pgfscope}%
\begin{pgfscope}%
\pgfsys@transformshift{6.560969in}{4.805102in}%
\pgfsys@useobject{currentmarker}{}%
\end{pgfscope}%
\begin{pgfscope}%
\pgfsys@transformshift{6.578563in}{4.981054in}%
\pgfsys@useobject{currentmarker}{}%
\end{pgfscope}%
\begin{pgfscope}%
\pgfsys@transformshift{6.596156in}{5.009122in}%
\pgfsys@useobject{currentmarker}{}%
\end{pgfscope}%
\begin{pgfscope}%
\pgfsys@transformshift{6.613750in}{4.838142in}%
\pgfsys@useobject{currentmarker}{}%
\end{pgfscope}%
\begin{pgfscope}%
\pgfsys@transformshift{6.631344in}{4.921769in}%
\pgfsys@useobject{currentmarker}{}%
\end{pgfscope}%
\begin{pgfscope}%
\pgfsys@transformshift{6.648938in}{4.832509in}%
\pgfsys@useobject{currentmarker}{}%
\end{pgfscope}%
\begin{pgfscope}%
\pgfsys@transformshift{6.666532in}{4.707571in}%
\pgfsys@useobject{currentmarker}{}%
\end{pgfscope}%
\begin{pgfscope}%
\pgfsys@transformshift{6.684126in}{4.788057in}%
\pgfsys@useobject{currentmarker}{}%
\end{pgfscope}%
\begin{pgfscope}%
\pgfsys@transformshift{6.701720in}{4.887373in}%
\pgfsys@useobject{currentmarker}{}%
\end{pgfscope}%
\begin{pgfscope}%
\pgfsys@transformshift{6.719313in}{4.709743in}%
\pgfsys@useobject{currentmarker}{}%
\end{pgfscope}%
\begin{pgfscope}%
\pgfsys@transformshift{6.736907in}{4.855561in}%
\pgfsys@useobject{currentmarker}{}%
\end{pgfscope}%
\begin{pgfscope}%
\pgfsys@transformshift{6.754501in}{4.417217in}%
\pgfsys@useobject{currentmarker}{}%
\end{pgfscope}%
\begin{pgfscope}%
\pgfsys@transformshift{6.772095in}{4.753521in}%
\pgfsys@useobject{currentmarker}{}%
\end{pgfscope}%
\begin{pgfscope}%
\pgfsys@transformshift{6.789689in}{4.668751in}%
\pgfsys@useobject{currentmarker}{}%
\end{pgfscope}%
\begin{pgfscope}%
\pgfsys@transformshift{6.807283in}{4.621427in}%
\pgfsys@useobject{currentmarker}{}%
\end{pgfscope}%
\begin{pgfscope}%
\pgfsys@transformshift{6.824876in}{4.654933in}%
\pgfsys@useobject{currentmarker}{}%
\end{pgfscope}%
\begin{pgfscope}%
\pgfsys@transformshift{6.842470in}{4.440302in}%
\pgfsys@useobject{currentmarker}{}%
\end{pgfscope}%
\begin{pgfscope}%
\pgfsys@transformshift{6.860064in}{4.617554in}%
\pgfsys@useobject{currentmarker}{}%
\end{pgfscope}%
\begin{pgfscope}%
\pgfsys@transformshift{6.877658in}{4.676183in}%
\pgfsys@useobject{currentmarker}{}%
\end{pgfscope}%
\begin{pgfscope}%
\pgfsys@transformshift{6.895252in}{4.791919in}%
\pgfsys@useobject{currentmarker}{}%
\end{pgfscope}%
\begin{pgfscope}%
\pgfsys@transformshift{6.912846in}{4.593766in}%
\pgfsys@useobject{currentmarker}{}%
\end{pgfscope}%
\begin{pgfscope}%
\pgfsys@transformshift{6.930440in}{4.570259in}%
\pgfsys@useobject{currentmarker}{}%
\end{pgfscope}%
\begin{pgfscope}%
\pgfsys@transformshift{6.948033in}{4.608920in}%
\pgfsys@useobject{currentmarker}{}%
\end{pgfscope}%
\begin{pgfscope}%
\pgfsys@transformshift{6.965627in}{4.761637in}%
\pgfsys@useobject{currentmarker}{}%
\end{pgfscope}%
\begin{pgfscope}%
\pgfsys@transformshift{6.983221in}{4.712738in}%
\pgfsys@useobject{currentmarker}{}%
\end{pgfscope}%
\begin{pgfscope}%
\pgfsys@transformshift{7.000815in}{4.637534in}%
\pgfsys@useobject{currentmarker}{}%
\end{pgfscope}%
\begin{pgfscope}%
\pgfsys@transformshift{7.018409in}{4.756049in}%
\pgfsys@useobject{currentmarker}{}%
\end{pgfscope}%
\begin{pgfscope}%
\pgfsys@transformshift{7.036003in}{4.727602in}%
\pgfsys@useobject{currentmarker}{}%
\end{pgfscope}%
\begin{pgfscope}%
\pgfsys@transformshift{7.053597in}{4.830310in}%
\pgfsys@useobject{currentmarker}{}%
\end{pgfscope}%
\begin{pgfscope}%
\pgfsys@transformshift{7.071190in}{4.675933in}%
\pgfsys@useobject{currentmarker}{}%
\end{pgfscope}%
\begin{pgfscope}%
\pgfsys@transformshift{7.088784in}{4.729039in}%
\pgfsys@useobject{currentmarker}{}%
\end{pgfscope}%
\begin{pgfscope}%
\pgfsys@transformshift{7.106378in}{4.737763in}%
\pgfsys@useobject{currentmarker}{}%
\end{pgfscope}%
\begin{pgfscope}%
\pgfsys@transformshift{7.123972in}{4.644339in}%
\pgfsys@useobject{currentmarker}{}%
\end{pgfscope}%
\begin{pgfscope}%
\pgfsys@transformshift{7.141566in}{4.837414in}%
\pgfsys@useobject{currentmarker}{}%
\end{pgfscope}%
\begin{pgfscope}%
\pgfsys@transformshift{7.159160in}{4.848128in}%
\pgfsys@useobject{currentmarker}{}%
\end{pgfscope}%
\begin{pgfscope}%
\pgfsys@transformshift{7.176754in}{4.835743in}%
\pgfsys@useobject{currentmarker}{}%
\end{pgfscope}%
\begin{pgfscope}%
\pgfsys@transformshift{7.194347in}{4.824096in}%
\pgfsys@useobject{currentmarker}{}%
\end{pgfscope}%
\begin{pgfscope}%
\pgfsys@transformshift{7.211941in}{4.716023in}%
\pgfsys@useobject{currentmarker}{}%
\end{pgfscope}%
\begin{pgfscope}%
\pgfsys@transformshift{7.229535in}{4.827119in}%
\pgfsys@useobject{currentmarker}{}%
\end{pgfscope}%
\begin{pgfscope}%
\pgfsys@transformshift{7.247129in}{4.843936in}%
\pgfsys@useobject{currentmarker}{}%
\end{pgfscope}%
\begin{pgfscope}%
\pgfsys@transformshift{7.264723in}{4.804768in}%
\pgfsys@useobject{currentmarker}{}%
\end{pgfscope}%
\begin{pgfscope}%
\pgfsys@transformshift{7.282317in}{4.875457in}%
\pgfsys@useobject{currentmarker}{}%
\end{pgfscope}%
\begin{pgfscope}%
\pgfsys@transformshift{7.299910in}{4.936734in}%
\pgfsys@useobject{currentmarker}{}%
\end{pgfscope}%
\begin{pgfscope}%
\pgfsys@transformshift{7.317504in}{5.089067in}%
\pgfsys@useobject{currentmarker}{}%
\end{pgfscope}%
\begin{pgfscope}%
\pgfsys@transformshift{7.335098in}{4.915931in}%
\pgfsys@useobject{currentmarker}{}%
\end{pgfscope}%
\begin{pgfscope}%
\pgfsys@transformshift{7.352692in}{4.922677in}%
\pgfsys@useobject{currentmarker}{}%
\end{pgfscope}%
\begin{pgfscope}%
\pgfsys@transformshift{7.370286in}{4.885421in}%
\pgfsys@useobject{currentmarker}{}%
\end{pgfscope}%
\begin{pgfscope}%
\pgfsys@transformshift{7.387880in}{4.692981in}%
\pgfsys@useobject{currentmarker}{}%
\end{pgfscope}%
\begin{pgfscope}%
\pgfsys@transformshift{7.405474in}{4.877173in}%
\pgfsys@useobject{currentmarker}{}%
\end{pgfscope}%
\begin{pgfscope}%
\pgfsys@transformshift{7.423067in}{4.876532in}%
\pgfsys@useobject{currentmarker}{}%
\end{pgfscope}%
\begin{pgfscope}%
\pgfsys@transformshift{7.440661in}{5.108721in}%
\pgfsys@useobject{currentmarker}{}%
\end{pgfscope}%
\begin{pgfscope}%
\pgfsys@transformshift{7.458255in}{4.826625in}%
\pgfsys@useobject{currentmarker}{}%
\end{pgfscope}%
\begin{pgfscope}%
\pgfsys@transformshift{7.475849in}{4.861953in}%
\pgfsys@useobject{currentmarker}{}%
\end{pgfscope}%
\begin{pgfscope}%
\pgfsys@transformshift{7.493443in}{4.811617in}%
\pgfsys@useobject{currentmarker}{}%
\end{pgfscope}%
\begin{pgfscope}%
\pgfsys@transformshift{7.511037in}{4.679035in}%
\pgfsys@useobject{currentmarker}{}%
\end{pgfscope}%
\begin{pgfscope}%
\pgfsys@transformshift{7.528631in}{4.894255in}%
\pgfsys@useobject{currentmarker}{}%
\end{pgfscope}%
\begin{pgfscope}%
\pgfsys@transformshift{7.546224in}{4.834562in}%
\pgfsys@useobject{currentmarker}{}%
\end{pgfscope}%
\begin{pgfscope}%
\pgfsys@transformshift{7.563818in}{4.817487in}%
\pgfsys@useobject{currentmarker}{}%
\end{pgfscope}%
\begin{pgfscope}%
\pgfsys@transformshift{7.581412in}{4.623408in}%
\pgfsys@useobject{currentmarker}{}%
\end{pgfscope}%
\begin{pgfscope}%
\pgfsys@transformshift{7.599006in}{4.835296in}%
\pgfsys@useobject{currentmarker}{}%
\end{pgfscope}%
\begin{pgfscope}%
\pgfsys@transformshift{7.616600in}{4.528394in}%
\pgfsys@useobject{currentmarker}{}%
\end{pgfscope}%
\begin{pgfscope}%
\pgfsys@transformshift{7.634194in}{4.706961in}%
\pgfsys@useobject{currentmarker}{}%
\end{pgfscope}%
\begin{pgfscope}%
\pgfsys@transformshift{7.651788in}{4.846544in}%
\pgfsys@useobject{currentmarker}{}%
\end{pgfscope}%
\begin{pgfscope}%
\pgfsys@transformshift{7.669381in}{4.501609in}%
\pgfsys@useobject{currentmarker}{}%
\end{pgfscope}%
\begin{pgfscope}%
\pgfsys@transformshift{7.686975in}{4.522383in}%
\pgfsys@useobject{currentmarker}{}%
\end{pgfscope}%
\begin{pgfscope}%
\pgfsys@transformshift{7.704569in}{4.568288in}%
\pgfsys@useobject{currentmarker}{}%
\end{pgfscope}%
\begin{pgfscope}%
\pgfsys@transformshift{7.722163in}{4.486451in}%
\pgfsys@useobject{currentmarker}{}%
\end{pgfscope}%
\begin{pgfscope}%
\pgfsys@transformshift{7.739757in}{4.360647in}%
\pgfsys@useobject{currentmarker}{}%
\end{pgfscope}%
\begin{pgfscope}%
\pgfsys@transformshift{7.757351in}{4.506193in}%
\pgfsys@useobject{currentmarker}{}%
\end{pgfscope}%
\begin{pgfscope}%
\pgfsys@transformshift{7.774944in}{4.374478in}%
\pgfsys@useobject{currentmarker}{}%
\end{pgfscope}%
\begin{pgfscope}%
\pgfsys@transformshift{7.792538in}{4.514468in}%
\pgfsys@useobject{currentmarker}{}%
\end{pgfscope}%
\begin{pgfscope}%
\pgfsys@transformshift{7.810132in}{4.359371in}%
\pgfsys@useobject{currentmarker}{}%
\end{pgfscope}%
\begin{pgfscope}%
\pgfsys@transformshift{7.827726in}{4.597375in}%
\pgfsys@useobject{currentmarker}{}%
\end{pgfscope}%
\begin{pgfscope}%
\pgfsys@transformshift{7.845320in}{4.350692in}%
\pgfsys@useobject{currentmarker}{}%
\end{pgfscope}%
\begin{pgfscope}%
\pgfsys@transformshift{7.862914in}{4.389098in}%
\pgfsys@useobject{currentmarker}{}%
\end{pgfscope}%
\begin{pgfscope}%
\pgfsys@transformshift{7.880508in}{4.497873in}%
\pgfsys@useobject{currentmarker}{}%
\end{pgfscope}%
\begin{pgfscope}%
\pgfsys@transformshift{7.898101in}{4.286558in}%
\pgfsys@useobject{currentmarker}{}%
\end{pgfscope}%
\begin{pgfscope}%
\pgfsys@transformshift{7.915695in}{4.432241in}%
\pgfsys@useobject{currentmarker}{}%
\end{pgfscope}%
\begin{pgfscope}%
\pgfsys@transformshift{7.933289in}{4.541689in}%
\pgfsys@useobject{currentmarker}{}%
\end{pgfscope}%
\begin{pgfscope}%
\pgfsys@transformshift{7.950883in}{4.248564in}%
\pgfsys@useobject{currentmarker}{}%
\end{pgfscope}%
\begin{pgfscope}%
\pgfsys@transformshift{7.968477in}{4.434386in}%
\pgfsys@useobject{currentmarker}{}%
\end{pgfscope}%
\begin{pgfscope}%
\pgfsys@transformshift{7.986071in}{4.448289in}%
\pgfsys@useobject{currentmarker}{}%
\end{pgfscope}%
\begin{pgfscope}%
\pgfsys@transformshift{8.003665in}{4.509403in}%
\pgfsys@useobject{currentmarker}{}%
\end{pgfscope}%
\begin{pgfscope}%
\pgfsys@transformshift{8.021258in}{4.314963in}%
\pgfsys@useobject{currentmarker}{}%
\end{pgfscope}%
\begin{pgfscope}%
\pgfsys@transformshift{8.038852in}{4.318347in}%
\pgfsys@useobject{currentmarker}{}%
\end{pgfscope}%
\begin{pgfscope}%
\pgfsys@transformshift{8.056446in}{4.518480in}%
\pgfsys@useobject{currentmarker}{}%
\end{pgfscope}%
\begin{pgfscope}%
\pgfsys@transformshift{8.074040in}{4.510638in}%
\pgfsys@useobject{currentmarker}{}%
\end{pgfscope}%
\begin{pgfscope}%
\pgfsys@transformshift{8.091634in}{4.522203in}%
\pgfsys@useobject{currentmarker}{}%
\end{pgfscope}%
\begin{pgfscope}%
\pgfsys@transformshift{8.109228in}{4.549355in}%
\pgfsys@useobject{currentmarker}{}%
\end{pgfscope}%
\begin{pgfscope}%
\pgfsys@transformshift{8.126822in}{4.463763in}%
\pgfsys@useobject{currentmarker}{}%
\end{pgfscope}%
\begin{pgfscope}%
\pgfsys@transformshift{8.144415in}{4.575384in}%
\pgfsys@useobject{currentmarker}{}%
\end{pgfscope}%
\begin{pgfscope}%
\pgfsys@transformshift{8.162009in}{4.601329in}%
\pgfsys@useobject{currentmarker}{}%
\end{pgfscope}%
\begin{pgfscope}%
\pgfsys@transformshift{8.179603in}{4.519432in}%
\pgfsys@useobject{currentmarker}{}%
\end{pgfscope}%
\begin{pgfscope}%
\pgfsys@transformshift{8.197197in}{4.801196in}%
\pgfsys@useobject{currentmarker}{}%
\end{pgfscope}%
\begin{pgfscope}%
\pgfsys@transformshift{8.214791in}{4.680506in}%
\pgfsys@useobject{currentmarker}{}%
\end{pgfscope}%
\begin{pgfscope}%
\pgfsys@transformshift{8.232385in}{4.531912in}%
\pgfsys@useobject{currentmarker}{}%
\end{pgfscope}%
\begin{pgfscope}%
\pgfsys@transformshift{8.249978in}{4.738813in}%
\pgfsys@useobject{currentmarker}{}%
\end{pgfscope}%
\begin{pgfscope}%
\pgfsys@transformshift{8.267572in}{4.592604in}%
\pgfsys@useobject{currentmarker}{}%
\end{pgfscope}%
\begin{pgfscope}%
\pgfsys@transformshift{8.285166in}{4.789348in}%
\pgfsys@useobject{currentmarker}{}%
\end{pgfscope}%
\begin{pgfscope}%
\pgfsys@transformshift{8.302760in}{4.844245in}%
\pgfsys@useobject{currentmarker}{}%
\end{pgfscope}%
\begin{pgfscope}%
\pgfsys@transformshift{8.320354in}{4.659809in}%
\pgfsys@useobject{currentmarker}{}%
\end{pgfscope}%
\begin{pgfscope}%
\pgfsys@transformshift{8.337948in}{4.855341in}%
\pgfsys@useobject{currentmarker}{}%
\end{pgfscope}%
\begin{pgfscope}%
\pgfsys@transformshift{8.355542in}{4.812876in}%
\pgfsys@useobject{currentmarker}{}%
\end{pgfscope}%
\begin{pgfscope}%
\pgfsys@transformshift{8.373135in}{4.866073in}%
\pgfsys@useobject{currentmarker}{}%
\end{pgfscope}%
\begin{pgfscope}%
\pgfsys@transformshift{8.390729in}{4.984994in}%
\pgfsys@useobject{currentmarker}{}%
\end{pgfscope}%
\begin{pgfscope}%
\pgfsys@transformshift{8.408323in}{4.776202in}%
\pgfsys@useobject{currentmarker}{}%
\end{pgfscope}%
\begin{pgfscope}%
\pgfsys@transformshift{8.425917in}{4.731079in}%
\pgfsys@useobject{currentmarker}{}%
\end{pgfscope}%
\begin{pgfscope}%
\pgfsys@transformshift{8.443511in}{4.721788in}%
\pgfsys@useobject{currentmarker}{}%
\end{pgfscope}%
\begin{pgfscope}%
\pgfsys@transformshift{8.461105in}{4.731770in}%
\pgfsys@useobject{currentmarker}{}%
\end{pgfscope}%
\begin{pgfscope}%
\pgfsys@transformshift{8.478699in}{4.807165in}%
\pgfsys@useobject{currentmarker}{}%
\end{pgfscope}%
\begin{pgfscope}%
\pgfsys@transformshift{8.496292in}{4.848137in}%
\pgfsys@useobject{currentmarker}{}%
\end{pgfscope}%
\begin{pgfscope}%
\pgfsys@transformshift{8.513886in}{4.838276in}%
\pgfsys@useobject{currentmarker}{}%
\end{pgfscope}%
\begin{pgfscope}%
\pgfsys@transformshift{8.531480in}{4.888840in}%
\pgfsys@useobject{currentmarker}{}%
\end{pgfscope}%
\begin{pgfscope}%
\pgfsys@transformshift{8.549074in}{4.799331in}%
\pgfsys@useobject{currentmarker}{}%
\end{pgfscope}%
\begin{pgfscope}%
\pgfsys@transformshift{8.566668in}{4.936537in}%
\pgfsys@useobject{currentmarker}{}%
\end{pgfscope}%
\begin{pgfscope}%
\pgfsys@transformshift{8.584262in}{4.752098in}%
\pgfsys@useobject{currentmarker}{}%
\end{pgfscope}%
\begin{pgfscope}%
\pgfsys@transformshift{8.601855in}{5.042653in}%
\pgfsys@useobject{currentmarker}{}%
\end{pgfscope}%
\begin{pgfscope}%
\pgfsys@transformshift{8.619449in}{4.817237in}%
\pgfsys@useobject{currentmarker}{}%
\end{pgfscope}%
\begin{pgfscope}%
\pgfsys@transformshift{8.637043in}{4.652590in}%
\pgfsys@useobject{currentmarker}{}%
\end{pgfscope}%
\begin{pgfscope}%
\pgfsys@transformshift{8.654637in}{4.615485in}%
\pgfsys@useobject{currentmarker}{}%
\end{pgfscope}%
\begin{pgfscope}%
\pgfsys@transformshift{8.672231in}{4.756555in}%
\pgfsys@useobject{currentmarker}{}%
\end{pgfscope}%
\begin{pgfscope}%
\pgfsys@transformshift{8.689825in}{4.668051in}%
\pgfsys@useobject{currentmarker}{}%
\end{pgfscope}%
\begin{pgfscope}%
\pgfsys@transformshift{8.707419in}{4.745571in}%
\pgfsys@useobject{currentmarker}{}%
\end{pgfscope}%
\begin{pgfscope}%
\pgfsys@transformshift{8.725012in}{4.703440in}%
\pgfsys@useobject{currentmarker}{}%
\end{pgfscope}%
\begin{pgfscope}%
\pgfsys@transformshift{8.742606in}{4.630304in}%
\pgfsys@useobject{currentmarker}{}%
\end{pgfscope}%
\begin{pgfscope}%
\pgfsys@transformshift{8.760200in}{4.534209in}%
\pgfsys@useobject{currentmarker}{}%
\end{pgfscope}%
\begin{pgfscope}%
\pgfsys@transformshift{8.777794in}{4.449180in}%
\pgfsys@useobject{currentmarker}{}%
\end{pgfscope}%
\begin{pgfscope}%
\pgfsys@transformshift{8.795388in}{4.540614in}%
\pgfsys@useobject{currentmarker}{}%
\end{pgfscope}%
\begin{pgfscope}%
\pgfsys@transformshift{8.812982in}{4.656551in}%
\pgfsys@useobject{currentmarker}{}%
\end{pgfscope}%
\begin{pgfscope}%
\pgfsys@transformshift{8.830576in}{4.576334in}%
\pgfsys@useobject{currentmarker}{}%
\end{pgfscope}%
\begin{pgfscope}%
\pgfsys@transformshift{8.848169in}{4.414398in}%
\pgfsys@useobject{currentmarker}{}%
\end{pgfscope}%
\begin{pgfscope}%
\pgfsys@transformshift{8.865763in}{4.545407in}%
\pgfsys@useobject{currentmarker}{}%
\end{pgfscope}%
\begin{pgfscope}%
\pgfsys@transformshift{8.883357in}{4.555593in}%
\pgfsys@useobject{currentmarker}{}%
\end{pgfscope}%
\begin{pgfscope}%
\pgfsys@transformshift{8.900951in}{4.417287in}%
\pgfsys@useobject{currentmarker}{}%
\end{pgfscope}%
\begin{pgfscope}%
\pgfsys@transformshift{8.918545in}{4.514411in}%
\pgfsys@useobject{currentmarker}{}%
\end{pgfscope}%
\begin{pgfscope}%
\pgfsys@transformshift{8.936139in}{4.498569in}%
\pgfsys@useobject{currentmarker}{}%
\end{pgfscope}%
\begin{pgfscope}%
\pgfsys@transformshift{8.953733in}{4.372637in}%
\pgfsys@useobject{currentmarker}{}%
\end{pgfscope}%
\begin{pgfscope}%
\pgfsys@transformshift{8.971326in}{4.522463in}%
\pgfsys@useobject{currentmarker}{}%
\end{pgfscope}%
\begin{pgfscope}%
\pgfsys@transformshift{8.988920in}{4.542874in}%
\pgfsys@useobject{currentmarker}{}%
\end{pgfscope}%
\begin{pgfscope}%
\pgfsys@transformshift{9.006514in}{4.597739in}%
\pgfsys@useobject{currentmarker}{}%
\end{pgfscope}%
\begin{pgfscope}%
\pgfsys@transformshift{9.024108in}{4.598852in}%
\pgfsys@useobject{currentmarker}{}%
\end{pgfscope}%
\begin{pgfscope}%
\pgfsys@transformshift{9.041702in}{4.358705in}%
\pgfsys@useobject{currentmarker}{}%
\end{pgfscope}%
\begin{pgfscope}%
\pgfsys@transformshift{9.059296in}{4.411568in}%
\pgfsys@useobject{currentmarker}{}%
\end{pgfscope}%
\begin{pgfscope}%
\pgfsys@transformshift{9.076889in}{4.569122in}%
\pgfsys@useobject{currentmarker}{}%
\end{pgfscope}%
\begin{pgfscope}%
\pgfsys@transformshift{9.094483in}{4.581346in}%
\pgfsys@useobject{currentmarker}{}%
\end{pgfscope}%
\begin{pgfscope}%
\pgfsys@transformshift{9.112077in}{4.595735in}%
\pgfsys@useobject{currentmarker}{}%
\end{pgfscope}%
\begin{pgfscope}%
\pgfsys@transformshift{9.129671in}{4.949966in}%
\pgfsys@useobject{currentmarker}{}%
\end{pgfscope}%
\begin{pgfscope}%
\pgfsys@transformshift{9.147265in}{4.635227in}%
\pgfsys@useobject{currentmarker}{}%
\end{pgfscope}%
\begin{pgfscope}%
\pgfsys@transformshift{9.164859in}{4.711757in}%
\pgfsys@useobject{currentmarker}{}%
\end{pgfscope}%
\begin{pgfscope}%
\pgfsys@transformshift{9.182453in}{4.714090in}%
\pgfsys@useobject{currentmarker}{}%
\end{pgfscope}%
\begin{pgfscope}%
\pgfsys@transformshift{9.200046in}{4.705403in}%
\pgfsys@useobject{currentmarker}{}%
\end{pgfscope}%
\begin{pgfscope}%
\pgfsys@transformshift{9.217640in}{4.630507in}%
\pgfsys@useobject{currentmarker}{}%
\end{pgfscope}%
\begin{pgfscope}%
\pgfsys@transformshift{9.235234in}{4.763272in}%
\pgfsys@useobject{currentmarker}{}%
\end{pgfscope}%
\begin{pgfscope}%
\pgfsys@transformshift{9.252828in}{4.632698in}%
\pgfsys@useobject{currentmarker}{}%
\end{pgfscope}%
\begin{pgfscope}%
\pgfsys@transformshift{9.270422in}{4.712118in}%
\pgfsys@useobject{currentmarker}{}%
\end{pgfscope}%
\begin{pgfscope}%
\pgfsys@transformshift{9.288016in}{4.712341in}%
\pgfsys@useobject{currentmarker}{}%
\end{pgfscope}%
\begin{pgfscope}%
\pgfsys@transformshift{9.305610in}{4.795302in}%
\pgfsys@useobject{currentmarker}{}%
\end{pgfscope}%
\begin{pgfscope}%
\pgfsys@transformshift{9.323203in}{5.046888in}%
\pgfsys@useobject{currentmarker}{}%
\end{pgfscope}%
\begin{pgfscope}%
\pgfsys@transformshift{9.340797in}{4.648241in}%
\pgfsys@useobject{currentmarker}{}%
\end{pgfscope}%
\begin{pgfscope}%
\pgfsys@transformshift{9.358391in}{4.931479in}%
\pgfsys@useobject{currentmarker}{}%
\end{pgfscope}%
\begin{pgfscope}%
\pgfsys@transformshift{9.375985in}{4.722368in}%
\pgfsys@useobject{currentmarker}{}%
\end{pgfscope}%
\begin{pgfscope}%
\pgfsys@transformshift{9.393579in}{4.860858in}%
\pgfsys@useobject{currentmarker}{}%
\end{pgfscope}%
\begin{pgfscope}%
\pgfsys@transformshift{9.411173in}{5.040836in}%
\pgfsys@useobject{currentmarker}{}%
\end{pgfscope}%
\begin{pgfscope}%
\pgfsys@transformshift{9.428767in}{4.957619in}%
\pgfsys@useobject{currentmarker}{}%
\end{pgfscope}%
\begin{pgfscope}%
\pgfsys@transformshift{9.446360in}{4.861123in}%
\pgfsys@useobject{currentmarker}{}%
\end{pgfscope}%
\begin{pgfscope}%
\pgfsys@transformshift{9.463954in}{4.915530in}%
\pgfsys@useobject{currentmarker}{}%
\end{pgfscope}%
\begin{pgfscope}%
\pgfsys@transformshift{9.481548in}{5.072899in}%
\pgfsys@useobject{currentmarker}{}%
\end{pgfscope}%
\begin{pgfscope}%
\pgfsys@transformshift{9.499142in}{4.944346in}%
\pgfsys@useobject{currentmarker}{}%
\end{pgfscope}%
\begin{pgfscope}%
\pgfsys@transformshift{9.516736in}{5.052747in}%
\pgfsys@useobject{currentmarker}{}%
\end{pgfscope}%
\begin{pgfscope}%
\pgfsys@transformshift{9.534330in}{5.046014in}%
\pgfsys@useobject{currentmarker}{}%
\end{pgfscope}%
\begin{pgfscope}%
\pgfsys@transformshift{9.551923in}{4.984025in}%
\pgfsys@useobject{currentmarker}{}%
\end{pgfscope}%
\begin{pgfscope}%
\pgfsys@transformshift{9.569517in}{5.273945in}%
\pgfsys@useobject{currentmarker}{}%
\end{pgfscope}%
\begin{pgfscope}%
\pgfsys@transformshift{9.587111in}{5.125680in}%
\pgfsys@useobject{currentmarker}{}%
\end{pgfscope}%
\begin{pgfscope}%
\pgfsys@transformshift{9.604705in}{4.859056in}%
\pgfsys@useobject{currentmarker}{}%
\end{pgfscope}%
\begin{pgfscope}%
\pgfsys@transformshift{9.622299in}{5.084000in}%
\pgfsys@useobject{currentmarker}{}%
\end{pgfscope}%
\begin{pgfscope}%
\pgfsys@transformshift{9.639893in}{4.997086in}%
\pgfsys@useobject{currentmarker}{}%
\end{pgfscope}%
\begin{pgfscope}%
\pgfsys@transformshift{9.657487in}{5.147756in}%
\pgfsys@useobject{currentmarker}{}%
\end{pgfscope}%
\begin{pgfscope}%
\pgfsys@transformshift{9.675080in}{4.976678in}%
\pgfsys@useobject{currentmarker}{}%
\end{pgfscope}%
\begin{pgfscope}%
\pgfsys@transformshift{9.692674in}{5.039369in}%
\pgfsys@useobject{currentmarker}{}%
\end{pgfscope}%
\begin{pgfscope}%
\pgfsys@transformshift{9.710268in}{5.094749in}%
\pgfsys@useobject{currentmarker}{}%
\end{pgfscope}%
\begin{pgfscope}%
\pgfsys@transformshift{9.727862in}{5.122608in}%
\pgfsys@useobject{currentmarker}{}%
\end{pgfscope}%
\begin{pgfscope}%
\pgfsys@transformshift{9.745456in}{4.903467in}%
\pgfsys@useobject{currentmarker}{}%
\end{pgfscope}%
\begin{pgfscope}%
\pgfsys@transformshift{9.763050in}{4.980417in}%
\pgfsys@useobject{currentmarker}{}%
\end{pgfscope}%
\begin{pgfscope}%
\pgfsys@transformshift{9.780644in}{4.954648in}%
\pgfsys@useobject{currentmarker}{}%
\end{pgfscope}%
\begin{pgfscope}%
\pgfsys@transformshift{9.798237in}{4.924456in}%
\pgfsys@useobject{currentmarker}{}%
\end{pgfscope}%
\begin{pgfscope}%
\pgfsys@transformshift{9.815831in}{5.157016in}%
\pgfsys@useobject{currentmarker}{}%
\end{pgfscope}%
\begin{pgfscope}%
\pgfsys@transformshift{9.833425in}{5.006498in}%
\pgfsys@useobject{currentmarker}{}%
\end{pgfscope}%
\begin{pgfscope}%
\pgfsys@transformshift{9.851019in}{4.825064in}%
\pgfsys@useobject{currentmarker}{}%
\end{pgfscope}%
\begin{pgfscope}%
\pgfsys@transformshift{9.868613in}{5.033381in}%
\pgfsys@useobject{currentmarker}{}%
\end{pgfscope}%
\begin{pgfscope}%
\pgfsys@transformshift{9.886207in}{5.143404in}%
\pgfsys@useobject{currentmarker}{}%
\end{pgfscope}%
\begin{pgfscope}%
\pgfsys@transformshift{9.903801in}{5.021645in}%
\pgfsys@useobject{currentmarker}{}%
\end{pgfscope}%
\begin{pgfscope}%
\pgfsys@transformshift{9.921394in}{4.752577in}%
\pgfsys@useobject{currentmarker}{}%
\end{pgfscope}%
\begin{pgfscope}%
\pgfsys@transformshift{9.938988in}{4.847942in}%
\pgfsys@useobject{currentmarker}{}%
\end{pgfscope}%
\end{pgfscope}%
\begin{pgfscope}%
\pgfsetbuttcap%
\pgfsetroundjoin%
\definecolor{currentfill}{rgb}{0.000000,0.000000,0.000000}%
\pgfsetfillcolor{currentfill}%
\pgfsetlinewidth{0.803000pt}%
\definecolor{currentstroke}{rgb}{0.000000,0.000000,0.000000}%
\pgfsetstrokecolor{currentstroke}%
\pgfsetdash{}{0pt}%
\pgfsys@defobject{currentmarker}{\pgfqpoint{0.000000in}{-0.048611in}}{\pgfqpoint{0.000000in}{0.000000in}}{%
\pgfpathmoveto{\pgfqpoint{0.000000in}{0.000000in}}%
\pgfpathlineto{\pgfqpoint{0.000000in}{-0.048611in}}%
\pgfusepath{stroke,fill}%
}%
\begin{pgfscope}%
\pgfsys@transformshift{5.562518in}{4.237239in}%
\pgfsys@useobject{currentmarker}{}%
\end{pgfscope}%
\end{pgfscope}%
\begin{pgfscope}%
\pgfsetbuttcap%
\pgfsetroundjoin%
\definecolor{currentfill}{rgb}{0.000000,0.000000,0.000000}%
\pgfsetfillcolor{currentfill}%
\pgfsetlinewidth{0.803000pt}%
\definecolor{currentstroke}{rgb}{0.000000,0.000000,0.000000}%
\pgfsetstrokecolor{currentstroke}%
\pgfsetdash{}{0pt}%
\pgfsys@defobject{currentmarker}{\pgfqpoint{0.000000in}{-0.048611in}}{\pgfqpoint{0.000000in}{0.000000in}}{%
\pgfpathmoveto{\pgfqpoint{0.000000in}{0.000000in}}%
\pgfpathlineto{\pgfqpoint{0.000000in}{-0.048611in}}%
\pgfusepath{stroke,fill}%
}%
\begin{pgfscope}%
\pgfsys@transformshift{6.437812in}{4.237239in}%
\pgfsys@useobject{currentmarker}{}%
\end{pgfscope}%
\end{pgfscope}%
\begin{pgfscope}%
\pgfsetbuttcap%
\pgfsetroundjoin%
\definecolor{currentfill}{rgb}{0.000000,0.000000,0.000000}%
\pgfsetfillcolor{currentfill}%
\pgfsetlinewidth{0.803000pt}%
\definecolor{currentstroke}{rgb}{0.000000,0.000000,0.000000}%
\pgfsetstrokecolor{currentstroke}%
\pgfsetdash{}{0pt}%
\pgfsys@defobject{currentmarker}{\pgfqpoint{0.000000in}{-0.048611in}}{\pgfqpoint{0.000000in}{0.000000in}}{%
\pgfpathmoveto{\pgfqpoint{0.000000in}{0.000000in}}%
\pgfpathlineto{\pgfqpoint{0.000000in}{-0.048611in}}%
\pgfusepath{stroke,fill}%
}%
\begin{pgfscope}%
\pgfsys@transformshift{7.313106in}{4.237239in}%
\pgfsys@useobject{currentmarker}{}%
\end{pgfscope}%
\end{pgfscope}%
\begin{pgfscope}%
\pgfsetbuttcap%
\pgfsetroundjoin%
\definecolor{currentfill}{rgb}{0.000000,0.000000,0.000000}%
\pgfsetfillcolor{currentfill}%
\pgfsetlinewidth{0.803000pt}%
\definecolor{currentstroke}{rgb}{0.000000,0.000000,0.000000}%
\pgfsetstrokecolor{currentstroke}%
\pgfsetdash{}{0pt}%
\pgfsys@defobject{currentmarker}{\pgfqpoint{0.000000in}{-0.048611in}}{\pgfqpoint{0.000000in}{0.000000in}}{%
\pgfpathmoveto{\pgfqpoint{0.000000in}{0.000000in}}%
\pgfpathlineto{\pgfqpoint{0.000000in}{-0.048611in}}%
\pgfusepath{stroke,fill}%
}%
\begin{pgfscope}%
\pgfsys@transformshift{8.188400in}{4.237239in}%
\pgfsys@useobject{currentmarker}{}%
\end{pgfscope}%
\end{pgfscope}%
\begin{pgfscope}%
\pgfsetbuttcap%
\pgfsetroundjoin%
\definecolor{currentfill}{rgb}{0.000000,0.000000,0.000000}%
\pgfsetfillcolor{currentfill}%
\pgfsetlinewidth{0.803000pt}%
\definecolor{currentstroke}{rgb}{0.000000,0.000000,0.000000}%
\pgfsetstrokecolor{currentstroke}%
\pgfsetdash{}{0pt}%
\pgfsys@defobject{currentmarker}{\pgfqpoint{0.000000in}{-0.048611in}}{\pgfqpoint{0.000000in}{0.000000in}}{%
\pgfpathmoveto{\pgfqpoint{0.000000in}{0.000000in}}%
\pgfpathlineto{\pgfqpoint{0.000000in}{-0.048611in}}%
\pgfusepath{stroke,fill}%
}%
\begin{pgfscope}%
\pgfsys@transformshift{9.063694in}{4.237239in}%
\pgfsys@useobject{currentmarker}{}%
\end{pgfscope}%
\end{pgfscope}%
\begin{pgfscope}%
\pgfsetbuttcap%
\pgfsetroundjoin%
\definecolor{currentfill}{rgb}{0.000000,0.000000,0.000000}%
\pgfsetfillcolor{currentfill}%
\pgfsetlinewidth{0.803000pt}%
\definecolor{currentstroke}{rgb}{0.000000,0.000000,0.000000}%
\pgfsetstrokecolor{currentstroke}%
\pgfsetdash{}{0pt}%
\pgfsys@defobject{currentmarker}{\pgfqpoint{0.000000in}{-0.048611in}}{\pgfqpoint{0.000000in}{0.000000in}}{%
\pgfpathmoveto{\pgfqpoint{0.000000in}{0.000000in}}%
\pgfpathlineto{\pgfqpoint{0.000000in}{-0.048611in}}%
\pgfusepath{stroke,fill}%
}%
\begin{pgfscope}%
\pgfsys@transformshift{9.938988in}{4.237239in}%
\pgfsys@useobject{currentmarker}{}%
\end{pgfscope}%
\end{pgfscope}%
\begin{pgfscope}%
\pgfsetbuttcap%
\pgfsetroundjoin%
\definecolor{currentfill}{rgb}{0.000000,0.000000,0.000000}%
\pgfsetfillcolor{currentfill}%
\pgfsetlinewidth{0.803000pt}%
\definecolor{currentstroke}{rgb}{0.000000,0.000000,0.000000}%
\pgfsetstrokecolor{currentstroke}%
\pgfsetdash{}{0pt}%
\pgfsys@defobject{currentmarker}{\pgfqpoint{-0.048611in}{0.000000in}}{\pgfqpoint{0.000000in}{0.000000in}}{%
\pgfpathmoveto{\pgfqpoint{0.000000in}{0.000000in}}%
\pgfpathlineto{\pgfqpoint{-0.048611in}{0.000000in}}%
\pgfusepath{stroke,fill}%
}%
\begin{pgfscope}%
\pgfsys@transformshift{5.562518in}{4.601976in}%
\pgfsys@useobject{currentmarker}{}%
\end{pgfscope}%
\end{pgfscope}%
\begin{pgfscope}%
\pgftext[x=5.395851in,y=4.549214in,left,base]{\rmfamily\fontsize{10.000000}{12.000000}\selectfont \(\displaystyle 0\)}%
\end{pgfscope}%
\begin{pgfscope}%
\pgfsetbuttcap%
\pgfsetroundjoin%
\definecolor{currentfill}{rgb}{0.000000,0.000000,0.000000}%
\pgfsetfillcolor{currentfill}%
\pgfsetlinewidth{0.803000pt}%
\definecolor{currentstroke}{rgb}{0.000000,0.000000,0.000000}%
\pgfsetstrokecolor{currentstroke}%
\pgfsetdash{}{0pt}%
\pgfsys@defobject{currentmarker}{\pgfqpoint{-0.048611in}{0.000000in}}{\pgfqpoint{0.000000in}{0.000000in}}{%
\pgfpathmoveto{\pgfqpoint{0.000000in}{0.000000in}}%
\pgfpathlineto{\pgfqpoint{-0.048611in}{0.000000in}}%
\pgfusepath{stroke,fill}%
}%
\begin{pgfscope}%
\pgfsys@transformshift{5.562518in}{5.007239in}%
\pgfsys@useobject{currentmarker}{}%
\end{pgfscope}%
\end{pgfscope}%
\begin{pgfscope}%
\pgftext[x=5.395851in,y=4.954477in,left,base]{\rmfamily\fontsize{10.000000}{12.000000}\selectfont \(\displaystyle 2\)}%
\end{pgfscope}%
\begin{pgfscope}%
\pgfpathrectangle{\pgfqpoint{5.562518in}{4.237239in}}{\pgfqpoint{4.376471in}{0.972632in}}%
\pgfusepath{clip}%
\pgfsetrectcap%
\pgfsetroundjoin%
\pgfsetlinewidth{1.505625pt}%
\definecolor{currentstroke}{rgb}{0.121569,0.466667,0.705882}%
\pgfsetstrokecolor{currentstroke}%
\pgfsetdash{}{0pt}%
\pgfpathmoveto{\pgfqpoint{6.437812in}{4.601976in}}%
\pgfpathlineto{\pgfqpoint{9.938988in}{4.601976in}}%
\pgfpathlineto{\pgfqpoint{9.938988in}{4.601976in}}%
\pgfusepath{stroke}%
\end{pgfscope}%
\begin{pgfscope}%
\pgfsetrectcap%
\pgfsetmiterjoin%
\pgfsetlinewidth{0.803000pt}%
\definecolor{currentstroke}{rgb}{0.000000,0.000000,0.000000}%
\pgfsetstrokecolor{currentstroke}%
\pgfsetdash{}{0pt}%
\pgfpathmoveto{\pgfqpoint{5.562518in}{4.237239in}}%
\pgfpathlineto{\pgfqpoint{5.562518in}{5.209870in}}%
\pgfusepath{stroke}%
\end{pgfscope}%
\begin{pgfscope}%
\pgfsetrectcap%
\pgfsetmiterjoin%
\pgfsetlinewidth{0.803000pt}%
\definecolor{currentstroke}{rgb}{0.000000,0.000000,0.000000}%
\pgfsetstrokecolor{currentstroke}%
\pgfsetdash{}{0pt}%
\pgfpathmoveto{\pgfqpoint{9.938988in}{4.237239in}}%
\pgfpathlineto{\pgfqpoint{9.938988in}{5.209870in}}%
\pgfusepath{stroke}%
\end{pgfscope}%
\begin{pgfscope}%
\pgfsetrectcap%
\pgfsetmiterjoin%
\pgfsetlinewidth{0.803000pt}%
\definecolor{currentstroke}{rgb}{0.000000,0.000000,0.000000}%
\pgfsetstrokecolor{currentstroke}%
\pgfsetdash{}{0pt}%
\pgfpathmoveto{\pgfqpoint{5.562518in}{4.237239in}}%
\pgfpathlineto{\pgfqpoint{9.938988in}{4.237239in}}%
\pgfusepath{stroke}%
\end{pgfscope}%
\begin{pgfscope}%
\pgfsetrectcap%
\pgfsetmiterjoin%
\pgfsetlinewidth{0.803000pt}%
\definecolor{currentstroke}{rgb}{0.000000,0.000000,0.000000}%
\pgfsetstrokecolor{currentstroke}%
\pgfsetdash{}{0pt}%
\pgfpathmoveto{\pgfqpoint{5.562518in}{5.209870in}}%
\pgfpathlineto{\pgfqpoint{9.938988in}{5.209870in}}%
\pgfusepath{stroke}%
\end{pgfscope}%
\begin{pgfscope}%
\pgfsetbuttcap%
\pgfsetmiterjoin%
\definecolor{currentfill}{rgb}{1.000000,1.000000,1.000000}%
\pgfsetfillcolor{currentfill}%
\pgfsetfillopacity{0.800000}%
\pgfsetlinewidth{1.003750pt}%
\definecolor{currentstroke}{rgb}{0.800000,0.800000,0.800000}%
\pgfsetstrokecolor{currentstroke}%
\pgfsetstrokeopacity{0.800000}%
\pgfsetdash{}{0pt}%
\pgfpathmoveto{\pgfqpoint{5.659740in}{4.306683in}}%
\pgfpathlineto{\pgfqpoint{6.443065in}{4.306683in}}%
\pgfpathquadraticcurveto{\pgfqpoint{6.470843in}{4.306683in}}{\pgfqpoint{6.470843in}{4.334461in}}%
\pgfpathlineto{\pgfqpoint{6.470843in}{4.933148in}}%
\pgfpathquadraticcurveto{\pgfqpoint{6.470843in}{4.960925in}}{\pgfqpoint{6.443065in}{4.960925in}}%
\pgfpathlineto{\pgfqpoint{5.659740in}{4.960925in}}%
\pgfpathquadraticcurveto{\pgfqpoint{5.631962in}{4.960925in}}{\pgfqpoint{5.631962in}{4.933148in}}%
\pgfpathlineto{\pgfqpoint{5.631962in}{4.334461in}}%
\pgfpathquadraticcurveto{\pgfqpoint{5.631962in}{4.306683in}}{\pgfqpoint{5.659740in}{4.306683in}}%
\pgfpathclose%
\pgfusepath{stroke,fill}%
\end{pgfscope}%
\begin{pgfscope}%
\pgfsetrectcap%
\pgfsetroundjoin%
\pgfsetlinewidth{1.505625pt}%
\definecolor{currentstroke}{rgb}{0.121569,0.466667,0.705882}%
\pgfsetstrokecolor{currentstroke}%
\pgfsetdash{}{0pt}%
\pgfpathmoveto{\pgfqpoint{5.687518in}{4.847454in}}%
\pgfpathlineto{\pgfqpoint{5.965295in}{4.847454in}}%
\pgfusepath{stroke}%
\end{pgfscope}%
\begin{pgfscope}%
\pgftext[x=6.076407in,y=4.798843in,left,base]{\rmfamily\fontsize{10.000000}{12.000000}\selectfont \(\displaystyle \widetilde{\Phi}^* \theta^{\parallel}\)}%
\end{pgfscope}%
\begin{pgfscope}%
\pgfsetbuttcap%
\pgfsetroundjoin%
\definecolor{currentfill}{rgb}{1.000000,0.000000,0.000000}%
\pgfsetfillcolor{currentfill}%
\pgfsetlinewidth{2.007500pt}%
\definecolor{currentstroke}{rgb}{1.000000,0.000000,0.000000}%
\pgfsetstrokecolor{currentstroke}%
\pgfsetdash{}{0pt}%
\pgfpathmoveto{\pgfqpoint{5.784740in}{4.631444in}}%
\pgfpathlineto{\pgfqpoint{5.868073in}{4.631444in}}%
\pgfpathmoveto{\pgfqpoint{5.826407in}{4.589777in}}%
\pgfpathlineto{\pgfqpoint{5.826407in}{4.673111in}}%
\pgfusepath{stroke,fill}%
\end{pgfscope}%
\begin{pgfscope}%
\pgftext[x=6.076407in,y=4.594986in,left,base]{\rmfamily\fontsize{10.000000}{12.000000}\selectfont train}%
\end{pgfscope}%
\begin{pgfscope}%
\pgfsetbuttcap%
\pgfsetroundjoin%
\definecolor{currentfill}{rgb}{0.000000,0.000000,0.000000}%
\pgfsetfillcolor{currentfill}%
\pgfsetlinewidth{1.003750pt}%
\definecolor{currentstroke}{rgb}{0.000000,0.000000,0.000000}%
\pgfsetstrokecolor{currentstroke}%
\pgfsetdash{}{0pt}%
\pgfsys@defobject{currentmarker}{\pgfqpoint{-0.020833in}{-0.020833in}}{\pgfqpoint{0.020833in}{0.020833in}}{%
\pgfpathmoveto{\pgfqpoint{0.000000in}{-0.020833in}}%
\pgfpathcurveto{\pgfqpoint{0.005525in}{-0.020833in}}{\pgfqpoint{0.010825in}{-0.018638in}}{\pgfqpoint{0.014731in}{-0.014731in}}%
\pgfpathcurveto{\pgfqpoint{0.018638in}{-0.010825in}}{\pgfqpoint{0.020833in}{-0.005525in}}{\pgfqpoint{0.020833in}{0.000000in}}%
\pgfpathcurveto{\pgfqpoint{0.020833in}{0.005525in}}{\pgfqpoint{0.018638in}{0.010825in}}{\pgfqpoint{0.014731in}{0.014731in}}%
\pgfpathcurveto{\pgfqpoint{0.010825in}{0.018638in}}{\pgfqpoint{0.005525in}{0.020833in}}{\pgfqpoint{0.000000in}{0.020833in}}%
\pgfpathcurveto{\pgfqpoint{-0.005525in}{0.020833in}}{\pgfqpoint{-0.010825in}{0.018638in}}{\pgfqpoint{-0.014731in}{0.014731in}}%
\pgfpathcurveto{\pgfqpoint{-0.018638in}{0.010825in}}{\pgfqpoint{-0.020833in}{0.005525in}}{\pgfqpoint{-0.020833in}{0.000000in}}%
\pgfpathcurveto{\pgfqpoint{-0.020833in}{-0.005525in}}{\pgfqpoint{-0.018638in}{-0.010825in}}{\pgfqpoint{-0.014731in}{-0.014731in}}%
\pgfpathcurveto{\pgfqpoint{-0.010825in}{-0.018638in}}{\pgfqpoint{-0.005525in}{-0.020833in}}{\pgfqpoint{0.000000in}{-0.020833in}}%
\pgfpathclose%
\pgfusepath{stroke,fill}%
}%
\begin{pgfscope}%
\pgfsys@transformshift{5.826407in}{4.427587in}%
\pgfsys@useobject{currentmarker}{}%
\end{pgfscope}%
\end{pgfscope}%
\begin{pgfscope}%
\pgftext[x=6.076407in,y=4.391128in,left,base]{\rmfamily\fontsize{10.000000}{12.000000}\selectfont test}%
\end{pgfscope}%
\begin{pgfscope}%
\pgfsetbuttcap%
\pgfsetmiterjoin%
\definecolor{currentfill}{rgb}{1.000000,1.000000,1.000000}%
\pgfsetfillcolor{currentfill}%
\pgfsetlinewidth{0.000000pt}%
\definecolor{currentstroke}{rgb}{0.000000,0.000000,0.000000}%
\pgfsetstrokecolor{currentstroke}%
\pgfsetstrokeopacity{0.000000}%
\pgfsetdash{}{0pt}%
\pgfpathmoveto{\pgfqpoint{10.668400in}{4.237239in}}%
\pgfpathlineto{\pgfqpoint{12.856635in}{4.237239in}}%
\pgfpathlineto{\pgfqpoint{12.856635in}{5.209870in}}%
\pgfpathlineto{\pgfqpoint{10.668400in}{5.209870in}}%
\pgfpathclose%
\pgfusepath{fill}%
\end{pgfscope}%
\begin{pgfscope}%
\pgfpathrectangle{\pgfqpoint{10.668400in}{4.237239in}}{\pgfqpoint{2.188235in}{0.972632in}}%
\pgfusepath{clip}%
\pgfsetbuttcap%
\pgfsetmiterjoin%
\definecolor{currentfill}{rgb}{0.121569,0.466667,0.705882}%
\pgfsetfillcolor{currentfill}%
\pgfsetlinewidth{0.000000pt}%
\definecolor{currentstroke}{rgb}{0.000000,0.000000,0.000000}%
\pgfsetstrokecolor{currentstroke}%
\pgfsetstrokeopacity{0.000000}%
\pgfsetdash{}{0pt}%
\pgfpathmoveto{\pgfqpoint{-319.224843in}{4.281449in}}%
\pgfpathlineto{\pgfqpoint{8.425867in}{4.281449in}}%
\pgfpathlineto{\pgfqpoint{8.425867in}{4.288537in}}%
\pgfpathlineto{\pgfqpoint{-319.224843in}{4.288537in}}%
\pgfpathclose%
\pgfusepath{fill}%
\end{pgfscope}%
\begin{pgfscope}%
\pgfpathrectangle{\pgfqpoint{10.668400in}{4.237239in}}{\pgfqpoint{2.188235in}{0.972632in}}%
\pgfusepath{clip}%
\pgfsetbuttcap%
\pgfsetmiterjoin%
\definecolor{currentfill}{rgb}{0.121569,0.466667,0.705882}%
\pgfsetfillcolor{currentfill}%
\pgfsetlinewidth{0.000000pt}%
\definecolor{currentstroke}{rgb}{0.000000,0.000000,0.000000}%
\pgfsetstrokecolor{currentstroke}%
\pgfsetstrokeopacity{0.000000}%
\pgfsetdash{}{0pt}%
\pgfpathmoveto{\pgfqpoint{-319.224843in}{4.290309in}}%
\pgfpathlineto{\pgfqpoint{8.075547in}{4.290309in}}%
\pgfpathlineto{\pgfqpoint{8.075547in}{4.297397in}}%
\pgfpathlineto{\pgfqpoint{-319.224843in}{4.297397in}}%
\pgfpathclose%
\pgfusepath{fill}%
\end{pgfscope}%
\begin{pgfscope}%
\pgfpathrectangle{\pgfqpoint{10.668400in}{4.237239in}}{\pgfqpoint{2.188235in}{0.972632in}}%
\pgfusepath{clip}%
\pgfsetbuttcap%
\pgfsetmiterjoin%
\definecolor{currentfill}{rgb}{0.121569,0.466667,0.705882}%
\pgfsetfillcolor{currentfill}%
\pgfsetlinewidth{0.000000pt}%
\definecolor{currentstroke}{rgb}{0.000000,0.000000,0.000000}%
\pgfsetstrokecolor{currentstroke}%
\pgfsetstrokeopacity{0.000000}%
\pgfsetdash{}{0pt}%
\pgfpathmoveto{\pgfqpoint{-319.224843in}{4.299169in}}%
\pgfpathlineto{\pgfqpoint{8.281322in}{4.299169in}}%
\pgfpathlineto{\pgfqpoint{8.281322in}{4.306257in}}%
\pgfpathlineto{\pgfqpoint{-319.224843in}{4.306257in}}%
\pgfpathclose%
\pgfusepath{fill}%
\end{pgfscope}%
\begin{pgfscope}%
\pgfpathrectangle{\pgfqpoint{10.668400in}{4.237239in}}{\pgfqpoint{2.188235in}{0.972632in}}%
\pgfusepath{clip}%
\pgfsetbuttcap%
\pgfsetmiterjoin%
\definecolor{currentfill}{rgb}{0.121569,0.466667,0.705882}%
\pgfsetfillcolor{currentfill}%
\pgfsetlinewidth{0.000000pt}%
\definecolor{currentstroke}{rgb}{0.000000,0.000000,0.000000}%
\pgfsetstrokecolor{currentstroke}%
\pgfsetstrokeopacity{0.000000}%
\pgfsetdash{}{0pt}%
\pgfpathmoveto{\pgfqpoint{-319.224843in}{4.308029in}}%
\pgfpathlineto{\pgfqpoint{8.097179in}{4.308029in}}%
\pgfpathlineto{\pgfqpoint{8.097179in}{4.315117in}}%
\pgfpathlineto{\pgfqpoint{-319.224843in}{4.315117in}}%
\pgfpathclose%
\pgfusepath{fill}%
\end{pgfscope}%
\begin{pgfscope}%
\pgfpathrectangle{\pgfqpoint{10.668400in}{4.237239in}}{\pgfqpoint{2.188235in}{0.972632in}}%
\pgfusepath{clip}%
\pgfsetbuttcap%
\pgfsetmiterjoin%
\definecolor{currentfill}{rgb}{0.121569,0.466667,0.705882}%
\pgfsetfillcolor{currentfill}%
\pgfsetlinewidth{0.000000pt}%
\definecolor{currentstroke}{rgb}{0.000000,0.000000,0.000000}%
\pgfsetstrokecolor{currentstroke}%
\pgfsetstrokeopacity{0.000000}%
\pgfsetdash{}{0pt}%
\pgfpathmoveto{\pgfqpoint{-319.224843in}{4.316889in}}%
\pgfpathlineto{\pgfqpoint{8.354986in}{4.316889in}}%
\pgfpathlineto{\pgfqpoint{8.354986in}{4.323976in}}%
\pgfpathlineto{\pgfqpoint{-319.224843in}{4.323976in}}%
\pgfpathclose%
\pgfusepath{fill}%
\end{pgfscope}%
\begin{pgfscope}%
\pgfpathrectangle{\pgfqpoint{10.668400in}{4.237239in}}{\pgfqpoint{2.188235in}{0.972632in}}%
\pgfusepath{clip}%
\pgfsetbuttcap%
\pgfsetmiterjoin%
\definecolor{currentfill}{rgb}{0.121569,0.466667,0.705882}%
\pgfsetfillcolor{currentfill}%
\pgfsetlinewidth{0.000000pt}%
\definecolor{currentstroke}{rgb}{0.000000,0.000000,0.000000}%
\pgfsetstrokecolor{currentstroke}%
\pgfsetstrokeopacity{0.000000}%
\pgfsetdash{}{0pt}%
\pgfpathmoveto{\pgfqpoint{-319.224843in}{4.325748in}}%
\pgfpathlineto{\pgfqpoint{8.202034in}{4.325748in}}%
\pgfpathlineto{\pgfqpoint{8.202034in}{4.332836in}}%
\pgfpathlineto{\pgfqpoint{-319.224843in}{4.332836in}}%
\pgfpathclose%
\pgfusepath{fill}%
\end{pgfscope}%
\begin{pgfscope}%
\pgfpathrectangle{\pgfqpoint{10.668400in}{4.237239in}}{\pgfqpoint{2.188235in}{0.972632in}}%
\pgfusepath{clip}%
\pgfsetbuttcap%
\pgfsetmiterjoin%
\definecolor{currentfill}{rgb}{0.121569,0.466667,0.705882}%
\pgfsetfillcolor{currentfill}%
\pgfsetlinewidth{0.000000pt}%
\definecolor{currentstroke}{rgb}{0.000000,0.000000,0.000000}%
\pgfsetstrokecolor{currentstroke}%
\pgfsetstrokeopacity{0.000000}%
\pgfsetdash{}{0pt}%
\pgfpathmoveto{\pgfqpoint{-319.224843in}{4.334608in}}%
\pgfpathlineto{\pgfqpoint{8.517480in}{4.334608in}}%
\pgfpathlineto{\pgfqpoint{8.517480in}{4.341696in}}%
\pgfpathlineto{\pgfqpoint{-319.224843in}{4.341696in}}%
\pgfpathclose%
\pgfusepath{fill}%
\end{pgfscope}%
\begin{pgfscope}%
\pgfpathrectangle{\pgfqpoint{10.668400in}{4.237239in}}{\pgfqpoint{2.188235in}{0.972632in}}%
\pgfusepath{clip}%
\pgfsetbuttcap%
\pgfsetmiterjoin%
\definecolor{currentfill}{rgb}{0.121569,0.466667,0.705882}%
\pgfsetfillcolor{currentfill}%
\pgfsetlinewidth{0.000000pt}%
\definecolor{currentstroke}{rgb}{0.000000,0.000000,0.000000}%
\pgfsetstrokecolor{currentstroke}%
\pgfsetstrokeopacity{0.000000}%
\pgfsetdash{}{0pt}%
\pgfpathmoveto{\pgfqpoint{-319.224843in}{4.343468in}}%
\pgfpathlineto{\pgfqpoint{8.310222in}{4.343468in}}%
\pgfpathlineto{\pgfqpoint{8.310222in}{4.350556in}}%
\pgfpathlineto{\pgfqpoint{-319.224843in}{4.350556in}}%
\pgfpathclose%
\pgfusepath{fill}%
\end{pgfscope}%
\begin{pgfscope}%
\pgfpathrectangle{\pgfqpoint{10.668400in}{4.237239in}}{\pgfqpoint{2.188235in}{0.972632in}}%
\pgfusepath{clip}%
\pgfsetbuttcap%
\pgfsetmiterjoin%
\definecolor{currentfill}{rgb}{0.121569,0.466667,0.705882}%
\pgfsetfillcolor{currentfill}%
\pgfsetlinewidth{0.000000pt}%
\definecolor{currentstroke}{rgb}{0.000000,0.000000,0.000000}%
\pgfsetstrokecolor{currentstroke}%
\pgfsetstrokeopacity{0.000000}%
\pgfsetdash{}{0pt}%
\pgfpathmoveto{\pgfqpoint{-319.224843in}{4.352328in}}%
\pgfpathlineto{\pgfqpoint{8.040028in}{4.352328in}}%
\pgfpathlineto{\pgfqpoint{8.040028in}{4.359416in}}%
\pgfpathlineto{\pgfqpoint{-319.224843in}{4.359416in}}%
\pgfpathclose%
\pgfusepath{fill}%
\end{pgfscope}%
\begin{pgfscope}%
\pgfpathrectangle{\pgfqpoint{10.668400in}{4.237239in}}{\pgfqpoint{2.188235in}{0.972632in}}%
\pgfusepath{clip}%
\pgfsetbuttcap%
\pgfsetmiterjoin%
\definecolor{currentfill}{rgb}{0.121569,0.466667,0.705882}%
\pgfsetfillcolor{currentfill}%
\pgfsetlinewidth{0.000000pt}%
\definecolor{currentstroke}{rgb}{0.000000,0.000000,0.000000}%
\pgfsetstrokecolor{currentstroke}%
\pgfsetstrokeopacity{0.000000}%
\pgfsetdash{}{0pt}%
\pgfpathmoveto{\pgfqpoint{-319.224843in}{4.361188in}}%
\pgfpathlineto{\pgfqpoint{8.283923in}{4.361188in}}%
\pgfpathlineto{\pgfqpoint{8.283923in}{4.368276in}}%
\pgfpathlineto{\pgfqpoint{-319.224843in}{4.368276in}}%
\pgfpathclose%
\pgfusepath{fill}%
\end{pgfscope}%
\begin{pgfscope}%
\pgfpathrectangle{\pgfqpoint{10.668400in}{4.237239in}}{\pgfqpoint{2.188235in}{0.972632in}}%
\pgfusepath{clip}%
\pgfsetbuttcap%
\pgfsetmiterjoin%
\definecolor{currentfill}{rgb}{0.121569,0.466667,0.705882}%
\pgfsetfillcolor{currentfill}%
\pgfsetlinewidth{0.000000pt}%
\definecolor{currentstroke}{rgb}{0.000000,0.000000,0.000000}%
\pgfsetstrokecolor{currentstroke}%
\pgfsetstrokeopacity{0.000000}%
\pgfsetdash{}{0pt}%
\pgfpathmoveto{\pgfqpoint{-319.224843in}{4.370047in}}%
\pgfpathlineto{\pgfqpoint{8.302100in}{4.370047in}}%
\pgfpathlineto{\pgfqpoint{8.302100in}{4.377135in}}%
\pgfpathlineto{\pgfqpoint{-319.224843in}{4.377135in}}%
\pgfpathclose%
\pgfusepath{fill}%
\end{pgfscope}%
\begin{pgfscope}%
\pgfpathrectangle{\pgfqpoint{10.668400in}{4.237239in}}{\pgfqpoint{2.188235in}{0.972632in}}%
\pgfusepath{clip}%
\pgfsetbuttcap%
\pgfsetmiterjoin%
\definecolor{currentfill}{rgb}{0.121569,0.466667,0.705882}%
\pgfsetfillcolor{currentfill}%
\pgfsetlinewidth{0.000000pt}%
\definecolor{currentstroke}{rgb}{0.000000,0.000000,0.000000}%
\pgfsetstrokecolor{currentstroke}%
\pgfsetstrokeopacity{0.000000}%
\pgfsetdash{}{0pt}%
\pgfpathmoveto{\pgfqpoint{-319.224843in}{4.378907in}}%
\pgfpathlineto{\pgfqpoint{8.401341in}{4.378907in}}%
\pgfpathlineto{\pgfqpoint{8.401341in}{4.385995in}}%
\pgfpathlineto{\pgfqpoint{-319.224843in}{4.385995in}}%
\pgfpathclose%
\pgfusepath{fill}%
\end{pgfscope}%
\begin{pgfscope}%
\pgfpathrectangle{\pgfqpoint{10.668400in}{4.237239in}}{\pgfqpoint{2.188235in}{0.972632in}}%
\pgfusepath{clip}%
\pgfsetbuttcap%
\pgfsetmiterjoin%
\definecolor{currentfill}{rgb}{0.121569,0.466667,0.705882}%
\pgfsetfillcolor{currentfill}%
\pgfsetlinewidth{0.000000pt}%
\definecolor{currentstroke}{rgb}{0.000000,0.000000,0.000000}%
\pgfsetstrokecolor{currentstroke}%
\pgfsetstrokeopacity{0.000000}%
\pgfsetdash{}{0pt}%
\pgfpathmoveto{\pgfqpoint{-319.224843in}{4.387767in}}%
\pgfpathlineto{\pgfqpoint{8.396649in}{4.387767in}}%
\pgfpathlineto{\pgfqpoint{8.396649in}{4.394855in}}%
\pgfpathlineto{\pgfqpoint{-319.224843in}{4.394855in}}%
\pgfpathclose%
\pgfusepath{fill}%
\end{pgfscope}%
\begin{pgfscope}%
\pgfpathrectangle{\pgfqpoint{10.668400in}{4.237239in}}{\pgfqpoint{2.188235in}{0.972632in}}%
\pgfusepath{clip}%
\pgfsetbuttcap%
\pgfsetmiterjoin%
\definecolor{currentfill}{rgb}{0.121569,0.466667,0.705882}%
\pgfsetfillcolor{currentfill}%
\pgfsetlinewidth{0.000000pt}%
\definecolor{currentstroke}{rgb}{0.000000,0.000000,0.000000}%
\pgfsetstrokecolor{currentstroke}%
\pgfsetstrokeopacity{0.000000}%
\pgfsetdash{}{0pt}%
\pgfpathmoveto{\pgfqpoint{-319.224843in}{4.396627in}}%
\pgfpathlineto{\pgfqpoint{8.064143in}{4.396627in}}%
\pgfpathlineto{\pgfqpoint{8.064143in}{4.403715in}}%
\pgfpathlineto{\pgfqpoint{-319.224843in}{4.403715in}}%
\pgfpathclose%
\pgfusepath{fill}%
\end{pgfscope}%
\begin{pgfscope}%
\pgfpathrectangle{\pgfqpoint{10.668400in}{4.237239in}}{\pgfqpoint{2.188235in}{0.972632in}}%
\pgfusepath{clip}%
\pgfsetbuttcap%
\pgfsetmiterjoin%
\definecolor{currentfill}{rgb}{0.121569,0.466667,0.705882}%
\pgfsetfillcolor{currentfill}%
\pgfsetlinewidth{0.000000pt}%
\definecolor{currentstroke}{rgb}{0.000000,0.000000,0.000000}%
\pgfsetstrokecolor{currentstroke}%
\pgfsetstrokeopacity{0.000000}%
\pgfsetdash{}{0pt}%
\pgfpathmoveto{\pgfqpoint{-319.224843in}{4.405487in}}%
\pgfpathlineto{\pgfqpoint{8.425805in}{4.405487in}}%
\pgfpathlineto{\pgfqpoint{8.425805in}{4.412575in}}%
\pgfpathlineto{\pgfqpoint{-319.224843in}{4.412575in}}%
\pgfpathclose%
\pgfusepath{fill}%
\end{pgfscope}%
\begin{pgfscope}%
\pgfpathrectangle{\pgfqpoint{10.668400in}{4.237239in}}{\pgfqpoint{2.188235in}{0.972632in}}%
\pgfusepath{clip}%
\pgfsetbuttcap%
\pgfsetmiterjoin%
\definecolor{currentfill}{rgb}{0.121569,0.466667,0.705882}%
\pgfsetfillcolor{currentfill}%
\pgfsetlinewidth{0.000000pt}%
\definecolor{currentstroke}{rgb}{0.000000,0.000000,0.000000}%
\pgfsetstrokecolor{currentstroke}%
\pgfsetstrokeopacity{0.000000}%
\pgfsetdash{}{0pt}%
\pgfpathmoveto{\pgfqpoint{-319.224843in}{4.414347in}}%
\pgfpathlineto{\pgfqpoint{8.357377in}{4.414347in}}%
\pgfpathlineto{\pgfqpoint{8.357377in}{4.421434in}}%
\pgfpathlineto{\pgfqpoint{-319.224843in}{4.421434in}}%
\pgfpathclose%
\pgfusepath{fill}%
\end{pgfscope}%
\begin{pgfscope}%
\pgfpathrectangle{\pgfqpoint{10.668400in}{4.237239in}}{\pgfqpoint{2.188235in}{0.972632in}}%
\pgfusepath{clip}%
\pgfsetbuttcap%
\pgfsetmiterjoin%
\definecolor{currentfill}{rgb}{0.121569,0.466667,0.705882}%
\pgfsetfillcolor{currentfill}%
\pgfsetlinewidth{0.000000pt}%
\definecolor{currentstroke}{rgb}{0.000000,0.000000,0.000000}%
\pgfsetstrokecolor{currentstroke}%
\pgfsetstrokeopacity{0.000000}%
\pgfsetdash{}{0pt}%
\pgfpathmoveto{\pgfqpoint{-319.224843in}{4.423206in}}%
\pgfpathlineto{\pgfqpoint{8.306290in}{4.423206in}}%
\pgfpathlineto{\pgfqpoint{8.306290in}{4.430294in}}%
\pgfpathlineto{\pgfqpoint{-319.224843in}{4.430294in}}%
\pgfpathclose%
\pgfusepath{fill}%
\end{pgfscope}%
\begin{pgfscope}%
\pgfpathrectangle{\pgfqpoint{10.668400in}{4.237239in}}{\pgfqpoint{2.188235in}{0.972632in}}%
\pgfusepath{clip}%
\pgfsetbuttcap%
\pgfsetmiterjoin%
\definecolor{currentfill}{rgb}{0.121569,0.466667,0.705882}%
\pgfsetfillcolor{currentfill}%
\pgfsetlinewidth{0.000000pt}%
\definecolor{currentstroke}{rgb}{0.000000,0.000000,0.000000}%
\pgfsetstrokecolor{currentstroke}%
\pgfsetstrokeopacity{0.000000}%
\pgfsetdash{}{0pt}%
\pgfpathmoveto{\pgfqpoint{-319.224843in}{4.432066in}}%
\pgfpathlineto{\pgfqpoint{8.237062in}{4.432066in}}%
\pgfpathlineto{\pgfqpoint{8.237062in}{4.439154in}}%
\pgfpathlineto{\pgfqpoint{-319.224843in}{4.439154in}}%
\pgfpathclose%
\pgfusepath{fill}%
\end{pgfscope}%
\begin{pgfscope}%
\pgfpathrectangle{\pgfqpoint{10.668400in}{4.237239in}}{\pgfqpoint{2.188235in}{0.972632in}}%
\pgfusepath{clip}%
\pgfsetbuttcap%
\pgfsetmiterjoin%
\definecolor{currentfill}{rgb}{0.121569,0.466667,0.705882}%
\pgfsetfillcolor{currentfill}%
\pgfsetlinewidth{0.000000pt}%
\definecolor{currentstroke}{rgb}{0.000000,0.000000,0.000000}%
\pgfsetstrokecolor{currentstroke}%
\pgfsetstrokeopacity{0.000000}%
\pgfsetdash{}{0pt}%
\pgfpathmoveto{\pgfqpoint{-319.224843in}{4.440926in}}%
\pgfpathlineto{\pgfqpoint{8.027750in}{4.440926in}}%
\pgfpathlineto{\pgfqpoint{8.027750in}{4.448014in}}%
\pgfpathlineto{\pgfqpoint{-319.224843in}{4.448014in}}%
\pgfpathclose%
\pgfusepath{fill}%
\end{pgfscope}%
\begin{pgfscope}%
\pgfpathrectangle{\pgfqpoint{10.668400in}{4.237239in}}{\pgfqpoint{2.188235in}{0.972632in}}%
\pgfusepath{clip}%
\pgfsetbuttcap%
\pgfsetmiterjoin%
\definecolor{currentfill}{rgb}{0.121569,0.466667,0.705882}%
\pgfsetfillcolor{currentfill}%
\pgfsetlinewidth{0.000000pt}%
\definecolor{currentstroke}{rgb}{0.000000,0.000000,0.000000}%
\pgfsetstrokecolor{currentstroke}%
\pgfsetstrokeopacity{0.000000}%
\pgfsetdash{}{0pt}%
\pgfpathmoveto{\pgfqpoint{-319.224843in}{4.449786in}}%
\pgfpathlineto{\pgfqpoint{8.233801in}{4.449786in}}%
\pgfpathlineto{\pgfqpoint{8.233801in}{4.456874in}}%
\pgfpathlineto{\pgfqpoint{-319.224843in}{4.456874in}}%
\pgfpathclose%
\pgfusepath{fill}%
\end{pgfscope}%
\begin{pgfscope}%
\pgfpathrectangle{\pgfqpoint{10.668400in}{4.237239in}}{\pgfqpoint{2.188235in}{0.972632in}}%
\pgfusepath{clip}%
\pgfsetbuttcap%
\pgfsetmiterjoin%
\definecolor{currentfill}{rgb}{0.121569,0.466667,0.705882}%
\pgfsetfillcolor{currentfill}%
\pgfsetlinewidth{0.000000pt}%
\definecolor{currentstroke}{rgb}{0.000000,0.000000,0.000000}%
\pgfsetstrokecolor{currentstroke}%
\pgfsetstrokeopacity{0.000000}%
\pgfsetdash{}{0pt}%
\pgfpathmoveto{\pgfqpoint{-319.224843in}{4.458646in}}%
\pgfpathlineto{\pgfqpoint{8.136712in}{4.458646in}}%
\pgfpathlineto{\pgfqpoint{8.136712in}{4.465734in}}%
\pgfpathlineto{\pgfqpoint{-319.224843in}{4.465734in}}%
\pgfpathclose%
\pgfusepath{fill}%
\end{pgfscope}%
\begin{pgfscope}%
\pgfpathrectangle{\pgfqpoint{10.668400in}{4.237239in}}{\pgfqpoint{2.188235in}{0.972632in}}%
\pgfusepath{clip}%
\pgfsetbuttcap%
\pgfsetmiterjoin%
\definecolor{currentfill}{rgb}{0.121569,0.466667,0.705882}%
\pgfsetfillcolor{currentfill}%
\pgfsetlinewidth{0.000000pt}%
\definecolor{currentstroke}{rgb}{0.000000,0.000000,0.000000}%
\pgfsetstrokecolor{currentstroke}%
\pgfsetstrokeopacity{0.000000}%
\pgfsetdash{}{0pt}%
\pgfpathmoveto{\pgfqpoint{-319.224843in}{4.467506in}}%
\pgfpathlineto{\pgfqpoint{8.211691in}{4.467506in}}%
\pgfpathlineto{\pgfqpoint{8.211691in}{4.474593in}}%
\pgfpathlineto{\pgfqpoint{-319.224843in}{4.474593in}}%
\pgfpathclose%
\pgfusepath{fill}%
\end{pgfscope}%
\begin{pgfscope}%
\pgfpathrectangle{\pgfqpoint{10.668400in}{4.237239in}}{\pgfqpoint{2.188235in}{0.972632in}}%
\pgfusepath{clip}%
\pgfsetbuttcap%
\pgfsetmiterjoin%
\definecolor{currentfill}{rgb}{0.121569,0.466667,0.705882}%
\pgfsetfillcolor{currentfill}%
\pgfsetlinewidth{0.000000pt}%
\definecolor{currentstroke}{rgb}{0.000000,0.000000,0.000000}%
\pgfsetstrokecolor{currentstroke}%
\pgfsetstrokeopacity{0.000000}%
\pgfsetdash{}{0pt}%
\pgfpathmoveto{\pgfqpoint{-319.224843in}{4.476365in}}%
\pgfpathlineto{\pgfqpoint{8.237177in}{4.476365in}}%
\pgfpathlineto{\pgfqpoint{8.237177in}{4.483453in}}%
\pgfpathlineto{\pgfqpoint{-319.224843in}{4.483453in}}%
\pgfpathclose%
\pgfusepath{fill}%
\end{pgfscope}%
\begin{pgfscope}%
\pgfpathrectangle{\pgfqpoint{10.668400in}{4.237239in}}{\pgfqpoint{2.188235in}{0.972632in}}%
\pgfusepath{clip}%
\pgfsetbuttcap%
\pgfsetmiterjoin%
\definecolor{currentfill}{rgb}{0.121569,0.466667,0.705882}%
\pgfsetfillcolor{currentfill}%
\pgfsetlinewidth{0.000000pt}%
\definecolor{currentstroke}{rgb}{0.000000,0.000000,0.000000}%
\pgfsetstrokecolor{currentstroke}%
\pgfsetstrokeopacity{0.000000}%
\pgfsetdash{}{0pt}%
\pgfpathmoveto{\pgfqpoint{-319.224843in}{4.485225in}}%
\pgfpathlineto{\pgfqpoint{8.287130in}{4.485225in}}%
\pgfpathlineto{\pgfqpoint{8.287130in}{4.492313in}}%
\pgfpathlineto{\pgfqpoint{-319.224843in}{4.492313in}}%
\pgfpathclose%
\pgfusepath{fill}%
\end{pgfscope}%
\begin{pgfscope}%
\pgfpathrectangle{\pgfqpoint{10.668400in}{4.237239in}}{\pgfqpoint{2.188235in}{0.972632in}}%
\pgfusepath{clip}%
\pgfsetbuttcap%
\pgfsetmiterjoin%
\definecolor{currentfill}{rgb}{0.121569,0.466667,0.705882}%
\pgfsetfillcolor{currentfill}%
\pgfsetlinewidth{0.000000pt}%
\definecolor{currentstroke}{rgb}{0.000000,0.000000,0.000000}%
\pgfsetstrokecolor{currentstroke}%
\pgfsetstrokeopacity{0.000000}%
\pgfsetdash{}{0pt}%
\pgfpathmoveto{\pgfqpoint{-319.224843in}{4.494085in}}%
\pgfpathlineto{\pgfqpoint{8.281829in}{4.494085in}}%
\pgfpathlineto{\pgfqpoint{8.281829in}{4.501173in}}%
\pgfpathlineto{\pgfqpoint{-319.224843in}{4.501173in}}%
\pgfpathclose%
\pgfusepath{fill}%
\end{pgfscope}%
\begin{pgfscope}%
\pgfpathrectangle{\pgfqpoint{10.668400in}{4.237239in}}{\pgfqpoint{2.188235in}{0.972632in}}%
\pgfusepath{clip}%
\pgfsetbuttcap%
\pgfsetmiterjoin%
\definecolor{currentfill}{rgb}{0.121569,0.466667,0.705882}%
\pgfsetfillcolor{currentfill}%
\pgfsetlinewidth{0.000000pt}%
\definecolor{currentstroke}{rgb}{0.000000,0.000000,0.000000}%
\pgfsetstrokecolor{currentstroke}%
\pgfsetstrokeopacity{0.000000}%
\pgfsetdash{}{0pt}%
\pgfpathmoveto{\pgfqpoint{-319.224843in}{4.502945in}}%
\pgfpathlineto{\pgfqpoint{8.269503in}{4.502945in}}%
\pgfpathlineto{\pgfqpoint{8.269503in}{4.510033in}}%
\pgfpathlineto{\pgfqpoint{-319.224843in}{4.510033in}}%
\pgfpathclose%
\pgfusepath{fill}%
\end{pgfscope}%
\begin{pgfscope}%
\pgfpathrectangle{\pgfqpoint{10.668400in}{4.237239in}}{\pgfqpoint{2.188235in}{0.972632in}}%
\pgfusepath{clip}%
\pgfsetbuttcap%
\pgfsetmiterjoin%
\definecolor{currentfill}{rgb}{0.121569,0.466667,0.705882}%
\pgfsetfillcolor{currentfill}%
\pgfsetlinewidth{0.000000pt}%
\definecolor{currentstroke}{rgb}{0.000000,0.000000,0.000000}%
\pgfsetstrokecolor{currentstroke}%
\pgfsetstrokeopacity{0.000000}%
\pgfsetdash{}{0pt}%
\pgfpathmoveto{\pgfqpoint{-319.224843in}{4.511805in}}%
\pgfpathlineto{\pgfqpoint{8.368123in}{4.511805in}}%
\pgfpathlineto{\pgfqpoint{8.368123in}{4.518893in}}%
\pgfpathlineto{\pgfqpoint{-319.224843in}{4.518893in}}%
\pgfpathclose%
\pgfusepath{fill}%
\end{pgfscope}%
\begin{pgfscope}%
\pgfpathrectangle{\pgfqpoint{10.668400in}{4.237239in}}{\pgfqpoint{2.188235in}{0.972632in}}%
\pgfusepath{clip}%
\pgfsetbuttcap%
\pgfsetmiterjoin%
\definecolor{currentfill}{rgb}{0.121569,0.466667,0.705882}%
\pgfsetfillcolor{currentfill}%
\pgfsetlinewidth{0.000000pt}%
\definecolor{currentstroke}{rgb}{0.000000,0.000000,0.000000}%
\pgfsetstrokecolor{currentstroke}%
\pgfsetstrokeopacity{0.000000}%
\pgfsetdash{}{0pt}%
\pgfpathmoveto{\pgfqpoint{-319.224843in}{4.520664in}}%
\pgfpathlineto{\pgfqpoint{8.314924in}{4.520664in}}%
\pgfpathlineto{\pgfqpoint{8.314924in}{4.527752in}}%
\pgfpathlineto{\pgfqpoint{-319.224843in}{4.527752in}}%
\pgfpathclose%
\pgfusepath{fill}%
\end{pgfscope}%
\begin{pgfscope}%
\pgfpathrectangle{\pgfqpoint{10.668400in}{4.237239in}}{\pgfqpoint{2.188235in}{0.972632in}}%
\pgfusepath{clip}%
\pgfsetbuttcap%
\pgfsetmiterjoin%
\definecolor{currentfill}{rgb}{0.121569,0.466667,0.705882}%
\pgfsetfillcolor{currentfill}%
\pgfsetlinewidth{0.000000pt}%
\definecolor{currentstroke}{rgb}{0.000000,0.000000,0.000000}%
\pgfsetstrokecolor{currentstroke}%
\pgfsetstrokeopacity{0.000000}%
\pgfsetdash{}{0pt}%
\pgfpathmoveto{\pgfqpoint{-319.224843in}{4.529524in}}%
\pgfpathlineto{\pgfqpoint{8.510217in}{4.529524in}}%
\pgfpathlineto{\pgfqpoint{8.510217in}{4.536612in}}%
\pgfpathlineto{\pgfqpoint{-319.224843in}{4.536612in}}%
\pgfpathclose%
\pgfusepath{fill}%
\end{pgfscope}%
\begin{pgfscope}%
\pgfpathrectangle{\pgfqpoint{10.668400in}{4.237239in}}{\pgfqpoint{2.188235in}{0.972632in}}%
\pgfusepath{clip}%
\pgfsetbuttcap%
\pgfsetmiterjoin%
\definecolor{currentfill}{rgb}{0.121569,0.466667,0.705882}%
\pgfsetfillcolor{currentfill}%
\pgfsetlinewidth{0.000000pt}%
\definecolor{currentstroke}{rgb}{0.000000,0.000000,0.000000}%
\pgfsetstrokecolor{currentstroke}%
\pgfsetstrokeopacity{0.000000}%
\pgfsetdash{}{0pt}%
\pgfpathmoveto{\pgfqpoint{-319.224843in}{4.538384in}}%
\pgfpathlineto{\pgfqpoint{8.057884in}{4.538384in}}%
\pgfpathlineto{\pgfqpoint{8.057884in}{4.545472in}}%
\pgfpathlineto{\pgfqpoint{-319.224843in}{4.545472in}}%
\pgfpathclose%
\pgfusepath{fill}%
\end{pgfscope}%
\begin{pgfscope}%
\pgfpathrectangle{\pgfqpoint{10.668400in}{4.237239in}}{\pgfqpoint{2.188235in}{0.972632in}}%
\pgfusepath{clip}%
\pgfsetbuttcap%
\pgfsetmiterjoin%
\definecolor{currentfill}{rgb}{0.121569,0.466667,0.705882}%
\pgfsetfillcolor{currentfill}%
\pgfsetlinewidth{0.000000pt}%
\definecolor{currentstroke}{rgb}{0.000000,0.000000,0.000000}%
\pgfsetstrokecolor{currentstroke}%
\pgfsetstrokeopacity{0.000000}%
\pgfsetdash{}{0pt}%
\pgfpathmoveto{\pgfqpoint{-319.224843in}{4.547244in}}%
\pgfpathlineto{\pgfqpoint{8.311188in}{4.547244in}}%
\pgfpathlineto{\pgfqpoint{8.311188in}{4.554332in}}%
\pgfpathlineto{\pgfqpoint{-319.224843in}{4.554332in}}%
\pgfpathclose%
\pgfusepath{fill}%
\end{pgfscope}%
\begin{pgfscope}%
\pgfpathrectangle{\pgfqpoint{10.668400in}{4.237239in}}{\pgfqpoint{2.188235in}{0.972632in}}%
\pgfusepath{clip}%
\pgfsetbuttcap%
\pgfsetmiterjoin%
\definecolor{currentfill}{rgb}{0.121569,0.466667,0.705882}%
\pgfsetfillcolor{currentfill}%
\pgfsetlinewidth{0.000000pt}%
\definecolor{currentstroke}{rgb}{0.000000,0.000000,0.000000}%
\pgfsetstrokecolor{currentstroke}%
\pgfsetstrokeopacity{0.000000}%
\pgfsetdash{}{0pt}%
\pgfpathmoveto{\pgfqpoint{-319.224843in}{4.556104in}}%
\pgfpathlineto{\pgfqpoint{8.286967in}{4.556104in}}%
\pgfpathlineto{\pgfqpoint{8.286967in}{4.563192in}}%
\pgfpathlineto{\pgfqpoint{-319.224843in}{4.563192in}}%
\pgfpathclose%
\pgfusepath{fill}%
\end{pgfscope}%
\begin{pgfscope}%
\pgfpathrectangle{\pgfqpoint{10.668400in}{4.237239in}}{\pgfqpoint{2.188235in}{0.972632in}}%
\pgfusepath{clip}%
\pgfsetbuttcap%
\pgfsetmiterjoin%
\definecolor{currentfill}{rgb}{0.121569,0.466667,0.705882}%
\pgfsetfillcolor{currentfill}%
\pgfsetlinewidth{0.000000pt}%
\definecolor{currentstroke}{rgb}{0.000000,0.000000,0.000000}%
\pgfsetstrokecolor{currentstroke}%
\pgfsetstrokeopacity{0.000000}%
\pgfsetdash{}{0pt}%
\pgfpathmoveto{\pgfqpoint{-319.224843in}{4.564964in}}%
\pgfpathlineto{\pgfqpoint{8.262149in}{4.564964in}}%
\pgfpathlineto{\pgfqpoint{8.262149in}{4.572051in}}%
\pgfpathlineto{\pgfqpoint{-319.224843in}{4.572051in}}%
\pgfpathclose%
\pgfusepath{fill}%
\end{pgfscope}%
\begin{pgfscope}%
\pgfpathrectangle{\pgfqpoint{10.668400in}{4.237239in}}{\pgfqpoint{2.188235in}{0.972632in}}%
\pgfusepath{clip}%
\pgfsetbuttcap%
\pgfsetmiterjoin%
\definecolor{currentfill}{rgb}{0.121569,0.466667,0.705882}%
\pgfsetfillcolor{currentfill}%
\pgfsetlinewidth{0.000000pt}%
\definecolor{currentstroke}{rgb}{0.000000,0.000000,0.000000}%
\pgfsetstrokecolor{currentstroke}%
\pgfsetstrokeopacity{0.000000}%
\pgfsetdash{}{0pt}%
\pgfpathmoveto{\pgfqpoint{-319.224843in}{4.573823in}}%
\pgfpathlineto{\pgfqpoint{7.902127in}{4.573823in}}%
\pgfpathlineto{\pgfqpoint{7.902127in}{4.580911in}}%
\pgfpathlineto{\pgfqpoint{-319.224843in}{4.580911in}}%
\pgfpathclose%
\pgfusepath{fill}%
\end{pgfscope}%
\begin{pgfscope}%
\pgfpathrectangle{\pgfqpoint{10.668400in}{4.237239in}}{\pgfqpoint{2.188235in}{0.972632in}}%
\pgfusepath{clip}%
\pgfsetbuttcap%
\pgfsetmiterjoin%
\definecolor{currentfill}{rgb}{0.121569,0.466667,0.705882}%
\pgfsetfillcolor{currentfill}%
\pgfsetlinewidth{0.000000pt}%
\definecolor{currentstroke}{rgb}{0.000000,0.000000,0.000000}%
\pgfsetstrokecolor{currentstroke}%
\pgfsetstrokeopacity{0.000000}%
\pgfsetdash{}{0pt}%
\pgfpathmoveto{\pgfqpoint{-319.224843in}{4.582683in}}%
\pgfpathlineto{\pgfqpoint{8.184358in}{4.582683in}}%
\pgfpathlineto{\pgfqpoint{8.184358in}{4.589771in}}%
\pgfpathlineto{\pgfqpoint{-319.224843in}{4.589771in}}%
\pgfpathclose%
\pgfusepath{fill}%
\end{pgfscope}%
\begin{pgfscope}%
\pgfpathrectangle{\pgfqpoint{10.668400in}{4.237239in}}{\pgfqpoint{2.188235in}{0.972632in}}%
\pgfusepath{clip}%
\pgfsetbuttcap%
\pgfsetmiterjoin%
\definecolor{currentfill}{rgb}{0.121569,0.466667,0.705882}%
\pgfsetfillcolor{currentfill}%
\pgfsetlinewidth{0.000000pt}%
\definecolor{currentstroke}{rgb}{0.000000,0.000000,0.000000}%
\pgfsetstrokecolor{currentstroke}%
\pgfsetstrokeopacity{0.000000}%
\pgfsetdash{}{0pt}%
\pgfpathmoveto{\pgfqpoint{-319.224843in}{4.591543in}}%
\pgfpathlineto{\pgfqpoint{8.181887in}{4.591543in}}%
\pgfpathlineto{\pgfqpoint{8.181887in}{4.598631in}}%
\pgfpathlineto{\pgfqpoint{-319.224843in}{4.598631in}}%
\pgfpathclose%
\pgfusepath{fill}%
\end{pgfscope}%
\begin{pgfscope}%
\pgfpathrectangle{\pgfqpoint{10.668400in}{4.237239in}}{\pgfqpoint{2.188235in}{0.972632in}}%
\pgfusepath{clip}%
\pgfsetbuttcap%
\pgfsetmiterjoin%
\definecolor{currentfill}{rgb}{0.121569,0.466667,0.705882}%
\pgfsetfillcolor{currentfill}%
\pgfsetlinewidth{0.000000pt}%
\definecolor{currentstroke}{rgb}{0.000000,0.000000,0.000000}%
\pgfsetstrokecolor{currentstroke}%
\pgfsetstrokeopacity{0.000000}%
\pgfsetdash{}{0pt}%
\pgfpathmoveto{\pgfqpoint{-319.224843in}{4.600403in}}%
\pgfpathlineto{\pgfqpoint{8.144257in}{4.600403in}}%
\pgfpathlineto{\pgfqpoint{8.144257in}{4.607491in}}%
\pgfpathlineto{\pgfqpoint{-319.224843in}{4.607491in}}%
\pgfpathclose%
\pgfusepath{fill}%
\end{pgfscope}%
\begin{pgfscope}%
\pgfpathrectangle{\pgfqpoint{10.668400in}{4.237239in}}{\pgfqpoint{2.188235in}{0.972632in}}%
\pgfusepath{clip}%
\pgfsetbuttcap%
\pgfsetmiterjoin%
\definecolor{currentfill}{rgb}{0.121569,0.466667,0.705882}%
\pgfsetfillcolor{currentfill}%
\pgfsetlinewidth{0.000000pt}%
\definecolor{currentstroke}{rgb}{0.000000,0.000000,0.000000}%
\pgfsetstrokecolor{currentstroke}%
\pgfsetstrokeopacity{0.000000}%
\pgfsetdash{}{0pt}%
\pgfpathmoveto{\pgfqpoint{-319.224843in}{4.609263in}}%
\pgfpathlineto{\pgfqpoint{8.206374in}{4.609263in}}%
\pgfpathlineto{\pgfqpoint{8.206374in}{4.616351in}}%
\pgfpathlineto{\pgfqpoint{-319.224843in}{4.616351in}}%
\pgfpathclose%
\pgfusepath{fill}%
\end{pgfscope}%
\begin{pgfscope}%
\pgfpathrectangle{\pgfqpoint{10.668400in}{4.237239in}}{\pgfqpoint{2.188235in}{0.972632in}}%
\pgfusepath{clip}%
\pgfsetbuttcap%
\pgfsetmiterjoin%
\definecolor{currentfill}{rgb}{0.121569,0.466667,0.705882}%
\pgfsetfillcolor{currentfill}%
\pgfsetlinewidth{0.000000pt}%
\definecolor{currentstroke}{rgb}{0.000000,0.000000,0.000000}%
\pgfsetstrokecolor{currentstroke}%
\pgfsetstrokeopacity{0.000000}%
\pgfsetdash{}{0pt}%
\pgfpathmoveto{\pgfqpoint{-319.224843in}{4.618123in}}%
\pgfpathlineto{\pgfqpoint{7.939347in}{4.618123in}}%
\pgfpathlineto{\pgfqpoint{7.939347in}{4.625210in}}%
\pgfpathlineto{\pgfqpoint{-319.224843in}{4.625210in}}%
\pgfpathclose%
\pgfusepath{fill}%
\end{pgfscope}%
\begin{pgfscope}%
\pgfpathrectangle{\pgfqpoint{10.668400in}{4.237239in}}{\pgfqpoint{2.188235in}{0.972632in}}%
\pgfusepath{clip}%
\pgfsetbuttcap%
\pgfsetmiterjoin%
\definecolor{currentfill}{rgb}{0.121569,0.466667,0.705882}%
\pgfsetfillcolor{currentfill}%
\pgfsetlinewidth{0.000000pt}%
\definecolor{currentstroke}{rgb}{0.000000,0.000000,0.000000}%
\pgfsetstrokecolor{currentstroke}%
\pgfsetstrokeopacity{0.000000}%
\pgfsetdash{}{0pt}%
\pgfpathmoveto{\pgfqpoint{-319.224843in}{4.626982in}}%
\pgfpathlineto{\pgfqpoint{8.260348in}{4.626982in}}%
\pgfpathlineto{\pgfqpoint{8.260348in}{4.634070in}}%
\pgfpathlineto{\pgfqpoint{-319.224843in}{4.634070in}}%
\pgfpathclose%
\pgfusepath{fill}%
\end{pgfscope}%
\begin{pgfscope}%
\pgfpathrectangle{\pgfqpoint{10.668400in}{4.237239in}}{\pgfqpoint{2.188235in}{0.972632in}}%
\pgfusepath{clip}%
\pgfsetbuttcap%
\pgfsetmiterjoin%
\definecolor{currentfill}{rgb}{0.121569,0.466667,0.705882}%
\pgfsetfillcolor{currentfill}%
\pgfsetlinewidth{0.000000pt}%
\definecolor{currentstroke}{rgb}{0.000000,0.000000,0.000000}%
\pgfsetstrokecolor{currentstroke}%
\pgfsetstrokeopacity{0.000000}%
\pgfsetdash{}{0pt}%
\pgfpathmoveto{\pgfqpoint{-319.224843in}{4.635842in}}%
\pgfpathlineto{\pgfqpoint{8.281660in}{4.635842in}}%
\pgfpathlineto{\pgfqpoint{8.281660in}{4.642930in}}%
\pgfpathlineto{\pgfqpoint{-319.224843in}{4.642930in}}%
\pgfpathclose%
\pgfusepath{fill}%
\end{pgfscope}%
\begin{pgfscope}%
\pgfpathrectangle{\pgfqpoint{10.668400in}{4.237239in}}{\pgfqpoint{2.188235in}{0.972632in}}%
\pgfusepath{clip}%
\pgfsetbuttcap%
\pgfsetmiterjoin%
\definecolor{currentfill}{rgb}{0.121569,0.466667,0.705882}%
\pgfsetfillcolor{currentfill}%
\pgfsetlinewidth{0.000000pt}%
\definecolor{currentstroke}{rgb}{0.000000,0.000000,0.000000}%
\pgfsetstrokecolor{currentstroke}%
\pgfsetstrokeopacity{0.000000}%
\pgfsetdash{}{0pt}%
\pgfpathmoveto{\pgfqpoint{-319.224843in}{4.644702in}}%
\pgfpathlineto{\pgfqpoint{8.225310in}{4.644702in}}%
\pgfpathlineto{\pgfqpoint{8.225310in}{4.651790in}}%
\pgfpathlineto{\pgfqpoint{-319.224843in}{4.651790in}}%
\pgfpathclose%
\pgfusepath{fill}%
\end{pgfscope}%
\begin{pgfscope}%
\pgfpathrectangle{\pgfqpoint{10.668400in}{4.237239in}}{\pgfqpoint{2.188235in}{0.972632in}}%
\pgfusepath{clip}%
\pgfsetbuttcap%
\pgfsetmiterjoin%
\definecolor{currentfill}{rgb}{0.121569,0.466667,0.705882}%
\pgfsetfillcolor{currentfill}%
\pgfsetlinewidth{0.000000pt}%
\definecolor{currentstroke}{rgb}{0.000000,0.000000,0.000000}%
\pgfsetstrokecolor{currentstroke}%
\pgfsetstrokeopacity{0.000000}%
\pgfsetdash{}{0pt}%
\pgfpathmoveto{\pgfqpoint{-319.224843in}{4.653562in}}%
\pgfpathlineto{\pgfqpoint{8.308058in}{4.653562in}}%
\pgfpathlineto{\pgfqpoint{8.308058in}{4.660650in}}%
\pgfpathlineto{\pgfqpoint{-319.224843in}{4.660650in}}%
\pgfpathclose%
\pgfusepath{fill}%
\end{pgfscope}%
\begin{pgfscope}%
\pgfpathrectangle{\pgfqpoint{10.668400in}{4.237239in}}{\pgfqpoint{2.188235in}{0.972632in}}%
\pgfusepath{clip}%
\pgfsetbuttcap%
\pgfsetmiterjoin%
\definecolor{currentfill}{rgb}{0.121569,0.466667,0.705882}%
\pgfsetfillcolor{currentfill}%
\pgfsetlinewidth{0.000000pt}%
\definecolor{currentstroke}{rgb}{0.000000,0.000000,0.000000}%
\pgfsetstrokecolor{currentstroke}%
\pgfsetstrokeopacity{0.000000}%
\pgfsetdash{}{0pt}%
\pgfpathmoveto{\pgfqpoint{-319.224843in}{4.662422in}}%
\pgfpathlineto{\pgfqpoint{8.150063in}{4.662422in}}%
\pgfpathlineto{\pgfqpoint{8.150063in}{4.669510in}}%
\pgfpathlineto{\pgfqpoint{-319.224843in}{4.669510in}}%
\pgfpathclose%
\pgfusepath{fill}%
\end{pgfscope}%
\begin{pgfscope}%
\pgfpathrectangle{\pgfqpoint{10.668400in}{4.237239in}}{\pgfqpoint{2.188235in}{0.972632in}}%
\pgfusepath{clip}%
\pgfsetbuttcap%
\pgfsetmiterjoin%
\definecolor{currentfill}{rgb}{0.121569,0.466667,0.705882}%
\pgfsetfillcolor{currentfill}%
\pgfsetlinewidth{0.000000pt}%
\definecolor{currentstroke}{rgb}{0.000000,0.000000,0.000000}%
\pgfsetstrokecolor{currentstroke}%
\pgfsetstrokeopacity{0.000000}%
\pgfsetdash{}{0pt}%
\pgfpathmoveto{\pgfqpoint{-319.224843in}{4.671282in}}%
\pgfpathlineto{\pgfqpoint{8.132131in}{4.671282in}}%
\pgfpathlineto{\pgfqpoint{8.132131in}{4.678369in}}%
\pgfpathlineto{\pgfqpoint{-319.224843in}{4.678369in}}%
\pgfpathclose%
\pgfusepath{fill}%
\end{pgfscope}%
\begin{pgfscope}%
\pgfpathrectangle{\pgfqpoint{10.668400in}{4.237239in}}{\pgfqpoint{2.188235in}{0.972632in}}%
\pgfusepath{clip}%
\pgfsetbuttcap%
\pgfsetmiterjoin%
\definecolor{currentfill}{rgb}{0.121569,0.466667,0.705882}%
\pgfsetfillcolor{currentfill}%
\pgfsetlinewidth{0.000000pt}%
\definecolor{currentstroke}{rgb}{0.000000,0.000000,0.000000}%
\pgfsetstrokecolor{currentstroke}%
\pgfsetstrokeopacity{0.000000}%
\pgfsetdash{}{0pt}%
\pgfpathmoveto{\pgfqpoint{-319.224843in}{4.680141in}}%
\pgfpathlineto{\pgfqpoint{8.117045in}{4.680141in}}%
\pgfpathlineto{\pgfqpoint{8.117045in}{4.687229in}}%
\pgfpathlineto{\pgfqpoint{-319.224843in}{4.687229in}}%
\pgfpathclose%
\pgfusepath{fill}%
\end{pgfscope}%
\begin{pgfscope}%
\pgfpathrectangle{\pgfqpoint{10.668400in}{4.237239in}}{\pgfqpoint{2.188235in}{0.972632in}}%
\pgfusepath{clip}%
\pgfsetbuttcap%
\pgfsetmiterjoin%
\definecolor{currentfill}{rgb}{0.121569,0.466667,0.705882}%
\pgfsetfillcolor{currentfill}%
\pgfsetlinewidth{0.000000pt}%
\definecolor{currentstroke}{rgb}{0.000000,0.000000,0.000000}%
\pgfsetstrokecolor{currentstroke}%
\pgfsetstrokeopacity{0.000000}%
\pgfsetdash{}{0pt}%
\pgfpathmoveto{\pgfqpoint{-319.224843in}{4.689001in}}%
\pgfpathlineto{\pgfqpoint{7.700202in}{4.689001in}}%
\pgfpathlineto{\pgfqpoint{7.700202in}{4.696089in}}%
\pgfpathlineto{\pgfqpoint{-319.224843in}{4.696089in}}%
\pgfpathclose%
\pgfusepath{fill}%
\end{pgfscope}%
\begin{pgfscope}%
\pgfpathrectangle{\pgfqpoint{10.668400in}{4.237239in}}{\pgfqpoint{2.188235in}{0.972632in}}%
\pgfusepath{clip}%
\pgfsetbuttcap%
\pgfsetmiterjoin%
\definecolor{currentfill}{rgb}{0.121569,0.466667,0.705882}%
\pgfsetfillcolor{currentfill}%
\pgfsetlinewidth{0.000000pt}%
\definecolor{currentstroke}{rgb}{0.000000,0.000000,0.000000}%
\pgfsetstrokecolor{currentstroke}%
\pgfsetstrokeopacity{0.000000}%
\pgfsetdash{}{0pt}%
\pgfpathmoveto{\pgfqpoint{-319.224843in}{4.697861in}}%
\pgfpathlineto{\pgfqpoint{8.206089in}{4.697861in}}%
\pgfpathlineto{\pgfqpoint{8.206089in}{4.704949in}}%
\pgfpathlineto{\pgfqpoint{-319.224843in}{4.704949in}}%
\pgfpathclose%
\pgfusepath{fill}%
\end{pgfscope}%
\begin{pgfscope}%
\pgfpathrectangle{\pgfqpoint{10.668400in}{4.237239in}}{\pgfqpoint{2.188235in}{0.972632in}}%
\pgfusepath{clip}%
\pgfsetbuttcap%
\pgfsetmiterjoin%
\definecolor{currentfill}{rgb}{0.121569,0.466667,0.705882}%
\pgfsetfillcolor{currentfill}%
\pgfsetlinewidth{0.000000pt}%
\definecolor{currentstroke}{rgb}{0.000000,0.000000,0.000000}%
\pgfsetstrokecolor{currentstroke}%
\pgfsetstrokeopacity{0.000000}%
\pgfsetdash{}{0pt}%
\pgfpathmoveto{\pgfqpoint{-319.224843in}{4.706721in}}%
\pgfpathlineto{\pgfqpoint{8.338210in}{4.706721in}}%
\pgfpathlineto{\pgfqpoint{8.338210in}{4.713809in}}%
\pgfpathlineto{\pgfqpoint{-319.224843in}{4.713809in}}%
\pgfpathclose%
\pgfusepath{fill}%
\end{pgfscope}%
\begin{pgfscope}%
\pgfpathrectangle{\pgfqpoint{10.668400in}{4.237239in}}{\pgfqpoint{2.188235in}{0.972632in}}%
\pgfusepath{clip}%
\pgfsetbuttcap%
\pgfsetmiterjoin%
\definecolor{currentfill}{rgb}{0.121569,0.466667,0.705882}%
\pgfsetfillcolor{currentfill}%
\pgfsetlinewidth{0.000000pt}%
\definecolor{currentstroke}{rgb}{0.000000,0.000000,0.000000}%
\pgfsetstrokecolor{currentstroke}%
\pgfsetstrokeopacity{0.000000}%
\pgfsetdash{}{0pt}%
\pgfpathmoveto{\pgfqpoint{-319.224843in}{4.715581in}}%
\pgfpathlineto{\pgfqpoint{8.251824in}{4.715581in}}%
\pgfpathlineto{\pgfqpoint{8.251824in}{4.722669in}}%
\pgfpathlineto{\pgfqpoint{-319.224843in}{4.722669in}}%
\pgfpathclose%
\pgfusepath{fill}%
\end{pgfscope}%
\begin{pgfscope}%
\pgfpathrectangle{\pgfqpoint{10.668400in}{4.237239in}}{\pgfqpoint{2.188235in}{0.972632in}}%
\pgfusepath{clip}%
\pgfsetbuttcap%
\pgfsetmiterjoin%
\definecolor{currentfill}{rgb}{0.121569,0.466667,0.705882}%
\pgfsetfillcolor{currentfill}%
\pgfsetlinewidth{0.000000pt}%
\definecolor{currentstroke}{rgb}{0.000000,0.000000,0.000000}%
\pgfsetstrokecolor{currentstroke}%
\pgfsetstrokeopacity{0.000000}%
\pgfsetdash{}{0pt}%
\pgfpathmoveto{\pgfqpoint{-319.224843in}{4.724440in}}%
\pgfpathlineto{\pgfqpoint{8.375218in}{4.724440in}}%
\pgfpathlineto{\pgfqpoint{8.375218in}{4.731528in}}%
\pgfpathlineto{\pgfqpoint{-319.224843in}{4.731528in}}%
\pgfpathclose%
\pgfusepath{fill}%
\end{pgfscope}%
\begin{pgfscope}%
\pgfpathrectangle{\pgfqpoint{10.668400in}{4.237239in}}{\pgfqpoint{2.188235in}{0.972632in}}%
\pgfusepath{clip}%
\pgfsetbuttcap%
\pgfsetmiterjoin%
\definecolor{currentfill}{rgb}{0.121569,0.466667,0.705882}%
\pgfsetfillcolor{currentfill}%
\pgfsetlinewidth{0.000000pt}%
\definecolor{currentstroke}{rgb}{0.000000,0.000000,0.000000}%
\pgfsetstrokecolor{currentstroke}%
\pgfsetstrokeopacity{0.000000}%
\pgfsetdash{}{0pt}%
\pgfpathmoveto{\pgfqpoint{-319.224843in}{4.733300in}}%
\pgfpathlineto{\pgfqpoint{7.911269in}{4.733300in}}%
\pgfpathlineto{\pgfqpoint{7.911269in}{4.740388in}}%
\pgfpathlineto{\pgfqpoint{-319.224843in}{4.740388in}}%
\pgfpathclose%
\pgfusepath{fill}%
\end{pgfscope}%
\begin{pgfscope}%
\pgfpathrectangle{\pgfqpoint{10.668400in}{4.237239in}}{\pgfqpoint{2.188235in}{0.972632in}}%
\pgfusepath{clip}%
\pgfsetbuttcap%
\pgfsetmiterjoin%
\definecolor{currentfill}{rgb}{0.121569,0.466667,0.705882}%
\pgfsetfillcolor{currentfill}%
\pgfsetlinewidth{0.000000pt}%
\definecolor{currentstroke}{rgb}{0.000000,0.000000,0.000000}%
\pgfsetstrokecolor{currentstroke}%
\pgfsetstrokeopacity{0.000000}%
\pgfsetdash{}{0pt}%
\pgfpathmoveto{\pgfqpoint{-319.224843in}{4.742160in}}%
\pgfpathlineto{\pgfqpoint{8.321605in}{4.742160in}}%
\pgfpathlineto{\pgfqpoint{8.321605in}{4.749248in}}%
\pgfpathlineto{\pgfqpoint{-319.224843in}{4.749248in}}%
\pgfpathclose%
\pgfusepath{fill}%
\end{pgfscope}%
\begin{pgfscope}%
\pgfpathrectangle{\pgfqpoint{10.668400in}{4.237239in}}{\pgfqpoint{2.188235in}{0.972632in}}%
\pgfusepath{clip}%
\pgfsetbuttcap%
\pgfsetmiterjoin%
\definecolor{currentfill}{rgb}{0.121569,0.466667,0.705882}%
\pgfsetfillcolor{currentfill}%
\pgfsetlinewidth{0.000000pt}%
\definecolor{currentstroke}{rgb}{0.000000,0.000000,0.000000}%
\pgfsetstrokecolor{currentstroke}%
\pgfsetstrokeopacity{0.000000}%
\pgfsetdash{}{0pt}%
\pgfpathmoveto{\pgfqpoint{-319.224843in}{4.751020in}}%
\pgfpathlineto{\pgfqpoint{8.126036in}{4.751020in}}%
\pgfpathlineto{\pgfqpoint{8.126036in}{4.758108in}}%
\pgfpathlineto{\pgfqpoint{-319.224843in}{4.758108in}}%
\pgfpathclose%
\pgfusepath{fill}%
\end{pgfscope}%
\begin{pgfscope}%
\pgfpathrectangle{\pgfqpoint{10.668400in}{4.237239in}}{\pgfqpoint{2.188235in}{0.972632in}}%
\pgfusepath{clip}%
\pgfsetbuttcap%
\pgfsetmiterjoin%
\definecolor{currentfill}{rgb}{0.121569,0.466667,0.705882}%
\pgfsetfillcolor{currentfill}%
\pgfsetlinewidth{0.000000pt}%
\definecolor{currentstroke}{rgb}{0.000000,0.000000,0.000000}%
\pgfsetstrokecolor{currentstroke}%
\pgfsetstrokeopacity{0.000000}%
\pgfsetdash{}{0pt}%
\pgfpathmoveto{\pgfqpoint{-319.224843in}{4.759880in}}%
\pgfpathlineto{\pgfqpoint{7.981664in}{4.759880in}}%
\pgfpathlineto{\pgfqpoint{7.981664in}{4.766968in}}%
\pgfpathlineto{\pgfqpoint{-319.224843in}{4.766968in}}%
\pgfpathclose%
\pgfusepath{fill}%
\end{pgfscope}%
\begin{pgfscope}%
\pgfpathrectangle{\pgfqpoint{10.668400in}{4.237239in}}{\pgfqpoint{2.188235in}{0.972632in}}%
\pgfusepath{clip}%
\pgfsetbuttcap%
\pgfsetmiterjoin%
\definecolor{currentfill}{rgb}{0.121569,0.466667,0.705882}%
\pgfsetfillcolor{currentfill}%
\pgfsetlinewidth{0.000000pt}%
\definecolor{currentstroke}{rgb}{0.000000,0.000000,0.000000}%
\pgfsetstrokecolor{currentstroke}%
\pgfsetstrokeopacity{0.000000}%
\pgfsetdash{}{0pt}%
\pgfpathmoveto{\pgfqpoint{-319.224843in}{4.768740in}}%
\pgfpathlineto{\pgfqpoint{8.139120in}{4.768740in}}%
\pgfpathlineto{\pgfqpoint{8.139120in}{4.775827in}}%
\pgfpathlineto{\pgfqpoint{-319.224843in}{4.775827in}}%
\pgfpathclose%
\pgfusepath{fill}%
\end{pgfscope}%
\begin{pgfscope}%
\pgfpathrectangle{\pgfqpoint{10.668400in}{4.237239in}}{\pgfqpoint{2.188235in}{0.972632in}}%
\pgfusepath{clip}%
\pgfsetbuttcap%
\pgfsetmiterjoin%
\definecolor{currentfill}{rgb}{0.121569,0.466667,0.705882}%
\pgfsetfillcolor{currentfill}%
\pgfsetlinewidth{0.000000pt}%
\definecolor{currentstroke}{rgb}{0.000000,0.000000,0.000000}%
\pgfsetstrokecolor{currentstroke}%
\pgfsetstrokeopacity{0.000000}%
\pgfsetdash{}{0pt}%
\pgfpathmoveto{\pgfqpoint{-319.224843in}{4.777599in}}%
\pgfpathlineto{\pgfqpoint{8.722224in}{4.777599in}}%
\pgfpathlineto{\pgfqpoint{8.722224in}{4.784687in}}%
\pgfpathlineto{\pgfqpoint{-319.224843in}{4.784687in}}%
\pgfpathclose%
\pgfusepath{fill}%
\end{pgfscope}%
\begin{pgfscope}%
\pgfpathrectangle{\pgfqpoint{10.668400in}{4.237239in}}{\pgfqpoint{2.188235in}{0.972632in}}%
\pgfusepath{clip}%
\pgfsetbuttcap%
\pgfsetmiterjoin%
\definecolor{currentfill}{rgb}{0.121569,0.466667,0.705882}%
\pgfsetfillcolor{currentfill}%
\pgfsetlinewidth{0.000000pt}%
\definecolor{currentstroke}{rgb}{0.000000,0.000000,0.000000}%
\pgfsetstrokecolor{currentstroke}%
\pgfsetstrokeopacity{0.000000}%
\pgfsetdash{}{0pt}%
\pgfpathmoveto{\pgfqpoint{-319.224843in}{4.786459in}}%
\pgfpathlineto{\pgfqpoint{8.096874in}{4.786459in}}%
\pgfpathlineto{\pgfqpoint{8.096874in}{4.793547in}}%
\pgfpathlineto{\pgfqpoint{-319.224843in}{4.793547in}}%
\pgfpathclose%
\pgfusepath{fill}%
\end{pgfscope}%
\begin{pgfscope}%
\pgfpathrectangle{\pgfqpoint{10.668400in}{4.237239in}}{\pgfqpoint{2.188235in}{0.972632in}}%
\pgfusepath{clip}%
\pgfsetbuttcap%
\pgfsetmiterjoin%
\definecolor{currentfill}{rgb}{0.121569,0.466667,0.705882}%
\pgfsetfillcolor{currentfill}%
\pgfsetlinewidth{0.000000pt}%
\definecolor{currentstroke}{rgb}{0.000000,0.000000,0.000000}%
\pgfsetstrokecolor{currentstroke}%
\pgfsetstrokeopacity{0.000000}%
\pgfsetdash{}{0pt}%
\pgfpathmoveto{\pgfqpoint{-319.224843in}{4.795319in}}%
\pgfpathlineto{\pgfqpoint{8.307283in}{4.795319in}}%
\pgfpathlineto{\pgfqpoint{8.307283in}{4.802407in}}%
\pgfpathlineto{\pgfqpoint{-319.224843in}{4.802407in}}%
\pgfpathclose%
\pgfusepath{fill}%
\end{pgfscope}%
\begin{pgfscope}%
\pgfpathrectangle{\pgfqpoint{10.668400in}{4.237239in}}{\pgfqpoint{2.188235in}{0.972632in}}%
\pgfusepath{clip}%
\pgfsetbuttcap%
\pgfsetmiterjoin%
\definecolor{currentfill}{rgb}{0.121569,0.466667,0.705882}%
\pgfsetfillcolor{currentfill}%
\pgfsetlinewidth{0.000000pt}%
\definecolor{currentstroke}{rgb}{0.000000,0.000000,0.000000}%
\pgfsetstrokecolor{currentstroke}%
\pgfsetstrokeopacity{0.000000}%
\pgfsetdash{}{0pt}%
\pgfpathmoveto{\pgfqpoint{-319.224843in}{4.804179in}}%
\pgfpathlineto{\pgfqpoint{8.374865in}{4.804179in}}%
\pgfpathlineto{\pgfqpoint{8.374865in}{4.811267in}}%
\pgfpathlineto{\pgfqpoint{-319.224843in}{4.811267in}}%
\pgfpathclose%
\pgfusepath{fill}%
\end{pgfscope}%
\begin{pgfscope}%
\pgfpathrectangle{\pgfqpoint{10.668400in}{4.237239in}}{\pgfqpoint{2.188235in}{0.972632in}}%
\pgfusepath{clip}%
\pgfsetbuttcap%
\pgfsetmiterjoin%
\definecolor{currentfill}{rgb}{0.121569,0.466667,0.705882}%
\pgfsetfillcolor{currentfill}%
\pgfsetlinewidth{0.000000pt}%
\definecolor{currentstroke}{rgb}{0.000000,0.000000,0.000000}%
\pgfsetstrokecolor{currentstroke}%
\pgfsetstrokeopacity{0.000000}%
\pgfsetdash{}{0pt}%
\pgfpathmoveto{\pgfqpoint{-319.224843in}{4.813039in}}%
\pgfpathlineto{\pgfqpoint{8.517390in}{4.813039in}}%
\pgfpathlineto{\pgfqpoint{8.517390in}{4.820127in}}%
\pgfpathlineto{\pgfqpoint{-319.224843in}{4.820127in}}%
\pgfpathclose%
\pgfusepath{fill}%
\end{pgfscope}%
\begin{pgfscope}%
\pgfpathrectangle{\pgfqpoint{10.668400in}{4.237239in}}{\pgfqpoint{2.188235in}{0.972632in}}%
\pgfusepath{clip}%
\pgfsetbuttcap%
\pgfsetmiterjoin%
\definecolor{currentfill}{rgb}{0.121569,0.466667,0.705882}%
\pgfsetfillcolor{currentfill}%
\pgfsetlinewidth{0.000000pt}%
\definecolor{currentstroke}{rgb}{0.000000,0.000000,0.000000}%
\pgfsetstrokecolor{currentstroke}%
\pgfsetstrokeopacity{0.000000}%
\pgfsetdash{}{0pt}%
\pgfpathmoveto{\pgfqpoint{-319.224843in}{4.821899in}}%
\pgfpathlineto{\pgfqpoint{8.280813in}{4.821899in}}%
\pgfpathlineto{\pgfqpoint{8.280813in}{4.828986in}}%
\pgfpathlineto{\pgfqpoint{-319.224843in}{4.828986in}}%
\pgfpathclose%
\pgfusepath{fill}%
\end{pgfscope}%
\begin{pgfscope}%
\pgfpathrectangle{\pgfqpoint{10.668400in}{4.237239in}}{\pgfqpoint{2.188235in}{0.972632in}}%
\pgfusepath{clip}%
\pgfsetbuttcap%
\pgfsetmiterjoin%
\definecolor{currentfill}{rgb}{0.121569,0.466667,0.705882}%
\pgfsetfillcolor{currentfill}%
\pgfsetlinewidth{0.000000pt}%
\definecolor{currentstroke}{rgb}{0.000000,0.000000,0.000000}%
\pgfsetstrokecolor{currentstroke}%
\pgfsetstrokeopacity{0.000000}%
\pgfsetdash{}{0pt}%
\pgfpathmoveto{\pgfqpoint{-319.224843in}{4.830758in}}%
\pgfpathlineto{\pgfqpoint{8.299064in}{4.830758in}}%
\pgfpathlineto{\pgfqpoint{8.299064in}{4.837846in}}%
\pgfpathlineto{\pgfqpoint{-319.224843in}{4.837846in}}%
\pgfpathclose%
\pgfusepath{fill}%
\end{pgfscope}%
\begin{pgfscope}%
\pgfpathrectangle{\pgfqpoint{10.668400in}{4.237239in}}{\pgfqpoint{2.188235in}{0.972632in}}%
\pgfusepath{clip}%
\pgfsetbuttcap%
\pgfsetmiterjoin%
\definecolor{currentfill}{rgb}{0.121569,0.466667,0.705882}%
\pgfsetfillcolor{currentfill}%
\pgfsetlinewidth{0.000000pt}%
\definecolor{currentstroke}{rgb}{0.000000,0.000000,0.000000}%
\pgfsetstrokecolor{currentstroke}%
\pgfsetstrokeopacity{0.000000}%
\pgfsetdash{}{0pt}%
\pgfpathmoveto{\pgfqpoint{-319.224843in}{4.839618in}}%
\pgfpathlineto{\pgfqpoint{8.006543in}{4.839618in}}%
\pgfpathlineto{\pgfqpoint{8.006543in}{4.846706in}}%
\pgfpathlineto{\pgfqpoint{-319.224843in}{4.846706in}}%
\pgfpathclose%
\pgfusepath{fill}%
\end{pgfscope}%
\begin{pgfscope}%
\pgfpathrectangle{\pgfqpoint{10.668400in}{4.237239in}}{\pgfqpoint{2.188235in}{0.972632in}}%
\pgfusepath{clip}%
\pgfsetbuttcap%
\pgfsetmiterjoin%
\definecolor{currentfill}{rgb}{0.121569,0.466667,0.705882}%
\pgfsetfillcolor{currentfill}%
\pgfsetlinewidth{0.000000pt}%
\definecolor{currentstroke}{rgb}{0.000000,0.000000,0.000000}%
\pgfsetstrokecolor{currentstroke}%
\pgfsetstrokeopacity{0.000000}%
\pgfsetdash{}{0pt}%
\pgfpathmoveto{\pgfqpoint{-319.224843in}{4.848478in}}%
\pgfpathlineto{\pgfqpoint{8.351285in}{4.848478in}}%
\pgfpathlineto{\pgfqpoint{8.351285in}{4.855566in}}%
\pgfpathlineto{\pgfqpoint{-319.224843in}{4.855566in}}%
\pgfpathclose%
\pgfusepath{fill}%
\end{pgfscope}%
\begin{pgfscope}%
\pgfpathrectangle{\pgfqpoint{10.668400in}{4.237239in}}{\pgfqpoint{2.188235in}{0.972632in}}%
\pgfusepath{clip}%
\pgfsetbuttcap%
\pgfsetmiterjoin%
\definecolor{currentfill}{rgb}{0.121569,0.466667,0.705882}%
\pgfsetfillcolor{currentfill}%
\pgfsetlinewidth{0.000000pt}%
\definecolor{currentstroke}{rgb}{0.000000,0.000000,0.000000}%
\pgfsetstrokecolor{currentstroke}%
\pgfsetstrokeopacity{0.000000}%
\pgfsetdash{}{0pt}%
\pgfpathmoveto{\pgfqpoint{-319.224843in}{4.857338in}}%
\pgfpathlineto{\pgfqpoint{8.233092in}{4.857338in}}%
\pgfpathlineto{\pgfqpoint{8.233092in}{4.864426in}}%
\pgfpathlineto{\pgfqpoint{-319.224843in}{4.864426in}}%
\pgfpathclose%
\pgfusepath{fill}%
\end{pgfscope}%
\begin{pgfscope}%
\pgfpathrectangle{\pgfqpoint{10.668400in}{4.237239in}}{\pgfqpoint{2.188235in}{0.972632in}}%
\pgfusepath{clip}%
\pgfsetbuttcap%
\pgfsetmiterjoin%
\definecolor{currentfill}{rgb}{0.121569,0.466667,0.705882}%
\pgfsetfillcolor{currentfill}%
\pgfsetlinewidth{0.000000pt}%
\definecolor{currentstroke}{rgb}{0.000000,0.000000,0.000000}%
\pgfsetstrokecolor{currentstroke}%
\pgfsetstrokeopacity{0.000000}%
\pgfsetdash{}{0pt}%
\pgfpathmoveto{\pgfqpoint{-319.224843in}{4.866198in}}%
\pgfpathlineto{\pgfqpoint{8.364225in}{4.866198in}}%
\pgfpathlineto{\pgfqpoint{8.364225in}{4.873286in}}%
\pgfpathlineto{\pgfqpoint{-319.224843in}{4.873286in}}%
\pgfpathclose%
\pgfusepath{fill}%
\end{pgfscope}%
\begin{pgfscope}%
\pgfpathrectangle{\pgfqpoint{10.668400in}{4.237239in}}{\pgfqpoint{2.188235in}{0.972632in}}%
\pgfusepath{clip}%
\pgfsetbuttcap%
\pgfsetmiterjoin%
\definecolor{currentfill}{rgb}{0.121569,0.466667,0.705882}%
\pgfsetfillcolor{currentfill}%
\pgfsetlinewidth{0.000000pt}%
\definecolor{currentstroke}{rgb}{0.000000,0.000000,0.000000}%
\pgfsetstrokecolor{currentstroke}%
\pgfsetstrokeopacity{0.000000}%
\pgfsetdash{}{0pt}%
\pgfpathmoveto{\pgfqpoint{-319.224843in}{4.875057in}}%
\pgfpathlineto{\pgfqpoint{7.602809in}{4.875057in}}%
\pgfpathlineto{\pgfqpoint{7.602809in}{4.882145in}}%
\pgfpathlineto{\pgfqpoint{-319.224843in}{4.882145in}}%
\pgfpathclose%
\pgfusepath{fill}%
\end{pgfscope}%
\begin{pgfscope}%
\pgfpathrectangle{\pgfqpoint{10.668400in}{4.237239in}}{\pgfqpoint{2.188235in}{0.972632in}}%
\pgfusepath{clip}%
\pgfsetbuttcap%
\pgfsetmiterjoin%
\definecolor{currentfill}{rgb}{0.121569,0.466667,0.705882}%
\pgfsetfillcolor{currentfill}%
\pgfsetlinewidth{0.000000pt}%
\definecolor{currentstroke}{rgb}{0.000000,0.000000,0.000000}%
\pgfsetstrokecolor{currentstroke}%
\pgfsetstrokeopacity{0.000000}%
\pgfsetdash{}{0pt}%
\pgfpathmoveto{\pgfqpoint{-319.224843in}{4.883917in}}%
\pgfpathlineto{\pgfqpoint{8.071603in}{4.883917in}}%
\pgfpathlineto{\pgfqpoint{8.071603in}{4.891005in}}%
\pgfpathlineto{\pgfqpoint{-319.224843in}{4.891005in}}%
\pgfpathclose%
\pgfusepath{fill}%
\end{pgfscope}%
\begin{pgfscope}%
\pgfpathrectangle{\pgfqpoint{10.668400in}{4.237239in}}{\pgfqpoint{2.188235in}{0.972632in}}%
\pgfusepath{clip}%
\pgfsetbuttcap%
\pgfsetmiterjoin%
\definecolor{currentfill}{rgb}{0.121569,0.466667,0.705882}%
\pgfsetfillcolor{currentfill}%
\pgfsetlinewidth{0.000000pt}%
\definecolor{currentstroke}{rgb}{0.000000,0.000000,0.000000}%
\pgfsetstrokecolor{currentstroke}%
\pgfsetstrokeopacity{0.000000}%
\pgfsetdash{}{0pt}%
\pgfpathmoveto{\pgfqpoint{-319.224843in}{4.892777in}}%
\pgfpathlineto{\pgfqpoint{8.239233in}{4.892777in}}%
\pgfpathlineto{\pgfqpoint{8.239233in}{4.899865in}}%
\pgfpathlineto{\pgfqpoint{-319.224843in}{4.899865in}}%
\pgfpathclose%
\pgfusepath{fill}%
\end{pgfscope}%
\begin{pgfscope}%
\pgfpathrectangle{\pgfqpoint{10.668400in}{4.237239in}}{\pgfqpoint{2.188235in}{0.972632in}}%
\pgfusepath{clip}%
\pgfsetbuttcap%
\pgfsetmiterjoin%
\definecolor{currentfill}{rgb}{0.121569,0.466667,0.705882}%
\pgfsetfillcolor{currentfill}%
\pgfsetlinewidth{0.000000pt}%
\definecolor{currentstroke}{rgb}{0.000000,0.000000,0.000000}%
\pgfsetstrokecolor{currentstroke}%
\pgfsetstrokeopacity{0.000000}%
\pgfsetdash{}{0pt}%
\pgfpathmoveto{\pgfqpoint{-319.224843in}{4.901637in}}%
\pgfpathlineto{\pgfqpoint{8.270237in}{4.901637in}}%
\pgfpathlineto{\pgfqpoint{8.270237in}{4.908725in}}%
\pgfpathlineto{\pgfqpoint{-319.224843in}{4.908725in}}%
\pgfpathclose%
\pgfusepath{fill}%
\end{pgfscope}%
\begin{pgfscope}%
\pgfpathrectangle{\pgfqpoint{10.668400in}{4.237239in}}{\pgfqpoint{2.188235in}{0.972632in}}%
\pgfusepath{clip}%
\pgfsetbuttcap%
\pgfsetmiterjoin%
\definecolor{currentfill}{rgb}{0.121569,0.466667,0.705882}%
\pgfsetfillcolor{currentfill}%
\pgfsetlinewidth{0.000000pt}%
\definecolor{currentstroke}{rgb}{0.000000,0.000000,0.000000}%
\pgfsetstrokecolor{currentstroke}%
\pgfsetstrokeopacity{0.000000}%
\pgfsetdash{}{0pt}%
\pgfpathmoveto{\pgfqpoint{-319.224843in}{4.910497in}}%
\pgfpathlineto{\pgfqpoint{8.179502in}{4.910497in}}%
\pgfpathlineto{\pgfqpoint{8.179502in}{4.917585in}}%
\pgfpathlineto{\pgfqpoint{-319.224843in}{4.917585in}}%
\pgfpathclose%
\pgfusepath{fill}%
\end{pgfscope}%
\begin{pgfscope}%
\pgfpathrectangle{\pgfqpoint{10.668400in}{4.237239in}}{\pgfqpoint{2.188235in}{0.972632in}}%
\pgfusepath{clip}%
\pgfsetbuttcap%
\pgfsetmiterjoin%
\definecolor{currentfill}{rgb}{0.121569,0.466667,0.705882}%
\pgfsetfillcolor{currentfill}%
\pgfsetlinewidth{0.000000pt}%
\definecolor{currentstroke}{rgb}{0.000000,0.000000,0.000000}%
\pgfsetstrokecolor{currentstroke}%
\pgfsetstrokeopacity{0.000000}%
\pgfsetdash{}{0pt}%
\pgfpathmoveto{\pgfqpoint{-319.224843in}{4.919357in}}%
\pgfpathlineto{\pgfqpoint{7.590498in}{4.919357in}}%
\pgfpathlineto{\pgfqpoint{7.590498in}{4.926444in}}%
\pgfpathlineto{\pgfqpoint{-319.224843in}{4.926444in}}%
\pgfpathclose%
\pgfusepath{fill}%
\end{pgfscope}%
\begin{pgfscope}%
\pgfpathrectangle{\pgfqpoint{10.668400in}{4.237239in}}{\pgfqpoint{2.188235in}{0.972632in}}%
\pgfusepath{clip}%
\pgfsetbuttcap%
\pgfsetmiterjoin%
\definecolor{currentfill}{rgb}{0.121569,0.466667,0.705882}%
\pgfsetfillcolor{currentfill}%
\pgfsetlinewidth{0.000000pt}%
\definecolor{currentstroke}{rgb}{0.000000,0.000000,0.000000}%
\pgfsetstrokecolor{currentstroke}%
\pgfsetstrokeopacity{0.000000}%
\pgfsetdash{}{0pt}%
\pgfpathmoveto{\pgfqpoint{-319.224843in}{4.928216in}}%
\pgfpathlineto{\pgfqpoint{8.260789in}{4.928216in}}%
\pgfpathlineto{\pgfqpoint{8.260789in}{4.935304in}}%
\pgfpathlineto{\pgfqpoint{-319.224843in}{4.935304in}}%
\pgfpathclose%
\pgfusepath{fill}%
\end{pgfscope}%
\begin{pgfscope}%
\pgfpathrectangle{\pgfqpoint{10.668400in}{4.237239in}}{\pgfqpoint{2.188235in}{0.972632in}}%
\pgfusepath{clip}%
\pgfsetbuttcap%
\pgfsetmiterjoin%
\definecolor{currentfill}{rgb}{0.121569,0.466667,0.705882}%
\pgfsetfillcolor{currentfill}%
\pgfsetlinewidth{0.000000pt}%
\definecolor{currentstroke}{rgb}{0.000000,0.000000,0.000000}%
\pgfsetstrokecolor{currentstroke}%
\pgfsetstrokeopacity{0.000000}%
\pgfsetdash{}{0pt}%
\pgfpathmoveto{\pgfqpoint{-319.224843in}{4.937076in}}%
\pgfpathlineto{\pgfqpoint{8.219444in}{4.937076in}}%
\pgfpathlineto{\pgfqpoint{8.219444in}{4.944164in}}%
\pgfpathlineto{\pgfqpoint{-319.224843in}{4.944164in}}%
\pgfpathclose%
\pgfusepath{fill}%
\end{pgfscope}%
\begin{pgfscope}%
\pgfpathrectangle{\pgfqpoint{10.668400in}{4.237239in}}{\pgfqpoint{2.188235in}{0.972632in}}%
\pgfusepath{clip}%
\pgfsetbuttcap%
\pgfsetmiterjoin%
\definecolor{currentfill}{rgb}{0.121569,0.466667,0.705882}%
\pgfsetfillcolor{currentfill}%
\pgfsetlinewidth{0.000000pt}%
\definecolor{currentstroke}{rgb}{0.000000,0.000000,0.000000}%
\pgfsetstrokecolor{currentstroke}%
\pgfsetstrokeopacity{0.000000}%
\pgfsetdash{}{0pt}%
\pgfpathmoveto{\pgfqpoint{-319.224843in}{4.945936in}}%
\pgfpathlineto{\pgfqpoint{8.031640in}{4.945936in}}%
\pgfpathlineto{\pgfqpoint{8.031640in}{4.953024in}}%
\pgfpathlineto{\pgfqpoint{-319.224843in}{4.953024in}}%
\pgfpathclose%
\pgfusepath{fill}%
\end{pgfscope}%
\begin{pgfscope}%
\pgfpathrectangle{\pgfqpoint{10.668400in}{4.237239in}}{\pgfqpoint{2.188235in}{0.972632in}}%
\pgfusepath{clip}%
\pgfsetbuttcap%
\pgfsetmiterjoin%
\definecolor{currentfill}{rgb}{0.121569,0.466667,0.705882}%
\pgfsetfillcolor{currentfill}%
\pgfsetlinewidth{0.000000pt}%
\definecolor{currentstroke}{rgb}{0.000000,0.000000,0.000000}%
\pgfsetstrokecolor{currentstroke}%
\pgfsetstrokeopacity{0.000000}%
\pgfsetdash{}{0pt}%
\pgfpathmoveto{\pgfqpoint{-319.224843in}{4.954796in}}%
\pgfpathlineto{\pgfqpoint{8.509871in}{4.954796in}}%
\pgfpathlineto{\pgfqpoint{8.509871in}{4.961884in}}%
\pgfpathlineto{\pgfqpoint{-319.224843in}{4.961884in}}%
\pgfpathclose%
\pgfusepath{fill}%
\end{pgfscope}%
\begin{pgfscope}%
\pgfpathrectangle{\pgfqpoint{10.668400in}{4.237239in}}{\pgfqpoint{2.188235in}{0.972632in}}%
\pgfusepath{clip}%
\pgfsetbuttcap%
\pgfsetmiterjoin%
\definecolor{currentfill}{rgb}{0.121569,0.466667,0.705882}%
\pgfsetfillcolor{currentfill}%
\pgfsetlinewidth{0.000000pt}%
\definecolor{currentstroke}{rgb}{0.000000,0.000000,0.000000}%
\pgfsetstrokecolor{currentstroke}%
\pgfsetstrokeopacity{0.000000}%
\pgfsetdash{}{0pt}%
\pgfpathmoveto{\pgfqpoint{-319.224843in}{4.963656in}}%
\pgfpathlineto{\pgfqpoint{8.376623in}{4.963656in}}%
\pgfpathlineto{\pgfqpoint{8.376623in}{4.970744in}}%
\pgfpathlineto{\pgfqpoint{-319.224843in}{4.970744in}}%
\pgfpathclose%
\pgfusepath{fill}%
\end{pgfscope}%
\begin{pgfscope}%
\pgfpathrectangle{\pgfqpoint{10.668400in}{4.237239in}}{\pgfqpoint{2.188235in}{0.972632in}}%
\pgfusepath{clip}%
\pgfsetbuttcap%
\pgfsetmiterjoin%
\definecolor{currentfill}{rgb}{0.121569,0.466667,0.705882}%
\pgfsetfillcolor{currentfill}%
\pgfsetlinewidth{0.000000pt}%
\definecolor{currentstroke}{rgb}{0.000000,0.000000,0.000000}%
\pgfsetstrokecolor{currentstroke}%
\pgfsetstrokeopacity{0.000000}%
\pgfsetdash{}{0pt}%
\pgfpathmoveto{\pgfqpoint{-319.224843in}{4.972516in}}%
\pgfpathlineto{\pgfqpoint{8.411099in}{4.972516in}}%
\pgfpathlineto{\pgfqpoint{8.411099in}{4.979603in}}%
\pgfpathlineto{\pgfqpoint{-319.224843in}{4.979603in}}%
\pgfpathclose%
\pgfusepath{fill}%
\end{pgfscope}%
\begin{pgfscope}%
\pgfpathrectangle{\pgfqpoint{10.668400in}{4.237239in}}{\pgfqpoint{2.188235in}{0.972632in}}%
\pgfusepath{clip}%
\pgfsetbuttcap%
\pgfsetmiterjoin%
\definecolor{currentfill}{rgb}{0.121569,0.466667,0.705882}%
\pgfsetfillcolor{currentfill}%
\pgfsetlinewidth{0.000000pt}%
\definecolor{currentstroke}{rgb}{0.000000,0.000000,0.000000}%
\pgfsetstrokecolor{currentstroke}%
\pgfsetstrokeopacity{0.000000}%
\pgfsetdash{}{0pt}%
\pgfpathmoveto{\pgfqpoint{-319.224843in}{4.981375in}}%
\pgfpathlineto{\pgfqpoint{8.064143in}{4.981375in}}%
\pgfpathlineto{\pgfqpoint{8.064143in}{4.988463in}}%
\pgfpathlineto{\pgfqpoint{-319.224843in}{4.988463in}}%
\pgfpathclose%
\pgfusepath{fill}%
\end{pgfscope}%
\begin{pgfscope}%
\pgfpathrectangle{\pgfqpoint{10.668400in}{4.237239in}}{\pgfqpoint{2.188235in}{0.972632in}}%
\pgfusepath{clip}%
\pgfsetbuttcap%
\pgfsetmiterjoin%
\definecolor{currentfill}{rgb}{0.121569,0.466667,0.705882}%
\pgfsetfillcolor{currentfill}%
\pgfsetlinewidth{0.000000pt}%
\definecolor{currentstroke}{rgb}{0.000000,0.000000,0.000000}%
\pgfsetstrokecolor{currentstroke}%
\pgfsetstrokeopacity{0.000000}%
\pgfsetdash{}{0pt}%
\pgfpathmoveto{\pgfqpoint{-319.224843in}{4.990235in}}%
\pgfpathlineto{\pgfqpoint{8.055465in}{4.990235in}}%
\pgfpathlineto{\pgfqpoint{8.055465in}{4.997323in}}%
\pgfpathlineto{\pgfqpoint{-319.224843in}{4.997323in}}%
\pgfpathclose%
\pgfusepath{fill}%
\end{pgfscope}%
\begin{pgfscope}%
\pgfpathrectangle{\pgfqpoint{10.668400in}{4.237239in}}{\pgfqpoint{2.188235in}{0.972632in}}%
\pgfusepath{clip}%
\pgfsetbuttcap%
\pgfsetmiterjoin%
\definecolor{currentfill}{rgb}{0.121569,0.466667,0.705882}%
\pgfsetfillcolor{currentfill}%
\pgfsetlinewidth{0.000000pt}%
\definecolor{currentstroke}{rgb}{0.000000,0.000000,0.000000}%
\pgfsetstrokecolor{currentstroke}%
\pgfsetstrokeopacity{0.000000}%
\pgfsetdash{}{0pt}%
\pgfpathmoveto{\pgfqpoint{-319.224843in}{4.999095in}}%
\pgfpathlineto{\pgfqpoint{8.222279in}{4.999095in}}%
\pgfpathlineto{\pgfqpoint{8.222279in}{5.006183in}}%
\pgfpathlineto{\pgfqpoint{-319.224843in}{5.006183in}}%
\pgfpathclose%
\pgfusepath{fill}%
\end{pgfscope}%
\begin{pgfscope}%
\pgfpathrectangle{\pgfqpoint{10.668400in}{4.237239in}}{\pgfqpoint{2.188235in}{0.972632in}}%
\pgfusepath{clip}%
\pgfsetbuttcap%
\pgfsetmiterjoin%
\definecolor{currentfill}{rgb}{0.121569,0.466667,0.705882}%
\pgfsetfillcolor{currentfill}%
\pgfsetlinewidth{0.000000pt}%
\definecolor{currentstroke}{rgb}{0.000000,0.000000,0.000000}%
\pgfsetstrokecolor{currentstroke}%
\pgfsetstrokeopacity{0.000000}%
\pgfsetdash{}{0pt}%
\pgfpathmoveto{\pgfqpoint{-319.224843in}{5.007955in}}%
\pgfpathlineto{\pgfqpoint{8.370739in}{5.007955in}}%
\pgfpathlineto{\pgfqpoint{8.370739in}{5.015043in}}%
\pgfpathlineto{\pgfqpoint{-319.224843in}{5.015043in}}%
\pgfpathclose%
\pgfusepath{fill}%
\end{pgfscope}%
\begin{pgfscope}%
\pgfpathrectangle{\pgfqpoint{10.668400in}{4.237239in}}{\pgfqpoint{2.188235in}{0.972632in}}%
\pgfusepath{clip}%
\pgfsetbuttcap%
\pgfsetmiterjoin%
\definecolor{currentfill}{rgb}{0.121569,0.466667,0.705882}%
\pgfsetfillcolor{currentfill}%
\pgfsetlinewidth{0.000000pt}%
\definecolor{currentstroke}{rgb}{0.000000,0.000000,0.000000}%
\pgfsetstrokecolor{currentstroke}%
\pgfsetstrokeopacity{0.000000}%
\pgfsetdash{}{0pt}%
\pgfpathmoveto{\pgfqpoint{-319.224843in}{5.016815in}}%
\pgfpathlineto{\pgfqpoint{7.911269in}{5.016815in}}%
\pgfpathlineto{\pgfqpoint{7.911269in}{5.023903in}}%
\pgfpathlineto{\pgfqpoint{-319.224843in}{5.023903in}}%
\pgfpathclose%
\pgfusepath{fill}%
\end{pgfscope}%
\begin{pgfscope}%
\pgfpathrectangle{\pgfqpoint{10.668400in}{4.237239in}}{\pgfqpoint{2.188235in}{0.972632in}}%
\pgfusepath{clip}%
\pgfsetbuttcap%
\pgfsetmiterjoin%
\definecolor{currentfill}{rgb}{0.121569,0.466667,0.705882}%
\pgfsetfillcolor{currentfill}%
\pgfsetlinewidth{0.000000pt}%
\definecolor{currentstroke}{rgb}{0.000000,0.000000,0.000000}%
\pgfsetstrokecolor{currentstroke}%
\pgfsetstrokeopacity{0.000000}%
\pgfsetdash{}{0pt}%
\pgfpathmoveto{\pgfqpoint{-319.224843in}{5.025675in}}%
\pgfpathlineto{\pgfqpoint{8.196647in}{5.025675in}}%
\pgfpathlineto{\pgfqpoint{8.196647in}{5.032762in}}%
\pgfpathlineto{\pgfqpoint{-319.224843in}{5.032762in}}%
\pgfpathclose%
\pgfusepath{fill}%
\end{pgfscope}%
\begin{pgfscope}%
\pgfpathrectangle{\pgfqpoint{10.668400in}{4.237239in}}{\pgfqpoint{2.188235in}{0.972632in}}%
\pgfusepath{clip}%
\pgfsetbuttcap%
\pgfsetmiterjoin%
\definecolor{currentfill}{rgb}{0.121569,0.466667,0.705882}%
\pgfsetfillcolor{currentfill}%
\pgfsetlinewidth{0.000000pt}%
\definecolor{currentstroke}{rgb}{0.000000,0.000000,0.000000}%
\pgfsetstrokecolor{currentstroke}%
\pgfsetstrokeopacity{0.000000}%
\pgfsetdash{}{0pt}%
\pgfpathmoveto{\pgfqpoint{-319.224843in}{5.034534in}}%
\pgfpathlineto{\pgfqpoint{8.324142in}{5.034534in}}%
\pgfpathlineto{\pgfqpoint{8.324142in}{5.041622in}}%
\pgfpathlineto{\pgfqpoint{-319.224843in}{5.041622in}}%
\pgfpathclose%
\pgfusepath{fill}%
\end{pgfscope}%
\begin{pgfscope}%
\pgfpathrectangle{\pgfqpoint{10.668400in}{4.237239in}}{\pgfqpoint{2.188235in}{0.972632in}}%
\pgfusepath{clip}%
\pgfsetbuttcap%
\pgfsetmiterjoin%
\definecolor{currentfill}{rgb}{0.121569,0.466667,0.705882}%
\pgfsetfillcolor{currentfill}%
\pgfsetlinewidth{0.000000pt}%
\definecolor{currentstroke}{rgb}{0.000000,0.000000,0.000000}%
\pgfsetstrokecolor{currentstroke}%
\pgfsetstrokeopacity{0.000000}%
\pgfsetdash{}{0pt}%
\pgfpathmoveto{\pgfqpoint{-319.224843in}{5.043394in}}%
\pgfpathlineto{\pgfqpoint{8.048807in}{5.043394in}}%
\pgfpathlineto{\pgfqpoint{8.048807in}{5.050482in}}%
\pgfpathlineto{\pgfqpoint{-319.224843in}{5.050482in}}%
\pgfpathclose%
\pgfusepath{fill}%
\end{pgfscope}%
\begin{pgfscope}%
\pgfpathrectangle{\pgfqpoint{10.668400in}{4.237239in}}{\pgfqpoint{2.188235in}{0.972632in}}%
\pgfusepath{clip}%
\pgfsetbuttcap%
\pgfsetmiterjoin%
\definecolor{currentfill}{rgb}{0.121569,0.466667,0.705882}%
\pgfsetfillcolor{currentfill}%
\pgfsetlinewidth{0.000000pt}%
\definecolor{currentstroke}{rgb}{0.000000,0.000000,0.000000}%
\pgfsetstrokecolor{currentstroke}%
\pgfsetstrokeopacity{0.000000}%
\pgfsetdash{}{0pt}%
\pgfpathmoveto{\pgfqpoint{-319.224843in}{5.052254in}}%
\pgfpathlineto{\pgfqpoint{8.259020in}{5.052254in}}%
\pgfpathlineto{\pgfqpoint{8.259020in}{5.059342in}}%
\pgfpathlineto{\pgfqpoint{-319.224843in}{5.059342in}}%
\pgfpathclose%
\pgfusepath{fill}%
\end{pgfscope}%
\begin{pgfscope}%
\pgfpathrectangle{\pgfqpoint{10.668400in}{4.237239in}}{\pgfqpoint{2.188235in}{0.972632in}}%
\pgfusepath{clip}%
\pgfsetbuttcap%
\pgfsetmiterjoin%
\definecolor{currentfill}{rgb}{0.121569,0.466667,0.705882}%
\pgfsetfillcolor{currentfill}%
\pgfsetlinewidth{0.000000pt}%
\definecolor{currentstroke}{rgb}{0.000000,0.000000,0.000000}%
\pgfsetstrokecolor{currentstroke}%
\pgfsetstrokeopacity{0.000000}%
\pgfsetdash{}{0pt}%
\pgfpathmoveto{\pgfqpoint{-319.224843in}{5.061114in}}%
\pgfpathlineto{\pgfqpoint{8.157657in}{5.061114in}}%
\pgfpathlineto{\pgfqpoint{8.157657in}{5.068202in}}%
\pgfpathlineto{\pgfqpoint{-319.224843in}{5.068202in}}%
\pgfpathclose%
\pgfusepath{fill}%
\end{pgfscope}%
\begin{pgfscope}%
\pgfpathrectangle{\pgfqpoint{10.668400in}{4.237239in}}{\pgfqpoint{2.188235in}{0.972632in}}%
\pgfusepath{clip}%
\pgfsetbuttcap%
\pgfsetmiterjoin%
\definecolor{currentfill}{rgb}{0.121569,0.466667,0.705882}%
\pgfsetfillcolor{currentfill}%
\pgfsetlinewidth{0.000000pt}%
\definecolor{currentstroke}{rgb}{0.000000,0.000000,0.000000}%
\pgfsetstrokecolor{currentstroke}%
\pgfsetstrokeopacity{0.000000}%
\pgfsetdash{}{0pt}%
\pgfpathmoveto{\pgfqpoint{-319.224843in}{5.069974in}}%
\pgfpathlineto{\pgfqpoint{8.244468in}{5.069974in}}%
\pgfpathlineto{\pgfqpoint{8.244468in}{5.077062in}}%
\pgfpathlineto{\pgfqpoint{-319.224843in}{5.077062in}}%
\pgfpathclose%
\pgfusepath{fill}%
\end{pgfscope}%
\begin{pgfscope}%
\pgfpathrectangle{\pgfqpoint{10.668400in}{4.237239in}}{\pgfqpoint{2.188235in}{0.972632in}}%
\pgfusepath{clip}%
\pgfsetbuttcap%
\pgfsetmiterjoin%
\definecolor{currentfill}{rgb}{0.121569,0.466667,0.705882}%
\pgfsetfillcolor{currentfill}%
\pgfsetlinewidth{0.000000pt}%
\definecolor{currentstroke}{rgb}{0.000000,0.000000,0.000000}%
\pgfsetstrokecolor{currentstroke}%
\pgfsetstrokeopacity{0.000000}%
\pgfsetdash{}{0pt}%
\pgfpathmoveto{\pgfqpoint{-319.224843in}{5.078833in}}%
\pgfpathlineto{\pgfqpoint{8.238323in}{5.078833in}}%
\pgfpathlineto{\pgfqpoint{8.238323in}{5.085921in}}%
\pgfpathlineto{\pgfqpoint{-319.224843in}{5.085921in}}%
\pgfpathclose%
\pgfusepath{fill}%
\end{pgfscope}%
\begin{pgfscope}%
\pgfpathrectangle{\pgfqpoint{10.668400in}{4.237239in}}{\pgfqpoint{2.188235in}{0.972632in}}%
\pgfusepath{clip}%
\pgfsetbuttcap%
\pgfsetmiterjoin%
\definecolor{currentfill}{rgb}{0.121569,0.466667,0.705882}%
\pgfsetfillcolor{currentfill}%
\pgfsetlinewidth{0.000000pt}%
\definecolor{currentstroke}{rgb}{0.000000,0.000000,0.000000}%
\pgfsetstrokecolor{currentstroke}%
\pgfsetstrokeopacity{0.000000}%
\pgfsetdash{}{0pt}%
\pgfpathmoveto{\pgfqpoint{-319.224843in}{5.087693in}}%
\pgfpathlineto{\pgfqpoint{8.348072in}{5.087693in}}%
\pgfpathlineto{\pgfqpoint{8.348072in}{5.094781in}}%
\pgfpathlineto{\pgfqpoint{-319.224843in}{5.094781in}}%
\pgfpathclose%
\pgfusepath{fill}%
\end{pgfscope}%
\begin{pgfscope}%
\pgfpathrectangle{\pgfqpoint{10.668400in}{4.237239in}}{\pgfqpoint{2.188235in}{0.972632in}}%
\pgfusepath{clip}%
\pgfsetbuttcap%
\pgfsetmiterjoin%
\definecolor{currentfill}{rgb}{0.121569,0.466667,0.705882}%
\pgfsetfillcolor{currentfill}%
\pgfsetlinewidth{0.000000pt}%
\definecolor{currentstroke}{rgb}{0.000000,0.000000,0.000000}%
\pgfsetstrokecolor{currentstroke}%
\pgfsetstrokeopacity{0.000000}%
\pgfsetdash{}{0pt}%
\pgfpathmoveto{\pgfqpoint{-319.224843in}{5.096553in}}%
\pgfpathlineto{\pgfqpoint{8.236831in}{5.096553in}}%
\pgfpathlineto{\pgfqpoint{8.236831in}{5.103641in}}%
\pgfpathlineto{\pgfqpoint{-319.224843in}{5.103641in}}%
\pgfpathclose%
\pgfusepath{fill}%
\end{pgfscope}%
\begin{pgfscope}%
\pgfpathrectangle{\pgfqpoint{10.668400in}{4.237239in}}{\pgfqpoint{2.188235in}{0.972632in}}%
\pgfusepath{clip}%
\pgfsetbuttcap%
\pgfsetmiterjoin%
\definecolor{currentfill}{rgb}{0.121569,0.466667,0.705882}%
\pgfsetfillcolor{currentfill}%
\pgfsetlinewidth{0.000000pt}%
\definecolor{currentstroke}{rgb}{0.000000,0.000000,0.000000}%
\pgfsetstrokecolor{currentstroke}%
\pgfsetstrokeopacity{0.000000}%
\pgfsetdash{}{0pt}%
\pgfpathmoveto{\pgfqpoint{-319.224843in}{5.105413in}}%
\pgfpathlineto{\pgfqpoint{8.199066in}{5.105413in}}%
\pgfpathlineto{\pgfqpoint{8.199066in}{5.112501in}}%
\pgfpathlineto{\pgfqpoint{-319.224843in}{5.112501in}}%
\pgfpathclose%
\pgfusepath{fill}%
\end{pgfscope}%
\begin{pgfscope}%
\pgfpathrectangle{\pgfqpoint{10.668400in}{4.237239in}}{\pgfqpoint{2.188235in}{0.972632in}}%
\pgfusepath{clip}%
\pgfsetbuttcap%
\pgfsetmiterjoin%
\definecolor{currentfill}{rgb}{0.121569,0.466667,0.705882}%
\pgfsetfillcolor{currentfill}%
\pgfsetlinewidth{0.000000pt}%
\definecolor{currentstroke}{rgb}{0.000000,0.000000,0.000000}%
\pgfsetstrokecolor{currentstroke}%
\pgfsetstrokeopacity{0.000000}%
\pgfsetdash{}{0pt}%
\pgfpathmoveto{\pgfqpoint{-319.224843in}{5.114273in}}%
\pgfpathlineto{\pgfqpoint{8.286967in}{5.114273in}}%
\pgfpathlineto{\pgfqpoint{8.286967in}{5.121361in}}%
\pgfpathlineto{\pgfqpoint{-319.224843in}{5.121361in}}%
\pgfpathclose%
\pgfusepath{fill}%
\end{pgfscope}%
\begin{pgfscope}%
\pgfpathrectangle{\pgfqpoint{10.668400in}{4.237239in}}{\pgfqpoint{2.188235in}{0.972632in}}%
\pgfusepath{clip}%
\pgfsetbuttcap%
\pgfsetmiterjoin%
\definecolor{currentfill}{rgb}{0.121569,0.466667,0.705882}%
\pgfsetfillcolor{currentfill}%
\pgfsetlinewidth{0.000000pt}%
\definecolor{currentstroke}{rgb}{0.000000,0.000000,0.000000}%
\pgfsetstrokecolor{currentstroke}%
\pgfsetstrokeopacity{0.000000}%
\pgfsetdash{}{0pt}%
\pgfpathmoveto{\pgfqpoint{-319.224843in}{5.123133in}}%
\pgfpathlineto{\pgfqpoint{7.855104in}{5.123133in}}%
\pgfpathlineto{\pgfqpoint{7.855104in}{5.130220in}}%
\pgfpathlineto{\pgfqpoint{-319.224843in}{5.130220in}}%
\pgfpathclose%
\pgfusepath{fill}%
\end{pgfscope}%
\begin{pgfscope}%
\pgfpathrectangle{\pgfqpoint{10.668400in}{4.237239in}}{\pgfqpoint{2.188235in}{0.972632in}}%
\pgfusepath{clip}%
\pgfsetbuttcap%
\pgfsetmiterjoin%
\definecolor{currentfill}{rgb}{0.121569,0.466667,0.705882}%
\pgfsetfillcolor{currentfill}%
\pgfsetlinewidth{0.000000pt}%
\definecolor{currentstroke}{rgb}{0.000000,0.000000,0.000000}%
\pgfsetstrokecolor{currentstroke}%
\pgfsetstrokeopacity{0.000000}%
\pgfsetdash{}{0pt}%
\pgfpathmoveto{\pgfqpoint{-319.224843in}{5.131992in}}%
\pgfpathlineto{\pgfqpoint{8.214546in}{5.131992in}}%
\pgfpathlineto{\pgfqpoint{8.214546in}{5.139080in}}%
\pgfpathlineto{\pgfqpoint{-319.224843in}{5.139080in}}%
\pgfpathclose%
\pgfusepath{fill}%
\end{pgfscope}%
\begin{pgfscope}%
\pgfpathrectangle{\pgfqpoint{10.668400in}{4.237239in}}{\pgfqpoint{2.188235in}{0.972632in}}%
\pgfusepath{clip}%
\pgfsetbuttcap%
\pgfsetmiterjoin%
\definecolor{currentfill}{rgb}{0.121569,0.466667,0.705882}%
\pgfsetfillcolor{currentfill}%
\pgfsetlinewidth{0.000000pt}%
\definecolor{currentstroke}{rgb}{0.000000,0.000000,0.000000}%
\pgfsetstrokecolor{currentstroke}%
\pgfsetstrokeopacity{0.000000}%
\pgfsetdash{}{0pt}%
\pgfpathmoveto{\pgfqpoint{-319.224843in}{5.140852in}}%
\pgfpathlineto{\pgfqpoint{8.200261in}{5.140852in}}%
\pgfpathlineto{\pgfqpoint{8.200261in}{5.147940in}}%
\pgfpathlineto{\pgfqpoint{-319.224843in}{5.147940in}}%
\pgfpathclose%
\pgfusepath{fill}%
\end{pgfscope}%
\begin{pgfscope}%
\pgfpathrectangle{\pgfqpoint{10.668400in}{4.237239in}}{\pgfqpoint{2.188235in}{0.972632in}}%
\pgfusepath{clip}%
\pgfsetbuttcap%
\pgfsetmiterjoin%
\definecolor{currentfill}{rgb}{0.121569,0.466667,0.705882}%
\pgfsetfillcolor{currentfill}%
\pgfsetlinewidth{0.000000pt}%
\definecolor{currentstroke}{rgb}{0.000000,0.000000,0.000000}%
\pgfsetstrokecolor{currentstroke}%
\pgfsetstrokeopacity{0.000000}%
\pgfsetdash{}{0pt}%
\pgfpathmoveto{\pgfqpoint{-319.224843in}{5.149712in}}%
\pgfpathlineto{\pgfqpoint{7.877759in}{5.149712in}}%
\pgfpathlineto{\pgfqpoint{7.877759in}{5.156800in}}%
\pgfpathlineto{\pgfqpoint{-319.224843in}{5.156800in}}%
\pgfpathclose%
\pgfusepath{fill}%
\end{pgfscope}%
\begin{pgfscope}%
\pgfpathrectangle{\pgfqpoint{10.668400in}{4.237239in}}{\pgfqpoint{2.188235in}{0.972632in}}%
\pgfusepath{clip}%
\pgfsetbuttcap%
\pgfsetmiterjoin%
\definecolor{currentfill}{rgb}{0.121569,0.466667,0.705882}%
\pgfsetfillcolor{currentfill}%
\pgfsetlinewidth{0.000000pt}%
\definecolor{currentstroke}{rgb}{0.000000,0.000000,0.000000}%
\pgfsetstrokecolor{currentstroke}%
\pgfsetstrokeopacity{0.000000}%
\pgfsetdash{}{0pt}%
\pgfpathmoveto{\pgfqpoint{-319.224843in}{5.158572in}}%
\pgfpathlineto{\pgfqpoint{8.098089in}{5.158572in}}%
\pgfpathlineto{\pgfqpoint{8.098089in}{5.165660in}}%
\pgfpathlineto{\pgfqpoint{-319.224843in}{5.165660in}}%
\pgfpathclose%
\pgfusepath{fill}%
\end{pgfscope}%
\begin{pgfscope}%
\pgfsetbuttcap%
\pgfsetroundjoin%
\definecolor{currentfill}{rgb}{0.000000,0.000000,0.000000}%
\pgfsetfillcolor{currentfill}%
\pgfsetlinewidth{0.803000pt}%
\definecolor{currentstroke}{rgb}{0.000000,0.000000,0.000000}%
\pgfsetstrokecolor{currentstroke}%
\pgfsetdash{}{0pt}%
\pgfsys@defobject{currentmarker}{\pgfqpoint{0.000000in}{-0.048611in}}{\pgfqpoint{0.000000in}{0.000000in}}{%
\pgfpathmoveto{\pgfqpoint{0.000000in}{0.000000in}}%
\pgfpathlineto{\pgfqpoint{0.000000in}{-0.048611in}}%
\pgfusepath{stroke,fill}%
}%
\begin{pgfscope}%
\pgfsys@transformshift{11.218880in}{4.237239in}%
\pgfsys@useobject{currentmarker}{}%
\end{pgfscope}%
\end{pgfscope}%
\begin{pgfscope}%
\pgfsetbuttcap%
\pgfsetroundjoin%
\definecolor{currentfill}{rgb}{0.000000,0.000000,0.000000}%
\pgfsetfillcolor{currentfill}%
\pgfsetlinewidth{0.803000pt}%
\definecolor{currentstroke}{rgb}{0.000000,0.000000,0.000000}%
\pgfsetstrokecolor{currentstroke}%
\pgfsetdash{}{0pt}%
\pgfsys@defobject{currentmarker}{\pgfqpoint{0.000000in}{-0.048611in}}{\pgfqpoint{0.000000in}{0.000000in}}{%
\pgfpathmoveto{\pgfqpoint{0.000000in}{0.000000in}}%
\pgfpathlineto{\pgfqpoint{0.000000in}{-0.048611in}}%
\pgfusepath{stroke,fill}%
}%
\begin{pgfscope}%
\pgfsys@transformshift{11.881756in}{4.237239in}%
\pgfsys@useobject{currentmarker}{}%
\end{pgfscope}%
\end{pgfscope}%
\begin{pgfscope}%
\pgfsetbuttcap%
\pgfsetroundjoin%
\definecolor{currentfill}{rgb}{0.000000,0.000000,0.000000}%
\pgfsetfillcolor{currentfill}%
\pgfsetlinewidth{0.803000pt}%
\definecolor{currentstroke}{rgb}{0.000000,0.000000,0.000000}%
\pgfsetstrokecolor{currentstroke}%
\pgfsetdash{}{0pt}%
\pgfsys@defobject{currentmarker}{\pgfqpoint{0.000000in}{-0.048611in}}{\pgfqpoint{0.000000in}{0.000000in}}{%
\pgfpathmoveto{\pgfqpoint{0.000000in}{0.000000in}}%
\pgfpathlineto{\pgfqpoint{0.000000in}{-0.048611in}}%
\pgfusepath{stroke,fill}%
}%
\begin{pgfscope}%
\pgfsys@transformshift{12.544632in}{4.237239in}%
\pgfsys@useobject{currentmarker}{}%
\end{pgfscope}%
\end{pgfscope}%
\begin{pgfscope}%
\pgfsetbuttcap%
\pgfsetroundjoin%
\definecolor{currentfill}{rgb}{0.000000,0.000000,0.000000}%
\pgfsetfillcolor{currentfill}%
\pgfsetlinewidth{0.803000pt}%
\definecolor{currentstroke}{rgb}{0.000000,0.000000,0.000000}%
\pgfsetstrokecolor{currentstroke}%
\pgfsetdash{}{0pt}%
\pgfsys@defobject{currentmarker}{\pgfqpoint{-0.048611in}{0.000000in}}{\pgfqpoint{0.000000in}{0.000000in}}{%
\pgfpathmoveto{\pgfqpoint{0.000000in}{0.000000in}}%
\pgfpathlineto{\pgfqpoint{-0.048611in}{0.000000in}}%
\pgfusepath{stroke,fill}%
}%
\begin{pgfscope}%
\pgfsys@transformshift{10.668400in}{4.284993in}%
\pgfsys@useobject{currentmarker}{}%
\end{pgfscope}%
\end{pgfscope}%
\begin{pgfscope}%
\pgftext[x=10.501733in,y=4.232232in,left,base]{\rmfamily\fontsize{10.000000}{12.000000}\selectfont \(\displaystyle 0\)}%
\end{pgfscope}%
\begin{pgfscope}%
\pgfsetbuttcap%
\pgfsetroundjoin%
\definecolor{currentfill}{rgb}{0.000000,0.000000,0.000000}%
\pgfsetfillcolor{currentfill}%
\pgfsetlinewidth{0.803000pt}%
\definecolor{currentstroke}{rgb}{0.000000,0.000000,0.000000}%
\pgfsetstrokecolor{currentstroke}%
\pgfsetdash{}{0pt}%
\pgfsys@defobject{currentmarker}{\pgfqpoint{-0.048611in}{0.000000in}}{\pgfqpoint{0.000000in}{0.000000in}}{%
\pgfpathmoveto{\pgfqpoint{0.000000in}{0.000000in}}%
\pgfpathlineto{\pgfqpoint{-0.048611in}{0.000000in}}%
\pgfusepath{stroke,fill}%
}%
\begin{pgfscope}%
\pgfsys@transformshift{10.668400in}{4.727984in}%
\pgfsys@useobject{currentmarker}{}%
\end{pgfscope}%
\end{pgfscope}%
\begin{pgfscope}%
\pgftext[x=10.432288in,y=4.675223in,left,base]{\rmfamily\fontsize{10.000000}{12.000000}\selectfont \(\displaystyle 50\)}%
\end{pgfscope}%
\begin{pgfscope}%
\pgfsetbuttcap%
\pgfsetroundjoin%
\definecolor{currentfill}{rgb}{0.000000,0.000000,0.000000}%
\pgfsetfillcolor{currentfill}%
\pgfsetlinewidth{0.803000pt}%
\definecolor{currentstroke}{rgb}{0.000000,0.000000,0.000000}%
\pgfsetstrokecolor{currentstroke}%
\pgfsetdash{}{0pt}%
\pgfsys@defobject{currentmarker}{\pgfqpoint{-0.048611in}{0.000000in}}{\pgfqpoint{0.000000in}{0.000000in}}{%
\pgfpathmoveto{\pgfqpoint{0.000000in}{0.000000in}}%
\pgfpathlineto{\pgfqpoint{-0.048611in}{0.000000in}}%
\pgfusepath{stroke,fill}%
}%
\begin{pgfscope}%
\pgfsys@transformshift{10.668400in}{5.170976in}%
\pgfsys@useobject{currentmarker}{}%
\end{pgfscope}%
\end{pgfscope}%
\begin{pgfscope}%
\pgftext[x=10.362844in,y=5.118214in,left,base]{\rmfamily\fontsize{10.000000}{12.000000}\selectfont \(\displaystyle 100\)}%
\end{pgfscope}%
\begin{pgfscope}%
\pgftext[x=10.307288in,y=4.723554in,,bottom,rotate=90.000000]{\rmfamily\fontsize{10.000000}{12.000000}\selectfont \(\displaystyle j\)}%
\end{pgfscope}%
\begin{pgfscope}%
\pgfsetrectcap%
\pgfsetmiterjoin%
\pgfsetlinewidth{0.803000pt}%
\definecolor{currentstroke}{rgb}{0.000000,0.000000,0.000000}%
\pgfsetstrokecolor{currentstroke}%
\pgfsetdash{}{0pt}%
\pgfpathmoveto{\pgfqpoint{10.668400in}{4.237239in}}%
\pgfpathlineto{\pgfqpoint{10.668400in}{5.209870in}}%
\pgfusepath{stroke}%
\end{pgfscope}%
\begin{pgfscope}%
\pgfsetrectcap%
\pgfsetmiterjoin%
\pgfsetlinewidth{0.803000pt}%
\definecolor{currentstroke}{rgb}{0.000000,0.000000,0.000000}%
\pgfsetstrokecolor{currentstroke}%
\pgfsetdash{}{0pt}%
\pgfpathmoveto{\pgfqpoint{12.856635in}{4.237239in}}%
\pgfpathlineto{\pgfqpoint{12.856635in}{5.209870in}}%
\pgfusepath{stroke}%
\end{pgfscope}%
\begin{pgfscope}%
\pgfsetrectcap%
\pgfsetmiterjoin%
\pgfsetlinewidth{0.803000pt}%
\definecolor{currentstroke}{rgb}{0.000000,0.000000,0.000000}%
\pgfsetstrokecolor{currentstroke}%
\pgfsetdash{}{0pt}%
\pgfpathmoveto{\pgfqpoint{10.668400in}{4.237239in}}%
\pgfpathlineto{\pgfqpoint{12.856635in}{4.237239in}}%
\pgfusepath{stroke}%
\end{pgfscope}%
\begin{pgfscope}%
\pgfsetrectcap%
\pgfsetmiterjoin%
\pgfsetlinewidth{0.803000pt}%
\definecolor{currentstroke}{rgb}{0.000000,0.000000,0.000000}%
\pgfsetstrokecolor{currentstroke}%
\pgfsetdash{}{0pt}%
\pgfpathmoveto{\pgfqpoint{10.668400in}{5.209870in}}%
\pgfpathlineto{\pgfqpoint{12.856635in}{5.209870in}}%
\pgfusepath{stroke}%
\end{pgfscope}%
\begin{pgfscope}%
\pgfsetbuttcap%
\pgfsetmiterjoin%
\definecolor{currentfill}{rgb}{1.000000,1.000000,1.000000}%
\pgfsetfillcolor{currentfill}%
\pgfsetlinewidth{0.000000pt}%
\definecolor{currentstroke}{rgb}{0.000000,0.000000,0.000000}%
\pgfsetstrokecolor{currentstroke}%
\pgfsetstrokeopacity{0.000000}%
\pgfsetdash{}{0pt}%
\pgfpathmoveto{\pgfqpoint{0.456635in}{3.021449in}}%
\pgfpathlineto{\pgfqpoint{4.833106in}{3.021449in}}%
\pgfpathlineto{\pgfqpoint{4.833106in}{3.994081in}}%
\pgfpathlineto{\pgfqpoint{0.456635in}{3.994081in}}%
\pgfpathclose%
\pgfusepath{fill}%
\end{pgfscope}%
\begin{pgfscope}%
\pgfpathrectangle{\pgfqpoint{0.456635in}{3.021449in}}{\pgfqpoint{4.376471in}{0.972632in}}%
\pgfusepath{clip}%
\pgfsetbuttcap%
\pgfsetroundjoin%
\definecolor{currentfill}{rgb}{1.000000,0.000000,0.000000}%
\pgfsetfillcolor{currentfill}%
\pgfsetlinewidth{2.007500pt}%
\definecolor{currentstroke}{rgb}{1.000000,0.000000,0.000000}%
\pgfsetstrokecolor{currentstroke}%
\pgfsetdash{}{0pt}%
\pgfpathmoveto{\pgfqpoint{2.601594in}{3.175317in}}%
\pgfpathlineto{\pgfqpoint{2.684927in}{3.175317in}}%
\pgfpathmoveto{\pgfqpoint{2.643260in}{3.133650in}}%
\pgfpathlineto{\pgfqpoint{2.643260in}{3.216983in}}%
\pgfusepath{stroke,fill}%
\end{pgfscope}%
\begin{pgfscope}%
\pgfpathrectangle{\pgfqpoint{0.456635in}{3.021449in}}{\pgfqpoint{4.376471in}{0.972632in}}%
\pgfusepath{clip}%
\pgfsetbuttcap%
\pgfsetroundjoin%
\definecolor{currentfill}{rgb}{1.000000,0.000000,0.000000}%
\pgfsetfillcolor{currentfill}%
\pgfsetlinewidth{2.007500pt}%
\definecolor{currentstroke}{rgb}{1.000000,0.000000,0.000000}%
\pgfsetstrokecolor{currentstroke}%
\pgfsetdash{}{0pt}%
\pgfpathmoveto{\pgfqpoint{4.618881in}{3.834419in}}%
\pgfpathlineto{\pgfqpoint{4.702215in}{3.834419in}}%
\pgfpathmoveto{\pgfqpoint{4.660548in}{3.792752in}}%
\pgfpathlineto{\pgfqpoint{4.660548in}{3.876085in}}%
\pgfusepath{stroke,fill}%
\end{pgfscope}%
\begin{pgfscope}%
\pgfpathrectangle{\pgfqpoint{0.456635in}{3.021449in}}{\pgfqpoint{4.376471in}{0.972632in}}%
\pgfusepath{clip}%
\pgfsetbuttcap%
\pgfsetroundjoin%
\definecolor{currentfill}{rgb}{1.000000,0.000000,0.000000}%
\pgfsetfillcolor{currentfill}%
\pgfsetlinewidth{2.007500pt}%
\definecolor{currentstroke}{rgb}{1.000000,0.000000,0.000000}%
\pgfsetstrokecolor{currentstroke}%
\pgfsetdash{}{0pt}%
\pgfpathmoveto{\pgfqpoint{3.853103in}{3.210480in}}%
\pgfpathlineto{\pgfqpoint{3.936436in}{3.210480in}}%
\pgfpathmoveto{\pgfqpoint{3.894769in}{3.168813in}}%
\pgfpathlineto{\pgfqpoint{3.894769in}{3.252146in}}%
\pgfusepath{stroke,fill}%
\end{pgfscope}%
\begin{pgfscope}%
\pgfpathrectangle{\pgfqpoint{0.456635in}{3.021449in}}{\pgfqpoint{4.376471in}{0.972632in}}%
\pgfusepath{clip}%
\pgfsetbuttcap%
\pgfsetroundjoin%
\definecolor{currentfill}{rgb}{1.000000,0.000000,0.000000}%
\pgfsetfillcolor{currentfill}%
\pgfsetlinewidth{2.007500pt}%
\definecolor{currentstroke}{rgb}{1.000000,0.000000,0.000000}%
\pgfsetstrokecolor{currentstroke}%
\pgfsetdash{}{0pt}%
\pgfpathmoveto{\pgfqpoint{3.386272in}{3.558860in}}%
\pgfpathlineto{\pgfqpoint{3.469605in}{3.558860in}}%
\pgfpathmoveto{\pgfqpoint{3.427938in}{3.517193in}}%
\pgfpathlineto{\pgfqpoint{3.427938in}{3.600526in}}%
\pgfusepath{stroke,fill}%
\end{pgfscope}%
\begin{pgfscope}%
\pgfpathrectangle{\pgfqpoint{0.456635in}{3.021449in}}{\pgfqpoint{4.376471in}{0.972632in}}%
\pgfusepath{clip}%
\pgfsetbuttcap%
\pgfsetroundjoin%
\definecolor{currentfill}{rgb}{1.000000,0.000000,0.000000}%
\pgfsetfillcolor{currentfill}%
\pgfsetlinewidth{2.007500pt}%
\definecolor{currentstroke}{rgb}{1.000000,0.000000,0.000000}%
\pgfsetstrokecolor{currentstroke}%
\pgfsetdash{}{0pt}%
\pgfpathmoveto{\pgfqpoint{1.836512in}{3.403213in}}%
\pgfpathlineto{\pgfqpoint{1.919845in}{3.403213in}}%
\pgfpathmoveto{\pgfqpoint{1.878178in}{3.361546in}}%
\pgfpathlineto{\pgfqpoint{1.878178in}{3.444880in}}%
\pgfusepath{stroke,fill}%
\end{pgfscope}%
\begin{pgfscope}%
\pgfpathrectangle{\pgfqpoint{0.456635in}{3.021449in}}{\pgfqpoint{4.376471in}{0.972632in}}%
\pgfusepath{clip}%
\pgfsetbuttcap%
\pgfsetroundjoin%
\definecolor{currentfill}{rgb}{1.000000,0.000000,0.000000}%
\pgfsetfillcolor{currentfill}%
\pgfsetlinewidth{2.007500pt}%
\definecolor{currentstroke}{rgb}{1.000000,0.000000,0.000000}%
\pgfsetstrokecolor{currentstroke}%
\pgfsetdash{}{0pt}%
\pgfpathmoveto{\pgfqpoint{1.836427in}{3.651787in}}%
\pgfpathlineto{\pgfqpoint{1.919760in}{3.651787in}}%
\pgfpathmoveto{\pgfqpoint{1.878094in}{3.610120in}}%
\pgfpathlineto{\pgfqpoint{1.878094in}{3.693453in}}%
\pgfusepath{stroke,fill}%
\end{pgfscope}%
\begin{pgfscope}%
\pgfpathrectangle{\pgfqpoint{0.456635in}{3.021449in}}{\pgfqpoint{4.376471in}{0.972632in}}%
\pgfusepath{clip}%
\pgfsetbuttcap%
\pgfsetroundjoin%
\definecolor{currentfill}{rgb}{1.000000,0.000000,0.000000}%
\pgfsetfillcolor{currentfill}%
\pgfsetlinewidth{2.007500pt}%
\definecolor{currentstroke}{rgb}{1.000000,0.000000,0.000000}%
\pgfsetstrokecolor{currentstroke}%
\pgfsetdash{}{0pt}%
\pgfpathmoveto{\pgfqpoint{1.493624in}{3.589256in}}%
\pgfpathlineto{\pgfqpoint{1.576957in}{3.589256in}}%
\pgfpathmoveto{\pgfqpoint{1.535290in}{3.547590in}}%
\pgfpathlineto{\pgfqpoint{1.535290in}{3.630923in}}%
\pgfusepath{stroke,fill}%
\end{pgfscope}%
\begin{pgfscope}%
\pgfpathrectangle{\pgfqpoint{0.456635in}{3.021449in}}{\pgfqpoint{4.376471in}{0.972632in}}%
\pgfusepath{clip}%
\pgfsetbuttcap%
\pgfsetroundjoin%
\definecolor{currentfill}{rgb}{1.000000,0.000000,0.000000}%
\pgfsetfillcolor{currentfill}%
\pgfsetlinewidth{2.007500pt}%
\definecolor{currentstroke}{rgb}{1.000000,0.000000,0.000000}%
\pgfsetstrokecolor{currentstroke}%
\pgfsetdash{}{0pt}%
\pgfpathmoveto{\pgfqpoint{4.322898in}{3.671167in}}%
\pgfpathlineto{\pgfqpoint{4.406232in}{3.671167in}}%
\pgfpathmoveto{\pgfqpoint{4.364565in}{3.629500in}}%
\pgfpathlineto{\pgfqpoint{4.364565in}{3.712834in}}%
\pgfusepath{stroke,fill}%
\end{pgfscope}%
\begin{pgfscope}%
\pgfpathrectangle{\pgfqpoint{0.456635in}{3.021449in}}{\pgfqpoint{4.376471in}{0.972632in}}%
\pgfusepath{clip}%
\pgfsetbuttcap%
\pgfsetroundjoin%
\definecolor{currentfill}{rgb}{1.000000,0.000000,0.000000}%
\pgfsetfillcolor{currentfill}%
\pgfsetlinewidth{2.007500pt}%
\definecolor{currentstroke}{rgb}{1.000000,0.000000,0.000000}%
\pgfsetstrokecolor{currentstroke}%
\pgfsetdash{}{0pt}%
\pgfpathmoveto{\pgfqpoint{3.394872in}{3.668423in}}%
\pgfpathlineto{\pgfqpoint{3.478206in}{3.668423in}}%
\pgfpathmoveto{\pgfqpoint{3.436539in}{3.626756in}}%
\pgfpathlineto{\pgfqpoint{3.436539in}{3.710090in}}%
\pgfusepath{stroke,fill}%
\end{pgfscope}%
\begin{pgfscope}%
\pgfpathrectangle{\pgfqpoint{0.456635in}{3.021449in}}{\pgfqpoint{4.376471in}{0.972632in}}%
\pgfusepath{clip}%
\pgfsetbuttcap%
\pgfsetroundjoin%
\definecolor{currentfill}{rgb}{1.000000,0.000000,0.000000}%
\pgfsetfillcolor{currentfill}%
\pgfsetlinewidth{2.007500pt}%
\definecolor{currentstroke}{rgb}{1.000000,0.000000,0.000000}%
\pgfsetstrokecolor{currentstroke}%
\pgfsetdash{}{0pt}%
\pgfpathmoveto{\pgfqpoint{3.769350in}{3.160028in}}%
\pgfpathlineto{\pgfqpoint{3.852683in}{3.160028in}}%
\pgfpathmoveto{\pgfqpoint{3.811016in}{3.118362in}}%
\pgfpathlineto{\pgfqpoint{3.811016in}{3.201695in}}%
\pgfusepath{stroke,fill}%
\end{pgfscope}%
\begin{pgfscope}%
\pgfpathrectangle{\pgfqpoint{0.456635in}{3.021449in}}{\pgfqpoint{4.376471in}{0.972632in}}%
\pgfusepath{clip}%
\pgfsetbuttcap%
\pgfsetroundjoin%
\definecolor{currentfill}{rgb}{1.000000,0.000000,0.000000}%
\pgfsetfillcolor{currentfill}%
\pgfsetlinewidth{2.007500pt}%
\definecolor{currentstroke}{rgb}{1.000000,0.000000,0.000000}%
\pgfsetstrokecolor{currentstroke}%
\pgfsetdash{}{0pt}%
\pgfpathmoveto{\pgfqpoint{1.362333in}{3.812918in}}%
\pgfpathlineto{\pgfqpoint{1.445666in}{3.812918in}}%
\pgfpathmoveto{\pgfqpoint{1.403999in}{3.771251in}}%
\pgfpathlineto{\pgfqpoint{1.403999in}{3.854585in}}%
\pgfusepath{stroke,fill}%
\end{pgfscope}%
\begin{pgfscope}%
\pgfpathrectangle{\pgfqpoint{0.456635in}{3.021449in}}{\pgfqpoint{4.376471in}{0.972632in}}%
\pgfusepath{clip}%
\pgfsetbuttcap%
\pgfsetroundjoin%
\definecolor{currentfill}{rgb}{1.000000,0.000000,0.000000}%
\pgfsetfillcolor{currentfill}%
\pgfsetlinewidth{2.007500pt}%
\definecolor{currentstroke}{rgb}{1.000000,0.000000,0.000000}%
\pgfsetstrokecolor{currentstroke}%
\pgfsetdash{}{0pt}%
\pgfpathmoveto{\pgfqpoint{4.686088in}{3.550979in}}%
\pgfpathlineto{\pgfqpoint{4.769422in}{3.550979in}}%
\pgfpathmoveto{\pgfqpoint{4.727755in}{3.509313in}}%
\pgfpathlineto{\pgfqpoint{4.727755in}{3.592646in}}%
\pgfusepath{stroke,fill}%
\end{pgfscope}%
\begin{pgfscope}%
\pgfpathrectangle{\pgfqpoint{0.456635in}{3.021449in}}{\pgfqpoint{4.376471in}{0.972632in}}%
\pgfusepath{clip}%
\pgfsetbuttcap%
\pgfsetroundjoin%
\definecolor{currentfill}{rgb}{1.000000,0.000000,0.000000}%
\pgfsetfillcolor{currentfill}%
\pgfsetlinewidth{2.007500pt}%
\definecolor{currentstroke}{rgb}{1.000000,0.000000,0.000000}%
\pgfsetstrokecolor{currentstroke}%
\pgfsetdash{}{0pt}%
\pgfpathmoveto{\pgfqpoint{4.204791in}{3.503231in}}%
\pgfpathlineto{\pgfqpoint{4.288125in}{3.503231in}}%
\pgfpathmoveto{\pgfqpoint{4.246458in}{3.461564in}}%
\pgfpathlineto{\pgfqpoint{4.246458in}{3.544898in}}%
\pgfusepath{stroke,fill}%
\end{pgfscope}%
\begin{pgfscope}%
\pgfpathrectangle{\pgfqpoint{0.456635in}{3.021449in}}{\pgfqpoint{4.376471in}{0.972632in}}%
\pgfusepath{clip}%
\pgfsetbuttcap%
\pgfsetroundjoin%
\definecolor{currentfill}{rgb}{1.000000,0.000000,0.000000}%
\pgfsetfillcolor{currentfill}%
\pgfsetlinewidth{2.007500pt}%
\definecolor{currentstroke}{rgb}{1.000000,0.000000,0.000000}%
\pgfsetstrokecolor{currentstroke}%
\pgfsetdash{}{0pt}%
\pgfpathmoveto{\pgfqpoint{2.033699in}{3.642703in}}%
\pgfpathlineto{\pgfqpoint{2.117033in}{3.642703in}}%
\pgfpathmoveto{\pgfqpoint{2.075366in}{3.601036in}}%
\pgfpathlineto{\pgfqpoint{2.075366in}{3.684369in}}%
\pgfusepath{stroke,fill}%
\end{pgfscope}%
\begin{pgfscope}%
\pgfpathrectangle{\pgfqpoint{0.456635in}{3.021449in}}{\pgfqpoint{4.376471in}{0.972632in}}%
\pgfusepath{clip}%
\pgfsetbuttcap%
\pgfsetroundjoin%
\definecolor{currentfill}{rgb}{1.000000,0.000000,0.000000}%
\pgfsetfillcolor{currentfill}%
\pgfsetlinewidth{2.007500pt}%
\definecolor{currentstroke}{rgb}{1.000000,0.000000,0.000000}%
\pgfsetstrokecolor{currentstroke}%
\pgfsetdash{}{0pt}%
\pgfpathmoveto{\pgfqpoint{1.926864in}{3.608855in}}%
\pgfpathlineto{\pgfqpoint{2.010197in}{3.608855in}}%
\pgfpathmoveto{\pgfqpoint{1.968531in}{3.567189in}}%
\pgfpathlineto{\pgfqpoint{1.968531in}{3.650522in}}%
\pgfusepath{stroke,fill}%
\end{pgfscope}%
\begin{pgfscope}%
\pgfpathrectangle{\pgfqpoint{0.456635in}{3.021449in}}{\pgfqpoint{4.376471in}{0.972632in}}%
\pgfusepath{clip}%
\pgfsetbuttcap%
\pgfsetroundjoin%
\definecolor{currentfill}{rgb}{1.000000,0.000000,0.000000}%
\pgfsetfillcolor{currentfill}%
\pgfsetlinewidth{2.007500pt}%
\definecolor{currentstroke}{rgb}{1.000000,0.000000,0.000000}%
\pgfsetstrokecolor{currentstroke}%
\pgfsetdash{}{0pt}%
\pgfpathmoveto{\pgfqpoint{1.932394in}{3.556162in}}%
\pgfpathlineto{\pgfqpoint{2.015728in}{3.556162in}}%
\pgfpathmoveto{\pgfqpoint{1.974061in}{3.514495in}}%
\pgfpathlineto{\pgfqpoint{1.974061in}{3.597828in}}%
\pgfusepath{stroke,fill}%
\end{pgfscope}%
\begin{pgfscope}%
\pgfpathrectangle{\pgfqpoint{0.456635in}{3.021449in}}{\pgfqpoint{4.376471in}{0.972632in}}%
\pgfusepath{clip}%
\pgfsetbuttcap%
\pgfsetroundjoin%
\definecolor{currentfill}{rgb}{1.000000,0.000000,0.000000}%
\pgfsetfillcolor{currentfill}%
\pgfsetlinewidth{2.007500pt}%
\definecolor{currentstroke}{rgb}{1.000000,0.000000,0.000000}%
\pgfsetstrokecolor{currentstroke}%
\pgfsetdash{}{0pt}%
\pgfpathmoveto{\pgfqpoint{2.355469in}{3.578166in}}%
\pgfpathlineto{\pgfqpoint{2.438802in}{3.578166in}}%
\pgfpathmoveto{\pgfqpoint{2.397135in}{3.536499in}}%
\pgfpathlineto{\pgfqpoint{2.397135in}{3.619833in}}%
\pgfusepath{stroke,fill}%
\end{pgfscope}%
\begin{pgfscope}%
\pgfpathrectangle{\pgfqpoint{0.456635in}{3.021449in}}{\pgfqpoint{4.376471in}{0.972632in}}%
\pgfusepath{clip}%
\pgfsetbuttcap%
\pgfsetroundjoin%
\definecolor{currentfill}{rgb}{1.000000,0.000000,0.000000}%
\pgfsetfillcolor{currentfill}%
\pgfsetlinewidth{2.007500pt}%
\definecolor{currentstroke}{rgb}{1.000000,0.000000,0.000000}%
\pgfsetstrokecolor{currentstroke}%
\pgfsetdash{}{0pt}%
\pgfpathmoveto{\pgfqpoint{3.127528in}{3.452952in}}%
\pgfpathlineto{\pgfqpoint{3.210861in}{3.452952in}}%
\pgfpathmoveto{\pgfqpoint{3.169194in}{3.411286in}}%
\pgfpathlineto{\pgfqpoint{3.169194in}{3.494619in}}%
\pgfusepath{stroke,fill}%
\end{pgfscope}%
\begin{pgfscope}%
\pgfpathrectangle{\pgfqpoint{0.456635in}{3.021449in}}{\pgfqpoint{4.376471in}{0.972632in}}%
\pgfusepath{clip}%
\pgfsetbuttcap%
\pgfsetroundjoin%
\definecolor{currentfill}{rgb}{1.000000,0.000000,0.000000}%
\pgfsetfillcolor{currentfill}%
\pgfsetlinewidth{2.007500pt}%
\definecolor{currentstroke}{rgb}{1.000000,0.000000,0.000000}%
\pgfsetstrokecolor{currentstroke}%
\pgfsetdash{}{0pt}%
\pgfpathmoveto{\pgfqpoint{2.802578in}{3.045704in}}%
\pgfpathlineto{\pgfqpoint{2.885912in}{3.045704in}}%
\pgfpathmoveto{\pgfqpoint{2.844245in}{3.004037in}}%
\pgfpathlineto{\pgfqpoint{2.844245in}{3.087370in}}%
\pgfusepath{stroke,fill}%
\end{pgfscope}%
\begin{pgfscope}%
\pgfpathrectangle{\pgfqpoint{0.456635in}{3.021449in}}{\pgfqpoint{4.376471in}{0.972632in}}%
\pgfusepath{clip}%
\pgfsetbuttcap%
\pgfsetroundjoin%
\definecolor{currentfill}{rgb}{1.000000,0.000000,0.000000}%
\pgfsetfillcolor{currentfill}%
\pgfsetlinewidth{2.007500pt}%
\definecolor{currentstroke}{rgb}{1.000000,0.000000,0.000000}%
\pgfsetstrokecolor{currentstroke}%
\pgfsetdash{}{0pt}%
\pgfpathmoveto{\pgfqpoint{2.309907in}{3.558022in}}%
\pgfpathlineto{\pgfqpoint{2.393241in}{3.558022in}}%
\pgfpathmoveto{\pgfqpoint{2.351574in}{3.516355in}}%
\pgfpathlineto{\pgfqpoint{2.351574in}{3.599689in}}%
\pgfusepath{stroke,fill}%
\end{pgfscope}%
\begin{pgfscope}%
\pgfpathrectangle{\pgfqpoint{0.456635in}{3.021449in}}{\pgfqpoint{4.376471in}{0.972632in}}%
\pgfusepath{clip}%
\pgfsetbuttcap%
\pgfsetroundjoin%
\definecolor{currentfill}{rgb}{1.000000,0.000000,0.000000}%
\pgfsetfillcolor{currentfill}%
\pgfsetlinewidth{2.007500pt}%
\definecolor{currentstroke}{rgb}{1.000000,0.000000,0.000000}%
\pgfsetstrokecolor{currentstroke}%
\pgfsetdash{}{0pt}%
\pgfpathmoveto{\pgfqpoint{3.432468in}{3.519094in}}%
\pgfpathlineto{\pgfqpoint{3.515801in}{3.519094in}}%
\pgfpathmoveto{\pgfqpoint{3.474134in}{3.477427in}}%
\pgfpathlineto{\pgfqpoint{3.474134in}{3.560760in}}%
\pgfusepath{stroke,fill}%
\end{pgfscope}%
\begin{pgfscope}%
\pgfpathrectangle{\pgfqpoint{0.456635in}{3.021449in}}{\pgfqpoint{4.376471in}{0.972632in}}%
\pgfusepath{clip}%
\pgfsetbuttcap%
\pgfsetroundjoin%
\definecolor{currentfill}{rgb}{1.000000,0.000000,0.000000}%
\pgfsetfillcolor{currentfill}%
\pgfsetlinewidth{2.007500pt}%
\definecolor{currentstroke}{rgb}{1.000000,0.000000,0.000000}%
\pgfsetstrokecolor{currentstroke}%
\pgfsetdash{}{0pt}%
\pgfpathmoveto{\pgfqpoint{1.778655in}{3.541907in}}%
\pgfpathlineto{\pgfqpoint{1.861989in}{3.541907in}}%
\pgfpathmoveto{\pgfqpoint{1.820322in}{3.500240in}}%
\pgfpathlineto{\pgfqpoint{1.820322in}{3.583574in}}%
\pgfusepath{stroke,fill}%
\end{pgfscope}%
\begin{pgfscope}%
\pgfpathrectangle{\pgfqpoint{0.456635in}{3.021449in}}{\pgfqpoint{4.376471in}{0.972632in}}%
\pgfusepath{clip}%
\pgfsetbuttcap%
\pgfsetroundjoin%
\definecolor{currentfill}{rgb}{1.000000,0.000000,0.000000}%
\pgfsetfillcolor{currentfill}%
\pgfsetlinewidth{2.007500pt}%
\definecolor{currentstroke}{rgb}{1.000000,0.000000,0.000000}%
\pgfsetstrokecolor{currentstroke}%
\pgfsetdash{}{0pt}%
\pgfpathmoveto{\pgfqpoint{2.313113in}{3.663222in}}%
\pgfpathlineto{\pgfqpoint{2.396446in}{3.663222in}}%
\pgfpathmoveto{\pgfqpoint{2.354779in}{3.621556in}}%
\pgfpathlineto{\pgfqpoint{2.354779in}{3.704889in}}%
\pgfusepath{stroke,fill}%
\end{pgfscope}%
\begin{pgfscope}%
\pgfpathrectangle{\pgfqpoint{0.456635in}{3.021449in}}{\pgfqpoint{4.376471in}{0.972632in}}%
\pgfusepath{clip}%
\pgfsetbuttcap%
\pgfsetroundjoin%
\definecolor{currentfill}{rgb}{1.000000,0.000000,0.000000}%
\pgfsetfillcolor{currentfill}%
\pgfsetlinewidth{2.007500pt}%
\definecolor{currentstroke}{rgb}{1.000000,0.000000,0.000000}%
\pgfsetstrokecolor{currentstroke}%
\pgfsetdash{}{0pt}%
\pgfpathmoveto{\pgfqpoint{2.572960in}{3.144955in}}%
\pgfpathlineto{\pgfqpoint{2.656294in}{3.144955in}}%
\pgfpathmoveto{\pgfqpoint{2.614627in}{3.103289in}}%
\pgfpathlineto{\pgfqpoint{2.614627in}{3.186622in}}%
\pgfusepath{stroke,fill}%
\end{pgfscope}%
\begin{pgfscope}%
\pgfpathrectangle{\pgfqpoint{0.456635in}{3.021449in}}{\pgfqpoint{4.376471in}{0.972632in}}%
\pgfusepath{clip}%
\pgfsetbuttcap%
\pgfsetroundjoin%
\definecolor{currentfill}{rgb}{1.000000,0.000000,0.000000}%
\pgfsetfillcolor{currentfill}%
\pgfsetlinewidth{2.007500pt}%
\definecolor{currentstroke}{rgb}{1.000000,0.000000,0.000000}%
\pgfsetstrokecolor{currentstroke}%
\pgfsetdash{}{0pt}%
\pgfpathmoveto{\pgfqpoint{2.887044in}{3.266154in}}%
\pgfpathlineto{\pgfqpoint{2.970378in}{3.266154in}}%
\pgfpathmoveto{\pgfqpoint{2.928711in}{3.224487in}}%
\pgfpathlineto{\pgfqpoint{2.928711in}{3.307821in}}%
\pgfusepath{stroke,fill}%
\end{pgfscope}%
\begin{pgfscope}%
\pgfpathrectangle{\pgfqpoint{0.456635in}{3.021449in}}{\pgfqpoint{4.376471in}{0.972632in}}%
\pgfusepath{clip}%
\pgfsetbuttcap%
\pgfsetroundjoin%
\definecolor{currentfill}{rgb}{1.000000,0.000000,0.000000}%
\pgfsetfillcolor{currentfill}%
\pgfsetlinewidth{2.007500pt}%
\definecolor{currentstroke}{rgb}{1.000000,0.000000,0.000000}%
\pgfsetstrokecolor{currentstroke}%
\pgfsetdash{}{0pt}%
\pgfpathmoveto{\pgfqpoint{4.039302in}{3.368013in}}%
\pgfpathlineto{\pgfqpoint{4.122636in}{3.368013in}}%
\pgfpathmoveto{\pgfqpoint{4.080969in}{3.326347in}}%
\pgfpathlineto{\pgfqpoint{4.080969in}{3.409680in}}%
\pgfusepath{stroke,fill}%
\end{pgfscope}%
\begin{pgfscope}%
\pgfpathrectangle{\pgfqpoint{0.456635in}{3.021449in}}{\pgfqpoint{4.376471in}{0.972632in}}%
\pgfusepath{clip}%
\pgfsetbuttcap%
\pgfsetroundjoin%
\definecolor{currentfill}{rgb}{1.000000,0.000000,0.000000}%
\pgfsetfillcolor{currentfill}%
\pgfsetlinewidth{2.007500pt}%
\definecolor{currentstroke}{rgb}{1.000000,0.000000,0.000000}%
\pgfsetstrokecolor{currentstroke}%
\pgfsetdash{}{0pt}%
\pgfpathmoveto{\pgfqpoint{1.989356in}{3.519163in}}%
\pgfpathlineto{\pgfqpoint{2.072689in}{3.519163in}}%
\pgfpathmoveto{\pgfqpoint{2.031023in}{3.477497in}}%
\pgfpathlineto{\pgfqpoint{2.031023in}{3.560830in}}%
\pgfusepath{stroke,fill}%
\end{pgfscope}%
\begin{pgfscope}%
\pgfpathrectangle{\pgfqpoint{0.456635in}{3.021449in}}{\pgfqpoint{4.376471in}{0.972632in}}%
\pgfusepath{clip}%
\pgfsetbuttcap%
\pgfsetroundjoin%
\definecolor{currentfill}{rgb}{1.000000,0.000000,0.000000}%
\pgfsetfillcolor{currentfill}%
\pgfsetlinewidth{2.007500pt}%
\definecolor{currentstroke}{rgb}{1.000000,0.000000,0.000000}%
\pgfsetstrokecolor{currentstroke}%
\pgfsetdash{}{0pt}%
\pgfpathmoveto{\pgfqpoint{3.090688in}{3.505398in}}%
\pgfpathlineto{\pgfqpoint{3.174022in}{3.505398in}}%
\pgfpathmoveto{\pgfqpoint{3.132355in}{3.463731in}}%
\pgfpathlineto{\pgfqpoint{3.132355in}{3.547065in}}%
\pgfusepath{stroke,fill}%
\end{pgfscope}%
\begin{pgfscope}%
\pgfpathrectangle{\pgfqpoint{0.456635in}{3.021449in}}{\pgfqpoint{4.376471in}{0.972632in}}%
\pgfusepath{clip}%
\pgfsetbuttcap%
\pgfsetroundjoin%
\definecolor{currentfill}{rgb}{1.000000,0.000000,0.000000}%
\pgfsetfillcolor{currentfill}%
\pgfsetlinewidth{2.007500pt}%
\definecolor{currentstroke}{rgb}{1.000000,0.000000,0.000000}%
\pgfsetstrokecolor{currentstroke}%
\pgfsetdash{}{0pt}%
\pgfpathmoveto{\pgfqpoint{3.364411in}{3.699367in}}%
\pgfpathlineto{\pgfqpoint{3.447744in}{3.699367in}}%
\pgfpathmoveto{\pgfqpoint{3.406077in}{3.657700in}}%
\pgfpathlineto{\pgfqpoint{3.406077in}{3.741034in}}%
\pgfusepath{stroke,fill}%
\end{pgfscope}%
\begin{pgfscope}%
\pgfpathrectangle{\pgfqpoint{0.456635in}{3.021449in}}{\pgfqpoint{4.376471in}{0.972632in}}%
\pgfusepath{clip}%
\pgfsetbuttcap%
\pgfsetroundjoin%
\definecolor{currentfill}{rgb}{1.000000,0.000000,0.000000}%
\pgfsetfillcolor{currentfill}%
\pgfsetlinewidth{2.007500pt}%
\definecolor{currentstroke}{rgb}{1.000000,0.000000,0.000000}%
\pgfsetstrokecolor{currentstroke}%
\pgfsetdash{}{0pt}%
\pgfpathmoveto{\pgfqpoint{1.452894in}{3.743800in}}%
\pgfpathlineto{\pgfqpoint{1.536227in}{3.743800in}}%
\pgfpathmoveto{\pgfqpoint{1.494560in}{3.702133in}}%
\pgfpathlineto{\pgfqpoint{1.494560in}{3.785466in}}%
\pgfusepath{stroke,fill}%
\end{pgfscope}%
\begin{pgfscope}%
\pgfpathrectangle{\pgfqpoint{0.456635in}{3.021449in}}{\pgfqpoint{4.376471in}{0.972632in}}%
\pgfusepath{clip}%
\pgfsetbuttcap%
\pgfsetroundjoin%
\definecolor{currentfill}{rgb}{0.000000,0.000000,0.000000}%
\pgfsetfillcolor{currentfill}%
\pgfsetlinewidth{1.003750pt}%
\definecolor{currentstroke}{rgb}{0.000000,0.000000,0.000000}%
\pgfsetstrokecolor{currentstroke}%
\pgfsetdash{}{0pt}%
\pgfsys@defobject{currentmarker}{\pgfqpoint{-0.020833in}{-0.020833in}}{\pgfqpoint{0.020833in}{0.020833in}}{%
\pgfpathmoveto{\pgfqpoint{0.000000in}{-0.020833in}}%
\pgfpathcurveto{\pgfqpoint{0.005525in}{-0.020833in}}{\pgfqpoint{0.010825in}{-0.018638in}}{\pgfqpoint{0.014731in}{-0.014731in}}%
\pgfpathcurveto{\pgfqpoint{0.018638in}{-0.010825in}}{\pgfqpoint{0.020833in}{-0.005525in}}{\pgfqpoint{0.020833in}{0.000000in}}%
\pgfpathcurveto{\pgfqpoint{0.020833in}{0.005525in}}{\pgfqpoint{0.018638in}{0.010825in}}{\pgfqpoint{0.014731in}{0.014731in}}%
\pgfpathcurveto{\pgfqpoint{0.010825in}{0.018638in}}{\pgfqpoint{0.005525in}{0.020833in}}{\pgfqpoint{0.000000in}{0.020833in}}%
\pgfpathcurveto{\pgfqpoint{-0.005525in}{0.020833in}}{\pgfqpoint{-0.010825in}{0.018638in}}{\pgfqpoint{-0.014731in}{0.014731in}}%
\pgfpathcurveto{\pgfqpoint{-0.018638in}{0.010825in}}{\pgfqpoint{-0.020833in}{0.005525in}}{\pgfqpoint{-0.020833in}{0.000000in}}%
\pgfpathcurveto{\pgfqpoint{-0.020833in}{-0.005525in}}{\pgfqpoint{-0.018638in}{-0.010825in}}{\pgfqpoint{-0.014731in}{-0.014731in}}%
\pgfpathcurveto{\pgfqpoint{-0.010825in}{-0.018638in}}{\pgfqpoint{-0.005525in}{-0.020833in}}{\pgfqpoint{0.000000in}{-0.020833in}}%
\pgfpathclose%
\pgfusepath{stroke,fill}%
}%
\begin{pgfscope}%
\pgfsys@transformshift{1.331929in}{3.816659in}%
\pgfsys@useobject{currentmarker}{}%
\end{pgfscope}%
\begin{pgfscope}%
\pgfsys@transformshift{1.349523in}{3.844637in}%
\pgfsys@useobject{currentmarker}{}%
\end{pgfscope}%
\begin{pgfscope}%
\pgfsys@transformshift{1.367117in}{3.882954in}%
\pgfsys@useobject{currentmarker}{}%
\end{pgfscope}%
\begin{pgfscope}%
\pgfsys@transformshift{1.384711in}{3.920991in}%
\pgfsys@useobject{currentmarker}{}%
\end{pgfscope}%
\begin{pgfscope}%
\pgfsys@transformshift{1.402305in}{3.745895in}%
\pgfsys@useobject{currentmarker}{}%
\end{pgfscope}%
\begin{pgfscope}%
\pgfsys@transformshift{1.419899in}{3.747653in}%
\pgfsys@useobject{currentmarker}{}%
\end{pgfscope}%
\begin{pgfscope}%
\pgfsys@transformshift{1.437493in}{3.626327in}%
\pgfsys@useobject{currentmarker}{}%
\end{pgfscope}%
\begin{pgfscope}%
\pgfsys@transformshift{1.455086in}{3.589313in}%
\pgfsys@useobject{currentmarker}{}%
\end{pgfscope}%
\begin{pgfscope}%
\pgfsys@transformshift{1.472680in}{3.765265in}%
\pgfsys@useobject{currentmarker}{}%
\end{pgfscope}%
\begin{pgfscope}%
\pgfsys@transformshift{1.490274in}{3.793332in}%
\pgfsys@useobject{currentmarker}{}%
\end{pgfscope}%
\begin{pgfscope}%
\pgfsys@transformshift{1.507868in}{3.622352in}%
\pgfsys@useobject{currentmarker}{}%
\end{pgfscope}%
\begin{pgfscope}%
\pgfsys@transformshift{1.525462in}{3.705980in}%
\pgfsys@useobject{currentmarker}{}%
\end{pgfscope}%
\begin{pgfscope}%
\pgfsys@transformshift{1.543056in}{3.616720in}%
\pgfsys@useobject{currentmarker}{}%
\end{pgfscope}%
\begin{pgfscope}%
\pgfsys@transformshift{1.560649in}{3.491782in}%
\pgfsys@useobject{currentmarker}{}%
\end{pgfscope}%
\begin{pgfscope}%
\pgfsys@transformshift{1.578243in}{3.572268in}%
\pgfsys@useobject{currentmarker}{}%
\end{pgfscope}%
\begin{pgfscope}%
\pgfsys@transformshift{1.595837in}{3.671583in}%
\pgfsys@useobject{currentmarker}{}%
\end{pgfscope}%
\begin{pgfscope}%
\pgfsys@transformshift{1.613431in}{3.493954in}%
\pgfsys@useobject{currentmarker}{}%
\end{pgfscope}%
\begin{pgfscope}%
\pgfsys@transformshift{1.631025in}{3.639772in}%
\pgfsys@useobject{currentmarker}{}%
\end{pgfscope}%
\begin{pgfscope}%
\pgfsys@transformshift{1.648619in}{3.201427in}%
\pgfsys@useobject{currentmarker}{}%
\end{pgfscope}%
\begin{pgfscope}%
\pgfsys@transformshift{1.666213in}{3.537732in}%
\pgfsys@useobject{currentmarker}{}%
\end{pgfscope}%
\begin{pgfscope}%
\pgfsys@transformshift{1.683806in}{3.452961in}%
\pgfsys@useobject{currentmarker}{}%
\end{pgfscope}%
\begin{pgfscope}%
\pgfsys@transformshift{1.701400in}{3.405637in}%
\pgfsys@useobject{currentmarker}{}%
\end{pgfscope}%
\begin{pgfscope}%
\pgfsys@transformshift{1.718994in}{3.439143in}%
\pgfsys@useobject{currentmarker}{}%
\end{pgfscope}%
\begin{pgfscope}%
\pgfsys@transformshift{1.736588in}{3.224512in}%
\pgfsys@useobject{currentmarker}{}%
\end{pgfscope}%
\begin{pgfscope}%
\pgfsys@transformshift{1.754182in}{3.401765in}%
\pgfsys@useobject{currentmarker}{}%
\end{pgfscope}%
\begin{pgfscope}%
\pgfsys@transformshift{1.771776in}{3.460394in}%
\pgfsys@useobject{currentmarker}{}%
\end{pgfscope}%
\begin{pgfscope}%
\pgfsys@transformshift{1.789370in}{3.576129in}%
\pgfsys@useobject{currentmarker}{}%
\end{pgfscope}%
\begin{pgfscope}%
\pgfsys@transformshift{1.806963in}{3.377977in}%
\pgfsys@useobject{currentmarker}{}%
\end{pgfscope}%
\begin{pgfscope}%
\pgfsys@transformshift{1.824557in}{3.354469in}%
\pgfsys@useobject{currentmarker}{}%
\end{pgfscope}%
\begin{pgfscope}%
\pgfsys@transformshift{1.842151in}{3.393130in}%
\pgfsys@useobject{currentmarker}{}%
\end{pgfscope}%
\begin{pgfscope}%
\pgfsys@transformshift{1.859745in}{3.545847in}%
\pgfsys@useobject{currentmarker}{}%
\end{pgfscope}%
\begin{pgfscope}%
\pgfsys@transformshift{1.877339in}{3.496948in}%
\pgfsys@useobject{currentmarker}{}%
\end{pgfscope}%
\begin{pgfscope}%
\pgfsys@transformshift{1.894933in}{3.421744in}%
\pgfsys@useobject{currentmarker}{}%
\end{pgfscope}%
\begin{pgfscope}%
\pgfsys@transformshift{1.912526in}{3.540260in}%
\pgfsys@useobject{currentmarker}{}%
\end{pgfscope}%
\begin{pgfscope}%
\pgfsys@transformshift{1.930120in}{3.511812in}%
\pgfsys@useobject{currentmarker}{}%
\end{pgfscope}%
\begin{pgfscope}%
\pgfsys@transformshift{1.947714in}{3.614520in}%
\pgfsys@useobject{currentmarker}{}%
\end{pgfscope}%
\begin{pgfscope}%
\pgfsys@transformshift{1.965308in}{3.460144in}%
\pgfsys@useobject{currentmarker}{}%
\end{pgfscope}%
\begin{pgfscope}%
\pgfsys@transformshift{1.982902in}{3.513250in}%
\pgfsys@useobject{currentmarker}{}%
\end{pgfscope}%
\begin{pgfscope}%
\pgfsys@transformshift{2.000496in}{3.521974in}%
\pgfsys@useobject{currentmarker}{}%
\end{pgfscope}%
\begin{pgfscope}%
\pgfsys@transformshift{2.018090in}{3.428549in}%
\pgfsys@useobject{currentmarker}{}%
\end{pgfscope}%
\begin{pgfscope}%
\pgfsys@transformshift{2.035683in}{3.621624in}%
\pgfsys@useobject{currentmarker}{}%
\end{pgfscope}%
\begin{pgfscope}%
\pgfsys@transformshift{2.053277in}{3.632339in}%
\pgfsys@useobject{currentmarker}{}%
\end{pgfscope}%
\begin{pgfscope}%
\pgfsys@transformshift{2.070871in}{3.619953in}%
\pgfsys@useobject{currentmarker}{}%
\end{pgfscope}%
\begin{pgfscope}%
\pgfsys@transformshift{2.088465in}{3.608307in}%
\pgfsys@useobject{currentmarker}{}%
\end{pgfscope}%
\begin{pgfscope}%
\pgfsys@transformshift{2.106059in}{3.500233in}%
\pgfsys@useobject{currentmarker}{}%
\end{pgfscope}%
\begin{pgfscope}%
\pgfsys@transformshift{2.123653in}{3.611329in}%
\pgfsys@useobject{currentmarker}{}%
\end{pgfscope}%
\begin{pgfscope}%
\pgfsys@transformshift{2.141247in}{3.628146in}%
\pgfsys@useobject{currentmarker}{}%
\end{pgfscope}%
\begin{pgfscope}%
\pgfsys@transformshift{2.158840in}{3.588978in}%
\pgfsys@useobject{currentmarker}{}%
\end{pgfscope}%
\begin{pgfscope}%
\pgfsys@transformshift{2.176434in}{3.659668in}%
\pgfsys@useobject{currentmarker}{}%
\end{pgfscope}%
\begin{pgfscope}%
\pgfsys@transformshift{2.194028in}{3.720944in}%
\pgfsys@useobject{currentmarker}{}%
\end{pgfscope}%
\begin{pgfscope}%
\pgfsys@transformshift{2.211622in}{3.873278in}%
\pgfsys@useobject{currentmarker}{}%
\end{pgfscope}%
\begin{pgfscope}%
\pgfsys@transformshift{2.229216in}{3.700141in}%
\pgfsys@useobject{currentmarker}{}%
\end{pgfscope}%
\begin{pgfscope}%
\pgfsys@transformshift{2.246810in}{3.706888in}%
\pgfsys@useobject{currentmarker}{}%
\end{pgfscope}%
\begin{pgfscope}%
\pgfsys@transformshift{2.264404in}{3.669632in}%
\pgfsys@useobject{currentmarker}{}%
\end{pgfscope}%
\begin{pgfscope}%
\pgfsys@transformshift{2.281997in}{3.477192in}%
\pgfsys@useobject{currentmarker}{}%
\end{pgfscope}%
\begin{pgfscope}%
\pgfsys@transformshift{2.299591in}{3.661383in}%
\pgfsys@useobject{currentmarker}{}%
\end{pgfscope}%
\begin{pgfscope}%
\pgfsys@transformshift{2.317185in}{3.660743in}%
\pgfsys@useobject{currentmarker}{}%
\end{pgfscope}%
\begin{pgfscope}%
\pgfsys@transformshift{2.334779in}{3.892932in}%
\pgfsys@useobject{currentmarker}{}%
\end{pgfscope}%
\begin{pgfscope}%
\pgfsys@transformshift{2.352373in}{3.610835in}%
\pgfsys@useobject{currentmarker}{}%
\end{pgfscope}%
\begin{pgfscope}%
\pgfsys@transformshift{2.369967in}{3.646164in}%
\pgfsys@useobject{currentmarker}{}%
\end{pgfscope}%
\begin{pgfscope}%
\pgfsys@transformshift{2.387560in}{3.595827in}%
\pgfsys@useobject{currentmarker}{}%
\end{pgfscope}%
\begin{pgfscope}%
\pgfsys@transformshift{2.405154in}{3.463245in}%
\pgfsys@useobject{currentmarker}{}%
\end{pgfscope}%
\begin{pgfscope}%
\pgfsys@transformshift{2.422748in}{3.678465in}%
\pgfsys@useobject{currentmarker}{}%
\end{pgfscope}%
\begin{pgfscope}%
\pgfsys@transformshift{2.440342in}{3.618772in}%
\pgfsys@useobject{currentmarker}{}%
\end{pgfscope}%
\begin{pgfscope}%
\pgfsys@transformshift{2.457936in}{3.601697in}%
\pgfsys@useobject{currentmarker}{}%
\end{pgfscope}%
\begin{pgfscope}%
\pgfsys@transformshift{2.475530in}{3.407618in}%
\pgfsys@useobject{currentmarker}{}%
\end{pgfscope}%
\begin{pgfscope}%
\pgfsys@transformshift{2.493124in}{3.619506in}%
\pgfsys@useobject{currentmarker}{}%
\end{pgfscope}%
\begin{pgfscope}%
\pgfsys@transformshift{2.510717in}{3.312604in}%
\pgfsys@useobject{currentmarker}{}%
\end{pgfscope}%
\begin{pgfscope}%
\pgfsys@transformshift{2.528311in}{3.491172in}%
\pgfsys@useobject{currentmarker}{}%
\end{pgfscope}%
\begin{pgfscope}%
\pgfsys@transformshift{2.545905in}{3.630754in}%
\pgfsys@useobject{currentmarker}{}%
\end{pgfscope}%
\begin{pgfscope}%
\pgfsys@transformshift{2.563499in}{3.285820in}%
\pgfsys@useobject{currentmarker}{}%
\end{pgfscope}%
\begin{pgfscope}%
\pgfsys@transformshift{2.581093in}{3.306594in}%
\pgfsys@useobject{currentmarker}{}%
\end{pgfscope}%
\begin{pgfscope}%
\pgfsys@transformshift{2.598687in}{3.352498in}%
\pgfsys@useobject{currentmarker}{}%
\end{pgfscope}%
\begin{pgfscope}%
\pgfsys@transformshift{2.616281in}{3.270661in}%
\pgfsys@useobject{currentmarker}{}%
\end{pgfscope}%
\begin{pgfscope}%
\pgfsys@transformshift{2.633874in}{3.144857in}%
\pgfsys@useobject{currentmarker}{}%
\end{pgfscope}%
\begin{pgfscope}%
\pgfsys@transformshift{2.651468in}{3.290404in}%
\pgfsys@useobject{currentmarker}{}%
\end{pgfscope}%
\begin{pgfscope}%
\pgfsys@transformshift{2.669062in}{3.158689in}%
\pgfsys@useobject{currentmarker}{}%
\end{pgfscope}%
\begin{pgfscope}%
\pgfsys@transformshift{2.686656in}{3.298679in}%
\pgfsys@useobject{currentmarker}{}%
\end{pgfscope}%
\begin{pgfscope}%
\pgfsys@transformshift{2.704250in}{3.143582in}%
\pgfsys@useobject{currentmarker}{}%
\end{pgfscope}%
\begin{pgfscope}%
\pgfsys@transformshift{2.721844in}{3.381586in}%
\pgfsys@useobject{currentmarker}{}%
\end{pgfscope}%
\begin{pgfscope}%
\pgfsys@transformshift{2.739438in}{3.134903in}%
\pgfsys@useobject{currentmarker}{}%
\end{pgfscope}%
\begin{pgfscope}%
\pgfsys@transformshift{2.757031in}{3.173309in}%
\pgfsys@useobject{currentmarker}{}%
\end{pgfscope}%
\begin{pgfscope}%
\pgfsys@transformshift{2.774625in}{3.282083in}%
\pgfsys@useobject{currentmarker}{}%
\end{pgfscope}%
\begin{pgfscope}%
\pgfsys@transformshift{2.792219in}{3.070768in}%
\pgfsys@useobject{currentmarker}{}%
\end{pgfscope}%
\begin{pgfscope}%
\pgfsys@transformshift{2.809813in}{3.216451in}%
\pgfsys@useobject{currentmarker}{}%
\end{pgfscope}%
\begin{pgfscope}%
\pgfsys@transformshift{2.827407in}{3.325899in}%
\pgfsys@useobject{currentmarker}{}%
\end{pgfscope}%
\begin{pgfscope}%
\pgfsys@transformshift{2.845001in}{3.032774in}%
\pgfsys@useobject{currentmarker}{}%
\end{pgfscope}%
\begin{pgfscope}%
\pgfsys@transformshift{2.862594in}{3.218597in}%
\pgfsys@useobject{currentmarker}{}%
\end{pgfscope}%
\begin{pgfscope}%
\pgfsys@transformshift{2.880188in}{3.232500in}%
\pgfsys@useobject{currentmarker}{}%
\end{pgfscope}%
\begin{pgfscope}%
\pgfsys@transformshift{2.897782in}{3.293613in}%
\pgfsys@useobject{currentmarker}{}%
\end{pgfscope}%
\begin{pgfscope}%
\pgfsys@transformshift{2.915376in}{3.099173in}%
\pgfsys@useobject{currentmarker}{}%
\end{pgfscope}%
\begin{pgfscope}%
\pgfsys@transformshift{2.932970in}{3.102558in}%
\pgfsys@useobject{currentmarker}{}%
\end{pgfscope}%
\begin{pgfscope}%
\pgfsys@transformshift{2.950564in}{3.302690in}%
\pgfsys@useobject{currentmarker}{}%
\end{pgfscope}%
\begin{pgfscope}%
\pgfsys@transformshift{2.968158in}{3.294849in}%
\pgfsys@useobject{currentmarker}{}%
\end{pgfscope}%
\begin{pgfscope}%
\pgfsys@transformshift{2.985751in}{3.306413in}%
\pgfsys@useobject{currentmarker}{}%
\end{pgfscope}%
\begin{pgfscope}%
\pgfsys@transformshift{3.003345in}{3.333566in}%
\pgfsys@useobject{currentmarker}{}%
\end{pgfscope}%
\begin{pgfscope}%
\pgfsys@transformshift{3.020939in}{3.247974in}%
\pgfsys@useobject{currentmarker}{}%
\end{pgfscope}%
\begin{pgfscope}%
\pgfsys@transformshift{3.038533in}{3.359594in}%
\pgfsys@useobject{currentmarker}{}%
\end{pgfscope}%
\begin{pgfscope}%
\pgfsys@transformshift{3.056127in}{3.385539in}%
\pgfsys@useobject{currentmarker}{}%
\end{pgfscope}%
\begin{pgfscope}%
\pgfsys@transformshift{3.073721in}{3.303643in}%
\pgfsys@useobject{currentmarker}{}%
\end{pgfscope}%
\begin{pgfscope}%
\pgfsys@transformshift{3.091315in}{3.585407in}%
\pgfsys@useobject{currentmarker}{}%
\end{pgfscope}%
\begin{pgfscope}%
\pgfsys@transformshift{3.108908in}{3.464717in}%
\pgfsys@useobject{currentmarker}{}%
\end{pgfscope}%
\begin{pgfscope}%
\pgfsys@transformshift{3.126502in}{3.316123in}%
\pgfsys@useobject{currentmarker}{}%
\end{pgfscope}%
\begin{pgfscope}%
\pgfsys@transformshift{3.144096in}{3.523023in}%
\pgfsys@useobject{currentmarker}{}%
\end{pgfscope}%
\begin{pgfscope}%
\pgfsys@transformshift{3.161690in}{3.376814in}%
\pgfsys@useobject{currentmarker}{}%
\end{pgfscope}%
\begin{pgfscope}%
\pgfsys@transformshift{3.179284in}{3.573558in}%
\pgfsys@useobject{currentmarker}{}%
\end{pgfscope}%
\begin{pgfscope}%
\pgfsys@transformshift{3.196878in}{3.628456in}%
\pgfsys@useobject{currentmarker}{}%
\end{pgfscope}%
\begin{pgfscope}%
\pgfsys@transformshift{3.214472in}{3.444020in}%
\pgfsys@useobject{currentmarker}{}%
\end{pgfscope}%
\begin{pgfscope}%
\pgfsys@transformshift{3.232065in}{3.639552in}%
\pgfsys@useobject{currentmarker}{}%
\end{pgfscope}%
\begin{pgfscope}%
\pgfsys@transformshift{3.249659in}{3.597086in}%
\pgfsys@useobject{currentmarker}{}%
\end{pgfscope}%
\begin{pgfscope}%
\pgfsys@transformshift{3.267253in}{3.650284in}%
\pgfsys@useobject{currentmarker}{}%
\end{pgfscope}%
\begin{pgfscope}%
\pgfsys@transformshift{3.284847in}{3.769205in}%
\pgfsys@useobject{currentmarker}{}%
\end{pgfscope}%
\begin{pgfscope}%
\pgfsys@transformshift{3.302441in}{3.560412in}%
\pgfsys@useobject{currentmarker}{}%
\end{pgfscope}%
\begin{pgfscope}%
\pgfsys@transformshift{3.320035in}{3.515290in}%
\pgfsys@useobject{currentmarker}{}%
\end{pgfscope}%
\begin{pgfscope}%
\pgfsys@transformshift{3.337628in}{3.505998in}%
\pgfsys@useobject{currentmarker}{}%
\end{pgfscope}%
\begin{pgfscope}%
\pgfsys@transformshift{3.355222in}{3.515981in}%
\pgfsys@useobject{currentmarker}{}%
\end{pgfscope}%
\begin{pgfscope}%
\pgfsys@transformshift{3.372816in}{3.591375in}%
\pgfsys@useobject{currentmarker}{}%
\end{pgfscope}%
\begin{pgfscope}%
\pgfsys@transformshift{3.390410in}{3.632348in}%
\pgfsys@useobject{currentmarker}{}%
\end{pgfscope}%
\begin{pgfscope}%
\pgfsys@transformshift{3.408004in}{3.622486in}%
\pgfsys@useobject{currentmarker}{}%
\end{pgfscope}%
\begin{pgfscope}%
\pgfsys@transformshift{3.425598in}{3.673051in}%
\pgfsys@useobject{currentmarker}{}%
\end{pgfscope}%
\begin{pgfscope}%
\pgfsys@transformshift{3.443192in}{3.583542in}%
\pgfsys@useobject{currentmarker}{}%
\end{pgfscope}%
\begin{pgfscope}%
\pgfsys@transformshift{3.460785in}{3.720748in}%
\pgfsys@useobject{currentmarker}{}%
\end{pgfscope}%
\begin{pgfscope}%
\pgfsys@transformshift{3.478379in}{3.536308in}%
\pgfsys@useobject{currentmarker}{}%
\end{pgfscope}%
\begin{pgfscope}%
\pgfsys@transformshift{3.495973in}{3.826864in}%
\pgfsys@useobject{currentmarker}{}%
\end{pgfscope}%
\begin{pgfscope}%
\pgfsys@transformshift{3.513567in}{3.601447in}%
\pgfsys@useobject{currentmarker}{}%
\end{pgfscope}%
\begin{pgfscope}%
\pgfsys@transformshift{3.531161in}{3.436800in}%
\pgfsys@useobject{currentmarker}{}%
\end{pgfscope}%
\begin{pgfscope}%
\pgfsys@transformshift{3.548755in}{3.399695in}%
\pgfsys@useobject{currentmarker}{}%
\end{pgfscope}%
\begin{pgfscope}%
\pgfsys@transformshift{3.566349in}{3.540766in}%
\pgfsys@useobject{currentmarker}{}%
\end{pgfscope}%
\begin{pgfscope}%
\pgfsys@transformshift{3.583942in}{3.452262in}%
\pgfsys@useobject{currentmarker}{}%
\end{pgfscope}%
\begin{pgfscope}%
\pgfsys@transformshift{3.601536in}{3.529782in}%
\pgfsys@useobject{currentmarker}{}%
\end{pgfscope}%
\begin{pgfscope}%
\pgfsys@transformshift{3.619130in}{3.487651in}%
\pgfsys@useobject{currentmarker}{}%
\end{pgfscope}%
\begin{pgfscope}%
\pgfsys@transformshift{3.636724in}{3.414514in}%
\pgfsys@useobject{currentmarker}{}%
\end{pgfscope}%
\begin{pgfscope}%
\pgfsys@transformshift{3.654318in}{3.318419in}%
\pgfsys@useobject{currentmarker}{}%
\end{pgfscope}%
\begin{pgfscope}%
\pgfsys@transformshift{3.671912in}{3.233391in}%
\pgfsys@useobject{currentmarker}{}%
\end{pgfscope}%
\begin{pgfscope}%
\pgfsys@transformshift{3.689506in}{3.324825in}%
\pgfsys@useobject{currentmarker}{}%
\end{pgfscope}%
\begin{pgfscope}%
\pgfsys@transformshift{3.707099in}{3.440761in}%
\pgfsys@useobject{currentmarker}{}%
\end{pgfscope}%
\begin{pgfscope}%
\pgfsys@transformshift{3.724693in}{3.360544in}%
\pgfsys@useobject{currentmarker}{}%
\end{pgfscope}%
\begin{pgfscope}%
\pgfsys@transformshift{3.742287in}{3.198609in}%
\pgfsys@useobject{currentmarker}{}%
\end{pgfscope}%
\begin{pgfscope}%
\pgfsys@transformshift{3.759881in}{3.329618in}%
\pgfsys@useobject{currentmarker}{}%
\end{pgfscope}%
\begin{pgfscope}%
\pgfsys@transformshift{3.777475in}{3.339804in}%
\pgfsys@useobject{currentmarker}{}%
\end{pgfscope}%
\begin{pgfscope}%
\pgfsys@transformshift{3.795069in}{3.201497in}%
\pgfsys@useobject{currentmarker}{}%
\end{pgfscope}%
\begin{pgfscope}%
\pgfsys@transformshift{3.812662in}{3.298622in}%
\pgfsys@useobject{currentmarker}{}%
\end{pgfscope}%
\begin{pgfscope}%
\pgfsys@transformshift{3.830256in}{3.282780in}%
\pgfsys@useobject{currentmarker}{}%
\end{pgfscope}%
\begin{pgfscope}%
\pgfsys@transformshift{3.847850in}{3.156848in}%
\pgfsys@useobject{currentmarker}{}%
\end{pgfscope}%
\begin{pgfscope}%
\pgfsys@transformshift{3.865444in}{3.306674in}%
\pgfsys@useobject{currentmarker}{}%
\end{pgfscope}%
\begin{pgfscope}%
\pgfsys@transformshift{3.883038in}{3.327085in}%
\pgfsys@useobject{currentmarker}{}%
\end{pgfscope}%
\begin{pgfscope}%
\pgfsys@transformshift{3.900632in}{3.381950in}%
\pgfsys@useobject{currentmarker}{}%
\end{pgfscope}%
\begin{pgfscope}%
\pgfsys@transformshift{3.918226in}{3.383063in}%
\pgfsys@useobject{currentmarker}{}%
\end{pgfscope}%
\begin{pgfscope}%
\pgfsys@transformshift{3.935819in}{3.142916in}%
\pgfsys@useobject{currentmarker}{}%
\end{pgfscope}%
\begin{pgfscope}%
\pgfsys@transformshift{3.953413in}{3.195779in}%
\pgfsys@useobject{currentmarker}{}%
\end{pgfscope}%
\begin{pgfscope}%
\pgfsys@transformshift{3.971007in}{3.353333in}%
\pgfsys@useobject{currentmarker}{}%
\end{pgfscope}%
\begin{pgfscope}%
\pgfsys@transformshift{3.988601in}{3.365556in}%
\pgfsys@useobject{currentmarker}{}%
\end{pgfscope}%
\begin{pgfscope}%
\pgfsys@transformshift{4.006195in}{3.379945in}%
\pgfsys@useobject{currentmarker}{}%
\end{pgfscope}%
\begin{pgfscope}%
\pgfsys@transformshift{4.023789in}{3.734177in}%
\pgfsys@useobject{currentmarker}{}%
\end{pgfscope}%
\begin{pgfscope}%
\pgfsys@transformshift{4.041383in}{3.419437in}%
\pgfsys@useobject{currentmarker}{}%
\end{pgfscope}%
\begin{pgfscope}%
\pgfsys@transformshift{4.058976in}{3.495968in}%
\pgfsys@useobject{currentmarker}{}%
\end{pgfscope}%
\begin{pgfscope}%
\pgfsys@transformshift{4.076570in}{3.498300in}%
\pgfsys@useobject{currentmarker}{}%
\end{pgfscope}%
\begin{pgfscope}%
\pgfsys@transformshift{4.094164in}{3.489614in}%
\pgfsys@useobject{currentmarker}{}%
\end{pgfscope}%
\begin{pgfscope}%
\pgfsys@transformshift{4.111758in}{3.414718in}%
\pgfsys@useobject{currentmarker}{}%
\end{pgfscope}%
\begin{pgfscope}%
\pgfsys@transformshift{4.129352in}{3.547483in}%
\pgfsys@useobject{currentmarker}{}%
\end{pgfscope}%
\begin{pgfscope}%
\pgfsys@transformshift{4.146946in}{3.416908in}%
\pgfsys@useobject{currentmarker}{}%
\end{pgfscope}%
\begin{pgfscope}%
\pgfsys@transformshift{4.164540in}{3.496328in}%
\pgfsys@useobject{currentmarker}{}%
\end{pgfscope}%
\begin{pgfscope}%
\pgfsys@transformshift{4.182133in}{3.496552in}%
\pgfsys@useobject{currentmarker}{}%
\end{pgfscope}%
\begin{pgfscope}%
\pgfsys@transformshift{4.199727in}{3.579512in}%
\pgfsys@useobject{currentmarker}{}%
\end{pgfscope}%
\begin{pgfscope}%
\pgfsys@transformshift{4.217321in}{3.831099in}%
\pgfsys@useobject{currentmarker}{}%
\end{pgfscope}%
\begin{pgfscope}%
\pgfsys@transformshift{4.234915in}{3.432452in}%
\pgfsys@useobject{currentmarker}{}%
\end{pgfscope}%
\begin{pgfscope}%
\pgfsys@transformshift{4.252509in}{3.715690in}%
\pgfsys@useobject{currentmarker}{}%
\end{pgfscope}%
\begin{pgfscope}%
\pgfsys@transformshift{4.270103in}{3.506578in}%
\pgfsys@useobject{currentmarker}{}%
\end{pgfscope}%
\begin{pgfscope}%
\pgfsys@transformshift{4.287696in}{3.645069in}%
\pgfsys@useobject{currentmarker}{}%
\end{pgfscope}%
\begin{pgfscope}%
\pgfsys@transformshift{4.305290in}{3.825047in}%
\pgfsys@useobject{currentmarker}{}%
\end{pgfscope}%
\begin{pgfscope}%
\pgfsys@transformshift{4.322884in}{3.741829in}%
\pgfsys@useobject{currentmarker}{}%
\end{pgfscope}%
\begin{pgfscope}%
\pgfsys@transformshift{4.340478in}{3.645333in}%
\pgfsys@useobject{currentmarker}{}%
\end{pgfscope}%
\begin{pgfscope}%
\pgfsys@transformshift{4.358072in}{3.699740in}%
\pgfsys@useobject{currentmarker}{}%
\end{pgfscope}%
\begin{pgfscope}%
\pgfsys@transformshift{4.375666in}{3.857109in}%
\pgfsys@useobject{currentmarker}{}%
\end{pgfscope}%
\begin{pgfscope}%
\pgfsys@transformshift{4.393260in}{3.728557in}%
\pgfsys@useobject{currentmarker}{}%
\end{pgfscope}%
\begin{pgfscope}%
\pgfsys@transformshift{4.410853in}{3.836957in}%
\pgfsys@useobject{currentmarker}{}%
\end{pgfscope}%
\begin{pgfscope}%
\pgfsys@transformshift{4.428447in}{3.830224in}%
\pgfsys@useobject{currentmarker}{}%
\end{pgfscope}%
\begin{pgfscope}%
\pgfsys@transformshift{4.446041in}{3.768236in}%
\pgfsys@useobject{currentmarker}{}%
\end{pgfscope}%
\begin{pgfscope}%
\pgfsys@transformshift{4.463635in}{4.058155in}%
\pgfsys@useobject{currentmarker}{}%
\end{pgfscope}%
\begin{pgfscope}%
\pgfsys@transformshift{4.481229in}{3.909891in}%
\pgfsys@useobject{currentmarker}{}%
\end{pgfscope}%
\begin{pgfscope}%
\pgfsys@transformshift{4.498823in}{3.643266in}%
\pgfsys@useobject{currentmarker}{}%
\end{pgfscope}%
\begin{pgfscope}%
\pgfsys@transformshift{4.516417in}{3.868210in}%
\pgfsys@useobject{currentmarker}{}%
\end{pgfscope}%
\begin{pgfscope}%
\pgfsys@transformshift{4.534010in}{3.781297in}%
\pgfsys@useobject{currentmarker}{}%
\end{pgfscope}%
\begin{pgfscope}%
\pgfsys@transformshift{4.551604in}{3.931967in}%
\pgfsys@useobject{currentmarker}{}%
\end{pgfscope}%
\begin{pgfscope}%
\pgfsys@transformshift{4.569198in}{3.760888in}%
\pgfsys@useobject{currentmarker}{}%
\end{pgfscope}%
\begin{pgfscope}%
\pgfsys@transformshift{4.586792in}{3.823580in}%
\pgfsys@useobject{currentmarker}{}%
\end{pgfscope}%
\begin{pgfscope}%
\pgfsys@transformshift{4.604386in}{3.878959in}%
\pgfsys@useobject{currentmarker}{}%
\end{pgfscope}%
\begin{pgfscope}%
\pgfsys@transformshift{4.621980in}{3.906818in}%
\pgfsys@useobject{currentmarker}{}%
\end{pgfscope}%
\begin{pgfscope}%
\pgfsys@transformshift{4.639573in}{3.687678in}%
\pgfsys@useobject{currentmarker}{}%
\end{pgfscope}%
\begin{pgfscope}%
\pgfsys@transformshift{4.657167in}{3.764627in}%
\pgfsys@useobject{currentmarker}{}%
\end{pgfscope}%
\begin{pgfscope}%
\pgfsys@transformshift{4.674761in}{3.738859in}%
\pgfsys@useobject{currentmarker}{}%
\end{pgfscope}%
\begin{pgfscope}%
\pgfsys@transformshift{4.692355in}{3.708667in}%
\pgfsys@useobject{currentmarker}{}%
\end{pgfscope}%
\begin{pgfscope}%
\pgfsys@transformshift{4.709949in}{3.941227in}%
\pgfsys@useobject{currentmarker}{}%
\end{pgfscope}%
\begin{pgfscope}%
\pgfsys@transformshift{4.727543in}{3.790709in}%
\pgfsys@useobject{currentmarker}{}%
\end{pgfscope}%
\begin{pgfscope}%
\pgfsys@transformshift{4.745137in}{3.609275in}%
\pgfsys@useobject{currentmarker}{}%
\end{pgfscope}%
\begin{pgfscope}%
\pgfsys@transformshift{4.762730in}{3.817591in}%
\pgfsys@useobject{currentmarker}{}%
\end{pgfscope}%
\begin{pgfscope}%
\pgfsys@transformshift{4.780324in}{3.927614in}%
\pgfsys@useobject{currentmarker}{}%
\end{pgfscope}%
\begin{pgfscope}%
\pgfsys@transformshift{4.797918in}{3.805855in}%
\pgfsys@useobject{currentmarker}{}%
\end{pgfscope}%
\begin{pgfscope}%
\pgfsys@transformshift{4.815512in}{3.536787in}%
\pgfsys@useobject{currentmarker}{}%
\end{pgfscope}%
\begin{pgfscope}%
\pgfsys@transformshift{4.833106in}{3.632153in}%
\pgfsys@useobject{currentmarker}{}%
\end{pgfscope}%
\end{pgfscope}%
\begin{pgfscope}%
\pgfsetbuttcap%
\pgfsetroundjoin%
\definecolor{currentfill}{rgb}{0.000000,0.000000,0.000000}%
\pgfsetfillcolor{currentfill}%
\pgfsetlinewidth{0.803000pt}%
\definecolor{currentstroke}{rgb}{0.000000,0.000000,0.000000}%
\pgfsetstrokecolor{currentstroke}%
\pgfsetdash{}{0pt}%
\pgfsys@defobject{currentmarker}{\pgfqpoint{0.000000in}{-0.048611in}}{\pgfqpoint{0.000000in}{0.000000in}}{%
\pgfpathmoveto{\pgfqpoint{0.000000in}{0.000000in}}%
\pgfpathlineto{\pgfqpoint{0.000000in}{-0.048611in}}%
\pgfusepath{stroke,fill}%
}%
\begin{pgfscope}%
\pgfsys@transformshift{0.456635in}{3.021449in}%
\pgfsys@useobject{currentmarker}{}%
\end{pgfscope}%
\end{pgfscope}%
\begin{pgfscope}%
\pgfsetbuttcap%
\pgfsetroundjoin%
\definecolor{currentfill}{rgb}{0.000000,0.000000,0.000000}%
\pgfsetfillcolor{currentfill}%
\pgfsetlinewidth{0.803000pt}%
\definecolor{currentstroke}{rgb}{0.000000,0.000000,0.000000}%
\pgfsetstrokecolor{currentstroke}%
\pgfsetdash{}{0pt}%
\pgfsys@defobject{currentmarker}{\pgfqpoint{0.000000in}{-0.048611in}}{\pgfqpoint{0.000000in}{0.000000in}}{%
\pgfpathmoveto{\pgfqpoint{0.000000in}{0.000000in}}%
\pgfpathlineto{\pgfqpoint{0.000000in}{-0.048611in}}%
\pgfusepath{stroke,fill}%
}%
\begin{pgfscope}%
\pgfsys@transformshift{1.331929in}{3.021449in}%
\pgfsys@useobject{currentmarker}{}%
\end{pgfscope}%
\end{pgfscope}%
\begin{pgfscope}%
\pgfsetbuttcap%
\pgfsetroundjoin%
\definecolor{currentfill}{rgb}{0.000000,0.000000,0.000000}%
\pgfsetfillcolor{currentfill}%
\pgfsetlinewidth{0.803000pt}%
\definecolor{currentstroke}{rgb}{0.000000,0.000000,0.000000}%
\pgfsetstrokecolor{currentstroke}%
\pgfsetdash{}{0pt}%
\pgfsys@defobject{currentmarker}{\pgfqpoint{0.000000in}{-0.048611in}}{\pgfqpoint{0.000000in}{0.000000in}}{%
\pgfpathmoveto{\pgfqpoint{0.000000in}{0.000000in}}%
\pgfpathlineto{\pgfqpoint{0.000000in}{-0.048611in}}%
\pgfusepath{stroke,fill}%
}%
\begin{pgfscope}%
\pgfsys@transformshift{2.207224in}{3.021449in}%
\pgfsys@useobject{currentmarker}{}%
\end{pgfscope}%
\end{pgfscope}%
\begin{pgfscope}%
\pgfsetbuttcap%
\pgfsetroundjoin%
\definecolor{currentfill}{rgb}{0.000000,0.000000,0.000000}%
\pgfsetfillcolor{currentfill}%
\pgfsetlinewidth{0.803000pt}%
\definecolor{currentstroke}{rgb}{0.000000,0.000000,0.000000}%
\pgfsetstrokecolor{currentstroke}%
\pgfsetdash{}{0pt}%
\pgfsys@defobject{currentmarker}{\pgfqpoint{0.000000in}{-0.048611in}}{\pgfqpoint{0.000000in}{0.000000in}}{%
\pgfpathmoveto{\pgfqpoint{0.000000in}{0.000000in}}%
\pgfpathlineto{\pgfqpoint{0.000000in}{-0.048611in}}%
\pgfusepath{stroke,fill}%
}%
\begin{pgfscope}%
\pgfsys@transformshift{3.082518in}{3.021449in}%
\pgfsys@useobject{currentmarker}{}%
\end{pgfscope}%
\end{pgfscope}%
\begin{pgfscope}%
\pgfsetbuttcap%
\pgfsetroundjoin%
\definecolor{currentfill}{rgb}{0.000000,0.000000,0.000000}%
\pgfsetfillcolor{currentfill}%
\pgfsetlinewidth{0.803000pt}%
\definecolor{currentstroke}{rgb}{0.000000,0.000000,0.000000}%
\pgfsetstrokecolor{currentstroke}%
\pgfsetdash{}{0pt}%
\pgfsys@defobject{currentmarker}{\pgfqpoint{0.000000in}{-0.048611in}}{\pgfqpoint{0.000000in}{0.000000in}}{%
\pgfpathmoveto{\pgfqpoint{0.000000in}{0.000000in}}%
\pgfpathlineto{\pgfqpoint{0.000000in}{-0.048611in}}%
\pgfusepath{stroke,fill}%
}%
\begin{pgfscope}%
\pgfsys@transformshift{3.957812in}{3.021449in}%
\pgfsys@useobject{currentmarker}{}%
\end{pgfscope}%
\end{pgfscope}%
\begin{pgfscope}%
\pgfsetbuttcap%
\pgfsetroundjoin%
\definecolor{currentfill}{rgb}{0.000000,0.000000,0.000000}%
\pgfsetfillcolor{currentfill}%
\pgfsetlinewidth{0.803000pt}%
\definecolor{currentstroke}{rgb}{0.000000,0.000000,0.000000}%
\pgfsetstrokecolor{currentstroke}%
\pgfsetdash{}{0pt}%
\pgfsys@defobject{currentmarker}{\pgfqpoint{0.000000in}{-0.048611in}}{\pgfqpoint{0.000000in}{0.000000in}}{%
\pgfpathmoveto{\pgfqpoint{0.000000in}{0.000000in}}%
\pgfpathlineto{\pgfqpoint{0.000000in}{-0.048611in}}%
\pgfusepath{stroke,fill}%
}%
\begin{pgfscope}%
\pgfsys@transformshift{4.833106in}{3.021449in}%
\pgfsys@useobject{currentmarker}{}%
\end{pgfscope}%
\end{pgfscope}%
\begin{pgfscope}%
\pgfsetbuttcap%
\pgfsetroundjoin%
\definecolor{currentfill}{rgb}{0.000000,0.000000,0.000000}%
\pgfsetfillcolor{currentfill}%
\pgfsetlinewidth{0.803000pt}%
\definecolor{currentstroke}{rgb}{0.000000,0.000000,0.000000}%
\pgfsetstrokecolor{currentstroke}%
\pgfsetdash{}{0pt}%
\pgfsys@defobject{currentmarker}{\pgfqpoint{-0.048611in}{0.000000in}}{\pgfqpoint{0.000000in}{0.000000in}}{%
\pgfpathmoveto{\pgfqpoint{0.000000in}{0.000000in}}%
\pgfpathlineto{\pgfqpoint{-0.048611in}{0.000000in}}%
\pgfusepath{stroke,fill}%
}%
\begin{pgfscope}%
\pgfsys@transformshift{0.456635in}{3.386186in}%
\pgfsys@useobject{currentmarker}{}%
\end{pgfscope}%
\end{pgfscope}%
\begin{pgfscope}%
\pgftext[x=0.289968in,y=3.333425in,left,base]{\rmfamily\fontsize{10.000000}{12.000000}\selectfont \(\displaystyle 0\)}%
\end{pgfscope}%
\begin{pgfscope}%
\pgfsetbuttcap%
\pgfsetroundjoin%
\definecolor{currentfill}{rgb}{0.000000,0.000000,0.000000}%
\pgfsetfillcolor{currentfill}%
\pgfsetlinewidth{0.803000pt}%
\definecolor{currentstroke}{rgb}{0.000000,0.000000,0.000000}%
\pgfsetstrokecolor{currentstroke}%
\pgfsetdash{}{0pt}%
\pgfsys@defobject{currentmarker}{\pgfqpoint{-0.048611in}{0.000000in}}{\pgfqpoint{0.000000in}{0.000000in}}{%
\pgfpathmoveto{\pgfqpoint{0.000000in}{0.000000in}}%
\pgfpathlineto{\pgfqpoint{-0.048611in}{0.000000in}}%
\pgfusepath{stroke,fill}%
}%
\begin{pgfscope}%
\pgfsys@transformshift{0.456635in}{3.791449in}%
\pgfsys@useobject{currentmarker}{}%
\end{pgfscope}%
\end{pgfscope}%
\begin{pgfscope}%
\pgftext[x=0.289968in,y=3.738688in,left,base]{\rmfamily\fontsize{10.000000}{12.000000}\selectfont \(\displaystyle 2\)}%
\end{pgfscope}%
\begin{pgfscope}%
\pgftext[x=0.234413in,y=3.507765in,,bottom,rotate=90.000000]{\rmfamily\fontsize{10.000000}{12.000000}\selectfont y}%
\end{pgfscope}%
\begin{pgfscope}%
\pgfpathrectangle{\pgfqpoint{0.456635in}{3.021449in}}{\pgfqpoint{4.376471in}{0.972632in}}%
\pgfusepath{clip}%
\pgfsetrectcap%
\pgfsetroundjoin%
\pgfsetlinewidth{1.505625pt}%
\definecolor{currentstroke}{rgb}{0.121569,0.466667,0.705882}%
\pgfsetstrokecolor{currentstroke}%
\pgfsetdash{}{0pt}%
\pgfpathmoveto{\pgfqpoint{1.331929in}{3.427999in}}%
\pgfpathlineto{\pgfqpoint{1.349523in}{3.587309in}}%
\pgfpathlineto{\pgfqpoint{1.367117in}{3.701521in}}%
\pgfpathlineto{\pgfqpoint{1.384711in}{3.776362in}}%
\pgfpathlineto{\pgfqpoint{1.402305in}{3.817764in}}%
\pgfpathlineto{\pgfqpoint{1.419899in}{3.831688in}}%
\pgfpathlineto{\pgfqpoint{1.437493in}{3.823956in}}%
\pgfpathlineto{\pgfqpoint{1.455086in}{3.800095in}}%
\pgfpathlineto{\pgfqpoint{1.472680in}{3.765203in}}%
\pgfpathlineto{\pgfqpoint{1.507868in}{3.679935in}}%
\pgfpathlineto{\pgfqpoint{1.525462in}{3.636740in}}%
\pgfpathlineto{\pgfqpoint{1.543056in}{3.596795in}}%
\pgfpathlineto{\pgfqpoint{1.560649in}{3.561939in}}%
\pgfpathlineto{\pgfqpoint{1.578243in}{3.533335in}}%
\pgfpathlineto{\pgfqpoint{1.595837in}{3.511530in}}%
\pgfpathlineto{\pgfqpoint{1.613431in}{3.496530in}}%
\pgfpathlineto{\pgfqpoint{1.631025in}{3.487893in}}%
\pgfpathlineto{\pgfqpoint{1.648619in}{3.484834in}}%
\pgfpathlineto{\pgfqpoint{1.666213in}{3.486330in}}%
\pgfpathlineto{\pgfqpoint{1.683806in}{3.491229in}}%
\pgfpathlineto{\pgfqpoint{1.718994in}{3.506562in}}%
\pgfpathlineto{\pgfqpoint{1.754182in}{3.522514in}}%
\pgfpathlineto{\pgfqpoint{1.771776in}{3.528878in}}%
\pgfpathlineto{\pgfqpoint{1.789370in}{3.533647in}}%
\pgfpathlineto{\pgfqpoint{1.806963in}{3.536732in}}%
\pgfpathlineto{\pgfqpoint{1.824557in}{3.538265in}}%
\pgfpathlineto{\pgfqpoint{1.859745in}{3.538084in}}%
\pgfpathlineto{\pgfqpoint{1.894933in}{3.537018in}}%
\pgfpathlineto{\pgfqpoint{1.912526in}{3.537565in}}%
\pgfpathlineto{\pgfqpoint{1.930120in}{3.539504in}}%
\pgfpathlineto{\pgfqpoint{1.947714in}{3.543209in}}%
\pgfpathlineto{\pgfqpoint{1.965308in}{3.548913in}}%
\pgfpathlineto{\pgfqpoint{1.982902in}{3.556687in}}%
\pgfpathlineto{\pgfqpoint{2.000496in}{3.566442in}}%
\pgfpathlineto{\pgfqpoint{2.018090in}{3.577931in}}%
\pgfpathlineto{\pgfqpoint{2.053277in}{3.604468in}}%
\pgfpathlineto{\pgfqpoint{2.088465in}{3.632171in}}%
\pgfpathlineto{\pgfqpoint{2.106059in}{3.645000in}}%
\pgfpathlineto{\pgfqpoint{2.123653in}{3.656416in}}%
\pgfpathlineto{\pgfqpoint{2.141247in}{3.665963in}}%
\pgfpathlineto{\pgfqpoint{2.158840in}{3.673289in}}%
\pgfpathlineto{\pgfqpoint{2.176434in}{3.678161in}}%
\pgfpathlineto{\pgfqpoint{2.194028in}{3.680473in}}%
\pgfpathlineto{\pgfqpoint{2.211622in}{3.680237in}}%
\pgfpathlineto{\pgfqpoint{2.229216in}{3.677572in}}%
\pgfpathlineto{\pgfqpoint{2.246810in}{3.672681in}}%
\pgfpathlineto{\pgfqpoint{2.264404in}{3.665817in}}%
\pgfpathlineto{\pgfqpoint{2.281997in}{3.657258in}}%
\pgfpathlineto{\pgfqpoint{2.317185in}{3.636071in}}%
\pgfpathlineto{\pgfqpoint{2.352373in}{3.610581in}}%
\pgfpathlineto{\pgfqpoint{2.387560in}{3.580922in}}%
\pgfpathlineto{\pgfqpoint{2.405154in}{3.564160in}}%
\pgfpathlineto{\pgfqpoint{2.422748in}{3.545758in}}%
\pgfpathlineto{\pgfqpoint{2.440342in}{3.525396in}}%
\pgfpathlineto{\pgfqpoint{2.457936in}{3.502753in}}%
\pgfpathlineto{\pgfqpoint{2.475530in}{3.477542in}}%
\pgfpathlineto{\pgfqpoint{2.493124in}{3.449555in}}%
\pgfpathlineto{\pgfqpoint{2.510717in}{3.418692in}}%
\pgfpathlineto{\pgfqpoint{2.528311in}{3.385003in}}%
\pgfpathlineto{\pgfqpoint{2.563499in}{3.310201in}}%
\pgfpathlineto{\pgfqpoint{2.651468in}{3.110691in}}%
\pgfpathlineto{\pgfqpoint{2.669062in}{3.076356in}}%
\pgfpathlineto{\pgfqpoint{2.686656in}{3.046416in}}%
\pgfpathlineto{\pgfqpoint{2.704250in}{3.021856in}}%
\pgfpathlineto{\pgfqpoint{2.714232in}{3.011449in}}%
\pgfpathmoveto{\pgfqpoint{2.802541in}{3.011449in}}%
\pgfpathlineto{\pgfqpoint{2.809813in}{3.018568in}}%
\pgfpathlineto{\pgfqpoint{2.827407in}{3.041943in}}%
\pgfpathlineto{\pgfqpoint{2.845001in}{3.070571in}}%
\pgfpathlineto{\pgfqpoint{2.862594in}{3.103379in}}%
\pgfpathlineto{\pgfqpoint{2.897782in}{3.176692in}}%
\pgfpathlineto{\pgfqpoint{2.932970in}{3.252008in}}%
\pgfpathlineto{\pgfqpoint{2.950564in}{3.287585in}}%
\pgfpathlineto{\pgfqpoint{2.968158in}{3.320568in}}%
\pgfpathlineto{\pgfqpoint{2.985751in}{3.350320in}}%
\pgfpathlineto{\pgfqpoint{3.003345in}{3.376452in}}%
\pgfpathlineto{\pgfqpoint{3.020939in}{3.398831in}}%
\pgfpathlineto{\pgfqpoint{3.038533in}{3.417576in}}%
\pgfpathlineto{\pgfqpoint{3.056127in}{3.433034in}}%
\pgfpathlineto{\pgfqpoint{3.073721in}{3.445741in}}%
\pgfpathlineto{\pgfqpoint{3.091315in}{3.456377in}}%
\pgfpathlineto{\pgfqpoint{3.161690in}{3.493446in}}%
\pgfpathlineto{\pgfqpoint{3.179284in}{3.504715in}}%
\pgfpathlineto{\pgfqpoint{3.196878in}{3.517649in}}%
\pgfpathlineto{\pgfqpoint{3.214472in}{3.532326in}}%
\pgfpathlineto{\pgfqpoint{3.249659in}{3.566042in}}%
\pgfpathlineto{\pgfqpoint{3.302441in}{3.618671in}}%
\pgfpathlineto{\pgfqpoint{3.320035in}{3.633299in}}%
\pgfpathlineto{\pgfqpoint{3.337628in}{3.644810in}}%
\pgfpathlineto{\pgfqpoint{3.355222in}{3.652243in}}%
\pgfpathlineto{\pgfqpoint{3.372816in}{3.654752in}}%
\pgfpathlineto{\pgfqpoint{3.390410in}{3.651656in}}%
\pgfpathlineto{\pgfqpoint{3.408004in}{3.642495in}}%
\pgfpathlineto{\pgfqpoint{3.425598in}{3.627055in}}%
\pgfpathlineto{\pgfqpoint{3.443192in}{3.605399in}}%
\pgfpathlineto{\pgfqpoint{3.460785in}{3.577861in}}%
\pgfpathlineto{\pgfqpoint{3.478379in}{3.545044in}}%
\pgfpathlineto{\pgfqpoint{3.495973in}{3.507786in}}%
\pgfpathlineto{\pgfqpoint{3.531161in}{3.424238in}}%
\pgfpathlineto{\pgfqpoint{3.583942in}{3.295016in}}%
\pgfpathlineto{\pgfqpoint{3.601536in}{3.255880in}}%
\pgfpathlineto{\pgfqpoint{3.619130in}{3.220507in}}%
\pgfpathlineto{\pgfqpoint{3.636724in}{3.189711in}}%
\pgfpathlineto{\pgfqpoint{3.654318in}{3.164080in}}%
\pgfpathlineto{\pgfqpoint{3.671912in}{3.143963in}}%
\pgfpathlineto{\pgfqpoint{3.689506in}{3.129464in}}%
\pgfpathlineto{\pgfqpoint{3.707099in}{3.120460in}}%
\pgfpathlineto{\pgfqpoint{3.724693in}{3.116616in}}%
\pgfpathlineto{\pgfqpoint{3.742287in}{3.117429in}}%
\pgfpathlineto{\pgfqpoint{3.759881in}{3.122264in}}%
\pgfpathlineto{\pgfqpoint{3.777475in}{3.130408in}}%
\pgfpathlineto{\pgfqpoint{3.795069in}{3.141110in}}%
\pgfpathlineto{\pgfqpoint{3.830256in}{3.167311in}}%
\pgfpathlineto{\pgfqpoint{3.900632in}{3.223584in}}%
\pgfpathlineto{\pgfqpoint{3.953413in}{3.261630in}}%
\pgfpathlineto{\pgfqpoint{4.023789in}{3.311559in}}%
\pgfpathlineto{\pgfqpoint{4.058976in}{3.339668in}}%
\pgfpathlineto{\pgfqpoint{4.094164in}{3.370711in}}%
\pgfpathlineto{\pgfqpoint{4.146946in}{3.421126in}}%
\pgfpathlineto{\pgfqpoint{4.287696in}{3.559336in}}%
\pgfpathlineto{\pgfqpoint{4.305290in}{3.579163in}}%
\pgfpathlineto{\pgfqpoint{4.322884in}{3.600761in}}%
\pgfpathlineto{\pgfqpoint{4.340478in}{3.624539in}}%
\pgfpathlineto{\pgfqpoint{4.358072in}{3.650815in}}%
\pgfpathlineto{\pgfqpoint{4.375666in}{3.679774in}}%
\pgfpathlineto{\pgfqpoint{4.393260in}{3.711411in}}%
\pgfpathlineto{\pgfqpoint{4.428447in}{3.781543in}}%
\pgfpathlineto{\pgfqpoint{4.481229in}{3.892423in}}%
\pgfpathlineto{\pgfqpoint{4.498823in}{3.926013in}}%
\pgfpathlineto{\pgfqpoint{4.516417in}{3.955268in}}%
\pgfpathlineto{\pgfqpoint{4.534010in}{3.978432in}}%
\pgfpathlineto{\pgfqpoint{4.551604in}{3.993756in}}%
\pgfpathlineto{\pgfqpoint{4.569198in}{3.999608in}}%
\pgfpathlineto{\pgfqpoint{4.586792in}{3.994575in}}%
\pgfpathlineto{\pgfqpoint{4.604386in}{3.977585in}}%
\pgfpathlineto{\pgfqpoint{4.621980in}{3.948012in}}%
\pgfpathlineto{\pgfqpoint{4.639573in}{3.905772in}}%
\pgfpathlineto{\pgfqpoint{4.657167in}{3.851414in}}%
\pgfpathlineto{\pgfqpoint{4.674761in}{3.786176in}}%
\pgfpathlineto{\pgfqpoint{4.692355in}{3.712029in}}%
\pgfpathlineto{\pgfqpoint{4.762730in}{3.391057in}}%
\pgfpathlineto{\pgfqpoint{4.780324in}{3.326524in}}%
\pgfpathlineto{\pgfqpoint{4.797918in}{3.278724in}}%
\pgfpathlineto{\pgfqpoint{4.815512in}{3.253377in}}%
\pgfpathlineto{\pgfqpoint{4.833106in}{3.256229in}}%
\pgfpathlineto{\pgfqpoint{4.833106in}{3.256229in}}%
\pgfusepath{stroke}%
\end{pgfscope}%
\begin{pgfscope}%
\pgfpathrectangle{\pgfqpoint{0.456635in}{3.021449in}}{\pgfqpoint{4.376471in}{0.972632in}}%
\pgfusepath{clip}%
\pgfsetbuttcap%
\pgfsetroundjoin%
\pgfsetlinewidth{1.505625pt}%
\definecolor{currentstroke}{rgb}{1.000000,0.498039,0.054902}%
\pgfsetstrokecolor{currentstroke}%
\pgfsetdash{{9.600000pt}{2.400000pt}{1.500000pt}{2.400000pt}}{0.000000pt}%
\pgfpathmoveto{\pgfqpoint{1.331929in}{3.427999in}}%
\pgfpathlineto{\pgfqpoint{1.349523in}{3.587309in}}%
\pgfpathlineto{\pgfqpoint{1.367117in}{3.701521in}}%
\pgfpathlineto{\pgfqpoint{1.384711in}{3.776362in}}%
\pgfpathlineto{\pgfqpoint{1.402305in}{3.817764in}}%
\pgfpathlineto{\pgfqpoint{1.419899in}{3.831688in}}%
\pgfpathlineto{\pgfqpoint{1.437493in}{3.823956in}}%
\pgfpathlineto{\pgfqpoint{1.455086in}{3.800095in}}%
\pgfpathlineto{\pgfqpoint{1.472680in}{3.765203in}}%
\pgfpathlineto{\pgfqpoint{1.507868in}{3.679935in}}%
\pgfpathlineto{\pgfqpoint{1.525462in}{3.636740in}}%
\pgfpathlineto{\pgfqpoint{1.543056in}{3.596795in}}%
\pgfpathlineto{\pgfqpoint{1.560649in}{3.561939in}}%
\pgfpathlineto{\pgfqpoint{1.578243in}{3.533335in}}%
\pgfpathlineto{\pgfqpoint{1.595837in}{3.511530in}}%
\pgfpathlineto{\pgfqpoint{1.613431in}{3.496530in}}%
\pgfpathlineto{\pgfqpoint{1.631025in}{3.487893in}}%
\pgfpathlineto{\pgfqpoint{1.648619in}{3.484834in}}%
\pgfpathlineto{\pgfqpoint{1.666213in}{3.486330in}}%
\pgfpathlineto{\pgfqpoint{1.683806in}{3.491229in}}%
\pgfpathlineto{\pgfqpoint{1.718994in}{3.506562in}}%
\pgfpathlineto{\pgfqpoint{1.754182in}{3.522514in}}%
\pgfpathlineto{\pgfqpoint{1.771776in}{3.528878in}}%
\pgfpathlineto{\pgfqpoint{1.789370in}{3.533647in}}%
\pgfpathlineto{\pgfqpoint{1.806963in}{3.536732in}}%
\pgfpathlineto{\pgfqpoint{1.824557in}{3.538265in}}%
\pgfpathlineto{\pgfqpoint{1.859745in}{3.538084in}}%
\pgfpathlineto{\pgfqpoint{1.894933in}{3.537018in}}%
\pgfpathlineto{\pgfqpoint{1.912526in}{3.537565in}}%
\pgfpathlineto{\pgfqpoint{1.930120in}{3.539504in}}%
\pgfpathlineto{\pgfqpoint{1.947714in}{3.543209in}}%
\pgfpathlineto{\pgfqpoint{1.965308in}{3.548913in}}%
\pgfpathlineto{\pgfqpoint{1.982902in}{3.556687in}}%
\pgfpathlineto{\pgfqpoint{2.000496in}{3.566442in}}%
\pgfpathlineto{\pgfqpoint{2.018090in}{3.577931in}}%
\pgfpathlineto{\pgfqpoint{2.053277in}{3.604468in}}%
\pgfpathlineto{\pgfqpoint{2.088465in}{3.632171in}}%
\pgfpathlineto{\pgfqpoint{2.106059in}{3.645000in}}%
\pgfpathlineto{\pgfqpoint{2.123653in}{3.656416in}}%
\pgfpathlineto{\pgfqpoint{2.141247in}{3.665963in}}%
\pgfpathlineto{\pgfqpoint{2.158840in}{3.673289in}}%
\pgfpathlineto{\pgfqpoint{2.176434in}{3.678161in}}%
\pgfpathlineto{\pgfqpoint{2.194028in}{3.680473in}}%
\pgfpathlineto{\pgfqpoint{2.211622in}{3.680237in}}%
\pgfpathlineto{\pgfqpoint{2.229216in}{3.677572in}}%
\pgfpathlineto{\pgfqpoint{2.246810in}{3.672681in}}%
\pgfpathlineto{\pgfqpoint{2.264404in}{3.665817in}}%
\pgfpathlineto{\pgfqpoint{2.281997in}{3.657258in}}%
\pgfpathlineto{\pgfqpoint{2.317185in}{3.636071in}}%
\pgfpathlineto{\pgfqpoint{2.352373in}{3.610581in}}%
\pgfpathlineto{\pgfqpoint{2.387560in}{3.580922in}}%
\pgfpathlineto{\pgfqpoint{2.405154in}{3.564160in}}%
\pgfpathlineto{\pgfqpoint{2.422748in}{3.545758in}}%
\pgfpathlineto{\pgfqpoint{2.440342in}{3.525396in}}%
\pgfpathlineto{\pgfqpoint{2.457936in}{3.502753in}}%
\pgfpathlineto{\pgfqpoint{2.475530in}{3.477542in}}%
\pgfpathlineto{\pgfqpoint{2.493124in}{3.449555in}}%
\pgfpathlineto{\pgfqpoint{2.510717in}{3.418692in}}%
\pgfpathlineto{\pgfqpoint{2.528311in}{3.385003in}}%
\pgfpathlineto{\pgfqpoint{2.563499in}{3.310201in}}%
\pgfpathlineto{\pgfqpoint{2.651468in}{3.110691in}}%
\pgfpathlineto{\pgfqpoint{2.669062in}{3.076356in}}%
\pgfpathlineto{\pgfqpoint{2.686656in}{3.046416in}}%
\pgfpathlineto{\pgfqpoint{2.704250in}{3.021856in}}%
\pgfpathlineto{\pgfqpoint{2.714232in}{3.011449in}}%
\pgfpathmoveto{\pgfqpoint{2.802541in}{3.011449in}}%
\pgfpathlineto{\pgfqpoint{2.809813in}{3.018568in}}%
\pgfpathlineto{\pgfqpoint{2.827407in}{3.041943in}}%
\pgfpathlineto{\pgfqpoint{2.845001in}{3.070571in}}%
\pgfpathlineto{\pgfqpoint{2.862594in}{3.103379in}}%
\pgfpathlineto{\pgfqpoint{2.897782in}{3.176692in}}%
\pgfpathlineto{\pgfqpoint{2.932970in}{3.252008in}}%
\pgfpathlineto{\pgfqpoint{2.950564in}{3.287585in}}%
\pgfpathlineto{\pgfqpoint{2.968158in}{3.320568in}}%
\pgfpathlineto{\pgfqpoint{2.985751in}{3.350320in}}%
\pgfpathlineto{\pgfqpoint{3.003345in}{3.376452in}}%
\pgfpathlineto{\pgfqpoint{3.020939in}{3.398831in}}%
\pgfpathlineto{\pgfqpoint{3.038533in}{3.417576in}}%
\pgfpathlineto{\pgfqpoint{3.056127in}{3.433034in}}%
\pgfpathlineto{\pgfqpoint{3.073721in}{3.445741in}}%
\pgfpathlineto{\pgfqpoint{3.091315in}{3.456377in}}%
\pgfpathlineto{\pgfqpoint{3.161690in}{3.493446in}}%
\pgfpathlineto{\pgfqpoint{3.179284in}{3.504715in}}%
\pgfpathlineto{\pgfqpoint{3.196878in}{3.517649in}}%
\pgfpathlineto{\pgfqpoint{3.214472in}{3.532326in}}%
\pgfpathlineto{\pgfqpoint{3.249659in}{3.566042in}}%
\pgfpathlineto{\pgfqpoint{3.302441in}{3.618671in}}%
\pgfpathlineto{\pgfqpoint{3.320035in}{3.633299in}}%
\pgfpathlineto{\pgfqpoint{3.337628in}{3.644810in}}%
\pgfpathlineto{\pgfqpoint{3.355222in}{3.652243in}}%
\pgfpathlineto{\pgfqpoint{3.372816in}{3.654752in}}%
\pgfpathlineto{\pgfqpoint{3.390410in}{3.651656in}}%
\pgfpathlineto{\pgfqpoint{3.408004in}{3.642495in}}%
\pgfpathlineto{\pgfqpoint{3.425598in}{3.627055in}}%
\pgfpathlineto{\pgfqpoint{3.443192in}{3.605399in}}%
\pgfpathlineto{\pgfqpoint{3.460785in}{3.577861in}}%
\pgfpathlineto{\pgfqpoint{3.478379in}{3.545044in}}%
\pgfpathlineto{\pgfqpoint{3.495973in}{3.507786in}}%
\pgfpathlineto{\pgfqpoint{3.531161in}{3.424238in}}%
\pgfpathlineto{\pgfqpoint{3.583942in}{3.295016in}}%
\pgfpathlineto{\pgfqpoint{3.601536in}{3.255880in}}%
\pgfpathlineto{\pgfqpoint{3.619130in}{3.220507in}}%
\pgfpathlineto{\pgfqpoint{3.636724in}{3.189711in}}%
\pgfpathlineto{\pgfqpoint{3.654318in}{3.164080in}}%
\pgfpathlineto{\pgfqpoint{3.671912in}{3.143963in}}%
\pgfpathlineto{\pgfqpoint{3.689506in}{3.129464in}}%
\pgfpathlineto{\pgfqpoint{3.707099in}{3.120460in}}%
\pgfpathlineto{\pgfqpoint{3.724693in}{3.116616in}}%
\pgfpathlineto{\pgfqpoint{3.742287in}{3.117429in}}%
\pgfpathlineto{\pgfqpoint{3.759881in}{3.122264in}}%
\pgfpathlineto{\pgfqpoint{3.777475in}{3.130408in}}%
\pgfpathlineto{\pgfqpoint{3.795069in}{3.141110in}}%
\pgfpathlineto{\pgfqpoint{3.830256in}{3.167311in}}%
\pgfpathlineto{\pgfqpoint{3.900632in}{3.223584in}}%
\pgfpathlineto{\pgfqpoint{3.953413in}{3.261630in}}%
\pgfpathlineto{\pgfqpoint{4.023789in}{3.311559in}}%
\pgfpathlineto{\pgfqpoint{4.058976in}{3.339668in}}%
\pgfpathlineto{\pgfqpoint{4.094164in}{3.370711in}}%
\pgfpathlineto{\pgfqpoint{4.146946in}{3.421126in}}%
\pgfpathlineto{\pgfqpoint{4.287696in}{3.559336in}}%
\pgfpathlineto{\pgfqpoint{4.305290in}{3.579163in}}%
\pgfpathlineto{\pgfqpoint{4.322884in}{3.600761in}}%
\pgfpathlineto{\pgfqpoint{4.340478in}{3.624539in}}%
\pgfpathlineto{\pgfqpoint{4.358072in}{3.650815in}}%
\pgfpathlineto{\pgfqpoint{4.375666in}{3.679774in}}%
\pgfpathlineto{\pgfqpoint{4.393260in}{3.711411in}}%
\pgfpathlineto{\pgfqpoint{4.428447in}{3.781543in}}%
\pgfpathlineto{\pgfqpoint{4.481229in}{3.892423in}}%
\pgfpathlineto{\pgfqpoint{4.498823in}{3.926013in}}%
\pgfpathlineto{\pgfqpoint{4.516417in}{3.955268in}}%
\pgfpathlineto{\pgfqpoint{4.534010in}{3.978432in}}%
\pgfpathlineto{\pgfqpoint{4.551604in}{3.993756in}}%
\pgfpathlineto{\pgfqpoint{4.569198in}{3.999608in}}%
\pgfpathlineto{\pgfqpoint{4.586792in}{3.994575in}}%
\pgfpathlineto{\pgfqpoint{4.604386in}{3.977585in}}%
\pgfpathlineto{\pgfqpoint{4.621980in}{3.948012in}}%
\pgfpathlineto{\pgfqpoint{4.639573in}{3.905772in}}%
\pgfpathlineto{\pgfqpoint{4.657167in}{3.851414in}}%
\pgfpathlineto{\pgfqpoint{4.674761in}{3.786176in}}%
\pgfpathlineto{\pgfqpoint{4.692355in}{3.712029in}}%
\pgfpathlineto{\pgfqpoint{4.762730in}{3.391057in}}%
\pgfpathlineto{\pgfqpoint{4.780324in}{3.326524in}}%
\pgfpathlineto{\pgfqpoint{4.797918in}{3.278724in}}%
\pgfpathlineto{\pgfqpoint{4.815512in}{3.253377in}}%
\pgfpathlineto{\pgfqpoint{4.833106in}{3.256229in}}%
\pgfpathlineto{\pgfqpoint{4.833106in}{3.256229in}}%
\pgfusepath{stroke}%
\end{pgfscope}%
\begin{pgfscope}%
\pgfsetrectcap%
\pgfsetmiterjoin%
\pgfsetlinewidth{0.803000pt}%
\definecolor{currentstroke}{rgb}{0.000000,0.000000,0.000000}%
\pgfsetstrokecolor{currentstroke}%
\pgfsetdash{}{0pt}%
\pgfpathmoveto{\pgfqpoint{0.456635in}{3.021449in}}%
\pgfpathlineto{\pgfqpoint{0.456635in}{3.994081in}}%
\pgfusepath{stroke}%
\end{pgfscope}%
\begin{pgfscope}%
\pgfsetrectcap%
\pgfsetmiterjoin%
\pgfsetlinewidth{0.803000pt}%
\definecolor{currentstroke}{rgb}{0.000000,0.000000,0.000000}%
\pgfsetstrokecolor{currentstroke}%
\pgfsetdash{}{0pt}%
\pgfpathmoveto{\pgfqpoint{4.833106in}{3.021449in}}%
\pgfpathlineto{\pgfqpoint{4.833106in}{3.994081in}}%
\pgfusepath{stroke}%
\end{pgfscope}%
\begin{pgfscope}%
\pgfsetrectcap%
\pgfsetmiterjoin%
\pgfsetlinewidth{0.803000pt}%
\definecolor{currentstroke}{rgb}{0.000000,0.000000,0.000000}%
\pgfsetstrokecolor{currentstroke}%
\pgfsetdash{}{0pt}%
\pgfpathmoveto{\pgfqpoint{0.456635in}{3.021449in}}%
\pgfpathlineto{\pgfqpoint{4.833106in}{3.021449in}}%
\pgfusepath{stroke}%
\end{pgfscope}%
\begin{pgfscope}%
\pgfsetrectcap%
\pgfsetmiterjoin%
\pgfsetlinewidth{0.803000pt}%
\definecolor{currentstroke}{rgb}{0.000000,0.000000,0.000000}%
\pgfsetstrokecolor{currentstroke}%
\pgfsetdash{}{0pt}%
\pgfpathmoveto{\pgfqpoint{0.456635in}{3.994081in}}%
\pgfpathlineto{\pgfqpoint{4.833106in}{3.994081in}}%
\pgfusepath{stroke}%
\end{pgfscope}%
\begin{pgfscope}%
\pgfsetbuttcap%
\pgfsetmiterjoin%
\definecolor{currentfill}{rgb}{1.000000,1.000000,1.000000}%
\pgfsetfillcolor{currentfill}%
\pgfsetfillopacity{0.800000}%
\pgfsetlinewidth{1.003750pt}%
\definecolor{currentstroke}{rgb}{0.800000,0.800000,0.800000}%
\pgfsetstrokecolor{currentstroke}%
\pgfsetstrokeopacity{0.800000}%
\pgfsetdash{}{0pt}%
\pgfpathmoveto{\pgfqpoint{0.553858in}{3.090894in}}%
\pgfpathlineto{\pgfqpoint{1.337183in}{3.090894in}}%
\pgfpathquadraticcurveto{\pgfqpoint{1.364960in}{3.090894in}}{\pgfqpoint{1.364960in}{3.118671in}}%
\pgfpathlineto{\pgfqpoint{1.364960in}{3.922219in}}%
\pgfpathquadraticcurveto{\pgfqpoint{1.364960in}{3.949997in}}{\pgfqpoint{1.337183in}{3.949997in}}%
\pgfpathlineto{\pgfqpoint{0.553858in}{3.949997in}}%
\pgfpathquadraticcurveto{\pgfqpoint{0.526080in}{3.949997in}}{\pgfqpoint{0.526080in}{3.922219in}}%
\pgfpathlineto{\pgfqpoint{0.526080in}{3.118671in}}%
\pgfpathquadraticcurveto{\pgfqpoint{0.526080in}{3.090894in}}{\pgfqpoint{0.553858in}{3.090894in}}%
\pgfpathclose%
\pgfusepath{stroke,fill}%
\end{pgfscope}%
\begin{pgfscope}%
\pgfsetrectcap%
\pgfsetroundjoin%
\pgfsetlinewidth{1.505625pt}%
\definecolor{currentstroke}{rgb}{0.121569,0.466667,0.705882}%
\pgfsetstrokecolor{currentstroke}%
\pgfsetdash{}{0pt}%
\pgfpathmoveto{\pgfqpoint{0.581635in}{3.836526in}}%
\pgfpathlineto{\pgfqpoint{0.859413in}{3.836526in}}%
\pgfusepath{stroke}%
\end{pgfscope}%
\begin{pgfscope}%
\pgftext[x=0.970524in,y=3.787915in,left,base]{\rmfamily\fontsize{10.000000}{12.000000}\selectfont \(\displaystyle \widetilde{\Phi}^* \theta\)}%
\end{pgfscope}%
\begin{pgfscope}%
\pgfsetbuttcap%
\pgfsetroundjoin%
\pgfsetlinewidth{1.505625pt}%
\definecolor{currentstroke}{rgb}{1.000000,0.498039,0.054902}%
\pgfsetstrokecolor{currentstroke}%
\pgfsetdash{{9.600000pt}{2.400000pt}{1.500000pt}{2.400000pt}}{0.000000pt}%
\pgfpathmoveto{\pgfqpoint{0.581635in}{3.631665in}}%
\pgfpathlineto{\pgfqpoint{0.859413in}{3.631665in}}%
\pgfusepath{stroke}%
\end{pgfscope}%
\begin{pgfscope}%
\pgftext[x=0.970524in,y=3.583053in,left,base]{\rmfamily\fontsize{10.000000}{12.000000}\selectfont \(\displaystyle \widetilde{K}u\)}%
\end{pgfscope}%
\begin{pgfscope}%
\pgfsetbuttcap%
\pgfsetroundjoin%
\definecolor{currentfill}{rgb}{1.000000,0.000000,0.000000}%
\pgfsetfillcolor{currentfill}%
\pgfsetlinewidth{2.007500pt}%
\definecolor{currentstroke}{rgb}{1.000000,0.000000,0.000000}%
\pgfsetstrokecolor{currentstroke}%
\pgfsetdash{}{0pt}%
\pgfpathmoveto{\pgfqpoint{0.678857in}{3.415655in}}%
\pgfpathlineto{\pgfqpoint{0.762191in}{3.415655in}}%
\pgfpathmoveto{\pgfqpoint{0.720524in}{3.373988in}}%
\pgfpathlineto{\pgfqpoint{0.720524in}{3.457321in}}%
\pgfusepath{stroke,fill}%
\end{pgfscope}%
\begin{pgfscope}%
\pgftext[x=0.970524in,y=3.379196in,left,base]{\rmfamily\fontsize{10.000000}{12.000000}\selectfont train}%
\end{pgfscope}%
\begin{pgfscope}%
\pgfsetbuttcap%
\pgfsetroundjoin%
\definecolor{currentfill}{rgb}{0.000000,0.000000,0.000000}%
\pgfsetfillcolor{currentfill}%
\pgfsetlinewidth{1.003750pt}%
\definecolor{currentstroke}{rgb}{0.000000,0.000000,0.000000}%
\pgfsetstrokecolor{currentstroke}%
\pgfsetdash{}{0pt}%
\pgfsys@defobject{currentmarker}{\pgfqpoint{-0.020833in}{-0.020833in}}{\pgfqpoint{0.020833in}{0.020833in}}{%
\pgfpathmoveto{\pgfqpoint{0.000000in}{-0.020833in}}%
\pgfpathcurveto{\pgfqpoint{0.005525in}{-0.020833in}}{\pgfqpoint{0.010825in}{-0.018638in}}{\pgfqpoint{0.014731in}{-0.014731in}}%
\pgfpathcurveto{\pgfqpoint{0.018638in}{-0.010825in}}{\pgfqpoint{0.020833in}{-0.005525in}}{\pgfqpoint{0.020833in}{0.000000in}}%
\pgfpathcurveto{\pgfqpoint{0.020833in}{0.005525in}}{\pgfqpoint{0.018638in}{0.010825in}}{\pgfqpoint{0.014731in}{0.014731in}}%
\pgfpathcurveto{\pgfqpoint{0.010825in}{0.018638in}}{\pgfqpoint{0.005525in}{0.020833in}}{\pgfqpoint{0.000000in}{0.020833in}}%
\pgfpathcurveto{\pgfqpoint{-0.005525in}{0.020833in}}{\pgfqpoint{-0.010825in}{0.018638in}}{\pgfqpoint{-0.014731in}{0.014731in}}%
\pgfpathcurveto{\pgfqpoint{-0.018638in}{0.010825in}}{\pgfqpoint{-0.020833in}{0.005525in}}{\pgfqpoint{-0.020833in}{0.000000in}}%
\pgfpathcurveto{\pgfqpoint{-0.020833in}{-0.005525in}}{\pgfqpoint{-0.018638in}{-0.010825in}}{\pgfqpoint{-0.014731in}{-0.014731in}}%
\pgfpathcurveto{\pgfqpoint{-0.010825in}{-0.018638in}}{\pgfqpoint{-0.005525in}{-0.020833in}}{\pgfqpoint{0.000000in}{-0.020833in}}%
\pgfpathclose%
\pgfusepath{stroke,fill}%
}%
\begin{pgfscope}%
\pgfsys@transformshift{0.720524in}{3.211797in}%
\pgfsys@useobject{currentmarker}{}%
\end{pgfscope}%
\end{pgfscope}%
\begin{pgfscope}%
\pgftext[x=0.970524in,y=3.175339in,left,base]{\rmfamily\fontsize{10.000000}{12.000000}\selectfont test}%
\end{pgfscope}%
\begin{pgfscope}%
\pgfsetbuttcap%
\pgfsetmiterjoin%
\definecolor{currentfill}{rgb}{1.000000,1.000000,1.000000}%
\pgfsetfillcolor{currentfill}%
\pgfsetlinewidth{0.000000pt}%
\definecolor{currentstroke}{rgb}{0.000000,0.000000,0.000000}%
\pgfsetstrokecolor{currentstroke}%
\pgfsetstrokeopacity{0.000000}%
\pgfsetdash{}{0pt}%
\pgfpathmoveto{\pgfqpoint{5.562518in}{3.021449in}}%
\pgfpathlineto{\pgfqpoint{9.938988in}{3.021449in}}%
\pgfpathlineto{\pgfqpoint{9.938988in}{3.994081in}}%
\pgfpathlineto{\pgfqpoint{5.562518in}{3.994081in}}%
\pgfpathclose%
\pgfusepath{fill}%
\end{pgfscope}%
\begin{pgfscope}%
\pgfpathrectangle{\pgfqpoint{5.562518in}{3.021449in}}{\pgfqpoint{4.376471in}{0.972632in}}%
\pgfusepath{clip}%
\pgfsetbuttcap%
\pgfsetroundjoin%
\definecolor{currentfill}{rgb}{1.000000,0.000000,0.000000}%
\pgfsetfillcolor{currentfill}%
\pgfsetlinewidth{2.007500pt}%
\definecolor{currentstroke}{rgb}{1.000000,0.000000,0.000000}%
\pgfsetstrokecolor{currentstroke}%
\pgfsetdash{}{0pt}%
\pgfpathmoveto{\pgfqpoint{7.707476in}{3.175317in}}%
\pgfpathlineto{\pgfqpoint{7.790809in}{3.175317in}}%
\pgfpathmoveto{\pgfqpoint{7.749143in}{3.133650in}}%
\pgfpathlineto{\pgfqpoint{7.749143in}{3.216983in}}%
\pgfusepath{stroke,fill}%
\end{pgfscope}%
\begin{pgfscope}%
\pgfpathrectangle{\pgfqpoint{5.562518in}{3.021449in}}{\pgfqpoint{4.376471in}{0.972632in}}%
\pgfusepath{clip}%
\pgfsetbuttcap%
\pgfsetroundjoin%
\definecolor{currentfill}{rgb}{1.000000,0.000000,0.000000}%
\pgfsetfillcolor{currentfill}%
\pgfsetlinewidth{2.007500pt}%
\definecolor{currentstroke}{rgb}{1.000000,0.000000,0.000000}%
\pgfsetstrokecolor{currentstroke}%
\pgfsetdash{}{0pt}%
\pgfpathmoveto{\pgfqpoint{9.724764in}{3.834419in}}%
\pgfpathlineto{\pgfqpoint{9.808097in}{3.834419in}}%
\pgfpathmoveto{\pgfqpoint{9.766430in}{3.792752in}}%
\pgfpathlineto{\pgfqpoint{9.766430in}{3.876085in}}%
\pgfusepath{stroke,fill}%
\end{pgfscope}%
\begin{pgfscope}%
\pgfpathrectangle{\pgfqpoint{5.562518in}{3.021449in}}{\pgfqpoint{4.376471in}{0.972632in}}%
\pgfusepath{clip}%
\pgfsetbuttcap%
\pgfsetroundjoin%
\definecolor{currentfill}{rgb}{1.000000,0.000000,0.000000}%
\pgfsetfillcolor{currentfill}%
\pgfsetlinewidth{2.007500pt}%
\definecolor{currentstroke}{rgb}{1.000000,0.000000,0.000000}%
\pgfsetstrokecolor{currentstroke}%
\pgfsetdash{}{0pt}%
\pgfpathmoveto{\pgfqpoint{8.958985in}{3.210480in}}%
\pgfpathlineto{\pgfqpoint{9.042318in}{3.210480in}}%
\pgfpathmoveto{\pgfqpoint{9.000652in}{3.168813in}}%
\pgfpathlineto{\pgfqpoint{9.000652in}{3.252146in}}%
\pgfusepath{stroke,fill}%
\end{pgfscope}%
\begin{pgfscope}%
\pgfpathrectangle{\pgfqpoint{5.562518in}{3.021449in}}{\pgfqpoint{4.376471in}{0.972632in}}%
\pgfusepath{clip}%
\pgfsetbuttcap%
\pgfsetroundjoin%
\definecolor{currentfill}{rgb}{1.000000,0.000000,0.000000}%
\pgfsetfillcolor{currentfill}%
\pgfsetlinewidth{2.007500pt}%
\definecolor{currentstroke}{rgb}{1.000000,0.000000,0.000000}%
\pgfsetstrokecolor{currentstroke}%
\pgfsetdash{}{0pt}%
\pgfpathmoveto{\pgfqpoint{8.492154in}{3.558860in}}%
\pgfpathlineto{\pgfqpoint{8.575487in}{3.558860in}}%
\pgfpathmoveto{\pgfqpoint{8.533821in}{3.517193in}}%
\pgfpathlineto{\pgfqpoint{8.533821in}{3.600526in}}%
\pgfusepath{stroke,fill}%
\end{pgfscope}%
\begin{pgfscope}%
\pgfpathrectangle{\pgfqpoint{5.562518in}{3.021449in}}{\pgfqpoint{4.376471in}{0.972632in}}%
\pgfusepath{clip}%
\pgfsetbuttcap%
\pgfsetroundjoin%
\definecolor{currentfill}{rgb}{1.000000,0.000000,0.000000}%
\pgfsetfillcolor{currentfill}%
\pgfsetlinewidth{2.007500pt}%
\definecolor{currentstroke}{rgb}{1.000000,0.000000,0.000000}%
\pgfsetstrokecolor{currentstroke}%
\pgfsetdash{}{0pt}%
\pgfpathmoveto{\pgfqpoint{6.942394in}{3.403213in}}%
\pgfpathlineto{\pgfqpoint{7.025727in}{3.403213in}}%
\pgfpathmoveto{\pgfqpoint{6.984061in}{3.361546in}}%
\pgfpathlineto{\pgfqpoint{6.984061in}{3.444880in}}%
\pgfusepath{stroke,fill}%
\end{pgfscope}%
\begin{pgfscope}%
\pgfpathrectangle{\pgfqpoint{5.562518in}{3.021449in}}{\pgfqpoint{4.376471in}{0.972632in}}%
\pgfusepath{clip}%
\pgfsetbuttcap%
\pgfsetroundjoin%
\definecolor{currentfill}{rgb}{1.000000,0.000000,0.000000}%
\pgfsetfillcolor{currentfill}%
\pgfsetlinewidth{2.007500pt}%
\definecolor{currentstroke}{rgb}{1.000000,0.000000,0.000000}%
\pgfsetstrokecolor{currentstroke}%
\pgfsetdash{}{0pt}%
\pgfpathmoveto{\pgfqpoint{6.942309in}{3.651787in}}%
\pgfpathlineto{\pgfqpoint{7.025643in}{3.651787in}}%
\pgfpathmoveto{\pgfqpoint{6.983976in}{3.610120in}}%
\pgfpathlineto{\pgfqpoint{6.983976in}{3.693453in}}%
\pgfusepath{stroke,fill}%
\end{pgfscope}%
\begin{pgfscope}%
\pgfpathrectangle{\pgfqpoint{5.562518in}{3.021449in}}{\pgfqpoint{4.376471in}{0.972632in}}%
\pgfusepath{clip}%
\pgfsetbuttcap%
\pgfsetroundjoin%
\definecolor{currentfill}{rgb}{1.000000,0.000000,0.000000}%
\pgfsetfillcolor{currentfill}%
\pgfsetlinewidth{2.007500pt}%
\definecolor{currentstroke}{rgb}{1.000000,0.000000,0.000000}%
\pgfsetstrokecolor{currentstroke}%
\pgfsetdash{}{0pt}%
\pgfpathmoveto{\pgfqpoint{6.599506in}{3.589256in}}%
\pgfpathlineto{\pgfqpoint{6.682839in}{3.589256in}}%
\pgfpathmoveto{\pgfqpoint{6.641173in}{3.547590in}}%
\pgfpathlineto{\pgfqpoint{6.641173in}{3.630923in}}%
\pgfusepath{stroke,fill}%
\end{pgfscope}%
\begin{pgfscope}%
\pgfpathrectangle{\pgfqpoint{5.562518in}{3.021449in}}{\pgfqpoint{4.376471in}{0.972632in}}%
\pgfusepath{clip}%
\pgfsetbuttcap%
\pgfsetroundjoin%
\definecolor{currentfill}{rgb}{1.000000,0.000000,0.000000}%
\pgfsetfillcolor{currentfill}%
\pgfsetlinewidth{2.007500pt}%
\definecolor{currentstroke}{rgb}{1.000000,0.000000,0.000000}%
\pgfsetstrokecolor{currentstroke}%
\pgfsetdash{}{0pt}%
\pgfpathmoveto{\pgfqpoint{9.428781in}{3.671167in}}%
\pgfpathlineto{\pgfqpoint{9.512114in}{3.671167in}}%
\pgfpathmoveto{\pgfqpoint{9.470447in}{3.629500in}}%
\pgfpathlineto{\pgfqpoint{9.470447in}{3.712834in}}%
\pgfusepath{stroke,fill}%
\end{pgfscope}%
\begin{pgfscope}%
\pgfpathrectangle{\pgfqpoint{5.562518in}{3.021449in}}{\pgfqpoint{4.376471in}{0.972632in}}%
\pgfusepath{clip}%
\pgfsetbuttcap%
\pgfsetroundjoin%
\definecolor{currentfill}{rgb}{1.000000,0.000000,0.000000}%
\pgfsetfillcolor{currentfill}%
\pgfsetlinewidth{2.007500pt}%
\definecolor{currentstroke}{rgb}{1.000000,0.000000,0.000000}%
\pgfsetstrokecolor{currentstroke}%
\pgfsetdash{}{0pt}%
\pgfpathmoveto{\pgfqpoint{8.500755in}{3.668423in}}%
\pgfpathlineto{\pgfqpoint{8.584088in}{3.668423in}}%
\pgfpathmoveto{\pgfqpoint{8.542421in}{3.626756in}}%
\pgfpathlineto{\pgfqpoint{8.542421in}{3.710090in}}%
\pgfusepath{stroke,fill}%
\end{pgfscope}%
\begin{pgfscope}%
\pgfpathrectangle{\pgfqpoint{5.562518in}{3.021449in}}{\pgfqpoint{4.376471in}{0.972632in}}%
\pgfusepath{clip}%
\pgfsetbuttcap%
\pgfsetroundjoin%
\definecolor{currentfill}{rgb}{1.000000,0.000000,0.000000}%
\pgfsetfillcolor{currentfill}%
\pgfsetlinewidth{2.007500pt}%
\definecolor{currentstroke}{rgb}{1.000000,0.000000,0.000000}%
\pgfsetstrokecolor{currentstroke}%
\pgfsetdash{}{0pt}%
\pgfpathmoveto{\pgfqpoint{8.875232in}{3.160028in}}%
\pgfpathlineto{\pgfqpoint{8.958565in}{3.160028in}}%
\pgfpathmoveto{\pgfqpoint{8.916899in}{3.118362in}}%
\pgfpathlineto{\pgfqpoint{8.916899in}{3.201695in}}%
\pgfusepath{stroke,fill}%
\end{pgfscope}%
\begin{pgfscope}%
\pgfpathrectangle{\pgfqpoint{5.562518in}{3.021449in}}{\pgfqpoint{4.376471in}{0.972632in}}%
\pgfusepath{clip}%
\pgfsetbuttcap%
\pgfsetroundjoin%
\definecolor{currentfill}{rgb}{1.000000,0.000000,0.000000}%
\pgfsetfillcolor{currentfill}%
\pgfsetlinewidth{2.007500pt}%
\definecolor{currentstroke}{rgb}{1.000000,0.000000,0.000000}%
\pgfsetstrokecolor{currentstroke}%
\pgfsetdash{}{0pt}%
\pgfpathmoveto{\pgfqpoint{6.468215in}{3.812918in}}%
\pgfpathlineto{\pgfqpoint{6.551548in}{3.812918in}}%
\pgfpathmoveto{\pgfqpoint{6.509882in}{3.771251in}}%
\pgfpathlineto{\pgfqpoint{6.509882in}{3.854585in}}%
\pgfusepath{stroke,fill}%
\end{pgfscope}%
\begin{pgfscope}%
\pgfpathrectangle{\pgfqpoint{5.562518in}{3.021449in}}{\pgfqpoint{4.376471in}{0.972632in}}%
\pgfusepath{clip}%
\pgfsetbuttcap%
\pgfsetroundjoin%
\definecolor{currentfill}{rgb}{1.000000,0.000000,0.000000}%
\pgfsetfillcolor{currentfill}%
\pgfsetlinewidth{2.007500pt}%
\definecolor{currentstroke}{rgb}{1.000000,0.000000,0.000000}%
\pgfsetstrokecolor{currentstroke}%
\pgfsetdash{}{0pt}%
\pgfpathmoveto{\pgfqpoint{9.791971in}{3.550979in}}%
\pgfpathlineto{\pgfqpoint{9.875304in}{3.550979in}}%
\pgfpathmoveto{\pgfqpoint{9.833637in}{3.509313in}}%
\pgfpathlineto{\pgfqpoint{9.833637in}{3.592646in}}%
\pgfusepath{stroke,fill}%
\end{pgfscope}%
\begin{pgfscope}%
\pgfpathrectangle{\pgfqpoint{5.562518in}{3.021449in}}{\pgfqpoint{4.376471in}{0.972632in}}%
\pgfusepath{clip}%
\pgfsetbuttcap%
\pgfsetroundjoin%
\definecolor{currentfill}{rgb}{1.000000,0.000000,0.000000}%
\pgfsetfillcolor{currentfill}%
\pgfsetlinewidth{2.007500pt}%
\definecolor{currentstroke}{rgb}{1.000000,0.000000,0.000000}%
\pgfsetstrokecolor{currentstroke}%
\pgfsetdash{}{0pt}%
\pgfpathmoveto{\pgfqpoint{9.310674in}{3.503231in}}%
\pgfpathlineto{\pgfqpoint{9.394007in}{3.503231in}}%
\pgfpathmoveto{\pgfqpoint{9.352340in}{3.461564in}}%
\pgfpathlineto{\pgfqpoint{9.352340in}{3.544898in}}%
\pgfusepath{stroke,fill}%
\end{pgfscope}%
\begin{pgfscope}%
\pgfpathrectangle{\pgfqpoint{5.562518in}{3.021449in}}{\pgfqpoint{4.376471in}{0.972632in}}%
\pgfusepath{clip}%
\pgfsetbuttcap%
\pgfsetroundjoin%
\definecolor{currentfill}{rgb}{1.000000,0.000000,0.000000}%
\pgfsetfillcolor{currentfill}%
\pgfsetlinewidth{2.007500pt}%
\definecolor{currentstroke}{rgb}{1.000000,0.000000,0.000000}%
\pgfsetstrokecolor{currentstroke}%
\pgfsetdash{}{0pt}%
\pgfpathmoveto{\pgfqpoint{7.139582in}{3.642703in}}%
\pgfpathlineto{\pgfqpoint{7.222915in}{3.642703in}}%
\pgfpathmoveto{\pgfqpoint{7.181248in}{3.601036in}}%
\pgfpathlineto{\pgfqpoint{7.181248in}{3.684369in}}%
\pgfusepath{stroke,fill}%
\end{pgfscope}%
\begin{pgfscope}%
\pgfpathrectangle{\pgfqpoint{5.562518in}{3.021449in}}{\pgfqpoint{4.376471in}{0.972632in}}%
\pgfusepath{clip}%
\pgfsetbuttcap%
\pgfsetroundjoin%
\definecolor{currentfill}{rgb}{1.000000,0.000000,0.000000}%
\pgfsetfillcolor{currentfill}%
\pgfsetlinewidth{2.007500pt}%
\definecolor{currentstroke}{rgb}{1.000000,0.000000,0.000000}%
\pgfsetstrokecolor{currentstroke}%
\pgfsetdash{}{0pt}%
\pgfpathmoveto{\pgfqpoint{7.032746in}{3.608855in}}%
\pgfpathlineto{\pgfqpoint{7.116080in}{3.608855in}}%
\pgfpathmoveto{\pgfqpoint{7.074413in}{3.567189in}}%
\pgfpathlineto{\pgfqpoint{7.074413in}{3.650522in}}%
\pgfusepath{stroke,fill}%
\end{pgfscope}%
\begin{pgfscope}%
\pgfpathrectangle{\pgfqpoint{5.562518in}{3.021449in}}{\pgfqpoint{4.376471in}{0.972632in}}%
\pgfusepath{clip}%
\pgfsetbuttcap%
\pgfsetroundjoin%
\definecolor{currentfill}{rgb}{1.000000,0.000000,0.000000}%
\pgfsetfillcolor{currentfill}%
\pgfsetlinewidth{2.007500pt}%
\definecolor{currentstroke}{rgb}{1.000000,0.000000,0.000000}%
\pgfsetstrokecolor{currentstroke}%
\pgfsetdash{}{0pt}%
\pgfpathmoveto{\pgfqpoint{7.038277in}{3.556162in}}%
\pgfpathlineto{\pgfqpoint{7.121610in}{3.556162in}}%
\pgfpathmoveto{\pgfqpoint{7.079943in}{3.514495in}}%
\pgfpathlineto{\pgfqpoint{7.079943in}{3.597828in}}%
\pgfusepath{stroke,fill}%
\end{pgfscope}%
\begin{pgfscope}%
\pgfpathrectangle{\pgfqpoint{5.562518in}{3.021449in}}{\pgfqpoint{4.376471in}{0.972632in}}%
\pgfusepath{clip}%
\pgfsetbuttcap%
\pgfsetroundjoin%
\definecolor{currentfill}{rgb}{1.000000,0.000000,0.000000}%
\pgfsetfillcolor{currentfill}%
\pgfsetlinewidth{2.007500pt}%
\definecolor{currentstroke}{rgb}{1.000000,0.000000,0.000000}%
\pgfsetstrokecolor{currentstroke}%
\pgfsetdash{}{0pt}%
\pgfpathmoveto{\pgfqpoint{7.461351in}{3.578166in}}%
\pgfpathlineto{\pgfqpoint{7.544684in}{3.578166in}}%
\pgfpathmoveto{\pgfqpoint{7.503018in}{3.536499in}}%
\pgfpathlineto{\pgfqpoint{7.503018in}{3.619833in}}%
\pgfusepath{stroke,fill}%
\end{pgfscope}%
\begin{pgfscope}%
\pgfpathrectangle{\pgfqpoint{5.562518in}{3.021449in}}{\pgfqpoint{4.376471in}{0.972632in}}%
\pgfusepath{clip}%
\pgfsetbuttcap%
\pgfsetroundjoin%
\definecolor{currentfill}{rgb}{1.000000,0.000000,0.000000}%
\pgfsetfillcolor{currentfill}%
\pgfsetlinewidth{2.007500pt}%
\definecolor{currentstroke}{rgb}{1.000000,0.000000,0.000000}%
\pgfsetstrokecolor{currentstroke}%
\pgfsetdash{}{0pt}%
\pgfpathmoveto{\pgfqpoint{8.233410in}{3.452952in}}%
\pgfpathlineto{\pgfqpoint{8.316743in}{3.452952in}}%
\pgfpathmoveto{\pgfqpoint{8.275077in}{3.411286in}}%
\pgfpathlineto{\pgfqpoint{8.275077in}{3.494619in}}%
\pgfusepath{stroke,fill}%
\end{pgfscope}%
\begin{pgfscope}%
\pgfpathrectangle{\pgfqpoint{5.562518in}{3.021449in}}{\pgfqpoint{4.376471in}{0.972632in}}%
\pgfusepath{clip}%
\pgfsetbuttcap%
\pgfsetroundjoin%
\definecolor{currentfill}{rgb}{1.000000,0.000000,0.000000}%
\pgfsetfillcolor{currentfill}%
\pgfsetlinewidth{2.007500pt}%
\definecolor{currentstroke}{rgb}{1.000000,0.000000,0.000000}%
\pgfsetstrokecolor{currentstroke}%
\pgfsetdash{}{0pt}%
\pgfpathmoveto{\pgfqpoint{7.908461in}{3.045704in}}%
\pgfpathlineto{\pgfqpoint{7.991794in}{3.045704in}}%
\pgfpathmoveto{\pgfqpoint{7.950127in}{3.004037in}}%
\pgfpathlineto{\pgfqpoint{7.950127in}{3.087370in}}%
\pgfusepath{stroke,fill}%
\end{pgfscope}%
\begin{pgfscope}%
\pgfpathrectangle{\pgfqpoint{5.562518in}{3.021449in}}{\pgfqpoint{4.376471in}{0.972632in}}%
\pgfusepath{clip}%
\pgfsetbuttcap%
\pgfsetroundjoin%
\definecolor{currentfill}{rgb}{1.000000,0.000000,0.000000}%
\pgfsetfillcolor{currentfill}%
\pgfsetlinewidth{2.007500pt}%
\definecolor{currentstroke}{rgb}{1.000000,0.000000,0.000000}%
\pgfsetstrokecolor{currentstroke}%
\pgfsetdash{}{0pt}%
\pgfpathmoveto{\pgfqpoint{7.415790in}{3.558022in}}%
\pgfpathlineto{\pgfqpoint{7.499123in}{3.558022in}}%
\pgfpathmoveto{\pgfqpoint{7.457456in}{3.516355in}}%
\pgfpathlineto{\pgfqpoint{7.457456in}{3.599689in}}%
\pgfusepath{stroke,fill}%
\end{pgfscope}%
\begin{pgfscope}%
\pgfpathrectangle{\pgfqpoint{5.562518in}{3.021449in}}{\pgfqpoint{4.376471in}{0.972632in}}%
\pgfusepath{clip}%
\pgfsetbuttcap%
\pgfsetroundjoin%
\definecolor{currentfill}{rgb}{1.000000,0.000000,0.000000}%
\pgfsetfillcolor{currentfill}%
\pgfsetlinewidth{2.007500pt}%
\definecolor{currentstroke}{rgb}{1.000000,0.000000,0.000000}%
\pgfsetstrokecolor{currentstroke}%
\pgfsetdash{}{0pt}%
\pgfpathmoveto{\pgfqpoint{8.538350in}{3.519094in}}%
\pgfpathlineto{\pgfqpoint{8.621683in}{3.519094in}}%
\pgfpathmoveto{\pgfqpoint{8.580017in}{3.477427in}}%
\pgfpathlineto{\pgfqpoint{8.580017in}{3.560760in}}%
\pgfusepath{stroke,fill}%
\end{pgfscope}%
\begin{pgfscope}%
\pgfpathrectangle{\pgfqpoint{5.562518in}{3.021449in}}{\pgfqpoint{4.376471in}{0.972632in}}%
\pgfusepath{clip}%
\pgfsetbuttcap%
\pgfsetroundjoin%
\definecolor{currentfill}{rgb}{1.000000,0.000000,0.000000}%
\pgfsetfillcolor{currentfill}%
\pgfsetlinewidth{2.007500pt}%
\definecolor{currentstroke}{rgb}{1.000000,0.000000,0.000000}%
\pgfsetstrokecolor{currentstroke}%
\pgfsetdash{}{0pt}%
\pgfpathmoveto{\pgfqpoint{6.884538in}{3.541907in}}%
\pgfpathlineto{\pgfqpoint{6.967871in}{3.541907in}}%
\pgfpathmoveto{\pgfqpoint{6.926204in}{3.500240in}}%
\pgfpathlineto{\pgfqpoint{6.926204in}{3.583574in}}%
\pgfusepath{stroke,fill}%
\end{pgfscope}%
\begin{pgfscope}%
\pgfpathrectangle{\pgfqpoint{5.562518in}{3.021449in}}{\pgfqpoint{4.376471in}{0.972632in}}%
\pgfusepath{clip}%
\pgfsetbuttcap%
\pgfsetroundjoin%
\definecolor{currentfill}{rgb}{1.000000,0.000000,0.000000}%
\pgfsetfillcolor{currentfill}%
\pgfsetlinewidth{2.007500pt}%
\definecolor{currentstroke}{rgb}{1.000000,0.000000,0.000000}%
\pgfsetstrokecolor{currentstroke}%
\pgfsetdash{}{0pt}%
\pgfpathmoveto{\pgfqpoint{7.418995in}{3.663222in}}%
\pgfpathlineto{\pgfqpoint{7.502328in}{3.663222in}}%
\pgfpathmoveto{\pgfqpoint{7.460662in}{3.621556in}}%
\pgfpathlineto{\pgfqpoint{7.460662in}{3.704889in}}%
\pgfusepath{stroke,fill}%
\end{pgfscope}%
\begin{pgfscope}%
\pgfpathrectangle{\pgfqpoint{5.562518in}{3.021449in}}{\pgfqpoint{4.376471in}{0.972632in}}%
\pgfusepath{clip}%
\pgfsetbuttcap%
\pgfsetroundjoin%
\definecolor{currentfill}{rgb}{1.000000,0.000000,0.000000}%
\pgfsetfillcolor{currentfill}%
\pgfsetlinewidth{2.007500pt}%
\definecolor{currentstroke}{rgb}{1.000000,0.000000,0.000000}%
\pgfsetstrokecolor{currentstroke}%
\pgfsetdash{}{0pt}%
\pgfpathmoveto{\pgfqpoint{7.678843in}{3.144955in}}%
\pgfpathlineto{\pgfqpoint{7.762176in}{3.144955in}}%
\pgfpathmoveto{\pgfqpoint{7.720509in}{3.103289in}}%
\pgfpathlineto{\pgfqpoint{7.720509in}{3.186622in}}%
\pgfusepath{stroke,fill}%
\end{pgfscope}%
\begin{pgfscope}%
\pgfpathrectangle{\pgfqpoint{5.562518in}{3.021449in}}{\pgfqpoint{4.376471in}{0.972632in}}%
\pgfusepath{clip}%
\pgfsetbuttcap%
\pgfsetroundjoin%
\definecolor{currentfill}{rgb}{1.000000,0.000000,0.000000}%
\pgfsetfillcolor{currentfill}%
\pgfsetlinewidth{2.007500pt}%
\definecolor{currentstroke}{rgb}{1.000000,0.000000,0.000000}%
\pgfsetstrokecolor{currentstroke}%
\pgfsetdash{}{0pt}%
\pgfpathmoveto{\pgfqpoint{7.992927in}{3.266154in}}%
\pgfpathlineto{\pgfqpoint{8.076260in}{3.266154in}}%
\pgfpathmoveto{\pgfqpoint{8.034593in}{3.224487in}}%
\pgfpathlineto{\pgfqpoint{8.034593in}{3.307821in}}%
\pgfusepath{stroke,fill}%
\end{pgfscope}%
\begin{pgfscope}%
\pgfpathrectangle{\pgfqpoint{5.562518in}{3.021449in}}{\pgfqpoint{4.376471in}{0.972632in}}%
\pgfusepath{clip}%
\pgfsetbuttcap%
\pgfsetroundjoin%
\definecolor{currentfill}{rgb}{1.000000,0.000000,0.000000}%
\pgfsetfillcolor{currentfill}%
\pgfsetlinewidth{2.007500pt}%
\definecolor{currentstroke}{rgb}{1.000000,0.000000,0.000000}%
\pgfsetstrokecolor{currentstroke}%
\pgfsetdash{}{0pt}%
\pgfpathmoveto{\pgfqpoint{9.145185in}{3.368013in}}%
\pgfpathlineto{\pgfqpoint{9.228518in}{3.368013in}}%
\pgfpathmoveto{\pgfqpoint{9.186851in}{3.326347in}}%
\pgfpathlineto{\pgfqpoint{9.186851in}{3.409680in}}%
\pgfusepath{stroke,fill}%
\end{pgfscope}%
\begin{pgfscope}%
\pgfpathrectangle{\pgfqpoint{5.562518in}{3.021449in}}{\pgfqpoint{4.376471in}{0.972632in}}%
\pgfusepath{clip}%
\pgfsetbuttcap%
\pgfsetroundjoin%
\definecolor{currentfill}{rgb}{1.000000,0.000000,0.000000}%
\pgfsetfillcolor{currentfill}%
\pgfsetlinewidth{2.007500pt}%
\definecolor{currentstroke}{rgb}{1.000000,0.000000,0.000000}%
\pgfsetstrokecolor{currentstroke}%
\pgfsetdash{}{0pt}%
\pgfpathmoveto{\pgfqpoint{7.095238in}{3.519163in}}%
\pgfpathlineto{\pgfqpoint{7.178572in}{3.519163in}}%
\pgfpathmoveto{\pgfqpoint{7.136905in}{3.477497in}}%
\pgfpathlineto{\pgfqpoint{7.136905in}{3.560830in}}%
\pgfusepath{stroke,fill}%
\end{pgfscope}%
\begin{pgfscope}%
\pgfpathrectangle{\pgfqpoint{5.562518in}{3.021449in}}{\pgfqpoint{4.376471in}{0.972632in}}%
\pgfusepath{clip}%
\pgfsetbuttcap%
\pgfsetroundjoin%
\definecolor{currentfill}{rgb}{1.000000,0.000000,0.000000}%
\pgfsetfillcolor{currentfill}%
\pgfsetlinewidth{2.007500pt}%
\definecolor{currentstroke}{rgb}{1.000000,0.000000,0.000000}%
\pgfsetstrokecolor{currentstroke}%
\pgfsetdash{}{0pt}%
\pgfpathmoveto{\pgfqpoint{8.196571in}{3.505398in}}%
\pgfpathlineto{\pgfqpoint{8.279904in}{3.505398in}}%
\pgfpathmoveto{\pgfqpoint{8.238237in}{3.463731in}}%
\pgfpathlineto{\pgfqpoint{8.238237in}{3.547065in}}%
\pgfusepath{stroke,fill}%
\end{pgfscope}%
\begin{pgfscope}%
\pgfpathrectangle{\pgfqpoint{5.562518in}{3.021449in}}{\pgfqpoint{4.376471in}{0.972632in}}%
\pgfusepath{clip}%
\pgfsetbuttcap%
\pgfsetroundjoin%
\definecolor{currentfill}{rgb}{1.000000,0.000000,0.000000}%
\pgfsetfillcolor{currentfill}%
\pgfsetlinewidth{2.007500pt}%
\definecolor{currentstroke}{rgb}{1.000000,0.000000,0.000000}%
\pgfsetstrokecolor{currentstroke}%
\pgfsetdash{}{0pt}%
\pgfpathmoveto{\pgfqpoint{8.470293in}{3.699367in}}%
\pgfpathlineto{\pgfqpoint{8.553626in}{3.699367in}}%
\pgfpathmoveto{\pgfqpoint{8.511960in}{3.657700in}}%
\pgfpathlineto{\pgfqpoint{8.511960in}{3.741034in}}%
\pgfusepath{stroke,fill}%
\end{pgfscope}%
\begin{pgfscope}%
\pgfpathrectangle{\pgfqpoint{5.562518in}{3.021449in}}{\pgfqpoint{4.376471in}{0.972632in}}%
\pgfusepath{clip}%
\pgfsetbuttcap%
\pgfsetroundjoin%
\definecolor{currentfill}{rgb}{1.000000,0.000000,0.000000}%
\pgfsetfillcolor{currentfill}%
\pgfsetlinewidth{2.007500pt}%
\definecolor{currentstroke}{rgb}{1.000000,0.000000,0.000000}%
\pgfsetstrokecolor{currentstroke}%
\pgfsetdash{}{0pt}%
\pgfpathmoveto{\pgfqpoint{6.558776in}{3.743800in}}%
\pgfpathlineto{\pgfqpoint{6.642110in}{3.743800in}}%
\pgfpathmoveto{\pgfqpoint{6.600443in}{3.702133in}}%
\pgfpathlineto{\pgfqpoint{6.600443in}{3.785466in}}%
\pgfusepath{stroke,fill}%
\end{pgfscope}%
\begin{pgfscope}%
\pgfpathrectangle{\pgfqpoint{5.562518in}{3.021449in}}{\pgfqpoint{4.376471in}{0.972632in}}%
\pgfusepath{clip}%
\pgfsetbuttcap%
\pgfsetroundjoin%
\definecolor{currentfill}{rgb}{0.000000,0.000000,0.000000}%
\pgfsetfillcolor{currentfill}%
\pgfsetlinewidth{1.003750pt}%
\definecolor{currentstroke}{rgb}{0.000000,0.000000,0.000000}%
\pgfsetstrokecolor{currentstroke}%
\pgfsetdash{}{0pt}%
\pgfsys@defobject{currentmarker}{\pgfqpoint{-0.020833in}{-0.020833in}}{\pgfqpoint{0.020833in}{0.020833in}}{%
\pgfpathmoveto{\pgfqpoint{0.000000in}{-0.020833in}}%
\pgfpathcurveto{\pgfqpoint{0.005525in}{-0.020833in}}{\pgfqpoint{0.010825in}{-0.018638in}}{\pgfqpoint{0.014731in}{-0.014731in}}%
\pgfpathcurveto{\pgfqpoint{0.018638in}{-0.010825in}}{\pgfqpoint{0.020833in}{-0.005525in}}{\pgfqpoint{0.020833in}{0.000000in}}%
\pgfpathcurveto{\pgfqpoint{0.020833in}{0.005525in}}{\pgfqpoint{0.018638in}{0.010825in}}{\pgfqpoint{0.014731in}{0.014731in}}%
\pgfpathcurveto{\pgfqpoint{0.010825in}{0.018638in}}{\pgfqpoint{0.005525in}{0.020833in}}{\pgfqpoint{0.000000in}{0.020833in}}%
\pgfpathcurveto{\pgfqpoint{-0.005525in}{0.020833in}}{\pgfqpoint{-0.010825in}{0.018638in}}{\pgfqpoint{-0.014731in}{0.014731in}}%
\pgfpathcurveto{\pgfqpoint{-0.018638in}{0.010825in}}{\pgfqpoint{-0.020833in}{0.005525in}}{\pgfqpoint{-0.020833in}{0.000000in}}%
\pgfpathcurveto{\pgfqpoint{-0.020833in}{-0.005525in}}{\pgfqpoint{-0.018638in}{-0.010825in}}{\pgfqpoint{-0.014731in}{-0.014731in}}%
\pgfpathcurveto{\pgfqpoint{-0.010825in}{-0.018638in}}{\pgfqpoint{-0.005525in}{-0.020833in}}{\pgfqpoint{0.000000in}{-0.020833in}}%
\pgfpathclose%
\pgfusepath{stroke,fill}%
}%
\begin{pgfscope}%
\pgfsys@transformshift{6.437812in}{3.816659in}%
\pgfsys@useobject{currentmarker}{}%
\end{pgfscope}%
\begin{pgfscope}%
\pgfsys@transformshift{6.455406in}{3.844637in}%
\pgfsys@useobject{currentmarker}{}%
\end{pgfscope}%
\begin{pgfscope}%
\pgfsys@transformshift{6.472999in}{3.882954in}%
\pgfsys@useobject{currentmarker}{}%
\end{pgfscope}%
\begin{pgfscope}%
\pgfsys@transformshift{6.490593in}{3.920991in}%
\pgfsys@useobject{currentmarker}{}%
\end{pgfscope}%
\begin{pgfscope}%
\pgfsys@transformshift{6.508187in}{3.745895in}%
\pgfsys@useobject{currentmarker}{}%
\end{pgfscope}%
\begin{pgfscope}%
\pgfsys@transformshift{6.525781in}{3.747653in}%
\pgfsys@useobject{currentmarker}{}%
\end{pgfscope}%
\begin{pgfscope}%
\pgfsys@transformshift{6.543375in}{3.626327in}%
\pgfsys@useobject{currentmarker}{}%
\end{pgfscope}%
\begin{pgfscope}%
\pgfsys@transformshift{6.560969in}{3.589313in}%
\pgfsys@useobject{currentmarker}{}%
\end{pgfscope}%
\begin{pgfscope}%
\pgfsys@transformshift{6.578563in}{3.765265in}%
\pgfsys@useobject{currentmarker}{}%
\end{pgfscope}%
\begin{pgfscope}%
\pgfsys@transformshift{6.596156in}{3.793332in}%
\pgfsys@useobject{currentmarker}{}%
\end{pgfscope}%
\begin{pgfscope}%
\pgfsys@transformshift{6.613750in}{3.622352in}%
\pgfsys@useobject{currentmarker}{}%
\end{pgfscope}%
\begin{pgfscope}%
\pgfsys@transformshift{6.631344in}{3.705980in}%
\pgfsys@useobject{currentmarker}{}%
\end{pgfscope}%
\begin{pgfscope}%
\pgfsys@transformshift{6.648938in}{3.616720in}%
\pgfsys@useobject{currentmarker}{}%
\end{pgfscope}%
\begin{pgfscope}%
\pgfsys@transformshift{6.666532in}{3.491782in}%
\pgfsys@useobject{currentmarker}{}%
\end{pgfscope}%
\begin{pgfscope}%
\pgfsys@transformshift{6.684126in}{3.572268in}%
\pgfsys@useobject{currentmarker}{}%
\end{pgfscope}%
\begin{pgfscope}%
\pgfsys@transformshift{6.701720in}{3.671583in}%
\pgfsys@useobject{currentmarker}{}%
\end{pgfscope}%
\begin{pgfscope}%
\pgfsys@transformshift{6.719313in}{3.493954in}%
\pgfsys@useobject{currentmarker}{}%
\end{pgfscope}%
\begin{pgfscope}%
\pgfsys@transformshift{6.736907in}{3.639772in}%
\pgfsys@useobject{currentmarker}{}%
\end{pgfscope}%
\begin{pgfscope}%
\pgfsys@transformshift{6.754501in}{3.201427in}%
\pgfsys@useobject{currentmarker}{}%
\end{pgfscope}%
\begin{pgfscope}%
\pgfsys@transformshift{6.772095in}{3.537732in}%
\pgfsys@useobject{currentmarker}{}%
\end{pgfscope}%
\begin{pgfscope}%
\pgfsys@transformshift{6.789689in}{3.452961in}%
\pgfsys@useobject{currentmarker}{}%
\end{pgfscope}%
\begin{pgfscope}%
\pgfsys@transformshift{6.807283in}{3.405637in}%
\pgfsys@useobject{currentmarker}{}%
\end{pgfscope}%
\begin{pgfscope}%
\pgfsys@transformshift{6.824876in}{3.439143in}%
\pgfsys@useobject{currentmarker}{}%
\end{pgfscope}%
\begin{pgfscope}%
\pgfsys@transformshift{6.842470in}{3.224512in}%
\pgfsys@useobject{currentmarker}{}%
\end{pgfscope}%
\begin{pgfscope}%
\pgfsys@transformshift{6.860064in}{3.401765in}%
\pgfsys@useobject{currentmarker}{}%
\end{pgfscope}%
\begin{pgfscope}%
\pgfsys@transformshift{6.877658in}{3.460394in}%
\pgfsys@useobject{currentmarker}{}%
\end{pgfscope}%
\begin{pgfscope}%
\pgfsys@transformshift{6.895252in}{3.576129in}%
\pgfsys@useobject{currentmarker}{}%
\end{pgfscope}%
\begin{pgfscope}%
\pgfsys@transformshift{6.912846in}{3.377977in}%
\pgfsys@useobject{currentmarker}{}%
\end{pgfscope}%
\begin{pgfscope}%
\pgfsys@transformshift{6.930440in}{3.354469in}%
\pgfsys@useobject{currentmarker}{}%
\end{pgfscope}%
\begin{pgfscope}%
\pgfsys@transformshift{6.948033in}{3.393130in}%
\pgfsys@useobject{currentmarker}{}%
\end{pgfscope}%
\begin{pgfscope}%
\pgfsys@transformshift{6.965627in}{3.545847in}%
\pgfsys@useobject{currentmarker}{}%
\end{pgfscope}%
\begin{pgfscope}%
\pgfsys@transformshift{6.983221in}{3.496948in}%
\pgfsys@useobject{currentmarker}{}%
\end{pgfscope}%
\begin{pgfscope}%
\pgfsys@transformshift{7.000815in}{3.421744in}%
\pgfsys@useobject{currentmarker}{}%
\end{pgfscope}%
\begin{pgfscope}%
\pgfsys@transformshift{7.018409in}{3.540260in}%
\pgfsys@useobject{currentmarker}{}%
\end{pgfscope}%
\begin{pgfscope}%
\pgfsys@transformshift{7.036003in}{3.511812in}%
\pgfsys@useobject{currentmarker}{}%
\end{pgfscope}%
\begin{pgfscope}%
\pgfsys@transformshift{7.053597in}{3.614520in}%
\pgfsys@useobject{currentmarker}{}%
\end{pgfscope}%
\begin{pgfscope}%
\pgfsys@transformshift{7.071190in}{3.460144in}%
\pgfsys@useobject{currentmarker}{}%
\end{pgfscope}%
\begin{pgfscope}%
\pgfsys@transformshift{7.088784in}{3.513250in}%
\pgfsys@useobject{currentmarker}{}%
\end{pgfscope}%
\begin{pgfscope}%
\pgfsys@transformshift{7.106378in}{3.521974in}%
\pgfsys@useobject{currentmarker}{}%
\end{pgfscope}%
\begin{pgfscope}%
\pgfsys@transformshift{7.123972in}{3.428549in}%
\pgfsys@useobject{currentmarker}{}%
\end{pgfscope}%
\begin{pgfscope}%
\pgfsys@transformshift{7.141566in}{3.621624in}%
\pgfsys@useobject{currentmarker}{}%
\end{pgfscope}%
\begin{pgfscope}%
\pgfsys@transformshift{7.159160in}{3.632339in}%
\pgfsys@useobject{currentmarker}{}%
\end{pgfscope}%
\begin{pgfscope}%
\pgfsys@transformshift{7.176754in}{3.619953in}%
\pgfsys@useobject{currentmarker}{}%
\end{pgfscope}%
\begin{pgfscope}%
\pgfsys@transformshift{7.194347in}{3.608307in}%
\pgfsys@useobject{currentmarker}{}%
\end{pgfscope}%
\begin{pgfscope}%
\pgfsys@transformshift{7.211941in}{3.500233in}%
\pgfsys@useobject{currentmarker}{}%
\end{pgfscope}%
\begin{pgfscope}%
\pgfsys@transformshift{7.229535in}{3.611329in}%
\pgfsys@useobject{currentmarker}{}%
\end{pgfscope}%
\begin{pgfscope}%
\pgfsys@transformshift{7.247129in}{3.628146in}%
\pgfsys@useobject{currentmarker}{}%
\end{pgfscope}%
\begin{pgfscope}%
\pgfsys@transformshift{7.264723in}{3.588978in}%
\pgfsys@useobject{currentmarker}{}%
\end{pgfscope}%
\begin{pgfscope}%
\pgfsys@transformshift{7.282317in}{3.659668in}%
\pgfsys@useobject{currentmarker}{}%
\end{pgfscope}%
\begin{pgfscope}%
\pgfsys@transformshift{7.299910in}{3.720944in}%
\pgfsys@useobject{currentmarker}{}%
\end{pgfscope}%
\begin{pgfscope}%
\pgfsys@transformshift{7.317504in}{3.873278in}%
\pgfsys@useobject{currentmarker}{}%
\end{pgfscope}%
\begin{pgfscope}%
\pgfsys@transformshift{7.335098in}{3.700141in}%
\pgfsys@useobject{currentmarker}{}%
\end{pgfscope}%
\begin{pgfscope}%
\pgfsys@transformshift{7.352692in}{3.706888in}%
\pgfsys@useobject{currentmarker}{}%
\end{pgfscope}%
\begin{pgfscope}%
\pgfsys@transformshift{7.370286in}{3.669632in}%
\pgfsys@useobject{currentmarker}{}%
\end{pgfscope}%
\begin{pgfscope}%
\pgfsys@transformshift{7.387880in}{3.477192in}%
\pgfsys@useobject{currentmarker}{}%
\end{pgfscope}%
\begin{pgfscope}%
\pgfsys@transformshift{7.405474in}{3.661383in}%
\pgfsys@useobject{currentmarker}{}%
\end{pgfscope}%
\begin{pgfscope}%
\pgfsys@transformshift{7.423067in}{3.660743in}%
\pgfsys@useobject{currentmarker}{}%
\end{pgfscope}%
\begin{pgfscope}%
\pgfsys@transformshift{7.440661in}{3.892932in}%
\pgfsys@useobject{currentmarker}{}%
\end{pgfscope}%
\begin{pgfscope}%
\pgfsys@transformshift{7.458255in}{3.610835in}%
\pgfsys@useobject{currentmarker}{}%
\end{pgfscope}%
\begin{pgfscope}%
\pgfsys@transformshift{7.475849in}{3.646164in}%
\pgfsys@useobject{currentmarker}{}%
\end{pgfscope}%
\begin{pgfscope}%
\pgfsys@transformshift{7.493443in}{3.595827in}%
\pgfsys@useobject{currentmarker}{}%
\end{pgfscope}%
\begin{pgfscope}%
\pgfsys@transformshift{7.511037in}{3.463245in}%
\pgfsys@useobject{currentmarker}{}%
\end{pgfscope}%
\begin{pgfscope}%
\pgfsys@transformshift{7.528631in}{3.678465in}%
\pgfsys@useobject{currentmarker}{}%
\end{pgfscope}%
\begin{pgfscope}%
\pgfsys@transformshift{7.546224in}{3.618772in}%
\pgfsys@useobject{currentmarker}{}%
\end{pgfscope}%
\begin{pgfscope}%
\pgfsys@transformshift{7.563818in}{3.601697in}%
\pgfsys@useobject{currentmarker}{}%
\end{pgfscope}%
\begin{pgfscope}%
\pgfsys@transformshift{7.581412in}{3.407618in}%
\pgfsys@useobject{currentmarker}{}%
\end{pgfscope}%
\begin{pgfscope}%
\pgfsys@transformshift{7.599006in}{3.619506in}%
\pgfsys@useobject{currentmarker}{}%
\end{pgfscope}%
\begin{pgfscope}%
\pgfsys@transformshift{7.616600in}{3.312604in}%
\pgfsys@useobject{currentmarker}{}%
\end{pgfscope}%
\begin{pgfscope}%
\pgfsys@transformshift{7.634194in}{3.491172in}%
\pgfsys@useobject{currentmarker}{}%
\end{pgfscope}%
\begin{pgfscope}%
\pgfsys@transformshift{7.651788in}{3.630754in}%
\pgfsys@useobject{currentmarker}{}%
\end{pgfscope}%
\begin{pgfscope}%
\pgfsys@transformshift{7.669381in}{3.285820in}%
\pgfsys@useobject{currentmarker}{}%
\end{pgfscope}%
\begin{pgfscope}%
\pgfsys@transformshift{7.686975in}{3.306594in}%
\pgfsys@useobject{currentmarker}{}%
\end{pgfscope}%
\begin{pgfscope}%
\pgfsys@transformshift{7.704569in}{3.352498in}%
\pgfsys@useobject{currentmarker}{}%
\end{pgfscope}%
\begin{pgfscope}%
\pgfsys@transformshift{7.722163in}{3.270661in}%
\pgfsys@useobject{currentmarker}{}%
\end{pgfscope}%
\begin{pgfscope}%
\pgfsys@transformshift{7.739757in}{3.144857in}%
\pgfsys@useobject{currentmarker}{}%
\end{pgfscope}%
\begin{pgfscope}%
\pgfsys@transformshift{7.757351in}{3.290404in}%
\pgfsys@useobject{currentmarker}{}%
\end{pgfscope}%
\begin{pgfscope}%
\pgfsys@transformshift{7.774944in}{3.158689in}%
\pgfsys@useobject{currentmarker}{}%
\end{pgfscope}%
\begin{pgfscope}%
\pgfsys@transformshift{7.792538in}{3.298679in}%
\pgfsys@useobject{currentmarker}{}%
\end{pgfscope}%
\begin{pgfscope}%
\pgfsys@transformshift{7.810132in}{3.143582in}%
\pgfsys@useobject{currentmarker}{}%
\end{pgfscope}%
\begin{pgfscope}%
\pgfsys@transformshift{7.827726in}{3.381586in}%
\pgfsys@useobject{currentmarker}{}%
\end{pgfscope}%
\begin{pgfscope}%
\pgfsys@transformshift{7.845320in}{3.134903in}%
\pgfsys@useobject{currentmarker}{}%
\end{pgfscope}%
\begin{pgfscope}%
\pgfsys@transformshift{7.862914in}{3.173309in}%
\pgfsys@useobject{currentmarker}{}%
\end{pgfscope}%
\begin{pgfscope}%
\pgfsys@transformshift{7.880508in}{3.282083in}%
\pgfsys@useobject{currentmarker}{}%
\end{pgfscope}%
\begin{pgfscope}%
\pgfsys@transformshift{7.898101in}{3.070768in}%
\pgfsys@useobject{currentmarker}{}%
\end{pgfscope}%
\begin{pgfscope}%
\pgfsys@transformshift{7.915695in}{3.216451in}%
\pgfsys@useobject{currentmarker}{}%
\end{pgfscope}%
\begin{pgfscope}%
\pgfsys@transformshift{7.933289in}{3.325899in}%
\pgfsys@useobject{currentmarker}{}%
\end{pgfscope}%
\begin{pgfscope}%
\pgfsys@transformshift{7.950883in}{3.032774in}%
\pgfsys@useobject{currentmarker}{}%
\end{pgfscope}%
\begin{pgfscope}%
\pgfsys@transformshift{7.968477in}{3.218597in}%
\pgfsys@useobject{currentmarker}{}%
\end{pgfscope}%
\begin{pgfscope}%
\pgfsys@transformshift{7.986071in}{3.232500in}%
\pgfsys@useobject{currentmarker}{}%
\end{pgfscope}%
\begin{pgfscope}%
\pgfsys@transformshift{8.003665in}{3.293613in}%
\pgfsys@useobject{currentmarker}{}%
\end{pgfscope}%
\begin{pgfscope}%
\pgfsys@transformshift{8.021258in}{3.099173in}%
\pgfsys@useobject{currentmarker}{}%
\end{pgfscope}%
\begin{pgfscope}%
\pgfsys@transformshift{8.038852in}{3.102558in}%
\pgfsys@useobject{currentmarker}{}%
\end{pgfscope}%
\begin{pgfscope}%
\pgfsys@transformshift{8.056446in}{3.302690in}%
\pgfsys@useobject{currentmarker}{}%
\end{pgfscope}%
\begin{pgfscope}%
\pgfsys@transformshift{8.074040in}{3.294849in}%
\pgfsys@useobject{currentmarker}{}%
\end{pgfscope}%
\begin{pgfscope}%
\pgfsys@transformshift{8.091634in}{3.306413in}%
\pgfsys@useobject{currentmarker}{}%
\end{pgfscope}%
\begin{pgfscope}%
\pgfsys@transformshift{8.109228in}{3.333566in}%
\pgfsys@useobject{currentmarker}{}%
\end{pgfscope}%
\begin{pgfscope}%
\pgfsys@transformshift{8.126822in}{3.247974in}%
\pgfsys@useobject{currentmarker}{}%
\end{pgfscope}%
\begin{pgfscope}%
\pgfsys@transformshift{8.144415in}{3.359594in}%
\pgfsys@useobject{currentmarker}{}%
\end{pgfscope}%
\begin{pgfscope}%
\pgfsys@transformshift{8.162009in}{3.385539in}%
\pgfsys@useobject{currentmarker}{}%
\end{pgfscope}%
\begin{pgfscope}%
\pgfsys@transformshift{8.179603in}{3.303643in}%
\pgfsys@useobject{currentmarker}{}%
\end{pgfscope}%
\begin{pgfscope}%
\pgfsys@transformshift{8.197197in}{3.585407in}%
\pgfsys@useobject{currentmarker}{}%
\end{pgfscope}%
\begin{pgfscope}%
\pgfsys@transformshift{8.214791in}{3.464717in}%
\pgfsys@useobject{currentmarker}{}%
\end{pgfscope}%
\begin{pgfscope}%
\pgfsys@transformshift{8.232385in}{3.316123in}%
\pgfsys@useobject{currentmarker}{}%
\end{pgfscope}%
\begin{pgfscope}%
\pgfsys@transformshift{8.249978in}{3.523023in}%
\pgfsys@useobject{currentmarker}{}%
\end{pgfscope}%
\begin{pgfscope}%
\pgfsys@transformshift{8.267572in}{3.376814in}%
\pgfsys@useobject{currentmarker}{}%
\end{pgfscope}%
\begin{pgfscope}%
\pgfsys@transformshift{8.285166in}{3.573558in}%
\pgfsys@useobject{currentmarker}{}%
\end{pgfscope}%
\begin{pgfscope}%
\pgfsys@transformshift{8.302760in}{3.628456in}%
\pgfsys@useobject{currentmarker}{}%
\end{pgfscope}%
\begin{pgfscope}%
\pgfsys@transformshift{8.320354in}{3.444020in}%
\pgfsys@useobject{currentmarker}{}%
\end{pgfscope}%
\begin{pgfscope}%
\pgfsys@transformshift{8.337948in}{3.639552in}%
\pgfsys@useobject{currentmarker}{}%
\end{pgfscope}%
\begin{pgfscope}%
\pgfsys@transformshift{8.355542in}{3.597086in}%
\pgfsys@useobject{currentmarker}{}%
\end{pgfscope}%
\begin{pgfscope}%
\pgfsys@transformshift{8.373135in}{3.650284in}%
\pgfsys@useobject{currentmarker}{}%
\end{pgfscope}%
\begin{pgfscope}%
\pgfsys@transformshift{8.390729in}{3.769205in}%
\pgfsys@useobject{currentmarker}{}%
\end{pgfscope}%
\begin{pgfscope}%
\pgfsys@transformshift{8.408323in}{3.560412in}%
\pgfsys@useobject{currentmarker}{}%
\end{pgfscope}%
\begin{pgfscope}%
\pgfsys@transformshift{8.425917in}{3.515290in}%
\pgfsys@useobject{currentmarker}{}%
\end{pgfscope}%
\begin{pgfscope}%
\pgfsys@transformshift{8.443511in}{3.505998in}%
\pgfsys@useobject{currentmarker}{}%
\end{pgfscope}%
\begin{pgfscope}%
\pgfsys@transformshift{8.461105in}{3.515981in}%
\pgfsys@useobject{currentmarker}{}%
\end{pgfscope}%
\begin{pgfscope}%
\pgfsys@transformshift{8.478699in}{3.591375in}%
\pgfsys@useobject{currentmarker}{}%
\end{pgfscope}%
\begin{pgfscope}%
\pgfsys@transformshift{8.496292in}{3.632348in}%
\pgfsys@useobject{currentmarker}{}%
\end{pgfscope}%
\begin{pgfscope}%
\pgfsys@transformshift{8.513886in}{3.622486in}%
\pgfsys@useobject{currentmarker}{}%
\end{pgfscope}%
\begin{pgfscope}%
\pgfsys@transformshift{8.531480in}{3.673051in}%
\pgfsys@useobject{currentmarker}{}%
\end{pgfscope}%
\begin{pgfscope}%
\pgfsys@transformshift{8.549074in}{3.583542in}%
\pgfsys@useobject{currentmarker}{}%
\end{pgfscope}%
\begin{pgfscope}%
\pgfsys@transformshift{8.566668in}{3.720748in}%
\pgfsys@useobject{currentmarker}{}%
\end{pgfscope}%
\begin{pgfscope}%
\pgfsys@transformshift{8.584262in}{3.536308in}%
\pgfsys@useobject{currentmarker}{}%
\end{pgfscope}%
\begin{pgfscope}%
\pgfsys@transformshift{8.601855in}{3.826864in}%
\pgfsys@useobject{currentmarker}{}%
\end{pgfscope}%
\begin{pgfscope}%
\pgfsys@transformshift{8.619449in}{3.601447in}%
\pgfsys@useobject{currentmarker}{}%
\end{pgfscope}%
\begin{pgfscope}%
\pgfsys@transformshift{8.637043in}{3.436800in}%
\pgfsys@useobject{currentmarker}{}%
\end{pgfscope}%
\begin{pgfscope}%
\pgfsys@transformshift{8.654637in}{3.399695in}%
\pgfsys@useobject{currentmarker}{}%
\end{pgfscope}%
\begin{pgfscope}%
\pgfsys@transformshift{8.672231in}{3.540766in}%
\pgfsys@useobject{currentmarker}{}%
\end{pgfscope}%
\begin{pgfscope}%
\pgfsys@transformshift{8.689825in}{3.452262in}%
\pgfsys@useobject{currentmarker}{}%
\end{pgfscope}%
\begin{pgfscope}%
\pgfsys@transformshift{8.707419in}{3.529782in}%
\pgfsys@useobject{currentmarker}{}%
\end{pgfscope}%
\begin{pgfscope}%
\pgfsys@transformshift{8.725012in}{3.487651in}%
\pgfsys@useobject{currentmarker}{}%
\end{pgfscope}%
\begin{pgfscope}%
\pgfsys@transformshift{8.742606in}{3.414514in}%
\pgfsys@useobject{currentmarker}{}%
\end{pgfscope}%
\begin{pgfscope}%
\pgfsys@transformshift{8.760200in}{3.318419in}%
\pgfsys@useobject{currentmarker}{}%
\end{pgfscope}%
\begin{pgfscope}%
\pgfsys@transformshift{8.777794in}{3.233391in}%
\pgfsys@useobject{currentmarker}{}%
\end{pgfscope}%
\begin{pgfscope}%
\pgfsys@transformshift{8.795388in}{3.324825in}%
\pgfsys@useobject{currentmarker}{}%
\end{pgfscope}%
\begin{pgfscope}%
\pgfsys@transformshift{8.812982in}{3.440761in}%
\pgfsys@useobject{currentmarker}{}%
\end{pgfscope}%
\begin{pgfscope}%
\pgfsys@transformshift{8.830576in}{3.360544in}%
\pgfsys@useobject{currentmarker}{}%
\end{pgfscope}%
\begin{pgfscope}%
\pgfsys@transformshift{8.848169in}{3.198609in}%
\pgfsys@useobject{currentmarker}{}%
\end{pgfscope}%
\begin{pgfscope}%
\pgfsys@transformshift{8.865763in}{3.329618in}%
\pgfsys@useobject{currentmarker}{}%
\end{pgfscope}%
\begin{pgfscope}%
\pgfsys@transformshift{8.883357in}{3.339804in}%
\pgfsys@useobject{currentmarker}{}%
\end{pgfscope}%
\begin{pgfscope}%
\pgfsys@transformshift{8.900951in}{3.201497in}%
\pgfsys@useobject{currentmarker}{}%
\end{pgfscope}%
\begin{pgfscope}%
\pgfsys@transformshift{8.918545in}{3.298622in}%
\pgfsys@useobject{currentmarker}{}%
\end{pgfscope}%
\begin{pgfscope}%
\pgfsys@transformshift{8.936139in}{3.282780in}%
\pgfsys@useobject{currentmarker}{}%
\end{pgfscope}%
\begin{pgfscope}%
\pgfsys@transformshift{8.953733in}{3.156848in}%
\pgfsys@useobject{currentmarker}{}%
\end{pgfscope}%
\begin{pgfscope}%
\pgfsys@transformshift{8.971326in}{3.306674in}%
\pgfsys@useobject{currentmarker}{}%
\end{pgfscope}%
\begin{pgfscope}%
\pgfsys@transformshift{8.988920in}{3.327085in}%
\pgfsys@useobject{currentmarker}{}%
\end{pgfscope}%
\begin{pgfscope}%
\pgfsys@transformshift{9.006514in}{3.381950in}%
\pgfsys@useobject{currentmarker}{}%
\end{pgfscope}%
\begin{pgfscope}%
\pgfsys@transformshift{9.024108in}{3.383063in}%
\pgfsys@useobject{currentmarker}{}%
\end{pgfscope}%
\begin{pgfscope}%
\pgfsys@transformshift{9.041702in}{3.142916in}%
\pgfsys@useobject{currentmarker}{}%
\end{pgfscope}%
\begin{pgfscope}%
\pgfsys@transformshift{9.059296in}{3.195779in}%
\pgfsys@useobject{currentmarker}{}%
\end{pgfscope}%
\begin{pgfscope}%
\pgfsys@transformshift{9.076889in}{3.353333in}%
\pgfsys@useobject{currentmarker}{}%
\end{pgfscope}%
\begin{pgfscope}%
\pgfsys@transformshift{9.094483in}{3.365556in}%
\pgfsys@useobject{currentmarker}{}%
\end{pgfscope}%
\begin{pgfscope}%
\pgfsys@transformshift{9.112077in}{3.379945in}%
\pgfsys@useobject{currentmarker}{}%
\end{pgfscope}%
\begin{pgfscope}%
\pgfsys@transformshift{9.129671in}{3.734177in}%
\pgfsys@useobject{currentmarker}{}%
\end{pgfscope}%
\begin{pgfscope}%
\pgfsys@transformshift{9.147265in}{3.419437in}%
\pgfsys@useobject{currentmarker}{}%
\end{pgfscope}%
\begin{pgfscope}%
\pgfsys@transformshift{9.164859in}{3.495968in}%
\pgfsys@useobject{currentmarker}{}%
\end{pgfscope}%
\begin{pgfscope}%
\pgfsys@transformshift{9.182453in}{3.498300in}%
\pgfsys@useobject{currentmarker}{}%
\end{pgfscope}%
\begin{pgfscope}%
\pgfsys@transformshift{9.200046in}{3.489614in}%
\pgfsys@useobject{currentmarker}{}%
\end{pgfscope}%
\begin{pgfscope}%
\pgfsys@transformshift{9.217640in}{3.414718in}%
\pgfsys@useobject{currentmarker}{}%
\end{pgfscope}%
\begin{pgfscope}%
\pgfsys@transformshift{9.235234in}{3.547483in}%
\pgfsys@useobject{currentmarker}{}%
\end{pgfscope}%
\begin{pgfscope}%
\pgfsys@transformshift{9.252828in}{3.416908in}%
\pgfsys@useobject{currentmarker}{}%
\end{pgfscope}%
\begin{pgfscope}%
\pgfsys@transformshift{9.270422in}{3.496328in}%
\pgfsys@useobject{currentmarker}{}%
\end{pgfscope}%
\begin{pgfscope}%
\pgfsys@transformshift{9.288016in}{3.496552in}%
\pgfsys@useobject{currentmarker}{}%
\end{pgfscope}%
\begin{pgfscope}%
\pgfsys@transformshift{9.305610in}{3.579512in}%
\pgfsys@useobject{currentmarker}{}%
\end{pgfscope}%
\begin{pgfscope}%
\pgfsys@transformshift{9.323203in}{3.831099in}%
\pgfsys@useobject{currentmarker}{}%
\end{pgfscope}%
\begin{pgfscope}%
\pgfsys@transformshift{9.340797in}{3.432452in}%
\pgfsys@useobject{currentmarker}{}%
\end{pgfscope}%
\begin{pgfscope}%
\pgfsys@transformshift{9.358391in}{3.715690in}%
\pgfsys@useobject{currentmarker}{}%
\end{pgfscope}%
\begin{pgfscope}%
\pgfsys@transformshift{9.375985in}{3.506578in}%
\pgfsys@useobject{currentmarker}{}%
\end{pgfscope}%
\begin{pgfscope}%
\pgfsys@transformshift{9.393579in}{3.645069in}%
\pgfsys@useobject{currentmarker}{}%
\end{pgfscope}%
\begin{pgfscope}%
\pgfsys@transformshift{9.411173in}{3.825047in}%
\pgfsys@useobject{currentmarker}{}%
\end{pgfscope}%
\begin{pgfscope}%
\pgfsys@transformshift{9.428767in}{3.741829in}%
\pgfsys@useobject{currentmarker}{}%
\end{pgfscope}%
\begin{pgfscope}%
\pgfsys@transformshift{9.446360in}{3.645333in}%
\pgfsys@useobject{currentmarker}{}%
\end{pgfscope}%
\begin{pgfscope}%
\pgfsys@transformshift{9.463954in}{3.699740in}%
\pgfsys@useobject{currentmarker}{}%
\end{pgfscope}%
\begin{pgfscope}%
\pgfsys@transformshift{9.481548in}{3.857109in}%
\pgfsys@useobject{currentmarker}{}%
\end{pgfscope}%
\begin{pgfscope}%
\pgfsys@transformshift{9.499142in}{3.728557in}%
\pgfsys@useobject{currentmarker}{}%
\end{pgfscope}%
\begin{pgfscope}%
\pgfsys@transformshift{9.516736in}{3.836957in}%
\pgfsys@useobject{currentmarker}{}%
\end{pgfscope}%
\begin{pgfscope}%
\pgfsys@transformshift{9.534330in}{3.830224in}%
\pgfsys@useobject{currentmarker}{}%
\end{pgfscope}%
\begin{pgfscope}%
\pgfsys@transformshift{9.551923in}{3.768236in}%
\pgfsys@useobject{currentmarker}{}%
\end{pgfscope}%
\begin{pgfscope}%
\pgfsys@transformshift{9.569517in}{4.058155in}%
\pgfsys@useobject{currentmarker}{}%
\end{pgfscope}%
\begin{pgfscope}%
\pgfsys@transformshift{9.587111in}{3.909891in}%
\pgfsys@useobject{currentmarker}{}%
\end{pgfscope}%
\begin{pgfscope}%
\pgfsys@transformshift{9.604705in}{3.643266in}%
\pgfsys@useobject{currentmarker}{}%
\end{pgfscope}%
\begin{pgfscope}%
\pgfsys@transformshift{9.622299in}{3.868210in}%
\pgfsys@useobject{currentmarker}{}%
\end{pgfscope}%
\begin{pgfscope}%
\pgfsys@transformshift{9.639893in}{3.781297in}%
\pgfsys@useobject{currentmarker}{}%
\end{pgfscope}%
\begin{pgfscope}%
\pgfsys@transformshift{9.657487in}{3.931967in}%
\pgfsys@useobject{currentmarker}{}%
\end{pgfscope}%
\begin{pgfscope}%
\pgfsys@transformshift{9.675080in}{3.760888in}%
\pgfsys@useobject{currentmarker}{}%
\end{pgfscope}%
\begin{pgfscope}%
\pgfsys@transformshift{9.692674in}{3.823580in}%
\pgfsys@useobject{currentmarker}{}%
\end{pgfscope}%
\begin{pgfscope}%
\pgfsys@transformshift{9.710268in}{3.878959in}%
\pgfsys@useobject{currentmarker}{}%
\end{pgfscope}%
\begin{pgfscope}%
\pgfsys@transformshift{9.727862in}{3.906818in}%
\pgfsys@useobject{currentmarker}{}%
\end{pgfscope}%
\begin{pgfscope}%
\pgfsys@transformshift{9.745456in}{3.687678in}%
\pgfsys@useobject{currentmarker}{}%
\end{pgfscope}%
\begin{pgfscope}%
\pgfsys@transformshift{9.763050in}{3.764627in}%
\pgfsys@useobject{currentmarker}{}%
\end{pgfscope}%
\begin{pgfscope}%
\pgfsys@transformshift{9.780644in}{3.738859in}%
\pgfsys@useobject{currentmarker}{}%
\end{pgfscope}%
\begin{pgfscope}%
\pgfsys@transformshift{9.798237in}{3.708667in}%
\pgfsys@useobject{currentmarker}{}%
\end{pgfscope}%
\begin{pgfscope}%
\pgfsys@transformshift{9.815831in}{3.941227in}%
\pgfsys@useobject{currentmarker}{}%
\end{pgfscope}%
\begin{pgfscope}%
\pgfsys@transformshift{9.833425in}{3.790709in}%
\pgfsys@useobject{currentmarker}{}%
\end{pgfscope}%
\begin{pgfscope}%
\pgfsys@transformshift{9.851019in}{3.609275in}%
\pgfsys@useobject{currentmarker}{}%
\end{pgfscope}%
\begin{pgfscope}%
\pgfsys@transformshift{9.868613in}{3.817591in}%
\pgfsys@useobject{currentmarker}{}%
\end{pgfscope}%
\begin{pgfscope}%
\pgfsys@transformshift{9.886207in}{3.927614in}%
\pgfsys@useobject{currentmarker}{}%
\end{pgfscope}%
\begin{pgfscope}%
\pgfsys@transformshift{9.903801in}{3.805855in}%
\pgfsys@useobject{currentmarker}{}%
\end{pgfscope}%
\begin{pgfscope}%
\pgfsys@transformshift{9.921394in}{3.536787in}%
\pgfsys@useobject{currentmarker}{}%
\end{pgfscope}%
\begin{pgfscope}%
\pgfsys@transformshift{9.938988in}{3.632153in}%
\pgfsys@useobject{currentmarker}{}%
\end{pgfscope}%
\end{pgfscope}%
\begin{pgfscope}%
\pgfsetbuttcap%
\pgfsetroundjoin%
\definecolor{currentfill}{rgb}{0.000000,0.000000,0.000000}%
\pgfsetfillcolor{currentfill}%
\pgfsetlinewidth{0.803000pt}%
\definecolor{currentstroke}{rgb}{0.000000,0.000000,0.000000}%
\pgfsetstrokecolor{currentstroke}%
\pgfsetdash{}{0pt}%
\pgfsys@defobject{currentmarker}{\pgfqpoint{0.000000in}{-0.048611in}}{\pgfqpoint{0.000000in}{0.000000in}}{%
\pgfpathmoveto{\pgfqpoint{0.000000in}{0.000000in}}%
\pgfpathlineto{\pgfqpoint{0.000000in}{-0.048611in}}%
\pgfusepath{stroke,fill}%
}%
\begin{pgfscope}%
\pgfsys@transformshift{5.562518in}{3.021449in}%
\pgfsys@useobject{currentmarker}{}%
\end{pgfscope}%
\end{pgfscope}%
\begin{pgfscope}%
\pgfsetbuttcap%
\pgfsetroundjoin%
\definecolor{currentfill}{rgb}{0.000000,0.000000,0.000000}%
\pgfsetfillcolor{currentfill}%
\pgfsetlinewidth{0.803000pt}%
\definecolor{currentstroke}{rgb}{0.000000,0.000000,0.000000}%
\pgfsetstrokecolor{currentstroke}%
\pgfsetdash{}{0pt}%
\pgfsys@defobject{currentmarker}{\pgfqpoint{0.000000in}{-0.048611in}}{\pgfqpoint{0.000000in}{0.000000in}}{%
\pgfpathmoveto{\pgfqpoint{0.000000in}{0.000000in}}%
\pgfpathlineto{\pgfqpoint{0.000000in}{-0.048611in}}%
\pgfusepath{stroke,fill}%
}%
\begin{pgfscope}%
\pgfsys@transformshift{6.437812in}{3.021449in}%
\pgfsys@useobject{currentmarker}{}%
\end{pgfscope}%
\end{pgfscope}%
\begin{pgfscope}%
\pgfsetbuttcap%
\pgfsetroundjoin%
\definecolor{currentfill}{rgb}{0.000000,0.000000,0.000000}%
\pgfsetfillcolor{currentfill}%
\pgfsetlinewidth{0.803000pt}%
\definecolor{currentstroke}{rgb}{0.000000,0.000000,0.000000}%
\pgfsetstrokecolor{currentstroke}%
\pgfsetdash{}{0pt}%
\pgfsys@defobject{currentmarker}{\pgfqpoint{0.000000in}{-0.048611in}}{\pgfqpoint{0.000000in}{0.000000in}}{%
\pgfpathmoveto{\pgfqpoint{0.000000in}{0.000000in}}%
\pgfpathlineto{\pgfqpoint{0.000000in}{-0.048611in}}%
\pgfusepath{stroke,fill}%
}%
\begin{pgfscope}%
\pgfsys@transformshift{7.313106in}{3.021449in}%
\pgfsys@useobject{currentmarker}{}%
\end{pgfscope}%
\end{pgfscope}%
\begin{pgfscope}%
\pgfsetbuttcap%
\pgfsetroundjoin%
\definecolor{currentfill}{rgb}{0.000000,0.000000,0.000000}%
\pgfsetfillcolor{currentfill}%
\pgfsetlinewidth{0.803000pt}%
\definecolor{currentstroke}{rgb}{0.000000,0.000000,0.000000}%
\pgfsetstrokecolor{currentstroke}%
\pgfsetdash{}{0pt}%
\pgfsys@defobject{currentmarker}{\pgfqpoint{0.000000in}{-0.048611in}}{\pgfqpoint{0.000000in}{0.000000in}}{%
\pgfpathmoveto{\pgfqpoint{0.000000in}{0.000000in}}%
\pgfpathlineto{\pgfqpoint{0.000000in}{-0.048611in}}%
\pgfusepath{stroke,fill}%
}%
\begin{pgfscope}%
\pgfsys@transformshift{8.188400in}{3.021449in}%
\pgfsys@useobject{currentmarker}{}%
\end{pgfscope}%
\end{pgfscope}%
\begin{pgfscope}%
\pgfsetbuttcap%
\pgfsetroundjoin%
\definecolor{currentfill}{rgb}{0.000000,0.000000,0.000000}%
\pgfsetfillcolor{currentfill}%
\pgfsetlinewidth{0.803000pt}%
\definecolor{currentstroke}{rgb}{0.000000,0.000000,0.000000}%
\pgfsetstrokecolor{currentstroke}%
\pgfsetdash{}{0pt}%
\pgfsys@defobject{currentmarker}{\pgfqpoint{0.000000in}{-0.048611in}}{\pgfqpoint{0.000000in}{0.000000in}}{%
\pgfpathmoveto{\pgfqpoint{0.000000in}{0.000000in}}%
\pgfpathlineto{\pgfqpoint{0.000000in}{-0.048611in}}%
\pgfusepath{stroke,fill}%
}%
\begin{pgfscope}%
\pgfsys@transformshift{9.063694in}{3.021449in}%
\pgfsys@useobject{currentmarker}{}%
\end{pgfscope}%
\end{pgfscope}%
\begin{pgfscope}%
\pgfsetbuttcap%
\pgfsetroundjoin%
\definecolor{currentfill}{rgb}{0.000000,0.000000,0.000000}%
\pgfsetfillcolor{currentfill}%
\pgfsetlinewidth{0.803000pt}%
\definecolor{currentstroke}{rgb}{0.000000,0.000000,0.000000}%
\pgfsetstrokecolor{currentstroke}%
\pgfsetdash{}{0pt}%
\pgfsys@defobject{currentmarker}{\pgfqpoint{0.000000in}{-0.048611in}}{\pgfqpoint{0.000000in}{0.000000in}}{%
\pgfpathmoveto{\pgfqpoint{0.000000in}{0.000000in}}%
\pgfpathlineto{\pgfqpoint{0.000000in}{-0.048611in}}%
\pgfusepath{stroke,fill}%
}%
\begin{pgfscope}%
\pgfsys@transformshift{9.938988in}{3.021449in}%
\pgfsys@useobject{currentmarker}{}%
\end{pgfscope}%
\end{pgfscope}%
\begin{pgfscope}%
\pgfsetbuttcap%
\pgfsetroundjoin%
\definecolor{currentfill}{rgb}{0.000000,0.000000,0.000000}%
\pgfsetfillcolor{currentfill}%
\pgfsetlinewidth{0.803000pt}%
\definecolor{currentstroke}{rgb}{0.000000,0.000000,0.000000}%
\pgfsetstrokecolor{currentstroke}%
\pgfsetdash{}{0pt}%
\pgfsys@defobject{currentmarker}{\pgfqpoint{-0.048611in}{0.000000in}}{\pgfqpoint{0.000000in}{0.000000in}}{%
\pgfpathmoveto{\pgfqpoint{0.000000in}{0.000000in}}%
\pgfpathlineto{\pgfqpoint{-0.048611in}{0.000000in}}%
\pgfusepath{stroke,fill}%
}%
\begin{pgfscope}%
\pgfsys@transformshift{5.562518in}{3.386186in}%
\pgfsys@useobject{currentmarker}{}%
\end{pgfscope}%
\end{pgfscope}%
\begin{pgfscope}%
\pgftext[x=5.395851in,y=3.333425in,left,base]{\rmfamily\fontsize{10.000000}{12.000000}\selectfont \(\displaystyle 0\)}%
\end{pgfscope}%
\begin{pgfscope}%
\pgfsetbuttcap%
\pgfsetroundjoin%
\definecolor{currentfill}{rgb}{0.000000,0.000000,0.000000}%
\pgfsetfillcolor{currentfill}%
\pgfsetlinewidth{0.803000pt}%
\definecolor{currentstroke}{rgb}{0.000000,0.000000,0.000000}%
\pgfsetstrokecolor{currentstroke}%
\pgfsetdash{}{0pt}%
\pgfsys@defobject{currentmarker}{\pgfqpoint{-0.048611in}{0.000000in}}{\pgfqpoint{0.000000in}{0.000000in}}{%
\pgfpathmoveto{\pgfqpoint{0.000000in}{0.000000in}}%
\pgfpathlineto{\pgfqpoint{-0.048611in}{0.000000in}}%
\pgfusepath{stroke,fill}%
}%
\begin{pgfscope}%
\pgfsys@transformshift{5.562518in}{3.791449in}%
\pgfsys@useobject{currentmarker}{}%
\end{pgfscope}%
\end{pgfscope}%
\begin{pgfscope}%
\pgftext[x=5.395851in,y=3.738688in,left,base]{\rmfamily\fontsize{10.000000}{12.000000}\selectfont \(\displaystyle 2\)}%
\end{pgfscope}%
\begin{pgfscope}%
\pgfpathrectangle{\pgfqpoint{5.562518in}{3.021449in}}{\pgfqpoint{4.376471in}{0.972632in}}%
\pgfusepath{clip}%
\pgfsetrectcap%
\pgfsetroundjoin%
\pgfsetlinewidth{1.505625pt}%
\definecolor{currentstroke}{rgb}{0.121569,0.466667,0.705882}%
\pgfsetstrokecolor{currentstroke}%
\pgfsetdash{}{0pt}%
\pgfpathmoveto{\pgfqpoint{6.437812in}{3.386186in}}%
\pgfpathlineto{\pgfqpoint{9.938988in}{3.386186in}}%
\pgfpathlineto{\pgfqpoint{9.938988in}{3.386186in}}%
\pgfusepath{stroke}%
\end{pgfscope}%
\begin{pgfscope}%
\pgfsetrectcap%
\pgfsetmiterjoin%
\pgfsetlinewidth{0.803000pt}%
\definecolor{currentstroke}{rgb}{0.000000,0.000000,0.000000}%
\pgfsetstrokecolor{currentstroke}%
\pgfsetdash{}{0pt}%
\pgfpathmoveto{\pgfqpoint{5.562518in}{3.021449in}}%
\pgfpathlineto{\pgfqpoint{5.562518in}{3.994081in}}%
\pgfusepath{stroke}%
\end{pgfscope}%
\begin{pgfscope}%
\pgfsetrectcap%
\pgfsetmiterjoin%
\pgfsetlinewidth{0.803000pt}%
\definecolor{currentstroke}{rgb}{0.000000,0.000000,0.000000}%
\pgfsetstrokecolor{currentstroke}%
\pgfsetdash{}{0pt}%
\pgfpathmoveto{\pgfqpoint{9.938988in}{3.021449in}}%
\pgfpathlineto{\pgfqpoint{9.938988in}{3.994081in}}%
\pgfusepath{stroke}%
\end{pgfscope}%
\begin{pgfscope}%
\pgfsetrectcap%
\pgfsetmiterjoin%
\pgfsetlinewidth{0.803000pt}%
\definecolor{currentstroke}{rgb}{0.000000,0.000000,0.000000}%
\pgfsetstrokecolor{currentstroke}%
\pgfsetdash{}{0pt}%
\pgfpathmoveto{\pgfqpoint{5.562518in}{3.021449in}}%
\pgfpathlineto{\pgfqpoint{9.938988in}{3.021449in}}%
\pgfusepath{stroke}%
\end{pgfscope}%
\begin{pgfscope}%
\pgfsetrectcap%
\pgfsetmiterjoin%
\pgfsetlinewidth{0.803000pt}%
\definecolor{currentstroke}{rgb}{0.000000,0.000000,0.000000}%
\pgfsetstrokecolor{currentstroke}%
\pgfsetdash{}{0pt}%
\pgfpathmoveto{\pgfqpoint{5.562518in}{3.994081in}}%
\pgfpathlineto{\pgfqpoint{9.938988in}{3.994081in}}%
\pgfusepath{stroke}%
\end{pgfscope}%
\begin{pgfscope}%
\pgfsetbuttcap%
\pgfsetmiterjoin%
\definecolor{currentfill}{rgb}{1.000000,1.000000,1.000000}%
\pgfsetfillcolor{currentfill}%
\pgfsetfillopacity{0.800000}%
\pgfsetlinewidth{1.003750pt}%
\definecolor{currentstroke}{rgb}{0.800000,0.800000,0.800000}%
\pgfsetstrokecolor{currentstroke}%
\pgfsetstrokeopacity{0.800000}%
\pgfsetdash{}{0pt}%
\pgfpathmoveto{\pgfqpoint{5.659740in}{3.090894in}}%
\pgfpathlineto{\pgfqpoint{6.443065in}{3.090894in}}%
\pgfpathquadraticcurveto{\pgfqpoint{6.470843in}{3.090894in}}{\pgfqpoint{6.470843in}{3.118671in}}%
\pgfpathlineto{\pgfqpoint{6.470843in}{3.717358in}}%
\pgfpathquadraticcurveto{\pgfqpoint{6.470843in}{3.745136in}}{\pgfqpoint{6.443065in}{3.745136in}}%
\pgfpathlineto{\pgfqpoint{5.659740in}{3.745136in}}%
\pgfpathquadraticcurveto{\pgfqpoint{5.631962in}{3.745136in}}{\pgfqpoint{5.631962in}{3.717358in}}%
\pgfpathlineto{\pgfqpoint{5.631962in}{3.118671in}}%
\pgfpathquadraticcurveto{\pgfqpoint{5.631962in}{3.090894in}}{\pgfqpoint{5.659740in}{3.090894in}}%
\pgfpathclose%
\pgfusepath{stroke,fill}%
\end{pgfscope}%
\begin{pgfscope}%
\pgfsetrectcap%
\pgfsetroundjoin%
\pgfsetlinewidth{1.505625pt}%
\definecolor{currentstroke}{rgb}{0.121569,0.466667,0.705882}%
\pgfsetstrokecolor{currentstroke}%
\pgfsetdash{}{0pt}%
\pgfpathmoveto{\pgfqpoint{5.687518in}{3.631665in}}%
\pgfpathlineto{\pgfqpoint{5.965295in}{3.631665in}}%
\pgfusepath{stroke}%
\end{pgfscope}%
\begin{pgfscope}%
\pgftext[x=6.076407in,y=3.583053in,left,base]{\rmfamily\fontsize{10.000000}{12.000000}\selectfont \(\displaystyle \widetilde{\Phi}^* \theta^{\parallel}\)}%
\end{pgfscope}%
\begin{pgfscope}%
\pgfsetbuttcap%
\pgfsetroundjoin%
\definecolor{currentfill}{rgb}{1.000000,0.000000,0.000000}%
\pgfsetfillcolor{currentfill}%
\pgfsetlinewidth{2.007500pt}%
\definecolor{currentstroke}{rgb}{1.000000,0.000000,0.000000}%
\pgfsetstrokecolor{currentstroke}%
\pgfsetdash{}{0pt}%
\pgfpathmoveto{\pgfqpoint{5.784740in}{3.415655in}}%
\pgfpathlineto{\pgfqpoint{5.868073in}{3.415655in}}%
\pgfpathmoveto{\pgfqpoint{5.826407in}{3.373988in}}%
\pgfpathlineto{\pgfqpoint{5.826407in}{3.457321in}}%
\pgfusepath{stroke,fill}%
\end{pgfscope}%
\begin{pgfscope}%
\pgftext[x=6.076407in,y=3.379196in,left,base]{\rmfamily\fontsize{10.000000}{12.000000}\selectfont train}%
\end{pgfscope}%
\begin{pgfscope}%
\pgfsetbuttcap%
\pgfsetroundjoin%
\definecolor{currentfill}{rgb}{0.000000,0.000000,0.000000}%
\pgfsetfillcolor{currentfill}%
\pgfsetlinewidth{1.003750pt}%
\definecolor{currentstroke}{rgb}{0.000000,0.000000,0.000000}%
\pgfsetstrokecolor{currentstroke}%
\pgfsetdash{}{0pt}%
\pgfsys@defobject{currentmarker}{\pgfqpoint{-0.020833in}{-0.020833in}}{\pgfqpoint{0.020833in}{0.020833in}}{%
\pgfpathmoveto{\pgfqpoint{0.000000in}{-0.020833in}}%
\pgfpathcurveto{\pgfqpoint{0.005525in}{-0.020833in}}{\pgfqpoint{0.010825in}{-0.018638in}}{\pgfqpoint{0.014731in}{-0.014731in}}%
\pgfpathcurveto{\pgfqpoint{0.018638in}{-0.010825in}}{\pgfqpoint{0.020833in}{-0.005525in}}{\pgfqpoint{0.020833in}{0.000000in}}%
\pgfpathcurveto{\pgfqpoint{0.020833in}{0.005525in}}{\pgfqpoint{0.018638in}{0.010825in}}{\pgfqpoint{0.014731in}{0.014731in}}%
\pgfpathcurveto{\pgfqpoint{0.010825in}{0.018638in}}{\pgfqpoint{0.005525in}{0.020833in}}{\pgfqpoint{0.000000in}{0.020833in}}%
\pgfpathcurveto{\pgfqpoint{-0.005525in}{0.020833in}}{\pgfqpoint{-0.010825in}{0.018638in}}{\pgfqpoint{-0.014731in}{0.014731in}}%
\pgfpathcurveto{\pgfqpoint{-0.018638in}{0.010825in}}{\pgfqpoint{-0.020833in}{0.005525in}}{\pgfqpoint{-0.020833in}{0.000000in}}%
\pgfpathcurveto{\pgfqpoint{-0.020833in}{-0.005525in}}{\pgfqpoint{-0.018638in}{-0.010825in}}{\pgfqpoint{-0.014731in}{-0.014731in}}%
\pgfpathcurveto{\pgfqpoint{-0.010825in}{-0.018638in}}{\pgfqpoint{-0.005525in}{-0.020833in}}{\pgfqpoint{0.000000in}{-0.020833in}}%
\pgfpathclose%
\pgfusepath{stroke,fill}%
}%
\begin{pgfscope}%
\pgfsys@transformshift{5.826407in}{3.211797in}%
\pgfsys@useobject{currentmarker}{}%
\end{pgfscope}%
\end{pgfscope}%
\begin{pgfscope}%
\pgftext[x=6.076407in,y=3.175339in,left,base]{\rmfamily\fontsize{10.000000}{12.000000}\selectfont test}%
\end{pgfscope}%
\begin{pgfscope}%
\pgfsetbuttcap%
\pgfsetmiterjoin%
\definecolor{currentfill}{rgb}{1.000000,1.000000,1.000000}%
\pgfsetfillcolor{currentfill}%
\pgfsetlinewidth{0.000000pt}%
\definecolor{currentstroke}{rgb}{0.000000,0.000000,0.000000}%
\pgfsetstrokecolor{currentstroke}%
\pgfsetstrokeopacity{0.000000}%
\pgfsetdash{}{0pt}%
\pgfpathmoveto{\pgfqpoint{10.668400in}{3.021449in}}%
\pgfpathlineto{\pgfqpoint{12.856635in}{3.021449in}}%
\pgfpathlineto{\pgfqpoint{12.856635in}{3.994081in}}%
\pgfpathlineto{\pgfqpoint{10.668400in}{3.994081in}}%
\pgfpathclose%
\pgfusepath{fill}%
\end{pgfscope}%
\begin{pgfscope}%
\pgfpathrectangle{\pgfqpoint{10.668400in}{3.021449in}}{\pgfqpoint{2.188235in}{0.972632in}}%
\pgfusepath{clip}%
\pgfsetbuttcap%
\pgfsetmiterjoin%
\definecolor{currentfill}{rgb}{0.121569,0.466667,0.705882}%
\pgfsetfillcolor{currentfill}%
\pgfsetlinewidth{0.000000pt}%
\definecolor{currentstroke}{rgb}{0.000000,0.000000,0.000000}%
\pgfsetstrokecolor{currentstroke}%
\pgfsetstrokeopacity{0.000000}%
\pgfsetdash{}{0pt}%
\pgfpathmoveto{\pgfqpoint{-319.224843in}{3.065660in}}%
\pgfpathlineto{\pgfqpoint{9.276031in}{3.065660in}}%
\pgfpathlineto{\pgfqpoint{9.276031in}{3.072748in}}%
\pgfpathlineto{\pgfqpoint{-319.224843in}{3.072748in}}%
\pgfpathclose%
\pgfusepath{fill}%
\end{pgfscope}%
\begin{pgfscope}%
\pgfpathrectangle{\pgfqpoint{10.668400in}{3.021449in}}{\pgfqpoint{2.188235in}{0.972632in}}%
\pgfusepath{clip}%
\pgfsetbuttcap%
\pgfsetmiterjoin%
\definecolor{currentfill}{rgb}{0.121569,0.466667,0.705882}%
\pgfsetfillcolor{currentfill}%
\pgfsetlinewidth{0.000000pt}%
\definecolor{currentstroke}{rgb}{0.000000,0.000000,0.000000}%
\pgfsetstrokecolor{currentstroke}%
\pgfsetstrokeopacity{0.000000}%
\pgfsetdash{}{0pt}%
\pgfpathmoveto{\pgfqpoint{-319.224843in}{3.074520in}}%
\pgfpathlineto{\pgfqpoint{9.234319in}{3.074520in}}%
\pgfpathlineto{\pgfqpoint{9.234319in}{3.081607in}}%
\pgfpathlineto{\pgfqpoint{-319.224843in}{3.081607in}}%
\pgfpathclose%
\pgfusepath{fill}%
\end{pgfscope}%
\begin{pgfscope}%
\pgfpathrectangle{\pgfqpoint{10.668400in}{3.021449in}}{\pgfqpoint{2.188235in}{0.972632in}}%
\pgfusepath{clip}%
\pgfsetbuttcap%
\pgfsetmiterjoin%
\definecolor{currentfill}{rgb}{0.121569,0.466667,0.705882}%
\pgfsetfillcolor{currentfill}%
\pgfsetlinewidth{0.000000pt}%
\definecolor{currentstroke}{rgb}{0.000000,0.000000,0.000000}%
\pgfsetstrokecolor{currentstroke}%
\pgfsetstrokeopacity{0.000000}%
\pgfsetdash{}{0pt}%
\pgfpathmoveto{\pgfqpoint{-319.224843in}{3.083379in}}%
\pgfpathlineto{\pgfqpoint{9.143664in}{3.083379in}}%
\pgfpathlineto{\pgfqpoint{9.143664in}{3.090467in}}%
\pgfpathlineto{\pgfqpoint{-319.224843in}{3.090467in}}%
\pgfpathclose%
\pgfusepath{fill}%
\end{pgfscope}%
\begin{pgfscope}%
\pgfpathrectangle{\pgfqpoint{10.668400in}{3.021449in}}{\pgfqpoint{2.188235in}{0.972632in}}%
\pgfusepath{clip}%
\pgfsetbuttcap%
\pgfsetmiterjoin%
\definecolor{currentfill}{rgb}{0.121569,0.466667,0.705882}%
\pgfsetfillcolor{currentfill}%
\pgfsetlinewidth{0.000000pt}%
\definecolor{currentstroke}{rgb}{0.000000,0.000000,0.000000}%
\pgfsetstrokecolor{currentstroke}%
\pgfsetstrokeopacity{0.000000}%
\pgfsetdash{}{0pt}%
\pgfpathmoveto{\pgfqpoint{-319.224843in}{3.092239in}}%
\pgfpathlineto{\pgfqpoint{9.252475in}{3.092239in}}%
\pgfpathlineto{\pgfqpoint{9.252475in}{3.099327in}}%
\pgfpathlineto{\pgfqpoint{-319.224843in}{3.099327in}}%
\pgfpathclose%
\pgfusepath{fill}%
\end{pgfscope}%
\begin{pgfscope}%
\pgfpathrectangle{\pgfqpoint{10.668400in}{3.021449in}}{\pgfqpoint{2.188235in}{0.972632in}}%
\pgfusepath{clip}%
\pgfsetbuttcap%
\pgfsetmiterjoin%
\definecolor{currentfill}{rgb}{0.121569,0.466667,0.705882}%
\pgfsetfillcolor{currentfill}%
\pgfsetlinewidth{0.000000pt}%
\definecolor{currentstroke}{rgb}{0.000000,0.000000,0.000000}%
\pgfsetstrokecolor{currentstroke}%
\pgfsetstrokeopacity{0.000000}%
\pgfsetdash{}{0pt}%
\pgfpathmoveto{\pgfqpoint{-319.224843in}{3.101099in}}%
\pgfpathlineto{\pgfqpoint{9.372085in}{3.101099in}}%
\pgfpathlineto{\pgfqpoint{9.372085in}{3.108187in}}%
\pgfpathlineto{\pgfqpoint{-319.224843in}{3.108187in}}%
\pgfpathclose%
\pgfusepath{fill}%
\end{pgfscope}%
\begin{pgfscope}%
\pgfpathrectangle{\pgfqpoint{10.668400in}{3.021449in}}{\pgfqpoint{2.188235in}{0.972632in}}%
\pgfusepath{clip}%
\pgfsetbuttcap%
\pgfsetmiterjoin%
\definecolor{currentfill}{rgb}{0.121569,0.466667,0.705882}%
\pgfsetfillcolor{currentfill}%
\pgfsetlinewidth{0.000000pt}%
\definecolor{currentstroke}{rgb}{0.000000,0.000000,0.000000}%
\pgfsetstrokecolor{currentstroke}%
\pgfsetstrokeopacity{0.000000}%
\pgfsetdash{}{0pt}%
\pgfpathmoveto{\pgfqpoint{-319.224843in}{3.109959in}}%
\pgfpathlineto{\pgfqpoint{9.207251in}{3.109959in}}%
\pgfpathlineto{\pgfqpoint{9.207251in}{3.117047in}}%
\pgfpathlineto{\pgfqpoint{-319.224843in}{3.117047in}}%
\pgfpathclose%
\pgfusepath{fill}%
\end{pgfscope}%
\begin{pgfscope}%
\pgfpathrectangle{\pgfqpoint{10.668400in}{3.021449in}}{\pgfqpoint{2.188235in}{0.972632in}}%
\pgfusepath{clip}%
\pgfsetbuttcap%
\pgfsetmiterjoin%
\definecolor{currentfill}{rgb}{0.121569,0.466667,0.705882}%
\pgfsetfillcolor{currentfill}%
\pgfsetlinewidth{0.000000pt}%
\definecolor{currentstroke}{rgb}{0.000000,0.000000,0.000000}%
\pgfsetstrokecolor{currentstroke}%
\pgfsetstrokeopacity{0.000000}%
\pgfsetdash{}{0pt}%
\pgfpathmoveto{\pgfqpoint{-319.224843in}{3.118819in}}%
\pgfpathlineto{\pgfqpoint{8.888978in}{3.118819in}}%
\pgfpathlineto{\pgfqpoint{8.888978in}{3.125907in}}%
\pgfpathlineto{\pgfqpoint{-319.224843in}{3.125907in}}%
\pgfpathclose%
\pgfusepath{fill}%
\end{pgfscope}%
\begin{pgfscope}%
\pgfpathrectangle{\pgfqpoint{10.668400in}{3.021449in}}{\pgfqpoint{2.188235in}{0.972632in}}%
\pgfusepath{clip}%
\pgfsetbuttcap%
\pgfsetmiterjoin%
\definecolor{currentfill}{rgb}{0.121569,0.466667,0.705882}%
\pgfsetfillcolor{currentfill}%
\pgfsetlinewidth{0.000000pt}%
\definecolor{currentstroke}{rgb}{0.000000,0.000000,0.000000}%
\pgfsetstrokecolor{currentstroke}%
\pgfsetstrokeopacity{0.000000}%
\pgfsetdash{}{0pt}%
\pgfpathmoveto{\pgfqpoint{-319.224843in}{3.127679in}}%
\pgfpathlineto{\pgfqpoint{9.278776in}{3.127679in}}%
\pgfpathlineto{\pgfqpoint{9.278776in}{3.134766in}}%
\pgfpathlineto{\pgfqpoint{-319.224843in}{3.134766in}}%
\pgfpathclose%
\pgfusepath{fill}%
\end{pgfscope}%
\begin{pgfscope}%
\pgfpathrectangle{\pgfqpoint{10.668400in}{3.021449in}}{\pgfqpoint{2.188235in}{0.972632in}}%
\pgfusepath{clip}%
\pgfsetbuttcap%
\pgfsetmiterjoin%
\definecolor{currentfill}{rgb}{0.121569,0.466667,0.705882}%
\pgfsetfillcolor{currentfill}%
\pgfsetlinewidth{0.000000pt}%
\definecolor{currentstroke}{rgb}{0.000000,0.000000,0.000000}%
\pgfsetstrokecolor{currentstroke}%
\pgfsetstrokeopacity{0.000000}%
\pgfsetdash{}{0pt}%
\pgfpathmoveto{\pgfqpoint{-319.224843in}{3.136538in}}%
\pgfpathlineto{\pgfqpoint{9.284914in}{3.136538in}}%
\pgfpathlineto{\pgfqpoint{9.284914in}{3.143626in}}%
\pgfpathlineto{\pgfqpoint{-319.224843in}{3.143626in}}%
\pgfpathclose%
\pgfusepath{fill}%
\end{pgfscope}%
\begin{pgfscope}%
\pgfpathrectangle{\pgfqpoint{10.668400in}{3.021449in}}{\pgfqpoint{2.188235in}{0.972632in}}%
\pgfusepath{clip}%
\pgfsetbuttcap%
\pgfsetmiterjoin%
\definecolor{currentfill}{rgb}{0.121569,0.466667,0.705882}%
\pgfsetfillcolor{currentfill}%
\pgfsetlinewidth{0.000000pt}%
\definecolor{currentstroke}{rgb}{0.000000,0.000000,0.000000}%
\pgfsetstrokecolor{currentstroke}%
\pgfsetstrokeopacity{0.000000}%
\pgfsetdash{}{0pt}%
\pgfpathmoveto{\pgfqpoint{-319.224843in}{3.145398in}}%
\pgfpathlineto{\pgfqpoint{9.214781in}{3.145398in}}%
\pgfpathlineto{\pgfqpoint{9.214781in}{3.152486in}}%
\pgfpathlineto{\pgfqpoint{-319.224843in}{3.152486in}}%
\pgfpathclose%
\pgfusepath{fill}%
\end{pgfscope}%
\begin{pgfscope}%
\pgfpathrectangle{\pgfqpoint{10.668400in}{3.021449in}}{\pgfqpoint{2.188235in}{0.972632in}}%
\pgfusepath{clip}%
\pgfsetbuttcap%
\pgfsetmiterjoin%
\definecolor{currentfill}{rgb}{0.121569,0.466667,0.705882}%
\pgfsetfillcolor{currentfill}%
\pgfsetlinewidth{0.000000pt}%
\definecolor{currentstroke}{rgb}{0.000000,0.000000,0.000000}%
\pgfsetstrokecolor{currentstroke}%
\pgfsetstrokeopacity{0.000000}%
\pgfsetdash{}{0pt}%
\pgfpathmoveto{\pgfqpoint{-319.224843in}{3.154258in}}%
\pgfpathlineto{\pgfqpoint{8.993670in}{3.154258in}}%
\pgfpathlineto{\pgfqpoint{8.993670in}{3.161346in}}%
\pgfpathlineto{\pgfqpoint{-319.224843in}{3.161346in}}%
\pgfpathclose%
\pgfusepath{fill}%
\end{pgfscope}%
\begin{pgfscope}%
\pgfpathrectangle{\pgfqpoint{10.668400in}{3.021449in}}{\pgfqpoint{2.188235in}{0.972632in}}%
\pgfusepath{clip}%
\pgfsetbuttcap%
\pgfsetmiterjoin%
\definecolor{currentfill}{rgb}{0.121569,0.466667,0.705882}%
\pgfsetfillcolor{currentfill}%
\pgfsetlinewidth{0.000000pt}%
\definecolor{currentstroke}{rgb}{0.000000,0.000000,0.000000}%
\pgfsetstrokecolor{currentstroke}%
\pgfsetstrokeopacity{0.000000}%
\pgfsetdash{}{0pt}%
\pgfpathmoveto{\pgfqpoint{-319.224843in}{3.163118in}}%
\pgfpathlineto{\pgfqpoint{9.174933in}{3.163118in}}%
\pgfpathlineto{\pgfqpoint{9.174933in}{3.170206in}}%
\pgfpathlineto{\pgfqpoint{-319.224843in}{3.170206in}}%
\pgfpathclose%
\pgfusepath{fill}%
\end{pgfscope}%
\begin{pgfscope}%
\pgfpathrectangle{\pgfqpoint{10.668400in}{3.021449in}}{\pgfqpoint{2.188235in}{0.972632in}}%
\pgfusepath{clip}%
\pgfsetbuttcap%
\pgfsetmiterjoin%
\definecolor{currentfill}{rgb}{0.121569,0.466667,0.705882}%
\pgfsetfillcolor{currentfill}%
\pgfsetlinewidth{0.000000pt}%
\definecolor{currentstroke}{rgb}{0.000000,0.000000,0.000000}%
\pgfsetstrokecolor{currentstroke}%
\pgfsetstrokeopacity{0.000000}%
\pgfsetdash{}{0pt}%
\pgfpathmoveto{\pgfqpoint{-319.224843in}{3.171978in}}%
\pgfpathlineto{\pgfqpoint{9.163080in}{3.171978in}}%
\pgfpathlineto{\pgfqpoint{9.163080in}{3.179066in}}%
\pgfpathlineto{\pgfqpoint{-319.224843in}{3.179066in}}%
\pgfpathclose%
\pgfusepath{fill}%
\end{pgfscope}%
\begin{pgfscope}%
\pgfpathrectangle{\pgfqpoint{10.668400in}{3.021449in}}{\pgfqpoint{2.188235in}{0.972632in}}%
\pgfusepath{clip}%
\pgfsetbuttcap%
\pgfsetmiterjoin%
\definecolor{currentfill}{rgb}{0.121569,0.466667,0.705882}%
\pgfsetfillcolor{currentfill}%
\pgfsetlinewidth{0.000000pt}%
\definecolor{currentstroke}{rgb}{0.000000,0.000000,0.000000}%
\pgfsetstrokecolor{currentstroke}%
\pgfsetstrokeopacity{0.000000}%
\pgfsetdash{}{0pt}%
\pgfpathmoveto{\pgfqpoint{-319.224843in}{3.180837in}}%
\pgfpathlineto{\pgfqpoint{9.234722in}{3.180837in}}%
\pgfpathlineto{\pgfqpoint{9.234722in}{3.187925in}}%
\pgfpathlineto{\pgfqpoint{-319.224843in}{3.187925in}}%
\pgfpathclose%
\pgfusepath{fill}%
\end{pgfscope}%
\begin{pgfscope}%
\pgfpathrectangle{\pgfqpoint{10.668400in}{3.021449in}}{\pgfqpoint{2.188235in}{0.972632in}}%
\pgfusepath{clip}%
\pgfsetbuttcap%
\pgfsetmiterjoin%
\definecolor{currentfill}{rgb}{0.121569,0.466667,0.705882}%
\pgfsetfillcolor{currentfill}%
\pgfsetlinewidth{0.000000pt}%
\definecolor{currentstroke}{rgb}{0.000000,0.000000,0.000000}%
\pgfsetstrokecolor{currentstroke}%
\pgfsetstrokeopacity{0.000000}%
\pgfsetdash{}{0pt}%
\pgfpathmoveto{\pgfqpoint{-319.224843in}{3.189697in}}%
\pgfpathlineto{\pgfqpoint{8.961384in}{3.189697in}}%
\pgfpathlineto{\pgfqpoint{8.961384in}{3.196785in}}%
\pgfpathlineto{\pgfqpoint{-319.224843in}{3.196785in}}%
\pgfpathclose%
\pgfusepath{fill}%
\end{pgfscope}%
\begin{pgfscope}%
\pgfpathrectangle{\pgfqpoint{10.668400in}{3.021449in}}{\pgfqpoint{2.188235in}{0.972632in}}%
\pgfusepath{clip}%
\pgfsetbuttcap%
\pgfsetmiterjoin%
\definecolor{currentfill}{rgb}{0.121569,0.466667,0.705882}%
\pgfsetfillcolor{currentfill}%
\pgfsetlinewidth{0.000000pt}%
\definecolor{currentstroke}{rgb}{0.000000,0.000000,0.000000}%
\pgfsetstrokecolor{currentstroke}%
\pgfsetstrokeopacity{0.000000}%
\pgfsetdash{}{0pt}%
\pgfpathmoveto{\pgfqpoint{-319.224843in}{3.198557in}}%
\pgfpathlineto{\pgfqpoint{9.050643in}{3.198557in}}%
\pgfpathlineto{\pgfqpoint{9.050643in}{3.205645in}}%
\pgfpathlineto{\pgfqpoint{-319.224843in}{3.205645in}}%
\pgfpathclose%
\pgfusepath{fill}%
\end{pgfscope}%
\begin{pgfscope}%
\pgfpathrectangle{\pgfqpoint{10.668400in}{3.021449in}}{\pgfqpoint{2.188235in}{0.972632in}}%
\pgfusepath{clip}%
\pgfsetbuttcap%
\pgfsetmiterjoin%
\definecolor{currentfill}{rgb}{0.121569,0.466667,0.705882}%
\pgfsetfillcolor{currentfill}%
\pgfsetlinewidth{0.000000pt}%
\definecolor{currentstroke}{rgb}{0.000000,0.000000,0.000000}%
\pgfsetstrokecolor{currentstroke}%
\pgfsetstrokeopacity{0.000000}%
\pgfsetdash{}{0pt}%
\pgfpathmoveto{\pgfqpoint{-319.224843in}{3.207417in}}%
\pgfpathlineto{\pgfqpoint{9.022211in}{3.207417in}}%
\pgfpathlineto{\pgfqpoint{9.022211in}{3.214505in}}%
\pgfpathlineto{\pgfqpoint{-319.224843in}{3.214505in}}%
\pgfpathclose%
\pgfusepath{fill}%
\end{pgfscope}%
\begin{pgfscope}%
\pgfpathrectangle{\pgfqpoint{10.668400in}{3.021449in}}{\pgfqpoint{2.188235in}{0.972632in}}%
\pgfusepath{clip}%
\pgfsetbuttcap%
\pgfsetmiterjoin%
\definecolor{currentfill}{rgb}{0.121569,0.466667,0.705882}%
\pgfsetfillcolor{currentfill}%
\pgfsetlinewidth{0.000000pt}%
\definecolor{currentstroke}{rgb}{0.000000,0.000000,0.000000}%
\pgfsetstrokecolor{currentstroke}%
\pgfsetstrokeopacity{0.000000}%
\pgfsetdash{}{0pt}%
\pgfpathmoveto{\pgfqpoint{-319.224843in}{3.216277in}}%
\pgfpathlineto{\pgfqpoint{8.937755in}{3.216277in}}%
\pgfpathlineto{\pgfqpoint{8.937755in}{3.223365in}}%
\pgfpathlineto{\pgfqpoint{-319.224843in}{3.223365in}}%
\pgfpathclose%
\pgfusepath{fill}%
\end{pgfscope}%
\begin{pgfscope}%
\pgfpathrectangle{\pgfqpoint{10.668400in}{3.021449in}}{\pgfqpoint{2.188235in}{0.972632in}}%
\pgfusepath{clip}%
\pgfsetbuttcap%
\pgfsetmiterjoin%
\definecolor{currentfill}{rgb}{0.121569,0.466667,0.705882}%
\pgfsetfillcolor{currentfill}%
\pgfsetlinewidth{0.000000pt}%
\definecolor{currentstroke}{rgb}{0.000000,0.000000,0.000000}%
\pgfsetstrokecolor{currentstroke}%
\pgfsetstrokeopacity{0.000000}%
\pgfsetdash{}{0pt}%
\pgfpathmoveto{\pgfqpoint{-319.224843in}{3.225137in}}%
\pgfpathlineto{\pgfqpoint{8.994816in}{3.225137in}}%
\pgfpathlineto{\pgfqpoint{8.994816in}{3.232224in}}%
\pgfpathlineto{\pgfqpoint{-319.224843in}{3.232224in}}%
\pgfpathclose%
\pgfusepath{fill}%
\end{pgfscope}%
\begin{pgfscope}%
\pgfpathrectangle{\pgfqpoint{10.668400in}{3.021449in}}{\pgfqpoint{2.188235in}{0.972632in}}%
\pgfusepath{clip}%
\pgfsetbuttcap%
\pgfsetmiterjoin%
\definecolor{currentfill}{rgb}{0.121569,0.466667,0.705882}%
\pgfsetfillcolor{currentfill}%
\pgfsetlinewidth{0.000000pt}%
\definecolor{currentstroke}{rgb}{0.000000,0.000000,0.000000}%
\pgfsetstrokecolor{currentstroke}%
\pgfsetstrokeopacity{0.000000}%
\pgfsetdash{}{0pt}%
\pgfpathmoveto{\pgfqpoint{-319.224843in}{3.233996in}}%
\pgfpathlineto{\pgfqpoint{9.019485in}{3.233996in}}%
\pgfpathlineto{\pgfqpoint{9.019485in}{3.241084in}}%
\pgfpathlineto{\pgfqpoint{-319.224843in}{3.241084in}}%
\pgfpathclose%
\pgfusepath{fill}%
\end{pgfscope}%
\begin{pgfscope}%
\pgfpathrectangle{\pgfqpoint{10.668400in}{3.021449in}}{\pgfqpoint{2.188235in}{0.972632in}}%
\pgfusepath{clip}%
\pgfsetbuttcap%
\pgfsetmiterjoin%
\definecolor{currentfill}{rgb}{0.121569,0.466667,0.705882}%
\pgfsetfillcolor{currentfill}%
\pgfsetlinewidth{0.000000pt}%
\definecolor{currentstroke}{rgb}{0.000000,0.000000,0.000000}%
\pgfsetstrokecolor{currentstroke}%
\pgfsetstrokeopacity{0.000000}%
\pgfsetdash{}{0pt}%
\pgfpathmoveto{\pgfqpoint{-319.224843in}{3.242856in}}%
\pgfpathlineto{\pgfqpoint{8.902585in}{3.242856in}}%
\pgfpathlineto{\pgfqpoint{8.902585in}{3.249944in}}%
\pgfpathlineto{\pgfqpoint{-319.224843in}{3.249944in}}%
\pgfpathclose%
\pgfusepath{fill}%
\end{pgfscope}%
\begin{pgfscope}%
\pgfpathrectangle{\pgfqpoint{10.668400in}{3.021449in}}{\pgfqpoint{2.188235in}{0.972632in}}%
\pgfusepath{clip}%
\pgfsetbuttcap%
\pgfsetmiterjoin%
\definecolor{currentfill}{rgb}{0.121569,0.466667,0.705882}%
\pgfsetfillcolor{currentfill}%
\pgfsetlinewidth{0.000000pt}%
\definecolor{currentstroke}{rgb}{0.000000,0.000000,0.000000}%
\pgfsetstrokecolor{currentstroke}%
\pgfsetstrokeopacity{0.000000}%
\pgfsetdash{}{0pt}%
\pgfpathmoveto{\pgfqpoint{-319.224843in}{3.251716in}}%
\pgfpathlineto{\pgfqpoint{9.033999in}{3.251716in}}%
\pgfpathlineto{\pgfqpoint{9.033999in}{3.258804in}}%
\pgfpathlineto{\pgfqpoint{-319.224843in}{3.258804in}}%
\pgfpathclose%
\pgfusepath{fill}%
\end{pgfscope}%
\begin{pgfscope}%
\pgfpathrectangle{\pgfqpoint{10.668400in}{3.021449in}}{\pgfqpoint{2.188235in}{0.972632in}}%
\pgfusepath{clip}%
\pgfsetbuttcap%
\pgfsetmiterjoin%
\definecolor{currentfill}{rgb}{0.121569,0.466667,0.705882}%
\pgfsetfillcolor{currentfill}%
\pgfsetlinewidth{0.000000pt}%
\definecolor{currentstroke}{rgb}{0.000000,0.000000,0.000000}%
\pgfsetstrokecolor{currentstroke}%
\pgfsetstrokeopacity{0.000000}%
\pgfsetdash{}{0pt}%
\pgfpathmoveto{\pgfqpoint{-319.224843in}{3.260576in}}%
\pgfpathlineto{\pgfqpoint{9.134957in}{3.260576in}}%
\pgfpathlineto{\pgfqpoint{9.134957in}{3.267664in}}%
\pgfpathlineto{\pgfqpoint{-319.224843in}{3.267664in}}%
\pgfpathclose%
\pgfusepath{fill}%
\end{pgfscope}%
\begin{pgfscope}%
\pgfpathrectangle{\pgfqpoint{10.668400in}{3.021449in}}{\pgfqpoint{2.188235in}{0.972632in}}%
\pgfusepath{clip}%
\pgfsetbuttcap%
\pgfsetmiterjoin%
\definecolor{currentfill}{rgb}{0.121569,0.466667,0.705882}%
\pgfsetfillcolor{currentfill}%
\pgfsetlinewidth{0.000000pt}%
\definecolor{currentstroke}{rgb}{0.000000,0.000000,0.000000}%
\pgfsetstrokecolor{currentstroke}%
\pgfsetstrokeopacity{0.000000}%
\pgfsetdash{}{0pt}%
\pgfpathmoveto{\pgfqpoint{-319.224843in}{3.269436in}}%
\pgfpathlineto{\pgfqpoint{9.161234in}{3.269436in}}%
\pgfpathlineto{\pgfqpoint{9.161234in}{3.276524in}}%
\pgfpathlineto{\pgfqpoint{-319.224843in}{3.276524in}}%
\pgfpathclose%
\pgfusepath{fill}%
\end{pgfscope}%
\begin{pgfscope}%
\pgfpathrectangle{\pgfqpoint{10.668400in}{3.021449in}}{\pgfqpoint{2.188235in}{0.972632in}}%
\pgfusepath{clip}%
\pgfsetbuttcap%
\pgfsetmiterjoin%
\definecolor{currentfill}{rgb}{0.121569,0.466667,0.705882}%
\pgfsetfillcolor{currentfill}%
\pgfsetlinewidth{0.000000pt}%
\definecolor{currentstroke}{rgb}{0.000000,0.000000,0.000000}%
\pgfsetstrokecolor{currentstroke}%
\pgfsetstrokeopacity{0.000000}%
\pgfsetdash{}{0pt}%
\pgfpathmoveto{\pgfqpoint{-319.224843in}{3.278296in}}%
\pgfpathlineto{\pgfqpoint{9.050415in}{3.278296in}}%
\pgfpathlineto{\pgfqpoint{9.050415in}{3.285383in}}%
\pgfpathlineto{\pgfqpoint{-319.224843in}{3.285383in}}%
\pgfpathclose%
\pgfusepath{fill}%
\end{pgfscope}%
\begin{pgfscope}%
\pgfpathrectangle{\pgfqpoint{10.668400in}{3.021449in}}{\pgfqpoint{2.188235in}{0.972632in}}%
\pgfusepath{clip}%
\pgfsetbuttcap%
\pgfsetmiterjoin%
\definecolor{currentfill}{rgb}{0.121569,0.466667,0.705882}%
\pgfsetfillcolor{currentfill}%
\pgfsetlinewidth{0.000000pt}%
\definecolor{currentstroke}{rgb}{0.000000,0.000000,0.000000}%
\pgfsetstrokecolor{currentstroke}%
\pgfsetstrokeopacity{0.000000}%
\pgfsetdash{}{0pt}%
\pgfpathmoveto{\pgfqpoint{-319.224843in}{3.287155in}}%
\pgfpathlineto{\pgfqpoint{8.975000in}{3.287155in}}%
\pgfpathlineto{\pgfqpoint{8.975000in}{3.294243in}}%
\pgfpathlineto{\pgfqpoint{-319.224843in}{3.294243in}}%
\pgfpathclose%
\pgfusepath{fill}%
\end{pgfscope}%
\begin{pgfscope}%
\pgfpathrectangle{\pgfqpoint{10.668400in}{3.021449in}}{\pgfqpoint{2.188235in}{0.972632in}}%
\pgfusepath{clip}%
\pgfsetbuttcap%
\pgfsetmiterjoin%
\definecolor{currentfill}{rgb}{0.121569,0.466667,0.705882}%
\pgfsetfillcolor{currentfill}%
\pgfsetlinewidth{0.000000pt}%
\definecolor{currentstroke}{rgb}{0.000000,0.000000,0.000000}%
\pgfsetstrokecolor{currentstroke}%
\pgfsetstrokeopacity{0.000000}%
\pgfsetdash{}{0pt}%
\pgfpathmoveto{\pgfqpoint{-319.224843in}{3.296015in}}%
\pgfpathlineto{\pgfqpoint{9.173948in}{3.296015in}}%
\pgfpathlineto{\pgfqpoint{9.173948in}{3.303103in}}%
\pgfpathlineto{\pgfqpoint{-319.224843in}{3.303103in}}%
\pgfpathclose%
\pgfusepath{fill}%
\end{pgfscope}%
\begin{pgfscope}%
\pgfpathrectangle{\pgfqpoint{10.668400in}{3.021449in}}{\pgfqpoint{2.188235in}{0.972632in}}%
\pgfusepath{clip}%
\pgfsetbuttcap%
\pgfsetmiterjoin%
\definecolor{currentfill}{rgb}{0.121569,0.466667,0.705882}%
\pgfsetfillcolor{currentfill}%
\pgfsetlinewidth{0.000000pt}%
\definecolor{currentstroke}{rgb}{0.000000,0.000000,0.000000}%
\pgfsetstrokecolor{currentstroke}%
\pgfsetstrokeopacity{0.000000}%
\pgfsetdash{}{0pt}%
\pgfpathmoveto{\pgfqpoint{-319.224843in}{3.304875in}}%
\pgfpathlineto{\pgfqpoint{9.011696in}{3.304875in}}%
\pgfpathlineto{\pgfqpoint{9.011696in}{3.311963in}}%
\pgfpathlineto{\pgfqpoint{-319.224843in}{3.311963in}}%
\pgfpathclose%
\pgfusepath{fill}%
\end{pgfscope}%
\begin{pgfscope}%
\pgfpathrectangle{\pgfqpoint{10.668400in}{3.021449in}}{\pgfqpoint{2.188235in}{0.972632in}}%
\pgfusepath{clip}%
\pgfsetbuttcap%
\pgfsetmiterjoin%
\definecolor{currentfill}{rgb}{0.121569,0.466667,0.705882}%
\pgfsetfillcolor{currentfill}%
\pgfsetlinewidth{0.000000pt}%
\definecolor{currentstroke}{rgb}{0.000000,0.000000,0.000000}%
\pgfsetstrokecolor{currentstroke}%
\pgfsetstrokeopacity{0.000000}%
\pgfsetdash{}{0pt}%
\pgfpathmoveto{\pgfqpoint{-319.224843in}{3.313735in}}%
\pgfpathlineto{\pgfqpoint{9.022869in}{3.313735in}}%
\pgfpathlineto{\pgfqpoint{9.022869in}{3.320823in}}%
\pgfpathlineto{\pgfqpoint{-319.224843in}{3.320823in}}%
\pgfpathclose%
\pgfusepath{fill}%
\end{pgfscope}%
\begin{pgfscope}%
\pgfpathrectangle{\pgfqpoint{10.668400in}{3.021449in}}{\pgfqpoint{2.188235in}{0.972632in}}%
\pgfusepath{clip}%
\pgfsetbuttcap%
\pgfsetmiterjoin%
\definecolor{currentfill}{rgb}{0.121569,0.466667,0.705882}%
\pgfsetfillcolor{currentfill}%
\pgfsetlinewidth{0.000000pt}%
\definecolor{currentstroke}{rgb}{0.000000,0.000000,0.000000}%
\pgfsetstrokecolor{currentstroke}%
\pgfsetstrokeopacity{0.000000}%
\pgfsetdash{}{0pt}%
\pgfpathmoveto{\pgfqpoint{-319.224843in}{3.322595in}}%
\pgfpathlineto{\pgfqpoint{9.111213in}{3.322595in}}%
\pgfpathlineto{\pgfqpoint{9.111213in}{3.329683in}}%
\pgfpathlineto{\pgfqpoint{-319.224843in}{3.329683in}}%
\pgfpathclose%
\pgfusepath{fill}%
\end{pgfscope}%
\begin{pgfscope}%
\pgfpathrectangle{\pgfqpoint{10.668400in}{3.021449in}}{\pgfqpoint{2.188235in}{0.972632in}}%
\pgfusepath{clip}%
\pgfsetbuttcap%
\pgfsetmiterjoin%
\definecolor{currentfill}{rgb}{0.121569,0.466667,0.705882}%
\pgfsetfillcolor{currentfill}%
\pgfsetlinewidth{0.000000pt}%
\definecolor{currentstroke}{rgb}{0.000000,0.000000,0.000000}%
\pgfsetstrokecolor{currentstroke}%
\pgfsetstrokeopacity{0.000000}%
\pgfsetdash{}{0pt}%
\pgfpathmoveto{\pgfqpoint{-319.224843in}{3.331454in}}%
\pgfpathlineto{\pgfqpoint{9.037068in}{3.331454in}}%
\pgfpathlineto{\pgfqpoint{9.037068in}{3.338542in}}%
\pgfpathlineto{\pgfqpoint{-319.224843in}{3.338542in}}%
\pgfpathclose%
\pgfusepath{fill}%
\end{pgfscope}%
\begin{pgfscope}%
\pgfpathrectangle{\pgfqpoint{10.668400in}{3.021449in}}{\pgfqpoint{2.188235in}{0.972632in}}%
\pgfusepath{clip}%
\pgfsetbuttcap%
\pgfsetmiterjoin%
\definecolor{currentfill}{rgb}{0.121569,0.466667,0.705882}%
\pgfsetfillcolor{currentfill}%
\pgfsetlinewidth{0.000000pt}%
\definecolor{currentstroke}{rgb}{0.000000,0.000000,0.000000}%
\pgfsetstrokecolor{currentstroke}%
\pgfsetstrokeopacity{0.000000}%
\pgfsetdash{}{0pt}%
\pgfpathmoveto{\pgfqpoint{-319.224843in}{3.340314in}}%
\pgfpathlineto{\pgfqpoint{9.040600in}{3.340314in}}%
\pgfpathlineto{\pgfqpoint{9.040600in}{3.347402in}}%
\pgfpathlineto{\pgfqpoint{-319.224843in}{3.347402in}}%
\pgfpathclose%
\pgfusepath{fill}%
\end{pgfscope}%
\begin{pgfscope}%
\pgfpathrectangle{\pgfqpoint{10.668400in}{3.021449in}}{\pgfqpoint{2.188235in}{0.972632in}}%
\pgfusepath{clip}%
\pgfsetbuttcap%
\pgfsetmiterjoin%
\definecolor{currentfill}{rgb}{0.121569,0.466667,0.705882}%
\pgfsetfillcolor{currentfill}%
\pgfsetlinewidth{0.000000pt}%
\definecolor{currentstroke}{rgb}{0.000000,0.000000,0.000000}%
\pgfsetstrokecolor{currentstroke}%
\pgfsetstrokeopacity{0.000000}%
\pgfsetdash{}{0pt}%
\pgfpathmoveto{\pgfqpoint{-319.224843in}{3.349174in}}%
\pgfpathlineto{\pgfqpoint{9.108536in}{3.349174in}}%
\pgfpathlineto{\pgfqpoint{9.108536in}{3.356262in}}%
\pgfpathlineto{\pgfqpoint{-319.224843in}{3.356262in}}%
\pgfpathclose%
\pgfusepath{fill}%
\end{pgfscope}%
\begin{pgfscope}%
\pgfpathrectangle{\pgfqpoint{10.668400in}{3.021449in}}{\pgfqpoint{2.188235in}{0.972632in}}%
\pgfusepath{clip}%
\pgfsetbuttcap%
\pgfsetmiterjoin%
\definecolor{currentfill}{rgb}{0.121569,0.466667,0.705882}%
\pgfsetfillcolor{currentfill}%
\pgfsetlinewidth{0.000000pt}%
\definecolor{currentstroke}{rgb}{0.000000,0.000000,0.000000}%
\pgfsetstrokecolor{currentstroke}%
\pgfsetstrokeopacity{0.000000}%
\pgfsetdash{}{0pt}%
\pgfpathmoveto{\pgfqpoint{-319.224843in}{3.358034in}}%
\pgfpathlineto{\pgfqpoint{9.184957in}{3.358034in}}%
\pgfpathlineto{\pgfqpoint{9.184957in}{3.365122in}}%
\pgfpathlineto{\pgfqpoint{-319.224843in}{3.365122in}}%
\pgfpathclose%
\pgfusepath{fill}%
\end{pgfscope}%
\begin{pgfscope}%
\pgfpathrectangle{\pgfqpoint{10.668400in}{3.021449in}}{\pgfqpoint{2.188235in}{0.972632in}}%
\pgfusepath{clip}%
\pgfsetbuttcap%
\pgfsetmiterjoin%
\definecolor{currentfill}{rgb}{0.121569,0.466667,0.705882}%
\pgfsetfillcolor{currentfill}%
\pgfsetlinewidth{0.000000pt}%
\definecolor{currentstroke}{rgb}{0.000000,0.000000,0.000000}%
\pgfsetstrokecolor{currentstroke}%
\pgfsetstrokeopacity{0.000000}%
\pgfsetdash{}{0pt}%
\pgfpathmoveto{\pgfqpoint{-319.224843in}{3.366894in}}%
\pgfpathlineto{\pgfqpoint{8.629789in}{3.366894in}}%
\pgfpathlineto{\pgfqpoint{8.629789in}{3.373982in}}%
\pgfpathlineto{\pgfqpoint{-319.224843in}{3.373982in}}%
\pgfpathclose%
\pgfusepath{fill}%
\end{pgfscope}%
\begin{pgfscope}%
\pgfpathrectangle{\pgfqpoint{10.668400in}{3.021449in}}{\pgfqpoint{2.188235in}{0.972632in}}%
\pgfusepath{clip}%
\pgfsetbuttcap%
\pgfsetmiterjoin%
\definecolor{currentfill}{rgb}{0.121569,0.466667,0.705882}%
\pgfsetfillcolor{currentfill}%
\pgfsetlinewidth{0.000000pt}%
\definecolor{currentstroke}{rgb}{0.000000,0.000000,0.000000}%
\pgfsetstrokecolor{currentstroke}%
\pgfsetstrokeopacity{0.000000}%
\pgfsetdash{}{0pt}%
\pgfpathmoveto{\pgfqpoint{-319.224843in}{3.375754in}}%
\pgfpathlineto{\pgfqpoint{9.125787in}{3.375754in}}%
\pgfpathlineto{\pgfqpoint{9.125787in}{3.382841in}}%
\pgfpathlineto{\pgfqpoint{-319.224843in}{3.382841in}}%
\pgfpathclose%
\pgfusepath{fill}%
\end{pgfscope}%
\begin{pgfscope}%
\pgfpathrectangle{\pgfqpoint{10.668400in}{3.021449in}}{\pgfqpoint{2.188235in}{0.972632in}}%
\pgfusepath{clip}%
\pgfsetbuttcap%
\pgfsetmiterjoin%
\definecolor{currentfill}{rgb}{0.121569,0.466667,0.705882}%
\pgfsetfillcolor{currentfill}%
\pgfsetlinewidth{0.000000pt}%
\definecolor{currentstroke}{rgb}{0.000000,0.000000,0.000000}%
\pgfsetstrokecolor{currentstroke}%
\pgfsetstrokeopacity{0.000000}%
\pgfsetdash{}{0pt}%
\pgfpathmoveto{\pgfqpoint{-319.224843in}{3.384613in}}%
\pgfpathlineto{\pgfqpoint{9.235496in}{3.384613in}}%
\pgfpathlineto{\pgfqpoint{9.235496in}{3.391701in}}%
\pgfpathlineto{\pgfqpoint{-319.224843in}{3.391701in}}%
\pgfpathclose%
\pgfusepath{fill}%
\end{pgfscope}%
\begin{pgfscope}%
\pgfpathrectangle{\pgfqpoint{10.668400in}{3.021449in}}{\pgfqpoint{2.188235in}{0.972632in}}%
\pgfusepath{clip}%
\pgfsetbuttcap%
\pgfsetmiterjoin%
\definecolor{currentfill}{rgb}{0.121569,0.466667,0.705882}%
\pgfsetfillcolor{currentfill}%
\pgfsetlinewidth{0.000000pt}%
\definecolor{currentstroke}{rgb}{0.000000,0.000000,0.000000}%
\pgfsetstrokecolor{currentstroke}%
\pgfsetstrokeopacity{0.000000}%
\pgfsetdash{}{0pt}%
\pgfpathmoveto{\pgfqpoint{-319.224843in}{3.393473in}}%
\pgfpathlineto{\pgfqpoint{9.171744in}{3.393473in}}%
\pgfpathlineto{\pgfqpoint{9.171744in}{3.400561in}}%
\pgfpathlineto{\pgfqpoint{-319.224843in}{3.400561in}}%
\pgfpathclose%
\pgfusepath{fill}%
\end{pgfscope}%
\begin{pgfscope}%
\pgfpathrectangle{\pgfqpoint{10.668400in}{3.021449in}}{\pgfqpoint{2.188235in}{0.972632in}}%
\pgfusepath{clip}%
\pgfsetbuttcap%
\pgfsetmiterjoin%
\definecolor{currentfill}{rgb}{0.121569,0.466667,0.705882}%
\pgfsetfillcolor{currentfill}%
\pgfsetlinewidth{0.000000pt}%
\definecolor{currentstroke}{rgb}{0.000000,0.000000,0.000000}%
\pgfsetstrokecolor{currentstroke}%
\pgfsetstrokeopacity{0.000000}%
\pgfsetdash{}{0pt}%
\pgfpathmoveto{\pgfqpoint{-319.224843in}{3.402333in}}%
\pgfpathlineto{\pgfqpoint{9.139996in}{3.402333in}}%
\pgfpathlineto{\pgfqpoint{9.139996in}{3.409421in}}%
\pgfpathlineto{\pgfqpoint{-319.224843in}{3.409421in}}%
\pgfpathclose%
\pgfusepath{fill}%
\end{pgfscope}%
\begin{pgfscope}%
\pgfpathrectangle{\pgfqpoint{10.668400in}{3.021449in}}{\pgfqpoint{2.188235in}{0.972632in}}%
\pgfusepath{clip}%
\pgfsetbuttcap%
\pgfsetmiterjoin%
\definecolor{currentfill}{rgb}{0.121569,0.466667,0.705882}%
\pgfsetfillcolor{currentfill}%
\pgfsetlinewidth{0.000000pt}%
\definecolor{currentstroke}{rgb}{0.000000,0.000000,0.000000}%
\pgfsetstrokecolor{currentstroke}%
\pgfsetstrokeopacity{0.000000}%
\pgfsetdash{}{0pt}%
\pgfpathmoveto{\pgfqpoint{-319.224843in}{3.411193in}}%
\pgfpathlineto{\pgfqpoint{8.883658in}{3.411193in}}%
\pgfpathlineto{\pgfqpoint{8.883658in}{3.418281in}}%
\pgfpathlineto{\pgfqpoint{-319.224843in}{3.418281in}}%
\pgfpathclose%
\pgfusepath{fill}%
\end{pgfscope}%
\begin{pgfscope}%
\pgfpathrectangle{\pgfqpoint{10.668400in}{3.021449in}}{\pgfqpoint{2.188235in}{0.972632in}}%
\pgfusepath{clip}%
\pgfsetbuttcap%
\pgfsetmiterjoin%
\definecolor{currentfill}{rgb}{0.121569,0.466667,0.705882}%
\pgfsetfillcolor{currentfill}%
\pgfsetlinewidth{0.000000pt}%
\definecolor{currentstroke}{rgb}{0.000000,0.000000,0.000000}%
\pgfsetstrokecolor{currentstroke}%
\pgfsetstrokeopacity{0.000000}%
\pgfsetdash{}{0pt}%
\pgfpathmoveto{\pgfqpoint{-319.224843in}{3.420053in}}%
\pgfpathlineto{\pgfqpoint{8.963472in}{3.420053in}}%
\pgfpathlineto{\pgfqpoint{8.963472in}{3.427141in}}%
\pgfpathlineto{\pgfqpoint{-319.224843in}{3.427141in}}%
\pgfpathclose%
\pgfusepath{fill}%
\end{pgfscope}%
\begin{pgfscope}%
\pgfpathrectangle{\pgfqpoint{10.668400in}{3.021449in}}{\pgfqpoint{2.188235in}{0.972632in}}%
\pgfusepath{clip}%
\pgfsetbuttcap%
\pgfsetmiterjoin%
\definecolor{currentfill}{rgb}{0.121569,0.466667,0.705882}%
\pgfsetfillcolor{currentfill}%
\pgfsetlinewidth{0.000000pt}%
\definecolor{currentstroke}{rgb}{0.000000,0.000000,0.000000}%
\pgfsetstrokecolor{currentstroke}%
\pgfsetstrokeopacity{0.000000}%
\pgfsetdash{}{0pt}%
\pgfpathmoveto{\pgfqpoint{-319.224843in}{3.428913in}}%
\pgfpathlineto{\pgfqpoint{8.863687in}{3.428913in}}%
\pgfpathlineto{\pgfqpoint{8.863687in}{3.436000in}}%
\pgfpathlineto{\pgfqpoint{-319.224843in}{3.436000in}}%
\pgfpathclose%
\pgfusepath{fill}%
\end{pgfscope}%
\begin{pgfscope}%
\pgfpathrectangle{\pgfqpoint{10.668400in}{3.021449in}}{\pgfqpoint{2.188235in}{0.972632in}}%
\pgfusepath{clip}%
\pgfsetbuttcap%
\pgfsetmiterjoin%
\definecolor{currentfill}{rgb}{0.121569,0.466667,0.705882}%
\pgfsetfillcolor{currentfill}%
\pgfsetlinewidth{0.000000pt}%
\definecolor{currentstroke}{rgb}{0.000000,0.000000,0.000000}%
\pgfsetstrokecolor{currentstroke}%
\pgfsetstrokeopacity{0.000000}%
\pgfsetdash{}{0pt}%
\pgfpathmoveto{\pgfqpoint{-319.224843in}{3.437772in}}%
\pgfpathlineto{\pgfqpoint{9.098055in}{3.437772in}}%
\pgfpathlineto{\pgfqpoint{9.098055in}{3.444860in}}%
\pgfpathlineto{\pgfqpoint{-319.224843in}{3.444860in}}%
\pgfpathclose%
\pgfusepath{fill}%
\end{pgfscope}%
\begin{pgfscope}%
\pgfpathrectangle{\pgfqpoint{10.668400in}{3.021449in}}{\pgfqpoint{2.188235in}{0.972632in}}%
\pgfusepath{clip}%
\pgfsetbuttcap%
\pgfsetmiterjoin%
\definecolor{currentfill}{rgb}{0.121569,0.466667,0.705882}%
\pgfsetfillcolor{currentfill}%
\pgfsetlinewidth{0.000000pt}%
\definecolor{currentstroke}{rgb}{0.000000,0.000000,0.000000}%
\pgfsetstrokecolor{currentstroke}%
\pgfsetstrokeopacity{0.000000}%
\pgfsetdash{}{0pt}%
\pgfpathmoveto{\pgfqpoint{-319.224843in}{3.446632in}}%
\pgfpathlineto{\pgfqpoint{9.057200in}{3.446632in}}%
\pgfpathlineto{\pgfqpoint{9.057200in}{3.453720in}}%
\pgfpathlineto{\pgfqpoint{-319.224843in}{3.453720in}}%
\pgfpathclose%
\pgfusepath{fill}%
\end{pgfscope}%
\begin{pgfscope}%
\pgfpathrectangle{\pgfqpoint{10.668400in}{3.021449in}}{\pgfqpoint{2.188235in}{0.972632in}}%
\pgfusepath{clip}%
\pgfsetbuttcap%
\pgfsetmiterjoin%
\definecolor{currentfill}{rgb}{0.121569,0.466667,0.705882}%
\pgfsetfillcolor{currentfill}%
\pgfsetlinewidth{0.000000pt}%
\definecolor{currentstroke}{rgb}{0.000000,0.000000,0.000000}%
\pgfsetstrokecolor{currentstroke}%
\pgfsetstrokeopacity{0.000000}%
\pgfsetdash{}{0pt}%
\pgfpathmoveto{\pgfqpoint{-319.224843in}{3.455492in}}%
\pgfpathlineto{\pgfqpoint{9.054934in}{3.455492in}}%
\pgfpathlineto{\pgfqpoint{9.054934in}{3.462580in}}%
\pgfpathlineto{\pgfqpoint{-319.224843in}{3.462580in}}%
\pgfpathclose%
\pgfusepath{fill}%
\end{pgfscope}%
\begin{pgfscope}%
\pgfpathrectangle{\pgfqpoint{10.668400in}{3.021449in}}{\pgfqpoint{2.188235in}{0.972632in}}%
\pgfusepath{clip}%
\pgfsetbuttcap%
\pgfsetmiterjoin%
\definecolor{currentfill}{rgb}{0.121569,0.466667,0.705882}%
\pgfsetfillcolor{currentfill}%
\pgfsetlinewidth{0.000000pt}%
\definecolor{currentstroke}{rgb}{0.000000,0.000000,0.000000}%
\pgfsetstrokecolor{currentstroke}%
\pgfsetstrokeopacity{0.000000}%
\pgfsetdash{}{0pt}%
\pgfpathmoveto{\pgfqpoint{-319.224843in}{3.464352in}}%
\pgfpathlineto{\pgfqpoint{9.099414in}{3.464352in}}%
\pgfpathlineto{\pgfqpoint{9.099414in}{3.471440in}}%
\pgfpathlineto{\pgfqpoint{-319.224843in}{3.471440in}}%
\pgfpathclose%
\pgfusepath{fill}%
\end{pgfscope}%
\begin{pgfscope}%
\pgfpathrectangle{\pgfqpoint{10.668400in}{3.021449in}}{\pgfqpoint{2.188235in}{0.972632in}}%
\pgfusepath{clip}%
\pgfsetbuttcap%
\pgfsetmiterjoin%
\definecolor{currentfill}{rgb}{0.121569,0.466667,0.705882}%
\pgfsetfillcolor{currentfill}%
\pgfsetlinewidth{0.000000pt}%
\definecolor{currentstroke}{rgb}{0.000000,0.000000,0.000000}%
\pgfsetstrokecolor{currentstroke}%
\pgfsetstrokeopacity{0.000000}%
\pgfsetdash{}{0pt}%
\pgfpathmoveto{\pgfqpoint{-319.224843in}{3.473212in}}%
\pgfpathlineto{\pgfqpoint{9.145455in}{3.473212in}}%
\pgfpathlineto{\pgfqpoint{9.145455in}{3.480300in}}%
\pgfpathlineto{\pgfqpoint{-319.224843in}{3.480300in}}%
\pgfpathclose%
\pgfusepath{fill}%
\end{pgfscope}%
\begin{pgfscope}%
\pgfpathrectangle{\pgfqpoint{10.668400in}{3.021449in}}{\pgfqpoint{2.188235in}{0.972632in}}%
\pgfusepath{clip}%
\pgfsetbuttcap%
\pgfsetmiterjoin%
\definecolor{currentfill}{rgb}{0.121569,0.466667,0.705882}%
\pgfsetfillcolor{currentfill}%
\pgfsetlinewidth{0.000000pt}%
\definecolor{currentstroke}{rgb}{0.000000,0.000000,0.000000}%
\pgfsetstrokecolor{currentstroke}%
\pgfsetstrokeopacity{0.000000}%
\pgfsetdash{}{0pt}%
\pgfpathmoveto{\pgfqpoint{-319.224843in}{3.482072in}}%
\pgfpathlineto{\pgfqpoint{8.862503in}{3.482072in}}%
\pgfpathlineto{\pgfqpoint{8.862503in}{3.489159in}}%
\pgfpathlineto{\pgfqpoint{-319.224843in}{3.489159in}}%
\pgfpathclose%
\pgfusepath{fill}%
\end{pgfscope}%
\begin{pgfscope}%
\pgfpathrectangle{\pgfqpoint{10.668400in}{3.021449in}}{\pgfqpoint{2.188235in}{0.972632in}}%
\pgfusepath{clip}%
\pgfsetbuttcap%
\pgfsetmiterjoin%
\definecolor{currentfill}{rgb}{0.121569,0.466667,0.705882}%
\pgfsetfillcolor{currentfill}%
\pgfsetlinewidth{0.000000pt}%
\definecolor{currentstroke}{rgb}{0.000000,0.000000,0.000000}%
\pgfsetstrokecolor{currentstroke}%
\pgfsetstrokeopacity{0.000000}%
\pgfsetdash{}{0pt}%
\pgfpathmoveto{\pgfqpoint{-319.224843in}{3.490931in}}%
\pgfpathlineto{\pgfqpoint{9.134390in}{3.490931in}}%
\pgfpathlineto{\pgfqpoint{9.134390in}{3.498019in}}%
\pgfpathlineto{\pgfqpoint{-319.224843in}{3.498019in}}%
\pgfpathclose%
\pgfusepath{fill}%
\end{pgfscope}%
\begin{pgfscope}%
\pgfpathrectangle{\pgfqpoint{10.668400in}{3.021449in}}{\pgfqpoint{2.188235in}{0.972632in}}%
\pgfusepath{clip}%
\pgfsetbuttcap%
\pgfsetmiterjoin%
\definecolor{currentfill}{rgb}{0.121569,0.466667,0.705882}%
\pgfsetfillcolor{currentfill}%
\pgfsetlinewidth{0.000000pt}%
\definecolor{currentstroke}{rgb}{0.000000,0.000000,0.000000}%
\pgfsetstrokecolor{currentstroke}%
\pgfsetstrokeopacity{0.000000}%
\pgfsetdash{}{0pt}%
\pgfpathmoveto{\pgfqpoint{-319.224843in}{3.499791in}}%
\pgfpathlineto{\pgfqpoint{9.190603in}{3.499791in}}%
\pgfpathlineto{\pgfqpoint{9.190603in}{3.506879in}}%
\pgfpathlineto{\pgfqpoint{-319.224843in}{3.506879in}}%
\pgfpathclose%
\pgfusepath{fill}%
\end{pgfscope}%
\begin{pgfscope}%
\pgfpathrectangle{\pgfqpoint{10.668400in}{3.021449in}}{\pgfqpoint{2.188235in}{0.972632in}}%
\pgfusepath{clip}%
\pgfsetbuttcap%
\pgfsetmiterjoin%
\definecolor{currentfill}{rgb}{0.121569,0.466667,0.705882}%
\pgfsetfillcolor{currentfill}%
\pgfsetlinewidth{0.000000pt}%
\definecolor{currentstroke}{rgb}{0.000000,0.000000,0.000000}%
\pgfsetstrokecolor{currentstroke}%
\pgfsetstrokeopacity{0.000000}%
\pgfsetdash{}{0pt}%
\pgfpathmoveto{\pgfqpoint{-319.224843in}{3.508651in}}%
\pgfpathlineto{\pgfqpoint{9.146823in}{3.508651in}}%
\pgfpathlineto{\pgfqpoint{9.146823in}{3.515739in}}%
\pgfpathlineto{\pgfqpoint{-319.224843in}{3.515739in}}%
\pgfpathclose%
\pgfusepath{fill}%
\end{pgfscope}%
\begin{pgfscope}%
\pgfpathrectangle{\pgfqpoint{10.668400in}{3.021449in}}{\pgfqpoint{2.188235in}{0.972632in}}%
\pgfusepath{clip}%
\pgfsetbuttcap%
\pgfsetmiterjoin%
\definecolor{currentfill}{rgb}{0.121569,0.466667,0.705882}%
\pgfsetfillcolor{currentfill}%
\pgfsetlinewidth{0.000000pt}%
\definecolor{currentstroke}{rgb}{0.000000,0.000000,0.000000}%
\pgfsetstrokecolor{currentstroke}%
\pgfsetstrokeopacity{0.000000}%
\pgfsetdash{}{0pt}%
\pgfpathmoveto{\pgfqpoint{-319.224843in}{3.517511in}}%
\pgfpathlineto{\pgfqpoint{9.132516in}{3.517511in}}%
\pgfpathlineto{\pgfqpoint{9.132516in}{3.524599in}}%
\pgfpathlineto{\pgfqpoint{-319.224843in}{3.524599in}}%
\pgfpathclose%
\pgfusepath{fill}%
\end{pgfscope}%
\begin{pgfscope}%
\pgfpathrectangle{\pgfqpoint{10.668400in}{3.021449in}}{\pgfqpoint{2.188235in}{0.972632in}}%
\pgfusepath{clip}%
\pgfsetbuttcap%
\pgfsetmiterjoin%
\definecolor{currentfill}{rgb}{0.121569,0.466667,0.705882}%
\pgfsetfillcolor{currentfill}%
\pgfsetlinewidth{0.000000pt}%
\definecolor{currentstroke}{rgb}{0.000000,0.000000,0.000000}%
\pgfsetstrokecolor{currentstroke}%
\pgfsetstrokeopacity{0.000000}%
\pgfsetdash{}{0pt}%
\pgfpathmoveto{\pgfqpoint{-319.224843in}{3.526371in}}%
\pgfpathlineto{\pgfqpoint{8.874559in}{3.526371in}}%
\pgfpathlineto{\pgfqpoint{8.874559in}{3.533459in}}%
\pgfpathlineto{\pgfqpoint{-319.224843in}{3.533459in}}%
\pgfpathclose%
\pgfusepath{fill}%
\end{pgfscope}%
\begin{pgfscope}%
\pgfpathrectangle{\pgfqpoint{10.668400in}{3.021449in}}{\pgfqpoint{2.188235in}{0.972632in}}%
\pgfusepath{clip}%
\pgfsetbuttcap%
\pgfsetmiterjoin%
\definecolor{currentfill}{rgb}{0.121569,0.466667,0.705882}%
\pgfsetfillcolor{currentfill}%
\pgfsetlinewidth{0.000000pt}%
\definecolor{currentstroke}{rgb}{0.000000,0.000000,0.000000}%
\pgfsetstrokecolor{currentstroke}%
\pgfsetstrokeopacity{0.000000}%
\pgfsetdash{}{0pt}%
\pgfpathmoveto{\pgfqpoint{-319.224843in}{3.535230in}}%
\pgfpathlineto{\pgfqpoint{8.985655in}{3.535230in}}%
\pgfpathlineto{\pgfqpoint{8.985655in}{3.542318in}}%
\pgfpathlineto{\pgfqpoint{-319.224843in}{3.542318in}}%
\pgfpathclose%
\pgfusepath{fill}%
\end{pgfscope}%
\begin{pgfscope}%
\pgfpathrectangle{\pgfqpoint{10.668400in}{3.021449in}}{\pgfqpoint{2.188235in}{0.972632in}}%
\pgfusepath{clip}%
\pgfsetbuttcap%
\pgfsetmiterjoin%
\definecolor{currentfill}{rgb}{0.121569,0.466667,0.705882}%
\pgfsetfillcolor{currentfill}%
\pgfsetlinewidth{0.000000pt}%
\definecolor{currentstroke}{rgb}{0.000000,0.000000,0.000000}%
\pgfsetstrokecolor{currentstroke}%
\pgfsetstrokeopacity{0.000000}%
\pgfsetdash{}{0pt}%
\pgfpathmoveto{\pgfqpoint{-319.224843in}{3.544090in}}%
\pgfpathlineto{\pgfqpoint{9.024454in}{3.544090in}}%
\pgfpathlineto{\pgfqpoint{9.024454in}{3.551178in}}%
\pgfpathlineto{\pgfqpoint{-319.224843in}{3.551178in}}%
\pgfpathclose%
\pgfusepath{fill}%
\end{pgfscope}%
\begin{pgfscope}%
\pgfpathrectangle{\pgfqpoint{10.668400in}{3.021449in}}{\pgfqpoint{2.188235in}{0.972632in}}%
\pgfusepath{clip}%
\pgfsetbuttcap%
\pgfsetmiterjoin%
\definecolor{currentfill}{rgb}{0.121569,0.466667,0.705882}%
\pgfsetfillcolor{currentfill}%
\pgfsetlinewidth{0.000000pt}%
\definecolor{currentstroke}{rgb}{0.000000,0.000000,0.000000}%
\pgfsetstrokecolor{currentstroke}%
\pgfsetstrokeopacity{0.000000}%
\pgfsetdash{}{0pt}%
\pgfpathmoveto{\pgfqpoint{-319.224843in}{3.552950in}}%
\pgfpathlineto{\pgfqpoint{9.040127in}{3.552950in}}%
\pgfpathlineto{\pgfqpoint{9.040127in}{3.560038in}}%
\pgfpathlineto{\pgfqpoint{-319.224843in}{3.560038in}}%
\pgfpathclose%
\pgfusepath{fill}%
\end{pgfscope}%
\begin{pgfscope}%
\pgfpathrectangle{\pgfqpoint{10.668400in}{3.021449in}}{\pgfqpoint{2.188235in}{0.972632in}}%
\pgfusepath{clip}%
\pgfsetbuttcap%
\pgfsetmiterjoin%
\definecolor{currentfill}{rgb}{0.121569,0.466667,0.705882}%
\pgfsetfillcolor{currentfill}%
\pgfsetlinewidth{0.000000pt}%
\definecolor{currentstroke}{rgb}{0.000000,0.000000,0.000000}%
\pgfsetstrokecolor{currentstroke}%
\pgfsetstrokeopacity{0.000000}%
\pgfsetdash{}{0pt}%
\pgfpathmoveto{\pgfqpoint{-319.224843in}{3.561810in}}%
\pgfpathlineto{\pgfqpoint{8.679966in}{3.561810in}}%
\pgfpathlineto{\pgfqpoint{8.679966in}{3.568898in}}%
\pgfpathlineto{\pgfqpoint{-319.224843in}{3.568898in}}%
\pgfpathclose%
\pgfusepath{fill}%
\end{pgfscope}%
\begin{pgfscope}%
\pgfpathrectangle{\pgfqpoint{10.668400in}{3.021449in}}{\pgfqpoint{2.188235in}{0.972632in}}%
\pgfusepath{clip}%
\pgfsetbuttcap%
\pgfsetmiterjoin%
\definecolor{currentfill}{rgb}{0.121569,0.466667,0.705882}%
\pgfsetfillcolor{currentfill}%
\pgfsetlinewidth{0.000000pt}%
\definecolor{currentstroke}{rgb}{0.000000,0.000000,0.000000}%
\pgfsetstrokecolor{currentstroke}%
\pgfsetstrokeopacity{0.000000}%
\pgfsetdash{}{0pt}%
\pgfpathmoveto{\pgfqpoint{-319.224843in}{3.570670in}}%
\pgfpathlineto{\pgfqpoint{9.044492in}{3.570670in}}%
\pgfpathlineto{\pgfqpoint{9.044492in}{3.577758in}}%
\pgfpathlineto{\pgfqpoint{-319.224843in}{3.577758in}}%
\pgfpathclose%
\pgfusepath{fill}%
\end{pgfscope}%
\begin{pgfscope}%
\pgfpathrectangle{\pgfqpoint{10.668400in}{3.021449in}}{\pgfqpoint{2.188235in}{0.972632in}}%
\pgfusepath{clip}%
\pgfsetbuttcap%
\pgfsetmiterjoin%
\definecolor{currentfill}{rgb}{0.121569,0.466667,0.705882}%
\pgfsetfillcolor{currentfill}%
\pgfsetlinewidth{0.000000pt}%
\definecolor{currentstroke}{rgb}{0.000000,0.000000,0.000000}%
\pgfsetstrokecolor{currentstroke}%
\pgfsetstrokeopacity{0.000000}%
\pgfsetdash{}{0pt}%
\pgfpathmoveto{\pgfqpoint{-319.224843in}{3.579530in}}%
\pgfpathlineto{\pgfqpoint{9.137058in}{3.579530in}}%
\pgfpathlineto{\pgfqpoint{9.137058in}{3.586617in}}%
\pgfpathlineto{\pgfqpoint{-319.224843in}{3.586617in}}%
\pgfpathclose%
\pgfusepath{fill}%
\end{pgfscope}%
\begin{pgfscope}%
\pgfpathrectangle{\pgfqpoint{10.668400in}{3.021449in}}{\pgfqpoint{2.188235in}{0.972632in}}%
\pgfusepath{clip}%
\pgfsetbuttcap%
\pgfsetmiterjoin%
\definecolor{currentfill}{rgb}{0.121569,0.466667,0.705882}%
\pgfsetfillcolor{currentfill}%
\pgfsetlinewidth{0.000000pt}%
\definecolor{currentstroke}{rgb}{0.000000,0.000000,0.000000}%
\pgfsetstrokecolor{currentstroke}%
\pgfsetstrokeopacity{0.000000}%
\pgfsetdash{}{0pt}%
\pgfpathmoveto{\pgfqpoint{-319.224843in}{3.588389in}}%
\pgfpathlineto{\pgfqpoint{9.190911in}{3.588389in}}%
\pgfpathlineto{\pgfqpoint{9.190911in}{3.595477in}}%
\pgfpathlineto{\pgfqpoint{-319.224843in}{3.595477in}}%
\pgfpathclose%
\pgfusepath{fill}%
\end{pgfscope}%
\begin{pgfscope}%
\pgfpathrectangle{\pgfqpoint{10.668400in}{3.021449in}}{\pgfqpoint{2.188235in}{0.972632in}}%
\pgfusepath{clip}%
\pgfsetbuttcap%
\pgfsetmiterjoin%
\definecolor{currentfill}{rgb}{0.121569,0.466667,0.705882}%
\pgfsetfillcolor{currentfill}%
\pgfsetlinewidth{0.000000pt}%
\definecolor{currentstroke}{rgb}{0.000000,0.000000,0.000000}%
\pgfsetstrokecolor{currentstroke}%
\pgfsetstrokeopacity{0.000000}%
\pgfsetdash{}{0pt}%
\pgfpathmoveto{\pgfqpoint{-319.224843in}{3.597249in}}%
\pgfpathlineto{\pgfqpoint{9.103582in}{3.597249in}}%
\pgfpathlineto{\pgfqpoint{9.103582in}{3.604337in}}%
\pgfpathlineto{\pgfqpoint{-319.224843in}{3.604337in}}%
\pgfpathclose%
\pgfusepath{fill}%
\end{pgfscope}%
\begin{pgfscope}%
\pgfpathrectangle{\pgfqpoint{10.668400in}{3.021449in}}{\pgfqpoint{2.188235in}{0.972632in}}%
\pgfusepath{clip}%
\pgfsetbuttcap%
\pgfsetmiterjoin%
\definecolor{currentfill}{rgb}{0.121569,0.466667,0.705882}%
\pgfsetfillcolor{currentfill}%
\pgfsetlinewidth{0.000000pt}%
\definecolor{currentstroke}{rgb}{0.000000,0.000000,0.000000}%
\pgfsetstrokecolor{currentstroke}%
\pgfsetstrokeopacity{0.000000}%
\pgfsetdash{}{0pt}%
\pgfpathmoveto{\pgfqpoint{-319.224843in}{3.606109in}}%
\pgfpathlineto{\pgfqpoint{8.655187in}{3.606109in}}%
\pgfpathlineto{\pgfqpoint{8.655187in}{3.613197in}}%
\pgfpathlineto{\pgfqpoint{-319.224843in}{3.613197in}}%
\pgfpathclose%
\pgfusepath{fill}%
\end{pgfscope}%
\begin{pgfscope}%
\pgfpathrectangle{\pgfqpoint{10.668400in}{3.021449in}}{\pgfqpoint{2.188235in}{0.972632in}}%
\pgfusepath{clip}%
\pgfsetbuttcap%
\pgfsetmiterjoin%
\definecolor{currentfill}{rgb}{0.121569,0.466667,0.705882}%
\pgfsetfillcolor{currentfill}%
\pgfsetlinewidth{0.000000pt}%
\definecolor{currentstroke}{rgb}{0.000000,0.000000,0.000000}%
\pgfsetstrokecolor{currentstroke}%
\pgfsetstrokeopacity{0.000000}%
\pgfsetdash{}{0pt}%
\pgfpathmoveto{\pgfqpoint{-319.224843in}{3.614969in}}%
\pgfpathlineto{\pgfqpoint{9.019429in}{3.614969in}}%
\pgfpathlineto{\pgfqpoint{9.019429in}{3.622057in}}%
\pgfpathlineto{\pgfqpoint{-319.224843in}{3.622057in}}%
\pgfpathclose%
\pgfusepath{fill}%
\end{pgfscope}%
\begin{pgfscope}%
\pgfpathrectangle{\pgfqpoint{10.668400in}{3.021449in}}{\pgfqpoint{2.188235in}{0.972632in}}%
\pgfusepath{clip}%
\pgfsetbuttcap%
\pgfsetmiterjoin%
\definecolor{currentfill}{rgb}{0.121569,0.466667,0.705882}%
\pgfsetfillcolor{currentfill}%
\pgfsetlinewidth{0.000000pt}%
\definecolor{currentstroke}{rgb}{0.000000,0.000000,0.000000}%
\pgfsetstrokecolor{currentstroke}%
\pgfsetstrokeopacity{0.000000}%
\pgfsetdash{}{0pt}%
\pgfpathmoveto{\pgfqpoint{-319.224843in}{3.623829in}}%
\pgfpathlineto{\pgfqpoint{8.850740in}{3.623829in}}%
\pgfpathlineto{\pgfqpoint{8.850740in}{3.630917in}}%
\pgfpathlineto{\pgfqpoint{-319.224843in}{3.630917in}}%
\pgfpathclose%
\pgfusepath{fill}%
\end{pgfscope}%
\begin{pgfscope}%
\pgfpathrectangle{\pgfqpoint{10.668400in}{3.021449in}}{\pgfqpoint{2.188235in}{0.972632in}}%
\pgfusepath{clip}%
\pgfsetbuttcap%
\pgfsetmiterjoin%
\definecolor{currentfill}{rgb}{0.121569,0.466667,0.705882}%
\pgfsetfillcolor{currentfill}%
\pgfsetlinewidth{0.000000pt}%
\definecolor{currentstroke}{rgb}{0.000000,0.000000,0.000000}%
\pgfsetstrokecolor{currentstroke}%
\pgfsetstrokeopacity{0.000000}%
\pgfsetdash{}{0pt}%
\pgfpathmoveto{\pgfqpoint{-319.224843in}{3.632689in}}%
\pgfpathlineto{\pgfqpoint{9.121055in}{3.632689in}}%
\pgfpathlineto{\pgfqpoint{9.121055in}{3.639776in}}%
\pgfpathlineto{\pgfqpoint{-319.224843in}{3.639776in}}%
\pgfpathclose%
\pgfusepath{fill}%
\end{pgfscope}%
\begin{pgfscope}%
\pgfpathrectangle{\pgfqpoint{10.668400in}{3.021449in}}{\pgfqpoint{2.188235in}{0.972632in}}%
\pgfusepath{clip}%
\pgfsetbuttcap%
\pgfsetmiterjoin%
\definecolor{currentfill}{rgb}{0.121569,0.466667,0.705882}%
\pgfsetfillcolor{currentfill}%
\pgfsetlinewidth{0.000000pt}%
\definecolor{currentstroke}{rgb}{0.000000,0.000000,0.000000}%
\pgfsetstrokecolor{currentstroke}%
\pgfsetstrokeopacity{0.000000}%
\pgfsetdash{}{0pt}%
\pgfpathmoveto{\pgfqpoint{-319.224843in}{3.641548in}}%
\pgfpathlineto{\pgfqpoint{9.096063in}{3.641548in}}%
\pgfpathlineto{\pgfqpoint{9.096063in}{3.648636in}}%
\pgfpathlineto{\pgfqpoint{-319.224843in}{3.648636in}}%
\pgfpathclose%
\pgfusepath{fill}%
\end{pgfscope}%
\begin{pgfscope}%
\pgfpathrectangle{\pgfqpoint{10.668400in}{3.021449in}}{\pgfqpoint{2.188235in}{0.972632in}}%
\pgfusepath{clip}%
\pgfsetbuttcap%
\pgfsetmiterjoin%
\definecolor{currentfill}{rgb}{0.121569,0.466667,0.705882}%
\pgfsetfillcolor{currentfill}%
\pgfsetlinewidth{0.000000pt}%
\definecolor{currentstroke}{rgb}{0.000000,0.000000,0.000000}%
\pgfsetstrokecolor{currentstroke}%
\pgfsetstrokeopacity{0.000000}%
\pgfsetdash{}{0pt}%
\pgfpathmoveto{\pgfqpoint{-319.224843in}{3.650408in}}%
\pgfpathlineto{\pgfqpoint{8.938662in}{3.650408in}}%
\pgfpathlineto{\pgfqpoint{8.938662in}{3.657496in}}%
\pgfpathlineto{\pgfqpoint{-319.224843in}{3.657496in}}%
\pgfpathclose%
\pgfusepath{fill}%
\end{pgfscope}%
\begin{pgfscope}%
\pgfpathrectangle{\pgfqpoint{10.668400in}{3.021449in}}{\pgfqpoint{2.188235in}{0.972632in}}%
\pgfusepath{clip}%
\pgfsetbuttcap%
\pgfsetmiterjoin%
\definecolor{currentfill}{rgb}{0.121569,0.466667,0.705882}%
\pgfsetfillcolor{currentfill}%
\pgfsetlinewidth{0.000000pt}%
\definecolor{currentstroke}{rgb}{0.000000,0.000000,0.000000}%
\pgfsetstrokecolor{currentstroke}%
\pgfsetstrokeopacity{0.000000}%
\pgfsetdash{}{0pt}%
\pgfpathmoveto{\pgfqpoint{-319.224843in}{3.659268in}}%
\pgfpathlineto{\pgfqpoint{8.656691in}{3.659268in}}%
\pgfpathlineto{\pgfqpoint{8.656691in}{3.666356in}}%
\pgfpathlineto{\pgfqpoint{-319.224843in}{3.666356in}}%
\pgfpathclose%
\pgfusepath{fill}%
\end{pgfscope}%
\begin{pgfscope}%
\pgfpathrectangle{\pgfqpoint{10.668400in}{3.021449in}}{\pgfqpoint{2.188235in}{0.972632in}}%
\pgfusepath{clip}%
\pgfsetbuttcap%
\pgfsetmiterjoin%
\definecolor{currentfill}{rgb}{0.121569,0.466667,0.705882}%
\pgfsetfillcolor{currentfill}%
\pgfsetlinewidth{0.000000pt}%
\definecolor{currentstroke}{rgb}{0.000000,0.000000,0.000000}%
\pgfsetstrokecolor{currentstroke}%
\pgfsetstrokeopacity{0.000000}%
\pgfsetdash{}{0pt}%
\pgfpathmoveto{\pgfqpoint{-319.224843in}{3.668128in}}%
\pgfpathlineto{\pgfqpoint{9.209362in}{3.668128in}}%
\pgfpathlineto{\pgfqpoint{9.209362in}{3.675216in}}%
\pgfpathlineto{\pgfqpoint{-319.224843in}{3.675216in}}%
\pgfpathclose%
\pgfusepath{fill}%
\end{pgfscope}%
\begin{pgfscope}%
\pgfpathrectangle{\pgfqpoint{10.668400in}{3.021449in}}{\pgfqpoint{2.188235in}{0.972632in}}%
\pgfusepath{clip}%
\pgfsetbuttcap%
\pgfsetmiterjoin%
\definecolor{currentfill}{rgb}{0.121569,0.466667,0.705882}%
\pgfsetfillcolor{currentfill}%
\pgfsetlinewidth{0.000000pt}%
\definecolor{currentstroke}{rgb}{0.000000,0.000000,0.000000}%
\pgfsetstrokecolor{currentstroke}%
\pgfsetstrokeopacity{0.000000}%
\pgfsetdash{}{0pt}%
\pgfpathmoveto{\pgfqpoint{-319.224843in}{3.676988in}}%
\pgfpathlineto{\pgfqpoint{9.121717in}{3.676988in}}%
\pgfpathlineto{\pgfqpoint{9.121717in}{3.684076in}}%
\pgfpathlineto{\pgfqpoint{-319.224843in}{3.684076in}}%
\pgfpathclose%
\pgfusepath{fill}%
\end{pgfscope}%
\begin{pgfscope}%
\pgfpathrectangle{\pgfqpoint{10.668400in}{3.021449in}}{\pgfqpoint{2.188235in}{0.972632in}}%
\pgfusepath{clip}%
\pgfsetbuttcap%
\pgfsetmiterjoin%
\definecolor{currentfill}{rgb}{0.121569,0.466667,0.705882}%
\pgfsetfillcolor{currentfill}%
\pgfsetlinewidth{0.000000pt}%
\definecolor{currentstroke}{rgb}{0.000000,0.000000,0.000000}%
\pgfsetstrokecolor{currentstroke}%
\pgfsetstrokeopacity{0.000000}%
\pgfsetdash{}{0pt}%
\pgfpathmoveto{\pgfqpoint{-319.224843in}{3.685847in}}%
\pgfpathlineto{\pgfqpoint{9.192075in}{3.685847in}}%
\pgfpathlineto{\pgfqpoint{9.192075in}{3.692935in}}%
\pgfpathlineto{\pgfqpoint{-319.224843in}{3.692935in}}%
\pgfpathclose%
\pgfusepath{fill}%
\end{pgfscope}%
\begin{pgfscope}%
\pgfpathrectangle{\pgfqpoint{10.668400in}{3.021449in}}{\pgfqpoint{2.188235in}{0.972632in}}%
\pgfusepath{clip}%
\pgfsetbuttcap%
\pgfsetmiterjoin%
\definecolor{currentfill}{rgb}{0.121569,0.466667,0.705882}%
\pgfsetfillcolor{currentfill}%
\pgfsetlinewidth{0.000000pt}%
\definecolor{currentstroke}{rgb}{0.000000,0.000000,0.000000}%
\pgfsetstrokecolor{currentstroke}%
\pgfsetstrokeopacity{0.000000}%
\pgfsetdash{}{0pt}%
\pgfpathmoveto{\pgfqpoint{-319.224843in}{3.694707in}}%
\pgfpathlineto{\pgfqpoint{8.301839in}{3.694707in}}%
\pgfpathlineto{\pgfqpoint{8.301839in}{3.701795in}}%
\pgfpathlineto{\pgfqpoint{-319.224843in}{3.701795in}}%
\pgfpathclose%
\pgfusepath{fill}%
\end{pgfscope}%
\begin{pgfscope}%
\pgfpathrectangle{\pgfqpoint{10.668400in}{3.021449in}}{\pgfqpoint{2.188235in}{0.972632in}}%
\pgfusepath{clip}%
\pgfsetbuttcap%
\pgfsetmiterjoin%
\definecolor{currentfill}{rgb}{0.121569,0.466667,0.705882}%
\pgfsetfillcolor{currentfill}%
\pgfsetlinewidth{0.000000pt}%
\definecolor{currentstroke}{rgb}{0.000000,0.000000,0.000000}%
\pgfsetstrokecolor{currentstroke}%
\pgfsetstrokeopacity{0.000000}%
\pgfsetdash{}{0pt}%
\pgfpathmoveto{\pgfqpoint{-319.224843in}{3.703567in}}%
\pgfpathlineto{\pgfqpoint{8.721785in}{3.703567in}}%
\pgfpathlineto{\pgfqpoint{8.721785in}{3.710655in}}%
\pgfpathlineto{\pgfqpoint{-319.224843in}{3.710655in}}%
\pgfpathclose%
\pgfusepath{fill}%
\end{pgfscope}%
\begin{pgfscope}%
\pgfpathrectangle{\pgfqpoint{10.668400in}{3.021449in}}{\pgfqpoint{2.188235in}{0.972632in}}%
\pgfusepath{clip}%
\pgfsetbuttcap%
\pgfsetmiterjoin%
\definecolor{currentfill}{rgb}{0.121569,0.466667,0.705882}%
\pgfsetfillcolor{currentfill}%
\pgfsetlinewidth{0.000000pt}%
\definecolor{currentstroke}{rgb}{0.000000,0.000000,0.000000}%
\pgfsetstrokecolor{currentstroke}%
\pgfsetstrokeopacity{0.000000}%
\pgfsetdash{}{0pt}%
\pgfpathmoveto{\pgfqpoint{-319.224843in}{3.712427in}}%
\pgfpathlineto{\pgfqpoint{9.223117in}{3.712427in}}%
\pgfpathlineto{\pgfqpoint{9.223117in}{3.719515in}}%
\pgfpathlineto{\pgfqpoint{-319.224843in}{3.719515in}}%
\pgfpathclose%
\pgfusepath{fill}%
\end{pgfscope}%
\begin{pgfscope}%
\pgfpathrectangle{\pgfqpoint{10.668400in}{3.021449in}}{\pgfqpoint{2.188235in}{0.972632in}}%
\pgfusepath{clip}%
\pgfsetbuttcap%
\pgfsetmiterjoin%
\definecolor{currentfill}{rgb}{0.121569,0.466667,0.705882}%
\pgfsetfillcolor{currentfill}%
\pgfsetlinewidth{0.000000pt}%
\definecolor{currentstroke}{rgb}{0.000000,0.000000,0.000000}%
\pgfsetstrokecolor{currentstroke}%
\pgfsetstrokeopacity{0.000000}%
\pgfsetdash{}{0pt}%
\pgfpathmoveto{\pgfqpoint{-319.224843in}{3.721287in}}%
\pgfpathlineto{\pgfqpoint{9.125517in}{3.721287in}}%
\pgfpathlineto{\pgfqpoint{9.125517in}{3.728375in}}%
\pgfpathlineto{\pgfqpoint{-319.224843in}{3.728375in}}%
\pgfpathclose%
\pgfusepath{fill}%
\end{pgfscope}%
\begin{pgfscope}%
\pgfpathrectangle{\pgfqpoint{10.668400in}{3.021449in}}{\pgfqpoint{2.188235in}{0.972632in}}%
\pgfusepath{clip}%
\pgfsetbuttcap%
\pgfsetmiterjoin%
\definecolor{currentfill}{rgb}{0.121569,0.466667,0.705882}%
\pgfsetfillcolor{currentfill}%
\pgfsetlinewidth{0.000000pt}%
\definecolor{currentstroke}{rgb}{0.000000,0.000000,0.000000}%
\pgfsetstrokecolor{currentstroke}%
\pgfsetstrokeopacity{0.000000}%
\pgfsetdash{}{0pt}%
\pgfpathmoveto{\pgfqpoint{-319.224843in}{3.730147in}}%
\pgfpathlineto{\pgfqpoint{9.096071in}{3.730147in}}%
\pgfpathlineto{\pgfqpoint{9.096071in}{3.737234in}}%
\pgfpathlineto{\pgfqpoint{-319.224843in}{3.737234in}}%
\pgfpathclose%
\pgfusepath{fill}%
\end{pgfscope}%
\begin{pgfscope}%
\pgfpathrectangle{\pgfqpoint{10.668400in}{3.021449in}}{\pgfqpoint{2.188235in}{0.972632in}}%
\pgfusepath{clip}%
\pgfsetbuttcap%
\pgfsetmiterjoin%
\definecolor{currentfill}{rgb}{0.121569,0.466667,0.705882}%
\pgfsetfillcolor{currentfill}%
\pgfsetlinewidth{0.000000pt}%
\definecolor{currentstroke}{rgb}{0.000000,0.000000,0.000000}%
\pgfsetstrokecolor{currentstroke}%
\pgfsetstrokeopacity{0.000000}%
\pgfsetdash{}{0pt}%
\pgfpathmoveto{\pgfqpoint{-319.224843in}{3.739006in}}%
\pgfpathlineto{\pgfqpoint{8.939520in}{3.739006in}}%
\pgfpathlineto{\pgfqpoint{8.939520in}{3.746094in}}%
\pgfpathlineto{\pgfqpoint{-319.224843in}{3.746094in}}%
\pgfpathclose%
\pgfusepath{fill}%
\end{pgfscope}%
\begin{pgfscope}%
\pgfpathrectangle{\pgfqpoint{10.668400in}{3.021449in}}{\pgfqpoint{2.188235in}{0.972632in}}%
\pgfusepath{clip}%
\pgfsetbuttcap%
\pgfsetmiterjoin%
\definecolor{currentfill}{rgb}{0.121569,0.466667,0.705882}%
\pgfsetfillcolor{currentfill}%
\pgfsetlinewidth{0.000000pt}%
\definecolor{currentstroke}{rgb}{0.000000,0.000000,0.000000}%
\pgfsetstrokecolor{currentstroke}%
\pgfsetstrokeopacity{0.000000}%
\pgfsetdash{}{0pt}%
\pgfpathmoveto{\pgfqpoint{-319.224843in}{3.747866in}}%
\pgfpathlineto{\pgfqpoint{9.218038in}{3.747866in}}%
\pgfpathlineto{\pgfqpoint{9.218038in}{3.754954in}}%
\pgfpathlineto{\pgfqpoint{-319.224843in}{3.754954in}}%
\pgfpathclose%
\pgfusepath{fill}%
\end{pgfscope}%
\begin{pgfscope}%
\pgfpathrectangle{\pgfqpoint{10.668400in}{3.021449in}}{\pgfqpoint{2.188235in}{0.972632in}}%
\pgfusepath{clip}%
\pgfsetbuttcap%
\pgfsetmiterjoin%
\definecolor{currentfill}{rgb}{0.121569,0.466667,0.705882}%
\pgfsetfillcolor{currentfill}%
\pgfsetlinewidth{0.000000pt}%
\definecolor{currentstroke}{rgb}{0.000000,0.000000,0.000000}%
\pgfsetstrokecolor{currentstroke}%
\pgfsetstrokeopacity{0.000000}%
\pgfsetdash{}{0pt}%
\pgfpathmoveto{\pgfqpoint{-319.224843in}{3.756726in}}%
\pgfpathlineto{\pgfqpoint{8.696099in}{3.756726in}}%
\pgfpathlineto{\pgfqpoint{8.696099in}{3.763814in}}%
\pgfpathlineto{\pgfqpoint{-319.224843in}{3.763814in}}%
\pgfpathclose%
\pgfusepath{fill}%
\end{pgfscope}%
\begin{pgfscope}%
\pgfpathrectangle{\pgfqpoint{10.668400in}{3.021449in}}{\pgfqpoint{2.188235in}{0.972632in}}%
\pgfusepath{clip}%
\pgfsetbuttcap%
\pgfsetmiterjoin%
\definecolor{currentfill}{rgb}{0.121569,0.466667,0.705882}%
\pgfsetfillcolor{currentfill}%
\pgfsetlinewidth{0.000000pt}%
\definecolor{currentstroke}{rgb}{0.000000,0.000000,0.000000}%
\pgfsetstrokecolor{currentstroke}%
\pgfsetstrokeopacity{0.000000}%
\pgfsetdash{}{0pt}%
\pgfpathmoveto{\pgfqpoint{-319.224843in}{3.765586in}}%
\pgfpathlineto{\pgfqpoint{8.988266in}{3.765586in}}%
\pgfpathlineto{\pgfqpoint{8.988266in}{3.772674in}}%
\pgfpathlineto{\pgfqpoint{-319.224843in}{3.772674in}}%
\pgfpathclose%
\pgfusepath{fill}%
\end{pgfscope}%
\begin{pgfscope}%
\pgfpathrectangle{\pgfqpoint{10.668400in}{3.021449in}}{\pgfqpoint{2.188235in}{0.972632in}}%
\pgfusepath{clip}%
\pgfsetbuttcap%
\pgfsetmiterjoin%
\definecolor{currentfill}{rgb}{0.121569,0.466667,0.705882}%
\pgfsetfillcolor{currentfill}%
\pgfsetlinewidth{0.000000pt}%
\definecolor{currentstroke}{rgb}{0.000000,0.000000,0.000000}%
\pgfsetstrokecolor{currentstroke}%
\pgfsetstrokeopacity{0.000000}%
\pgfsetdash{}{0pt}%
\pgfpathmoveto{\pgfqpoint{-319.224843in}{3.774446in}}%
\pgfpathlineto{\pgfqpoint{9.118382in}{3.774446in}}%
\pgfpathlineto{\pgfqpoint{9.118382in}{3.781534in}}%
\pgfpathlineto{\pgfqpoint{-319.224843in}{3.781534in}}%
\pgfpathclose%
\pgfusepath{fill}%
\end{pgfscope}%
\begin{pgfscope}%
\pgfpathrectangle{\pgfqpoint{10.668400in}{3.021449in}}{\pgfqpoint{2.188235in}{0.972632in}}%
\pgfusepath{clip}%
\pgfsetbuttcap%
\pgfsetmiterjoin%
\definecolor{currentfill}{rgb}{0.121569,0.466667,0.705882}%
\pgfsetfillcolor{currentfill}%
\pgfsetlinewidth{0.000000pt}%
\definecolor{currentstroke}{rgb}{0.000000,0.000000,0.000000}%
\pgfsetstrokecolor{currentstroke}%
\pgfsetstrokeopacity{0.000000}%
\pgfsetdash{}{0pt}%
\pgfpathmoveto{\pgfqpoint{-319.224843in}{3.783306in}}%
\pgfpathlineto{\pgfqpoint{9.224050in}{3.783306in}}%
\pgfpathlineto{\pgfqpoint{9.224050in}{3.790393in}}%
\pgfpathlineto{\pgfqpoint{-319.224843in}{3.790393in}}%
\pgfpathclose%
\pgfusepath{fill}%
\end{pgfscope}%
\begin{pgfscope}%
\pgfpathrectangle{\pgfqpoint{10.668400in}{3.021449in}}{\pgfqpoint{2.188235in}{0.972632in}}%
\pgfusepath{clip}%
\pgfsetbuttcap%
\pgfsetmiterjoin%
\definecolor{currentfill}{rgb}{0.121569,0.466667,0.705882}%
\pgfsetfillcolor{currentfill}%
\pgfsetlinewidth{0.000000pt}%
\definecolor{currentstroke}{rgb}{0.000000,0.000000,0.000000}%
\pgfsetstrokecolor{currentstroke}%
\pgfsetstrokeopacity{0.000000}%
\pgfsetdash{}{0pt}%
\pgfpathmoveto{\pgfqpoint{-319.224843in}{3.792165in}}%
\pgfpathlineto{\pgfqpoint{8.861082in}{3.792165in}}%
\pgfpathlineto{\pgfqpoint{8.861082in}{3.799253in}}%
\pgfpathlineto{\pgfqpoint{-319.224843in}{3.799253in}}%
\pgfpathclose%
\pgfusepath{fill}%
\end{pgfscope}%
\begin{pgfscope}%
\pgfpathrectangle{\pgfqpoint{10.668400in}{3.021449in}}{\pgfqpoint{2.188235in}{0.972632in}}%
\pgfusepath{clip}%
\pgfsetbuttcap%
\pgfsetmiterjoin%
\definecolor{currentfill}{rgb}{0.121569,0.466667,0.705882}%
\pgfsetfillcolor{currentfill}%
\pgfsetlinewidth{0.000000pt}%
\definecolor{currentstroke}{rgb}{0.000000,0.000000,0.000000}%
\pgfsetstrokecolor{currentstroke}%
\pgfsetstrokeopacity{0.000000}%
\pgfsetdash{}{0pt}%
\pgfpathmoveto{\pgfqpoint{-319.224843in}{3.801025in}}%
\pgfpathlineto{\pgfqpoint{9.022491in}{3.801025in}}%
\pgfpathlineto{\pgfqpoint{9.022491in}{3.808113in}}%
\pgfpathlineto{\pgfqpoint{-319.224843in}{3.808113in}}%
\pgfpathclose%
\pgfusepath{fill}%
\end{pgfscope}%
\begin{pgfscope}%
\pgfpathrectangle{\pgfqpoint{10.668400in}{3.021449in}}{\pgfqpoint{2.188235in}{0.972632in}}%
\pgfusepath{clip}%
\pgfsetbuttcap%
\pgfsetmiterjoin%
\definecolor{currentfill}{rgb}{0.121569,0.466667,0.705882}%
\pgfsetfillcolor{currentfill}%
\pgfsetlinewidth{0.000000pt}%
\definecolor{currentstroke}{rgb}{0.000000,0.000000,0.000000}%
\pgfsetstrokecolor{currentstroke}%
\pgfsetstrokeopacity{0.000000}%
\pgfsetdash{}{0pt}%
\pgfpathmoveto{\pgfqpoint{-319.224843in}{3.809885in}}%
\pgfpathlineto{\pgfqpoint{8.930714in}{3.809885in}}%
\pgfpathlineto{\pgfqpoint{8.930714in}{3.816973in}}%
\pgfpathlineto{\pgfqpoint{-319.224843in}{3.816973in}}%
\pgfpathclose%
\pgfusepath{fill}%
\end{pgfscope}%
\begin{pgfscope}%
\pgfpathrectangle{\pgfqpoint{10.668400in}{3.021449in}}{\pgfqpoint{2.188235in}{0.972632in}}%
\pgfusepath{clip}%
\pgfsetbuttcap%
\pgfsetmiterjoin%
\definecolor{currentfill}{rgb}{0.121569,0.466667,0.705882}%
\pgfsetfillcolor{currentfill}%
\pgfsetlinewidth{0.000000pt}%
\definecolor{currentstroke}{rgb}{0.000000,0.000000,0.000000}%
\pgfsetstrokecolor{currentstroke}%
\pgfsetstrokeopacity{0.000000}%
\pgfsetdash{}{0pt}%
\pgfpathmoveto{\pgfqpoint{-319.224843in}{3.818745in}}%
\pgfpathlineto{\pgfqpoint{9.183285in}{3.818745in}}%
\pgfpathlineto{\pgfqpoint{9.183285in}{3.825833in}}%
\pgfpathlineto{\pgfqpoint{-319.224843in}{3.825833in}}%
\pgfpathclose%
\pgfusepath{fill}%
\end{pgfscope}%
\begin{pgfscope}%
\pgfpathrectangle{\pgfqpoint{10.668400in}{3.021449in}}{\pgfqpoint{2.188235in}{0.972632in}}%
\pgfusepath{clip}%
\pgfsetbuttcap%
\pgfsetmiterjoin%
\definecolor{currentfill}{rgb}{0.121569,0.466667,0.705882}%
\pgfsetfillcolor{currentfill}%
\pgfsetlinewidth{0.000000pt}%
\definecolor{currentstroke}{rgb}{0.000000,0.000000,0.000000}%
\pgfsetstrokecolor{currentstroke}%
\pgfsetstrokeopacity{0.000000}%
\pgfsetdash{}{0pt}%
\pgfpathmoveto{\pgfqpoint{-319.224843in}{3.827605in}}%
\pgfpathlineto{\pgfqpoint{9.068263in}{3.827605in}}%
\pgfpathlineto{\pgfqpoint{9.068263in}{3.834693in}}%
\pgfpathlineto{\pgfqpoint{-319.224843in}{3.834693in}}%
\pgfpathclose%
\pgfusepath{fill}%
\end{pgfscope}%
\begin{pgfscope}%
\pgfpathrectangle{\pgfqpoint{10.668400in}{3.021449in}}{\pgfqpoint{2.188235in}{0.972632in}}%
\pgfusepath{clip}%
\pgfsetbuttcap%
\pgfsetmiterjoin%
\definecolor{currentfill}{rgb}{0.121569,0.466667,0.705882}%
\pgfsetfillcolor{currentfill}%
\pgfsetlinewidth{0.000000pt}%
\definecolor{currentstroke}{rgb}{0.000000,0.000000,0.000000}%
\pgfsetstrokecolor{currentstroke}%
\pgfsetstrokeopacity{0.000000}%
\pgfsetdash{}{0pt}%
\pgfpathmoveto{\pgfqpoint{-319.224843in}{3.836465in}}%
\pgfpathlineto{\pgfqpoint{8.995585in}{3.836465in}}%
\pgfpathlineto{\pgfqpoint{8.995585in}{3.843552in}}%
\pgfpathlineto{\pgfqpoint{-319.224843in}{3.843552in}}%
\pgfpathclose%
\pgfusepath{fill}%
\end{pgfscope}%
\begin{pgfscope}%
\pgfpathrectangle{\pgfqpoint{10.668400in}{3.021449in}}{\pgfqpoint{2.188235in}{0.972632in}}%
\pgfusepath{clip}%
\pgfsetbuttcap%
\pgfsetmiterjoin%
\definecolor{currentfill}{rgb}{0.121569,0.466667,0.705882}%
\pgfsetfillcolor{currentfill}%
\pgfsetlinewidth{0.000000pt}%
\definecolor{currentstroke}{rgb}{0.000000,0.000000,0.000000}%
\pgfsetstrokecolor{currentstroke}%
\pgfsetstrokeopacity{0.000000}%
\pgfsetdash{}{0pt}%
\pgfpathmoveto{\pgfqpoint{-319.224843in}{3.845324in}}%
\pgfpathlineto{\pgfqpoint{8.748825in}{3.845324in}}%
\pgfpathlineto{\pgfqpoint{8.748825in}{3.852412in}}%
\pgfpathlineto{\pgfqpoint{-319.224843in}{3.852412in}}%
\pgfpathclose%
\pgfusepath{fill}%
\end{pgfscope}%
\begin{pgfscope}%
\pgfpathrectangle{\pgfqpoint{10.668400in}{3.021449in}}{\pgfqpoint{2.188235in}{0.972632in}}%
\pgfusepath{clip}%
\pgfsetbuttcap%
\pgfsetmiterjoin%
\definecolor{currentfill}{rgb}{0.121569,0.466667,0.705882}%
\pgfsetfillcolor{currentfill}%
\pgfsetlinewidth{0.000000pt}%
\definecolor{currentstroke}{rgb}{0.000000,0.000000,0.000000}%
\pgfsetstrokecolor{currentstroke}%
\pgfsetstrokeopacity{0.000000}%
\pgfsetdash{}{0pt}%
\pgfpathmoveto{\pgfqpoint{-319.224843in}{3.854184in}}%
\pgfpathlineto{\pgfqpoint{9.142526in}{3.854184in}}%
\pgfpathlineto{\pgfqpoint{9.142526in}{3.861272in}}%
\pgfpathlineto{\pgfqpoint{-319.224843in}{3.861272in}}%
\pgfpathclose%
\pgfusepath{fill}%
\end{pgfscope}%
\begin{pgfscope}%
\pgfpathrectangle{\pgfqpoint{10.668400in}{3.021449in}}{\pgfqpoint{2.188235in}{0.972632in}}%
\pgfusepath{clip}%
\pgfsetbuttcap%
\pgfsetmiterjoin%
\definecolor{currentfill}{rgb}{0.121569,0.466667,0.705882}%
\pgfsetfillcolor{currentfill}%
\pgfsetlinewidth{0.000000pt}%
\definecolor{currentstroke}{rgb}{0.000000,0.000000,0.000000}%
\pgfsetstrokecolor{currentstroke}%
\pgfsetstrokeopacity{0.000000}%
\pgfsetdash{}{0pt}%
\pgfpathmoveto{\pgfqpoint{-319.224843in}{3.863044in}}%
\pgfpathlineto{\pgfqpoint{8.880970in}{3.863044in}}%
\pgfpathlineto{\pgfqpoint{8.880970in}{3.870132in}}%
\pgfpathlineto{\pgfqpoint{-319.224843in}{3.870132in}}%
\pgfpathclose%
\pgfusepath{fill}%
\end{pgfscope}%
\begin{pgfscope}%
\pgfpathrectangle{\pgfqpoint{10.668400in}{3.021449in}}{\pgfqpoint{2.188235in}{0.972632in}}%
\pgfusepath{clip}%
\pgfsetbuttcap%
\pgfsetmiterjoin%
\definecolor{currentfill}{rgb}{0.121569,0.466667,0.705882}%
\pgfsetfillcolor{currentfill}%
\pgfsetlinewidth{0.000000pt}%
\definecolor{currentstroke}{rgb}{0.000000,0.000000,0.000000}%
\pgfsetstrokecolor{currentstroke}%
\pgfsetstrokeopacity{0.000000}%
\pgfsetdash{}{0pt}%
\pgfpathmoveto{\pgfqpoint{-319.224843in}{3.871904in}}%
\pgfpathlineto{\pgfqpoint{8.923011in}{3.871904in}}%
\pgfpathlineto{\pgfqpoint{8.923011in}{3.878992in}}%
\pgfpathlineto{\pgfqpoint{-319.224843in}{3.878992in}}%
\pgfpathclose%
\pgfusepath{fill}%
\end{pgfscope}%
\begin{pgfscope}%
\pgfpathrectangle{\pgfqpoint{10.668400in}{3.021449in}}{\pgfqpoint{2.188235in}{0.972632in}}%
\pgfusepath{clip}%
\pgfsetbuttcap%
\pgfsetmiterjoin%
\definecolor{currentfill}{rgb}{0.121569,0.466667,0.705882}%
\pgfsetfillcolor{currentfill}%
\pgfsetlinewidth{0.000000pt}%
\definecolor{currentstroke}{rgb}{0.000000,0.000000,0.000000}%
\pgfsetstrokecolor{currentstroke}%
\pgfsetstrokeopacity{0.000000}%
\pgfsetdash{}{0pt}%
\pgfpathmoveto{\pgfqpoint{-319.224843in}{3.880764in}}%
\pgfpathlineto{\pgfqpoint{9.299276in}{3.880764in}}%
\pgfpathlineto{\pgfqpoint{9.299276in}{3.887852in}}%
\pgfpathlineto{\pgfqpoint{-319.224843in}{3.887852in}}%
\pgfpathclose%
\pgfusepath{fill}%
\end{pgfscope}%
\begin{pgfscope}%
\pgfpathrectangle{\pgfqpoint{10.668400in}{3.021449in}}{\pgfqpoint{2.188235in}{0.972632in}}%
\pgfusepath{clip}%
\pgfsetbuttcap%
\pgfsetmiterjoin%
\definecolor{currentfill}{rgb}{0.121569,0.466667,0.705882}%
\pgfsetfillcolor{currentfill}%
\pgfsetlinewidth{0.000000pt}%
\definecolor{currentstroke}{rgb}{0.000000,0.000000,0.000000}%
\pgfsetstrokecolor{currentstroke}%
\pgfsetstrokeopacity{0.000000}%
\pgfsetdash{}{0pt}%
\pgfpathmoveto{\pgfqpoint{-319.224843in}{3.889623in}}%
\pgfpathlineto{\pgfqpoint{9.096842in}{3.889623in}}%
\pgfpathlineto{\pgfqpoint{9.096842in}{3.896711in}}%
\pgfpathlineto{\pgfqpoint{-319.224843in}{3.896711in}}%
\pgfpathclose%
\pgfusepath{fill}%
\end{pgfscope}%
\begin{pgfscope}%
\pgfpathrectangle{\pgfqpoint{10.668400in}{3.021449in}}{\pgfqpoint{2.188235in}{0.972632in}}%
\pgfusepath{clip}%
\pgfsetbuttcap%
\pgfsetmiterjoin%
\definecolor{currentfill}{rgb}{0.121569,0.466667,0.705882}%
\pgfsetfillcolor{currentfill}%
\pgfsetlinewidth{0.000000pt}%
\definecolor{currentstroke}{rgb}{0.000000,0.000000,0.000000}%
\pgfsetstrokecolor{currentstroke}%
\pgfsetstrokeopacity{0.000000}%
\pgfsetdash{}{0pt}%
\pgfpathmoveto{\pgfqpoint{-319.224843in}{3.898483in}}%
\pgfpathlineto{\pgfqpoint{8.858909in}{3.898483in}}%
\pgfpathlineto{\pgfqpoint{8.858909in}{3.905571in}}%
\pgfpathlineto{\pgfqpoint{-319.224843in}{3.905571in}}%
\pgfpathclose%
\pgfusepath{fill}%
\end{pgfscope}%
\begin{pgfscope}%
\pgfpathrectangle{\pgfqpoint{10.668400in}{3.021449in}}{\pgfqpoint{2.188235in}{0.972632in}}%
\pgfusepath{clip}%
\pgfsetbuttcap%
\pgfsetmiterjoin%
\definecolor{currentfill}{rgb}{0.121569,0.466667,0.705882}%
\pgfsetfillcolor{currentfill}%
\pgfsetlinewidth{0.000000pt}%
\definecolor{currentstroke}{rgb}{0.000000,0.000000,0.000000}%
\pgfsetstrokecolor{currentstroke}%
\pgfsetstrokeopacity{0.000000}%
\pgfsetdash{}{0pt}%
\pgfpathmoveto{\pgfqpoint{-319.224843in}{3.907343in}}%
\pgfpathlineto{\pgfqpoint{9.059992in}{3.907343in}}%
\pgfpathlineto{\pgfqpoint{9.059992in}{3.914431in}}%
\pgfpathlineto{\pgfqpoint{-319.224843in}{3.914431in}}%
\pgfpathclose%
\pgfusepath{fill}%
\end{pgfscope}%
\begin{pgfscope}%
\pgfpathrectangle{\pgfqpoint{10.668400in}{3.021449in}}{\pgfqpoint{2.188235in}{0.972632in}}%
\pgfusepath{clip}%
\pgfsetbuttcap%
\pgfsetmiterjoin%
\definecolor{currentfill}{rgb}{0.121569,0.466667,0.705882}%
\pgfsetfillcolor{currentfill}%
\pgfsetlinewidth{0.000000pt}%
\definecolor{currentstroke}{rgb}{0.000000,0.000000,0.000000}%
\pgfsetstrokecolor{currentstroke}%
\pgfsetstrokeopacity{0.000000}%
\pgfsetdash{}{0pt}%
\pgfpathmoveto{\pgfqpoint{-319.224843in}{3.916203in}}%
\pgfpathlineto{\pgfqpoint{8.767411in}{3.916203in}}%
\pgfpathlineto{\pgfqpoint{8.767411in}{3.923291in}}%
\pgfpathlineto{\pgfqpoint{-319.224843in}{3.923291in}}%
\pgfpathclose%
\pgfusepath{fill}%
\end{pgfscope}%
\begin{pgfscope}%
\pgfpathrectangle{\pgfqpoint{10.668400in}{3.021449in}}{\pgfqpoint{2.188235in}{0.972632in}}%
\pgfusepath{clip}%
\pgfsetbuttcap%
\pgfsetmiterjoin%
\definecolor{currentfill}{rgb}{0.121569,0.466667,0.705882}%
\pgfsetfillcolor{currentfill}%
\pgfsetlinewidth{0.000000pt}%
\definecolor{currentstroke}{rgb}{0.000000,0.000000,0.000000}%
\pgfsetstrokecolor{currentstroke}%
\pgfsetstrokeopacity{0.000000}%
\pgfsetdash{}{0pt}%
\pgfpathmoveto{\pgfqpoint{-319.224843in}{3.925063in}}%
\pgfpathlineto{\pgfqpoint{9.073417in}{3.925063in}}%
\pgfpathlineto{\pgfqpoint{9.073417in}{3.932151in}}%
\pgfpathlineto{\pgfqpoint{-319.224843in}{3.932151in}}%
\pgfpathclose%
\pgfusepath{fill}%
\end{pgfscope}%
\begin{pgfscope}%
\pgfpathrectangle{\pgfqpoint{10.668400in}{3.021449in}}{\pgfqpoint{2.188235in}{0.972632in}}%
\pgfusepath{clip}%
\pgfsetbuttcap%
\pgfsetmiterjoin%
\definecolor{currentfill}{rgb}{0.121569,0.466667,0.705882}%
\pgfsetfillcolor{currentfill}%
\pgfsetlinewidth{0.000000pt}%
\definecolor{currentstroke}{rgb}{0.000000,0.000000,0.000000}%
\pgfsetstrokecolor{currentstroke}%
\pgfsetstrokeopacity{0.000000}%
\pgfsetdash{}{0pt}%
\pgfpathmoveto{\pgfqpoint{-319.224843in}{3.933923in}}%
\pgfpathlineto{\pgfqpoint{8.726180in}{3.933923in}}%
\pgfpathlineto{\pgfqpoint{8.726180in}{3.941010in}}%
\pgfpathlineto{\pgfqpoint{-319.224843in}{3.941010in}}%
\pgfpathclose%
\pgfusepath{fill}%
\end{pgfscope}%
\begin{pgfscope}%
\pgfpathrectangle{\pgfqpoint{10.668400in}{3.021449in}}{\pgfqpoint{2.188235in}{0.972632in}}%
\pgfusepath{clip}%
\pgfsetbuttcap%
\pgfsetmiterjoin%
\definecolor{currentfill}{rgb}{0.121569,0.466667,0.705882}%
\pgfsetfillcolor{currentfill}%
\pgfsetlinewidth{0.000000pt}%
\definecolor{currentstroke}{rgb}{0.000000,0.000000,0.000000}%
\pgfsetstrokecolor{currentstroke}%
\pgfsetstrokeopacity{0.000000}%
\pgfsetdash{}{0pt}%
\pgfpathmoveto{\pgfqpoint{-319.224843in}{3.942782in}}%
\pgfpathlineto{\pgfqpoint{8.780373in}{3.942782in}}%
\pgfpathlineto{\pgfqpoint{8.780373in}{3.949870in}}%
\pgfpathlineto{\pgfqpoint{-319.224843in}{3.949870in}}%
\pgfpathclose%
\pgfusepath{fill}%
\end{pgfscope}%
\begin{pgfscope}%
\pgfsetbuttcap%
\pgfsetroundjoin%
\definecolor{currentfill}{rgb}{0.000000,0.000000,0.000000}%
\pgfsetfillcolor{currentfill}%
\pgfsetlinewidth{0.803000pt}%
\definecolor{currentstroke}{rgb}{0.000000,0.000000,0.000000}%
\pgfsetstrokecolor{currentstroke}%
\pgfsetdash{}{0pt}%
\pgfsys@defobject{currentmarker}{\pgfqpoint{0.000000in}{-0.048611in}}{\pgfqpoint{0.000000in}{0.000000in}}{%
\pgfpathmoveto{\pgfqpoint{0.000000in}{0.000000in}}%
\pgfpathlineto{\pgfqpoint{0.000000in}{-0.048611in}}%
\pgfusepath{stroke,fill}%
}%
\begin{pgfscope}%
\pgfsys@transformshift{11.218880in}{3.021449in}%
\pgfsys@useobject{currentmarker}{}%
\end{pgfscope}%
\end{pgfscope}%
\begin{pgfscope}%
\pgfsetbuttcap%
\pgfsetroundjoin%
\definecolor{currentfill}{rgb}{0.000000,0.000000,0.000000}%
\pgfsetfillcolor{currentfill}%
\pgfsetlinewidth{0.803000pt}%
\definecolor{currentstroke}{rgb}{0.000000,0.000000,0.000000}%
\pgfsetstrokecolor{currentstroke}%
\pgfsetdash{}{0pt}%
\pgfsys@defobject{currentmarker}{\pgfqpoint{0.000000in}{-0.048611in}}{\pgfqpoint{0.000000in}{0.000000in}}{%
\pgfpathmoveto{\pgfqpoint{0.000000in}{0.000000in}}%
\pgfpathlineto{\pgfqpoint{0.000000in}{-0.048611in}}%
\pgfusepath{stroke,fill}%
}%
\begin{pgfscope}%
\pgfsys@transformshift{11.881756in}{3.021449in}%
\pgfsys@useobject{currentmarker}{}%
\end{pgfscope}%
\end{pgfscope}%
\begin{pgfscope}%
\pgfsetbuttcap%
\pgfsetroundjoin%
\definecolor{currentfill}{rgb}{0.000000,0.000000,0.000000}%
\pgfsetfillcolor{currentfill}%
\pgfsetlinewidth{0.803000pt}%
\definecolor{currentstroke}{rgb}{0.000000,0.000000,0.000000}%
\pgfsetstrokecolor{currentstroke}%
\pgfsetdash{}{0pt}%
\pgfsys@defobject{currentmarker}{\pgfqpoint{0.000000in}{-0.048611in}}{\pgfqpoint{0.000000in}{0.000000in}}{%
\pgfpathmoveto{\pgfqpoint{0.000000in}{0.000000in}}%
\pgfpathlineto{\pgfqpoint{0.000000in}{-0.048611in}}%
\pgfusepath{stroke,fill}%
}%
\begin{pgfscope}%
\pgfsys@transformshift{12.544632in}{3.021449in}%
\pgfsys@useobject{currentmarker}{}%
\end{pgfscope}%
\end{pgfscope}%
\begin{pgfscope}%
\pgfsetbuttcap%
\pgfsetroundjoin%
\definecolor{currentfill}{rgb}{0.000000,0.000000,0.000000}%
\pgfsetfillcolor{currentfill}%
\pgfsetlinewidth{0.803000pt}%
\definecolor{currentstroke}{rgb}{0.000000,0.000000,0.000000}%
\pgfsetstrokecolor{currentstroke}%
\pgfsetdash{}{0pt}%
\pgfsys@defobject{currentmarker}{\pgfqpoint{-0.048611in}{0.000000in}}{\pgfqpoint{0.000000in}{0.000000in}}{%
\pgfpathmoveto{\pgfqpoint{0.000000in}{0.000000in}}%
\pgfpathlineto{\pgfqpoint{-0.048611in}{0.000000in}}%
\pgfusepath{stroke,fill}%
}%
\begin{pgfscope}%
\pgfsys@transformshift{10.668400in}{3.069204in}%
\pgfsys@useobject{currentmarker}{}%
\end{pgfscope}%
\end{pgfscope}%
\begin{pgfscope}%
\pgftext[x=10.501733in,y=3.016442in,left,base]{\rmfamily\fontsize{10.000000}{12.000000}\selectfont \(\displaystyle 0\)}%
\end{pgfscope}%
\begin{pgfscope}%
\pgfsetbuttcap%
\pgfsetroundjoin%
\definecolor{currentfill}{rgb}{0.000000,0.000000,0.000000}%
\pgfsetfillcolor{currentfill}%
\pgfsetlinewidth{0.803000pt}%
\definecolor{currentstroke}{rgb}{0.000000,0.000000,0.000000}%
\pgfsetstrokecolor{currentstroke}%
\pgfsetdash{}{0pt}%
\pgfsys@defobject{currentmarker}{\pgfqpoint{-0.048611in}{0.000000in}}{\pgfqpoint{0.000000in}{0.000000in}}{%
\pgfpathmoveto{\pgfqpoint{0.000000in}{0.000000in}}%
\pgfpathlineto{\pgfqpoint{-0.048611in}{0.000000in}}%
\pgfusepath{stroke,fill}%
}%
\begin{pgfscope}%
\pgfsys@transformshift{10.668400in}{3.512195in}%
\pgfsys@useobject{currentmarker}{}%
\end{pgfscope}%
\end{pgfscope}%
\begin{pgfscope}%
\pgftext[x=10.432288in,y=3.459433in,left,base]{\rmfamily\fontsize{10.000000}{12.000000}\selectfont \(\displaystyle 50\)}%
\end{pgfscope}%
\begin{pgfscope}%
\pgfsetbuttcap%
\pgfsetroundjoin%
\definecolor{currentfill}{rgb}{0.000000,0.000000,0.000000}%
\pgfsetfillcolor{currentfill}%
\pgfsetlinewidth{0.803000pt}%
\definecolor{currentstroke}{rgb}{0.000000,0.000000,0.000000}%
\pgfsetstrokecolor{currentstroke}%
\pgfsetdash{}{0pt}%
\pgfsys@defobject{currentmarker}{\pgfqpoint{-0.048611in}{0.000000in}}{\pgfqpoint{0.000000in}{0.000000in}}{%
\pgfpathmoveto{\pgfqpoint{0.000000in}{0.000000in}}%
\pgfpathlineto{\pgfqpoint{-0.048611in}{0.000000in}}%
\pgfusepath{stroke,fill}%
}%
\begin{pgfscope}%
\pgfsys@transformshift{10.668400in}{3.955186in}%
\pgfsys@useobject{currentmarker}{}%
\end{pgfscope}%
\end{pgfscope}%
\begin{pgfscope}%
\pgftext[x=10.362844in,y=3.902425in,left,base]{\rmfamily\fontsize{10.000000}{12.000000}\selectfont \(\displaystyle 100\)}%
\end{pgfscope}%
\begin{pgfscope}%
\pgftext[x=10.307288in,y=3.507765in,,bottom,rotate=90.000000]{\rmfamily\fontsize{10.000000}{12.000000}\selectfont \(\displaystyle j\)}%
\end{pgfscope}%
\begin{pgfscope}%
\pgfsetrectcap%
\pgfsetmiterjoin%
\pgfsetlinewidth{0.803000pt}%
\definecolor{currentstroke}{rgb}{0.000000,0.000000,0.000000}%
\pgfsetstrokecolor{currentstroke}%
\pgfsetdash{}{0pt}%
\pgfpathmoveto{\pgfqpoint{10.668400in}{3.021449in}}%
\pgfpathlineto{\pgfqpoint{10.668400in}{3.994081in}}%
\pgfusepath{stroke}%
\end{pgfscope}%
\begin{pgfscope}%
\pgfsetrectcap%
\pgfsetmiterjoin%
\pgfsetlinewidth{0.803000pt}%
\definecolor{currentstroke}{rgb}{0.000000,0.000000,0.000000}%
\pgfsetstrokecolor{currentstroke}%
\pgfsetdash{}{0pt}%
\pgfpathmoveto{\pgfqpoint{12.856635in}{3.021449in}}%
\pgfpathlineto{\pgfqpoint{12.856635in}{3.994081in}}%
\pgfusepath{stroke}%
\end{pgfscope}%
\begin{pgfscope}%
\pgfsetrectcap%
\pgfsetmiterjoin%
\pgfsetlinewidth{0.803000pt}%
\definecolor{currentstroke}{rgb}{0.000000,0.000000,0.000000}%
\pgfsetstrokecolor{currentstroke}%
\pgfsetdash{}{0pt}%
\pgfpathmoveto{\pgfqpoint{10.668400in}{3.021449in}}%
\pgfpathlineto{\pgfqpoint{12.856635in}{3.021449in}}%
\pgfusepath{stroke}%
\end{pgfscope}%
\begin{pgfscope}%
\pgfsetrectcap%
\pgfsetmiterjoin%
\pgfsetlinewidth{0.803000pt}%
\definecolor{currentstroke}{rgb}{0.000000,0.000000,0.000000}%
\pgfsetstrokecolor{currentstroke}%
\pgfsetdash{}{0pt}%
\pgfpathmoveto{\pgfqpoint{10.668400in}{3.994081in}}%
\pgfpathlineto{\pgfqpoint{12.856635in}{3.994081in}}%
\pgfusepath{stroke}%
\end{pgfscope}%
\begin{pgfscope}%
\pgfsetbuttcap%
\pgfsetmiterjoin%
\definecolor{currentfill}{rgb}{1.000000,1.000000,1.000000}%
\pgfsetfillcolor{currentfill}%
\pgfsetlinewidth{0.000000pt}%
\definecolor{currentstroke}{rgb}{0.000000,0.000000,0.000000}%
\pgfsetstrokecolor{currentstroke}%
\pgfsetstrokeopacity{0.000000}%
\pgfsetdash{}{0pt}%
\pgfpathmoveto{\pgfqpoint{0.456635in}{1.805660in}}%
\pgfpathlineto{\pgfqpoint{4.833106in}{1.805660in}}%
\pgfpathlineto{\pgfqpoint{4.833106in}{2.778291in}}%
\pgfpathlineto{\pgfqpoint{0.456635in}{2.778291in}}%
\pgfpathclose%
\pgfusepath{fill}%
\end{pgfscope}%
\begin{pgfscope}%
\pgfpathrectangle{\pgfqpoint{0.456635in}{1.805660in}}{\pgfqpoint{4.376471in}{0.972632in}}%
\pgfusepath{clip}%
\pgfsetbuttcap%
\pgfsetroundjoin%
\definecolor{currentfill}{rgb}{1.000000,0.000000,0.000000}%
\pgfsetfillcolor{currentfill}%
\pgfsetlinewidth{2.007500pt}%
\definecolor{currentstroke}{rgb}{1.000000,0.000000,0.000000}%
\pgfsetstrokecolor{currentstroke}%
\pgfsetdash{}{0pt}%
\pgfpathmoveto{\pgfqpoint{2.601594in}{1.959527in}}%
\pgfpathlineto{\pgfqpoint{2.684927in}{1.959527in}}%
\pgfpathmoveto{\pgfqpoint{2.643260in}{1.917861in}}%
\pgfpathlineto{\pgfqpoint{2.643260in}{2.001194in}}%
\pgfusepath{stroke,fill}%
\end{pgfscope}%
\begin{pgfscope}%
\pgfpathrectangle{\pgfqpoint{0.456635in}{1.805660in}}{\pgfqpoint{4.376471in}{0.972632in}}%
\pgfusepath{clip}%
\pgfsetbuttcap%
\pgfsetroundjoin%
\definecolor{currentfill}{rgb}{1.000000,0.000000,0.000000}%
\pgfsetfillcolor{currentfill}%
\pgfsetlinewidth{2.007500pt}%
\definecolor{currentstroke}{rgb}{1.000000,0.000000,0.000000}%
\pgfsetstrokecolor{currentstroke}%
\pgfsetdash{}{0pt}%
\pgfpathmoveto{\pgfqpoint{4.618881in}{2.618629in}}%
\pgfpathlineto{\pgfqpoint{4.702215in}{2.618629in}}%
\pgfpathmoveto{\pgfqpoint{4.660548in}{2.576962in}}%
\pgfpathlineto{\pgfqpoint{4.660548in}{2.660296in}}%
\pgfusepath{stroke,fill}%
\end{pgfscope}%
\begin{pgfscope}%
\pgfpathrectangle{\pgfqpoint{0.456635in}{1.805660in}}{\pgfqpoint{4.376471in}{0.972632in}}%
\pgfusepath{clip}%
\pgfsetbuttcap%
\pgfsetroundjoin%
\definecolor{currentfill}{rgb}{1.000000,0.000000,0.000000}%
\pgfsetfillcolor{currentfill}%
\pgfsetlinewidth{2.007500pt}%
\definecolor{currentstroke}{rgb}{1.000000,0.000000,0.000000}%
\pgfsetstrokecolor{currentstroke}%
\pgfsetdash{}{0pt}%
\pgfpathmoveto{\pgfqpoint{3.853103in}{1.994690in}}%
\pgfpathlineto{\pgfqpoint{3.936436in}{1.994690in}}%
\pgfpathmoveto{\pgfqpoint{3.894769in}{1.953024in}}%
\pgfpathlineto{\pgfqpoint{3.894769in}{2.036357in}}%
\pgfusepath{stroke,fill}%
\end{pgfscope}%
\begin{pgfscope}%
\pgfpathrectangle{\pgfqpoint{0.456635in}{1.805660in}}{\pgfqpoint{4.376471in}{0.972632in}}%
\pgfusepath{clip}%
\pgfsetbuttcap%
\pgfsetroundjoin%
\definecolor{currentfill}{rgb}{1.000000,0.000000,0.000000}%
\pgfsetfillcolor{currentfill}%
\pgfsetlinewidth{2.007500pt}%
\definecolor{currentstroke}{rgb}{1.000000,0.000000,0.000000}%
\pgfsetstrokecolor{currentstroke}%
\pgfsetdash{}{0pt}%
\pgfpathmoveto{\pgfqpoint{3.386272in}{2.343070in}}%
\pgfpathlineto{\pgfqpoint{3.469605in}{2.343070in}}%
\pgfpathmoveto{\pgfqpoint{3.427938in}{2.301403in}}%
\pgfpathlineto{\pgfqpoint{3.427938in}{2.384737in}}%
\pgfusepath{stroke,fill}%
\end{pgfscope}%
\begin{pgfscope}%
\pgfpathrectangle{\pgfqpoint{0.456635in}{1.805660in}}{\pgfqpoint{4.376471in}{0.972632in}}%
\pgfusepath{clip}%
\pgfsetbuttcap%
\pgfsetroundjoin%
\definecolor{currentfill}{rgb}{1.000000,0.000000,0.000000}%
\pgfsetfillcolor{currentfill}%
\pgfsetlinewidth{2.007500pt}%
\definecolor{currentstroke}{rgb}{1.000000,0.000000,0.000000}%
\pgfsetstrokecolor{currentstroke}%
\pgfsetdash{}{0pt}%
\pgfpathmoveto{\pgfqpoint{1.836512in}{2.187424in}}%
\pgfpathlineto{\pgfqpoint{1.919845in}{2.187424in}}%
\pgfpathmoveto{\pgfqpoint{1.878178in}{2.145757in}}%
\pgfpathlineto{\pgfqpoint{1.878178in}{2.229090in}}%
\pgfusepath{stroke,fill}%
\end{pgfscope}%
\begin{pgfscope}%
\pgfpathrectangle{\pgfqpoint{0.456635in}{1.805660in}}{\pgfqpoint{4.376471in}{0.972632in}}%
\pgfusepath{clip}%
\pgfsetbuttcap%
\pgfsetroundjoin%
\definecolor{currentfill}{rgb}{1.000000,0.000000,0.000000}%
\pgfsetfillcolor{currentfill}%
\pgfsetlinewidth{2.007500pt}%
\definecolor{currentstroke}{rgb}{1.000000,0.000000,0.000000}%
\pgfsetstrokecolor{currentstroke}%
\pgfsetdash{}{0pt}%
\pgfpathmoveto{\pgfqpoint{1.836427in}{2.435997in}}%
\pgfpathlineto{\pgfqpoint{1.919760in}{2.435997in}}%
\pgfpathmoveto{\pgfqpoint{1.878094in}{2.394330in}}%
\pgfpathlineto{\pgfqpoint{1.878094in}{2.477664in}}%
\pgfusepath{stroke,fill}%
\end{pgfscope}%
\begin{pgfscope}%
\pgfpathrectangle{\pgfqpoint{0.456635in}{1.805660in}}{\pgfqpoint{4.376471in}{0.972632in}}%
\pgfusepath{clip}%
\pgfsetbuttcap%
\pgfsetroundjoin%
\definecolor{currentfill}{rgb}{1.000000,0.000000,0.000000}%
\pgfsetfillcolor{currentfill}%
\pgfsetlinewidth{2.007500pt}%
\definecolor{currentstroke}{rgb}{1.000000,0.000000,0.000000}%
\pgfsetstrokecolor{currentstroke}%
\pgfsetdash{}{0pt}%
\pgfpathmoveto{\pgfqpoint{1.493624in}{2.373467in}}%
\pgfpathlineto{\pgfqpoint{1.576957in}{2.373467in}}%
\pgfpathmoveto{\pgfqpoint{1.535290in}{2.331800in}}%
\pgfpathlineto{\pgfqpoint{1.535290in}{2.415133in}}%
\pgfusepath{stroke,fill}%
\end{pgfscope}%
\begin{pgfscope}%
\pgfpathrectangle{\pgfqpoint{0.456635in}{1.805660in}}{\pgfqpoint{4.376471in}{0.972632in}}%
\pgfusepath{clip}%
\pgfsetbuttcap%
\pgfsetroundjoin%
\definecolor{currentfill}{rgb}{1.000000,0.000000,0.000000}%
\pgfsetfillcolor{currentfill}%
\pgfsetlinewidth{2.007500pt}%
\definecolor{currentstroke}{rgb}{1.000000,0.000000,0.000000}%
\pgfsetstrokecolor{currentstroke}%
\pgfsetdash{}{0pt}%
\pgfpathmoveto{\pgfqpoint{4.322898in}{2.455378in}}%
\pgfpathlineto{\pgfqpoint{4.406232in}{2.455378in}}%
\pgfpathmoveto{\pgfqpoint{4.364565in}{2.413711in}}%
\pgfpathlineto{\pgfqpoint{4.364565in}{2.497044in}}%
\pgfusepath{stroke,fill}%
\end{pgfscope}%
\begin{pgfscope}%
\pgfpathrectangle{\pgfqpoint{0.456635in}{1.805660in}}{\pgfqpoint{4.376471in}{0.972632in}}%
\pgfusepath{clip}%
\pgfsetbuttcap%
\pgfsetroundjoin%
\definecolor{currentfill}{rgb}{1.000000,0.000000,0.000000}%
\pgfsetfillcolor{currentfill}%
\pgfsetlinewidth{2.007500pt}%
\definecolor{currentstroke}{rgb}{1.000000,0.000000,0.000000}%
\pgfsetstrokecolor{currentstroke}%
\pgfsetdash{}{0pt}%
\pgfpathmoveto{\pgfqpoint{3.394872in}{2.452634in}}%
\pgfpathlineto{\pgfqpoint{3.478206in}{2.452634in}}%
\pgfpathmoveto{\pgfqpoint{3.436539in}{2.410967in}}%
\pgfpathlineto{\pgfqpoint{3.436539in}{2.494300in}}%
\pgfusepath{stroke,fill}%
\end{pgfscope}%
\begin{pgfscope}%
\pgfpathrectangle{\pgfqpoint{0.456635in}{1.805660in}}{\pgfqpoint{4.376471in}{0.972632in}}%
\pgfusepath{clip}%
\pgfsetbuttcap%
\pgfsetroundjoin%
\definecolor{currentfill}{rgb}{1.000000,0.000000,0.000000}%
\pgfsetfillcolor{currentfill}%
\pgfsetlinewidth{2.007500pt}%
\definecolor{currentstroke}{rgb}{1.000000,0.000000,0.000000}%
\pgfsetstrokecolor{currentstroke}%
\pgfsetdash{}{0pt}%
\pgfpathmoveto{\pgfqpoint{3.769350in}{1.944239in}}%
\pgfpathlineto{\pgfqpoint{3.852683in}{1.944239in}}%
\pgfpathmoveto{\pgfqpoint{3.811016in}{1.902572in}}%
\pgfpathlineto{\pgfqpoint{3.811016in}{1.985906in}}%
\pgfusepath{stroke,fill}%
\end{pgfscope}%
\begin{pgfscope}%
\pgfpathrectangle{\pgfqpoint{0.456635in}{1.805660in}}{\pgfqpoint{4.376471in}{0.972632in}}%
\pgfusepath{clip}%
\pgfsetbuttcap%
\pgfsetroundjoin%
\definecolor{currentfill}{rgb}{1.000000,0.000000,0.000000}%
\pgfsetfillcolor{currentfill}%
\pgfsetlinewidth{2.007500pt}%
\definecolor{currentstroke}{rgb}{1.000000,0.000000,0.000000}%
\pgfsetstrokecolor{currentstroke}%
\pgfsetdash{}{0pt}%
\pgfpathmoveto{\pgfqpoint{1.362333in}{2.597129in}}%
\pgfpathlineto{\pgfqpoint{1.445666in}{2.597129in}}%
\pgfpathmoveto{\pgfqpoint{1.403999in}{2.555462in}}%
\pgfpathlineto{\pgfqpoint{1.403999in}{2.638795in}}%
\pgfusepath{stroke,fill}%
\end{pgfscope}%
\begin{pgfscope}%
\pgfpathrectangle{\pgfqpoint{0.456635in}{1.805660in}}{\pgfqpoint{4.376471in}{0.972632in}}%
\pgfusepath{clip}%
\pgfsetbuttcap%
\pgfsetroundjoin%
\definecolor{currentfill}{rgb}{1.000000,0.000000,0.000000}%
\pgfsetfillcolor{currentfill}%
\pgfsetlinewidth{2.007500pt}%
\definecolor{currentstroke}{rgb}{1.000000,0.000000,0.000000}%
\pgfsetstrokecolor{currentstroke}%
\pgfsetdash{}{0pt}%
\pgfpathmoveto{\pgfqpoint{4.686088in}{2.335190in}}%
\pgfpathlineto{\pgfqpoint{4.769422in}{2.335190in}}%
\pgfpathmoveto{\pgfqpoint{4.727755in}{2.293523in}}%
\pgfpathlineto{\pgfqpoint{4.727755in}{2.376856in}}%
\pgfusepath{stroke,fill}%
\end{pgfscope}%
\begin{pgfscope}%
\pgfpathrectangle{\pgfqpoint{0.456635in}{1.805660in}}{\pgfqpoint{4.376471in}{0.972632in}}%
\pgfusepath{clip}%
\pgfsetbuttcap%
\pgfsetroundjoin%
\definecolor{currentfill}{rgb}{1.000000,0.000000,0.000000}%
\pgfsetfillcolor{currentfill}%
\pgfsetlinewidth{2.007500pt}%
\definecolor{currentstroke}{rgb}{1.000000,0.000000,0.000000}%
\pgfsetstrokecolor{currentstroke}%
\pgfsetdash{}{0pt}%
\pgfpathmoveto{\pgfqpoint{4.204791in}{2.287442in}}%
\pgfpathlineto{\pgfqpoint{4.288125in}{2.287442in}}%
\pgfpathmoveto{\pgfqpoint{4.246458in}{2.245775in}}%
\pgfpathlineto{\pgfqpoint{4.246458in}{2.329108in}}%
\pgfusepath{stroke,fill}%
\end{pgfscope}%
\begin{pgfscope}%
\pgfpathrectangle{\pgfqpoint{0.456635in}{1.805660in}}{\pgfqpoint{4.376471in}{0.972632in}}%
\pgfusepath{clip}%
\pgfsetbuttcap%
\pgfsetroundjoin%
\definecolor{currentfill}{rgb}{1.000000,0.000000,0.000000}%
\pgfsetfillcolor{currentfill}%
\pgfsetlinewidth{2.007500pt}%
\definecolor{currentstroke}{rgb}{1.000000,0.000000,0.000000}%
\pgfsetstrokecolor{currentstroke}%
\pgfsetdash{}{0pt}%
\pgfpathmoveto{\pgfqpoint{2.033699in}{2.426913in}}%
\pgfpathlineto{\pgfqpoint{2.117033in}{2.426913in}}%
\pgfpathmoveto{\pgfqpoint{2.075366in}{2.385247in}}%
\pgfpathlineto{\pgfqpoint{2.075366in}{2.468580in}}%
\pgfusepath{stroke,fill}%
\end{pgfscope}%
\begin{pgfscope}%
\pgfpathrectangle{\pgfqpoint{0.456635in}{1.805660in}}{\pgfqpoint{4.376471in}{0.972632in}}%
\pgfusepath{clip}%
\pgfsetbuttcap%
\pgfsetroundjoin%
\definecolor{currentfill}{rgb}{1.000000,0.000000,0.000000}%
\pgfsetfillcolor{currentfill}%
\pgfsetlinewidth{2.007500pt}%
\definecolor{currentstroke}{rgb}{1.000000,0.000000,0.000000}%
\pgfsetstrokecolor{currentstroke}%
\pgfsetdash{}{0pt}%
\pgfpathmoveto{\pgfqpoint{1.926864in}{2.393066in}}%
\pgfpathlineto{\pgfqpoint{2.010197in}{2.393066in}}%
\pgfpathmoveto{\pgfqpoint{1.968531in}{2.351399in}}%
\pgfpathlineto{\pgfqpoint{1.968531in}{2.434732in}}%
\pgfusepath{stroke,fill}%
\end{pgfscope}%
\begin{pgfscope}%
\pgfpathrectangle{\pgfqpoint{0.456635in}{1.805660in}}{\pgfqpoint{4.376471in}{0.972632in}}%
\pgfusepath{clip}%
\pgfsetbuttcap%
\pgfsetroundjoin%
\definecolor{currentfill}{rgb}{1.000000,0.000000,0.000000}%
\pgfsetfillcolor{currentfill}%
\pgfsetlinewidth{2.007500pt}%
\definecolor{currentstroke}{rgb}{1.000000,0.000000,0.000000}%
\pgfsetstrokecolor{currentstroke}%
\pgfsetdash{}{0pt}%
\pgfpathmoveto{\pgfqpoint{1.932394in}{2.340372in}}%
\pgfpathlineto{\pgfqpoint{2.015728in}{2.340372in}}%
\pgfpathmoveto{\pgfqpoint{1.974061in}{2.298706in}}%
\pgfpathlineto{\pgfqpoint{1.974061in}{2.382039in}}%
\pgfusepath{stroke,fill}%
\end{pgfscope}%
\begin{pgfscope}%
\pgfpathrectangle{\pgfqpoint{0.456635in}{1.805660in}}{\pgfqpoint{4.376471in}{0.972632in}}%
\pgfusepath{clip}%
\pgfsetbuttcap%
\pgfsetroundjoin%
\definecolor{currentfill}{rgb}{1.000000,0.000000,0.000000}%
\pgfsetfillcolor{currentfill}%
\pgfsetlinewidth{2.007500pt}%
\definecolor{currentstroke}{rgb}{1.000000,0.000000,0.000000}%
\pgfsetstrokecolor{currentstroke}%
\pgfsetdash{}{0pt}%
\pgfpathmoveto{\pgfqpoint{2.355469in}{2.362376in}}%
\pgfpathlineto{\pgfqpoint{2.438802in}{2.362376in}}%
\pgfpathmoveto{\pgfqpoint{2.397135in}{2.320710in}}%
\pgfpathlineto{\pgfqpoint{2.397135in}{2.404043in}}%
\pgfusepath{stroke,fill}%
\end{pgfscope}%
\begin{pgfscope}%
\pgfpathrectangle{\pgfqpoint{0.456635in}{1.805660in}}{\pgfqpoint{4.376471in}{0.972632in}}%
\pgfusepath{clip}%
\pgfsetbuttcap%
\pgfsetroundjoin%
\definecolor{currentfill}{rgb}{1.000000,0.000000,0.000000}%
\pgfsetfillcolor{currentfill}%
\pgfsetlinewidth{2.007500pt}%
\definecolor{currentstroke}{rgb}{1.000000,0.000000,0.000000}%
\pgfsetstrokecolor{currentstroke}%
\pgfsetdash{}{0pt}%
\pgfpathmoveto{\pgfqpoint{3.127528in}{2.237163in}}%
\pgfpathlineto{\pgfqpoint{3.210861in}{2.237163in}}%
\pgfpathmoveto{\pgfqpoint{3.169194in}{2.195496in}}%
\pgfpathlineto{\pgfqpoint{3.169194in}{2.278830in}}%
\pgfusepath{stroke,fill}%
\end{pgfscope}%
\begin{pgfscope}%
\pgfpathrectangle{\pgfqpoint{0.456635in}{1.805660in}}{\pgfqpoint{4.376471in}{0.972632in}}%
\pgfusepath{clip}%
\pgfsetbuttcap%
\pgfsetroundjoin%
\definecolor{currentfill}{rgb}{1.000000,0.000000,0.000000}%
\pgfsetfillcolor{currentfill}%
\pgfsetlinewidth{2.007500pt}%
\definecolor{currentstroke}{rgb}{1.000000,0.000000,0.000000}%
\pgfsetstrokecolor{currentstroke}%
\pgfsetdash{}{0pt}%
\pgfpathmoveto{\pgfqpoint{2.802578in}{1.829914in}}%
\pgfpathlineto{\pgfqpoint{2.885912in}{1.829914in}}%
\pgfpathmoveto{\pgfqpoint{2.844245in}{1.788247in}}%
\pgfpathlineto{\pgfqpoint{2.844245in}{1.871581in}}%
\pgfusepath{stroke,fill}%
\end{pgfscope}%
\begin{pgfscope}%
\pgfpathrectangle{\pgfqpoint{0.456635in}{1.805660in}}{\pgfqpoint{4.376471in}{0.972632in}}%
\pgfusepath{clip}%
\pgfsetbuttcap%
\pgfsetroundjoin%
\definecolor{currentfill}{rgb}{1.000000,0.000000,0.000000}%
\pgfsetfillcolor{currentfill}%
\pgfsetlinewidth{2.007500pt}%
\definecolor{currentstroke}{rgb}{1.000000,0.000000,0.000000}%
\pgfsetstrokecolor{currentstroke}%
\pgfsetdash{}{0pt}%
\pgfpathmoveto{\pgfqpoint{2.309907in}{2.342233in}}%
\pgfpathlineto{\pgfqpoint{2.393241in}{2.342233in}}%
\pgfpathmoveto{\pgfqpoint{2.351574in}{2.300566in}}%
\pgfpathlineto{\pgfqpoint{2.351574in}{2.383899in}}%
\pgfusepath{stroke,fill}%
\end{pgfscope}%
\begin{pgfscope}%
\pgfpathrectangle{\pgfqpoint{0.456635in}{1.805660in}}{\pgfqpoint{4.376471in}{0.972632in}}%
\pgfusepath{clip}%
\pgfsetbuttcap%
\pgfsetroundjoin%
\definecolor{currentfill}{rgb}{1.000000,0.000000,0.000000}%
\pgfsetfillcolor{currentfill}%
\pgfsetlinewidth{2.007500pt}%
\definecolor{currentstroke}{rgb}{1.000000,0.000000,0.000000}%
\pgfsetstrokecolor{currentstroke}%
\pgfsetdash{}{0pt}%
\pgfpathmoveto{\pgfqpoint{3.432468in}{2.303304in}}%
\pgfpathlineto{\pgfqpoint{3.515801in}{2.303304in}}%
\pgfpathmoveto{\pgfqpoint{3.474134in}{2.261637in}}%
\pgfpathlineto{\pgfqpoint{3.474134in}{2.344971in}}%
\pgfusepath{stroke,fill}%
\end{pgfscope}%
\begin{pgfscope}%
\pgfpathrectangle{\pgfqpoint{0.456635in}{1.805660in}}{\pgfqpoint{4.376471in}{0.972632in}}%
\pgfusepath{clip}%
\pgfsetbuttcap%
\pgfsetroundjoin%
\definecolor{currentfill}{rgb}{1.000000,0.000000,0.000000}%
\pgfsetfillcolor{currentfill}%
\pgfsetlinewidth{2.007500pt}%
\definecolor{currentstroke}{rgb}{1.000000,0.000000,0.000000}%
\pgfsetstrokecolor{currentstroke}%
\pgfsetdash{}{0pt}%
\pgfpathmoveto{\pgfqpoint{1.778655in}{2.326118in}}%
\pgfpathlineto{\pgfqpoint{1.861989in}{2.326118in}}%
\pgfpathmoveto{\pgfqpoint{1.820322in}{2.284451in}}%
\pgfpathlineto{\pgfqpoint{1.820322in}{2.367784in}}%
\pgfusepath{stroke,fill}%
\end{pgfscope}%
\begin{pgfscope}%
\pgfpathrectangle{\pgfqpoint{0.456635in}{1.805660in}}{\pgfqpoint{4.376471in}{0.972632in}}%
\pgfusepath{clip}%
\pgfsetbuttcap%
\pgfsetroundjoin%
\definecolor{currentfill}{rgb}{1.000000,0.000000,0.000000}%
\pgfsetfillcolor{currentfill}%
\pgfsetlinewidth{2.007500pt}%
\definecolor{currentstroke}{rgb}{1.000000,0.000000,0.000000}%
\pgfsetstrokecolor{currentstroke}%
\pgfsetdash{}{0pt}%
\pgfpathmoveto{\pgfqpoint{2.313113in}{2.447433in}}%
\pgfpathlineto{\pgfqpoint{2.396446in}{2.447433in}}%
\pgfpathmoveto{\pgfqpoint{2.354779in}{2.405766in}}%
\pgfpathlineto{\pgfqpoint{2.354779in}{2.489099in}}%
\pgfusepath{stroke,fill}%
\end{pgfscope}%
\begin{pgfscope}%
\pgfpathrectangle{\pgfqpoint{0.456635in}{1.805660in}}{\pgfqpoint{4.376471in}{0.972632in}}%
\pgfusepath{clip}%
\pgfsetbuttcap%
\pgfsetroundjoin%
\definecolor{currentfill}{rgb}{1.000000,0.000000,0.000000}%
\pgfsetfillcolor{currentfill}%
\pgfsetlinewidth{2.007500pt}%
\definecolor{currentstroke}{rgb}{1.000000,0.000000,0.000000}%
\pgfsetstrokecolor{currentstroke}%
\pgfsetdash{}{0pt}%
\pgfpathmoveto{\pgfqpoint{2.572960in}{1.929166in}}%
\pgfpathlineto{\pgfqpoint{2.656294in}{1.929166in}}%
\pgfpathmoveto{\pgfqpoint{2.614627in}{1.887499in}}%
\pgfpathlineto{\pgfqpoint{2.614627in}{1.970833in}}%
\pgfusepath{stroke,fill}%
\end{pgfscope}%
\begin{pgfscope}%
\pgfpathrectangle{\pgfqpoint{0.456635in}{1.805660in}}{\pgfqpoint{4.376471in}{0.972632in}}%
\pgfusepath{clip}%
\pgfsetbuttcap%
\pgfsetroundjoin%
\definecolor{currentfill}{rgb}{1.000000,0.000000,0.000000}%
\pgfsetfillcolor{currentfill}%
\pgfsetlinewidth{2.007500pt}%
\definecolor{currentstroke}{rgb}{1.000000,0.000000,0.000000}%
\pgfsetstrokecolor{currentstroke}%
\pgfsetdash{}{0pt}%
\pgfpathmoveto{\pgfqpoint{2.887044in}{2.050365in}}%
\pgfpathlineto{\pgfqpoint{2.970378in}{2.050365in}}%
\pgfpathmoveto{\pgfqpoint{2.928711in}{2.008698in}}%
\pgfpathlineto{\pgfqpoint{2.928711in}{2.092031in}}%
\pgfusepath{stroke,fill}%
\end{pgfscope}%
\begin{pgfscope}%
\pgfpathrectangle{\pgfqpoint{0.456635in}{1.805660in}}{\pgfqpoint{4.376471in}{0.972632in}}%
\pgfusepath{clip}%
\pgfsetbuttcap%
\pgfsetroundjoin%
\definecolor{currentfill}{rgb}{1.000000,0.000000,0.000000}%
\pgfsetfillcolor{currentfill}%
\pgfsetlinewidth{2.007500pt}%
\definecolor{currentstroke}{rgb}{1.000000,0.000000,0.000000}%
\pgfsetstrokecolor{currentstroke}%
\pgfsetdash{}{0pt}%
\pgfpathmoveto{\pgfqpoint{4.039302in}{2.152224in}}%
\pgfpathlineto{\pgfqpoint{4.122636in}{2.152224in}}%
\pgfpathmoveto{\pgfqpoint{4.080969in}{2.110557in}}%
\pgfpathlineto{\pgfqpoint{4.080969in}{2.193891in}}%
\pgfusepath{stroke,fill}%
\end{pgfscope}%
\begin{pgfscope}%
\pgfpathrectangle{\pgfqpoint{0.456635in}{1.805660in}}{\pgfqpoint{4.376471in}{0.972632in}}%
\pgfusepath{clip}%
\pgfsetbuttcap%
\pgfsetroundjoin%
\definecolor{currentfill}{rgb}{1.000000,0.000000,0.000000}%
\pgfsetfillcolor{currentfill}%
\pgfsetlinewidth{2.007500pt}%
\definecolor{currentstroke}{rgb}{1.000000,0.000000,0.000000}%
\pgfsetstrokecolor{currentstroke}%
\pgfsetdash{}{0pt}%
\pgfpathmoveto{\pgfqpoint{1.989356in}{2.303374in}}%
\pgfpathlineto{\pgfqpoint{2.072689in}{2.303374in}}%
\pgfpathmoveto{\pgfqpoint{2.031023in}{2.261707in}}%
\pgfpathlineto{\pgfqpoint{2.031023in}{2.345040in}}%
\pgfusepath{stroke,fill}%
\end{pgfscope}%
\begin{pgfscope}%
\pgfpathrectangle{\pgfqpoint{0.456635in}{1.805660in}}{\pgfqpoint{4.376471in}{0.972632in}}%
\pgfusepath{clip}%
\pgfsetbuttcap%
\pgfsetroundjoin%
\definecolor{currentfill}{rgb}{1.000000,0.000000,0.000000}%
\pgfsetfillcolor{currentfill}%
\pgfsetlinewidth{2.007500pt}%
\definecolor{currentstroke}{rgb}{1.000000,0.000000,0.000000}%
\pgfsetstrokecolor{currentstroke}%
\pgfsetdash{}{0pt}%
\pgfpathmoveto{\pgfqpoint{3.090688in}{2.289608in}}%
\pgfpathlineto{\pgfqpoint{3.174022in}{2.289608in}}%
\pgfpathmoveto{\pgfqpoint{3.132355in}{2.247942in}}%
\pgfpathlineto{\pgfqpoint{3.132355in}{2.331275in}}%
\pgfusepath{stroke,fill}%
\end{pgfscope}%
\begin{pgfscope}%
\pgfpathrectangle{\pgfqpoint{0.456635in}{1.805660in}}{\pgfqpoint{4.376471in}{0.972632in}}%
\pgfusepath{clip}%
\pgfsetbuttcap%
\pgfsetroundjoin%
\definecolor{currentfill}{rgb}{1.000000,0.000000,0.000000}%
\pgfsetfillcolor{currentfill}%
\pgfsetlinewidth{2.007500pt}%
\definecolor{currentstroke}{rgb}{1.000000,0.000000,0.000000}%
\pgfsetstrokecolor{currentstroke}%
\pgfsetdash{}{0pt}%
\pgfpathmoveto{\pgfqpoint{3.364411in}{2.483577in}}%
\pgfpathlineto{\pgfqpoint{3.447744in}{2.483577in}}%
\pgfpathmoveto{\pgfqpoint{3.406077in}{2.441911in}}%
\pgfpathlineto{\pgfqpoint{3.406077in}{2.525244in}}%
\pgfusepath{stroke,fill}%
\end{pgfscope}%
\begin{pgfscope}%
\pgfpathrectangle{\pgfqpoint{0.456635in}{1.805660in}}{\pgfqpoint{4.376471in}{0.972632in}}%
\pgfusepath{clip}%
\pgfsetbuttcap%
\pgfsetroundjoin%
\definecolor{currentfill}{rgb}{1.000000,0.000000,0.000000}%
\pgfsetfillcolor{currentfill}%
\pgfsetlinewidth{2.007500pt}%
\definecolor{currentstroke}{rgb}{1.000000,0.000000,0.000000}%
\pgfsetstrokecolor{currentstroke}%
\pgfsetdash{}{0pt}%
\pgfpathmoveto{\pgfqpoint{1.452894in}{2.528010in}}%
\pgfpathlineto{\pgfqpoint{1.536227in}{2.528010in}}%
\pgfpathmoveto{\pgfqpoint{1.494560in}{2.486343in}}%
\pgfpathlineto{\pgfqpoint{1.494560in}{2.569677in}}%
\pgfusepath{stroke,fill}%
\end{pgfscope}%
\begin{pgfscope}%
\pgfpathrectangle{\pgfqpoint{0.456635in}{1.805660in}}{\pgfqpoint{4.376471in}{0.972632in}}%
\pgfusepath{clip}%
\pgfsetbuttcap%
\pgfsetroundjoin%
\definecolor{currentfill}{rgb}{0.000000,0.000000,0.000000}%
\pgfsetfillcolor{currentfill}%
\pgfsetlinewidth{1.003750pt}%
\definecolor{currentstroke}{rgb}{0.000000,0.000000,0.000000}%
\pgfsetstrokecolor{currentstroke}%
\pgfsetdash{}{0pt}%
\pgfsys@defobject{currentmarker}{\pgfqpoint{-0.020833in}{-0.020833in}}{\pgfqpoint{0.020833in}{0.020833in}}{%
\pgfpathmoveto{\pgfqpoint{0.000000in}{-0.020833in}}%
\pgfpathcurveto{\pgfqpoint{0.005525in}{-0.020833in}}{\pgfqpoint{0.010825in}{-0.018638in}}{\pgfqpoint{0.014731in}{-0.014731in}}%
\pgfpathcurveto{\pgfqpoint{0.018638in}{-0.010825in}}{\pgfqpoint{0.020833in}{-0.005525in}}{\pgfqpoint{0.020833in}{0.000000in}}%
\pgfpathcurveto{\pgfqpoint{0.020833in}{0.005525in}}{\pgfqpoint{0.018638in}{0.010825in}}{\pgfqpoint{0.014731in}{0.014731in}}%
\pgfpathcurveto{\pgfqpoint{0.010825in}{0.018638in}}{\pgfqpoint{0.005525in}{0.020833in}}{\pgfqpoint{0.000000in}{0.020833in}}%
\pgfpathcurveto{\pgfqpoint{-0.005525in}{0.020833in}}{\pgfqpoint{-0.010825in}{0.018638in}}{\pgfqpoint{-0.014731in}{0.014731in}}%
\pgfpathcurveto{\pgfqpoint{-0.018638in}{0.010825in}}{\pgfqpoint{-0.020833in}{0.005525in}}{\pgfqpoint{-0.020833in}{0.000000in}}%
\pgfpathcurveto{\pgfqpoint{-0.020833in}{-0.005525in}}{\pgfqpoint{-0.018638in}{-0.010825in}}{\pgfqpoint{-0.014731in}{-0.014731in}}%
\pgfpathcurveto{\pgfqpoint{-0.010825in}{-0.018638in}}{\pgfqpoint{-0.005525in}{-0.020833in}}{\pgfqpoint{0.000000in}{-0.020833in}}%
\pgfpathclose%
\pgfusepath{stroke,fill}%
}%
\begin{pgfscope}%
\pgfsys@transformshift{1.331929in}{2.600870in}%
\pgfsys@useobject{currentmarker}{}%
\end{pgfscope}%
\begin{pgfscope}%
\pgfsys@transformshift{1.349523in}{2.628847in}%
\pgfsys@useobject{currentmarker}{}%
\end{pgfscope}%
\begin{pgfscope}%
\pgfsys@transformshift{1.367117in}{2.667164in}%
\pgfsys@useobject{currentmarker}{}%
\end{pgfscope}%
\begin{pgfscope}%
\pgfsys@transformshift{1.384711in}{2.705202in}%
\pgfsys@useobject{currentmarker}{}%
\end{pgfscope}%
\begin{pgfscope}%
\pgfsys@transformshift{1.402305in}{2.530106in}%
\pgfsys@useobject{currentmarker}{}%
\end{pgfscope}%
\begin{pgfscope}%
\pgfsys@transformshift{1.419899in}{2.531863in}%
\pgfsys@useobject{currentmarker}{}%
\end{pgfscope}%
\begin{pgfscope}%
\pgfsys@transformshift{1.437493in}{2.410537in}%
\pgfsys@useobject{currentmarker}{}%
\end{pgfscope}%
\begin{pgfscope}%
\pgfsys@transformshift{1.455086in}{2.373524in}%
\pgfsys@useobject{currentmarker}{}%
\end{pgfscope}%
\begin{pgfscope}%
\pgfsys@transformshift{1.472680in}{2.549475in}%
\pgfsys@useobject{currentmarker}{}%
\end{pgfscope}%
\begin{pgfscope}%
\pgfsys@transformshift{1.490274in}{2.577543in}%
\pgfsys@useobject{currentmarker}{}%
\end{pgfscope}%
\begin{pgfscope}%
\pgfsys@transformshift{1.507868in}{2.406563in}%
\pgfsys@useobject{currentmarker}{}%
\end{pgfscope}%
\begin{pgfscope}%
\pgfsys@transformshift{1.525462in}{2.490190in}%
\pgfsys@useobject{currentmarker}{}%
\end{pgfscope}%
\begin{pgfscope}%
\pgfsys@transformshift{1.543056in}{2.400930in}%
\pgfsys@useobject{currentmarker}{}%
\end{pgfscope}%
\begin{pgfscope}%
\pgfsys@transformshift{1.560649in}{2.275992in}%
\pgfsys@useobject{currentmarker}{}%
\end{pgfscope}%
\begin{pgfscope}%
\pgfsys@transformshift{1.578243in}{2.356478in}%
\pgfsys@useobject{currentmarker}{}%
\end{pgfscope}%
\begin{pgfscope}%
\pgfsys@transformshift{1.595837in}{2.455794in}%
\pgfsys@useobject{currentmarker}{}%
\end{pgfscope}%
\begin{pgfscope}%
\pgfsys@transformshift{1.613431in}{2.278164in}%
\pgfsys@useobject{currentmarker}{}%
\end{pgfscope}%
\begin{pgfscope}%
\pgfsys@transformshift{1.631025in}{2.423983in}%
\pgfsys@useobject{currentmarker}{}%
\end{pgfscope}%
\begin{pgfscope}%
\pgfsys@transformshift{1.648619in}{1.985638in}%
\pgfsys@useobject{currentmarker}{}%
\end{pgfscope}%
\begin{pgfscope}%
\pgfsys@transformshift{1.666213in}{2.321942in}%
\pgfsys@useobject{currentmarker}{}%
\end{pgfscope}%
\begin{pgfscope}%
\pgfsys@transformshift{1.683806in}{2.237172in}%
\pgfsys@useobject{currentmarker}{}%
\end{pgfscope}%
\begin{pgfscope}%
\pgfsys@transformshift{1.701400in}{2.189848in}%
\pgfsys@useobject{currentmarker}{}%
\end{pgfscope}%
\begin{pgfscope}%
\pgfsys@transformshift{1.718994in}{2.223354in}%
\pgfsys@useobject{currentmarker}{}%
\end{pgfscope}%
\begin{pgfscope}%
\pgfsys@transformshift{1.736588in}{2.008723in}%
\pgfsys@useobject{currentmarker}{}%
\end{pgfscope}%
\begin{pgfscope}%
\pgfsys@transformshift{1.754182in}{2.185975in}%
\pgfsys@useobject{currentmarker}{}%
\end{pgfscope}%
\begin{pgfscope}%
\pgfsys@transformshift{1.771776in}{2.244604in}%
\pgfsys@useobject{currentmarker}{}%
\end{pgfscope}%
\begin{pgfscope}%
\pgfsys@transformshift{1.789370in}{2.360340in}%
\pgfsys@useobject{currentmarker}{}%
\end{pgfscope}%
\begin{pgfscope}%
\pgfsys@transformshift{1.806963in}{2.162187in}%
\pgfsys@useobject{currentmarker}{}%
\end{pgfscope}%
\begin{pgfscope}%
\pgfsys@transformshift{1.824557in}{2.138680in}%
\pgfsys@useobject{currentmarker}{}%
\end{pgfscope}%
\begin{pgfscope}%
\pgfsys@transformshift{1.842151in}{2.177341in}%
\pgfsys@useobject{currentmarker}{}%
\end{pgfscope}%
\begin{pgfscope}%
\pgfsys@transformshift{1.859745in}{2.330058in}%
\pgfsys@useobject{currentmarker}{}%
\end{pgfscope}%
\begin{pgfscope}%
\pgfsys@transformshift{1.877339in}{2.281159in}%
\pgfsys@useobject{currentmarker}{}%
\end{pgfscope}%
\begin{pgfscope}%
\pgfsys@transformshift{1.894933in}{2.205955in}%
\pgfsys@useobject{currentmarker}{}%
\end{pgfscope}%
\begin{pgfscope}%
\pgfsys@transformshift{1.912526in}{2.324470in}%
\pgfsys@useobject{currentmarker}{}%
\end{pgfscope}%
\begin{pgfscope}%
\pgfsys@transformshift{1.930120in}{2.296023in}%
\pgfsys@useobject{currentmarker}{}%
\end{pgfscope}%
\begin{pgfscope}%
\pgfsys@transformshift{1.947714in}{2.398731in}%
\pgfsys@useobject{currentmarker}{}%
\end{pgfscope}%
\begin{pgfscope}%
\pgfsys@transformshift{1.965308in}{2.244354in}%
\pgfsys@useobject{currentmarker}{}%
\end{pgfscope}%
\begin{pgfscope}%
\pgfsys@transformshift{1.982902in}{2.297461in}%
\pgfsys@useobject{currentmarker}{}%
\end{pgfscope}%
\begin{pgfscope}%
\pgfsys@transformshift{2.000496in}{2.306184in}%
\pgfsys@useobject{currentmarker}{}%
\end{pgfscope}%
\begin{pgfscope}%
\pgfsys@transformshift{2.018090in}{2.212760in}%
\pgfsys@useobject{currentmarker}{}%
\end{pgfscope}%
\begin{pgfscope}%
\pgfsys@transformshift{2.035683in}{2.405835in}%
\pgfsys@useobject{currentmarker}{}%
\end{pgfscope}%
\begin{pgfscope}%
\pgfsys@transformshift{2.053277in}{2.416549in}%
\pgfsys@useobject{currentmarker}{}%
\end{pgfscope}%
\begin{pgfscope}%
\pgfsys@transformshift{2.070871in}{2.404164in}%
\pgfsys@useobject{currentmarker}{}%
\end{pgfscope}%
\begin{pgfscope}%
\pgfsys@transformshift{2.088465in}{2.392518in}%
\pgfsys@useobject{currentmarker}{}%
\end{pgfscope}%
\begin{pgfscope}%
\pgfsys@transformshift{2.106059in}{2.284444in}%
\pgfsys@useobject{currentmarker}{}%
\end{pgfscope}%
\begin{pgfscope}%
\pgfsys@transformshift{2.123653in}{2.395540in}%
\pgfsys@useobject{currentmarker}{}%
\end{pgfscope}%
\begin{pgfscope}%
\pgfsys@transformshift{2.141247in}{2.412357in}%
\pgfsys@useobject{currentmarker}{}%
\end{pgfscope}%
\begin{pgfscope}%
\pgfsys@transformshift{2.158840in}{2.373189in}%
\pgfsys@useobject{currentmarker}{}%
\end{pgfscope}%
\begin{pgfscope}%
\pgfsys@transformshift{2.176434in}{2.443878in}%
\pgfsys@useobject{currentmarker}{}%
\end{pgfscope}%
\begin{pgfscope}%
\pgfsys@transformshift{2.194028in}{2.505155in}%
\pgfsys@useobject{currentmarker}{}%
\end{pgfscope}%
\begin{pgfscope}%
\pgfsys@transformshift{2.211622in}{2.657488in}%
\pgfsys@useobject{currentmarker}{}%
\end{pgfscope}%
\begin{pgfscope}%
\pgfsys@transformshift{2.229216in}{2.484352in}%
\pgfsys@useobject{currentmarker}{}%
\end{pgfscope}%
\begin{pgfscope}%
\pgfsys@transformshift{2.246810in}{2.491098in}%
\pgfsys@useobject{currentmarker}{}%
\end{pgfscope}%
\begin{pgfscope}%
\pgfsys@transformshift{2.264404in}{2.453842in}%
\pgfsys@useobject{currentmarker}{}%
\end{pgfscope}%
\begin{pgfscope}%
\pgfsys@transformshift{2.281997in}{2.261402in}%
\pgfsys@useobject{currentmarker}{}%
\end{pgfscope}%
\begin{pgfscope}%
\pgfsys@transformshift{2.299591in}{2.445594in}%
\pgfsys@useobject{currentmarker}{}%
\end{pgfscope}%
\begin{pgfscope}%
\pgfsys@transformshift{2.317185in}{2.444953in}%
\pgfsys@useobject{currentmarker}{}%
\end{pgfscope}%
\begin{pgfscope}%
\pgfsys@transformshift{2.334779in}{2.677142in}%
\pgfsys@useobject{currentmarker}{}%
\end{pgfscope}%
\begin{pgfscope}%
\pgfsys@transformshift{2.352373in}{2.395046in}%
\pgfsys@useobject{currentmarker}{}%
\end{pgfscope}%
\begin{pgfscope}%
\pgfsys@transformshift{2.369967in}{2.430374in}%
\pgfsys@useobject{currentmarker}{}%
\end{pgfscope}%
\begin{pgfscope}%
\pgfsys@transformshift{2.387560in}{2.380038in}%
\pgfsys@useobject{currentmarker}{}%
\end{pgfscope}%
\begin{pgfscope}%
\pgfsys@transformshift{2.405154in}{2.247456in}%
\pgfsys@useobject{currentmarker}{}%
\end{pgfscope}%
\begin{pgfscope}%
\pgfsys@transformshift{2.422748in}{2.462676in}%
\pgfsys@useobject{currentmarker}{}%
\end{pgfscope}%
\begin{pgfscope}%
\pgfsys@transformshift{2.440342in}{2.402983in}%
\pgfsys@useobject{currentmarker}{}%
\end{pgfscope}%
\begin{pgfscope}%
\pgfsys@transformshift{2.457936in}{2.385908in}%
\pgfsys@useobject{currentmarker}{}%
\end{pgfscope}%
\begin{pgfscope}%
\pgfsys@transformshift{2.475530in}{2.191829in}%
\pgfsys@useobject{currentmarker}{}%
\end{pgfscope}%
\begin{pgfscope}%
\pgfsys@transformshift{2.493124in}{2.403717in}%
\pgfsys@useobject{currentmarker}{}%
\end{pgfscope}%
\begin{pgfscope}%
\pgfsys@transformshift{2.510717in}{2.096815in}%
\pgfsys@useobject{currentmarker}{}%
\end{pgfscope}%
\begin{pgfscope}%
\pgfsys@transformshift{2.528311in}{2.275382in}%
\pgfsys@useobject{currentmarker}{}%
\end{pgfscope}%
\begin{pgfscope}%
\pgfsys@transformshift{2.545905in}{2.414965in}%
\pgfsys@useobject{currentmarker}{}%
\end{pgfscope}%
\begin{pgfscope}%
\pgfsys@transformshift{2.563499in}{2.070030in}%
\pgfsys@useobject{currentmarker}{}%
\end{pgfscope}%
\begin{pgfscope}%
\pgfsys@transformshift{2.581093in}{2.090804in}%
\pgfsys@useobject{currentmarker}{}%
\end{pgfscope}%
\begin{pgfscope}%
\pgfsys@transformshift{2.598687in}{2.136709in}%
\pgfsys@useobject{currentmarker}{}%
\end{pgfscope}%
\begin{pgfscope}%
\pgfsys@transformshift{2.616281in}{2.054872in}%
\pgfsys@useobject{currentmarker}{}%
\end{pgfscope}%
\begin{pgfscope}%
\pgfsys@transformshift{2.633874in}{1.929068in}%
\pgfsys@useobject{currentmarker}{}%
\end{pgfscope}%
\begin{pgfscope}%
\pgfsys@transformshift{2.651468in}{2.074614in}%
\pgfsys@useobject{currentmarker}{}%
\end{pgfscope}%
\begin{pgfscope}%
\pgfsys@transformshift{2.669062in}{1.942899in}%
\pgfsys@useobject{currentmarker}{}%
\end{pgfscope}%
\begin{pgfscope}%
\pgfsys@transformshift{2.686656in}{2.082889in}%
\pgfsys@useobject{currentmarker}{}%
\end{pgfscope}%
\begin{pgfscope}%
\pgfsys@transformshift{2.704250in}{1.927792in}%
\pgfsys@useobject{currentmarker}{}%
\end{pgfscope}%
\begin{pgfscope}%
\pgfsys@transformshift{2.721844in}{2.165797in}%
\pgfsys@useobject{currentmarker}{}%
\end{pgfscope}%
\begin{pgfscope}%
\pgfsys@transformshift{2.739438in}{1.919114in}%
\pgfsys@useobject{currentmarker}{}%
\end{pgfscope}%
\begin{pgfscope}%
\pgfsys@transformshift{2.757031in}{1.957519in}%
\pgfsys@useobject{currentmarker}{}%
\end{pgfscope}%
\begin{pgfscope}%
\pgfsys@transformshift{2.774625in}{2.066294in}%
\pgfsys@useobject{currentmarker}{}%
\end{pgfscope}%
\begin{pgfscope}%
\pgfsys@transformshift{2.792219in}{1.854979in}%
\pgfsys@useobject{currentmarker}{}%
\end{pgfscope}%
\begin{pgfscope}%
\pgfsys@transformshift{2.809813in}{2.000662in}%
\pgfsys@useobject{currentmarker}{}%
\end{pgfscope}%
\begin{pgfscope}%
\pgfsys@transformshift{2.827407in}{2.110110in}%
\pgfsys@useobject{currentmarker}{}%
\end{pgfscope}%
\begin{pgfscope}%
\pgfsys@transformshift{2.845001in}{1.816985in}%
\pgfsys@useobject{currentmarker}{}%
\end{pgfscope}%
\begin{pgfscope}%
\pgfsys@transformshift{2.862594in}{2.002807in}%
\pgfsys@useobject{currentmarker}{}%
\end{pgfscope}%
\begin{pgfscope}%
\pgfsys@transformshift{2.880188in}{2.016710in}%
\pgfsys@useobject{currentmarker}{}%
\end{pgfscope}%
\begin{pgfscope}%
\pgfsys@transformshift{2.897782in}{2.077824in}%
\pgfsys@useobject{currentmarker}{}%
\end{pgfscope}%
\begin{pgfscope}%
\pgfsys@transformshift{2.915376in}{1.883384in}%
\pgfsys@useobject{currentmarker}{}%
\end{pgfscope}%
\begin{pgfscope}%
\pgfsys@transformshift{2.932970in}{1.886768in}%
\pgfsys@useobject{currentmarker}{}%
\end{pgfscope}%
\begin{pgfscope}%
\pgfsys@transformshift{2.950564in}{2.086901in}%
\pgfsys@useobject{currentmarker}{}%
\end{pgfscope}%
\begin{pgfscope}%
\pgfsys@transformshift{2.968158in}{2.079059in}%
\pgfsys@useobject{currentmarker}{}%
\end{pgfscope}%
\begin{pgfscope}%
\pgfsys@transformshift{2.985751in}{2.090624in}%
\pgfsys@useobject{currentmarker}{}%
\end{pgfscope}%
\begin{pgfscope}%
\pgfsys@transformshift{3.003345in}{2.117776in}%
\pgfsys@useobject{currentmarker}{}%
\end{pgfscope}%
\begin{pgfscope}%
\pgfsys@transformshift{3.020939in}{2.032184in}%
\pgfsys@useobject{currentmarker}{}%
\end{pgfscope}%
\begin{pgfscope}%
\pgfsys@transformshift{3.038533in}{2.143805in}%
\pgfsys@useobject{currentmarker}{}%
\end{pgfscope}%
\begin{pgfscope}%
\pgfsys@transformshift{3.056127in}{2.169750in}%
\pgfsys@useobject{currentmarker}{}%
\end{pgfscope}%
\begin{pgfscope}%
\pgfsys@transformshift{3.073721in}{2.087854in}%
\pgfsys@useobject{currentmarker}{}%
\end{pgfscope}%
\begin{pgfscope}%
\pgfsys@transformshift{3.091315in}{2.369617in}%
\pgfsys@useobject{currentmarker}{}%
\end{pgfscope}%
\begin{pgfscope}%
\pgfsys@transformshift{3.108908in}{2.248927in}%
\pgfsys@useobject{currentmarker}{}%
\end{pgfscope}%
\begin{pgfscope}%
\pgfsys@transformshift{3.126502in}{2.100333in}%
\pgfsys@useobject{currentmarker}{}%
\end{pgfscope}%
\begin{pgfscope}%
\pgfsys@transformshift{3.144096in}{2.307234in}%
\pgfsys@useobject{currentmarker}{}%
\end{pgfscope}%
\begin{pgfscope}%
\pgfsys@transformshift{3.161690in}{2.161025in}%
\pgfsys@useobject{currentmarker}{}%
\end{pgfscope}%
\begin{pgfscope}%
\pgfsys@transformshift{3.179284in}{2.357769in}%
\pgfsys@useobject{currentmarker}{}%
\end{pgfscope}%
\begin{pgfscope}%
\pgfsys@transformshift{3.196878in}{2.412666in}%
\pgfsys@useobject{currentmarker}{}%
\end{pgfscope}%
\begin{pgfscope}%
\pgfsys@transformshift{3.214472in}{2.228230in}%
\pgfsys@useobject{currentmarker}{}%
\end{pgfscope}%
\begin{pgfscope}%
\pgfsys@transformshift{3.232065in}{2.423762in}%
\pgfsys@useobject{currentmarker}{}%
\end{pgfscope}%
\begin{pgfscope}%
\pgfsys@transformshift{3.249659in}{2.381297in}%
\pgfsys@useobject{currentmarker}{}%
\end{pgfscope}%
\begin{pgfscope}%
\pgfsys@transformshift{3.267253in}{2.434494in}%
\pgfsys@useobject{currentmarker}{}%
\end{pgfscope}%
\begin{pgfscope}%
\pgfsys@transformshift{3.284847in}{2.553415in}%
\pgfsys@useobject{currentmarker}{}%
\end{pgfscope}%
\begin{pgfscope}%
\pgfsys@transformshift{3.302441in}{2.344623in}%
\pgfsys@useobject{currentmarker}{}%
\end{pgfscope}%
\begin{pgfscope}%
\pgfsys@transformshift{3.320035in}{2.299500in}%
\pgfsys@useobject{currentmarker}{}%
\end{pgfscope}%
\begin{pgfscope}%
\pgfsys@transformshift{3.337628in}{2.290209in}%
\pgfsys@useobject{currentmarker}{}%
\end{pgfscope}%
\begin{pgfscope}%
\pgfsys@transformshift{3.355222in}{2.300191in}%
\pgfsys@useobject{currentmarker}{}%
\end{pgfscope}%
\begin{pgfscope}%
\pgfsys@transformshift{3.372816in}{2.375586in}%
\pgfsys@useobject{currentmarker}{}%
\end{pgfscope}%
\begin{pgfscope}%
\pgfsys@transformshift{3.390410in}{2.416558in}%
\pgfsys@useobject{currentmarker}{}%
\end{pgfscope}%
\begin{pgfscope}%
\pgfsys@transformshift{3.408004in}{2.406697in}%
\pgfsys@useobject{currentmarker}{}%
\end{pgfscope}%
\begin{pgfscope}%
\pgfsys@transformshift{3.425598in}{2.457261in}%
\pgfsys@useobject{currentmarker}{}%
\end{pgfscope}%
\begin{pgfscope}%
\pgfsys@transformshift{3.443192in}{2.367752in}%
\pgfsys@useobject{currentmarker}{}%
\end{pgfscope}%
\begin{pgfscope}%
\pgfsys@transformshift{3.460785in}{2.504959in}%
\pgfsys@useobject{currentmarker}{}%
\end{pgfscope}%
\begin{pgfscope}%
\pgfsys@transformshift{3.478379in}{2.320519in}%
\pgfsys@useobject{currentmarker}{}%
\end{pgfscope}%
\begin{pgfscope}%
\pgfsys@transformshift{3.495973in}{2.611074in}%
\pgfsys@useobject{currentmarker}{}%
\end{pgfscope}%
\begin{pgfscope}%
\pgfsys@transformshift{3.513567in}{2.385658in}%
\pgfsys@useobject{currentmarker}{}%
\end{pgfscope}%
\begin{pgfscope}%
\pgfsys@transformshift{3.531161in}{2.221011in}%
\pgfsys@useobject{currentmarker}{}%
\end{pgfscope}%
\begin{pgfscope}%
\pgfsys@transformshift{3.548755in}{2.183906in}%
\pgfsys@useobject{currentmarker}{}%
\end{pgfscope}%
\begin{pgfscope}%
\pgfsys@transformshift{3.566349in}{2.324977in}%
\pgfsys@useobject{currentmarker}{}%
\end{pgfscope}%
\begin{pgfscope}%
\pgfsys@transformshift{3.583942in}{2.236472in}%
\pgfsys@useobject{currentmarker}{}%
\end{pgfscope}%
\begin{pgfscope}%
\pgfsys@transformshift{3.601536in}{2.313992in}%
\pgfsys@useobject{currentmarker}{}%
\end{pgfscope}%
\begin{pgfscope}%
\pgfsys@transformshift{3.619130in}{2.271861in}%
\pgfsys@useobject{currentmarker}{}%
\end{pgfscope}%
\begin{pgfscope}%
\pgfsys@transformshift{3.636724in}{2.198725in}%
\pgfsys@useobject{currentmarker}{}%
\end{pgfscope}%
\begin{pgfscope}%
\pgfsys@transformshift{3.654318in}{2.102630in}%
\pgfsys@useobject{currentmarker}{}%
\end{pgfscope}%
\begin{pgfscope}%
\pgfsys@transformshift{3.671912in}{2.017601in}%
\pgfsys@useobject{currentmarker}{}%
\end{pgfscope}%
\begin{pgfscope}%
\pgfsys@transformshift{3.689506in}{2.109035in}%
\pgfsys@useobject{currentmarker}{}%
\end{pgfscope}%
\begin{pgfscope}%
\pgfsys@transformshift{3.707099in}{2.224972in}%
\pgfsys@useobject{currentmarker}{}%
\end{pgfscope}%
\begin{pgfscope}%
\pgfsys@transformshift{3.724693in}{2.144755in}%
\pgfsys@useobject{currentmarker}{}%
\end{pgfscope}%
\begin{pgfscope}%
\pgfsys@transformshift{3.742287in}{1.982819in}%
\pgfsys@useobject{currentmarker}{}%
\end{pgfscope}%
\begin{pgfscope}%
\pgfsys@transformshift{3.759881in}{2.113829in}%
\pgfsys@useobject{currentmarker}{}%
\end{pgfscope}%
\begin{pgfscope}%
\pgfsys@transformshift{3.777475in}{2.124014in}%
\pgfsys@useobject{currentmarker}{}%
\end{pgfscope}%
\begin{pgfscope}%
\pgfsys@transformshift{3.795069in}{1.985708in}%
\pgfsys@useobject{currentmarker}{}%
\end{pgfscope}%
\begin{pgfscope}%
\pgfsys@transformshift{3.812662in}{2.082832in}%
\pgfsys@useobject{currentmarker}{}%
\end{pgfscope}%
\begin{pgfscope}%
\pgfsys@transformshift{3.830256in}{2.066990in}%
\pgfsys@useobject{currentmarker}{}%
\end{pgfscope}%
\begin{pgfscope}%
\pgfsys@transformshift{3.847850in}{1.941058in}%
\pgfsys@useobject{currentmarker}{}%
\end{pgfscope}%
\begin{pgfscope}%
\pgfsys@transformshift{3.865444in}{2.090884in}%
\pgfsys@useobject{currentmarker}{}%
\end{pgfscope}%
\begin{pgfscope}%
\pgfsys@transformshift{3.883038in}{2.111295in}%
\pgfsys@useobject{currentmarker}{}%
\end{pgfscope}%
\begin{pgfscope}%
\pgfsys@transformshift{3.900632in}{2.166160in}%
\pgfsys@useobject{currentmarker}{}%
\end{pgfscope}%
\begin{pgfscope}%
\pgfsys@transformshift{3.918226in}{2.167273in}%
\pgfsys@useobject{currentmarker}{}%
\end{pgfscope}%
\begin{pgfscope}%
\pgfsys@transformshift{3.935819in}{1.927126in}%
\pgfsys@useobject{currentmarker}{}%
\end{pgfscope}%
\begin{pgfscope}%
\pgfsys@transformshift{3.953413in}{1.979989in}%
\pgfsys@useobject{currentmarker}{}%
\end{pgfscope}%
\begin{pgfscope}%
\pgfsys@transformshift{3.971007in}{2.137543in}%
\pgfsys@useobject{currentmarker}{}%
\end{pgfscope}%
\begin{pgfscope}%
\pgfsys@transformshift{3.988601in}{2.149767in}%
\pgfsys@useobject{currentmarker}{}%
\end{pgfscope}%
\begin{pgfscope}%
\pgfsys@transformshift{4.006195in}{2.164156in}%
\pgfsys@useobject{currentmarker}{}%
\end{pgfscope}%
\begin{pgfscope}%
\pgfsys@transformshift{4.023789in}{2.518387in}%
\pgfsys@useobject{currentmarker}{}%
\end{pgfscope}%
\begin{pgfscope}%
\pgfsys@transformshift{4.041383in}{2.203648in}%
\pgfsys@useobject{currentmarker}{}%
\end{pgfscope}%
\begin{pgfscope}%
\pgfsys@transformshift{4.058976in}{2.280178in}%
\pgfsys@useobject{currentmarker}{}%
\end{pgfscope}%
\begin{pgfscope}%
\pgfsys@transformshift{4.076570in}{2.282511in}%
\pgfsys@useobject{currentmarker}{}%
\end{pgfscope}%
\begin{pgfscope}%
\pgfsys@transformshift{4.094164in}{2.273824in}%
\pgfsys@useobject{currentmarker}{}%
\end{pgfscope}%
\begin{pgfscope}%
\pgfsys@transformshift{4.111758in}{2.198928in}%
\pgfsys@useobject{currentmarker}{}%
\end{pgfscope}%
\begin{pgfscope}%
\pgfsys@transformshift{4.129352in}{2.331694in}%
\pgfsys@useobject{currentmarker}{}%
\end{pgfscope}%
\begin{pgfscope}%
\pgfsys@transformshift{4.146946in}{2.201119in}%
\pgfsys@useobject{currentmarker}{}%
\end{pgfscope}%
\begin{pgfscope}%
\pgfsys@transformshift{4.164540in}{2.280539in}%
\pgfsys@useobject{currentmarker}{}%
\end{pgfscope}%
\begin{pgfscope}%
\pgfsys@transformshift{4.182133in}{2.280762in}%
\pgfsys@useobject{currentmarker}{}%
\end{pgfscope}%
\begin{pgfscope}%
\pgfsys@transformshift{4.199727in}{2.363723in}%
\pgfsys@useobject{currentmarker}{}%
\end{pgfscope}%
\begin{pgfscope}%
\pgfsys@transformshift{4.217321in}{2.615309in}%
\pgfsys@useobject{currentmarker}{}%
\end{pgfscope}%
\begin{pgfscope}%
\pgfsys@transformshift{4.234915in}{2.216662in}%
\pgfsys@useobject{currentmarker}{}%
\end{pgfscope}%
\begin{pgfscope}%
\pgfsys@transformshift{4.252509in}{2.499901in}%
\pgfsys@useobject{currentmarker}{}%
\end{pgfscope}%
\begin{pgfscope}%
\pgfsys@transformshift{4.270103in}{2.290789in}%
\pgfsys@useobject{currentmarker}{}%
\end{pgfscope}%
\begin{pgfscope}%
\pgfsys@transformshift{4.287696in}{2.429279in}%
\pgfsys@useobject{currentmarker}{}%
\end{pgfscope}%
\begin{pgfscope}%
\pgfsys@transformshift{4.305290in}{2.609257in}%
\pgfsys@useobject{currentmarker}{}%
\end{pgfscope}%
\begin{pgfscope}%
\pgfsys@transformshift{4.322884in}{2.526040in}%
\pgfsys@useobject{currentmarker}{}%
\end{pgfscope}%
\begin{pgfscope}%
\pgfsys@transformshift{4.340478in}{2.429544in}%
\pgfsys@useobject{currentmarker}{}%
\end{pgfscope}%
\begin{pgfscope}%
\pgfsys@transformshift{4.358072in}{2.483951in}%
\pgfsys@useobject{currentmarker}{}%
\end{pgfscope}%
\begin{pgfscope}%
\pgfsys@transformshift{4.375666in}{2.641320in}%
\pgfsys@useobject{currentmarker}{}%
\end{pgfscope}%
\begin{pgfscope}%
\pgfsys@transformshift{4.393260in}{2.512767in}%
\pgfsys@useobject{currentmarker}{}%
\end{pgfscope}%
\begin{pgfscope}%
\pgfsys@transformshift{4.410853in}{2.621168in}%
\pgfsys@useobject{currentmarker}{}%
\end{pgfscope}%
\begin{pgfscope}%
\pgfsys@transformshift{4.428447in}{2.614435in}%
\pgfsys@useobject{currentmarker}{}%
\end{pgfscope}%
\begin{pgfscope}%
\pgfsys@transformshift{4.446041in}{2.552446in}%
\pgfsys@useobject{currentmarker}{}%
\end{pgfscope}%
\begin{pgfscope}%
\pgfsys@transformshift{4.463635in}{2.842366in}%
\pgfsys@useobject{currentmarker}{}%
\end{pgfscope}%
\begin{pgfscope}%
\pgfsys@transformshift{4.481229in}{2.694101in}%
\pgfsys@useobject{currentmarker}{}%
\end{pgfscope}%
\begin{pgfscope}%
\pgfsys@transformshift{4.498823in}{2.427477in}%
\pgfsys@useobject{currentmarker}{}%
\end{pgfscope}%
\begin{pgfscope}%
\pgfsys@transformshift{4.516417in}{2.652421in}%
\pgfsys@useobject{currentmarker}{}%
\end{pgfscope}%
\begin{pgfscope}%
\pgfsys@transformshift{4.534010in}{2.565507in}%
\pgfsys@useobject{currentmarker}{}%
\end{pgfscope}%
\begin{pgfscope}%
\pgfsys@transformshift{4.551604in}{2.716177in}%
\pgfsys@useobject{currentmarker}{}%
\end{pgfscope}%
\begin{pgfscope}%
\pgfsys@transformshift{4.569198in}{2.545099in}%
\pgfsys@useobject{currentmarker}{}%
\end{pgfscope}%
\begin{pgfscope}%
\pgfsys@transformshift{4.586792in}{2.607790in}%
\pgfsys@useobject{currentmarker}{}%
\end{pgfscope}%
\begin{pgfscope}%
\pgfsys@transformshift{4.604386in}{2.663170in}%
\pgfsys@useobject{currentmarker}{}%
\end{pgfscope}%
\begin{pgfscope}%
\pgfsys@transformshift{4.621980in}{2.691029in}%
\pgfsys@useobject{currentmarker}{}%
\end{pgfscope}%
\begin{pgfscope}%
\pgfsys@transformshift{4.639573in}{2.471888in}%
\pgfsys@useobject{currentmarker}{}%
\end{pgfscope}%
\begin{pgfscope}%
\pgfsys@transformshift{4.657167in}{2.548838in}%
\pgfsys@useobject{currentmarker}{}%
\end{pgfscope}%
\begin{pgfscope}%
\pgfsys@transformshift{4.674761in}{2.523069in}%
\pgfsys@useobject{currentmarker}{}%
\end{pgfscope}%
\begin{pgfscope}%
\pgfsys@transformshift{4.692355in}{2.492877in}%
\pgfsys@useobject{currentmarker}{}%
\end{pgfscope}%
\begin{pgfscope}%
\pgfsys@transformshift{4.709949in}{2.725437in}%
\pgfsys@useobject{currentmarker}{}%
\end{pgfscope}%
\begin{pgfscope}%
\pgfsys@transformshift{4.727543in}{2.574920in}%
\pgfsys@useobject{currentmarker}{}%
\end{pgfscope}%
\begin{pgfscope}%
\pgfsys@transformshift{4.745137in}{2.393485in}%
\pgfsys@useobject{currentmarker}{}%
\end{pgfscope}%
\begin{pgfscope}%
\pgfsys@transformshift{4.762730in}{2.601802in}%
\pgfsys@useobject{currentmarker}{}%
\end{pgfscope}%
\begin{pgfscope}%
\pgfsys@transformshift{4.780324in}{2.711825in}%
\pgfsys@useobject{currentmarker}{}%
\end{pgfscope}%
\begin{pgfscope}%
\pgfsys@transformshift{4.797918in}{2.590066in}%
\pgfsys@useobject{currentmarker}{}%
\end{pgfscope}%
\begin{pgfscope}%
\pgfsys@transformshift{4.815512in}{2.320998in}%
\pgfsys@useobject{currentmarker}{}%
\end{pgfscope}%
\begin{pgfscope}%
\pgfsys@transformshift{4.833106in}{2.416363in}%
\pgfsys@useobject{currentmarker}{}%
\end{pgfscope}%
\end{pgfscope}%
\begin{pgfscope}%
\pgfsetbuttcap%
\pgfsetroundjoin%
\definecolor{currentfill}{rgb}{0.000000,0.000000,0.000000}%
\pgfsetfillcolor{currentfill}%
\pgfsetlinewidth{0.803000pt}%
\definecolor{currentstroke}{rgb}{0.000000,0.000000,0.000000}%
\pgfsetstrokecolor{currentstroke}%
\pgfsetdash{}{0pt}%
\pgfsys@defobject{currentmarker}{\pgfqpoint{0.000000in}{-0.048611in}}{\pgfqpoint{0.000000in}{0.000000in}}{%
\pgfpathmoveto{\pgfqpoint{0.000000in}{0.000000in}}%
\pgfpathlineto{\pgfqpoint{0.000000in}{-0.048611in}}%
\pgfusepath{stroke,fill}%
}%
\begin{pgfscope}%
\pgfsys@transformshift{0.456635in}{1.805660in}%
\pgfsys@useobject{currentmarker}{}%
\end{pgfscope}%
\end{pgfscope}%
\begin{pgfscope}%
\pgfsetbuttcap%
\pgfsetroundjoin%
\definecolor{currentfill}{rgb}{0.000000,0.000000,0.000000}%
\pgfsetfillcolor{currentfill}%
\pgfsetlinewidth{0.803000pt}%
\definecolor{currentstroke}{rgb}{0.000000,0.000000,0.000000}%
\pgfsetstrokecolor{currentstroke}%
\pgfsetdash{}{0pt}%
\pgfsys@defobject{currentmarker}{\pgfqpoint{0.000000in}{-0.048611in}}{\pgfqpoint{0.000000in}{0.000000in}}{%
\pgfpathmoveto{\pgfqpoint{0.000000in}{0.000000in}}%
\pgfpathlineto{\pgfqpoint{0.000000in}{-0.048611in}}%
\pgfusepath{stroke,fill}%
}%
\begin{pgfscope}%
\pgfsys@transformshift{1.331929in}{1.805660in}%
\pgfsys@useobject{currentmarker}{}%
\end{pgfscope}%
\end{pgfscope}%
\begin{pgfscope}%
\pgfsetbuttcap%
\pgfsetroundjoin%
\definecolor{currentfill}{rgb}{0.000000,0.000000,0.000000}%
\pgfsetfillcolor{currentfill}%
\pgfsetlinewidth{0.803000pt}%
\definecolor{currentstroke}{rgb}{0.000000,0.000000,0.000000}%
\pgfsetstrokecolor{currentstroke}%
\pgfsetdash{}{0pt}%
\pgfsys@defobject{currentmarker}{\pgfqpoint{0.000000in}{-0.048611in}}{\pgfqpoint{0.000000in}{0.000000in}}{%
\pgfpathmoveto{\pgfqpoint{0.000000in}{0.000000in}}%
\pgfpathlineto{\pgfqpoint{0.000000in}{-0.048611in}}%
\pgfusepath{stroke,fill}%
}%
\begin{pgfscope}%
\pgfsys@transformshift{2.207224in}{1.805660in}%
\pgfsys@useobject{currentmarker}{}%
\end{pgfscope}%
\end{pgfscope}%
\begin{pgfscope}%
\pgfsetbuttcap%
\pgfsetroundjoin%
\definecolor{currentfill}{rgb}{0.000000,0.000000,0.000000}%
\pgfsetfillcolor{currentfill}%
\pgfsetlinewidth{0.803000pt}%
\definecolor{currentstroke}{rgb}{0.000000,0.000000,0.000000}%
\pgfsetstrokecolor{currentstroke}%
\pgfsetdash{}{0pt}%
\pgfsys@defobject{currentmarker}{\pgfqpoint{0.000000in}{-0.048611in}}{\pgfqpoint{0.000000in}{0.000000in}}{%
\pgfpathmoveto{\pgfqpoint{0.000000in}{0.000000in}}%
\pgfpathlineto{\pgfqpoint{0.000000in}{-0.048611in}}%
\pgfusepath{stroke,fill}%
}%
\begin{pgfscope}%
\pgfsys@transformshift{3.082518in}{1.805660in}%
\pgfsys@useobject{currentmarker}{}%
\end{pgfscope}%
\end{pgfscope}%
\begin{pgfscope}%
\pgfsetbuttcap%
\pgfsetroundjoin%
\definecolor{currentfill}{rgb}{0.000000,0.000000,0.000000}%
\pgfsetfillcolor{currentfill}%
\pgfsetlinewidth{0.803000pt}%
\definecolor{currentstroke}{rgb}{0.000000,0.000000,0.000000}%
\pgfsetstrokecolor{currentstroke}%
\pgfsetdash{}{0pt}%
\pgfsys@defobject{currentmarker}{\pgfqpoint{0.000000in}{-0.048611in}}{\pgfqpoint{0.000000in}{0.000000in}}{%
\pgfpathmoveto{\pgfqpoint{0.000000in}{0.000000in}}%
\pgfpathlineto{\pgfqpoint{0.000000in}{-0.048611in}}%
\pgfusepath{stroke,fill}%
}%
\begin{pgfscope}%
\pgfsys@transformshift{3.957812in}{1.805660in}%
\pgfsys@useobject{currentmarker}{}%
\end{pgfscope}%
\end{pgfscope}%
\begin{pgfscope}%
\pgfsetbuttcap%
\pgfsetroundjoin%
\definecolor{currentfill}{rgb}{0.000000,0.000000,0.000000}%
\pgfsetfillcolor{currentfill}%
\pgfsetlinewidth{0.803000pt}%
\definecolor{currentstroke}{rgb}{0.000000,0.000000,0.000000}%
\pgfsetstrokecolor{currentstroke}%
\pgfsetdash{}{0pt}%
\pgfsys@defobject{currentmarker}{\pgfqpoint{0.000000in}{-0.048611in}}{\pgfqpoint{0.000000in}{0.000000in}}{%
\pgfpathmoveto{\pgfqpoint{0.000000in}{0.000000in}}%
\pgfpathlineto{\pgfqpoint{0.000000in}{-0.048611in}}%
\pgfusepath{stroke,fill}%
}%
\begin{pgfscope}%
\pgfsys@transformshift{4.833106in}{1.805660in}%
\pgfsys@useobject{currentmarker}{}%
\end{pgfscope}%
\end{pgfscope}%
\begin{pgfscope}%
\pgfsetbuttcap%
\pgfsetroundjoin%
\definecolor{currentfill}{rgb}{0.000000,0.000000,0.000000}%
\pgfsetfillcolor{currentfill}%
\pgfsetlinewidth{0.803000pt}%
\definecolor{currentstroke}{rgb}{0.000000,0.000000,0.000000}%
\pgfsetstrokecolor{currentstroke}%
\pgfsetdash{}{0pt}%
\pgfsys@defobject{currentmarker}{\pgfqpoint{-0.048611in}{0.000000in}}{\pgfqpoint{0.000000in}{0.000000in}}{%
\pgfpathmoveto{\pgfqpoint{0.000000in}{0.000000in}}%
\pgfpathlineto{\pgfqpoint{-0.048611in}{0.000000in}}%
\pgfusepath{stroke,fill}%
}%
\begin{pgfscope}%
\pgfsys@transformshift{0.456635in}{2.170397in}%
\pgfsys@useobject{currentmarker}{}%
\end{pgfscope}%
\end{pgfscope}%
\begin{pgfscope}%
\pgftext[x=0.289968in,y=2.117635in,left,base]{\rmfamily\fontsize{10.000000}{12.000000}\selectfont \(\displaystyle 0\)}%
\end{pgfscope}%
\begin{pgfscope}%
\pgfsetbuttcap%
\pgfsetroundjoin%
\definecolor{currentfill}{rgb}{0.000000,0.000000,0.000000}%
\pgfsetfillcolor{currentfill}%
\pgfsetlinewidth{0.803000pt}%
\definecolor{currentstroke}{rgb}{0.000000,0.000000,0.000000}%
\pgfsetstrokecolor{currentstroke}%
\pgfsetdash{}{0pt}%
\pgfsys@defobject{currentmarker}{\pgfqpoint{-0.048611in}{0.000000in}}{\pgfqpoint{0.000000in}{0.000000in}}{%
\pgfpathmoveto{\pgfqpoint{0.000000in}{0.000000in}}%
\pgfpathlineto{\pgfqpoint{-0.048611in}{0.000000in}}%
\pgfusepath{stroke,fill}%
}%
\begin{pgfscope}%
\pgfsys@transformshift{0.456635in}{2.575660in}%
\pgfsys@useobject{currentmarker}{}%
\end{pgfscope}%
\end{pgfscope}%
\begin{pgfscope}%
\pgftext[x=0.289968in,y=2.522898in,left,base]{\rmfamily\fontsize{10.000000}{12.000000}\selectfont \(\displaystyle 2\)}%
\end{pgfscope}%
\begin{pgfscope}%
\pgftext[x=0.234413in,y=2.291976in,,bottom,rotate=90.000000]{\rmfamily\fontsize{10.000000}{12.000000}\selectfont y}%
\end{pgfscope}%
\begin{pgfscope}%
\pgfpathrectangle{\pgfqpoint{0.456635in}{1.805660in}}{\pgfqpoint{4.376471in}{0.972632in}}%
\pgfusepath{clip}%
\pgfsetrectcap%
\pgfsetroundjoin%
\pgfsetlinewidth{1.505625pt}%
\definecolor{currentstroke}{rgb}{0.121569,0.466667,0.705882}%
\pgfsetstrokecolor{currentstroke}%
\pgfsetdash{}{0pt}%
\pgfpathmoveto{\pgfqpoint{1.341344in}{1.795660in}}%
\pgfpathlineto{\pgfqpoint{1.349523in}{1.975955in}}%
\pgfpathlineto{\pgfqpoint{1.367117in}{2.263062in}}%
\pgfpathlineto{\pgfqpoint{1.384711in}{2.462672in}}%
\pgfpathlineto{\pgfqpoint{1.402305in}{2.588237in}}%
\pgfpathlineto{\pgfqpoint{1.419899in}{2.653034in}}%
\pgfpathlineto{\pgfqpoint{1.437493in}{2.669812in}}%
\pgfpathlineto{\pgfqpoint{1.455086in}{2.650484in}}%
\pgfpathlineto{\pgfqpoint{1.472680in}{2.605879in}}%
\pgfpathlineto{\pgfqpoint{1.490274in}{2.545555in}}%
\pgfpathlineto{\pgfqpoint{1.543056in}{2.344434in}}%
\pgfpathlineto{\pgfqpoint{1.560649in}{2.288072in}}%
\pgfpathlineto{\pgfqpoint{1.578243in}{2.242232in}}%
\pgfpathlineto{\pgfqpoint{1.595837in}{2.208140in}}%
\pgfpathlineto{\pgfqpoint{1.613431in}{2.185984in}}%
\pgfpathlineto{\pgfqpoint{1.631025in}{2.175099in}}%
\pgfpathlineto{\pgfqpoint{1.648619in}{2.174176in}}%
\pgfpathlineto{\pgfqpoint{1.666213in}{2.181453in}}%
\pgfpathlineto{\pgfqpoint{1.683806in}{2.194923in}}%
\pgfpathlineto{\pgfqpoint{1.701400in}{2.212505in}}%
\pgfpathlineto{\pgfqpoint{1.754182in}{2.271112in}}%
\pgfpathlineto{\pgfqpoint{1.771776in}{2.287749in}}%
\pgfpathlineto{\pgfqpoint{1.789370in}{2.301431in}}%
\pgfpathlineto{\pgfqpoint{1.806963in}{2.311850in}}%
\pgfpathlineto{\pgfqpoint{1.824557in}{2.319057in}}%
\pgfpathlineto{\pgfqpoint{1.842151in}{2.323416in}}%
\pgfpathlineto{\pgfqpoint{1.859745in}{2.325530in}}%
\pgfpathlineto{\pgfqpoint{1.894933in}{2.326137in}}%
\pgfpathlineto{\pgfqpoint{1.930120in}{2.327358in}}%
\pgfpathlineto{\pgfqpoint{1.947714in}{2.329941in}}%
\pgfpathlineto{\pgfqpoint{1.965308in}{2.334450in}}%
\pgfpathlineto{\pgfqpoint{1.982902in}{2.341084in}}%
\pgfpathlineto{\pgfqpoint{2.000496in}{2.349822in}}%
\pgfpathlineto{\pgfqpoint{2.018090in}{2.360434in}}%
\pgfpathlineto{\pgfqpoint{2.053277in}{2.385519in}}%
\pgfpathlineto{\pgfqpoint{2.088465in}{2.411733in}}%
\pgfpathlineto{\pgfqpoint{2.106059in}{2.423632in}}%
\pgfpathlineto{\pgfqpoint{2.123653in}{2.433938in}}%
\pgfpathlineto{\pgfqpoint{2.141247in}{2.442192in}}%
\pgfpathlineto{\pgfqpoint{2.158840in}{2.448076in}}%
\pgfpathlineto{\pgfqpoint{2.176434in}{2.451431in}}%
\pgfpathlineto{\pgfqpoint{2.194028in}{2.452259in}}%
\pgfpathlineto{\pgfqpoint{2.211622in}{2.450708in}}%
\pgfpathlineto{\pgfqpoint{2.229216in}{2.447051in}}%
\pgfpathlineto{\pgfqpoint{2.246810in}{2.441641in}}%
\pgfpathlineto{\pgfqpoint{2.281997in}{2.427156in}}%
\pgfpathlineto{\pgfqpoint{2.334779in}{2.401144in}}%
\pgfpathlineto{\pgfqpoint{2.369967in}{2.382221in}}%
\pgfpathlineto{\pgfqpoint{2.405154in}{2.360010in}}%
\pgfpathlineto{\pgfqpoint{2.422748in}{2.346469in}}%
\pgfpathlineto{\pgfqpoint{2.440342in}{2.330511in}}%
\pgfpathlineto{\pgfqpoint{2.457936in}{2.311536in}}%
\pgfpathlineto{\pgfqpoint{2.475530in}{2.289004in}}%
\pgfpathlineto{\pgfqpoint{2.493124in}{2.262497in}}%
\pgfpathlineto{\pgfqpoint{2.510717in}{2.231770in}}%
\pgfpathlineto{\pgfqpoint{2.528311in}{2.196798in}}%
\pgfpathlineto{\pgfqpoint{2.545905in}{2.157807in}}%
\pgfpathlineto{\pgfqpoint{2.581093in}{2.070006in}}%
\pgfpathlineto{\pgfqpoint{2.651468in}{1.884227in}}%
\pgfpathlineto{\pgfqpoint{2.669062in}{1.843610in}}%
\pgfpathlineto{\pgfqpoint{2.686656in}{1.808227in}}%
\pgfpathlineto{\pgfqpoint{2.694318in}{1.795660in}}%
\pgfpathmoveto{\pgfqpoint{2.818328in}{1.795660in}}%
\pgfpathlineto{\pgfqpoint{2.827407in}{1.810579in}}%
\pgfpathlineto{\pgfqpoint{2.845001in}{1.845400in}}%
\pgfpathlineto{\pgfqpoint{2.862594in}{1.884690in}}%
\pgfpathlineto{\pgfqpoint{2.950564in}{2.093340in}}%
\pgfpathlineto{\pgfqpoint{2.968158in}{2.127801in}}%
\pgfpathlineto{\pgfqpoint{2.985751in}{2.157638in}}%
\pgfpathlineto{\pgfqpoint{3.003345in}{2.182531in}}%
\pgfpathlineto{\pgfqpoint{3.020939in}{2.202504in}}%
\pgfpathlineto{\pgfqpoint{3.038533in}{2.217904in}}%
\pgfpathlineto{\pgfqpoint{3.056127in}{2.229367in}}%
\pgfpathlineto{\pgfqpoint{3.073721in}{2.237755in}}%
\pgfpathlineto{\pgfqpoint{3.108908in}{2.249453in}}%
\pgfpathlineto{\pgfqpoint{3.144096in}{2.261552in}}%
\pgfpathlineto{\pgfqpoint{3.161690in}{2.270095in}}%
\pgfpathlineto{\pgfqpoint{3.179284in}{2.281160in}}%
\pgfpathlineto{\pgfqpoint{3.196878in}{2.295051in}}%
\pgfpathlineto{\pgfqpoint{3.214472in}{2.311756in}}%
\pgfpathlineto{\pgfqpoint{3.232065in}{2.330943in}}%
\pgfpathlineto{\pgfqpoint{3.302441in}{2.416048in}}%
\pgfpathlineto{\pgfqpoint{3.320035in}{2.433569in}}%
\pgfpathlineto{\pgfqpoint{3.337628in}{2.446970in}}%
\pgfpathlineto{\pgfqpoint{3.355222in}{2.455036in}}%
\pgfpathlineto{\pgfqpoint{3.372816in}{2.456738in}}%
\pgfpathlineto{\pgfqpoint{3.390410in}{2.451303in}}%
\pgfpathlineto{\pgfqpoint{3.408004in}{2.438270in}}%
\pgfpathlineto{\pgfqpoint{3.425598in}{2.417522in}}%
\pgfpathlineto{\pgfqpoint{3.443192in}{2.389310in}}%
\pgfpathlineto{\pgfqpoint{3.460785in}{2.354237in}}%
\pgfpathlineto{\pgfqpoint{3.478379in}{2.313240in}}%
\pgfpathlineto{\pgfqpoint{3.495973in}{2.267535in}}%
\pgfpathlineto{\pgfqpoint{3.531161in}{2.167897in}}%
\pgfpathlineto{\pgfqpoint{3.566349in}{2.068053in}}%
\pgfpathlineto{\pgfqpoint{3.583942in}{2.022004in}}%
\pgfpathlineto{\pgfqpoint{3.601536in}{1.980372in}}%
\pgfpathlineto{\pgfqpoint{3.619130in}{1.944247in}}%
\pgfpathlineto{\pgfqpoint{3.636724in}{1.914432in}}%
\pgfpathlineto{\pgfqpoint{3.654318in}{1.891411in}}%
\pgfpathlineto{\pgfqpoint{3.671912in}{1.875348in}}%
\pgfpathlineto{\pgfqpoint{3.689506in}{1.866092in}}%
\pgfpathlineto{\pgfqpoint{3.707099in}{1.863212in}}%
\pgfpathlineto{\pgfqpoint{3.724693in}{1.866039in}}%
\pgfpathlineto{\pgfqpoint{3.742287in}{1.873722in}}%
\pgfpathlineto{\pgfqpoint{3.759881in}{1.885296in}}%
\pgfpathlineto{\pgfqpoint{3.777475in}{1.899748in}}%
\pgfpathlineto{\pgfqpoint{3.812662in}{1.933380in}}%
\pgfpathlineto{\pgfqpoint{3.847850in}{1.967885in}}%
\pgfpathlineto{\pgfqpoint{3.883038in}{1.999072in}}%
\pgfpathlineto{\pgfqpoint{3.918226in}{2.025755in}}%
\pgfpathlineto{\pgfqpoint{3.971007in}{2.060692in}}%
\pgfpathlineto{\pgfqpoint{4.006195in}{2.084596in}}%
\pgfpathlineto{\pgfqpoint{4.041383in}{2.111601in}}%
\pgfpathlineto{\pgfqpoint{4.076570in}{2.142209in}}%
\pgfpathlineto{\pgfqpoint{4.182133in}{2.239076in}}%
\pgfpathlineto{\pgfqpoint{4.270103in}{2.311773in}}%
\pgfpathlineto{\pgfqpoint{4.287696in}{2.329621in}}%
\pgfpathlineto{\pgfqpoint{4.305290in}{2.350505in}}%
\pgfpathlineto{\pgfqpoint{4.322884in}{2.375325in}}%
\pgfpathlineto{\pgfqpoint{4.340478in}{2.404873in}}%
\pgfpathlineto{\pgfqpoint{4.358072in}{2.439738in}}%
\pgfpathlineto{\pgfqpoint{4.375666in}{2.480208in}}%
\pgfpathlineto{\pgfqpoint{4.393260in}{2.526183in}}%
\pgfpathlineto{\pgfqpoint{4.410853in}{2.577103in}}%
\pgfpathlineto{\pgfqpoint{4.446041in}{2.688965in}}%
\pgfpathlineto{\pgfqpoint{4.477158in}{2.788291in}}%
\pgfpathmoveto{\pgfqpoint{4.625892in}{2.788291in}}%
\pgfpathlineto{\pgfqpoint{4.639573in}{2.727493in}}%
\pgfpathlineto{\pgfqpoint{4.674761in}{2.541356in}}%
\pgfpathlineto{\pgfqpoint{4.692355in}{2.450108in}}%
\pgfpathlineto{\pgfqpoint{4.709949in}{2.376079in}}%
\pgfpathlineto{\pgfqpoint{4.727543in}{2.335308in}}%
\pgfpathlineto{\pgfqpoint{4.745137in}{2.347323in}}%
\pgfpathlineto{\pgfqpoint{4.762730in}{2.435288in}}%
\pgfpathlineto{\pgfqpoint{4.780324in}{2.626113in}}%
\pgfpathlineto{\pgfqpoint{4.789121in}{2.788291in}}%
\pgfpathlineto{\pgfqpoint{4.789121in}{2.788291in}}%
\pgfusepath{stroke}%
\end{pgfscope}%
\begin{pgfscope}%
\pgfpathrectangle{\pgfqpoint{0.456635in}{1.805660in}}{\pgfqpoint{4.376471in}{0.972632in}}%
\pgfusepath{clip}%
\pgfsetbuttcap%
\pgfsetroundjoin%
\pgfsetlinewidth{1.505625pt}%
\definecolor{currentstroke}{rgb}{1.000000,0.498039,0.054902}%
\pgfsetstrokecolor{currentstroke}%
\pgfsetdash{{9.600000pt}{2.400000pt}{1.500000pt}{2.400000pt}}{0.000000pt}%
\pgfpathmoveto{\pgfqpoint{1.341344in}{1.795660in}}%
\pgfpathlineto{\pgfqpoint{1.349523in}{1.975956in}}%
\pgfpathlineto{\pgfqpoint{1.367117in}{2.263063in}}%
\pgfpathlineto{\pgfqpoint{1.384711in}{2.462673in}}%
\pgfpathlineto{\pgfqpoint{1.402305in}{2.588238in}}%
\pgfpathlineto{\pgfqpoint{1.419899in}{2.653035in}}%
\pgfpathlineto{\pgfqpoint{1.437493in}{2.669813in}}%
\pgfpathlineto{\pgfqpoint{1.455086in}{2.650484in}}%
\pgfpathlineto{\pgfqpoint{1.472680in}{2.605880in}}%
\pgfpathlineto{\pgfqpoint{1.490274in}{2.545555in}}%
\pgfpathlineto{\pgfqpoint{1.543056in}{2.344436in}}%
\pgfpathlineto{\pgfqpoint{1.560649in}{2.288073in}}%
\pgfpathlineto{\pgfqpoint{1.578243in}{2.242233in}}%
\pgfpathlineto{\pgfqpoint{1.595837in}{2.208142in}}%
\pgfpathlineto{\pgfqpoint{1.613431in}{2.185985in}}%
\pgfpathlineto{\pgfqpoint{1.631025in}{2.175100in}}%
\pgfpathlineto{\pgfqpoint{1.648619in}{2.174177in}}%
\pgfpathlineto{\pgfqpoint{1.666213in}{2.181455in}}%
\pgfpathlineto{\pgfqpoint{1.683806in}{2.194925in}}%
\pgfpathlineto{\pgfqpoint{1.701400in}{2.212506in}}%
\pgfpathlineto{\pgfqpoint{1.754182in}{2.271114in}}%
\pgfpathlineto{\pgfqpoint{1.771776in}{2.287748in}}%
\pgfpathlineto{\pgfqpoint{1.789370in}{2.301432in}}%
\pgfpathlineto{\pgfqpoint{1.806963in}{2.311850in}}%
\pgfpathlineto{\pgfqpoint{1.824557in}{2.319057in}}%
\pgfpathlineto{\pgfqpoint{1.842151in}{2.323415in}}%
\pgfpathlineto{\pgfqpoint{1.859745in}{2.325529in}}%
\pgfpathlineto{\pgfqpoint{1.894933in}{2.326138in}}%
\pgfpathlineto{\pgfqpoint{1.930120in}{2.327357in}}%
\pgfpathlineto{\pgfqpoint{1.947714in}{2.329941in}}%
\pgfpathlineto{\pgfqpoint{1.965308in}{2.334450in}}%
\pgfpathlineto{\pgfqpoint{1.982902in}{2.341083in}}%
\pgfpathlineto{\pgfqpoint{2.000496in}{2.349821in}}%
\pgfpathlineto{\pgfqpoint{2.018090in}{2.360435in}}%
\pgfpathlineto{\pgfqpoint{2.053277in}{2.385519in}}%
\pgfpathlineto{\pgfqpoint{2.088465in}{2.411733in}}%
\pgfpathlineto{\pgfqpoint{2.106059in}{2.423632in}}%
\pgfpathlineto{\pgfqpoint{2.123653in}{2.433938in}}%
\pgfpathlineto{\pgfqpoint{2.141247in}{2.442192in}}%
\pgfpathlineto{\pgfqpoint{2.158840in}{2.448076in}}%
\pgfpathlineto{\pgfqpoint{2.176434in}{2.451431in}}%
\pgfpathlineto{\pgfqpoint{2.194028in}{2.452259in}}%
\pgfpathlineto{\pgfqpoint{2.211622in}{2.450708in}}%
\pgfpathlineto{\pgfqpoint{2.229216in}{2.447051in}}%
\pgfpathlineto{\pgfqpoint{2.246810in}{2.441641in}}%
\pgfpathlineto{\pgfqpoint{2.281997in}{2.427156in}}%
\pgfpathlineto{\pgfqpoint{2.334779in}{2.401144in}}%
\pgfpathlineto{\pgfqpoint{2.369967in}{2.382221in}}%
\pgfpathlineto{\pgfqpoint{2.405154in}{2.360010in}}%
\pgfpathlineto{\pgfqpoint{2.422748in}{2.346469in}}%
\pgfpathlineto{\pgfqpoint{2.440342in}{2.330512in}}%
\pgfpathlineto{\pgfqpoint{2.457936in}{2.311537in}}%
\pgfpathlineto{\pgfqpoint{2.475530in}{2.289005in}}%
\pgfpathlineto{\pgfqpoint{2.493124in}{2.262497in}}%
\pgfpathlineto{\pgfqpoint{2.510717in}{2.231770in}}%
\pgfpathlineto{\pgfqpoint{2.528311in}{2.196798in}}%
\pgfpathlineto{\pgfqpoint{2.545905in}{2.157806in}}%
\pgfpathlineto{\pgfqpoint{2.581093in}{2.070006in}}%
\pgfpathlineto{\pgfqpoint{2.651468in}{1.884226in}}%
\pgfpathlineto{\pgfqpoint{2.669062in}{1.843609in}}%
\pgfpathlineto{\pgfqpoint{2.686656in}{1.808227in}}%
\pgfpathlineto{\pgfqpoint{2.694318in}{1.795660in}}%
\pgfpathmoveto{\pgfqpoint{2.818328in}{1.795660in}}%
\pgfpathlineto{\pgfqpoint{2.827407in}{1.810579in}}%
\pgfpathlineto{\pgfqpoint{2.845001in}{1.845400in}}%
\pgfpathlineto{\pgfqpoint{2.862594in}{1.884691in}}%
\pgfpathlineto{\pgfqpoint{2.950564in}{2.093342in}}%
\pgfpathlineto{\pgfqpoint{2.968158in}{2.127802in}}%
\pgfpathlineto{\pgfqpoint{2.985751in}{2.157638in}}%
\pgfpathlineto{\pgfqpoint{3.003345in}{2.182531in}}%
\pgfpathlineto{\pgfqpoint{3.020939in}{2.202505in}}%
\pgfpathlineto{\pgfqpoint{3.038533in}{2.217905in}}%
\pgfpathlineto{\pgfqpoint{3.056127in}{2.229368in}}%
\pgfpathlineto{\pgfqpoint{3.073721in}{2.237755in}}%
\pgfpathlineto{\pgfqpoint{3.108908in}{2.249454in}}%
\pgfpathlineto{\pgfqpoint{3.144096in}{2.261552in}}%
\pgfpathlineto{\pgfqpoint{3.161690in}{2.270095in}}%
\pgfpathlineto{\pgfqpoint{3.179284in}{2.281160in}}%
\pgfpathlineto{\pgfqpoint{3.196878in}{2.295050in}}%
\pgfpathlineto{\pgfqpoint{3.214472in}{2.311755in}}%
\pgfpathlineto{\pgfqpoint{3.232065in}{2.330943in}}%
\pgfpathlineto{\pgfqpoint{3.302441in}{2.416048in}}%
\pgfpathlineto{\pgfqpoint{3.320035in}{2.433568in}}%
\pgfpathlineto{\pgfqpoint{3.337628in}{2.446968in}}%
\pgfpathlineto{\pgfqpoint{3.355222in}{2.455035in}}%
\pgfpathlineto{\pgfqpoint{3.372816in}{2.456737in}}%
\pgfpathlineto{\pgfqpoint{3.390410in}{2.451303in}}%
\pgfpathlineto{\pgfqpoint{3.408004in}{2.438269in}}%
\pgfpathlineto{\pgfqpoint{3.425598in}{2.417522in}}%
\pgfpathlineto{\pgfqpoint{3.443192in}{2.389309in}}%
\pgfpathlineto{\pgfqpoint{3.460785in}{2.354236in}}%
\pgfpathlineto{\pgfqpoint{3.478379in}{2.313240in}}%
\pgfpathlineto{\pgfqpoint{3.495973in}{2.267535in}}%
\pgfpathlineto{\pgfqpoint{3.531161in}{2.167897in}}%
\pgfpathlineto{\pgfqpoint{3.566349in}{2.068053in}}%
\pgfpathlineto{\pgfqpoint{3.583942in}{2.022004in}}%
\pgfpathlineto{\pgfqpoint{3.601536in}{1.980373in}}%
\pgfpathlineto{\pgfqpoint{3.619130in}{1.944249in}}%
\pgfpathlineto{\pgfqpoint{3.636724in}{1.914432in}}%
\pgfpathlineto{\pgfqpoint{3.654318in}{1.891412in}}%
\pgfpathlineto{\pgfqpoint{3.671912in}{1.875349in}}%
\pgfpathlineto{\pgfqpoint{3.689506in}{1.866093in}}%
\pgfpathlineto{\pgfqpoint{3.707099in}{1.863213in}}%
\pgfpathlineto{\pgfqpoint{3.724693in}{1.866040in}}%
\pgfpathlineto{\pgfqpoint{3.742287in}{1.873723in}}%
\pgfpathlineto{\pgfqpoint{3.759881in}{1.885297in}}%
\pgfpathlineto{\pgfqpoint{3.777475in}{1.899749in}}%
\pgfpathlineto{\pgfqpoint{3.812662in}{1.933380in}}%
\pgfpathlineto{\pgfqpoint{3.847850in}{1.967885in}}%
\pgfpathlineto{\pgfqpoint{3.883038in}{1.999072in}}%
\pgfpathlineto{\pgfqpoint{3.918226in}{2.025755in}}%
\pgfpathlineto{\pgfqpoint{3.971007in}{2.060692in}}%
\pgfpathlineto{\pgfqpoint{4.006195in}{2.084596in}}%
\pgfpathlineto{\pgfqpoint{4.041383in}{2.111601in}}%
\pgfpathlineto{\pgfqpoint{4.076570in}{2.142209in}}%
\pgfpathlineto{\pgfqpoint{4.182133in}{2.239076in}}%
\pgfpathlineto{\pgfqpoint{4.270103in}{2.311773in}}%
\pgfpathlineto{\pgfqpoint{4.287696in}{2.329621in}}%
\pgfpathlineto{\pgfqpoint{4.305290in}{2.350504in}}%
\pgfpathlineto{\pgfqpoint{4.322884in}{2.375324in}}%
\pgfpathlineto{\pgfqpoint{4.340478in}{2.404872in}}%
\pgfpathlineto{\pgfqpoint{4.358072in}{2.439737in}}%
\pgfpathlineto{\pgfqpoint{4.375666in}{2.480207in}}%
\pgfpathlineto{\pgfqpoint{4.393260in}{2.526182in}}%
\pgfpathlineto{\pgfqpoint{4.410853in}{2.577103in}}%
\pgfpathlineto{\pgfqpoint{4.446041in}{2.688965in}}%
\pgfpathlineto{\pgfqpoint{4.477159in}{2.788291in}}%
\pgfpathmoveto{\pgfqpoint{4.625892in}{2.788291in}}%
\pgfpathlineto{\pgfqpoint{4.639573in}{2.727494in}}%
\pgfpathlineto{\pgfqpoint{4.674761in}{2.541357in}}%
\pgfpathlineto{\pgfqpoint{4.692355in}{2.450109in}}%
\pgfpathlineto{\pgfqpoint{4.709949in}{2.376079in}}%
\pgfpathlineto{\pgfqpoint{4.727543in}{2.335309in}}%
\pgfpathlineto{\pgfqpoint{4.745137in}{2.347323in}}%
\pgfpathlineto{\pgfqpoint{4.762730in}{2.435287in}}%
\pgfpathlineto{\pgfqpoint{4.780324in}{2.626109in}}%
\pgfpathlineto{\pgfqpoint{4.789121in}{2.788291in}}%
\pgfpathlineto{\pgfqpoint{4.789121in}{2.788291in}}%
\pgfusepath{stroke}%
\end{pgfscope}%
\begin{pgfscope}%
\pgfsetrectcap%
\pgfsetmiterjoin%
\pgfsetlinewidth{0.803000pt}%
\definecolor{currentstroke}{rgb}{0.000000,0.000000,0.000000}%
\pgfsetstrokecolor{currentstroke}%
\pgfsetdash{}{0pt}%
\pgfpathmoveto{\pgfqpoint{0.456635in}{1.805660in}}%
\pgfpathlineto{\pgfqpoint{0.456635in}{2.778291in}}%
\pgfusepath{stroke}%
\end{pgfscope}%
\begin{pgfscope}%
\pgfsetrectcap%
\pgfsetmiterjoin%
\pgfsetlinewidth{0.803000pt}%
\definecolor{currentstroke}{rgb}{0.000000,0.000000,0.000000}%
\pgfsetstrokecolor{currentstroke}%
\pgfsetdash{}{0pt}%
\pgfpathmoveto{\pgfqpoint{4.833106in}{1.805660in}}%
\pgfpathlineto{\pgfqpoint{4.833106in}{2.778291in}}%
\pgfusepath{stroke}%
\end{pgfscope}%
\begin{pgfscope}%
\pgfsetrectcap%
\pgfsetmiterjoin%
\pgfsetlinewidth{0.803000pt}%
\definecolor{currentstroke}{rgb}{0.000000,0.000000,0.000000}%
\pgfsetstrokecolor{currentstroke}%
\pgfsetdash{}{0pt}%
\pgfpathmoveto{\pgfqpoint{0.456635in}{1.805660in}}%
\pgfpathlineto{\pgfqpoint{4.833106in}{1.805660in}}%
\pgfusepath{stroke}%
\end{pgfscope}%
\begin{pgfscope}%
\pgfsetrectcap%
\pgfsetmiterjoin%
\pgfsetlinewidth{0.803000pt}%
\definecolor{currentstroke}{rgb}{0.000000,0.000000,0.000000}%
\pgfsetstrokecolor{currentstroke}%
\pgfsetdash{}{0pt}%
\pgfpathmoveto{\pgfqpoint{0.456635in}{2.778291in}}%
\pgfpathlineto{\pgfqpoint{4.833106in}{2.778291in}}%
\pgfusepath{stroke}%
\end{pgfscope}%
\begin{pgfscope}%
\pgfsetbuttcap%
\pgfsetmiterjoin%
\definecolor{currentfill}{rgb}{1.000000,1.000000,1.000000}%
\pgfsetfillcolor{currentfill}%
\pgfsetfillopacity{0.800000}%
\pgfsetlinewidth{1.003750pt}%
\definecolor{currentstroke}{rgb}{0.800000,0.800000,0.800000}%
\pgfsetstrokecolor{currentstroke}%
\pgfsetstrokeopacity{0.800000}%
\pgfsetdash{}{0pt}%
\pgfpathmoveto{\pgfqpoint{0.553858in}{1.875104in}}%
\pgfpathlineto{\pgfqpoint{1.337183in}{1.875104in}}%
\pgfpathquadraticcurveto{\pgfqpoint{1.364960in}{1.875104in}}{\pgfqpoint{1.364960in}{1.902882in}}%
\pgfpathlineto{\pgfqpoint{1.364960in}{2.706430in}}%
\pgfpathquadraticcurveto{\pgfqpoint{1.364960in}{2.734208in}}{\pgfqpoint{1.337183in}{2.734208in}}%
\pgfpathlineto{\pgfqpoint{0.553858in}{2.734208in}}%
\pgfpathquadraticcurveto{\pgfqpoint{0.526080in}{2.734208in}}{\pgfqpoint{0.526080in}{2.706430in}}%
\pgfpathlineto{\pgfqpoint{0.526080in}{1.902882in}}%
\pgfpathquadraticcurveto{\pgfqpoint{0.526080in}{1.875104in}}{\pgfqpoint{0.553858in}{1.875104in}}%
\pgfpathclose%
\pgfusepath{stroke,fill}%
\end{pgfscope}%
\begin{pgfscope}%
\pgfsetrectcap%
\pgfsetroundjoin%
\pgfsetlinewidth{1.505625pt}%
\definecolor{currentstroke}{rgb}{0.121569,0.466667,0.705882}%
\pgfsetstrokecolor{currentstroke}%
\pgfsetdash{}{0pt}%
\pgfpathmoveto{\pgfqpoint{0.581635in}{2.620736in}}%
\pgfpathlineto{\pgfqpoint{0.859413in}{2.620736in}}%
\pgfusepath{stroke}%
\end{pgfscope}%
\begin{pgfscope}%
\pgftext[x=0.970524in,y=2.572125in,left,base]{\rmfamily\fontsize{10.000000}{12.000000}\selectfont \(\displaystyle \widetilde{\Phi}^* \theta\)}%
\end{pgfscope}%
\begin{pgfscope}%
\pgfsetbuttcap%
\pgfsetroundjoin%
\pgfsetlinewidth{1.505625pt}%
\definecolor{currentstroke}{rgb}{1.000000,0.498039,0.054902}%
\pgfsetstrokecolor{currentstroke}%
\pgfsetdash{{9.600000pt}{2.400000pt}{1.500000pt}{2.400000pt}}{0.000000pt}%
\pgfpathmoveto{\pgfqpoint{0.581635in}{2.415875in}}%
\pgfpathlineto{\pgfqpoint{0.859413in}{2.415875in}}%
\pgfusepath{stroke}%
\end{pgfscope}%
\begin{pgfscope}%
\pgftext[x=0.970524in,y=2.367264in,left,base]{\rmfamily\fontsize{10.000000}{12.000000}\selectfont \(\displaystyle \widetilde{K}u\)}%
\end{pgfscope}%
\begin{pgfscope}%
\pgfsetbuttcap%
\pgfsetroundjoin%
\definecolor{currentfill}{rgb}{1.000000,0.000000,0.000000}%
\pgfsetfillcolor{currentfill}%
\pgfsetlinewidth{2.007500pt}%
\definecolor{currentstroke}{rgb}{1.000000,0.000000,0.000000}%
\pgfsetstrokecolor{currentstroke}%
\pgfsetdash{}{0pt}%
\pgfpathmoveto{\pgfqpoint{0.678857in}{2.199865in}}%
\pgfpathlineto{\pgfqpoint{0.762191in}{2.199865in}}%
\pgfpathmoveto{\pgfqpoint{0.720524in}{2.158198in}}%
\pgfpathlineto{\pgfqpoint{0.720524in}{2.241532in}}%
\pgfusepath{stroke,fill}%
\end{pgfscope}%
\begin{pgfscope}%
\pgftext[x=0.970524in,y=2.163407in,left,base]{\rmfamily\fontsize{10.000000}{12.000000}\selectfont train}%
\end{pgfscope}%
\begin{pgfscope}%
\pgfsetbuttcap%
\pgfsetroundjoin%
\definecolor{currentfill}{rgb}{0.000000,0.000000,0.000000}%
\pgfsetfillcolor{currentfill}%
\pgfsetlinewidth{1.003750pt}%
\definecolor{currentstroke}{rgb}{0.000000,0.000000,0.000000}%
\pgfsetstrokecolor{currentstroke}%
\pgfsetdash{}{0pt}%
\pgfsys@defobject{currentmarker}{\pgfqpoint{-0.020833in}{-0.020833in}}{\pgfqpoint{0.020833in}{0.020833in}}{%
\pgfpathmoveto{\pgfqpoint{0.000000in}{-0.020833in}}%
\pgfpathcurveto{\pgfqpoint{0.005525in}{-0.020833in}}{\pgfqpoint{0.010825in}{-0.018638in}}{\pgfqpoint{0.014731in}{-0.014731in}}%
\pgfpathcurveto{\pgfqpoint{0.018638in}{-0.010825in}}{\pgfqpoint{0.020833in}{-0.005525in}}{\pgfqpoint{0.020833in}{0.000000in}}%
\pgfpathcurveto{\pgfqpoint{0.020833in}{0.005525in}}{\pgfqpoint{0.018638in}{0.010825in}}{\pgfqpoint{0.014731in}{0.014731in}}%
\pgfpathcurveto{\pgfqpoint{0.010825in}{0.018638in}}{\pgfqpoint{0.005525in}{0.020833in}}{\pgfqpoint{0.000000in}{0.020833in}}%
\pgfpathcurveto{\pgfqpoint{-0.005525in}{0.020833in}}{\pgfqpoint{-0.010825in}{0.018638in}}{\pgfqpoint{-0.014731in}{0.014731in}}%
\pgfpathcurveto{\pgfqpoint{-0.018638in}{0.010825in}}{\pgfqpoint{-0.020833in}{0.005525in}}{\pgfqpoint{-0.020833in}{0.000000in}}%
\pgfpathcurveto{\pgfqpoint{-0.020833in}{-0.005525in}}{\pgfqpoint{-0.018638in}{-0.010825in}}{\pgfqpoint{-0.014731in}{-0.014731in}}%
\pgfpathcurveto{\pgfqpoint{-0.010825in}{-0.018638in}}{\pgfqpoint{-0.005525in}{-0.020833in}}{\pgfqpoint{0.000000in}{-0.020833in}}%
\pgfpathclose%
\pgfusepath{stroke,fill}%
}%
\begin{pgfscope}%
\pgfsys@transformshift{0.720524in}{1.996008in}%
\pgfsys@useobject{currentmarker}{}%
\end{pgfscope}%
\end{pgfscope}%
\begin{pgfscope}%
\pgftext[x=0.970524in,y=1.959549in,left,base]{\rmfamily\fontsize{10.000000}{12.000000}\selectfont test}%
\end{pgfscope}%
\begin{pgfscope}%
\pgfsetbuttcap%
\pgfsetmiterjoin%
\definecolor{currentfill}{rgb}{1.000000,1.000000,1.000000}%
\pgfsetfillcolor{currentfill}%
\pgfsetlinewidth{0.000000pt}%
\definecolor{currentstroke}{rgb}{0.000000,0.000000,0.000000}%
\pgfsetstrokecolor{currentstroke}%
\pgfsetstrokeopacity{0.000000}%
\pgfsetdash{}{0pt}%
\pgfpathmoveto{\pgfqpoint{5.562518in}{1.805660in}}%
\pgfpathlineto{\pgfqpoint{9.938988in}{1.805660in}}%
\pgfpathlineto{\pgfqpoint{9.938988in}{2.778291in}}%
\pgfpathlineto{\pgfqpoint{5.562518in}{2.778291in}}%
\pgfpathclose%
\pgfusepath{fill}%
\end{pgfscope}%
\begin{pgfscope}%
\pgfpathrectangle{\pgfqpoint{5.562518in}{1.805660in}}{\pgfqpoint{4.376471in}{0.972632in}}%
\pgfusepath{clip}%
\pgfsetbuttcap%
\pgfsetroundjoin%
\definecolor{currentfill}{rgb}{1.000000,0.000000,0.000000}%
\pgfsetfillcolor{currentfill}%
\pgfsetlinewidth{2.007500pt}%
\definecolor{currentstroke}{rgb}{1.000000,0.000000,0.000000}%
\pgfsetstrokecolor{currentstroke}%
\pgfsetdash{}{0pt}%
\pgfpathmoveto{\pgfqpoint{7.707476in}{1.959527in}}%
\pgfpathlineto{\pgfqpoint{7.790809in}{1.959527in}}%
\pgfpathmoveto{\pgfqpoint{7.749143in}{1.917861in}}%
\pgfpathlineto{\pgfqpoint{7.749143in}{2.001194in}}%
\pgfusepath{stroke,fill}%
\end{pgfscope}%
\begin{pgfscope}%
\pgfpathrectangle{\pgfqpoint{5.562518in}{1.805660in}}{\pgfqpoint{4.376471in}{0.972632in}}%
\pgfusepath{clip}%
\pgfsetbuttcap%
\pgfsetroundjoin%
\definecolor{currentfill}{rgb}{1.000000,0.000000,0.000000}%
\pgfsetfillcolor{currentfill}%
\pgfsetlinewidth{2.007500pt}%
\definecolor{currentstroke}{rgb}{1.000000,0.000000,0.000000}%
\pgfsetstrokecolor{currentstroke}%
\pgfsetdash{}{0pt}%
\pgfpathmoveto{\pgfqpoint{9.724764in}{2.618629in}}%
\pgfpathlineto{\pgfqpoint{9.808097in}{2.618629in}}%
\pgfpathmoveto{\pgfqpoint{9.766430in}{2.576962in}}%
\pgfpathlineto{\pgfqpoint{9.766430in}{2.660296in}}%
\pgfusepath{stroke,fill}%
\end{pgfscope}%
\begin{pgfscope}%
\pgfpathrectangle{\pgfqpoint{5.562518in}{1.805660in}}{\pgfqpoint{4.376471in}{0.972632in}}%
\pgfusepath{clip}%
\pgfsetbuttcap%
\pgfsetroundjoin%
\definecolor{currentfill}{rgb}{1.000000,0.000000,0.000000}%
\pgfsetfillcolor{currentfill}%
\pgfsetlinewidth{2.007500pt}%
\definecolor{currentstroke}{rgb}{1.000000,0.000000,0.000000}%
\pgfsetstrokecolor{currentstroke}%
\pgfsetdash{}{0pt}%
\pgfpathmoveto{\pgfqpoint{8.958985in}{1.994690in}}%
\pgfpathlineto{\pgfqpoint{9.042318in}{1.994690in}}%
\pgfpathmoveto{\pgfqpoint{9.000652in}{1.953024in}}%
\pgfpathlineto{\pgfqpoint{9.000652in}{2.036357in}}%
\pgfusepath{stroke,fill}%
\end{pgfscope}%
\begin{pgfscope}%
\pgfpathrectangle{\pgfqpoint{5.562518in}{1.805660in}}{\pgfqpoint{4.376471in}{0.972632in}}%
\pgfusepath{clip}%
\pgfsetbuttcap%
\pgfsetroundjoin%
\definecolor{currentfill}{rgb}{1.000000,0.000000,0.000000}%
\pgfsetfillcolor{currentfill}%
\pgfsetlinewidth{2.007500pt}%
\definecolor{currentstroke}{rgb}{1.000000,0.000000,0.000000}%
\pgfsetstrokecolor{currentstroke}%
\pgfsetdash{}{0pt}%
\pgfpathmoveto{\pgfqpoint{8.492154in}{2.343070in}}%
\pgfpathlineto{\pgfqpoint{8.575487in}{2.343070in}}%
\pgfpathmoveto{\pgfqpoint{8.533821in}{2.301403in}}%
\pgfpathlineto{\pgfqpoint{8.533821in}{2.384737in}}%
\pgfusepath{stroke,fill}%
\end{pgfscope}%
\begin{pgfscope}%
\pgfpathrectangle{\pgfqpoint{5.562518in}{1.805660in}}{\pgfqpoint{4.376471in}{0.972632in}}%
\pgfusepath{clip}%
\pgfsetbuttcap%
\pgfsetroundjoin%
\definecolor{currentfill}{rgb}{1.000000,0.000000,0.000000}%
\pgfsetfillcolor{currentfill}%
\pgfsetlinewidth{2.007500pt}%
\definecolor{currentstroke}{rgb}{1.000000,0.000000,0.000000}%
\pgfsetstrokecolor{currentstroke}%
\pgfsetdash{}{0pt}%
\pgfpathmoveto{\pgfqpoint{6.942394in}{2.187424in}}%
\pgfpathlineto{\pgfqpoint{7.025727in}{2.187424in}}%
\pgfpathmoveto{\pgfqpoint{6.984061in}{2.145757in}}%
\pgfpathlineto{\pgfqpoint{6.984061in}{2.229090in}}%
\pgfusepath{stroke,fill}%
\end{pgfscope}%
\begin{pgfscope}%
\pgfpathrectangle{\pgfqpoint{5.562518in}{1.805660in}}{\pgfqpoint{4.376471in}{0.972632in}}%
\pgfusepath{clip}%
\pgfsetbuttcap%
\pgfsetroundjoin%
\definecolor{currentfill}{rgb}{1.000000,0.000000,0.000000}%
\pgfsetfillcolor{currentfill}%
\pgfsetlinewidth{2.007500pt}%
\definecolor{currentstroke}{rgb}{1.000000,0.000000,0.000000}%
\pgfsetstrokecolor{currentstroke}%
\pgfsetdash{}{0pt}%
\pgfpathmoveto{\pgfqpoint{6.942309in}{2.435997in}}%
\pgfpathlineto{\pgfqpoint{7.025643in}{2.435997in}}%
\pgfpathmoveto{\pgfqpoint{6.983976in}{2.394330in}}%
\pgfpathlineto{\pgfqpoint{6.983976in}{2.477664in}}%
\pgfusepath{stroke,fill}%
\end{pgfscope}%
\begin{pgfscope}%
\pgfpathrectangle{\pgfqpoint{5.562518in}{1.805660in}}{\pgfqpoint{4.376471in}{0.972632in}}%
\pgfusepath{clip}%
\pgfsetbuttcap%
\pgfsetroundjoin%
\definecolor{currentfill}{rgb}{1.000000,0.000000,0.000000}%
\pgfsetfillcolor{currentfill}%
\pgfsetlinewidth{2.007500pt}%
\definecolor{currentstroke}{rgb}{1.000000,0.000000,0.000000}%
\pgfsetstrokecolor{currentstroke}%
\pgfsetdash{}{0pt}%
\pgfpathmoveto{\pgfqpoint{6.599506in}{2.373467in}}%
\pgfpathlineto{\pgfqpoint{6.682839in}{2.373467in}}%
\pgfpathmoveto{\pgfqpoint{6.641173in}{2.331800in}}%
\pgfpathlineto{\pgfqpoint{6.641173in}{2.415133in}}%
\pgfusepath{stroke,fill}%
\end{pgfscope}%
\begin{pgfscope}%
\pgfpathrectangle{\pgfqpoint{5.562518in}{1.805660in}}{\pgfqpoint{4.376471in}{0.972632in}}%
\pgfusepath{clip}%
\pgfsetbuttcap%
\pgfsetroundjoin%
\definecolor{currentfill}{rgb}{1.000000,0.000000,0.000000}%
\pgfsetfillcolor{currentfill}%
\pgfsetlinewidth{2.007500pt}%
\definecolor{currentstroke}{rgb}{1.000000,0.000000,0.000000}%
\pgfsetstrokecolor{currentstroke}%
\pgfsetdash{}{0pt}%
\pgfpathmoveto{\pgfqpoint{9.428781in}{2.455378in}}%
\pgfpathlineto{\pgfqpoint{9.512114in}{2.455378in}}%
\pgfpathmoveto{\pgfqpoint{9.470447in}{2.413711in}}%
\pgfpathlineto{\pgfqpoint{9.470447in}{2.497044in}}%
\pgfusepath{stroke,fill}%
\end{pgfscope}%
\begin{pgfscope}%
\pgfpathrectangle{\pgfqpoint{5.562518in}{1.805660in}}{\pgfqpoint{4.376471in}{0.972632in}}%
\pgfusepath{clip}%
\pgfsetbuttcap%
\pgfsetroundjoin%
\definecolor{currentfill}{rgb}{1.000000,0.000000,0.000000}%
\pgfsetfillcolor{currentfill}%
\pgfsetlinewidth{2.007500pt}%
\definecolor{currentstroke}{rgb}{1.000000,0.000000,0.000000}%
\pgfsetstrokecolor{currentstroke}%
\pgfsetdash{}{0pt}%
\pgfpathmoveto{\pgfqpoint{8.500755in}{2.452634in}}%
\pgfpathlineto{\pgfqpoint{8.584088in}{2.452634in}}%
\pgfpathmoveto{\pgfqpoint{8.542421in}{2.410967in}}%
\pgfpathlineto{\pgfqpoint{8.542421in}{2.494300in}}%
\pgfusepath{stroke,fill}%
\end{pgfscope}%
\begin{pgfscope}%
\pgfpathrectangle{\pgfqpoint{5.562518in}{1.805660in}}{\pgfqpoint{4.376471in}{0.972632in}}%
\pgfusepath{clip}%
\pgfsetbuttcap%
\pgfsetroundjoin%
\definecolor{currentfill}{rgb}{1.000000,0.000000,0.000000}%
\pgfsetfillcolor{currentfill}%
\pgfsetlinewidth{2.007500pt}%
\definecolor{currentstroke}{rgb}{1.000000,0.000000,0.000000}%
\pgfsetstrokecolor{currentstroke}%
\pgfsetdash{}{0pt}%
\pgfpathmoveto{\pgfqpoint{8.875232in}{1.944239in}}%
\pgfpathlineto{\pgfqpoint{8.958565in}{1.944239in}}%
\pgfpathmoveto{\pgfqpoint{8.916899in}{1.902572in}}%
\pgfpathlineto{\pgfqpoint{8.916899in}{1.985906in}}%
\pgfusepath{stroke,fill}%
\end{pgfscope}%
\begin{pgfscope}%
\pgfpathrectangle{\pgfqpoint{5.562518in}{1.805660in}}{\pgfqpoint{4.376471in}{0.972632in}}%
\pgfusepath{clip}%
\pgfsetbuttcap%
\pgfsetroundjoin%
\definecolor{currentfill}{rgb}{1.000000,0.000000,0.000000}%
\pgfsetfillcolor{currentfill}%
\pgfsetlinewidth{2.007500pt}%
\definecolor{currentstroke}{rgb}{1.000000,0.000000,0.000000}%
\pgfsetstrokecolor{currentstroke}%
\pgfsetdash{}{0pt}%
\pgfpathmoveto{\pgfqpoint{6.468215in}{2.597129in}}%
\pgfpathlineto{\pgfqpoint{6.551548in}{2.597129in}}%
\pgfpathmoveto{\pgfqpoint{6.509882in}{2.555462in}}%
\pgfpathlineto{\pgfqpoint{6.509882in}{2.638795in}}%
\pgfusepath{stroke,fill}%
\end{pgfscope}%
\begin{pgfscope}%
\pgfpathrectangle{\pgfqpoint{5.562518in}{1.805660in}}{\pgfqpoint{4.376471in}{0.972632in}}%
\pgfusepath{clip}%
\pgfsetbuttcap%
\pgfsetroundjoin%
\definecolor{currentfill}{rgb}{1.000000,0.000000,0.000000}%
\pgfsetfillcolor{currentfill}%
\pgfsetlinewidth{2.007500pt}%
\definecolor{currentstroke}{rgb}{1.000000,0.000000,0.000000}%
\pgfsetstrokecolor{currentstroke}%
\pgfsetdash{}{0pt}%
\pgfpathmoveto{\pgfqpoint{9.791971in}{2.335190in}}%
\pgfpathlineto{\pgfqpoint{9.875304in}{2.335190in}}%
\pgfpathmoveto{\pgfqpoint{9.833637in}{2.293523in}}%
\pgfpathlineto{\pgfqpoint{9.833637in}{2.376856in}}%
\pgfusepath{stroke,fill}%
\end{pgfscope}%
\begin{pgfscope}%
\pgfpathrectangle{\pgfqpoint{5.562518in}{1.805660in}}{\pgfqpoint{4.376471in}{0.972632in}}%
\pgfusepath{clip}%
\pgfsetbuttcap%
\pgfsetroundjoin%
\definecolor{currentfill}{rgb}{1.000000,0.000000,0.000000}%
\pgfsetfillcolor{currentfill}%
\pgfsetlinewidth{2.007500pt}%
\definecolor{currentstroke}{rgb}{1.000000,0.000000,0.000000}%
\pgfsetstrokecolor{currentstroke}%
\pgfsetdash{}{0pt}%
\pgfpathmoveto{\pgfqpoint{9.310674in}{2.287442in}}%
\pgfpathlineto{\pgfqpoint{9.394007in}{2.287442in}}%
\pgfpathmoveto{\pgfqpoint{9.352340in}{2.245775in}}%
\pgfpathlineto{\pgfqpoint{9.352340in}{2.329108in}}%
\pgfusepath{stroke,fill}%
\end{pgfscope}%
\begin{pgfscope}%
\pgfpathrectangle{\pgfqpoint{5.562518in}{1.805660in}}{\pgfqpoint{4.376471in}{0.972632in}}%
\pgfusepath{clip}%
\pgfsetbuttcap%
\pgfsetroundjoin%
\definecolor{currentfill}{rgb}{1.000000,0.000000,0.000000}%
\pgfsetfillcolor{currentfill}%
\pgfsetlinewidth{2.007500pt}%
\definecolor{currentstroke}{rgb}{1.000000,0.000000,0.000000}%
\pgfsetstrokecolor{currentstroke}%
\pgfsetdash{}{0pt}%
\pgfpathmoveto{\pgfqpoint{7.139582in}{2.426913in}}%
\pgfpathlineto{\pgfqpoint{7.222915in}{2.426913in}}%
\pgfpathmoveto{\pgfqpoint{7.181248in}{2.385247in}}%
\pgfpathlineto{\pgfqpoint{7.181248in}{2.468580in}}%
\pgfusepath{stroke,fill}%
\end{pgfscope}%
\begin{pgfscope}%
\pgfpathrectangle{\pgfqpoint{5.562518in}{1.805660in}}{\pgfqpoint{4.376471in}{0.972632in}}%
\pgfusepath{clip}%
\pgfsetbuttcap%
\pgfsetroundjoin%
\definecolor{currentfill}{rgb}{1.000000,0.000000,0.000000}%
\pgfsetfillcolor{currentfill}%
\pgfsetlinewidth{2.007500pt}%
\definecolor{currentstroke}{rgb}{1.000000,0.000000,0.000000}%
\pgfsetstrokecolor{currentstroke}%
\pgfsetdash{}{0pt}%
\pgfpathmoveto{\pgfqpoint{7.032746in}{2.393066in}}%
\pgfpathlineto{\pgfqpoint{7.116080in}{2.393066in}}%
\pgfpathmoveto{\pgfqpoint{7.074413in}{2.351399in}}%
\pgfpathlineto{\pgfqpoint{7.074413in}{2.434732in}}%
\pgfusepath{stroke,fill}%
\end{pgfscope}%
\begin{pgfscope}%
\pgfpathrectangle{\pgfqpoint{5.562518in}{1.805660in}}{\pgfqpoint{4.376471in}{0.972632in}}%
\pgfusepath{clip}%
\pgfsetbuttcap%
\pgfsetroundjoin%
\definecolor{currentfill}{rgb}{1.000000,0.000000,0.000000}%
\pgfsetfillcolor{currentfill}%
\pgfsetlinewidth{2.007500pt}%
\definecolor{currentstroke}{rgb}{1.000000,0.000000,0.000000}%
\pgfsetstrokecolor{currentstroke}%
\pgfsetdash{}{0pt}%
\pgfpathmoveto{\pgfqpoint{7.038277in}{2.340372in}}%
\pgfpathlineto{\pgfqpoint{7.121610in}{2.340372in}}%
\pgfpathmoveto{\pgfqpoint{7.079943in}{2.298706in}}%
\pgfpathlineto{\pgfqpoint{7.079943in}{2.382039in}}%
\pgfusepath{stroke,fill}%
\end{pgfscope}%
\begin{pgfscope}%
\pgfpathrectangle{\pgfqpoint{5.562518in}{1.805660in}}{\pgfqpoint{4.376471in}{0.972632in}}%
\pgfusepath{clip}%
\pgfsetbuttcap%
\pgfsetroundjoin%
\definecolor{currentfill}{rgb}{1.000000,0.000000,0.000000}%
\pgfsetfillcolor{currentfill}%
\pgfsetlinewidth{2.007500pt}%
\definecolor{currentstroke}{rgb}{1.000000,0.000000,0.000000}%
\pgfsetstrokecolor{currentstroke}%
\pgfsetdash{}{0pt}%
\pgfpathmoveto{\pgfqpoint{7.461351in}{2.362376in}}%
\pgfpathlineto{\pgfqpoint{7.544684in}{2.362376in}}%
\pgfpathmoveto{\pgfqpoint{7.503018in}{2.320710in}}%
\pgfpathlineto{\pgfqpoint{7.503018in}{2.404043in}}%
\pgfusepath{stroke,fill}%
\end{pgfscope}%
\begin{pgfscope}%
\pgfpathrectangle{\pgfqpoint{5.562518in}{1.805660in}}{\pgfqpoint{4.376471in}{0.972632in}}%
\pgfusepath{clip}%
\pgfsetbuttcap%
\pgfsetroundjoin%
\definecolor{currentfill}{rgb}{1.000000,0.000000,0.000000}%
\pgfsetfillcolor{currentfill}%
\pgfsetlinewidth{2.007500pt}%
\definecolor{currentstroke}{rgb}{1.000000,0.000000,0.000000}%
\pgfsetstrokecolor{currentstroke}%
\pgfsetdash{}{0pt}%
\pgfpathmoveto{\pgfqpoint{8.233410in}{2.237163in}}%
\pgfpathlineto{\pgfqpoint{8.316743in}{2.237163in}}%
\pgfpathmoveto{\pgfqpoint{8.275077in}{2.195496in}}%
\pgfpathlineto{\pgfqpoint{8.275077in}{2.278830in}}%
\pgfusepath{stroke,fill}%
\end{pgfscope}%
\begin{pgfscope}%
\pgfpathrectangle{\pgfqpoint{5.562518in}{1.805660in}}{\pgfqpoint{4.376471in}{0.972632in}}%
\pgfusepath{clip}%
\pgfsetbuttcap%
\pgfsetroundjoin%
\definecolor{currentfill}{rgb}{1.000000,0.000000,0.000000}%
\pgfsetfillcolor{currentfill}%
\pgfsetlinewidth{2.007500pt}%
\definecolor{currentstroke}{rgb}{1.000000,0.000000,0.000000}%
\pgfsetstrokecolor{currentstroke}%
\pgfsetdash{}{0pt}%
\pgfpathmoveto{\pgfqpoint{7.908461in}{1.829914in}}%
\pgfpathlineto{\pgfqpoint{7.991794in}{1.829914in}}%
\pgfpathmoveto{\pgfqpoint{7.950127in}{1.788247in}}%
\pgfpathlineto{\pgfqpoint{7.950127in}{1.871581in}}%
\pgfusepath{stroke,fill}%
\end{pgfscope}%
\begin{pgfscope}%
\pgfpathrectangle{\pgfqpoint{5.562518in}{1.805660in}}{\pgfqpoint{4.376471in}{0.972632in}}%
\pgfusepath{clip}%
\pgfsetbuttcap%
\pgfsetroundjoin%
\definecolor{currentfill}{rgb}{1.000000,0.000000,0.000000}%
\pgfsetfillcolor{currentfill}%
\pgfsetlinewidth{2.007500pt}%
\definecolor{currentstroke}{rgb}{1.000000,0.000000,0.000000}%
\pgfsetstrokecolor{currentstroke}%
\pgfsetdash{}{0pt}%
\pgfpathmoveto{\pgfqpoint{7.415790in}{2.342233in}}%
\pgfpathlineto{\pgfqpoint{7.499123in}{2.342233in}}%
\pgfpathmoveto{\pgfqpoint{7.457456in}{2.300566in}}%
\pgfpathlineto{\pgfqpoint{7.457456in}{2.383899in}}%
\pgfusepath{stroke,fill}%
\end{pgfscope}%
\begin{pgfscope}%
\pgfpathrectangle{\pgfqpoint{5.562518in}{1.805660in}}{\pgfqpoint{4.376471in}{0.972632in}}%
\pgfusepath{clip}%
\pgfsetbuttcap%
\pgfsetroundjoin%
\definecolor{currentfill}{rgb}{1.000000,0.000000,0.000000}%
\pgfsetfillcolor{currentfill}%
\pgfsetlinewidth{2.007500pt}%
\definecolor{currentstroke}{rgb}{1.000000,0.000000,0.000000}%
\pgfsetstrokecolor{currentstroke}%
\pgfsetdash{}{0pt}%
\pgfpathmoveto{\pgfqpoint{8.538350in}{2.303304in}}%
\pgfpathlineto{\pgfqpoint{8.621683in}{2.303304in}}%
\pgfpathmoveto{\pgfqpoint{8.580017in}{2.261637in}}%
\pgfpathlineto{\pgfqpoint{8.580017in}{2.344971in}}%
\pgfusepath{stroke,fill}%
\end{pgfscope}%
\begin{pgfscope}%
\pgfpathrectangle{\pgfqpoint{5.562518in}{1.805660in}}{\pgfqpoint{4.376471in}{0.972632in}}%
\pgfusepath{clip}%
\pgfsetbuttcap%
\pgfsetroundjoin%
\definecolor{currentfill}{rgb}{1.000000,0.000000,0.000000}%
\pgfsetfillcolor{currentfill}%
\pgfsetlinewidth{2.007500pt}%
\definecolor{currentstroke}{rgb}{1.000000,0.000000,0.000000}%
\pgfsetstrokecolor{currentstroke}%
\pgfsetdash{}{0pt}%
\pgfpathmoveto{\pgfqpoint{6.884538in}{2.326118in}}%
\pgfpathlineto{\pgfqpoint{6.967871in}{2.326118in}}%
\pgfpathmoveto{\pgfqpoint{6.926204in}{2.284451in}}%
\pgfpathlineto{\pgfqpoint{6.926204in}{2.367784in}}%
\pgfusepath{stroke,fill}%
\end{pgfscope}%
\begin{pgfscope}%
\pgfpathrectangle{\pgfqpoint{5.562518in}{1.805660in}}{\pgfqpoint{4.376471in}{0.972632in}}%
\pgfusepath{clip}%
\pgfsetbuttcap%
\pgfsetroundjoin%
\definecolor{currentfill}{rgb}{1.000000,0.000000,0.000000}%
\pgfsetfillcolor{currentfill}%
\pgfsetlinewidth{2.007500pt}%
\definecolor{currentstroke}{rgb}{1.000000,0.000000,0.000000}%
\pgfsetstrokecolor{currentstroke}%
\pgfsetdash{}{0pt}%
\pgfpathmoveto{\pgfqpoint{7.418995in}{2.447433in}}%
\pgfpathlineto{\pgfqpoint{7.502328in}{2.447433in}}%
\pgfpathmoveto{\pgfqpoint{7.460662in}{2.405766in}}%
\pgfpathlineto{\pgfqpoint{7.460662in}{2.489099in}}%
\pgfusepath{stroke,fill}%
\end{pgfscope}%
\begin{pgfscope}%
\pgfpathrectangle{\pgfqpoint{5.562518in}{1.805660in}}{\pgfqpoint{4.376471in}{0.972632in}}%
\pgfusepath{clip}%
\pgfsetbuttcap%
\pgfsetroundjoin%
\definecolor{currentfill}{rgb}{1.000000,0.000000,0.000000}%
\pgfsetfillcolor{currentfill}%
\pgfsetlinewidth{2.007500pt}%
\definecolor{currentstroke}{rgb}{1.000000,0.000000,0.000000}%
\pgfsetstrokecolor{currentstroke}%
\pgfsetdash{}{0pt}%
\pgfpathmoveto{\pgfqpoint{7.678843in}{1.929166in}}%
\pgfpathlineto{\pgfqpoint{7.762176in}{1.929166in}}%
\pgfpathmoveto{\pgfqpoint{7.720509in}{1.887499in}}%
\pgfpathlineto{\pgfqpoint{7.720509in}{1.970833in}}%
\pgfusepath{stroke,fill}%
\end{pgfscope}%
\begin{pgfscope}%
\pgfpathrectangle{\pgfqpoint{5.562518in}{1.805660in}}{\pgfqpoint{4.376471in}{0.972632in}}%
\pgfusepath{clip}%
\pgfsetbuttcap%
\pgfsetroundjoin%
\definecolor{currentfill}{rgb}{1.000000,0.000000,0.000000}%
\pgfsetfillcolor{currentfill}%
\pgfsetlinewidth{2.007500pt}%
\definecolor{currentstroke}{rgb}{1.000000,0.000000,0.000000}%
\pgfsetstrokecolor{currentstroke}%
\pgfsetdash{}{0pt}%
\pgfpathmoveto{\pgfqpoint{7.992927in}{2.050365in}}%
\pgfpathlineto{\pgfqpoint{8.076260in}{2.050365in}}%
\pgfpathmoveto{\pgfqpoint{8.034593in}{2.008698in}}%
\pgfpathlineto{\pgfqpoint{8.034593in}{2.092031in}}%
\pgfusepath{stroke,fill}%
\end{pgfscope}%
\begin{pgfscope}%
\pgfpathrectangle{\pgfqpoint{5.562518in}{1.805660in}}{\pgfqpoint{4.376471in}{0.972632in}}%
\pgfusepath{clip}%
\pgfsetbuttcap%
\pgfsetroundjoin%
\definecolor{currentfill}{rgb}{1.000000,0.000000,0.000000}%
\pgfsetfillcolor{currentfill}%
\pgfsetlinewidth{2.007500pt}%
\definecolor{currentstroke}{rgb}{1.000000,0.000000,0.000000}%
\pgfsetstrokecolor{currentstroke}%
\pgfsetdash{}{0pt}%
\pgfpathmoveto{\pgfqpoint{9.145185in}{2.152224in}}%
\pgfpathlineto{\pgfqpoint{9.228518in}{2.152224in}}%
\pgfpathmoveto{\pgfqpoint{9.186851in}{2.110557in}}%
\pgfpathlineto{\pgfqpoint{9.186851in}{2.193891in}}%
\pgfusepath{stroke,fill}%
\end{pgfscope}%
\begin{pgfscope}%
\pgfpathrectangle{\pgfqpoint{5.562518in}{1.805660in}}{\pgfqpoint{4.376471in}{0.972632in}}%
\pgfusepath{clip}%
\pgfsetbuttcap%
\pgfsetroundjoin%
\definecolor{currentfill}{rgb}{1.000000,0.000000,0.000000}%
\pgfsetfillcolor{currentfill}%
\pgfsetlinewidth{2.007500pt}%
\definecolor{currentstroke}{rgb}{1.000000,0.000000,0.000000}%
\pgfsetstrokecolor{currentstroke}%
\pgfsetdash{}{0pt}%
\pgfpathmoveto{\pgfqpoint{7.095238in}{2.303374in}}%
\pgfpathlineto{\pgfqpoint{7.178572in}{2.303374in}}%
\pgfpathmoveto{\pgfqpoint{7.136905in}{2.261707in}}%
\pgfpathlineto{\pgfqpoint{7.136905in}{2.345040in}}%
\pgfusepath{stroke,fill}%
\end{pgfscope}%
\begin{pgfscope}%
\pgfpathrectangle{\pgfqpoint{5.562518in}{1.805660in}}{\pgfqpoint{4.376471in}{0.972632in}}%
\pgfusepath{clip}%
\pgfsetbuttcap%
\pgfsetroundjoin%
\definecolor{currentfill}{rgb}{1.000000,0.000000,0.000000}%
\pgfsetfillcolor{currentfill}%
\pgfsetlinewidth{2.007500pt}%
\definecolor{currentstroke}{rgb}{1.000000,0.000000,0.000000}%
\pgfsetstrokecolor{currentstroke}%
\pgfsetdash{}{0pt}%
\pgfpathmoveto{\pgfqpoint{8.196571in}{2.289608in}}%
\pgfpathlineto{\pgfqpoint{8.279904in}{2.289608in}}%
\pgfpathmoveto{\pgfqpoint{8.238237in}{2.247942in}}%
\pgfpathlineto{\pgfqpoint{8.238237in}{2.331275in}}%
\pgfusepath{stroke,fill}%
\end{pgfscope}%
\begin{pgfscope}%
\pgfpathrectangle{\pgfqpoint{5.562518in}{1.805660in}}{\pgfqpoint{4.376471in}{0.972632in}}%
\pgfusepath{clip}%
\pgfsetbuttcap%
\pgfsetroundjoin%
\definecolor{currentfill}{rgb}{1.000000,0.000000,0.000000}%
\pgfsetfillcolor{currentfill}%
\pgfsetlinewidth{2.007500pt}%
\definecolor{currentstroke}{rgb}{1.000000,0.000000,0.000000}%
\pgfsetstrokecolor{currentstroke}%
\pgfsetdash{}{0pt}%
\pgfpathmoveto{\pgfqpoint{8.470293in}{2.483577in}}%
\pgfpathlineto{\pgfqpoint{8.553626in}{2.483577in}}%
\pgfpathmoveto{\pgfqpoint{8.511960in}{2.441911in}}%
\pgfpathlineto{\pgfqpoint{8.511960in}{2.525244in}}%
\pgfusepath{stroke,fill}%
\end{pgfscope}%
\begin{pgfscope}%
\pgfpathrectangle{\pgfqpoint{5.562518in}{1.805660in}}{\pgfqpoint{4.376471in}{0.972632in}}%
\pgfusepath{clip}%
\pgfsetbuttcap%
\pgfsetroundjoin%
\definecolor{currentfill}{rgb}{1.000000,0.000000,0.000000}%
\pgfsetfillcolor{currentfill}%
\pgfsetlinewidth{2.007500pt}%
\definecolor{currentstroke}{rgb}{1.000000,0.000000,0.000000}%
\pgfsetstrokecolor{currentstroke}%
\pgfsetdash{}{0pt}%
\pgfpathmoveto{\pgfqpoint{6.558776in}{2.528010in}}%
\pgfpathlineto{\pgfqpoint{6.642110in}{2.528010in}}%
\pgfpathmoveto{\pgfqpoint{6.600443in}{2.486343in}}%
\pgfpathlineto{\pgfqpoint{6.600443in}{2.569677in}}%
\pgfusepath{stroke,fill}%
\end{pgfscope}%
\begin{pgfscope}%
\pgfpathrectangle{\pgfqpoint{5.562518in}{1.805660in}}{\pgfqpoint{4.376471in}{0.972632in}}%
\pgfusepath{clip}%
\pgfsetbuttcap%
\pgfsetroundjoin%
\definecolor{currentfill}{rgb}{0.000000,0.000000,0.000000}%
\pgfsetfillcolor{currentfill}%
\pgfsetlinewidth{1.003750pt}%
\definecolor{currentstroke}{rgb}{0.000000,0.000000,0.000000}%
\pgfsetstrokecolor{currentstroke}%
\pgfsetdash{}{0pt}%
\pgfsys@defobject{currentmarker}{\pgfqpoint{-0.020833in}{-0.020833in}}{\pgfqpoint{0.020833in}{0.020833in}}{%
\pgfpathmoveto{\pgfqpoint{0.000000in}{-0.020833in}}%
\pgfpathcurveto{\pgfqpoint{0.005525in}{-0.020833in}}{\pgfqpoint{0.010825in}{-0.018638in}}{\pgfqpoint{0.014731in}{-0.014731in}}%
\pgfpathcurveto{\pgfqpoint{0.018638in}{-0.010825in}}{\pgfqpoint{0.020833in}{-0.005525in}}{\pgfqpoint{0.020833in}{0.000000in}}%
\pgfpathcurveto{\pgfqpoint{0.020833in}{0.005525in}}{\pgfqpoint{0.018638in}{0.010825in}}{\pgfqpoint{0.014731in}{0.014731in}}%
\pgfpathcurveto{\pgfqpoint{0.010825in}{0.018638in}}{\pgfqpoint{0.005525in}{0.020833in}}{\pgfqpoint{0.000000in}{0.020833in}}%
\pgfpathcurveto{\pgfqpoint{-0.005525in}{0.020833in}}{\pgfqpoint{-0.010825in}{0.018638in}}{\pgfqpoint{-0.014731in}{0.014731in}}%
\pgfpathcurveto{\pgfqpoint{-0.018638in}{0.010825in}}{\pgfqpoint{-0.020833in}{0.005525in}}{\pgfqpoint{-0.020833in}{0.000000in}}%
\pgfpathcurveto{\pgfqpoint{-0.020833in}{-0.005525in}}{\pgfqpoint{-0.018638in}{-0.010825in}}{\pgfqpoint{-0.014731in}{-0.014731in}}%
\pgfpathcurveto{\pgfqpoint{-0.010825in}{-0.018638in}}{\pgfqpoint{-0.005525in}{-0.020833in}}{\pgfqpoint{0.000000in}{-0.020833in}}%
\pgfpathclose%
\pgfusepath{stroke,fill}%
}%
\begin{pgfscope}%
\pgfsys@transformshift{6.437812in}{2.600870in}%
\pgfsys@useobject{currentmarker}{}%
\end{pgfscope}%
\begin{pgfscope}%
\pgfsys@transformshift{6.455406in}{2.628847in}%
\pgfsys@useobject{currentmarker}{}%
\end{pgfscope}%
\begin{pgfscope}%
\pgfsys@transformshift{6.472999in}{2.667164in}%
\pgfsys@useobject{currentmarker}{}%
\end{pgfscope}%
\begin{pgfscope}%
\pgfsys@transformshift{6.490593in}{2.705202in}%
\pgfsys@useobject{currentmarker}{}%
\end{pgfscope}%
\begin{pgfscope}%
\pgfsys@transformshift{6.508187in}{2.530106in}%
\pgfsys@useobject{currentmarker}{}%
\end{pgfscope}%
\begin{pgfscope}%
\pgfsys@transformshift{6.525781in}{2.531863in}%
\pgfsys@useobject{currentmarker}{}%
\end{pgfscope}%
\begin{pgfscope}%
\pgfsys@transformshift{6.543375in}{2.410537in}%
\pgfsys@useobject{currentmarker}{}%
\end{pgfscope}%
\begin{pgfscope}%
\pgfsys@transformshift{6.560969in}{2.373524in}%
\pgfsys@useobject{currentmarker}{}%
\end{pgfscope}%
\begin{pgfscope}%
\pgfsys@transformshift{6.578563in}{2.549475in}%
\pgfsys@useobject{currentmarker}{}%
\end{pgfscope}%
\begin{pgfscope}%
\pgfsys@transformshift{6.596156in}{2.577543in}%
\pgfsys@useobject{currentmarker}{}%
\end{pgfscope}%
\begin{pgfscope}%
\pgfsys@transformshift{6.613750in}{2.406563in}%
\pgfsys@useobject{currentmarker}{}%
\end{pgfscope}%
\begin{pgfscope}%
\pgfsys@transformshift{6.631344in}{2.490190in}%
\pgfsys@useobject{currentmarker}{}%
\end{pgfscope}%
\begin{pgfscope}%
\pgfsys@transformshift{6.648938in}{2.400930in}%
\pgfsys@useobject{currentmarker}{}%
\end{pgfscope}%
\begin{pgfscope}%
\pgfsys@transformshift{6.666532in}{2.275992in}%
\pgfsys@useobject{currentmarker}{}%
\end{pgfscope}%
\begin{pgfscope}%
\pgfsys@transformshift{6.684126in}{2.356478in}%
\pgfsys@useobject{currentmarker}{}%
\end{pgfscope}%
\begin{pgfscope}%
\pgfsys@transformshift{6.701720in}{2.455794in}%
\pgfsys@useobject{currentmarker}{}%
\end{pgfscope}%
\begin{pgfscope}%
\pgfsys@transformshift{6.719313in}{2.278164in}%
\pgfsys@useobject{currentmarker}{}%
\end{pgfscope}%
\begin{pgfscope}%
\pgfsys@transformshift{6.736907in}{2.423983in}%
\pgfsys@useobject{currentmarker}{}%
\end{pgfscope}%
\begin{pgfscope}%
\pgfsys@transformshift{6.754501in}{1.985638in}%
\pgfsys@useobject{currentmarker}{}%
\end{pgfscope}%
\begin{pgfscope}%
\pgfsys@transformshift{6.772095in}{2.321942in}%
\pgfsys@useobject{currentmarker}{}%
\end{pgfscope}%
\begin{pgfscope}%
\pgfsys@transformshift{6.789689in}{2.237172in}%
\pgfsys@useobject{currentmarker}{}%
\end{pgfscope}%
\begin{pgfscope}%
\pgfsys@transformshift{6.807283in}{2.189848in}%
\pgfsys@useobject{currentmarker}{}%
\end{pgfscope}%
\begin{pgfscope}%
\pgfsys@transformshift{6.824876in}{2.223354in}%
\pgfsys@useobject{currentmarker}{}%
\end{pgfscope}%
\begin{pgfscope}%
\pgfsys@transformshift{6.842470in}{2.008723in}%
\pgfsys@useobject{currentmarker}{}%
\end{pgfscope}%
\begin{pgfscope}%
\pgfsys@transformshift{6.860064in}{2.185975in}%
\pgfsys@useobject{currentmarker}{}%
\end{pgfscope}%
\begin{pgfscope}%
\pgfsys@transformshift{6.877658in}{2.244604in}%
\pgfsys@useobject{currentmarker}{}%
\end{pgfscope}%
\begin{pgfscope}%
\pgfsys@transformshift{6.895252in}{2.360340in}%
\pgfsys@useobject{currentmarker}{}%
\end{pgfscope}%
\begin{pgfscope}%
\pgfsys@transformshift{6.912846in}{2.162187in}%
\pgfsys@useobject{currentmarker}{}%
\end{pgfscope}%
\begin{pgfscope}%
\pgfsys@transformshift{6.930440in}{2.138680in}%
\pgfsys@useobject{currentmarker}{}%
\end{pgfscope}%
\begin{pgfscope}%
\pgfsys@transformshift{6.948033in}{2.177341in}%
\pgfsys@useobject{currentmarker}{}%
\end{pgfscope}%
\begin{pgfscope}%
\pgfsys@transformshift{6.965627in}{2.330058in}%
\pgfsys@useobject{currentmarker}{}%
\end{pgfscope}%
\begin{pgfscope}%
\pgfsys@transformshift{6.983221in}{2.281159in}%
\pgfsys@useobject{currentmarker}{}%
\end{pgfscope}%
\begin{pgfscope}%
\pgfsys@transformshift{7.000815in}{2.205955in}%
\pgfsys@useobject{currentmarker}{}%
\end{pgfscope}%
\begin{pgfscope}%
\pgfsys@transformshift{7.018409in}{2.324470in}%
\pgfsys@useobject{currentmarker}{}%
\end{pgfscope}%
\begin{pgfscope}%
\pgfsys@transformshift{7.036003in}{2.296023in}%
\pgfsys@useobject{currentmarker}{}%
\end{pgfscope}%
\begin{pgfscope}%
\pgfsys@transformshift{7.053597in}{2.398731in}%
\pgfsys@useobject{currentmarker}{}%
\end{pgfscope}%
\begin{pgfscope}%
\pgfsys@transformshift{7.071190in}{2.244354in}%
\pgfsys@useobject{currentmarker}{}%
\end{pgfscope}%
\begin{pgfscope}%
\pgfsys@transformshift{7.088784in}{2.297461in}%
\pgfsys@useobject{currentmarker}{}%
\end{pgfscope}%
\begin{pgfscope}%
\pgfsys@transformshift{7.106378in}{2.306184in}%
\pgfsys@useobject{currentmarker}{}%
\end{pgfscope}%
\begin{pgfscope}%
\pgfsys@transformshift{7.123972in}{2.212760in}%
\pgfsys@useobject{currentmarker}{}%
\end{pgfscope}%
\begin{pgfscope}%
\pgfsys@transformshift{7.141566in}{2.405835in}%
\pgfsys@useobject{currentmarker}{}%
\end{pgfscope}%
\begin{pgfscope}%
\pgfsys@transformshift{7.159160in}{2.416549in}%
\pgfsys@useobject{currentmarker}{}%
\end{pgfscope}%
\begin{pgfscope}%
\pgfsys@transformshift{7.176754in}{2.404164in}%
\pgfsys@useobject{currentmarker}{}%
\end{pgfscope}%
\begin{pgfscope}%
\pgfsys@transformshift{7.194347in}{2.392518in}%
\pgfsys@useobject{currentmarker}{}%
\end{pgfscope}%
\begin{pgfscope}%
\pgfsys@transformshift{7.211941in}{2.284444in}%
\pgfsys@useobject{currentmarker}{}%
\end{pgfscope}%
\begin{pgfscope}%
\pgfsys@transformshift{7.229535in}{2.395540in}%
\pgfsys@useobject{currentmarker}{}%
\end{pgfscope}%
\begin{pgfscope}%
\pgfsys@transformshift{7.247129in}{2.412357in}%
\pgfsys@useobject{currentmarker}{}%
\end{pgfscope}%
\begin{pgfscope}%
\pgfsys@transformshift{7.264723in}{2.373189in}%
\pgfsys@useobject{currentmarker}{}%
\end{pgfscope}%
\begin{pgfscope}%
\pgfsys@transformshift{7.282317in}{2.443878in}%
\pgfsys@useobject{currentmarker}{}%
\end{pgfscope}%
\begin{pgfscope}%
\pgfsys@transformshift{7.299910in}{2.505155in}%
\pgfsys@useobject{currentmarker}{}%
\end{pgfscope}%
\begin{pgfscope}%
\pgfsys@transformshift{7.317504in}{2.657488in}%
\pgfsys@useobject{currentmarker}{}%
\end{pgfscope}%
\begin{pgfscope}%
\pgfsys@transformshift{7.335098in}{2.484352in}%
\pgfsys@useobject{currentmarker}{}%
\end{pgfscope}%
\begin{pgfscope}%
\pgfsys@transformshift{7.352692in}{2.491098in}%
\pgfsys@useobject{currentmarker}{}%
\end{pgfscope}%
\begin{pgfscope}%
\pgfsys@transformshift{7.370286in}{2.453842in}%
\pgfsys@useobject{currentmarker}{}%
\end{pgfscope}%
\begin{pgfscope}%
\pgfsys@transformshift{7.387880in}{2.261402in}%
\pgfsys@useobject{currentmarker}{}%
\end{pgfscope}%
\begin{pgfscope}%
\pgfsys@transformshift{7.405474in}{2.445594in}%
\pgfsys@useobject{currentmarker}{}%
\end{pgfscope}%
\begin{pgfscope}%
\pgfsys@transformshift{7.423067in}{2.444953in}%
\pgfsys@useobject{currentmarker}{}%
\end{pgfscope}%
\begin{pgfscope}%
\pgfsys@transformshift{7.440661in}{2.677142in}%
\pgfsys@useobject{currentmarker}{}%
\end{pgfscope}%
\begin{pgfscope}%
\pgfsys@transformshift{7.458255in}{2.395046in}%
\pgfsys@useobject{currentmarker}{}%
\end{pgfscope}%
\begin{pgfscope}%
\pgfsys@transformshift{7.475849in}{2.430374in}%
\pgfsys@useobject{currentmarker}{}%
\end{pgfscope}%
\begin{pgfscope}%
\pgfsys@transformshift{7.493443in}{2.380038in}%
\pgfsys@useobject{currentmarker}{}%
\end{pgfscope}%
\begin{pgfscope}%
\pgfsys@transformshift{7.511037in}{2.247456in}%
\pgfsys@useobject{currentmarker}{}%
\end{pgfscope}%
\begin{pgfscope}%
\pgfsys@transformshift{7.528631in}{2.462676in}%
\pgfsys@useobject{currentmarker}{}%
\end{pgfscope}%
\begin{pgfscope}%
\pgfsys@transformshift{7.546224in}{2.402983in}%
\pgfsys@useobject{currentmarker}{}%
\end{pgfscope}%
\begin{pgfscope}%
\pgfsys@transformshift{7.563818in}{2.385908in}%
\pgfsys@useobject{currentmarker}{}%
\end{pgfscope}%
\begin{pgfscope}%
\pgfsys@transformshift{7.581412in}{2.191829in}%
\pgfsys@useobject{currentmarker}{}%
\end{pgfscope}%
\begin{pgfscope}%
\pgfsys@transformshift{7.599006in}{2.403717in}%
\pgfsys@useobject{currentmarker}{}%
\end{pgfscope}%
\begin{pgfscope}%
\pgfsys@transformshift{7.616600in}{2.096815in}%
\pgfsys@useobject{currentmarker}{}%
\end{pgfscope}%
\begin{pgfscope}%
\pgfsys@transformshift{7.634194in}{2.275382in}%
\pgfsys@useobject{currentmarker}{}%
\end{pgfscope}%
\begin{pgfscope}%
\pgfsys@transformshift{7.651788in}{2.414965in}%
\pgfsys@useobject{currentmarker}{}%
\end{pgfscope}%
\begin{pgfscope}%
\pgfsys@transformshift{7.669381in}{2.070030in}%
\pgfsys@useobject{currentmarker}{}%
\end{pgfscope}%
\begin{pgfscope}%
\pgfsys@transformshift{7.686975in}{2.090804in}%
\pgfsys@useobject{currentmarker}{}%
\end{pgfscope}%
\begin{pgfscope}%
\pgfsys@transformshift{7.704569in}{2.136709in}%
\pgfsys@useobject{currentmarker}{}%
\end{pgfscope}%
\begin{pgfscope}%
\pgfsys@transformshift{7.722163in}{2.054872in}%
\pgfsys@useobject{currentmarker}{}%
\end{pgfscope}%
\begin{pgfscope}%
\pgfsys@transformshift{7.739757in}{1.929068in}%
\pgfsys@useobject{currentmarker}{}%
\end{pgfscope}%
\begin{pgfscope}%
\pgfsys@transformshift{7.757351in}{2.074614in}%
\pgfsys@useobject{currentmarker}{}%
\end{pgfscope}%
\begin{pgfscope}%
\pgfsys@transformshift{7.774944in}{1.942899in}%
\pgfsys@useobject{currentmarker}{}%
\end{pgfscope}%
\begin{pgfscope}%
\pgfsys@transformshift{7.792538in}{2.082889in}%
\pgfsys@useobject{currentmarker}{}%
\end{pgfscope}%
\begin{pgfscope}%
\pgfsys@transformshift{7.810132in}{1.927792in}%
\pgfsys@useobject{currentmarker}{}%
\end{pgfscope}%
\begin{pgfscope}%
\pgfsys@transformshift{7.827726in}{2.165797in}%
\pgfsys@useobject{currentmarker}{}%
\end{pgfscope}%
\begin{pgfscope}%
\pgfsys@transformshift{7.845320in}{1.919114in}%
\pgfsys@useobject{currentmarker}{}%
\end{pgfscope}%
\begin{pgfscope}%
\pgfsys@transformshift{7.862914in}{1.957519in}%
\pgfsys@useobject{currentmarker}{}%
\end{pgfscope}%
\begin{pgfscope}%
\pgfsys@transformshift{7.880508in}{2.066294in}%
\pgfsys@useobject{currentmarker}{}%
\end{pgfscope}%
\begin{pgfscope}%
\pgfsys@transformshift{7.898101in}{1.854979in}%
\pgfsys@useobject{currentmarker}{}%
\end{pgfscope}%
\begin{pgfscope}%
\pgfsys@transformshift{7.915695in}{2.000662in}%
\pgfsys@useobject{currentmarker}{}%
\end{pgfscope}%
\begin{pgfscope}%
\pgfsys@transformshift{7.933289in}{2.110110in}%
\pgfsys@useobject{currentmarker}{}%
\end{pgfscope}%
\begin{pgfscope}%
\pgfsys@transformshift{7.950883in}{1.816985in}%
\pgfsys@useobject{currentmarker}{}%
\end{pgfscope}%
\begin{pgfscope}%
\pgfsys@transformshift{7.968477in}{2.002807in}%
\pgfsys@useobject{currentmarker}{}%
\end{pgfscope}%
\begin{pgfscope}%
\pgfsys@transformshift{7.986071in}{2.016710in}%
\pgfsys@useobject{currentmarker}{}%
\end{pgfscope}%
\begin{pgfscope}%
\pgfsys@transformshift{8.003665in}{2.077824in}%
\pgfsys@useobject{currentmarker}{}%
\end{pgfscope}%
\begin{pgfscope}%
\pgfsys@transformshift{8.021258in}{1.883384in}%
\pgfsys@useobject{currentmarker}{}%
\end{pgfscope}%
\begin{pgfscope}%
\pgfsys@transformshift{8.038852in}{1.886768in}%
\pgfsys@useobject{currentmarker}{}%
\end{pgfscope}%
\begin{pgfscope}%
\pgfsys@transformshift{8.056446in}{2.086901in}%
\pgfsys@useobject{currentmarker}{}%
\end{pgfscope}%
\begin{pgfscope}%
\pgfsys@transformshift{8.074040in}{2.079059in}%
\pgfsys@useobject{currentmarker}{}%
\end{pgfscope}%
\begin{pgfscope}%
\pgfsys@transformshift{8.091634in}{2.090624in}%
\pgfsys@useobject{currentmarker}{}%
\end{pgfscope}%
\begin{pgfscope}%
\pgfsys@transformshift{8.109228in}{2.117776in}%
\pgfsys@useobject{currentmarker}{}%
\end{pgfscope}%
\begin{pgfscope}%
\pgfsys@transformshift{8.126822in}{2.032184in}%
\pgfsys@useobject{currentmarker}{}%
\end{pgfscope}%
\begin{pgfscope}%
\pgfsys@transformshift{8.144415in}{2.143805in}%
\pgfsys@useobject{currentmarker}{}%
\end{pgfscope}%
\begin{pgfscope}%
\pgfsys@transformshift{8.162009in}{2.169750in}%
\pgfsys@useobject{currentmarker}{}%
\end{pgfscope}%
\begin{pgfscope}%
\pgfsys@transformshift{8.179603in}{2.087854in}%
\pgfsys@useobject{currentmarker}{}%
\end{pgfscope}%
\begin{pgfscope}%
\pgfsys@transformshift{8.197197in}{2.369617in}%
\pgfsys@useobject{currentmarker}{}%
\end{pgfscope}%
\begin{pgfscope}%
\pgfsys@transformshift{8.214791in}{2.248927in}%
\pgfsys@useobject{currentmarker}{}%
\end{pgfscope}%
\begin{pgfscope}%
\pgfsys@transformshift{8.232385in}{2.100333in}%
\pgfsys@useobject{currentmarker}{}%
\end{pgfscope}%
\begin{pgfscope}%
\pgfsys@transformshift{8.249978in}{2.307234in}%
\pgfsys@useobject{currentmarker}{}%
\end{pgfscope}%
\begin{pgfscope}%
\pgfsys@transformshift{8.267572in}{2.161025in}%
\pgfsys@useobject{currentmarker}{}%
\end{pgfscope}%
\begin{pgfscope}%
\pgfsys@transformshift{8.285166in}{2.357769in}%
\pgfsys@useobject{currentmarker}{}%
\end{pgfscope}%
\begin{pgfscope}%
\pgfsys@transformshift{8.302760in}{2.412666in}%
\pgfsys@useobject{currentmarker}{}%
\end{pgfscope}%
\begin{pgfscope}%
\pgfsys@transformshift{8.320354in}{2.228230in}%
\pgfsys@useobject{currentmarker}{}%
\end{pgfscope}%
\begin{pgfscope}%
\pgfsys@transformshift{8.337948in}{2.423762in}%
\pgfsys@useobject{currentmarker}{}%
\end{pgfscope}%
\begin{pgfscope}%
\pgfsys@transformshift{8.355542in}{2.381297in}%
\pgfsys@useobject{currentmarker}{}%
\end{pgfscope}%
\begin{pgfscope}%
\pgfsys@transformshift{8.373135in}{2.434494in}%
\pgfsys@useobject{currentmarker}{}%
\end{pgfscope}%
\begin{pgfscope}%
\pgfsys@transformshift{8.390729in}{2.553415in}%
\pgfsys@useobject{currentmarker}{}%
\end{pgfscope}%
\begin{pgfscope}%
\pgfsys@transformshift{8.408323in}{2.344623in}%
\pgfsys@useobject{currentmarker}{}%
\end{pgfscope}%
\begin{pgfscope}%
\pgfsys@transformshift{8.425917in}{2.299500in}%
\pgfsys@useobject{currentmarker}{}%
\end{pgfscope}%
\begin{pgfscope}%
\pgfsys@transformshift{8.443511in}{2.290209in}%
\pgfsys@useobject{currentmarker}{}%
\end{pgfscope}%
\begin{pgfscope}%
\pgfsys@transformshift{8.461105in}{2.300191in}%
\pgfsys@useobject{currentmarker}{}%
\end{pgfscope}%
\begin{pgfscope}%
\pgfsys@transformshift{8.478699in}{2.375586in}%
\pgfsys@useobject{currentmarker}{}%
\end{pgfscope}%
\begin{pgfscope}%
\pgfsys@transformshift{8.496292in}{2.416558in}%
\pgfsys@useobject{currentmarker}{}%
\end{pgfscope}%
\begin{pgfscope}%
\pgfsys@transformshift{8.513886in}{2.406697in}%
\pgfsys@useobject{currentmarker}{}%
\end{pgfscope}%
\begin{pgfscope}%
\pgfsys@transformshift{8.531480in}{2.457261in}%
\pgfsys@useobject{currentmarker}{}%
\end{pgfscope}%
\begin{pgfscope}%
\pgfsys@transformshift{8.549074in}{2.367752in}%
\pgfsys@useobject{currentmarker}{}%
\end{pgfscope}%
\begin{pgfscope}%
\pgfsys@transformshift{8.566668in}{2.504959in}%
\pgfsys@useobject{currentmarker}{}%
\end{pgfscope}%
\begin{pgfscope}%
\pgfsys@transformshift{8.584262in}{2.320519in}%
\pgfsys@useobject{currentmarker}{}%
\end{pgfscope}%
\begin{pgfscope}%
\pgfsys@transformshift{8.601855in}{2.611074in}%
\pgfsys@useobject{currentmarker}{}%
\end{pgfscope}%
\begin{pgfscope}%
\pgfsys@transformshift{8.619449in}{2.385658in}%
\pgfsys@useobject{currentmarker}{}%
\end{pgfscope}%
\begin{pgfscope}%
\pgfsys@transformshift{8.637043in}{2.221011in}%
\pgfsys@useobject{currentmarker}{}%
\end{pgfscope}%
\begin{pgfscope}%
\pgfsys@transformshift{8.654637in}{2.183906in}%
\pgfsys@useobject{currentmarker}{}%
\end{pgfscope}%
\begin{pgfscope}%
\pgfsys@transformshift{8.672231in}{2.324977in}%
\pgfsys@useobject{currentmarker}{}%
\end{pgfscope}%
\begin{pgfscope}%
\pgfsys@transformshift{8.689825in}{2.236472in}%
\pgfsys@useobject{currentmarker}{}%
\end{pgfscope}%
\begin{pgfscope}%
\pgfsys@transformshift{8.707419in}{2.313992in}%
\pgfsys@useobject{currentmarker}{}%
\end{pgfscope}%
\begin{pgfscope}%
\pgfsys@transformshift{8.725012in}{2.271861in}%
\pgfsys@useobject{currentmarker}{}%
\end{pgfscope}%
\begin{pgfscope}%
\pgfsys@transformshift{8.742606in}{2.198725in}%
\pgfsys@useobject{currentmarker}{}%
\end{pgfscope}%
\begin{pgfscope}%
\pgfsys@transformshift{8.760200in}{2.102630in}%
\pgfsys@useobject{currentmarker}{}%
\end{pgfscope}%
\begin{pgfscope}%
\pgfsys@transformshift{8.777794in}{2.017601in}%
\pgfsys@useobject{currentmarker}{}%
\end{pgfscope}%
\begin{pgfscope}%
\pgfsys@transformshift{8.795388in}{2.109035in}%
\pgfsys@useobject{currentmarker}{}%
\end{pgfscope}%
\begin{pgfscope}%
\pgfsys@transformshift{8.812982in}{2.224972in}%
\pgfsys@useobject{currentmarker}{}%
\end{pgfscope}%
\begin{pgfscope}%
\pgfsys@transformshift{8.830576in}{2.144755in}%
\pgfsys@useobject{currentmarker}{}%
\end{pgfscope}%
\begin{pgfscope}%
\pgfsys@transformshift{8.848169in}{1.982819in}%
\pgfsys@useobject{currentmarker}{}%
\end{pgfscope}%
\begin{pgfscope}%
\pgfsys@transformshift{8.865763in}{2.113829in}%
\pgfsys@useobject{currentmarker}{}%
\end{pgfscope}%
\begin{pgfscope}%
\pgfsys@transformshift{8.883357in}{2.124014in}%
\pgfsys@useobject{currentmarker}{}%
\end{pgfscope}%
\begin{pgfscope}%
\pgfsys@transformshift{8.900951in}{1.985708in}%
\pgfsys@useobject{currentmarker}{}%
\end{pgfscope}%
\begin{pgfscope}%
\pgfsys@transformshift{8.918545in}{2.082832in}%
\pgfsys@useobject{currentmarker}{}%
\end{pgfscope}%
\begin{pgfscope}%
\pgfsys@transformshift{8.936139in}{2.066990in}%
\pgfsys@useobject{currentmarker}{}%
\end{pgfscope}%
\begin{pgfscope}%
\pgfsys@transformshift{8.953733in}{1.941058in}%
\pgfsys@useobject{currentmarker}{}%
\end{pgfscope}%
\begin{pgfscope}%
\pgfsys@transformshift{8.971326in}{2.090884in}%
\pgfsys@useobject{currentmarker}{}%
\end{pgfscope}%
\begin{pgfscope}%
\pgfsys@transformshift{8.988920in}{2.111295in}%
\pgfsys@useobject{currentmarker}{}%
\end{pgfscope}%
\begin{pgfscope}%
\pgfsys@transformshift{9.006514in}{2.166160in}%
\pgfsys@useobject{currentmarker}{}%
\end{pgfscope}%
\begin{pgfscope}%
\pgfsys@transformshift{9.024108in}{2.167273in}%
\pgfsys@useobject{currentmarker}{}%
\end{pgfscope}%
\begin{pgfscope}%
\pgfsys@transformshift{9.041702in}{1.927126in}%
\pgfsys@useobject{currentmarker}{}%
\end{pgfscope}%
\begin{pgfscope}%
\pgfsys@transformshift{9.059296in}{1.979989in}%
\pgfsys@useobject{currentmarker}{}%
\end{pgfscope}%
\begin{pgfscope}%
\pgfsys@transformshift{9.076889in}{2.137543in}%
\pgfsys@useobject{currentmarker}{}%
\end{pgfscope}%
\begin{pgfscope}%
\pgfsys@transformshift{9.094483in}{2.149767in}%
\pgfsys@useobject{currentmarker}{}%
\end{pgfscope}%
\begin{pgfscope}%
\pgfsys@transformshift{9.112077in}{2.164156in}%
\pgfsys@useobject{currentmarker}{}%
\end{pgfscope}%
\begin{pgfscope}%
\pgfsys@transformshift{9.129671in}{2.518387in}%
\pgfsys@useobject{currentmarker}{}%
\end{pgfscope}%
\begin{pgfscope}%
\pgfsys@transformshift{9.147265in}{2.203648in}%
\pgfsys@useobject{currentmarker}{}%
\end{pgfscope}%
\begin{pgfscope}%
\pgfsys@transformshift{9.164859in}{2.280178in}%
\pgfsys@useobject{currentmarker}{}%
\end{pgfscope}%
\begin{pgfscope}%
\pgfsys@transformshift{9.182453in}{2.282511in}%
\pgfsys@useobject{currentmarker}{}%
\end{pgfscope}%
\begin{pgfscope}%
\pgfsys@transformshift{9.200046in}{2.273824in}%
\pgfsys@useobject{currentmarker}{}%
\end{pgfscope}%
\begin{pgfscope}%
\pgfsys@transformshift{9.217640in}{2.198928in}%
\pgfsys@useobject{currentmarker}{}%
\end{pgfscope}%
\begin{pgfscope}%
\pgfsys@transformshift{9.235234in}{2.331694in}%
\pgfsys@useobject{currentmarker}{}%
\end{pgfscope}%
\begin{pgfscope}%
\pgfsys@transformshift{9.252828in}{2.201119in}%
\pgfsys@useobject{currentmarker}{}%
\end{pgfscope}%
\begin{pgfscope}%
\pgfsys@transformshift{9.270422in}{2.280539in}%
\pgfsys@useobject{currentmarker}{}%
\end{pgfscope}%
\begin{pgfscope}%
\pgfsys@transformshift{9.288016in}{2.280762in}%
\pgfsys@useobject{currentmarker}{}%
\end{pgfscope}%
\begin{pgfscope}%
\pgfsys@transformshift{9.305610in}{2.363723in}%
\pgfsys@useobject{currentmarker}{}%
\end{pgfscope}%
\begin{pgfscope}%
\pgfsys@transformshift{9.323203in}{2.615309in}%
\pgfsys@useobject{currentmarker}{}%
\end{pgfscope}%
\begin{pgfscope}%
\pgfsys@transformshift{9.340797in}{2.216662in}%
\pgfsys@useobject{currentmarker}{}%
\end{pgfscope}%
\begin{pgfscope}%
\pgfsys@transformshift{9.358391in}{2.499901in}%
\pgfsys@useobject{currentmarker}{}%
\end{pgfscope}%
\begin{pgfscope}%
\pgfsys@transformshift{9.375985in}{2.290789in}%
\pgfsys@useobject{currentmarker}{}%
\end{pgfscope}%
\begin{pgfscope}%
\pgfsys@transformshift{9.393579in}{2.429279in}%
\pgfsys@useobject{currentmarker}{}%
\end{pgfscope}%
\begin{pgfscope}%
\pgfsys@transformshift{9.411173in}{2.609257in}%
\pgfsys@useobject{currentmarker}{}%
\end{pgfscope}%
\begin{pgfscope}%
\pgfsys@transformshift{9.428767in}{2.526040in}%
\pgfsys@useobject{currentmarker}{}%
\end{pgfscope}%
\begin{pgfscope}%
\pgfsys@transformshift{9.446360in}{2.429544in}%
\pgfsys@useobject{currentmarker}{}%
\end{pgfscope}%
\begin{pgfscope}%
\pgfsys@transformshift{9.463954in}{2.483951in}%
\pgfsys@useobject{currentmarker}{}%
\end{pgfscope}%
\begin{pgfscope}%
\pgfsys@transformshift{9.481548in}{2.641320in}%
\pgfsys@useobject{currentmarker}{}%
\end{pgfscope}%
\begin{pgfscope}%
\pgfsys@transformshift{9.499142in}{2.512767in}%
\pgfsys@useobject{currentmarker}{}%
\end{pgfscope}%
\begin{pgfscope}%
\pgfsys@transformshift{9.516736in}{2.621168in}%
\pgfsys@useobject{currentmarker}{}%
\end{pgfscope}%
\begin{pgfscope}%
\pgfsys@transformshift{9.534330in}{2.614435in}%
\pgfsys@useobject{currentmarker}{}%
\end{pgfscope}%
\begin{pgfscope}%
\pgfsys@transformshift{9.551923in}{2.552446in}%
\pgfsys@useobject{currentmarker}{}%
\end{pgfscope}%
\begin{pgfscope}%
\pgfsys@transformshift{9.569517in}{2.842366in}%
\pgfsys@useobject{currentmarker}{}%
\end{pgfscope}%
\begin{pgfscope}%
\pgfsys@transformshift{9.587111in}{2.694101in}%
\pgfsys@useobject{currentmarker}{}%
\end{pgfscope}%
\begin{pgfscope}%
\pgfsys@transformshift{9.604705in}{2.427477in}%
\pgfsys@useobject{currentmarker}{}%
\end{pgfscope}%
\begin{pgfscope}%
\pgfsys@transformshift{9.622299in}{2.652421in}%
\pgfsys@useobject{currentmarker}{}%
\end{pgfscope}%
\begin{pgfscope}%
\pgfsys@transformshift{9.639893in}{2.565507in}%
\pgfsys@useobject{currentmarker}{}%
\end{pgfscope}%
\begin{pgfscope}%
\pgfsys@transformshift{9.657487in}{2.716177in}%
\pgfsys@useobject{currentmarker}{}%
\end{pgfscope}%
\begin{pgfscope}%
\pgfsys@transformshift{9.675080in}{2.545099in}%
\pgfsys@useobject{currentmarker}{}%
\end{pgfscope}%
\begin{pgfscope}%
\pgfsys@transformshift{9.692674in}{2.607790in}%
\pgfsys@useobject{currentmarker}{}%
\end{pgfscope}%
\begin{pgfscope}%
\pgfsys@transformshift{9.710268in}{2.663170in}%
\pgfsys@useobject{currentmarker}{}%
\end{pgfscope}%
\begin{pgfscope}%
\pgfsys@transformshift{9.727862in}{2.691029in}%
\pgfsys@useobject{currentmarker}{}%
\end{pgfscope}%
\begin{pgfscope}%
\pgfsys@transformshift{9.745456in}{2.471888in}%
\pgfsys@useobject{currentmarker}{}%
\end{pgfscope}%
\begin{pgfscope}%
\pgfsys@transformshift{9.763050in}{2.548838in}%
\pgfsys@useobject{currentmarker}{}%
\end{pgfscope}%
\begin{pgfscope}%
\pgfsys@transformshift{9.780644in}{2.523069in}%
\pgfsys@useobject{currentmarker}{}%
\end{pgfscope}%
\begin{pgfscope}%
\pgfsys@transformshift{9.798237in}{2.492877in}%
\pgfsys@useobject{currentmarker}{}%
\end{pgfscope}%
\begin{pgfscope}%
\pgfsys@transformshift{9.815831in}{2.725437in}%
\pgfsys@useobject{currentmarker}{}%
\end{pgfscope}%
\begin{pgfscope}%
\pgfsys@transformshift{9.833425in}{2.574920in}%
\pgfsys@useobject{currentmarker}{}%
\end{pgfscope}%
\begin{pgfscope}%
\pgfsys@transformshift{9.851019in}{2.393485in}%
\pgfsys@useobject{currentmarker}{}%
\end{pgfscope}%
\begin{pgfscope}%
\pgfsys@transformshift{9.868613in}{2.601802in}%
\pgfsys@useobject{currentmarker}{}%
\end{pgfscope}%
\begin{pgfscope}%
\pgfsys@transformshift{9.886207in}{2.711825in}%
\pgfsys@useobject{currentmarker}{}%
\end{pgfscope}%
\begin{pgfscope}%
\pgfsys@transformshift{9.903801in}{2.590066in}%
\pgfsys@useobject{currentmarker}{}%
\end{pgfscope}%
\begin{pgfscope}%
\pgfsys@transformshift{9.921394in}{2.320998in}%
\pgfsys@useobject{currentmarker}{}%
\end{pgfscope}%
\begin{pgfscope}%
\pgfsys@transformshift{9.938988in}{2.416363in}%
\pgfsys@useobject{currentmarker}{}%
\end{pgfscope}%
\end{pgfscope}%
\begin{pgfscope}%
\pgfsetbuttcap%
\pgfsetroundjoin%
\definecolor{currentfill}{rgb}{0.000000,0.000000,0.000000}%
\pgfsetfillcolor{currentfill}%
\pgfsetlinewidth{0.803000pt}%
\definecolor{currentstroke}{rgb}{0.000000,0.000000,0.000000}%
\pgfsetstrokecolor{currentstroke}%
\pgfsetdash{}{0pt}%
\pgfsys@defobject{currentmarker}{\pgfqpoint{0.000000in}{-0.048611in}}{\pgfqpoint{0.000000in}{0.000000in}}{%
\pgfpathmoveto{\pgfqpoint{0.000000in}{0.000000in}}%
\pgfpathlineto{\pgfqpoint{0.000000in}{-0.048611in}}%
\pgfusepath{stroke,fill}%
}%
\begin{pgfscope}%
\pgfsys@transformshift{5.562518in}{1.805660in}%
\pgfsys@useobject{currentmarker}{}%
\end{pgfscope}%
\end{pgfscope}%
\begin{pgfscope}%
\pgfsetbuttcap%
\pgfsetroundjoin%
\definecolor{currentfill}{rgb}{0.000000,0.000000,0.000000}%
\pgfsetfillcolor{currentfill}%
\pgfsetlinewidth{0.803000pt}%
\definecolor{currentstroke}{rgb}{0.000000,0.000000,0.000000}%
\pgfsetstrokecolor{currentstroke}%
\pgfsetdash{}{0pt}%
\pgfsys@defobject{currentmarker}{\pgfqpoint{0.000000in}{-0.048611in}}{\pgfqpoint{0.000000in}{0.000000in}}{%
\pgfpathmoveto{\pgfqpoint{0.000000in}{0.000000in}}%
\pgfpathlineto{\pgfqpoint{0.000000in}{-0.048611in}}%
\pgfusepath{stroke,fill}%
}%
\begin{pgfscope}%
\pgfsys@transformshift{6.437812in}{1.805660in}%
\pgfsys@useobject{currentmarker}{}%
\end{pgfscope}%
\end{pgfscope}%
\begin{pgfscope}%
\pgfsetbuttcap%
\pgfsetroundjoin%
\definecolor{currentfill}{rgb}{0.000000,0.000000,0.000000}%
\pgfsetfillcolor{currentfill}%
\pgfsetlinewidth{0.803000pt}%
\definecolor{currentstroke}{rgb}{0.000000,0.000000,0.000000}%
\pgfsetstrokecolor{currentstroke}%
\pgfsetdash{}{0pt}%
\pgfsys@defobject{currentmarker}{\pgfqpoint{0.000000in}{-0.048611in}}{\pgfqpoint{0.000000in}{0.000000in}}{%
\pgfpathmoveto{\pgfqpoint{0.000000in}{0.000000in}}%
\pgfpathlineto{\pgfqpoint{0.000000in}{-0.048611in}}%
\pgfusepath{stroke,fill}%
}%
\begin{pgfscope}%
\pgfsys@transformshift{7.313106in}{1.805660in}%
\pgfsys@useobject{currentmarker}{}%
\end{pgfscope}%
\end{pgfscope}%
\begin{pgfscope}%
\pgfsetbuttcap%
\pgfsetroundjoin%
\definecolor{currentfill}{rgb}{0.000000,0.000000,0.000000}%
\pgfsetfillcolor{currentfill}%
\pgfsetlinewidth{0.803000pt}%
\definecolor{currentstroke}{rgb}{0.000000,0.000000,0.000000}%
\pgfsetstrokecolor{currentstroke}%
\pgfsetdash{}{0pt}%
\pgfsys@defobject{currentmarker}{\pgfqpoint{0.000000in}{-0.048611in}}{\pgfqpoint{0.000000in}{0.000000in}}{%
\pgfpathmoveto{\pgfqpoint{0.000000in}{0.000000in}}%
\pgfpathlineto{\pgfqpoint{0.000000in}{-0.048611in}}%
\pgfusepath{stroke,fill}%
}%
\begin{pgfscope}%
\pgfsys@transformshift{8.188400in}{1.805660in}%
\pgfsys@useobject{currentmarker}{}%
\end{pgfscope}%
\end{pgfscope}%
\begin{pgfscope}%
\pgfsetbuttcap%
\pgfsetroundjoin%
\definecolor{currentfill}{rgb}{0.000000,0.000000,0.000000}%
\pgfsetfillcolor{currentfill}%
\pgfsetlinewidth{0.803000pt}%
\definecolor{currentstroke}{rgb}{0.000000,0.000000,0.000000}%
\pgfsetstrokecolor{currentstroke}%
\pgfsetdash{}{0pt}%
\pgfsys@defobject{currentmarker}{\pgfqpoint{0.000000in}{-0.048611in}}{\pgfqpoint{0.000000in}{0.000000in}}{%
\pgfpathmoveto{\pgfqpoint{0.000000in}{0.000000in}}%
\pgfpathlineto{\pgfqpoint{0.000000in}{-0.048611in}}%
\pgfusepath{stroke,fill}%
}%
\begin{pgfscope}%
\pgfsys@transformshift{9.063694in}{1.805660in}%
\pgfsys@useobject{currentmarker}{}%
\end{pgfscope}%
\end{pgfscope}%
\begin{pgfscope}%
\pgfsetbuttcap%
\pgfsetroundjoin%
\definecolor{currentfill}{rgb}{0.000000,0.000000,0.000000}%
\pgfsetfillcolor{currentfill}%
\pgfsetlinewidth{0.803000pt}%
\definecolor{currentstroke}{rgb}{0.000000,0.000000,0.000000}%
\pgfsetstrokecolor{currentstroke}%
\pgfsetdash{}{0pt}%
\pgfsys@defobject{currentmarker}{\pgfqpoint{0.000000in}{-0.048611in}}{\pgfqpoint{0.000000in}{0.000000in}}{%
\pgfpathmoveto{\pgfqpoint{0.000000in}{0.000000in}}%
\pgfpathlineto{\pgfqpoint{0.000000in}{-0.048611in}}%
\pgfusepath{stroke,fill}%
}%
\begin{pgfscope}%
\pgfsys@transformshift{9.938988in}{1.805660in}%
\pgfsys@useobject{currentmarker}{}%
\end{pgfscope}%
\end{pgfscope}%
\begin{pgfscope}%
\pgfsetbuttcap%
\pgfsetroundjoin%
\definecolor{currentfill}{rgb}{0.000000,0.000000,0.000000}%
\pgfsetfillcolor{currentfill}%
\pgfsetlinewidth{0.803000pt}%
\definecolor{currentstroke}{rgb}{0.000000,0.000000,0.000000}%
\pgfsetstrokecolor{currentstroke}%
\pgfsetdash{}{0pt}%
\pgfsys@defobject{currentmarker}{\pgfqpoint{-0.048611in}{0.000000in}}{\pgfqpoint{0.000000in}{0.000000in}}{%
\pgfpathmoveto{\pgfqpoint{0.000000in}{0.000000in}}%
\pgfpathlineto{\pgfqpoint{-0.048611in}{0.000000in}}%
\pgfusepath{stroke,fill}%
}%
\begin{pgfscope}%
\pgfsys@transformshift{5.562518in}{2.170397in}%
\pgfsys@useobject{currentmarker}{}%
\end{pgfscope}%
\end{pgfscope}%
\begin{pgfscope}%
\pgftext[x=5.395851in,y=2.117635in,left,base]{\rmfamily\fontsize{10.000000}{12.000000}\selectfont \(\displaystyle 0\)}%
\end{pgfscope}%
\begin{pgfscope}%
\pgfsetbuttcap%
\pgfsetroundjoin%
\definecolor{currentfill}{rgb}{0.000000,0.000000,0.000000}%
\pgfsetfillcolor{currentfill}%
\pgfsetlinewidth{0.803000pt}%
\definecolor{currentstroke}{rgb}{0.000000,0.000000,0.000000}%
\pgfsetstrokecolor{currentstroke}%
\pgfsetdash{}{0pt}%
\pgfsys@defobject{currentmarker}{\pgfqpoint{-0.048611in}{0.000000in}}{\pgfqpoint{0.000000in}{0.000000in}}{%
\pgfpathmoveto{\pgfqpoint{0.000000in}{0.000000in}}%
\pgfpathlineto{\pgfqpoint{-0.048611in}{0.000000in}}%
\pgfusepath{stroke,fill}%
}%
\begin{pgfscope}%
\pgfsys@transformshift{5.562518in}{2.575660in}%
\pgfsys@useobject{currentmarker}{}%
\end{pgfscope}%
\end{pgfscope}%
\begin{pgfscope}%
\pgftext[x=5.395851in,y=2.522898in,left,base]{\rmfamily\fontsize{10.000000}{12.000000}\selectfont \(\displaystyle 2\)}%
\end{pgfscope}%
\begin{pgfscope}%
\pgfpathrectangle{\pgfqpoint{5.562518in}{1.805660in}}{\pgfqpoint{4.376471in}{0.972632in}}%
\pgfusepath{clip}%
\pgfsetrectcap%
\pgfsetroundjoin%
\pgfsetlinewidth{1.505625pt}%
\definecolor{currentstroke}{rgb}{0.121569,0.466667,0.705882}%
\pgfsetstrokecolor{currentstroke}%
\pgfsetdash{}{0pt}%
\pgfpathmoveto{\pgfqpoint{6.437812in}{2.170397in}}%
\pgfpathlineto{\pgfqpoint{9.938988in}{2.170396in}}%
\pgfpathlineto{\pgfqpoint{9.938988in}{2.170396in}}%
\pgfusepath{stroke}%
\end{pgfscope}%
\begin{pgfscope}%
\pgfsetrectcap%
\pgfsetmiterjoin%
\pgfsetlinewidth{0.803000pt}%
\definecolor{currentstroke}{rgb}{0.000000,0.000000,0.000000}%
\pgfsetstrokecolor{currentstroke}%
\pgfsetdash{}{0pt}%
\pgfpathmoveto{\pgfqpoint{5.562518in}{1.805660in}}%
\pgfpathlineto{\pgfqpoint{5.562518in}{2.778291in}}%
\pgfusepath{stroke}%
\end{pgfscope}%
\begin{pgfscope}%
\pgfsetrectcap%
\pgfsetmiterjoin%
\pgfsetlinewidth{0.803000pt}%
\definecolor{currentstroke}{rgb}{0.000000,0.000000,0.000000}%
\pgfsetstrokecolor{currentstroke}%
\pgfsetdash{}{0pt}%
\pgfpathmoveto{\pgfqpoint{9.938988in}{1.805660in}}%
\pgfpathlineto{\pgfqpoint{9.938988in}{2.778291in}}%
\pgfusepath{stroke}%
\end{pgfscope}%
\begin{pgfscope}%
\pgfsetrectcap%
\pgfsetmiterjoin%
\pgfsetlinewidth{0.803000pt}%
\definecolor{currentstroke}{rgb}{0.000000,0.000000,0.000000}%
\pgfsetstrokecolor{currentstroke}%
\pgfsetdash{}{0pt}%
\pgfpathmoveto{\pgfqpoint{5.562518in}{1.805660in}}%
\pgfpathlineto{\pgfqpoint{9.938988in}{1.805660in}}%
\pgfusepath{stroke}%
\end{pgfscope}%
\begin{pgfscope}%
\pgfsetrectcap%
\pgfsetmiterjoin%
\pgfsetlinewidth{0.803000pt}%
\definecolor{currentstroke}{rgb}{0.000000,0.000000,0.000000}%
\pgfsetstrokecolor{currentstroke}%
\pgfsetdash{}{0pt}%
\pgfpathmoveto{\pgfqpoint{5.562518in}{2.778291in}}%
\pgfpathlineto{\pgfqpoint{9.938988in}{2.778291in}}%
\pgfusepath{stroke}%
\end{pgfscope}%
\begin{pgfscope}%
\pgfsetbuttcap%
\pgfsetmiterjoin%
\definecolor{currentfill}{rgb}{1.000000,1.000000,1.000000}%
\pgfsetfillcolor{currentfill}%
\pgfsetfillopacity{0.800000}%
\pgfsetlinewidth{1.003750pt}%
\definecolor{currentstroke}{rgb}{0.800000,0.800000,0.800000}%
\pgfsetstrokecolor{currentstroke}%
\pgfsetstrokeopacity{0.800000}%
\pgfsetdash{}{0pt}%
\pgfpathmoveto{\pgfqpoint{5.659740in}{1.875104in}}%
\pgfpathlineto{\pgfqpoint{6.443065in}{1.875104in}}%
\pgfpathquadraticcurveto{\pgfqpoint{6.470843in}{1.875104in}}{\pgfqpoint{6.470843in}{1.902882in}}%
\pgfpathlineto{\pgfqpoint{6.470843in}{2.501569in}}%
\pgfpathquadraticcurveto{\pgfqpoint{6.470843in}{2.529346in}}{\pgfqpoint{6.443065in}{2.529346in}}%
\pgfpathlineto{\pgfqpoint{5.659740in}{2.529346in}}%
\pgfpathquadraticcurveto{\pgfqpoint{5.631962in}{2.529346in}}{\pgfqpoint{5.631962in}{2.501569in}}%
\pgfpathlineto{\pgfqpoint{5.631962in}{1.902882in}}%
\pgfpathquadraticcurveto{\pgfqpoint{5.631962in}{1.875104in}}{\pgfqpoint{5.659740in}{1.875104in}}%
\pgfpathclose%
\pgfusepath{stroke,fill}%
\end{pgfscope}%
\begin{pgfscope}%
\pgfsetrectcap%
\pgfsetroundjoin%
\pgfsetlinewidth{1.505625pt}%
\definecolor{currentstroke}{rgb}{0.121569,0.466667,0.705882}%
\pgfsetstrokecolor{currentstroke}%
\pgfsetdash{}{0pt}%
\pgfpathmoveto{\pgfqpoint{5.687518in}{2.415875in}}%
\pgfpathlineto{\pgfqpoint{5.965295in}{2.415875in}}%
\pgfusepath{stroke}%
\end{pgfscope}%
\begin{pgfscope}%
\pgftext[x=6.076407in,y=2.367264in,left,base]{\rmfamily\fontsize{10.000000}{12.000000}\selectfont \(\displaystyle \widetilde{\Phi}^* \theta^{\parallel}\)}%
\end{pgfscope}%
\begin{pgfscope}%
\pgfsetbuttcap%
\pgfsetroundjoin%
\definecolor{currentfill}{rgb}{1.000000,0.000000,0.000000}%
\pgfsetfillcolor{currentfill}%
\pgfsetlinewidth{2.007500pt}%
\definecolor{currentstroke}{rgb}{1.000000,0.000000,0.000000}%
\pgfsetstrokecolor{currentstroke}%
\pgfsetdash{}{0pt}%
\pgfpathmoveto{\pgfqpoint{5.784740in}{2.199865in}}%
\pgfpathlineto{\pgfqpoint{5.868073in}{2.199865in}}%
\pgfpathmoveto{\pgfqpoint{5.826407in}{2.158198in}}%
\pgfpathlineto{\pgfqpoint{5.826407in}{2.241532in}}%
\pgfusepath{stroke,fill}%
\end{pgfscope}%
\begin{pgfscope}%
\pgftext[x=6.076407in,y=2.163407in,left,base]{\rmfamily\fontsize{10.000000}{12.000000}\selectfont train}%
\end{pgfscope}%
\begin{pgfscope}%
\pgfsetbuttcap%
\pgfsetroundjoin%
\definecolor{currentfill}{rgb}{0.000000,0.000000,0.000000}%
\pgfsetfillcolor{currentfill}%
\pgfsetlinewidth{1.003750pt}%
\definecolor{currentstroke}{rgb}{0.000000,0.000000,0.000000}%
\pgfsetstrokecolor{currentstroke}%
\pgfsetdash{}{0pt}%
\pgfsys@defobject{currentmarker}{\pgfqpoint{-0.020833in}{-0.020833in}}{\pgfqpoint{0.020833in}{0.020833in}}{%
\pgfpathmoveto{\pgfqpoint{0.000000in}{-0.020833in}}%
\pgfpathcurveto{\pgfqpoint{0.005525in}{-0.020833in}}{\pgfqpoint{0.010825in}{-0.018638in}}{\pgfqpoint{0.014731in}{-0.014731in}}%
\pgfpathcurveto{\pgfqpoint{0.018638in}{-0.010825in}}{\pgfqpoint{0.020833in}{-0.005525in}}{\pgfqpoint{0.020833in}{0.000000in}}%
\pgfpathcurveto{\pgfqpoint{0.020833in}{0.005525in}}{\pgfqpoint{0.018638in}{0.010825in}}{\pgfqpoint{0.014731in}{0.014731in}}%
\pgfpathcurveto{\pgfqpoint{0.010825in}{0.018638in}}{\pgfqpoint{0.005525in}{0.020833in}}{\pgfqpoint{0.000000in}{0.020833in}}%
\pgfpathcurveto{\pgfqpoint{-0.005525in}{0.020833in}}{\pgfqpoint{-0.010825in}{0.018638in}}{\pgfqpoint{-0.014731in}{0.014731in}}%
\pgfpathcurveto{\pgfqpoint{-0.018638in}{0.010825in}}{\pgfqpoint{-0.020833in}{0.005525in}}{\pgfqpoint{-0.020833in}{0.000000in}}%
\pgfpathcurveto{\pgfqpoint{-0.020833in}{-0.005525in}}{\pgfqpoint{-0.018638in}{-0.010825in}}{\pgfqpoint{-0.014731in}{-0.014731in}}%
\pgfpathcurveto{\pgfqpoint{-0.010825in}{-0.018638in}}{\pgfqpoint{-0.005525in}{-0.020833in}}{\pgfqpoint{0.000000in}{-0.020833in}}%
\pgfpathclose%
\pgfusepath{stroke,fill}%
}%
\begin{pgfscope}%
\pgfsys@transformshift{5.826407in}{1.996008in}%
\pgfsys@useobject{currentmarker}{}%
\end{pgfscope}%
\end{pgfscope}%
\begin{pgfscope}%
\pgftext[x=6.076407in,y=1.959549in,left,base]{\rmfamily\fontsize{10.000000}{12.000000}\selectfont test}%
\end{pgfscope}%
\begin{pgfscope}%
\pgfsetbuttcap%
\pgfsetmiterjoin%
\definecolor{currentfill}{rgb}{1.000000,1.000000,1.000000}%
\pgfsetfillcolor{currentfill}%
\pgfsetlinewidth{0.000000pt}%
\definecolor{currentstroke}{rgb}{0.000000,0.000000,0.000000}%
\pgfsetstrokecolor{currentstroke}%
\pgfsetstrokeopacity{0.000000}%
\pgfsetdash{}{0pt}%
\pgfpathmoveto{\pgfqpoint{10.668400in}{1.805660in}}%
\pgfpathlineto{\pgfqpoint{12.856635in}{1.805660in}}%
\pgfpathlineto{\pgfqpoint{12.856635in}{2.778291in}}%
\pgfpathlineto{\pgfqpoint{10.668400in}{2.778291in}}%
\pgfpathclose%
\pgfusepath{fill}%
\end{pgfscope}%
\begin{pgfscope}%
\pgfpathrectangle{\pgfqpoint{10.668400in}{1.805660in}}{\pgfqpoint{2.188235in}{0.972632in}}%
\pgfusepath{clip}%
\pgfsetbuttcap%
\pgfsetmiterjoin%
\definecolor{currentfill}{rgb}{0.121569,0.466667,0.705882}%
\pgfsetfillcolor{currentfill}%
\pgfsetlinewidth{0.000000pt}%
\definecolor{currentstroke}{rgb}{0.000000,0.000000,0.000000}%
\pgfsetstrokecolor{currentstroke}%
\pgfsetstrokeopacity{0.000000}%
\pgfsetdash{}{0pt}%
\pgfpathmoveto{\pgfqpoint{-319.224843in}{1.849870in}}%
\pgfpathlineto{\pgfqpoint{11.066973in}{1.849870in}}%
\pgfpathlineto{\pgfqpoint{11.066973in}{1.856958in}}%
\pgfpathlineto{\pgfqpoint{-319.224843in}{1.856958in}}%
\pgfpathclose%
\pgfusepath{fill}%
\end{pgfscope}%
\begin{pgfscope}%
\pgfpathrectangle{\pgfqpoint{10.668400in}{1.805660in}}{\pgfqpoint{2.188235in}{0.972632in}}%
\pgfusepath{clip}%
\pgfsetbuttcap%
\pgfsetmiterjoin%
\definecolor{currentfill}{rgb}{0.121569,0.466667,0.705882}%
\pgfsetfillcolor{currentfill}%
\pgfsetlinewidth{0.000000pt}%
\definecolor{currentstroke}{rgb}{0.000000,0.000000,0.000000}%
\pgfsetstrokecolor{currentstroke}%
\pgfsetstrokeopacity{0.000000}%
\pgfsetdash{}{0pt}%
\pgfpathmoveto{\pgfqpoint{-319.224843in}{1.858730in}}%
\pgfpathlineto{\pgfqpoint{10.938860in}{1.858730in}}%
\pgfpathlineto{\pgfqpoint{10.938860in}{1.865818in}}%
\pgfpathlineto{\pgfqpoint{-319.224843in}{1.865818in}}%
\pgfpathclose%
\pgfusepath{fill}%
\end{pgfscope}%
\begin{pgfscope}%
\pgfpathrectangle{\pgfqpoint{10.668400in}{1.805660in}}{\pgfqpoint{2.188235in}{0.972632in}}%
\pgfusepath{clip}%
\pgfsetbuttcap%
\pgfsetmiterjoin%
\definecolor{currentfill}{rgb}{0.121569,0.466667,0.705882}%
\pgfsetfillcolor{currentfill}%
\pgfsetlinewidth{0.000000pt}%
\definecolor{currentstroke}{rgb}{0.000000,0.000000,0.000000}%
\pgfsetstrokecolor{currentstroke}%
\pgfsetstrokeopacity{0.000000}%
\pgfsetdash{}{0pt}%
\pgfpathmoveto{\pgfqpoint{-319.224843in}{1.867590in}}%
\pgfpathlineto{\pgfqpoint{10.970311in}{1.867590in}}%
\pgfpathlineto{\pgfqpoint{10.970311in}{1.874678in}}%
\pgfpathlineto{\pgfqpoint{-319.224843in}{1.874678in}}%
\pgfpathclose%
\pgfusepath{fill}%
\end{pgfscope}%
\begin{pgfscope}%
\pgfpathrectangle{\pgfqpoint{10.668400in}{1.805660in}}{\pgfqpoint{2.188235in}{0.972632in}}%
\pgfusepath{clip}%
\pgfsetbuttcap%
\pgfsetmiterjoin%
\definecolor{currentfill}{rgb}{0.121569,0.466667,0.705882}%
\pgfsetfillcolor{currentfill}%
\pgfsetlinewidth{0.000000pt}%
\definecolor{currentstroke}{rgb}{0.000000,0.000000,0.000000}%
\pgfsetstrokecolor{currentstroke}%
\pgfsetstrokeopacity{0.000000}%
\pgfsetdash{}{0pt}%
\pgfpathmoveto{\pgfqpoint{-319.224843in}{1.876450in}}%
\pgfpathlineto{\pgfqpoint{11.014654in}{1.876450in}}%
\pgfpathlineto{\pgfqpoint{11.014654in}{1.883538in}}%
\pgfpathlineto{\pgfqpoint{-319.224843in}{1.883538in}}%
\pgfpathclose%
\pgfusepath{fill}%
\end{pgfscope}%
\begin{pgfscope}%
\pgfpathrectangle{\pgfqpoint{10.668400in}{1.805660in}}{\pgfqpoint{2.188235in}{0.972632in}}%
\pgfusepath{clip}%
\pgfsetbuttcap%
\pgfsetmiterjoin%
\definecolor{currentfill}{rgb}{0.121569,0.466667,0.705882}%
\pgfsetfillcolor{currentfill}%
\pgfsetlinewidth{0.000000pt}%
\definecolor{currentstroke}{rgb}{0.000000,0.000000,0.000000}%
\pgfsetstrokecolor{currentstroke}%
\pgfsetstrokeopacity{0.000000}%
\pgfsetdash{}{0pt}%
\pgfpathmoveto{\pgfqpoint{-319.224843in}{1.885310in}}%
\pgfpathlineto{\pgfqpoint{10.739579in}{1.885310in}}%
\pgfpathlineto{\pgfqpoint{10.739579in}{1.892397in}}%
\pgfpathlineto{\pgfqpoint{-319.224843in}{1.892397in}}%
\pgfpathclose%
\pgfusepath{fill}%
\end{pgfscope}%
\begin{pgfscope}%
\pgfpathrectangle{\pgfqpoint{10.668400in}{1.805660in}}{\pgfqpoint{2.188235in}{0.972632in}}%
\pgfusepath{clip}%
\pgfsetbuttcap%
\pgfsetmiterjoin%
\definecolor{currentfill}{rgb}{0.121569,0.466667,0.705882}%
\pgfsetfillcolor{currentfill}%
\pgfsetlinewidth{0.000000pt}%
\definecolor{currentstroke}{rgb}{0.000000,0.000000,0.000000}%
\pgfsetstrokecolor{currentstroke}%
\pgfsetstrokeopacity{0.000000}%
\pgfsetdash{}{0pt}%
\pgfpathmoveto{\pgfqpoint{-319.224843in}{1.894169in}}%
\pgfpathlineto{\pgfqpoint{10.975948in}{1.894169in}}%
\pgfpathlineto{\pgfqpoint{10.975948in}{1.901257in}}%
\pgfpathlineto{\pgfqpoint{-319.224843in}{1.901257in}}%
\pgfpathclose%
\pgfusepath{fill}%
\end{pgfscope}%
\begin{pgfscope}%
\pgfpathrectangle{\pgfqpoint{10.668400in}{1.805660in}}{\pgfqpoint{2.188235in}{0.972632in}}%
\pgfusepath{clip}%
\pgfsetbuttcap%
\pgfsetmiterjoin%
\definecolor{currentfill}{rgb}{0.121569,0.466667,0.705882}%
\pgfsetfillcolor{currentfill}%
\pgfsetlinewidth{0.000000pt}%
\definecolor{currentstroke}{rgb}{0.000000,0.000000,0.000000}%
\pgfsetstrokecolor{currentstroke}%
\pgfsetstrokeopacity{0.000000}%
\pgfsetdash{}{0pt}%
\pgfpathmoveto{\pgfqpoint{-319.224843in}{1.903029in}}%
\pgfpathlineto{\pgfqpoint{9.891604in}{1.903029in}}%
\pgfpathlineto{\pgfqpoint{9.891604in}{1.910117in}}%
\pgfpathlineto{\pgfqpoint{-319.224843in}{1.910117in}}%
\pgfpathclose%
\pgfusepath{fill}%
\end{pgfscope}%
\begin{pgfscope}%
\pgfpathrectangle{\pgfqpoint{10.668400in}{1.805660in}}{\pgfqpoint{2.188235in}{0.972632in}}%
\pgfusepath{clip}%
\pgfsetbuttcap%
\pgfsetmiterjoin%
\definecolor{currentfill}{rgb}{0.121569,0.466667,0.705882}%
\pgfsetfillcolor{currentfill}%
\pgfsetlinewidth{0.000000pt}%
\definecolor{currentstroke}{rgb}{0.000000,0.000000,0.000000}%
\pgfsetstrokecolor{currentstroke}%
\pgfsetstrokeopacity{0.000000}%
\pgfsetdash{}{0pt}%
\pgfpathmoveto{\pgfqpoint{-319.224843in}{1.911889in}}%
\pgfpathlineto{\pgfqpoint{11.082853in}{1.911889in}}%
\pgfpathlineto{\pgfqpoint{11.082853in}{1.918977in}}%
\pgfpathlineto{\pgfqpoint{-319.224843in}{1.918977in}}%
\pgfpathclose%
\pgfusepath{fill}%
\end{pgfscope}%
\begin{pgfscope}%
\pgfpathrectangle{\pgfqpoint{10.668400in}{1.805660in}}{\pgfqpoint{2.188235in}{0.972632in}}%
\pgfusepath{clip}%
\pgfsetbuttcap%
\pgfsetmiterjoin%
\definecolor{currentfill}{rgb}{0.121569,0.466667,0.705882}%
\pgfsetfillcolor{currentfill}%
\pgfsetlinewidth{0.000000pt}%
\definecolor{currentstroke}{rgb}{0.000000,0.000000,0.000000}%
\pgfsetstrokecolor{currentstroke}%
\pgfsetstrokeopacity{0.000000}%
\pgfsetdash{}{0pt}%
\pgfpathmoveto{\pgfqpoint{-319.224843in}{1.920749in}}%
\pgfpathlineto{\pgfqpoint{10.961076in}{1.920749in}}%
\pgfpathlineto{\pgfqpoint{10.961076in}{1.927837in}}%
\pgfpathlineto{\pgfqpoint{-319.224843in}{1.927837in}}%
\pgfpathclose%
\pgfusepath{fill}%
\end{pgfscope}%
\begin{pgfscope}%
\pgfpathrectangle{\pgfqpoint{10.668400in}{1.805660in}}{\pgfqpoint{2.188235in}{0.972632in}}%
\pgfusepath{clip}%
\pgfsetbuttcap%
\pgfsetmiterjoin%
\definecolor{currentfill}{rgb}{0.121569,0.466667,0.705882}%
\pgfsetfillcolor{currentfill}%
\pgfsetlinewidth{0.000000pt}%
\definecolor{currentstroke}{rgb}{0.000000,0.000000,0.000000}%
\pgfsetstrokecolor{currentstroke}%
\pgfsetstrokeopacity{0.000000}%
\pgfsetdash{}{0pt}%
\pgfpathmoveto{\pgfqpoint{-319.224843in}{1.929609in}}%
\pgfpathlineto{\pgfqpoint{11.149276in}{1.929609in}}%
\pgfpathlineto{\pgfqpoint{11.149276in}{1.936697in}}%
\pgfpathlineto{\pgfqpoint{-319.224843in}{1.936697in}}%
\pgfpathclose%
\pgfusepath{fill}%
\end{pgfscope}%
\begin{pgfscope}%
\pgfpathrectangle{\pgfqpoint{10.668400in}{1.805660in}}{\pgfqpoint{2.188235in}{0.972632in}}%
\pgfusepath{clip}%
\pgfsetbuttcap%
\pgfsetmiterjoin%
\definecolor{currentfill}{rgb}{0.121569,0.466667,0.705882}%
\pgfsetfillcolor{currentfill}%
\pgfsetlinewidth{0.000000pt}%
\definecolor{currentstroke}{rgb}{0.000000,0.000000,0.000000}%
\pgfsetstrokecolor{currentstroke}%
\pgfsetstrokeopacity{0.000000}%
\pgfsetdash{}{0pt}%
\pgfpathmoveto{\pgfqpoint{-319.224843in}{1.938469in}}%
\pgfpathlineto{\pgfqpoint{11.121076in}{1.938469in}}%
\pgfpathlineto{\pgfqpoint{11.121076in}{1.945556in}}%
\pgfpathlineto{\pgfqpoint{-319.224843in}{1.945556in}}%
\pgfpathclose%
\pgfusepath{fill}%
\end{pgfscope}%
\begin{pgfscope}%
\pgfpathrectangle{\pgfqpoint{10.668400in}{1.805660in}}{\pgfqpoint{2.188235in}{0.972632in}}%
\pgfusepath{clip}%
\pgfsetbuttcap%
\pgfsetmiterjoin%
\definecolor{currentfill}{rgb}{0.121569,0.466667,0.705882}%
\pgfsetfillcolor{currentfill}%
\pgfsetlinewidth{0.000000pt}%
\definecolor{currentstroke}{rgb}{0.000000,0.000000,0.000000}%
\pgfsetstrokecolor{currentstroke}%
\pgfsetstrokeopacity{0.000000}%
\pgfsetdash{}{0pt}%
\pgfpathmoveto{\pgfqpoint{-319.224843in}{1.947328in}}%
\pgfpathlineto{\pgfqpoint{10.906602in}{1.947328in}}%
\pgfpathlineto{\pgfqpoint{10.906602in}{1.954416in}}%
\pgfpathlineto{\pgfqpoint{-319.224843in}{1.954416in}}%
\pgfpathclose%
\pgfusepath{fill}%
\end{pgfscope}%
\begin{pgfscope}%
\pgfpathrectangle{\pgfqpoint{10.668400in}{1.805660in}}{\pgfqpoint{2.188235in}{0.972632in}}%
\pgfusepath{clip}%
\pgfsetbuttcap%
\pgfsetmiterjoin%
\definecolor{currentfill}{rgb}{0.121569,0.466667,0.705882}%
\pgfsetfillcolor{currentfill}%
\pgfsetlinewidth{0.000000pt}%
\definecolor{currentstroke}{rgb}{0.000000,0.000000,0.000000}%
\pgfsetstrokecolor{currentstroke}%
\pgfsetstrokeopacity{0.000000}%
\pgfsetdash{}{0pt}%
\pgfpathmoveto{\pgfqpoint{-319.224843in}{1.956188in}}%
\pgfpathlineto{\pgfqpoint{10.556206in}{1.956188in}}%
\pgfpathlineto{\pgfqpoint{10.556206in}{1.963276in}}%
\pgfpathlineto{\pgfqpoint{-319.224843in}{1.963276in}}%
\pgfpathclose%
\pgfusepath{fill}%
\end{pgfscope}%
\begin{pgfscope}%
\pgfpathrectangle{\pgfqpoint{10.668400in}{1.805660in}}{\pgfqpoint{2.188235in}{0.972632in}}%
\pgfusepath{clip}%
\pgfsetbuttcap%
\pgfsetmiterjoin%
\definecolor{currentfill}{rgb}{0.121569,0.466667,0.705882}%
\pgfsetfillcolor{currentfill}%
\pgfsetlinewidth{0.000000pt}%
\definecolor{currentstroke}{rgb}{0.000000,0.000000,0.000000}%
\pgfsetstrokecolor{currentstroke}%
\pgfsetstrokeopacity{0.000000}%
\pgfsetdash{}{0pt}%
\pgfpathmoveto{\pgfqpoint{-319.224843in}{1.965048in}}%
\pgfpathlineto{\pgfqpoint{10.894216in}{1.965048in}}%
\pgfpathlineto{\pgfqpoint{10.894216in}{1.972136in}}%
\pgfpathlineto{\pgfqpoint{-319.224843in}{1.972136in}}%
\pgfpathclose%
\pgfusepath{fill}%
\end{pgfscope}%
\begin{pgfscope}%
\pgfpathrectangle{\pgfqpoint{10.668400in}{1.805660in}}{\pgfqpoint{2.188235in}{0.972632in}}%
\pgfusepath{clip}%
\pgfsetbuttcap%
\pgfsetmiterjoin%
\definecolor{currentfill}{rgb}{0.121569,0.466667,0.705882}%
\pgfsetfillcolor{currentfill}%
\pgfsetlinewidth{0.000000pt}%
\definecolor{currentstroke}{rgb}{0.000000,0.000000,0.000000}%
\pgfsetstrokecolor{currentstroke}%
\pgfsetstrokeopacity{0.000000}%
\pgfsetdash{}{0pt}%
\pgfpathmoveto{\pgfqpoint{-319.224843in}{1.973908in}}%
\pgfpathlineto{\pgfqpoint{10.747699in}{1.973908in}}%
\pgfpathlineto{\pgfqpoint{10.747699in}{1.980996in}}%
\pgfpathlineto{\pgfqpoint{-319.224843in}{1.980996in}}%
\pgfpathclose%
\pgfusepath{fill}%
\end{pgfscope}%
\begin{pgfscope}%
\pgfpathrectangle{\pgfqpoint{10.668400in}{1.805660in}}{\pgfqpoint{2.188235in}{0.972632in}}%
\pgfusepath{clip}%
\pgfsetbuttcap%
\pgfsetmiterjoin%
\definecolor{currentfill}{rgb}{0.121569,0.466667,0.705882}%
\pgfsetfillcolor{currentfill}%
\pgfsetlinewidth{0.000000pt}%
\definecolor{currentstroke}{rgb}{0.000000,0.000000,0.000000}%
\pgfsetstrokecolor{currentstroke}%
\pgfsetstrokeopacity{0.000000}%
\pgfsetdash{}{0pt}%
\pgfpathmoveto{\pgfqpoint{-319.224843in}{1.982768in}}%
\pgfpathlineto{\pgfqpoint{10.808963in}{1.982768in}}%
\pgfpathlineto{\pgfqpoint{10.808963in}{1.989856in}}%
\pgfpathlineto{\pgfqpoint{-319.224843in}{1.989856in}}%
\pgfpathclose%
\pgfusepath{fill}%
\end{pgfscope}%
\begin{pgfscope}%
\pgfpathrectangle{\pgfqpoint{10.668400in}{1.805660in}}{\pgfqpoint{2.188235in}{0.972632in}}%
\pgfusepath{clip}%
\pgfsetbuttcap%
\pgfsetmiterjoin%
\definecolor{currentfill}{rgb}{0.121569,0.466667,0.705882}%
\pgfsetfillcolor{currentfill}%
\pgfsetlinewidth{0.000000pt}%
\definecolor{currentstroke}{rgb}{0.000000,0.000000,0.000000}%
\pgfsetstrokecolor{currentstroke}%
\pgfsetstrokeopacity{0.000000}%
\pgfsetdash{}{0pt}%
\pgfpathmoveto{\pgfqpoint{-319.224843in}{1.991627in}}%
\pgfpathlineto{\pgfqpoint{11.087019in}{1.991627in}}%
\pgfpathlineto{\pgfqpoint{11.087019in}{1.998715in}}%
\pgfpathlineto{\pgfqpoint{-319.224843in}{1.998715in}}%
\pgfpathclose%
\pgfusepath{fill}%
\end{pgfscope}%
\begin{pgfscope}%
\pgfpathrectangle{\pgfqpoint{10.668400in}{1.805660in}}{\pgfqpoint{2.188235in}{0.972632in}}%
\pgfusepath{clip}%
\pgfsetbuttcap%
\pgfsetmiterjoin%
\definecolor{currentfill}{rgb}{0.121569,0.466667,0.705882}%
\pgfsetfillcolor{currentfill}%
\pgfsetlinewidth{0.000000pt}%
\definecolor{currentstroke}{rgb}{0.000000,0.000000,0.000000}%
\pgfsetstrokecolor{currentstroke}%
\pgfsetstrokeopacity{0.000000}%
\pgfsetdash{}{0pt}%
\pgfpathmoveto{\pgfqpoint{-319.224843in}{2.000487in}}%
\pgfpathlineto{\pgfqpoint{10.965305in}{2.000487in}}%
\pgfpathlineto{\pgfqpoint{10.965305in}{2.007575in}}%
\pgfpathlineto{\pgfqpoint{-319.224843in}{2.007575in}}%
\pgfpathclose%
\pgfusepath{fill}%
\end{pgfscope}%
\begin{pgfscope}%
\pgfpathrectangle{\pgfqpoint{10.668400in}{1.805660in}}{\pgfqpoint{2.188235in}{0.972632in}}%
\pgfusepath{clip}%
\pgfsetbuttcap%
\pgfsetmiterjoin%
\definecolor{currentfill}{rgb}{0.121569,0.466667,0.705882}%
\pgfsetfillcolor{currentfill}%
\pgfsetlinewidth{0.000000pt}%
\definecolor{currentstroke}{rgb}{0.000000,0.000000,0.000000}%
\pgfsetstrokecolor{currentstroke}%
\pgfsetstrokeopacity{0.000000}%
\pgfsetdash{}{0pt}%
\pgfpathmoveto{\pgfqpoint{-319.224843in}{2.009347in}}%
\pgfpathlineto{\pgfqpoint{11.098159in}{2.009347in}}%
\pgfpathlineto{\pgfqpoint{11.098159in}{2.016435in}}%
\pgfpathlineto{\pgfqpoint{-319.224843in}{2.016435in}}%
\pgfpathclose%
\pgfusepath{fill}%
\end{pgfscope}%
\begin{pgfscope}%
\pgfpathrectangle{\pgfqpoint{10.668400in}{1.805660in}}{\pgfqpoint{2.188235in}{0.972632in}}%
\pgfusepath{clip}%
\pgfsetbuttcap%
\pgfsetmiterjoin%
\definecolor{currentfill}{rgb}{0.121569,0.466667,0.705882}%
\pgfsetfillcolor{currentfill}%
\pgfsetlinewidth{0.000000pt}%
\definecolor{currentstroke}{rgb}{0.000000,0.000000,0.000000}%
\pgfsetstrokecolor{currentstroke}%
\pgfsetstrokeopacity{0.000000}%
\pgfsetdash{}{0pt}%
\pgfpathmoveto{\pgfqpoint{-319.224843in}{2.018207in}}%
\pgfpathlineto{\pgfqpoint{10.698074in}{2.018207in}}%
\pgfpathlineto{\pgfqpoint{10.698074in}{2.025295in}}%
\pgfpathlineto{\pgfqpoint{-319.224843in}{2.025295in}}%
\pgfpathclose%
\pgfusepath{fill}%
\end{pgfscope}%
\begin{pgfscope}%
\pgfpathrectangle{\pgfqpoint{10.668400in}{1.805660in}}{\pgfqpoint{2.188235in}{0.972632in}}%
\pgfusepath{clip}%
\pgfsetbuttcap%
\pgfsetmiterjoin%
\definecolor{currentfill}{rgb}{0.121569,0.466667,0.705882}%
\pgfsetfillcolor{currentfill}%
\pgfsetlinewidth{0.000000pt}%
\definecolor{currentstroke}{rgb}{0.000000,0.000000,0.000000}%
\pgfsetstrokecolor{currentstroke}%
\pgfsetstrokeopacity{0.000000}%
\pgfsetdash{}{0pt}%
\pgfpathmoveto{\pgfqpoint{-319.224843in}{2.027067in}}%
\pgfpathlineto{\pgfqpoint{10.897310in}{2.027067in}}%
\pgfpathlineto{\pgfqpoint{10.897310in}{2.034155in}}%
\pgfpathlineto{\pgfqpoint{-319.224843in}{2.034155in}}%
\pgfpathclose%
\pgfusepath{fill}%
\end{pgfscope}%
\begin{pgfscope}%
\pgfpathrectangle{\pgfqpoint{10.668400in}{1.805660in}}{\pgfqpoint{2.188235in}{0.972632in}}%
\pgfusepath{clip}%
\pgfsetbuttcap%
\pgfsetmiterjoin%
\definecolor{currentfill}{rgb}{0.121569,0.466667,0.705882}%
\pgfsetfillcolor{currentfill}%
\pgfsetlinewidth{0.000000pt}%
\definecolor{currentstroke}{rgb}{0.000000,0.000000,0.000000}%
\pgfsetstrokecolor{currentstroke}%
\pgfsetstrokeopacity{0.000000}%
\pgfsetdash{}{0pt}%
\pgfpathmoveto{\pgfqpoint{-319.224843in}{2.035927in}}%
\pgfpathlineto{\pgfqpoint{10.789992in}{2.035927in}}%
\pgfpathlineto{\pgfqpoint{10.789992in}{2.043014in}}%
\pgfpathlineto{\pgfqpoint{-319.224843in}{2.043014in}}%
\pgfpathclose%
\pgfusepath{fill}%
\end{pgfscope}%
\begin{pgfscope}%
\pgfpathrectangle{\pgfqpoint{10.668400in}{1.805660in}}{\pgfqpoint{2.188235in}{0.972632in}}%
\pgfusepath{clip}%
\pgfsetbuttcap%
\pgfsetmiterjoin%
\definecolor{currentfill}{rgb}{0.121569,0.466667,0.705882}%
\pgfsetfillcolor{currentfill}%
\pgfsetlinewidth{0.000000pt}%
\definecolor{currentstroke}{rgb}{0.000000,0.000000,0.000000}%
\pgfsetstrokecolor{currentstroke}%
\pgfsetstrokeopacity{0.000000}%
\pgfsetdash{}{0pt}%
\pgfpathmoveto{\pgfqpoint{-319.224843in}{2.044786in}}%
\pgfpathlineto{\pgfqpoint{10.822453in}{2.044786in}}%
\pgfpathlineto{\pgfqpoint{10.822453in}{2.051874in}}%
\pgfpathlineto{\pgfqpoint{-319.224843in}{2.051874in}}%
\pgfpathclose%
\pgfusepath{fill}%
\end{pgfscope}%
\begin{pgfscope}%
\pgfpathrectangle{\pgfqpoint{10.668400in}{1.805660in}}{\pgfqpoint{2.188235in}{0.972632in}}%
\pgfusepath{clip}%
\pgfsetbuttcap%
\pgfsetmiterjoin%
\definecolor{currentfill}{rgb}{0.121569,0.466667,0.705882}%
\pgfsetfillcolor{currentfill}%
\pgfsetlinewidth{0.000000pt}%
\definecolor{currentstroke}{rgb}{0.000000,0.000000,0.000000}%
\pgfsetstrokecolor{currentstroke}%
\pgfsetstrokeopacity{0.000000}%
\pgfsetdash{}{0pt}%
\pgfpathmoveto{\pgfqpoint{-319.224843in}{2.053646in}}%
\pgfpathlineto{\pgfqpoint{10.998144in}{2.053646in}}%
\pgfpathlineto{\pgfqpoint{10.998144in}{2.060734in}}%
\pgfpathlineto{\pgfqpoint{-319.224843in}{2.060734in}}%
\pgfpathclose%
\pgfusepath{fill}%
\end{pgfscope}%
\begin{pgfscope}%
\pgfpathrectangle{\pgfqpoint{10.668400in}{1.805660in}}{\pgfqpoint{2.188235in}{0.972632in}}%
\pgfusepath{clip}%
\pgfsetbuttcap%
\pgfsetmiterjoin%
\definecolor{currentfill}{rgb}{0.121569,0.466667,0.705882}%
\pgfsetfillcolor{currentfill}%
\pgfsetlinewidth{0.000000pt}%
\definecolor{currentstroke}{rgb}{0.000000,0.000000,0.000000}%
\pgfsetstrokecolor{currentstroke}%
\pgfsetstrokeopacity{0.000000}%
\pgfsetdash{}{0pt}%
\pgfpathmoveto{\pgfqpoint{-319.224843in}{2.062506in}}%
\pgfpathlineto{\pgfqpoint{10.972010in}{2.062506in}}%
\pgfpathlineto{\pgfqpoint{10.972010in}{2.069594in}}%
\pgfpathlineto{\pgfqpoint{-319.224843in}{2.069594in}}%
\pgfpathclose%
\pgfusepath{fill}%
\end{pgfscope}%
\begin{pgfscope}%
\pgfpathrectangle{\pgfqpoint{10.668400in}{1.805660in}}{\pgfqpoint{2.188235in}{0.972632in}}%
\pgfusepath{clip}%
\pgfsetbuttcap%
\pgfsetmiterjoin%
\definecolor{currentfill}{rgb}{0.121569,0.466667,0.705882}%
\pgfsetfillcolor{currentfill}%
\pgfsetlinewidth{0.000000pt}%
\definecolor{currentstroke}{rgb}{0.000000,0.000000,0.000000}%
\pgfsetstrokecolor{currentstroke}%
\pgfsetstrokeopacity{0.000000}%
\pgfsetdash{}{0pt}%
\pgfpathmoveto{\pgfqpoint{-319.224843in}{2.071366in}}%
\pgfpathlineto{\pgfqpoint{10.549465in}{2.071366in}}%
\pgfpathlineto{\pgfqpoint{10.549465in}{2.078454in}}%
\pgfpathlineto{\pgfqpoint{-319.224843in}{2.078454in}}%
\pgfpathclose%
\pgfusepath{fill}%
\end{pgfscope}%
\begin{pgfscope}%
\pgfpathrectangle{\pgfqpoint{10.668400in}{1.805660in}}{\pgfqpoint{2.188235in}{0.972632in}}%
\pgfusepath{clip}%
\pgfsetbuttcap%
\pgfsetmiterjoin%
\definecolor{currentfill}{rgb}{0.121569,0.466667,0.705882}%
\pgfsetfillcolor{currentfill}%
\pgfsetlinewidth{0.000000pt}%
\definecolor{currentstroke}{rgb}{0.000000,0.000000,0.000000}%
\pgfsetstrokecolor{currentstroke}%
\pgfsetstrokeopacity{0.000000}%
\pgfsetdash{}{0pt}%
\pgfpathmoveto{\pgfqpoint{-319.224843in}{2.080226in}}%
\pgfpathlineto{\pgfqpoint{11.129297in}{2.080226in}}%
\pgfpathlineto{\pgfqpoint{11.129297in}{2.087314in}}%
\pgfpathlineto{\pgfqpoint{-319.224843in}{2.087314in}}%
\pgfpathclose%
\pgfusepath{fill}%
\end{pgfscope}%
\begin{pgfscope}%
\pgfpathrectangle{\pgfqpoint{10.668400in}{1.805660in}}{\pgfqpoint{2.188235in}{0.972632in}}%
\pgfusepath{clip}%
\pgfsetbuttcap%
\pgfsetmiterjoin%
\definecolor{currentfill}{rgb}{0.121569,0.466667,0.705882}%
\pgfsetfillcolor{currentfill}%
\pgfsetlinewidth{0.000000pt}%
\definecolor{currentstroke}{rgb}{0.000000,0.000000,0.000000}%
\pgfsetstrokecolor{currentstroke}%
\pgfsetstrokeopacity{0.000000}%
\pgfsetdash{}{0pt}%
\pgfpathmoveto{\pgfqpoint{-319.224843in}{2.089086in}}%
\pgfpathlineto{\pgfqpoint{10.914117in}{2.089086in}}%
\pgfpathlineto{\pgfqpoint{10.914117in}{2.096173in}}%
\pgfpathlineto{\pgfqpoint{-319.224843in}{2.096173in}}%
\pgfpathclose%
\pgfusepath{fill}%
\end{pgfscope}%
\begin{pgfscope}%
\pgfpathrectangle{\pgfqpoint{10.668400in}{1.805660in}}{\pgfqpoint{2.188235in}{0.972632in}}%
\pgfusepath{clip}%
\pgfsetbuttcap%
\pgfsetmiterjoin%
\definecolor{currentfill}{rgb}{0.121569,0.466667,0.705882}%
\pgfsetfillcolor{currentfill}%
\pgfsetlinewidth{0.000000pt}%
\definecolor{currentstroke}{rgb}{0.000000,0.000000,0.000000}%
\pgfsetstrokecolor{currentstroke}%
\pgfsetstrokeopacity{0.000000}%
\pgfsetdash{}{0pt}%
\pgfpathmoveto{\pgfqpoint{-319.224843in}{2.097945in}}%
\pgfpathlineto{\pgfqpoint{10.520609in}{2.097945in}}%
\pgfpathlineto{\pgfqpoint{10.520609in}{2.105033in}}%
\pgfpathlineto{\pgfqpoint{-319.224843in}{2.105033in}}%
\pgfpathclose%
\pgfusepath{fill}%
\end{pgfscope}%
\begin{pgfscope}%
\pgfpathrectangle{\pgfqpoint{10.668400in}{1.805660in}}{\pgfqpoint{2.188235in}{0.972632in}}%
\pgfusepath{clip}%
\pgfsetbuttcap%
\pgfsetmiterjoin%
\definecolor{currentfill}{rgb}{0.121569,0.466667,0.705882}%
\pgfsetfillcolor{currentfill}%
\pgfsetlinewidth{0.000000pt}%
\definecolor{currentstroke}{rgb}{0.000000,0.000000,0.000000}%
\pgfsetstrokecolor{currentstroke}%
\pgfsetstrokeopacity{0.000000}%
\pgfsetdash{}{0pt}%
\pgfpathmoveto{\pgfqpoint{-319.224843in}{2.106805in}}%
\pgfpathlineto{\pgfqpoint{10.721783in}{2.106805in}}%
\pgfpathlineto{\pgfqpoint{10.721783in}{2.113893in}}%
\pgfpathlineto{\pgfqpoint{-319.224843in}{2.113893in}}%
\pgfpathclose%
\pgfusepath{fill}%
\end{pgfscope}%
\begin{pgfscope}%
\pgfpathrectangle{\pgfqpoint{10.668400in}{1.805660in}}{\pgfqpoint{2.188235in}{0.972632in}}%
\pgfusepath{clip}%
\pgfsetbuttcap%
\pgfsetmiterjoin%
\definecolor{currentfill}{rgb}{0.121569,0.466667,0.705882}%
\pgfsetfillcolor{currentfill}%
\pgfsetlinewidth{0.000000pt}%
\definecolor{currentstroke}{rgb}{0.000000,0.000000,0.000000}%
\pgfsetstrokecolor{currentstroke}%
\pgfsetstrokeopacity{0.000000}%
\pgfsetdash{}{0pt}%
\pgfpathmoveto{\pgfqpoint{-319.224843in}{2.115665in}}%
\pgfpathlineto{\pgfqpoint{10.961430in}{2.115665in}}%
\pgfpathlineto{\pgfqpoint{10.961430in}{2.122753in}}%
\pgfpathlineto{\pgfqpoint{-319.224843in}{2.122753in}}%
\pgfpathclose%
\pgfusepath{fill}%
\end{pgfscope}%
\begin{pgfscope}%
\pgfpathrectangle{\pgfqpoint{10.668400in}{1.805660in}}{\pgfqpoint{2.188235in}{0.972632in}}%
\pgfusepath{clip}%
\pgfsetbuttcap%
\pgfsetmiterjoin%
\definecolor{currentfill}{rgb}{0.121569,0.466667,0.705882}%
\pgfsetfillcolor{currentfill}%
\pgfsetlinewidth{0.000000pt}%
\definecolor{currentstroke}{rgb}{0.000000,0.000000,0.000000}%
\pgfsetstrokecolor{currentstroke}%
\pgfsetstrokeopacity{0.000000}%
\pgfsetdash{}{0pt}%
\pgfpathmoveto{\pgfqpoint{-319.224843in}{2.124525in}}%
\pgfpathlineto{\pgfqpoint{10.909788in}{2.124525in}}%
\pgfpathlineto{\pgfqpoint{10.909788in}{2.131613in}}%
\pgfpathlineto{\pgfqpoint{-319.224843in}{2.131613in}}%
\pgfpathclose%
\pgfusepath{fill}%
\end{pgfscope}%
\begin{pgfscope}%
\pgfpathrectangle{\pgfqpoint{10.668400in}{1.805660in}}{\pgfqpoint{2.188235in}{0.972632in}}%
\pgfusepath{clip}%
\pgfsetbuttcap%
\pgfsetmiterjoin%
\definecolor{currentfill}{rgb}{0.121569,0.466667,0.705882}%
\pgfsetfillcolor{currentfill}%
\pgfsetlinewidth{0.000000pt}%
\definecolor{currentstroke}{rgb}{0.000000,0.000000,0.000000}%
\pgfsetstrokecolor{currentstroke}%
\pgfsetstrokeopacity{0.000000}%
\pgfsetdash{}{0pt}%
\pgfpathmoveto{\pgfqpoint{-319.224843in}{2.133385in}}%
\pgfpathlineto{\pgfqpoint{11.036442in}{2.133385in}}%
\pgfpathlineto{\pgfqpoint{11.036442in}{2.140473in}}%
\pgfpathlineto{\pgfqpoint{-319.224843in}{2.140473in}}%
\pgfpathclose%
\pgfusepath{fill}%
\end{pgfscope}%
\begin{pgfscope}%
\pgfpathrectangle{\pgfqpoint{10.668400in}{1.805660in}}{\pgfqpoint{2.188235in}{0.972632in}}%
\pgfusepath{clip}%
\pgfsetbuttcap%
\pgfsetmiterjoin%
\definecolor{currentfill}{rgb}{0.121569,0.466667,0.705882}%
\pgfsetfillcolor{currentfill}%
\pgfsetlinewidth{0.000000pt}%
\definecolor{currentstroke}{rgb}{0.000000,0.000000,0.000000}%
\pgfsetstrokecolor{currentstroke}%
\pgfsetstrokeopacity{0.000000}%
\pgfsetdash{}{0pt}%
\pgfpathmoveto{\pgfqpoint{-319.224843in}{2.142244in}}%
\pgfpathlineto{\pgfqpoint{10.876557in}{2.142244in}}%
\pgfpathlineto{\pgfqpoint{10.876557in}{2.149332in}}%
\pgfpathlineto{\pgfqpoint{-319.224843in}{2.149332in}}%
\pgfpathclose%
\pgfusepath{fill}%
\end{pgfscope}%
\begin{pgfscope}%
\pgfpathrectangle{\pgfqpoint{10.668400in}{1.805660in}}{\pgfqpoint{2.188235in}{0.972632in}}%
\pgfusepath{clip}%
\pgfsetbuttcap%
\pgfsetmiterjoin%
\definecolor{currentfill}{rgb}{0.121569,0.466667,0.705882}%
\pgfsetfillcolor{currentfill}%
\pgfsetlinewidth{0.000000pt}%
\definecolor{currentstroke}{rgb}{0.000000,0.000000,0.000000}%
\pgfsetstrokecolor{currentstroke}%
\pgfsetstrokeopacity{0.000000}%
\pgfsetdash{}{0pt}%
\pgfpathmoveto{\pgfqpoint{-319.224843in}{2.151104in}}%
\pgfpathlineto{\pgfqpoint{11.043845in}{2.151104in}}%
\pgfpathlineto{\pgfqpoint{11.043845in}{2.158192in}}%
\pgfpathlineto{\pgfqpoint{-319.224843in}{2.158192in}}%
\pgfpathclose%
\pgfusepath{fill}%
\end{pgfscope}%
\begin{pgfscope}%
\pgfpathrectangle{\pgfqpoint{10.668400in}{1.805660in}}{\pgfqpoint{2.188235in}{0.972632in}}%
\pgfusepath{clip}%
\pgfsetbuttcap%
\pgfsetmiterjoin%
\definecolor{currentfill}{rgb}{0.121569,0.466667,0.705882}%
\pgfsetfillcolor{currentfill}%
\pgfsetlinewidth{0.000000pt}%
\definecolor{currentstroke}{rgb}{0.000000,0.000000,0.000000}%
\pgfsetstrokecolor{currentstroke}%
\pgfsetstrokeopacity{0.000000}%
\pgfsetdash{}{0pt}%
\pgfpathmoveto{\pgfqpoint{-319.224843in}{2.159964in}}%
\pgfpathlineto{\pgfqpoint{10.784269in}{2.159964in}}%
\pgfpathlineto{\pgfqpoint{10.784269in}{2.167052in}}%
\pgfpathlineto{\pgfqpoint{-319.224843in}{2.167052in}}%
\pgfpathclose%
\pgfusepath{fill}%
\end{pgfscope}%
\begin{pgfscope}%
\pgfpathrectangle{\pgfqpoint{10.668400in}{1.805660in}}{\pgfqpoint{2.188235in}{0.972632in}}%
\pgfusepath{clip}%
\pgfsetbuttcap%
\pgfsetmiterjoin%
\definecolor{currentfill}{rgb}{0.121569,0.466667,0.705882}%
\pgfsetfillcolor{currentfill}%
\pgfsetlinewidth{0.000000pt}%
\definecolor{currentstroke}{rgb}{0.000000,0.000000,0.000000}%
\pgfsetstrokecolor{currentstroke}%
\pgfsetstrokeopacity{0.000000}%
\pgfsetdash{}{0pt}%
\pgfpathmoveto{\pgfqpoint{-319.224843in}{2.168824in}}%
\pgfpathlineto{\pgfqpoint{11.057117in}{2.168824in}}%
\pgfpathlineto{\pgfqpoint{11.057117in}{2.175912in}}%
\pgfpathlineto{\pgfqpoint{-319.224843in}{2.175912in}}%
\pgfpathclose%
\pgfusepath{fill}%
\end{pgfscope}%
\begin{pgfscope}%
\pgfpathrectangle{\pgfqpoint{10.668400in}{1.805660in}}{\pgfqpoint{2.188235in}{0.972632in}}%
\pgfusepath{clip}%
\pgfsetbuttcap%
\pgfsetmiterjoin%
\definecolor{currentfill}{rgb}{0.121569,0.466667,0.705882}%
\pgfsetfillcolor{currentfill}%
\pgfsetlinewidth{0.000000pt}%
\definecolor{currentstroke}{rgb}{0.000000,0.000000,0.000000}%
\pgfsetstrokecolor{currentstroke}%
\pgfsetstrokeopacity{0.000000}%
\pgfsetdash{}{0pt}%
\pgfpathmoveto{\pgfqpoint{-319.224843in}{2.177684in}}%
\pgfpathlineto{\pgfqpoint{10.956544in}{2.177684in}}%
\pgfpathlineto{\pgfqpoint{10.956544in}{2.184772in}}%
\pgfpathlineto{\pgfqpoint{-319.224843in}{2.184772in}}%
\pgfpathclose%
\pgfusepath{fill}%
\end{pgfscope}%
\begin{pgfscope}%
\pgfpathrectangle{\pgfqpoint{10.668400in}{1.805660in}}{\pgfqpoint{2.188235in}{0.972632in}}%
\pgfusepath{clip}%
\pgfsetbuttcap%
\pgfsetmiterjoin%
\definecolor{currentfill}{rgb}{0.121569,0.466667,0.705882}%
\pgfsetfillcolor{currentfill}%
\pgfsetlinewidth{0.000000pt}%
\definecolor{currentstroke}{rgb}{0.000000,0.000000,0.000000}%
\pgfsetstrokecolor{currentstroke}%
\pgfsetstrokeopacity{0.000000}%
\pgfsetdash{}{0pt}%
\pgfpathmoveto{\pgfqpoint{-319.224843in}{2.186544in}}%
\pgfpathlineto{\pgfqpoint{10.973474in}{2.186544in}}%
\pgfpathlineto{\pgfqpoint{10.973474in}{2.193631in}}%
\pgfpathlineto{\pgfqpoint{-319.224843in}{2.193631in}}%
\pgfpathclose%
\pgfusepath{fill}%
\end{pgfscope}%
\begin{pgfscope}%
\pgfpathrectangle{\pgfqpoint{10.668400in}{1.805660in}}{\pgfqpoint{2.188235in}{0.972632in}}%
\pgfusepath{clip}%
\pgfsetbuttcap%
\pgfsetmiterjoin%
\definecolor{currentfill}{rgb}{0.121569,0.466667,0.705882}%
\pgfsetfillcolor{currentfill}%
\pgfsetlinewidth{0.000000pt}%
\definecolor{currentstroke}{rgb}{0.000000,0.000000,0.000000}%
\pgfsetstrokecolor{currentstroke}%
\pgfsetstrokeopacity{0.000000}%
\pgfsetdash{}{0pt}%
\pgfpathmoveto{\pgfqpoint{-319.224843in}{2.195403in}}%
\pgfpathlineto{\pgfqpoint{10.937399in}{2.195403in}}%
\pgfpathlineto{\pgfqpoint{10.937399in}{2.202491in}}%
\pgfpathlineto{\pgfqpoint{-319.224843in}{2.202491in}}%
\pgfpathclose%
\pgfusepath{fill}%
\end{pgfscope}%
\begin{pgfscope}%
\pgfpathrectangle{\pgfqpoint{10.668400in}{1.805660in}}{\pgfqpoint{2.188235in}{0.972632in}}%
\pgfusepath{clip}%
\pgfsetbuttcap%
\pgfsetmiterjoin%
\definecolor{currentfill}{rgb}{0.121569,0.466667,0.705882}%
\pgfsetfillcolor{currentfill}%
\pgfsetlinewidth{0.000000pt}%
\definecolor{currentstroke}{rgb}{0.000000,0.000000,0.000000}%
\pgfsetstrokecolor{currentstroke}%
\pgfsetstrokeopacity{0.000000}%
\pgfsetdash{}{0pt}%
\pgfpathmoveto{\pgfqpoint{-319.224843in}{2.204263in}}%
\pgfpathlineto{\pgfqpoint{10.895319in}{2.204263in}}%
\pgfpathlineto{\pgfqpoint{10.895319in}{2.211351in}}%
\pgfpathlineto{\pgfqpoint{-319.224843in}{2.211351in}}%
\pgfpathclose%
\pgfusepath{fill}%
\end{pgfscope}%
\begin{pgfscope}%
\pgfpathrectangle{\pgfqpoint{10.668400in}{1.805660in}}{\pgfqpoint{2.188235in}{0.972632in}}%
\pgfusepath{clip}%
\pgfsetbuttcap%
\pgfsetmiterjoin%
\definecolor{currentfill}{rgb}{0.121569,0.466667,0.705882}%
\pgfsetfillcolor{currentfill}%
\pgfsetlinewidth{0.000000pt}%
\definecolor{currentstroke}{rgb}{0.000000,0.000000,0.000000}%
\pgfsetstrokecolor{currentstroke}%
\pgfsetstrokeopacity{0.000000}%
\pgfsetdash{}{0pt}%
\pgfpathmoveto{\pgfqpoint{-319.224843in}{2.213123in}}%
\pgfpathlineto{\pgfqpoint{11.005607in}{2.213123in}}%
\pgfpathlineto{\pgfqpoint{11.005607in}{2.220211in}}%
\pgfpathlineto{\pgfqpoint{-319.224843in}{2.220211in}}%
\pgfpathclose%
\pgfusepath{fill}%
\end{pgfscope}%
\begin{pgfscope}%
\pgfpathrectangle{\pgfqpoint{10.668400in}{1.805660in}}{\pgfqpoint{2.188235in}{0.972632in}}%
\pgfusepath{clip}%
\pgfsetbuttcap%
\pgfsetmiterjoin%
\definecolor{currentfill}{rgb}{0.121569,0.466667,0.705882}%
\pgfsetfillcolor{currentfill}%
\pgfsetlinewidth{0.000000pt}%
\definecolor{currentstroke}{rgb}{0.000000,0.000000,0.000000}%
\pgfsetstrokecolor{currentstroke}%
\pgfsetstrokeopacity{0.000000}%
\pgfsetdash{}{0pt}%
\pgfpathmoveto{\pgfqpoint{-319.224843in}{2.221983in}}%
\pgfpathlineto{\pgfqpoint{11.049086in}{2.221983in}}%
\pgfpathlineto{\pgfqpoint{11.049086in}{2.229071in}}%
\pgfpathlineto{\pgfqpoint{-319.224843in}{2.229071in}}%
\pgfpathclose%
\pgfusepath{fill}%
\end{pgfscope}%
\begin{pgfscope}%
\pgfpathrectangle{\pgfqpoint{10.668400in}{1.805660in}}{\pgfqpoint{2.188235in}{0.972632in}}%
\pgfusepath{clip}%
\pgfsetbuttcap%
\pgfsetmiterjoin%
\definecolor{currentfill}{rgb}{0.121569,0.466667,0.705882}%
\pgfsetfillcolor{currentfill}%
\pgfsetlinewidth{0.000000pt}%
\definecolor{currentstroke}{rgb}{0.000000,0.000000,0.000000}%
\pgfsetstrokecolor{currentstroke}%
\pgfsetstrokeopacity{0.000000}%
\pgfsetdash{}{0pt}%
\pgfpathmoveto{\pgfqpoint{-319.224843in}{2.230843in}}%
\pgfpathlineto{\pgfqpoint{10.976603in}{2.230843in}}%
\pgfpathlineto{\pgfqpoint{10.976603in}{2.237931in}}%
\pgfpathlineto{\pgfqpoint{-319.224843in}{2.237931in}}%
\pgfpathclose%
\pgfusepath{fill}%
\end{pgfscope}%
\begin{pgfscope}%
\pgfpathrectangle{\pgfqpoint{10.668400in}{1.805660in}}{\pgfqpoint{2.188235in}{0.972632in}}%
\pgfusepath{clip}%
\pgfsetbuttcap%
\pgfsetmiterjoin%
\definecolor{currentfill}{rgb}{0.121569,0.466667,0.705882}%
\pgfsetfillcolor{currentfill}%
\pgfsetlinewidth{0.000000pt}%
\definecolor{currentstroke}{rgb}{0.000000,0.000000,0.000000}%
\pgfsetstrokecolor{currentstroke}%
\pgfsetstrokeopacity{0.000000}%
\pgfsetdash{}{0pt}%
\pgfpathmoveto{\pgfqpoint{-319.224843in}{2.239703in}}%
\pgfpathlineto{\pgfqpoint{10.985386in}{2.239703in}}%
\pgfpathlineto{\pgfqpoint{10.985386in}{2.246790in}}%
\pgfpathlineto{\pgfqpoint{-319.224843in}{2.246790in}}%
\pgfpathclose%
\pgfusepath{fill}%
\end{pgfscope}%
\begin{pgfscope}%
\pgfpathrectangle{\pgfqpoint{10.668400in}{1.805660in}}{\pgfqpoint{2.188235in}{0.972632in}}%
\pgfusepath{clip}%
\pgfsetbuttcap%
\pgfsetmiterjoin%
\definecolor{currentfill}{rgb}{0.121569,0.466667,0.705882}%
\pgfsetfillcolor{currentfill}%
\pgfsetlinewidth{0.000000pt}%
\definecolor{currentstroke}{rgb}{0.000000,0.000000,0.000000}%
\pgfsetstrokecolor{currentstroke}%
\pgfsetstrokeopacity{0.000000}%
\pgfsetdash{}{0pt}%
\pgfpathmoveto{\pgfqpoint{-319.224843in}{2.248562in}}%
\pgfpathlineto{\pgfqpoint{10.792389in}{2.248562in}}%
\pgfpathlineto{\pgfqpoint{10.792389in}{2.255650in}}%
\pgfpathlineto{\pgfqpoint{-319.224843in}{2.255650in}}%
\pgfpathclose%
\pgfusepath{fill}%
\end{pgfscope}%
\begin{pgfscope}%
\pgfpathrectangle{\pgfqpoint{10.668400in}{1.805660in}}{\pgfqpoint{2.188235in}{0.972632in}}%
\pgfusepath{clip}%
\pgfsetbuttcap%
\pgfsetmiterjoin%
\definecolor{currentfill}{rgb}{0.121569,0.466667,0.705882}%
\pgfsetfillcolor{currentfill}%
\pgfsetlinewidth{0.000000pt}%
\definecolor{currentstroke}{rgb}{0.000000,0.000000,0.000000}%
\pgfsetstrokecolor{currentstroke}%
\pgfsetstrokeopacity{0.000000}%
\pgfsetdash{}{0pt}%
\pgfpathmoveto{\pgfqpoint{-319.224843in}{2.257422in}}%
\pgfpathlineto{\pgfqpoint{11.006607in}{2.257422in}}%
\pgfpathlineto{\pgfqpoint{11.006607in}{2.264510in}}%
\pgfpathlineto{\pgfqpoint{-319.224843in}{2.264510in}}%
\pgfpathclose%
\pgfusepath{fill}%
\end{pgfscope}%
\begin{pgfscope}%
\pgfpathrectangle{\pgfqpoint{10.668400in}{1.805660in}}{\pgfqpoint{2.188235in}{0.972632in}}%
\pgfusepath{clip}%
\pgfsetbuttcap%
\pgfsetmiterjoin%
\definecolor{currentfill}{rgb}{0.121569,0.466667,0.705882}%
\pgfsetfillcolor{currentfill}%
\pgfsetlinewidth{0.000000pt}%
\definecolor{currentstroke}{rgb}{0.000000,0.000000,0.000000}%
\pgfsetstrokecolor{currentstroke}%
\pgfsetstrokeopacity{0.000000}%
\pgfsetdash{}{0pt}%
\pgfpathmoveto{\pgfqpoint{-319.224843in}{2.266282in}}%
\pgfpathlineto{\pgfqpoint{10.848052in}{2.266282in}}%
\pgfpathlineto{\pgfqpoint{10.848052in}{2.273370in}}%
\pgfpathlineto{\pgfqpoint{-319.224843in}{2.273370in}}%
\pgfpathclose%
\pgfusepath{fill}%
\end{pgfscope}%
\begin{pgfscope}%
\pgfpathrectangle{\pgfqpoint{10.668400in}{1.805660in}}{\pgfqpoint{2.188235in}{0.972632in}}%
\pgfusepath{clip}%
\pgfsetbuttcap%
\pgfsetmiterjoin%
\definecolor{currentfill}{rgb}{0.121569,0.466667,0.705882}%
\pgfsetfillcolor{currentfill}%
\pgfsetlinewidth{0.000000pt}%
\definecolor{currentstroke}{rgb}{0.000000,0.000000,0.000000}%
\pgfsetstrokecolor{currentstroke}%
\pgfsetstrokeopacity{0.000000}%
\pgfsetdash{}{0pt}%
\pgfpathmoveto{\pgfqpoint{-319.224843in}{2.275142in}}%
\pgfpathlineto{\pgfqpoint{10.989920in}{2.275142in}}%
\pgfpathlineto{\pgfqpoint{10.989920in}{2.282230in}}%
\pgfpathlineto{\pgfqpoint{-319.224843in}{2.282230in}}%
\pgfpathclose%
\pgfusepath{fill}%
\end{pgfscope}%
\begin{pgfscope}%
\pgfpathrectangle{\pgfqpoint{10.668400in}{1.805660in}}{\pgfqpoint{2.188235in}{0.972632in}}%
\pgfusepath{clip}%
\pgfsetbuttcap%
\pgfsetmiterjoin%
\definecolor{currentfill}{rgb}{0.121569,0.466667,0.705882}%
\pgfsetfillcolor{currentfill}%
\pgfsetlinewidth{0.000000pt}%
\definecolor{currentstroke}{rgb}{0.000000,0.000000,0.000000}%
\pgfsetstrokecolor{currentstroke}%
\pgfsetstrokeopacity{0.000000}%
\pgfsetdash{}{0pt}%
\pgfpathmoveto{\pgfqpoint{-319.224843in}{2.284002in}}%
\pgfpathlineto{\pgfqpoint{10.896386in}{2.284002in}}%
\pgfpathlineto{\pgfqpoint{10.896386in}{2.291090in}}%
\pgfpathlineto{\pgfqpoint{-319.224843in}{2.291090in}}%
\pgfpathclose%
\pgfusepath{fill}%
\end{pgfscope}%
\begin{pgfscope}%
\pgfpathrectangle{\pgfqpoint{10.668400in}{1.805660in}}{\pgfqpoint{2.188235in}{0.972632in}}%
\pgfusepath{clip}%
\pgfsetbuttcap%
\pgfsetmiterjoin%
\definecolor{currentfill}{rgb}{0.121569,0.466667,0.705882}%
\pgfsetfillcolor{currentfill}%
\pgfsetlinewidth{0.000000pt}%
\definecolor{currentstroke}{rgb}{0.000000,0.000000,0.000000}%
\pgfsetstrokecolor{currentstroke}%
\pgfsetstrokeopacity{0.000000}%
\pgfsetdash{}{0pt}%
\pgfpathmoveto{\pgfqpoint{-319.224843in}{2.292862in}}%
\pgfpathlineto{\pgfqpoint{10.849513in}{2.292862in}}%
\pgfpathlineto{\pgfqpoint{10.849513in}{2.299949in}}%
\pgfpathlineto{\pgfqpoint{-319.224843in}{2.299949in}}%
\pgfpathclose%
\pgfusepath{fill}%
\end{pgfscope}%
\begin{pgfscope}%
\pgfpathrectangle{\pgfqpoint{10.668400in}{1.805660in}}{\pgfqpoint{2.188235in}{0.972632in}}%
\pgfusepath{clip}%
\pgfsetbuttcap%
\pgfsetmiterjoin%
\definecolor{currentfill}{rgb}{0.121569,0.466667,0.705882}%
\pgfsetfillcolor{currentfill}%
\pgfsetlinewidth{0.000000pt}%
\definecolor{currentstroke}{rgb}{0.000000,0.000000,0.000000}%
\pgfsetstrokecolor{currentstroke}%
\pgfsetstrokeopacity{0.000000}%
\pgfsetdash{}{0pt}%
\pgfpathmoveto{\pgfqpoint{-319.224843in}{2.301721in}}%
\pgfpathlineto{\pgfqpoint{10.835033in}{2.301721in}}%
\pgfpathlineto{\pgfqpoint{10.835033in}{2.308809in}}%
\pgfpathlineto{\pgfqpoint{-319.224843in}{2.308809in}}%
\pgfpathclose%
\pgfusepath{fill}%
\end{pgfscope}%
\begin{pgfscope}%
\pgfpathrectangle{\pgfqpoint{10.668400in}{1.805660in}}{\pgfqpoint{2.188235in}{0.972632in}}%
\pgfusepath{clip}%
\pgfsetbuttcap%
\pgfsetmiterjoin%
\definecolor{currentfill}{rgb}{0.121569,0.466667,0.705882}%
\pgfsetfillcolor{currentfill}%
\pgfsetlinewidth{0.000000pt}%
\definecolor{currentstroke}{rgb}{0.000000,0.000000,0.000000}%
\pgfsetstrokecolor{currentstroke}%
\pgfsetstrokeopacity{0.000000}%
\pgfsetdash{}{0pt}%
\pgfpathmoveto{\pgfqpoint{-319.224843in}{2.310581in}}%
\pgfpathlineto{\pgfqpoint{10.818102in}{2.310581in}}%
\pgfpathlineto{\pgfqpoint{10.818102in}{2.317669in}}%
\pgfpathlineto{\pgfqpoint{-319.224843in}{2.317669in}}%
\pgfpathclose%
\pgfusepath{fill}%
\end{pgfscope}%
\begin{pgfscope}%
\pgfpathrectangle{\pgfqpoint{10.668400in}{1.805660in}}{\pgfqpoint{2.188235in}{0.972632in}}%
\pgfusepath{clip}%
\pgfsetbuttcap%
\pgfsetmiterjoin%
\definecolor{currentfill}{rgb}{0.121569,0.466667,0.705882}%
\pgfsetfillcolor{currentfill}%
\pgfsetlinewidth{0.000000pt}%
\definecolor{currentstroke}{rgb}{0.000000,0.000000,0.000000}%
\pgfsetstrokecolor{currentstroke}%
\pgfsetstrokeopacity{0.000000}%
\pgfsetdash{}{0pt}%
\pgfpathmoveto{\pgfqpoint{-319.224843in}{2.319441in}}%
\pgfpathlineto{\pgfqpoint{10.870693in}{2.319441in}}%
\pgfpathlineto{\pgfqpoint{10.870693in}{2.326529in}}%
\pgfpathlineto{\pgfqpoint{-319.224843in}{2.326529in}}%
\pgfpathclose%
\pgfusepath{fill}%
\end{pgfscope}%
\begin{pgfscope}%
\pgfpathrectangle{\pgfqpoint{10.668400in}{1.805660in}}{\pgfqpoint{2.188235in}{0.972632in}}%
\pgfusepath{clip}%
\pgfsetbuttcap%
\pgfsetmiterjoin%
\definecolor{currentfill}{rgb}{0.121569,0.466667,0.705882}%
\pgfsetfillcolor{currentfill}%
\pgfsetlinewidth{0.000000pt}%
\definecolor{currentstroke}{rgb}{0.000000,0.000000,0.000000}%
\pgfsetstrokecolor{currentstroke}%
\pgfsetstrokeopacity{0.000000}%
\pgfsetdash{}{0pt}%
\pgfpathmoveto{\pgfqpoint{-319.224843in}{2.328301in}}%
\pgfpathlineto{\pgfqpoint{10.891618in}{2.328301in}}%
\pgfpathlineto{\pgfqpoint{10.891618in}{2.335389in}}%
\pgfpathlineto{\pgfqpoint{-319.224843in}{2.335389in}}%
\pgfpathclose%
\pgfusepath{fill}%
\end{pgfscope}%
\begin{pgfscope}%
\pgfpathrectangle{\pgfqpoint{10.668400in}{1.805660in}}{\pgfqpoint{2.188235in}{0.972632in}}%
\pgfusepath{clip}%
\pgfsetbuttcap%
\pgfsetmiterjoin%
\definecolor{currentfill}{rgb}{0.121569,0.466667,0.705882}%
\pgfsetfillcolor{currentfill}%
\pgfsetlinewidth{0.000000pt}%
\definecolor{currentstroke}{rgb}{0.000000,0.000000,0.000000}%
\pgfsetstrokecolor{currentstroke}%
\pgfsetstrokeopacity{0.000000}%
\pgfsetdash{}{0pt}%
\pgfpathmoveto{\pgfqpoint{-319.224843in}{2.337161in}}%
\pgfpathlineto{\pgfqpoint{10.631425in}{2.337161in}}%
\pgfpathlineto{\pgfqpoint{10.631425in}{2.344249in}}%
\pgfpathlineto{\pgfqpoint{-319.224843in}{2.344249in}}%
\pgfpathclose%
\pgfusepath{fill}%
\end{pgfscope}%
\begin{pgfscope}%
\pgfpathrectangle{\pgfqpoint{10.668400in}{1.805660in}}{\pgfqpoint{2.188235in}{0.972632in}}%
\pgfusepath{clip}%
\pgfsetbuttcap%
\pgfsetmiterjoin%
\definecolor{currentfill}{rgb}{0.121569,0.466667,0.705882}%
\pgfsetfillcolor{currentfill}%
\pgfsetlinewidth{0.000000pt}%
\definecolor{currentstroke}{rgb}{0.000000,0.000000,0.000000}%
\pgfsetstrokecolor{currentstroke}%
\pgfsetstrokeopacity{0.000000}%
\pgfsetdash{}{0pt}%
\pgfpathmoveto{\pgfqpoint{-319.224843in}{2.346020in}}%
\pgfpathlineto{\pgfqpoint{9.717725in}{2.346020in}}%
\pgfpathlineto{\pgfqpoint{9.717725in}{2.353108in}}%
\pgfpathlineto{\pgfqpoint{-319.224843in}{2.353108in}}%
\pgfpathclose%
\pgfusepath{fill}%
\end{pgfscope}%
\begin{pgfscope}%
\pgfpathrectangle{\pgfqpoint{10.668400in}{1.805660in}}{\pgfqpoint{2.188235in}{0.972632in}}%
\pgfusepath{clip}%
\pgfsetbuttcap%
\pgfsetmiterjoin%
\definecolor{currentfill}{rgb}{0.121569,0.466667,0.705882}%
\pgfsetfillcolor{currentfill}%
\pgfsetlinewidth{0.000000pt}%
\definecolor{currentstroke}{rgb}{0.000000,0.000000,0.000000}%
\pgfsetstrokecolor{currentstroke}%
\pgfsetstrokeopacity{0.000000}%
\pgfsetdash{}{0pt}%
\pgfpathmoveto{\pgfqpoint{-319.224843in}{2.354880in}}%
\pgfpathlineto{\pgfqpoint{10.720501in}{2.354880in}}%
\pgfpathlineto{\pgfqpoint{10.720501in}{2.361968in}}%
\pgfpathlineto{\pgfqpoint{-319.224843in}{2.361968in}}%
\pgfpathclose%
\pgfusepath{fill}%
\end{pgfscope}%
\begin{pgfscope}%
\pgfpathrectangle{\pgfqpoint{10.668400in}{1.805660in}}{\pgfqpoint{2.188235in}{0.972632in}}%
\pgfusepath{clip}%
\pgfsetbuttcap%
\pgfsetmiterjoin%
\definecolor{currentfill}{rgb}{0.121569,0.466667,0.705882}%
\pgfsetfillcolor{currentfill}%
\pgfsetlinewidth{0.000000pt}%
\definecolor{currentstroke}{rgb}{0.000000,0.000000,0.000000}%
\pgfsetstrokecolor{currentstroke}%
\pgfsetstrokeopacity{0.000000}%
\pgfsetdash{}{0pt}%
\pgfpathmoveto{\pgfqpoint{-319.224843in}{2.363740in}}%
\pgfpathlineto{\pgfqpoint{10.430936in}{2.363740in}}%
\pgfpathlineto{\pgfqpoint{10.430936in}{2.370828in}}%
\pgfpathlineto{\pgfqpoint{-319.224843in}{2.370828in}}%
\pgfpathclose%
\pgfusepath{fill}%
\end{pgfscope}%
\begin{pgfscope}%
\pgfpathrectangle{\pgfqpoint{10.668400in}{1.805660in}}{\pgfqpoint{2.188235in}{0.972632in}}%
\pgfusepath{clip}%
\pgfsetbuttcap%
\pgfsetmiterjoin%
\definecolor{currentfill}{rgb}{0.121569,0.466667,0.705882}%
\pgfsetfillcolor{currentfill}%
\pgfsetlinewidth{0.000000pt}%
\definecolor{currentstroke}{rgb}{0.000000,0.000000,0.000000}%
\pgfsetstrokecolor{currentstroke}%
\pgfsetstrokeopacity{0.000000}%
\pgfsetdash{}{0pt}%
\pgfpathmoveto{\pgfqpoint{-319.224843in}{2.372600in}}%
\pgfpathlineto{\pgfqpoint{10.166259in}{2.372600in}}%
\pgfpathlineto{\pgfqpoint{10.166259in}{2.379688in}}%
\pgfpathlineto{\pgfqpoint{-319.224843in}{2.379688in}}%
\pgfpathclose%
\pgfusepath{fill}%
\end{pgfscope}%
\begin{pgfscope}%
\pgfpathrectangle{\pgfqpoint{10.668400in}{1.805660in}}{\pgfqpoint{2.188235in}{0.972632in}}%
\pgfusepath{clip}%
\pgfsetbuttcap%
\pgfsetmiterjoin%
\definecolor{currentfill}{rgb}{0.121569,0.466667,0.705882}%
\pgfsetfillcolor{currentfill}%
\pgfsetlinewidth{0.000000pt}%
\definecolor{currentstroke}{rgb}{0.000000,0.000000,0.000000}%
\pgfsetstrokecolor{currentstroke}%
\pgfsetstrokeopacity{0.000000}%
\pgfsetdash{}{0pt}%
\pgfpathmoveto{\pgfqpoint{-319.224843in}{2.381460in}}%
\pgfpathlineto{\pgfqpoint{10.931779in}{2.381460in}}%
\pgfpathlineto{\pgfqpoint{10.931779in}{2.388548in}}%
\pgfpathlineto{\pgfqpoint{-319.224843in}{2.388548in}}%
\pgfpathclose%
\pgfusepath{fill}%
\end{pgfscope}%
\begin{pgfscope}%
\pgfpathrectangle{\pgfqpoint{10.668400in}{1.805660in}}{\pgfqpoint{2.188235in}{0.972632in}}%
\pgfusepath{clip}%
\pgfsetbuttcap%
\pgfsetmiterjoin%
\definecolor{currentfill}{rgb}{0.121569,0.466667,0.705882}%
\pgfsetfillcolor{currentfill}%
\pgfsetlinewidth{0.000000pt}%
\definecolor{currentstroke}{rgb}{0.000000,0.000000,0.000000}%
\pgfsetstrokecolor{currentstroke}%
\pgfsetstrokeopacity{0.000000}%
\pgfsetdash{}{0pt}%
\pgfpathmoveto{\pgfqpoint{-319.224843in}{2.390320in}}%
\pgfpathlineto{\pgfqpoint{10.809520in}{2.390320in}}%
\pgfpathlineto{\pgfqpoint{10.809520in}{2.397407in}}%
\pgfpathlineto{\pgfqpoint{-319.224843in}{2.397407in}}%
\pgfpathclose%
\pgfusepath{fill}%
\end{pgfscope}%
\begin{pgfscope}%
\pgfpathrectangle{\pgfqpoint{10.668400in}{1.805660in}}{\pgfqpoint{2.188235in}{0.972632in}}%
\pgfusepath{clip}%
\pgfsetbuttcap%
\pgfsetmiterjoin%
\definecolor{currentfill}{rgb}{0.121569,0.466667,0.705882}%
\pgfsetfillcolor{currentfill}%
\pgfsetlinewidth{0.000000pt}%
\definecolor{currentstroke}{rgb}{0.000000,0.000000,0.000000}%
\pgfsetstrokecolor{currentstroke}%
\pgfsetstrokeopacity{0.000000}%
\pgfsetdash{}{0pt}%
\pgfpathmoveto{\pgfqpoint{-319.224843in}{2.399179in}}%
\pgfpathlineto{\pgfqpoint{10.831905in}{2.399179in}}%
\pgfpathlineto{\pgfqpoint{10.831905in}{2.406267in}}%
\pgfpathlineto{\pgfqpoint{-319.224843in}{2.406267in}}%
\pgfpathclose%
\pgfusepath{fill}%
\end{pgfscope}%
\begin{pgfscope}%
\pgfpathrectangle{\pgfqpoint{10.668400in}{1.805660in}}{\pgfqpoint{2.188235in}{0.972632in}}%
\pgfusepath{clip}%
\pgfsetbuttcap%
\pgfsetmiterjoin%
\definecolor{currentfill}{rgb}{0.121569,0.466667,0.705882}%
\pgfsetfillcolor{currentfill}%
\pgfsetlinewidth{0.000000pt}%
\definecolor{currentstroke}{rgb}{0.000000,0.000000,0.000000}%
\pgfsetstrokecolor{currentstroke}%
\pgfsetstrokeopacity{0.000000}%
\pgfsetdash{}{0pt}%
\pgfpathmoveto{\pgfqpoint{-319.224843in}{2.408039in}}%
\pgfpathlineto{\pgfqpoint{10.569678in}{2.408039in}}%
\pgfpathlineto{\pgfqpoint{10.569678in}{2.415127in}}%
\pgfpathlineto{\pgfqpoint{-319.224843in}{2.415127in}}%
\pgfpathclose%
\pgfusepath{fill}%
\end{pgfscope}%
\begin{pgfscope}%
\pgfpathrectangle{\pgfqpoint{10.668400in}{1.805660in}}{\pgfqpoint{2.188235in}{0.972632in}}%
\pgfusepath{clip}%
\pgfsetbuttcap%
\pgfsetmiterjoin%
\definecolor{currentfill}{rgb}{0.121569,0.466667,0.705882}%
\pgfsetfillcolor{currentfill}%
\pgfsetlinewidth{0.000000pt}%
\definecolor{currentstroke}{rgb}{0.000000,0.000000,0.000000}%
\pgfsetstrokecolor{currentstroke}%
\pgfsetstrokeopacity{0.000000}%
\pgfsetdash{}{0pt}%
\pgfpathmoveto{\pgfqpoint{-319.224843in}{2.416899in}}%
\pgfpathlineto{\pgfqpoint{10.812600in}{2.416899in}}%
\pgfpathlineto{\pgfqpoint{10.812600in}{2.423987in}}%
\pgfpathlineto{\pgfqpoint{-319.224843in}{2.423987in}}%
\pgfpathclose%
\pgfusepath{fill}%
\end{pgfscope}%
\begin{pgfscope}%
\pgfpathrectangle{\pgfqpoint{10.668400in}{1.805660in}}{\pgfqpoint{2.188235in}{0.972632in}}%
\pgfusepath{clip}%
\pgfsetbuttcap%
\pgfsetmiterjoin%
\definecolor{currentfill}{rgb}{0.121569,0.466667,0.705882}%
\pgfsetfillcolor{currentfill}%
\pgfsetlinewidth{0.000000pt}%
\definecolor{currentstroke}{rgb}{0.000000,0.000000,0.000000}%
\pgfsetstrokecolor{currentstroke}%
\pgfsetstrokeopacity{0.000000}%
\pgfsetdash{}{0pt}%
\pgfpathmoveto{\pgfqpoint{-319.224843in}{2.425759in}}%
\pgfpathlineto{\pgfqpoint{10.621952in}{2.425759in}}%
\pgfpathlineto{\pgfqpoint{10.621952in}{2.432847in}}%
\pgfpathlineto{\pgfqpoint{-319.224843in}{2.432847in}}%
\pgfpathclose%
\pgfusepath{fill}%
\end{pgfscope}%
\begin{pgfscope}%
\pgfpathrectangle{\pgfqpoint{10.668400in}{1.805660in}}{\pgfqpoint{2.188235in}{0.972632in}}%
\pgfusepath{clip}%
\pgfsetbuttcap%
\pgfsetmiterjoin%
\definecolor{currentfill}{rgb}{0.121569,0.466667,0.705882}%
\pgfsetfillcolor{currentfill}%
\pgfsetlinewidth{0.000000pt}%
\definecolor{currentstroke}{rgb}{0.000000,0.000000,0.000000}%
\pgfsetstrokecolor{currentstroke}%
\pgfsetstrokeopacity{0.000000}%
\pgfsetdash{}{0pt}%
\pgfpathmoveto{\pgfqpoint{-319.224843in}{2.434619in}}%
\pgfpathlineto{\pgfqpoint{10.725870in}{2.434619in}}%
\pgfpathlineto{\pgfqpoint{10.725870in}{2.441707in}}%
\pgfpathlineto{\pgfqpoint{-319.224843in}{2.441707in}}%
\pgfpathclose%
\pgfusepath{fill}%
\end{pgfscope}%
\begin{pgfscope}%
\pgfpathrectangle{\pgfqpoint{10.668400in}{1.805660in}}{\pgfqpoint{2.188235in}{0.972632in}}%
\pgfusepath{clip}%
\pgfsetbuttcap%
\pgfsetmiterjoin%
\definecolor{currentfill}{rgb}{0.121569,0.466667,0.705882}%
\pgfsetfillcolor{currentfill}%
\pgfsetlinewidth{0.000000pt}%
\definecolor{currentstroke}{rgb}{0.000000,0.000000,0.000000}%
\pgfsetstrokecolor{currentstroke}%
\pgfsetstrokeopacity{0.000000}%
\pgfsetdash{}{0pt}%
\pgfpathmoveto{\pgfqpoint{-319.224843in}{2.443479in}}%
\pgfpathlineto{\pgfqpoint{10.329196in}{2.443479in}}%
\pgfpathlineto{\pgfqpoint{10.329196in}{2.450566in}}%
\pgfpathlineto{\pgfqpoint{-319.224843in}{2.450566in}}%
\pgfpathclose%
\pgfusepath{fill}%
\end{pgfscope}%
\begin{pgfscope}%
\pgfpathrectangle{\pgfqpoint{10.668400in}{1.805660in}}{\pgfqpoint{2.188235in}{0.972632in}}%
\pgfusepath{clip}%
\pgfsetbuttcap%
\pgfsetmiterjoin%
\definecolor{currentfill}{rgb}{0.121569,0.466667,0.705882}%
\pgfsetfillcolor{currentfill}%
\pgfsetlinewidth{0.000000pt}%
\definecolor{currentstroke}{rgb}{0.000000,0.000000,0.000000}%
\pgfsetstrokecolor{currentstroke}%
\pgfsetstrokeopacity{0.000000}%
\pgfsetdash{}{0pt}%
\pgfpathmoveto{\pgfqpoint{-319.224843in}{2.452338in}}%
\pgfpathlineto{\pgfqpoint{10.338696in}{2.452338in}}%
\pgfpathlineto{\pgfqpoint{10.338696in}{2.459426in}}%
\pgfpathlineto{\pgfqpoint{-319.224843in}{2.459426in}}%
\pgfpathclose%
\pgfusepath{fill}%
\end{pgfscope}%
\begin{pgfscope}%
\pgfpathrectangle{\pgfqpoint{10.668400in}{1.805660in}}{\pgfqpoint{2.188235in}{0.972632in}}%
\pgfusepath{clip}%
\pgfsetbuttcap%
\pgfsetmiterjoin%
\definecolor{currentfill}{rgb}{0.121569,0.466667,0.705882}%
\pgfsetfillcolor{currentfill}%
\pgfsetlinewidth{0.000000pt}%
\definecolor{currentstroke}{rgb}{0.000000,0.000000,0.000000}%
\pgfsetstrokecolor{currentstroke}%
\pgfsetstrokeopacity{0.000000}%
\pgfsetdash{}{0pt}%
\pgfpathmoveto{\pgfqpoint{-319.224843in}{2.461198in}}%
\pgfpathlineto{\pgfqpoint{10.858213in}{2.461198in}}%
\pgfpathlineto{\pgfqpoint{10.858213in}{2.468286in}}%
\pgfpathlineto{\pgfqpoint{-319.224843in}{2.468286in}}%
\pgfpathclose%
\pgfusepath{fill}%
\end{pgfscope}%
\begin{pgfscope}%
\pgfpathrectangle{\pgfqpoint{10.668400in}{1.805660in}}{\pgfqpoint{2.188235in}{0.972632in}}%
\pgfusepath{clip}%
\pgfsetbuttcap%
\pgfsetmiterjoin%
\definecolor{currentfill}{rgb}{0.121569,0.466667,0.705882}%
\pgfsetfillcolor{currentfill}%
\pgfsetlinewidth{0.000000pt}%
\definecolor{currentstroke}{rgb}{0.000000,0.000000,0.000000}%
\pgfsetstrokecolor{currentstroke}%
\pgfsetstrokeopacity{0.000000}%
\pgfsetdash{}{0pt}%
\pgfpathmoveto{\pgfqpoint{-319.224843in}{2.470058in}}%
\pgfpathlineto{\pgfqpoint{10.792032in}{2.470058in}}%
\pgfpathlineto{\pgfqpoint{10.792032in}{2.477146in}}%
\pgfpathlineto{\pgfqpoint{-319.224843in}{2.477146in}}%
\pgfpathclose%
\pgfusepath{fill}%
\end{pgfscope}%
\begin{pgfscope}%
\pgfpathrectangle{\pgfqpoint{10.668400in}{1.805660in}}{\pgfqpoint{2.188235in}{0.972632in}}%
\pgfusepath{clip}%
\pgfsetbuttcap%
\pgfsetmiterjoin%
\definecolor{currentfill}{rgb}{0.121569,0.466667,0.705882}%
\pgfsetfillcolor{currentfill}%
\pgfsetlinewidth{0.000000pt}%
\definecolor{currentstroke}{rgb}{0.000000,0.000000,0.000000}%
\pgfsetstrokecolor{currentstroke}%
\pgfsetstrokeopacity{0.000000}%
\pgfsetdash{}{0pt}%
\pgfpathmoveto{\pgfqpoint{-319.224843in}{2.478918in}}%
\pgfpathlineto{\pgfqpoint{10.522374in}{2.478918in}}%
\pgfpathlineto{\pgfqpoint{10.522374in}{2.486006in}}%
\pgfpathlineto{\pgfqpoint{-319.224843in}{2.486006in}}%
\pgfpathclose%
\pgfusepath{fill}%
\end{pgfscope}%
\begin{pgfscope}%
\pgfpathrectangle{\pgfqpoint{10.668400in}{1.805660in}}{\pgfqpoint{2.188235in}{0.972632in}}%
\pgfusepath{clip}%
\pgfsetbuttcap%
\pgfsetmiterjoin%
\definecolor{currentfill}{rgb}{0.121569,0.466667,0.705882}%
\pgfsetfillcolor{currentfill}%
\pgfsetlinewidth{0.000000pt}%
\definecolor{currentstroke}{rgb}{0.000000,0.000000,0.000000}%
\pgfsetstrokecolor{currentstroke}%
\pgfsetstrokeopacity{0.000000}%
\pgfsetdash{}{0pt}%
\pgfpathmoveto{\pgfqpoint{-319.224843in}{2.487778in}}%
\pgfpathlineto{\pgfqpoint{10.676465in}{2.487778in}}%
\pgfpathlineto{\pgfqpoint{10.676465in}{2.494866in}}%
\pgfpathlineto{\pgfqpoint{-319.224843in}{2.494866in}}%
\pgfpathclose%
\pgfusepath{fill}%
\end{pgfscope}%
\begin{pgfscope}%
\pgfpathrectangle{\pgfqpoint{10.668400in}{1.805660in}}{\pgfqpoint{2.188235in}{0.972632in}}%
\pgfusepath{clip}%
\pgfsetbuttcap%
\pgfsetmiterjoin%
\definecolor{currentfill}{rgb}{0.121569,0.466667,0.705882}%
\pgfsetfillcolor{currentfill}%
\pgfsetlinewidth{0.000000pt}%
\definecolor{currentstroke}{rgb}{0.000000,0.000000,0.000000}%
\pgfsetstrokecolor{currentstroke}%
\pgfsetstrokeopacity{0.000000}%
\pgfsetdash{}{0pt}%
\pgfpathmoveto{\pgfqpoint{-319.224843in}{2.496637in}}%
\pgfpathlineto{\pgfqpoint{10.571642in}{2.496637in}}%
\pgfpathlineto{\pgfqpoint{10.571642in}{2.503725in}}%
\pgfpathlineto{\pgfqpoint{-319.224843in}{2.503725in}}%
\pgfpathclose%
\pgfusepath{fill}%
\end{pgfscope}%
\begin{pgfscope}%
\pgfpathrectangle{\pgfqpoint{10.668400in}{1.805660in}}{\pgfqpoint{2.188235in}{0.972632in}}%
\pgfusepath{clip}%
\pgfsetbuttcap%
\pgfsetmiterjoin%
\definecolor{currentfill}{rgb}{0.121569,0.466667,0.705882}%
\pgfsetfillcolor{currentfill}%
\pgfsetlinewidth{0.000000pt}%
\definecolor{currentstroke}{rgb}{0.000000,0.000000,0.000000}%
\pgfsetstrokecolor{currentstroke}%
\pgfsetstrokeopacity{0.000000}%
\pgfsetdash{}{0pt}%
\pgfpathmoveto{\pgfqpoint{-319.224843in}{2.505497in}}%
\pgfpathlineto{\pgfqpoint{10.600391in}{2.505497in}}%
\pgfpathlineto{\pgfqpoint{10.600391in}{2.512585in}}%
\pgfpathlineto{\pgfqpoint{-319.224843in}{2.512585in}}%
\pgfpathclose%
\pgfusepath{fill}%
\end{pgfscope}%
\begin{pgfscope}%
\pgfpathrectangle{\pgfqpoint{10.668400in}{1.805660in}}{\pgfqpoint{2.188235in}{0.972632in}}%
\pgfusepath{clip}%
\pgfsetbuttcap%
\pgfsetmiterjoin%
\definecolor{currentfill}{rgb}{0.121569,0.466667,0.705882}%
\pgfsetfillcolor{currentfill}%
\pgfsetlinewidth{0.000000pt}%
\definecolor{currentstroke}{rgb}{0.000000,0.000000,0.000000}%
\pgfsetstrokecolor{currentstroke}%
\pgfsetstrokeopacity{0.000000}%
\pgfsetdash{}{0pt}%
\pgfpathmoveto{\pgfqpoint{-319.224843in}{2.514357in}}%
\pgfpathlineto{\pgfqpoint{10.513973in}{2.514357in}}%
\pgfpathlineto{\pgfqpoint{10.513973in}{2.521445in}}%
\pgfpathlineto{\pgfqpoint{-319.224843in}{2.521445in}}%
\pgfpathclose%
\pgfusepath{fill}%
\end{pgfscope}%
\begin{pgfscope}%
\pgfpathrectangle{\pgfqpoint{10.668400in}{1.805660in}}{\pgfqpoint{2.188235in}{0.972632in}}%
\pgfusepath{clip}%
\pgfsetbuttcap%
\pgfsetmiterjoin%
\definecolor{currentfill}{rgb}{0.121569,0.466667,0.705882}%
\pgfsetfillcolor{currentfill}%
\pgfsetlinewidth{0.000000pt}%
\definecolor{currentstroke}{rgb}{0.000000,0.000000,0.000000}%
\pgfsetstrokecolor{currentstroke}%
\pgfsetstrokeopacity{0.000000}%
\pgfsetdash{}{0pt}%
\pgfpathmoveto{\pgfqpoint{-319.224843in}{2.523217in}}%
\pgfpathlineto{\pgfqpoint{10.943953in}{2.523217in}}%
\pgfpathlineto{\pgfqpoint{10.943953in}{2.530305in}}%
\pgfpathlineto{\pgfqpoint{-319.224843in}{2.530305in}}%
\pgfpathclose%
\pgfusepath{fill}%
\end{pgfscope}%
\begin{pgfscope}%
\pgfpathrectangle{\pgfqpoint{10.668400in}{1.805660in}}{\pgfqpoint{2.188235in}{0.972632in}}%
\pgfusepath{clip}%
\pgfsetbuttcap%
\pgfsetmiterjoin%
\definecolor{currentfill}{rgb}{0.121569,0.466667,0.705882}%
\pgfsetfillcolor{currentfill}%
\pgfsetlinewidth{0.000000pt}%
\definecolor{currentstroke}{rgb}{0.000000,0.000000,0.000000}%
\pgfsetstrokecolor{currentstroke}%
\pgfsetstrokeopacity{0.000000}%
\pgfsetdash{}{0pt}%
\pgfpathmoveto{\pgfqpoint{-319.224843in}{2.532077in}}%
\pgfpathlineto{\pgfqpoint{10.648881in}{2.532077in}}%
\pgfpathlineto{\pgfqpoint{10.648881in}{2.539165in}}%
\pgfpathlineto{\pgfqpoint{-319.224843in}{2.539165in}}%
\pgfpathclose%
\pgfusepath{fill}%
\end{pgfscope}%
\begin{pgfscope}%
\pgfpathrectangle{\pgfqpoint{10.668400in}{1.805660in}}{\pgfqpoint{2.188235in}{0.972632in}}%
\pgfusepath{clip}%
\pgfsetbuttcap%
\pgfsetmiterjoin%
\definecolor{currentfill}{rgb}{0.121569,0.466667,0.705882}%
\pgfsetfillcolor{currentfill}%
\pgfsetlinewidth{0.000000pt}%
\definecolor{currentstroke}{rgb}{0.000000,0.000000,0.000000}%
\pgfsetstrokecolor{currentstroke}%
\pgfsetstrokeopacity{0.000000}%
\pgfsetdash{}{0pt}%
\pgfpathmoveto{\pgfqpoint{-319.224843in}{2.540937in}}%
\pgfpathlineto{\pgfqpoint{10.343806in}{2.540937in}}%
\pgfpathlineto{\pgfqpoint{10.343806in}{2.548024in}}%
\pgfpathlineto{\pgfqpoint{-319.224843in}{2.548024in}}%
\pgfpathclose%
\pgfusepath{fill}%
\end{pgfscope}%
\begin{pgfscope}%
\pgfpathrectangle{\pgfqpoint{10.668400in}{1.805660in}}{\pgfqpoint{2.188235in}{0.972632in}}%
\pgfusepath{clip}%
\pgfsetbuttcap%
\pgfsetmiterjoin%
\definecolor{currentfill}{rgb}{0.121569,0.466667,0.705882}%
\pgfsetfillcolor{currentfill}%
\pgfsetlinewidth{0.000000pt}%
\definecolor{currentstroke}{rgb}{0.000000,0.000000,0.000000}%
\pgfsetstrokecolor{currentstroke}%
\pgfsetstrokeopacity{0.000000}%
\pgfsetdash{}{0pt}%
\pgfpathmoveto{\pgfqpoint{-319.224843in}{2.549796in}}%
\pgfpathlineto{\pgfqpoint{10.536788in}{2.549796in}}%
\pgfpathlineto{\pgfqpoint{10.536788in}{2.556884in}}%
\pgfpathlineto{\pgfqpoint{-319.224843in}{2.556884in}}%
\pgfpathclose%
\pgfusepath{fill}%
\end{pgfscope}%
\begin{pgfscope}%
\pgfpathrectangle{\pgfqpoint{10.668400in}{1.805660in}}{\pgfqpoint{2.188235in}{0.972632in}}%
\pgfusepath{clip}%
\pgfsetbuttcap%
\pgfsetmiterjoin%
\definecolor{currentfill}{rgb}{0.121569,0.466667,0.705882}%
\pgfsetfillcolor{currentfill}%
\pgfsetlinewidth{0.000000pt}%
\definecolor{currentstroke}{rgb}{0.000000,0.000000,0.000000}%
\pgfsetstrokecolor{currentstroke}%
\pgfsetstrokeopacity{0.000000}%
\pgfsetdash{}{0pt}%
\pgfpathmoveto{\pgfqpoint{-319.224843in}{2.558656in}}%
\pgfpathlineto{\pgfqpoint{10.906573in}{2.558656in}}%
\pgfpathlineto{\pgfqpoint{10.906573in}{2.565744in}}%
\pgfpathlineto{\pgfqpoint{-319.224843in}{2.565744in}}%
\pgfpathclose%
\pgfusepath{fill}%
\end{pgfscope}%
\begin{pgfscope}%
\pgfpathrectangle{\pgfqpoint{10.668400in}{1.805660in}}{\pgfqpoint{2.188235in}{0.972632in}}%
\pgfusepath{clip}%
\pgfsetbuttcap%
\pgfsetmiterjoin%
\definecolor{currentfill}{rgb}{0.121569,0.466667,0.705882}%
\pgfsetfillcolor{currentfill}%
\pgfsetlinewidth{0.000000pt}%
\definecolor{currentstroke}{rgb}{0.000000,0.000000,0.000000}%
\pgfsetstrokecolor{currentstroke}%
\pgfsetstrokeopacity{0.000000}%
\pgfsetdash{}{0pt}%
\pgfpathmoveto{\pgfqpoint{-319.224843in}{2.567516in}}%
\pgfpathlineto{\pgfqpoint{10.404239in}{2.567516in}}%
\pgfpathlineto{\pgfqpoint{10.404239in}{2.574604in}}%
\pgfpathlineto{\pgfqpoint{-319.224843in}{2.574604in}}%
\pgfpathclose%
\pgfusepath{fill}%
\end{pgfscope}%
\begin{pgfscope}%
\pgfpathrectangle{\pgfqpoint{10.668400in}{1.805660in}}{\pgfqpoint{2.188235in}{0.972632in}}%
\pgfusepath{clip}%
\pgfsetbuttcap%
\pgfsetmiterjoin%
\definecolor{currentfill}{rgb}{0.121569,0.466667,0.705882}%
\pgfsetfillcolor{currentfill}%
\pgfsetlinewidth{0.000000pt}%
\definecolor{currentstroke}{rgb}{0.000000,0.000000,0.000000}%
\pgfsetstrokecolor{currentstroke}%
\pgfsetstrokeopacity{0.000000}%
\pgfsetdash{}{0pt}%
\pgfpathmoveto{\pgfqpoint{-319.224843in}{2.576376in}}%
\pgfpathlineto{\pgfqpoint{10.791640in}{2.576376in}}%
\pgfpathlineto{\pgfqpoint{10.791640in}{2.583464in}}%
\pgfpathlineto{\pgfqpoint{-319.224843in}{2.583464in}}%
\pgfpathclose%
\pgfusepath{fill}%
\end{pgfscope}%
\begin{pgfscope}%
\pgfpathrectangle{\pgfqpoint{10.668400in}{1.805660in}}{\pgfqpoint{2.188235in}{0.972632in}}%
\pgfusepath{clip}%
\pgfsetbuttcap%
\pgfsetmiterjoin%
\definecolor{currentfill}{rgb}{0.121569,0.466667,0.705882}%
\pgfsetfillcolor{currentfill}%
\pgfsetlinewidth{0.000000pt}%
\definecolor{currentstroke}{rgb}{0.000000,0.000000,0.000000}%
\pgfsetstrokecolor{currentstroke}%
\pgfsetstrokeopacity{0.000000}%
\pgfsetdash{}{0pt}%
\pgfpathmoveto{\pgfqpoint{-319.224843in}{2.585236in}}%
\pgfpathlineto{\pgfqpoint{10.733352in}{2.585236in}}%
\pgfpathlineto{\pgfqpoint{10.733352in}{2.592324in}}%
\pgfpathlineto{\pgfqpoint{-319.224843in}{2.592324in}}%
\pgfpathclose%
\pgfusepath{fill}%
\end{pgfscope}%
\begin{pgfscope}%
\pgfpathrectangle{\pgfqpoint{10.668400in}{1.805660in}}{\pgfqpoint{2.188235in}{0.972632in}}%
\pgfusepath{clip}%
\pgfsetbuttcap%
\pgfsetmiterjoin%
\definecolor{currentfill}{rgb}{0.121569,0.466667,0.705882}%
\pgfsetfillcolor{currentfill}%
\pgfsetlinewidth{0.000000pt}%
\definecolor{currentstroke}{rgb}{0.000000,0.000000,0.000000}%
\pgfsetstrokecolor{currentstroke}%
\pgfsetstrokeopacity{0.000000}%
\pgfsetdash{}{0pt}%
\pgfpathmoveto{\pgfqpoint{-319.224843in}{2.594096in}}%
\pgfpathlineto{\pgfqpoint{10.764020in}{2.594096in}}%
\pgfpathlineto{\pgfqpoint{10.764020in}{2.601183in}}%
\pgfpathlineto{\pgfqpoint{-319.224843in}{2.601183in}}%
\pgfpathclose%
\pgfusepath{fill}%
\end{pgfscope}%
\begin{pgfscope}%
\pgfpathrectangle{\pgfqpoint{10.668400in}{1.805660in}}{\pgfqpoint{2.188235in}{0.972632in}}%
\pgfusepath{clip}%
\pgfsetbuttcap%
\pgfsetmiterjoin%
\definecolor{currentfill}{rgb}{0.121569,0.466667,0.705882}%
\pgfsetfillcolor{currentfill}%
\pgfsetlinewidth{0.000000pt}%
\definecolor{currentstroke}{rgb}{0.000000,0.000000,0.000000}%
\pgfsetstrokecolor{currentstroke}%
\pgfsetstrokeopacity{0.000000}%
\pgfsetdash{}{0pt}%
\pgfpathmoveto{\pgfqpoint{-319.224843in}{2.602955in}}%
\pgfpathlineto{\pgfqpoint{10.651399in}{2.602955in}}%
\pgfpathlineto{\pgfqpoint{10.651399in}{2.610043in}}%
\pgfpathlineto{\pgfqpoint{-319.224843in}{2.610043in}}%
\pgfpathclose%
\pgfusepath{fill}%
\end{pgfscope}%
\begin{pgfscope}%
\pgfpathrectangle{\pgfqpoint{10.668400in}{1.805660in}}{\pgfqpoint{2.188235in}{0.972632in}}%
\pgfusepath{clip}%
\pgfsetbuttcap%
\pgfsetmiterjoin%
\definecolor{currentfill}{rgb}{0.121569,0.466667,0.705882}%
\pgfsetfillcolor{currentfill}%
\pgfsetlinewidth{0.000000pt}%
\definecolor{currentstroke}{rgb}{0.000000,0.000000,0.000000}%
\pgfsetstrokecolor{currentstroke}%
\pgfsetstrokeopacity{0.000000}%
\pgfsetdash{}{0pt}%
\pgfpathmoveto{\pgfqpoint{-319.224843in}{2.611815in}}%
\pgfpathlineto{\pgfqpoint{10.637015in}{2.611815in}}%
\pgfpathlineto{\pgfqpoint{10.637015in}{2.618903in}}%
\pgfpathlineto{\pgfqpoint{-319.224843in}{2.618903in}}%
\pgfpathclose%
\pgfusepath{fill}%
\end{pgfscope}%
\begin{pgfscope}%
\pgfpathrectangle{\pgfqpoint{10.668400in}{1.805660in}}{\pgfqpoint{2.188235in}{0.972632in}}%
\pgfusepath{clip}%
\pgfsetbuttcap%
\pgfsetmiterjoin%
\definecolor{currentfill}{rgb}{0.121569,0.466667,0.705882}%
\pgfsetfillcolor{currentfill}%
\pgfsetlinewidth{0.000000pt}%
\definecolor{currentstroke}{rgb}{0.000000,0.000000,0.000000}%
\pgfsetstrokecolor{currentstroke}%
\pgfsetstrokeopacity{0.000000}%
\pgfsetdash{}{0pt}%
\pgfpathmoveto{\pgfqpoint{-319.224843in}{2.620675in}}%
\pgfpathlineto{\pgfqpoint{10.657030in}{2.620675in}}%
\pgfpathlineto{\pgfqpoint{10.657030in}{2.627763in}}%
\pgfpathlineto{\pgfqpoint{-319.224843in}{2.627763in}}%
\pgfpathclose%
\pgfusepath{fill}%
\end{pgfscope}%
\begin{pgfscope}%
\pgfpathrectangle{\pgfqpoint{10.668400in}{1.805660in}}{\pgfqpoint{2.188235in}{0.972632in}}%
\pgfusepath{clip}%
\pgfsetbuttcap%
\pgfsetmiterjoin%
\definecolor{currentfill}{rgb}{0.121569,0.466667,0.705882}%
\pgfsetfillcolor{currentfill}%
\pgfsetlinewidth{0.000000pt}%
\definecolor{currentstroke}{rgb}{0.000000,0.000000,0.000000}%
\pgfsetstrokecolor{currentstroke}%
\pgfsetstrokeopacity{0.000000}%
\pgfsetdash{}{0pt}%
\pgfpathmoveto{\pgfqpoint{-319.224843in}{2.629535in}}%
\pgfpathlineto{\pgfqpoint{10.882340in}{2.629535in}}%
\pgfpathlineto{\pgfqpoint{10.882340in}{2.636623in}}%
\pgfpathlineto{\pgfqpoint{-319.224843in}{2.636623in}}%
\pgfpathclose%
\pgfusepath{fill}%
\end{pgfscope}%
\begin{pgfscope}%
\pgfpathrectangle{\pgfqpoint{10.668400in}{1.805660in}}{\pgfqpoint{2.188235in}{0.972632in}}%
\pgfusepath{clip}%
\pgfsetbuttcap%
\pgfsetmiterjoin%
\definecolor{currentfill}{rgb}{0.121569,0.466667,0.705882}%
\pgfsetfillcolor{currentfill}%
\pgfsetlinewidth{0.000000pt}%
\definecolor{currentstroke}{rgb}{0.000000,0.000000,0.000000}%
\pgfsetstrokecolor{currentstroke}%
\pgfsetstrokeopacity{0.000000}%
\pgfsetdash{}{0pt}%
\pgfpathmoveto{\pgfqpoint{-319.224843in}{2.638395in}}%
\pgfpathlineto{\pgfqpoint{10.891787in}{2.638395in}}%
\pgfpathlineto{\pgfqpoint{10.891787in}{2.645483in}}%
\pgfpathlineto{\pgfqpoint{-319.224843in}{2.645483in}}%
\pgfpathclose%
\pgfusepath{fill}%
\end{pgfscope}%
\begin{pgfscope}%
\pgfpathrectangle{\pgfqpoint{10.668400in}{1.805660in}}{\pgfqpoint{2.188235in}{0.972632in}}%
\pgfusepath{clip}%
\pgfsetbuttcap%
\pgfsetmiterjoin%
\definecolor{currentfill}{rgb}{0.121569,0.466667,0.705882}%
\pgfsetfillcolor{currentfill}%
\pgfsetlinewidth{0.000000pt}%
\definecolor{currentstroke}{rgb}{0.000000,0.000000,0.000000}%
\pgfsetstrokecolor{currentstroke}%
\pgfsetstrokeopacity{0.000000}%
\pgfsetdash{}{0pt}%
\pgfpathmoveto{\pgfqpoint{-319.224843in}{2.647255in}}%
\pgfpathlineto{\pgfqpoint{10.955432in}{2.647255in}}%
\pgfpathlineto{\pgfqpoint{10.955432in}{2.654342in}}%
\pgfpathlineto{\pgfqpoint{-319.224843in}{2.654342in}}%
\pgfpathclose%
\pgfusepath{fill}%
\end{pgfscope}%
\begin{pgfscope}%
\pgfpathrectangle{\pgfqpoint{10.668400in}{1.805660in}}{\pgfqpoint{2.188235in}{0.972632in}}%
\pgfusepath{clip}%
\pgfsetbuttcap%
\pgfsetmiterjoin%
\definecolor{currentfill}{rgb}{0.121569,0.466667,0.705882}%
\pgfsetfillcolor{currentfill}%
\pgfsetlinewidth{0.000000pt}%
\definecolor{currentstroke}{rgb}{0.000000,0.000000,0.000000}%
\pgfsetstrokecolor{currentstroke}%
\pgfsetstrokeopacity{0.000000}%
\pgfsetdash{}{0pt}%
\pgfpathmoveto{\pgfqpoint{-319.224843in}{2.656114in}}%
\pgfpathlineto{\pgfqpoint{10.693721in}{2.656114in}}%
\pgfpathlineto{\pgfqpoint{10.693721in}{2.663202in}}%
\pgfpathlineto{\pgfqpoint{-319.224843in}{2.663202in}}%
\pgfpathclose%
\pgfusepath{fill}%
\end{pgfscope}%
\begin{pgfscope}%
\pgfpathrectangle{\pgfqpoint{10.668400in}{1.805660in}}{\pgfqpoint{2.188235in}{0.972632in}}%
\pgfusepath{clip}%
\pgfsetbuttcap%
\pgfsetmiterjoin%
\definecolor{currentfill}{rgb}{0.121569,0.466667,0.705882}%
\pgfsetfillcolor{currentfill}%
\pgfsetlinewidth{0.000000pt}%
\definecolor{currentstroke}{rgb}{0.000000,0.000000,0.000000}%
\pgfsetstrokecolor{currentstroke}%
\pgfsetstrokeopacity{0.000000}%
\pgfsetdash{}{0pt}%
\pgfpathmoveto{\pgfqpoint{-319.224843in}{2.664974in}}%
\pgfpathlineto{\pgfqpoint{10.909472in}{2.664974in}}%
\pgfpathlineto{\pgfqpoint{10.909472in}{2.672062in}}%
\pgfpathlineto{\pgfqpoint{-319.224843in}{2.672062in}}%
\pgfpathclose%
\pgfusepath{fill}%
\end{pgfscope}%
\begin{pgfscope}%
\pgfpathrectangle{\pgfqpoint{10.668400in}{1.805660in}}{\pgfqpoint{2.188235in}{0.972632in}}%
\pgfusepath{clip}%
\pgfsetbuttcap%
\pgfsetmiterjoin%
\definecolor{currentfill}{rgb}{0.121569,0.466667,0.705882}%
\pgfsetfillcolor{currentfill}%
\pgfsetlinewidth{0.000000pt}%
\definecolor{currentstroke}{rgb}{0.000000,0.000000,0.000000}%
\pgfsetstrokecolor{currentstroke}%
\pgfsetstrokeopacity{0.000000}%
\pgfsetdash{}{0pt}%
\pgfpathmoveto{\pgfqpoint{-319.224843in}{2.673834in}}%
\pgfpathlineto{\pgfqpoint{10.722839in}{2.673834in}}%
\pgfpathlineto{\pgfqpoint{10.722839in}{2.680922in}}%
\pgfpathlineto{\pgfqpoint{-319.224843in}{2.680922in}}%
\pgfpathclose%
\pgfusepath{fill}%
\end{pgfscope}%
\begin{pgfscope}%
\pgfpathrectangle{\pgfqpoint{10.668400in}{1.805660in}}{\pgfqpoint{2.188235in}{0.972632in}}%
\pgfusepath{clip}%
\pgfsetbuttcap%
\pgfsetmiterjoin%
\definecolor{currentfill}{rgb}{0.121569,0.466667,0.705882}%
\pgfsetfillcolor{currentfill}%
\pgfsetlinewidth{0.000000pt}%
\definecolor{currentstroke}{rgb}{0.000000,0.000000,0.000000}%
\pgfsetstrokecolor{currentstroke}%
\pgfsetstrokeopacity{0.000000}%
\pgfsetdash{}{0pt}%
\pgfpathmoveto{\pgfqpoint{-319.224843in}{2.682694in}}%
\pgfpathlineto{\pgfqpoint{10.911673in}{2.682694in}}%
\pgfpathlineto{\pgfqpoint{10.911673in}{2.689782in}}%
\pgfpathlineto{\pgfqpoint{-319.224843in}{2.689782in}}%
\pgfpathclose%
\pgfusepath{fill}%
\end{pgfscope}%
\begin{pgfscope}%
\pgfpathrectangle{\pgfqpoint{10.668400in}{1.805660in}}{\pgfqpoint{2.188235in}{0.972632in}}%
\pgfusepath{clip}%
\pgfsetbuttcap%
\pgfsetmiterjoin%
\definecolor{currentfill}{rgb}{0.121569,0.466667,0.705882}%
\pgfsetfillcolor{currentfill}%
\pgfsetlinewidth{0.000000pt}%
\definecolor{currentstroke}{rgb}{0.000000,0.000000,0.000000}%
\pgfsetstrokecolor{currentstroke}%
\pgfsetstrokeopacity{0.000000}%
\pgfsetdash{}{0pt}%
\pgfpathmoveto{\pgfqpoint{-319.224843in}{2.691554in}}%
\pgfpathlineto{\pgfqpoint{10.843502in}{2.691554in}}%
\pgfpathlineto{\pgfqpoint{10.843502in}{2.698642in}}%
\pgfpathlineto{\pgfqpoint{-319.224843in}{2.698642in}}%
\pgfpathclose%
\pgfusepath{fill}%
\end{pgfscope}%
\begin{pgfscope}%
\pgfpathrectangle{\pgfqpoint{10.668400in}{1.805660in}}{\pgfqpoint{2.188235in}{0.972632in}}%
\pgfusepath{clip}%
\pgfsetbuttcap%
\pgfsetmiterjoin%
\definecolor{currentfill}{rgb}{0.121569,0.466667,0.705882}%
\pgfsetfillcolor{currentfill}%
\pgfsetlinewidth{0.000000pt}%
\definecolor{currentstroke}{rgb}{0.000000,0.000000,0.000000}%
\pgfsetstrokecolor{currentstroke}%
\pgfsetstrokeopacity{0.000000}%
\pgfsetdash{}{0pt}%
\pgfpathmoveto{\pgfqpoint{-319.224843in}{2.700413in}}%
\pgfpathlineto{\pgfqpoint{10.688765in}{2.700413in}}%
\pgfpathlineto{\pgfqpoint{10.688765in}{2.707501in}}%
\pgfpathlineto{\pgfqpoint{-319.224843in}{2.707501in}}%
\pgfpathclose%
\pgfusepath{fill}%
\end{pgfscope}%
\begin{pgfscope}%
\pgfpathrectangle{\pgfqpoint{10.668400in}{1.805660in}}{\pgfqpoint{2.188235in}{0.972632in}}%
\pgfusepath{clip}%
\pgfsetbuttcap%
\pgfsetmiterjoin%
\definecolor{currentfill}{rgb}{0.121569,0.466667,0.705882}%
\pgfsetfillcolor{currentfill}%
\pgfsetlinewidth{0.000000pt}%
\definecolor{currentstroke}{rgb}{0.000000,0.000000,0.000000}%
\pgfsetstrokecolor{currentstroke}%
\pgfsetstrokeopacity{0.000000}%
\pgfsetdash{}{0pt}%
\pgfpathmoveto{\pgfqpoint{-319.224843in}{2.709273in}}%
\pgfpathlineto{\pgfqpoint{10.739689in}{2.709273in}}%
\pgfpathlineto{\pgfqpoint{10.739689in}{2.716361in}}%
\pgfpathlineto{\pgfqpoint{-319.224843in}{2.716361in}}%
\pgfpathclose%
\pgfusepath{fill}%
\end{pgfscope}%
\begin{pgfscope}%
\pgfpathrectangle{\pgfqpoint{10.668400in}{1.805660in}}{\pgfqpoint{2.188235in}{0.972632in}}%
\pgfusepath{clip}%
\pgfsetbuttcap%
\pgfsetmiterjoin%
\definecolor{currentfill}{rgb}{0.121569,0.466667,0.705882}%
\pgfsetfillcolor{currentfill}%
\pgfsetlinewidth{0.000000pt}%
\definecolor{currentstroke}{rgb}{0.000000,0.000000,0.000000}%
\pgfsetstrokecolor{currentstroke}%
\pgfsetstrokeopacity{0.000000}%
\pgfsetdash{}{0pt}%
\pgfpathmoveto{\pgfqpoint{-319.224843in}{2.718133in}}%
\pgfpathlineto{\pgfqpoint{10.622124in}{2.718133in}}%
\pgfpathlineto{\pgfqpoint{10.622124in}{2.725221in}}%
\pgfpathlineto{\pgfqpoint{-319.224843in}{2.725221in}}%
\pgfpathclose%
\pgfusepath{fill}%
\end{pgfscope}%
\begin{pgfscope}%
\pgfpathrectangle{\pgfqpoint{10.668400in}{1.805660in}}{\pgfqpoint{2.188235in}{0.972632in}}%
\pgfusepath{clip}%
\pgfsetbuttcap%
\pgfsetmiterjoin%
\definecolor{currentfill}{rgb}{0.121569,0.466667,0.705882}%
\pgfsetfillcolor{currentfill}%
\pgfsetlinewidth{0.000000pt}%
\definecolor{currentstroke}{rgb}{0.000000,0.000000,0.000000}%
\pgfsetstrokecolor{currentstroke}%
\pgfsetstrokeopacity{0.000000}%
\pgfsetdash{}{0pt}%
\pgfpathmoveto{\pgfqpoint{-319.224843in}{2.726993in}}%
\pgfpathlineto{\pgfqpoint{10.479110in}{2.726993in}}%
\pgfpathlineto{\pgfqpoint{10.479110in}{2.734081in}}%
\pgfpathlineto{\pgfqpoint{-319.224843in}{2.734081in}}%
\pgfpathclose%
\pgfusepath{fill}%
\end{pgfscope}%
\begin{pgfscope}%
\pgfsetbuttcap%
\pgfsetroundjoin%
\definecolor{currentfill}{rgb}{0.000000,0.000000,0.000000}%
\pgfsetfillcolor{currentfill}%
\pgfsetlinewidth{0.803000pt}%
\definecolor{currentstroke}{rgb}{0.000000,0.000000,0.000000}%
\pgfsetstrokecolor{currentstroke}%
\pgfsetdash{}{0pt}%
\pgfsys@defobject{currentmarker}{\pgfqpoint{0.000000in}{-0.048611in}}{\pgfqpoint{0.000000in}{0.000000in}}{%
\pgfpathmoveto{\pgfqpoint{0.000000in}{0.000000in}}%
\pgfpathlineto{\pgfqpoint{0.000000in}{-0.048611in}}%
\pgfusepath{stroke,fill}%
}%
\begin{pgfscope}%
\pgfsys@transformshift{11.218880in}{1.805660in}%
\pgfsys@useobject{currentmarker}{}%
\end{pgfscope}%
\end{pgfscope}%
\begin{pgfscope}%
\pgfsetbuttcap%
\pgfsetroundjoin%
\definecolor{currentfill}{rgb}{0.000000,0.000000,0.000000}%
\pgfsetfillcolor{currentfill}%
\pgfsetlinewidth{0.803000pt}%
\definecolor{currentstroke}{rgb}{0.000000,0.000000,0.000000}%
\pgfsetstrokecolor{currentstroke}%
\pgfsetdash{}{0pt}%
\pgfsys@defobject{currentmarker}{\pgfqpoint{0.000000in}{-0.048611in}}{\pgfqpoint{0.000000in}{0.000000in}}{%
\pgfpathmoveto{\pgfqpoint{0.000000in}{0.000000in}}%
\pgfpathlineto{\pgfqpoint{0.000000in}{-0.048611in}}%
\pgfusepath{stroke,fill}%
}%
\begin{pgfscope}%
\pgfsys@transformshift{11.881756in}{1.805660in}%
\pgfsys@useobject{currentmarker}{}%
\end{pgfscope}%
\end{pgfscope}%
\begin{pgfscope}%
\pgfsetbuttcap%
\pgfsetroundjoin%
\definecolor{currentfill}{rgb}{0.000000,0.000000,0.000000}%
\pgfsetfillcolor{currentfill}%
\pgfsetlinewidth{0.803000pt}%
\definecolor{currentstroke}{rgb}{0.000000,0.000000,0.000000}%
\pgfsetstrokecolor{currentstroke}%
\pgfsetdash{}{0pt}%
\pgfsys@defobject{currentmarker}{\pgfqpoint{0.000000in}{-0.048611in}}{\pgfqpoint{0.000000in}{0.000000in}}{%
\pgfpathmoveto{\pgfqpoint{0.000000in}{0.000000in}}%
\pgfpathlineto{\pgfqpoint{0.000000in}{-0.048611in}}%
\pgfusepath{stroke,fill}%
}%
\begin{pgfscope}%
\pgfsys@transformshift{12.544632in}{1.805660in}%
\pgfsys@useobject{currentmarker}{}%
\end{pgfscope}%
\end{pgfscope}%
\begin{pgfscope}%
\pgfsetbuttcap%
\pgfsetroundjoin%
\definecolor{currentfill}{rgb}{0.000000,0.000000,0.000000}%
\pgfsetfillcolor{currentfill}%
\pgfsetlinewidth{0.803000pt}%
\definecolor{currentstroke}{rgb}{0.000000,0.000000,0.000000}%
\pgfsetstrokecolor{currentstroke}%
\pgfsetdash{}{0pt}%
\pgfsys@defobject{currentmarker}{\pgfqpoint{-0.048611in}{0.000000in}}{\pgfqpoint{0.000000in}{0.000000in}}{%
\pgfpathmoveto{\pgfqpoint{0.000000in}{0.000000in}}%
\pgfpathlineto{\pgfqpoint{-0.048611in}{0.000000in}}%
\pgfusepath{stroke,fill}%
}%
\begin{pgfscope}%
\pgfsys@transformshift{10.668400in}{1.853414in}%
\pgfsys@useobject{currentmarker}{}%
\end{pgfscope}%
\end{pgfscope}%
\begin{pgfscope}%
\pgftext[x=10.501733in,y=1.800653in,left,base]{\rmfamily\fontsize{10.000000}{12.000000}\selectfont \(\displaystyle 0\)}%
\end{pgfscope}%
\begin{pgfscope}%
\pgfsetbuttcap%
\pgfsetroundjoin%
\definecolor{currentfill}{rgb}{0.000000,0.000000,0.000000}%
\pgfsetfillcolor{currentfill}%
\pgfsetlinewidth{0.803000pt}%
\definecolor{currentstroke}{rgb}{0.000000,0.000000,0.000000}%
\pgfsetstrokecolor{currentstroke}%
\pgfsetdash{}{0pt}%
\pgfsys@defobject{currentmarker}{\pgfqpoint{-0.048611in}{0.000000in}}{\pgfqpoint{0.000000in}{0.000000in}}{%
\pgfpathmoveto{\pgfqpoint{0.000000in}{0.000000in}}%
\pgfpathlineto{\pgfqpoint{-0.048611in}{0.000000in}}%
\pgfusepath{stroke,fill}%
}%
\begin{pgfscope}%
\pgfsys@transformshift{10.668400in}{2.296405in}%
\pgfsys@useobject{currentmarker}{}%
\end{pgfscope}%
\end{pgfscope}%
\begin{pgfscope}%
\pgftext[x=10.432288in,y=2.243644in,left,base]{\rmfamily\fontsize{10.000000}{12.000000}\selectfont \(\displaystyle 50\)}%
\end{pgfscope}%
\begin{pgfscope}%
\pgfsetbuttcap%
\pgfsetroundjoin%
\definecolor{currentfill}{rgb}{0.000000,0.000000,0.000000}%
\pgfsetfillcolor{currentfill}%
\pgfsetlinewidth{0.803000pt}%
\definecolor{currentstroke}{rgb}{0.000000,0.000000,0.000000}%
\pgfsetstrokecolor{currentstroke}%
\pgfsetdash{}{0pt}%
\pgfsys@defobject{currentmarker}{\pgfqpoint{-0.048611in}{0.000000in}}{\pgfqpoint{0.000000in}{0.000000in}}{%
\pgfpathmoveto{\pgfqpoint{0.000000in}{0.000000in}}%
\pgfpathlineto{\pgfqpoint{-0.048611in}{0.000000in}}%
\pgfusepath{stroke,fill}%
}%
\begin{pgfscope}%
\pgfsys@transformshift{10.668400in}{2.739397in}%
\pgfsys@useobject{currentmarker}{}%
\end{pgfscope}%
\end{pgfscope}%
\begin{pgfscope}%
\pgftext[x=10.362844in,y=2.686635in,left,base]{\rmfamily\fontsize{10.000000}{12.000000}\selectfont \(\displaystyle 100\)}%
\end{pgfscope}%
\begin{pgfscope}%
\pgftext[x=10.307288in,y=2.291976in,,bottom,rotate=90.000000]{\rmfamily\fontsize{10.000000}{12.000000}\selectfont \(\displaystyle j\)}%
\end{pgfscope}%
\begin{pgfscope}%
\pgfsetrectcap%
\pgfsetmiterjoin%
\pgfsetlinewidth{0.803000pt}%
\definecolor{currentstroke}{rgb}{0.000000,0.000000,0.000000}%
\pgfsetstrokecolor{currentstroke}%
\pgfsetdash{}{0pt}%
\pgfpathmoveto{\pgfqpoint{10.668400in}{1.805660in}}%
\pgfpathlineto{\pgfqpoint{10.668400in}{2.778291in}}%
\pgfusepath{stroke}%
\end{pgfscope}%
\begin{pgfscope}%
\pgfsetrectcap%
\pgfsetmiterjoin%
\pgfsetlinewidth{0.803000pt}%
\definecolor{currentstroke}{rgb}{0.000000,0.000000,0.000000}%
\pgfsetstrokecolor{currentstroke}%
\pgfsetdash{}{0pt}%
\pgfpathmoveto{\pgfqpoint{12.856635in}{1.805660in}}%
\pgfpathlineto{\pgfqpoint{12.856635in}{2.778291in}}%
\pgfusepath{stroke}%
\end{pgfscope}%
\begin{pgfscope}%
\pgfsetrectcap%
\pgfsetmiterjoin%
\pgfsetlinewidth{0.803000pt}%
\definecolor{currentstroke}{rgb}{0.000000,0.000000,0.000000}%
\pgfsetstrokecolor{currentstroke}%
\pgfsetdash{}{0pt}%
\pgfpathmoveto{\pgfqpoint{10.668400in}{1.805660in}}%
\pgfpathlineto{\pgfqpoint{12.856635in}{1.805660in}}%
\pgfusepath{stroke}%
\end{pgfscope}%
\begin{pgfscope}%
\pgfsetrectcap%
\pgfsetmiterjoin%
\pgfsetlinewidth{0.803000pt}%
\definecolor{currentstroke}{rgb}{0.000000,0.000000,0.000000}%
\pgfsetstrokecolor{currentstroke}%
\pgfsetdash{}{0pt}%
\pgfpathmoveto{\pgfqpoint{10.668400in}{2.778291in}}%
\pgfpathlineto{\pgfqpoint{12.856635in}{2.778291in}}%
\pgfusepath{stroke}%
\end{pgfscope}%
\begin{pgfscope}%
\pgfsetbuttcap%
\pgfsetmiterjoin%
\definecolor{currentfill}{rgb}{1.000000,1.000000,1.000000}%
\pgfsetfillcolor{currentfill}%
\pgfsetlinewidth{0.000000pt}%
\definecolor{currentstroke}{rgb}{0.000000,0.000000,0.000000}%
\pgfsetstrokecolor{currentstroke}%
\pgfsetstrokeopacity{0.000000}%
\pgfsetdash{}{0pt}%
\pgfpathmoveto{\pgfqpoint{0.456635in}{0.589870in}}%
\pgfpathlineto{\pgfqpoint{4.833106in}{0.589870in}}%
\pgfpathlineto{\pgfqpoint{4.833106in}{1.562502in}}%
\pgfpathlineto{\pgfqpoint{0.456635in}{1.562502in}}%
\pgfpathclose%
\pgfusepath{fill}%
\end{pgfscope}%
\begin{pgfscope}%
\pgfpathrectangle{\pgfqpoint{0.456635in}{0.589870in}}{\pgfqpoint{4.376471in}{0.972632in}}%
\pgfusepath{clip}%
\pgfsetbuttcap%
\pgfsetroundjoin%
\definecolor{currentfill}{rgb}{1.000000,0.000000,0.000000}%
\pgfsetfillcolor{currentfill}%
\pgfsetlinewidth{2.007500pt}%
\definecolor{currentstroke}{rgb}{1.000000,0.000000,0.000000}%
\pgfsetstrokecolor{currentstroke}%
\pgfsetdash{}{0pt}%
\pgfpathmoveto{\pgfqpoint{2.601594in}{0.743738in}}%
\pgfpathlineto{\pgfqpoint{2.684927in}{0.743738in}}%
\pgfpathmoveto{\pgfqpoint{2.643260in}{0.702071in}}%
\pgfpathlineto{\pgfqpoint{2.643260in}{0.785404in}}%
\pgfusepath{stroke,fill}%
\end{pgfscope}%
\begin{pgfscope}%
\pgfpathrectangle{\pgfqpoint{0.456635in}{0.589870in}}{\pgfqpoint{4.376471in}{0.972632in}}%
\pgfusepath{clip}%
\pgfsetbuttcap%
\pgfsetroundjoin%
\definecolor{currentfill}{rgb}{1.000000,0.000000,0.000000}%
\pgfsetfillcolor{currentfill}%
\pgfsetlinewidth{2.007500pt}%
\definecolor{currentstroke}{rgb}{1.000000,0.000000,0.000000}%
\pgfsetstrokecolor{currentstroke}%
\pgfsetdash{}{0pt}%
\pgfpathmoveto{\pgfqpoint{4.618881in}{1.402840in}}%
\pgfpathlineto{\pgfqpoint{4.702215in}{1.402840in}}%
\pgfpathmoveto{\pgfqpoint{4.660548in}{1.361173in}}%
\pgfpathlineto{\pgfqpoint{4.660548in}{1.444506in}}%
\pgfusepath{stroke,fill}%
\end{pgfscope}%
\begin{pgfscope}%
\pgfpathrectangle{\pgfqpoint{0.456635in}{0.589870in}}{\pgfqpoint{4.376471in}{0.972632in}}%
\pgfusepath{clip}%
\pgfsetbuttcap%
\pgfsetroundjoin%
\definecolor{currentfill}{rgb}{1.000000,0.000000,0.000000}%
\pgfsetfillcolor{currentfill}%
\pgfsetlinewidth{2.007500pt}%
\definecolor{currentstroke}{rgb}{1.000000,0.000000,0.000000}%
\pgfsetstrokecolor{currentstroke}%
\pgfsetdash{}{0pt}%
\pgfpathmoveto{\pgfqpoint{3.853103in}{0.778901in}}%
\pgfpathlineto{\pgfqpoint{3.936436in}{0.778901in}}%
\pgfpathmoveto{\pgfqpoint{3.894769in}{0.737234in}}%
\pgfpathlineto{\pgfqpoint{3.894769in}{0.820567in}}%
\pgfusepath{stroke,fill}%
\end{pgfscope}%
\begin{pgfscope}%
\pgfpathrectangle{\pgfqpoint{0.456635in}{0.589870in}}{\pgfqpoint{4.376471in}{0.972632in}}%
\pgfusepath{clip}%
\pgfsetbuttcap%
\pgfsetroundjoin%
\definecolor{currentfill}{rgb}{1.000000,0.000000,0.000000}%
\pgfsetfillcolor{currentfill}%
\pgfsetlinewidth{2.007500pt}%
\definecolor{currentstroke}{rgb}{1.000000,0.000000,0.000000}%
\pgfsetstrokecolor{currentstroke}%
\pgfsetdash{}{0pt}%
\pgfpathmoveto{\pgfqpoint{3.386272in}{1.127281in}}%
\pgfpathlineto{\pgfqpoint{3.469605in}{1.127281in}}%
\pgfpathmoveto{\pgfqpoint{3.427938in}{1.085614in}}%
\pgfpathlineto{\pgfqpoint{3.427938in}{1.168947in}}%
\pgfusepath{stroke,fill}%
\end{pgfscope}%
\begin{pgfscope}%
\pgfpathrectangle{\pgfqpoint{0.456635in}{0.589870in}}{\pgfqpoint{4.376471in}{0.972632in}}%
\pgfusepath{clip}%
\pgfsetbuttcap%
\pgfsetroundjoin%
\definecolor{currentfill}{rgb}{1.000000,0.000000,0.000000}%
\pgfsetfillcolor{currentfill}%
\pgfsetlinewidth{2.007500pt}%
\definecolor{currentstroke}{rgb}{1.000000,0.000000,0.000000}%
\pgfsetstrokecolor{currentstroke}%
\pgfsetdash{}{0pt}%
\pgfpathmoveto{\pgfqpoint{1.836512in}{0.971634in}}%
\pgfpathlineto{\pgfqpoint{1.919845in}{0.971634in}}%
\pgfpathmoveto{\pgfqpoint{1.878178in}{0.929968in}}%
\pgfpathlineto{\pgfqpoint{1.878178in}{1.013301in}}%
\pgfusepath{stroke,fill}%
\end{pgfscope}%
\begin{pgfscope}%
\pgfpathrectangle{\pgfqpoint{0.456635in}{0.589870in}}{\pgfqpoint{4.376471in}{0.972632in}}%
\pgfusepath{clip}%
\pgfsetbuttcap%
\pgfsetroundjoin%
\definecolor{currentfill}{rgb}{1.000000,0.000000,0.000000}%
\pgfsetfillcolor{currentfill}%
\pgfsetlinewidth{2.007500pt}%
\definecolor{currentstroke}{rgb}{1.000000,0.000000,0.000000}%
\pgfsetstrokecolor{currentstroke}%
\pgfsetdash{}{0pt}%
\pgfpathmoveto{\pgfqpoint{1.836427in}{1.220208in}}%
\pgfpathlineto{\pgfqpoint{1.919760in}{1.220208in}}%
\pgfpathmoveto{\pgfqpoint{1.878094in}{1.178541in}}%
\pgfpathlineto{\pgfqpoint{1.878094in}{1.261874in}}%
\pgfusepath{stroke,fill}%
\end{pgfscope}%
\begin{pgfscope}%
\pgfpathrectangle{\pgfqpoint{0.456635in}{0.589870in}}{\pgfqpoint{4.376471in}{0.972632in}}%
\pgfusepath{clip}%
\pgfsetbuttcap%
\pgfsetroundjoin%
\definecolor{currentfill}{rgb}{1.000000,0.000000,0.000000}%
\pgfsetfillcolor{currentfill}%
\pgfsetlinewidth{2.007500pt}%
\definecolor{currentstroke}{rgb}{1.000000,0.000000,0.000000}%
\pgfsetstrokecolor{currentstroke}%
\pgfsetdash{}{0pt}%
\pgfpathmoveto{\pgfqpoint{1.493624in}{1.157677in}}%
\pgfpathlineto{\pgfqpoint{1.576957in}{1.157677in}}%
\pgfpathmoveto{\pgfqpoint{1.535290in}{1.116011in}}%
\pgfpathlineto{\pgfqpoint{1.535290in}{1.199344in}}%
\pgfusepath{stroke,fill}%
\end{pgfscope}%
\begin{pgfscope}%
\pgfpathrectangle{\pgfqpoint{0.456635in}{0.589870in}}{\pgfqpoint{4.376471in}{0.972632in}}%
\pgfusepath{clip}%
\pgfsetbuttcap%
\pgfsetroundjoin%
\definecolor{currentfill}{rgb}{1.000000,0.000000,0.000000}%
\pgfsetfillcolor{currentfill}%
\pgfsetlinewidth{2.007500pt}%
\definecolor{currentstroke}{rgb}{1.000000,0.000000,0.000000}%
\pgfsetstrokecolor{currentstroke}%
\pgfsetdash{}{0pt}%
\pgfpathmoveto{\pgfqpoint{4.322898in}{1.239588in}}%
\pgfpathlineto{\pgfqpoint{4.406232in}{1.239588in}}%
\pgfpathmoveto{\pgfqpoint{4.364565in}{1.197921in}}%
\pgfpathlineto{\pgfqpoint{4.364565in}{1.281255in}}%
\pgfusepath{stroke,fill}%
\end{pgfscope}%
\begin{pgfscope}%
\pgfpathrectangle{\pgfqpoint{0.456635in}{0.589870in}}{\pgfqpoint{4.376471in}{0.972632in}}%
\pgfusepath{clip}%
\pgfsetbuttcap%
\pgfsetroundjoin%
\definecolor{currentfill}{rgb}{1.000000,0.000000,0.000000}%
\pgfsetfillcolor{currentfill}%
\pgfsetlinewidth{2.007500pt}%
\definecolor{currentstroke}{rgb}{1.000000,0.000000,0.000000}%
\pgfsetstrokecolor{currentstroke}%
\pgfsetdash{}{0pt}%
\pgfpathmoveto{\pgfqpoint{3.394872in}{1.236844in}}%
\pgfpathlineto{\pgfqpoint{3.478206in}{1.236844in}}%
\pgfpathmoveto{\pgfqpoint{3.436539in}{1.195178in}}%
\pgfpathlineto{\pgfqpoint{3.436539in}{1.278511in}}%
\pgfusepath{stroke,fill}%
\end{pgfscope}%
\begin{pgfscope}%
\pgfpathrectangle{\pgfqpoint{0.456635in}{0.589870in}}{\pgfqpoint{4.376471in}{0.972632in}}%
\pgfusepath{clip}%
\pgfsetbuttcap%
\pgfsetroundjoin%
\definecolor{currentfill}{rgb}{1.000000,0.000000,0.000000}%
\pgfsetfillcolor{currentfill}%
\pgfsetlinewidth{2.007500pt}%
\definecolor{currentstroke}{rgb}{1.000000,0.000000,0.000000}%
\pgfsetstrokecolor{currentstroke}%
\pgfsetdash{}{0pt}%
\pgfpathmoveto{\pgfqpoint{3.769350in}{0.728450in}}%
\pgfpathlineto{\pgfqpoint{3.852683in}{0.728450in}}%
\pgfpathmoveto{\pgfqpoint{3.811016in}{0.686783in}}%
\pgfpathlineto{\pgfqpoint{3.811016in}{0.770116in}}%
\pgfusepath{stroke,fill}%
\end{pgfscope}%
\begin{pgfscope}%
\pgfpathrectangle{\pgfqpoint{0.456635in}{0.589870in}}{\pgfqpoint{4.376471in}{0.972632in}}%
\pgfusepath{clip}%
\pgfsetbuttcap%
\pgfsetroundjoin%
\definecolor{currentfill}{rgb}{1.000000,0.000000,0.000000}%
\pgfsetfillcolor{currentfill}%
\pgfsetlinewidth{2.007500pt}%
\definecolor{currentstroke}{rgb}{1.000000,0.000000,0.000000}%
\pgfsetstrokecolor{currentstroke}%
\pgfsetdash{}{0pt}%
\pgfpathmoveto{\pgfqpoint{1.362333in}{1.381339in}}%
\pgfpathlineto{\pgfqpoint{1.445666in}{1.381339in}}%
\pgfpathmoveto{\pgfqpoint{1.403999in}{1.339672in}}%
\pgfpathlineto{\pgfqpoint{1.403999in}{1.423006in}}%
\pgfusepath{stroke,fill}%
\end{pgfscope}%
\begin{pgfscope}%
\pgfpathrectangle{\pgfqpoint{0.456635in}{0.589870in}}{\pgfqpoint{4.376471in}{0.972632in}}%
\pgfusepath{clip}%
\pgfsetbuttcap%
\pgfsetroundjoin%
\definecolor{currentfill}{rgb}{1.000000,0.000000,0.000000}%
\pgfsetfillcolor{currentfill}%
\pgfsetlinewidth{2.007500pt}%
\definecolor{currentstroke}{rgb}{1.000000,0.000000,0.000000}%
\pgfsetstrokecolor{currentstroke}%
\pgfsetdash{}{0pt}%
\pgfpathmoveto{\pgfqpoint{4.686088in}{1.119400in}}%
\pgfpathlineto{\pgfqpoint{4.769422in}{1.119400in}}%
\pgfpathmoveto{\pgfqpoint{4.727755in}{1.077734in}}%
\pgfpathlineto{\pgfqpoint{4.727755in}{1.161067in}}%
\pgfusepath{stroke,fill}%
\end{pgfscope}%
\begin{pgfscope}%
\pgfpathrectangle{\pgfqpoint{0.456635in}{0.589870in}}{\pgfqpoint{4.376471in}{0.972632in}}%
\pgfusepath{clip}%
\pgfsetbuttcap%
\pgfsetroundjoin%
\definecolor{currentfill}{rgb}{1.000000,0.000000,0.000000}%
\pgfsetfillcolor{currentfill}%
\pgfsetlinewidth{2.007500pt}%
\definecolor{currentstroke}{rgb}{1.000000,0.000000,0.000000}%
\pgfsetstrokecolor{currentstroke}%
\pgfsetdash{}{0pt}%
\pgfpathmoveto{\pgfqpoint{4.204791in}{1.071652in}}%
\pgfpathlineto{\pgfqpoint{4.288125in}{1.071652in}}%
\pgfpathmoveto{\pgfqpoint{4.246458in}{1.029985in}}%
\pgfpathlineto{\pgfqpoint{4.246458in}{1.113319in}}%
\pgfusepath{stroke,fill}%
\end{pgfscope}%
\begin{pgfscope}%
\pgfpathrectangle{\pgfqpoint{0.456635in}{0.589870in}}{\pgfqpoint{4.376471in}{0.972632in}}%
\pgfusepath{clip}%
\pgfsetbuttcap%
\pgfsetroundjoin%
\definecolor{currentfill}{rgb}{1.000000,0.000000,0.000000}%
\pgfsetfillcolor{currentfill}%
\pgfsetlinewidth{2.007500pt}%
\definecolor{currentstroke}{rgb}{1.000000,0.000000,0.000000}%
\pgfsetstrokecolor{currentstroke}%
\pgfsetdash{}{0pt}%
\pgfpathmoveto{\pgfqpoint{2.033699in}{1.211124in}}%
\pgfpathlineto{\pgfqpoint{2.117033in}{1.211124in}}%
\pgfpathmoveto{\pgfqpoint{2.075366in}{1.169457in}}%
\pgfpathlineto{\pgfqpoint{2.075366in}{1.252790in}}%
\pgfusepath{stroke,fill}%
\end{pgfscope}%
\begin{pgfscope}%
\pgfpathrectangle{\pgfqpoint{0.456635in}{0.589870in}}{\pgfqpoint{4.376471in}{0.972632in}}%
\pgfusepath{clip}%
\pgfsetbuttcap%
\pgfsetroundjoin%
\definecolor{currentfill}{rgb}{1.000000,0.000000,0.000000}%
\pgfsetfillcolor{currentfill}%
\pgfsetlinewidth{2.007500pt}%
\definecolor{currentstroke}{rgb}{1.000000,0.000000,0.000000}%
\pgfsetstrokecolor{currentstroke}%
\pgfsetdash{}{0pt}%
\pgfpathmoveto{\pgfqpoint{1.926864in}{1.177276in}}%
\pgfpathlineto{\pgfqpoint{2.010197in}{1.177276in}}%
\pgfpathmoveto{\pgfqpoint{1.968531in}{1.135610in}}%
\pgfpathlineto{\pgfqpoint{1.968531in}{1.218943in}}%
\pgfusepath{stroke,fill}%
\end{pgfscope}%
\begin{pgfscope}%
\pgfpathrectangle{\pgfqpoint{0.456635in}{0.589870in}}{\pgfqpoint{4.376471in}{0.972632in}}%
\pgfusepath{clip}%
\pgfsetbuttcap%
\pgfsetroundjoin%
\definecolor{currentfill}{rgb}{1.000000,0.000000,0.000000}%
\pgfsetfillcolor{currentfill}%
\pgfsetlinewidth{2.007500pt}%
\definecolor{currentstroke}{rgb}{1.000000,0.000000,0.000000}%
\pgfsetstrokecolor{currentstroke}%
\pgfsetdash{}{0pt}%
\pgfpathmoveto{\pgfqpoint{1.932394in}{1.124583in}}%
\pgfpathlineto{\pgfqpoint{2.015728in}{1.124583in}}%
\pgfpathmoveto{\pgfqpoint{1.974061in}{1.082916in}}%
\pgfpathlineto{\pgfqpoint{1.974061in}{1.166249in}}%
\pgfusepath{stroke,fill}%
\end{pgfscope}%
\begin{pgfscope}%
\pgfpathrectangle{\pgfqpoint{0.456635in}{0.589870in}}{\pgfqpoint{4.376471in}{0.972632in}}%
\pgfusepath{clip}%
\pgfsetbuttcap%
\pgfsetroundjoin%
\definecolor{currentfill}{rgb}{1.000000,0.000000,0.000000}%
\pgfsetfillcolor{currentfill}%
\pgfsetlinewidth{2.007500pt}%
\definecolor{currentstroke}{rgb}{1.000000,0.000000,0.000000}%
\pgfsetstrokecolor{currentstroke}%
\pgfsetdash{}{0pt}%
\pgfpathmoveto{\pgfqpoint{2.355469in}{1.146587in}}%
\pgfpathlineto{\pgfqpoint{2.438802in}{1.146587in}}%
\pgfpathmoveto{\pgfqpoint{2.397135in}{1.104920in}}%
\pgfpathlineto{\pgfqpoint{2.397135in}{1.188254in}}%
\pgfusepath{stroke,fill}%
\end{pgfscope}%
\begin{pgfscope}%
\pgfpathrectangle{\pgfqpoint{0.456635in}{0.589870in}}{\pgfqpoint{4.376471in}{0.972632in}}%
\pgfusepath{clip}%
\pgfsetbuttcap%
\pgfsetroundjoin%
\definecolor{currentfill}{rgb}{1.000000,0.000000,0.000000}%
\pgfsetfillcolor{currentfill}%
\pgfsetlinewidth{2.007500pt}%
\definecolor{currentstroke}{rgb}{1.000000,0.000000,0.000000}%
\pgfsetstrokecolor{currentstroke}%
\pgfsetdash{}{0pt}%
\pgfpathmoveto{\pgfqpoint{3.127528in}{1.021373in}}%
\pgfpathlineto{\pgfqpoint{3.210861in}{1.021373in}}%
\pgfpathmoveto{\pgfqpoint{3.169194in}{0.979707in}}%
\pgfpathlineto{\pgfqpoint{3.169194in}{1.063040in}}%
\pgfusepath{stroke,fill}%
\end{pgfscope}%
\begin{pgfscope}%
\pgfpathrectangle{\pgfqpoint{0.456635in}{0.589870in}}{\pgfqpoint{4.376471in}{0.972632in}}%
\pgfusepath{clip}%
\pgfsetbuttcap%
\pgfsetroundjoin%
\definecolor{currentfill}{rgb}{1.000000,0.000000,0.000000}%
\pgfsetfillcolor{currentfill}%
\pgfsetlinewidth{2.007500pt}%
\definecolor{currentstroke}{rgb}{1.000000,0.000000,0.000000}%
\pgfsetstrokecolor{currentstroke}%
\pgfsetdash{}{0pt}%
\pgfpathmoveto{\pgfqpoint{2.802578in}{0.614125in}}%
\pgfpathlineto{\pgfqpoint{2.885912in}{0.614125in}}%
\pgfpathmoveto{\pgfqpoint{2.844245in}{0.572458in}}%
\pgfpathlineto{\pgfqpoint{2.844245in}{0.655791in}}%
\pgfusepath{stroke,fill}%
\end{pgfscope}%
\begin{pgfscope}%
\pgfpathrectangle{\pgfqpoint{0.456635in}{0.589870in}}{\pgfqpoint{4.376471in}{0.972632in}}%
\pgfusepath{clip}%
\pgfsetbuttcap%
\pgfsetroundjoin%
\definecolor{currentfill}{rgb}{1.000000,0.000000,0.000000}%
\pgfsetfillcolor{currentfill}%
\pgfsetlinewidth{2.007500pt}%
\definecolor{currentstroke}{rgb}{1.000000,0.000000,0.000000}%
\pgfsetstrokecolor{currentstroke}%
\pgfsetdash{}{0pt}%
\pgfpathmoveto{\pgfqpoint{2.309907in}{1.126443in}}%
\pgfpathlineto{\pgfqpoint{2.393241in}{1.126443in}}%
\pgfpathmoveto{\pgfqpoint{2.351574in}{1.084776in}}%
\pgfpathlineto{\pgfqpoint{2.351574in}{1.168110in}}%
\pgfusepath{stroke,fill}%
\end{pgfscope}%
\begin{pgfscope}%
\pgfpathrectangle{\pgfqpoint{0.456635in}{0.589870in}}{\pgfqpoint{4.376471in}{0.972632in}}%
\pgfusepath{clip}%
\pgfsetbuttcap%
\pgfsetroundjoin%
\definecolor{currentfill}{rgb}{1.000000,0.000000,0.000000}%
\pgfsetfillcolor{currentfill}%
\pgfsetlinewidth{2.007500pt}%
\definecolor{currentstroke}{rgb}{1.000000,0.000000,0.000000}%
\pgfsetstrokecolor{currentstroke}%
\pgfsetdash{}{0pt}%
\pgfpathmoveto{\pgfqpoint{3.432468in}{1.087515in}}%
\pgfpathlineto{\pgfqpoint{3.515801in}{1.087515in}}%
\pgfpathmoveto{\pgfqpoint{3.474134in}{1.045848in}}%
\pgfpathlineto{\pgfqpoint{3.474134in}{1.129181in}}%
\pgfusepath{stroke,fill}%
\end{pgfscope}%
\begin{pgfscope}%
\pgfpathrectangle{\pgfqpoint{0.456635in}{0.589870in}}{\pgfqpoint{4.376471in}{0.972632in}}%
\pgfusepath{clip}%
\pgfsetbuttcap%
\pgfsetroundjoin%
\definecolor{currentfill}{rgb}{1.000000,0.000000,0.000000}%
\pgfsetfillcolor{currentfill}%
\pgfsetlinewidth{2.007500pt}%
\definecolor{currentstroke}{rgb}{1.000000,0.000000,0.000000}%
\pgfsetstrokecolor{currentstroke}%
\pgfsetdash{}{0pt}%
\pgfpathmoveto{\pgfqpoint{1.778655in}{1.110328in}}%
\pgfpathlineto{\pgfqpoint{1.861989in}{1.110328in}}%
\pgfpathmoveto{\pgfqpoint{1.820322in}{1.068661in}}%
\pgfpathlineto{\pgfqpoint{1.820322in}{1.151995in}}%
\pgfusepath{stroke,fill}%
\end{pgfscope}%
\begin{pgfscope}%
\pgfpathrectangle{\pgfqpoint{0.456635in}{0.589870in}}{\pgfqpoint{4.376471in}{0.972632in}}%
\pgfusepath{clip}%
\pgfsetbuttcap%
\pgfsetroundjoin%
\definecolor{currentfill}{rgb}{1.000000,0.000000,0.000000}%
\pgfsetfillcolor{currentfill}%
\pgfsetlinewidth{2.007500pt}%
\definecolor{currentstroke}{rgb}{1.000000,0.000000,0.000000}%
\pgfsetstrokecolor{currentstroke}%
\pgfsetdash{}{0pt}%
\pgfpathmoveto{\pgfqpoint{2.313113in}{1.231643in}}%
\pgfpathlineto{\pgfqpoint{2.396446in}{1.231643in}}%
\pgfpathmoveto{\pgfqpoint{2.354779in}{1.189977in}}%
\pgfpathlineto{\pgfqpoint{2.354779in}{1.273310in}}%
\pgfusepath{stroke,fill}%
\end{pgfscope}%
\begin{pgfscope}%
\pgfpathrectangle{\pgfqpoint{0.456635in}{0.589870in}}{\pgfqpoint{4.376471in}{0.972632in}}%
\pgfusepath{clip}%
\pgfsetbuttcap%
\pgfsetroundjoin%
\definecolor{currentfill}{rgb}{1.000000,0.000000,0.000000}%
\pgfsetfillcolor{currentfill}%
\pgfsetlinewidth{2.007500pt}%
\definecolor{currentstroke}{rgb}{1.000000,0.000000,0.000000}%
\pgfsetstrokecolor{currentstroke}%
\pgfsetdash{}{0pt}%
\pgfpathmoveto{\pgfqpoint{2.572960in}{0.713376in}}%
\pgfpathlineto{\pgfqpoint{2.656294in}{0.713376in}}%
\pgfpathmoveto{\pgfqpoint{2.614627in}{0.671710in}}%
\pgfpathlineto{\pgfqpoint{2.614627in}{0.755043in}}%
\pgfusepath{stroke,fill}%
\end{pgfscope}%
\begin{pgfscope}%
\pgfpathrectangle{\pgfqpoint{0.456635in}{0.589870in}}{\pgfqpoint{4.376471in}{0.972632in}}%
\pgfusepath{clip}%
\pgfsetbuttcap%
\pgfsetroundjoin%
\definecolor{currentfill}{rgb}{1.000000,0.000000,0.000000}%
\pgfsetfillcolor{currentfill}%
\pgfsetlinewidth{2.007500pt}%
\definecolor{currentstroke}{rgb}{1.000000,0.000000,0.000000}%
\pgfsetstrokecolor{currentstroke}%
\pgfsetdash{}{0pt}%
\pgfpathmoveto{\pgfqpoint{2.887044in}{0.834575in}}%
\pgfpathlineto{\pgfqpoint{2.970378in}{0.834575in}}%
\pgfpathmoveto{\pgfqpoint{2.928711in}{0.792908in}}%
\pgfpathlineto{\pgfqpoint{2.928711in}{0.876242in}}%
\pgfusepath{stroke,fill}%
\end{pgfscope}%
\begin{pgfscope}%
\pgfpathrectangle{\pgfqpoint{0.456635in}{0.589870in}}{\pgfqpoint{4.376471in}{0.972632in}}%
\pgfusepath{clip}%
\pgfsetbuttcap%
\pgfsetroundjoin%
\definecolor{currentfill}{rgb}{1.000000,0.000000,0.000000}%
\pgfsetfillcolor{currentfill}%
\pgfsetlinewidth{2.007500pt}%
\definecolor{currentstroke}{rgb}{1.000000,0.000000,0.000000}%
\pgfsetstrokecolor{currentstroke}%
\pgfsetdash{}{0pt}%
\pgfpathmoveto{\pgfqpoint{4.039302in}{0.936434in}}%
\pgfpathlineto{\pgfqpoint{4.122636in}{0.936434in}}%
\pgfpathmoveto{\pgfqpoint{4.080969in}{0.894768in}}%
\pgfpathlineto{\pgfqpoint{4.080969in}{0.978101in}}%
\pgfusepath{stroke,fill}%
\end{pgfscope}%
\begin{pgfscope}%
\pgfpathrectangle{\pgfqpoint{0.456635in}{0.589870in}}{\pgfqpoint{4.376471in}{0.972632in}}%
\pgfusepath{clip}%
\pgfsetbuttcap%
\pgfsetroundjoin%
\definecolor{currentfill}{rgb}{1.000000,0.000000,0.000000}%
\pgfsetfillcolor{currentfill}%
\pgfsetlinewidth{2.007500pt}%
\definecolor{currentstroke}{rgb}{1.000000,0.000000,0.000000}%
\pgfsetstrokecolor{currentstroke}%
\pgfsetdash{}{0pt}%
\pgfpathmoveto{\pgfqpoint{1.989356in}{1.087584in}}%
\pgfpathlineto{\pgfqpoint{2.072689in}{1.087584in}}%
\pgfpathmoveto{\pgfqpoint{2.031023in}{1.045918in}}%
\pgfpathlineto{\pgfqpoint{2.031023in}{1.129251in}}%
\pgfusepath{stroke,fill}%
\end{pgfscope}%
\begin{pgfscope}%
\pgfpathrectangle{\pgfqpoint{0.456635in}{0.589870in}}{\pgfqpoint{4.376471in}{0.972632in}}%
\pgfusepath{clip}%
\pgfsetbuttcap%
\pgfsetroundjoin%
\definecolor{currentfill}{rgb}{1.000000,0.000000,0.000000}%
\pgfsetfillcolor{currentfill}%
\pgfsetlinewidth{2.007500pt}%
\definecolor{currentstroke}{rgb}{1.000000,0.000000,0.000000}%
\pgfsetstrokecolor{currentstroke}%
\pgfsetdash{}{0pt}%
\pgfpathmoveto{\pgfqpoint{3.090688in}{1.073819in}}%
\pgfpathlineto{\pgfqpoint{3.174022in}{1.073819in}}%
\pgfpathmoveto{\pgfqpoint{3.132355in}{1.032152in}}%
\pgfpathlineto{\pgfqpoint{3.132355in}{1.115486in}}%
\pgfusepath{stroke,fill}%
\end{pgfscope}%
\begin{pgfscope}%
\pgfpathrectangle{\pgfqpoint{0.456635in}{0.589870in}}{\pgfqpoint{4.376471in}{0.972632in}}%
\pgfusepath{clip}%
\pgfsetbuttcap%
\pgfsetroundjoin%
\definecolor{currentfill}{rgb}{1.000000,0.000000,0.000000}%
\pgfsetfillcolor{currentfill}%
\pgfsetlinewidth{2.007500pt}%
\definecolor{currentstroke}{rgb}{1.000000,0.000000,0.000000}%
\pgfsetstrokecolor{currentstroke}%
\pgfsetdash{}{0pt}%
\pgfpathmoveto{\pgfqpoint{3.364411in}{1.267788in}}%
\pgfpathlineto{\pgfqpoint{3.447744in}{1.267788in}}%
\pgfpathmoveto{\pgfqpoint{3.406077in}{1.226121in}}%
\pgfpathlineto{\pgfqpoint{3.406077in}{1.309455in}}%
\pgfusepath{stroke,fill}%
\end{pgfscope}%
\begin{pgfscope}%
\pgfpathrectangle{\pgfqpoint{0.456635in}{0.589870in}}{\pgfqpoint{4.376471in}{0.972632in}}%
\pgfusepath{clip}%
\pgfsetbuttcap%
\pgfsetroundjoin%
\definecolor{currentfill}{rgb}{1.000000,0.000000,0.000000}%
\pgfsetfillcolor{currentfill}%
\pgfsetlinewidth{2.007500pt}%
\definecolor{currentstroke}{rgb}{1.000000,0.000000,0.000000}%
\pgfsetstrokecolor{currentstroke}%
\pgfsetdash{}{0pt}%
\pgfpathmoveto{\pgfqpoint{1.452894in}{1.312221in}}%
\pgfpathlineto{\pgfqpoint{1.536227in}{1.312221in}}%
\pgfpathmoveto{\pgfqpoint{1.494560in}{1.270554in}}%
\pgfpathlineto{\pgfqpoint{1.494560in}{1.353887in}}%
\pgfusepath{stroke,fill}%
\end{pgfscope}%
\begin{pgfscope}%
\pgfpathrectangle{\pgfqpoint{0.456635in}{0.589870in}}{\pgfqpoint{4.376471in}{0.972632in}}%
\pgfusepath{clip}%
\pgfsetbuttcap%
\pgfsetroundjoin%
\definecolor{currentfill}{rgb}{0.000000,0.000000,0.000000}%
\pgfsetfillcolor{currentfill}%
\pgfsetlinewidth{1.003750pt}%
\definecolor{currentstroke}{rgb}{0.000000,0.000000,0.000000}%
\pgfsetstrokecolor{currentstroke}%
\pgfsetdash{}{0pt}%
\pgfsys@defobject{currentmarker}{\pgfqpoint{-0.020833in}{-0.020833in}}{\pgfqpoint{0.020833in}{0.020833in}}{%
\pgfpathmoveto{\pgfqpoint{0.000000in}{-0.020833in}}%
\pgfpathcurveto{\pgfqpoint{0.005525in}{-0.020833in}}{\pgfqpoint{0.010825in}{-0.018638in}}{\pgfqpoint{0.014731in}{-0.014731in}}%
\pgfpathcurveto{\pgfqpoint{0.018638in}{-0.010825in}}{\pgfqpoint{0.020833in}{-0.005525in}}{\pgfqpoint{0.020833in}{0.000000in}}%
\pgfpathcurveto{\pgfqpoint{0.020833in}{0.005525in}}{\pgfqpoint{0.018638in}{0.010825in}}{\pgfqpoint{0.014731in}{0.014731in}}%
\pgfpathcurveto{\pgfqpoint{0.010825in}{0.018638in}}{\pgfqpoint{0.005525in}{0.020833in}}{\pgfqpoint{0.000000in}{0.020833in}}%
\pgfpathcurveto{\pgfqpoint{-0.005525in}{0.020833in}}{\pgfqpoint{-0.010825in}{0.018638in}}{\pgfqpoint{-0.014731in}{0.014731in}}%
\pgfpathcurveto{\pgfqpoint{-0.018638in}{0.010825in}}{\pgfqpoint{-0.020833in}{0.005525in}}{\pgfqpoint{-0.020833in}{0.000000in}}%
\pgfpathcurveto{\pgfqpoint{-0.020833in}{-0.005525in}}{\pgfqpoint{-0.018638in}{-0.010825in}}{\pgfqpoint{-0.014731in}{-0.014731in}}%
\pgfpathcurveto{\pgfqpoint{-0.010825in}{-0.018638in}}{\pgfqpoint{-0.005525in}{-0.020833in}}{\pgfqpoint{0.000000in}{-0.020833in}}%
\pgfpathclose%
\pgfusepath{stroke,fill}%
}%
\begin{pgfscope}%
\pgfsys@transformshift{1.331929in}{1.385080in}%
\pgfsys@useobject{currentmarker}{}%
\end{pgfscope}%
\begin{pgfscope}%
\pgfsys@transformshift{1.349523in}{1.413058in}%
\pgfsys@useobject{currentmarker}{}%
\end{pgfscope}%
\begin{pgfscope}%
\pgfsys@transformshift{1.367117in}{1.451375in}%
\pgfsys@useobject{currentmarker}{}%
\end{pgfscope}%
\begin{pgfscope}%
\pgfsys@transformshift{1.384711in}{1.489412in}%
\pgfsys@useobject{currentmarker}{}%
\end{pgfscope}%
\begin{pgfscope}%
\pgfsys@transformshift{1.402305in}{1.314316in}%
\pgfsys@useobject{currentmarker}{}%
\end{pgfscope}%
\begin{pgfscope}%
\pgfsys@transformshift{1.419899in}{1.316074in}%
\pgfsys@useobject{currentmarker}{}%
\end{pgfscope}%
\begin{pgfscope}%
\pgfsys@transformshift{1.437493in}{1.194748in}%
\pgfsys@useobject{currentmarker}{}%
\end{pgfscope}%
\begin{pgfscope}%
\pgfsys@transformshift{1.455086in}{1.157734in}%
\pgfsys@useobject{currentmarker}{}%
\end{pgfscope}%
\begin{pgfscope}%
\pgfsys@transformshift{1.472680in}{1.333686in}%
\pgfsys@useobject{currentmarker}{}%
\end{pgfscope}%
\begin{pgfscope}%
\pgfsys@transformshift{1.490274in}{1.361753in}%
\pgfsys@useobject{currentmarker}{}%
\end{pgfscope}%
\begin{pgfscope}%
\pgfsys@transformshift{1.507868in}{1.190773in}%
\pgfsys@useobject{currentmarker}{}%
\end{pgfscope}%
\begin{pgfscope}%
\pgfsys@transformshift{1.525462in}{1.274401in}%
\pgfsys@useobject{currentmarker}{}%
\end{pgfscope}%
\begin{pgfscope}%
\pgfsys@transformshift{1.543056in}{1.185141in}%
\pgfsys@useobject{currentmarker}{}%
\end{pgfscope}%
\begin{pgfscope}%
\pgfsys@transformshift{1.560649in}{1.060203in}%
\pgfsys@useobject{currentmarker}{}%
\end{pgfscope}%
\begin{pgfscope}%
\pgfsys@transformshift{1.578243in}{1.140689in}%
\pgfsys@useobject{currentmarker}{}%
\end{pgfscope}%
\begin{pgfscope}%
\pgfsys@transformshift{1.595837in}{1.240004in}%
\pgfsys@useobject{currentmarker}{}%
\end{pgfscope}%
\begin{pgfscope}%
\pgfsys@transformshift{1.613431in}{1.062375in}%
\pgfsys@useobject{currentmarker}{}%
\end{pgfscope}%
\begin{pgfscope}%
\pgfsys@transformshift{1.631025in}{1.208193in}%
\pgfsys@useobject{currentmarker}{}%
\end{pgfscope}%
\begin{pgfscope}%
\pgfsys@transformshift{1.648619in}{0.769848in}%
\pgfsys@useobject{currentmarker}{}%
\end{pgfscope}%
\begin{pgfscope}%
\pgfsys@transformshift{1.666213in}{1.106153in}%
\pgfsys@useobject{currentmarker}{}%
\end{pgfscope}%
\begin{pgfscope}%
\pgfsys@transformshift{1.683806in}{1.021382in}%
\pgfsys@useobject{currentmarker}{}%
\end{pgfscope}%
\begin{pgfscope}%
\pgfsys@transformshift{1.701400in}{0.974058in}%
\pgfsys@useobject{currentmarker}{}%
\end{pgfscope}%
\begin{pgfscope}%
\pgfsys@transformshift{1.718994in}{1.007564in}%
\pgfsys@useobject{currentmarker}{}%
\end{pgfscope}%
\begin{pgfscope}%
\pgfsys@transformshift{1.736588in}{0.792933in}%
\pgfsys@useobject{currentmarker}{}%
\end{pgfscope}%
\begin{pgfscope}%
\pgfsys@transformshift{1.754182in}{0.970186in}%
\pgfsys@useobject{currentmarker}{}%
\end{pgfscope}%
\begin{pgfscope}%
\pgfsys@transformshift{1.771776in}{1.028815in}%
\pgfsys@useobject{currentmarker}{}%
\end{pgfscope}%
\begin{pgfscope}%
\pgfsys@transformshift{1.789370in}{1.144550in}%
\pgfsys@useobject{currentmarker}{}%
\end{pgfscope}%
\begin{pgfscope}%
\pgfsys@transformshift{1.806963in}{0.946398in}%
\pgfsys@useobject{currentmarker}{}%
\end{pgfscope}%
\begin{pgfscope}%
\pgfsys@transformshift{1.824557in}{0.922890in}%
\pgfsys@useobject{currentmarker}{}%
\end{pgfscope}%
\begin{pgfscope}%
\pgfsys@transformshift{1.842151in}{0.961551in}%
\pgfsys@useobject{currentmarker}{}%
\end{pgfscope}%
\begin{pgfscope}%
\pgfsys@transformshift{1.859745in}{1.114268in}%
\pgfsys@useobject{currentmarker}{}%
\end{pgfscope}%
\begin{pgfscope}%
\pgfsys@transformshift{1.877339in}{1.065369in}%
\pgfsys@useobject{currentmarker}{}%
\end{pgfscope}%
\begin{pgfscope}%
\pgfsys@transformshift{1.894933in}{0.990165in}%
\pgfsys@useobject{currentmarker}{}%
\end{pgfscope}%
\begin{pgfscope}%
\pgfsys@transformshift{1.912526in}{1.108681in}%
\pgfsys@useobject{currentmarker}{}%
\end{pgfscope}%
\begin{pgfscope}%
\pgfsys@transformshift{1.930120in}{1.080233in}%
\pgfsys@useobject{currentmarker}{}%
\end{pgfscope}%
\begin{pgfscope}%
\pgfsys@transformshift{1.947714in}{1.182941in}%
\pgfsys@useobject{currentmarker}{}%
\end{pgfscope}%
\begin{pgfscope}%
\pgfsys@transformshift{1.965308in}{1.028565in}%
\pgfsys@useobject{currentmarker}{}%
\end{pgfscope}%
\begin{pgfscope}%
\pgfsys@transformshift{1.982902in}{1.081671in}%
\pgfsys@useobject{currentmarker}{}%
\end{pgfscope}%
\begin{pgfscope}%
\pgfsys@transformshift{2.000496in}{1.090395in}%
\pgfsys@useobject{currentmarker}{}%
\end{pgfscope}%
\begin{pgfscope}%
\pgfsys@transformshift{2.018090in}{0.996970in}%
\pgfsys@useobject{currentmarker}{}%
\end{pgfscope}%
\begin{pgfscope}%
\pgfsys@transformshift{2.035683in}{1.190045in}%
\pgfsys@useobject{currentmarker}{}%
\end{pgfscope}%
\begin{pgfscope}%
\pgfsys@transformshift{2.053277in}{1.200760in}%
\pgfsys@useobject{currentmarker}{}%
\end{pgfscope}%
\begin{pgfscope}%
\pgfsys@transformshift{2.070871in}{1.188374in}%
\pgfsys@useobject{currentmarker}{}%
\end{pgfscope}%
\begin{pgfscope}%
\pgfsys@transformshift{2.088465in}{1.176728in}%
\pgfsys@useobject{currentmarker}{}%
\end{pgfscope}%
\begin{pgfscope}%
\pgfsys@transformshift{2.106059in}{1.068654in}%
\pgfsys@useobject{currentmarker}{}%
\end{pgfscope}%
\begin{pgfscope}%
\pgfsys@transformshift{2.123653in}{1.179750in}%
\pgfsys@useobject{currentmarker}{}%
\end{pgfscope}%
\begin{pgfscope}%
\pgfsys@transformshift{2.141247in}{1.196567in}%
\pgfsys@useobject{currentmarker}{}%
\end{pgfscope}%
\begin{pgfscope}%
\pgfsys@transformshift{2.158840in}{1.157399in}%
\pgfsys@useobject{currentmarker}{}%
\end{pgfscope}%
\begin{pgfscope}%
\pgfsys@transformshift{2.176434in}{1.228089in}%
\pgfsys@useobject{currentmarker}{}%
\end{pgfscope}%
\begin{pgfscope}%
\pgfsys@transformshift{2.194028in}{1.289366in}%
\pgfsys@useobject{currentmarker}{}%
\end{pgfscope}%
\begin{pgfscope}%
\pgfsys@transformshift{2.211622in}{1.441699in}%
\pgfsys@useobject{currentmarker}{}%
\end{pgfscope}%
\begin{pgfscope}%
\pgfsys@transformshift{2.229216in}{1.268562in}%
\pgfsys@useobject{currentmarker}{}%
\end{pgfscope}%
\begin{pgfscope}%
\pgfsys@transformshift{2.246810in}{1.275309in}%
\pgfsys@useobject{currentmarker}{}%
\end{pgfscope}%
\begin{pgfscope}%
\pgfsys@transformshift{2.264404in}{1.238053in}%
\pgfsys@useobject{currentmarker}{}%
\end{pgfscope}%
\begin{pgfscope}%
\pgfsys@transformshift{2.281997in}{1.045613in}%
\pgfsys@useobject{currentmarker}{}%
\end{pgfscope}%
\begin{pgfscope}%
\pgfsys@transformshift{2.299591in}{1.229804in}%
\pgfsys@useobject{currentmarker}{}%
\end{pgfscope}%
\begin{pgfscope}%
\pgfsys@transformshift{2.317185in}{1.229164in}%
\pgfsys@useobject{currentmarker}{}%
\end{pgfscope}%
\begin{pgfscope}%
\pgfsys@transformshift{2.334779in}{1.461353in}%
\pgfsys@useobject{currentmarker}{}%
\end{pgfscope}%
\begin{pgfscope}%
\pgfsys@transformshift{2.352373in}{1.179256in}%
\pgfsys@useobject{currentmarker}{}%
\end{pgfscope}%
\begin{pgfscope}%
\pgfsys@transformshift{2.369967in}{1.214585in}%
\pgfsys@useobject{currentmarker}{}%
\end{pgfscope}%
\begin{pgfscope}%
\pgfsys@transformshift{2.387560in}{1.164248in}%
\pgfsys@useobject{currentmarker}{}%
\end{pgfscope}%
\begin{pgfscope}%
\pgfsys@transformshift{2.405154in}{1.031666in}%
\pgfsys@useobject{currentmarker}{}%
\end{pgfscope}%
\begin{pgfscope}%
\pgfsys@transformshift{2.422748in}{1.246886in}%
\pgfsys@useobject{currentmarker}{}%
\end{pgfscope}%
\begin{pgfscope}%
\pgfsys@transformshift{2.440342in}{1.187193in}%
\pgfsys@useobject{currentmarker}{}%
\end{pgfscope}%
\begin{pgfscope}%
\pgfsys@transformshift{2.457936in}{1.170118in}%
\pgfsys@useobject{currentmarker}{}%
\end{pgfscope}%
\begin{pgfscope}%
\pgfsys@transformshift{2.475530in}{0.976039in}%
\pgfsys@useobject{currentmarker}{}%
\end{pgfscope}%
\begin{pgfscope}%
\pgfsys@transformshift{2.493124in}{1.187927in}%
\pgfsys@useobject{currentmarker}{}%
\end{pgfscope}%
\begin{pgfscope}%
\pgfsys@transformshift{2.510717in}{0.881025in}%
\pgfsys@useobject{currentmarker}{}%
\end{pgfscope}%
\begin{pgfscope}%
\pgfsys@transformshift{2.528311in}{1.059593in}%
\pgfsys@useobject{currentmarker}{}%
\end{pgfscope}%
\begin{pgfscope}%
\pgfsys@transformshift{2.545905in}{1.199175in}%
\pgfsys@useobject{currentmarker}{}%
\end{pgfscope}%
\begin{pgfscope}%
\pgfsys@transformshift{2.563499in}{0.854241in}%
\pgfsys@useobject{currentmarker}{}%
\end{pgfscope}%
\begin{pgfscope}%
\pgfsys@transformshift{2.581093in}{0.875015in}%
\pgfsys@useobject{currentmarker}{}%
\end{pgfscope}%
\begin{pgfscope}%
\pgfsys@transformshift{2.598687in}{0.920919in}%
\pgfsys@useobject{currentmarker}{}%
\end{pgfscope}%
\begin{pgfscope}%
\pgfsys@transformshift{2.616281in}{0.839082in}%
\pgfsys@useobject{currentmarker}{}%
\end{pgfscope}%
\begin{pgfscope}%
\pgfsys@transformshift{2.633874in}{0.713278in}%
\pgfsys@useobject{currentmarker}{}%
\end{pgfscope}%
\begin{pgfscope}%
\pgfsys@transformshift{2.651468in}{0.858825in}%
\pgfsys@useobject{currentmarker}{}%
\end{pgfscope}%
\begin{pgfscope}%
\pgfsys@transformshift{2.669062in}{0.727110in}%
\pgfsys@useobject{currentmarker}{}%
\end{pgfscope}%
\begin{pgfscope}%
\pgfsys@transformshift{2.686656in}{0.867100in}%
\pgfsys@useobject{currentmarker}{}%
\end{pgfscope}%
\begin{pgfscope}%
\pgfsys@transformshift{2.704250in}{0.712003in}%
\pgfsys@useobject{currentmarker}{}%
\end{pgfscope}%
\begin{pgfscope}%
\pgfsys@transformshift{2.721844in}{0.950007in}%
\pgfsys@useobject{currentmarker}{}%
\end{pgfscope}%
\begin{pgfscope}%
\pgfsys@transformshift{2.739438in}{0.703324in}%
\pgfsys@useobject{currentmarker}{}%
\end{pgfscope}%
\begin{pgfscope}%
\pgfsys@transformshift{2.757031in}{0.741730in}%
\pgfsys@useobject{currentmarker}{}%
\end{pgfscope}%
\begin{pgfscope}%
\pgfsys@transformshift{2.774625in}{0.850504in}%
\pgfsys@useobject{currentmarker}{}%
\end{pgfscope}%
\begin{pgfscope}%
\pgfsys@transformshift{2.792219in}{0.639189in}%
\pgfsys@useobject{currentmarker}{}%
\end{pgfscope}%
\begin{pgfscope}%
\pgfsys@transformshift{2.809813in}{0.784873in}%
\pgfsys@useobject{currentmarker}{}%
\end{pgfscope}%
\begin{pgfscope}%
\pgfsys@transformshift{2.827407in}{0.894320in}%
\pgfsys@useobject{currentmarker}{}%
\end{pgfscope}%
\begin{pgfscope}%
\pgfsys@transformshift{2.845001in}{0.601195in}%
\pgfsys@useobject{currentmarker}{}%
\end{pgfscope}%
\begin{pgfscope}%
\pgfsys@transformshift{2.862594in}{0.787018in}%
\pgfsys@useobject{currentmarker}{}%
\end{pgfscope}%
\begin{pgfscope}%
\pgfsys@transformshift{2.880188in}{0.800921in}%
\pgfsys@useobject{currentmarker}{}%
\end{pgfscope}%
\begin{pgfscope}%
\pgfsys@transformshift{2.897782in}{0.862034in}%
\pgfsys@useobject{currentmarker}{}%
\end{pgfscope}%
\begin{pgfscope}%
\pgfsys@transformshift{2.915376in}{0.667594in}%
\pgfsys@useobject{currentmarker}{}%
\end{pgfscope}%
\begin{pgfscope}%
\pgfsys@transformshift{2.932970in}{0.670979in}%
\pgfsys@useobject{currentmarker}{}%
\end{pgfscope}%
\begin{pgfscope}%
\pgfsys@transformshift{2.950564in}{0.871111in}%
\pgfsys@useobject{currentmarker}{}%
\end{pgfscope}%
\begin{pgfscope}%
\pgfsys@transformshift{2.968158in}{0.863270in}%
\pgfsys@useobject{currentmarker}{}%
\end{pgfscope}%
\begin{pgfscope}%
\pgfsys@transformshift{2.985751in}{0.874834in}%
\pgfsys@useobject{currentmarker}{}%
\end{pgfscope}%
\begin{pgfscope}%
\pgfsys@transformshift{3.003345in}{0.901987in}%
\pgfsys@useobject{currentmarker}{}%
\end{pgfscope}%
\begin{pgfscope}%
\pgfsys@transformshift{3.020939in}{0.816395in}%
\pgfsys@useobject{currentmarker}{}%
\end{pgfscope}%
\begin{pgfscope}%
\pgfsys@transformshift{3.038533in}{0.928015in}%
\pgfsys@useobject{currentmarker}{}%
\end{pgfscope}%
\begin{pgfscope}%
\pgfsys@transformshift{3.056127in}{0.953960in}%
\pgfsys@useobject{currentmarker}{}%
\end{pgfscope}%
\begin{pgfscope}%
\pgfsys@transformshift{3.073721in}{0.872064in}%
\pgfsys@useobject{currentmarker}{}%
\end{pgfscope}%
\begin{pgfscope}%
\pgfsys@transformshift{3.091315in}{1.153828in}%
\pgfsys@useobject{currentmarker}{}%
\end{pgfscope}%
\begin{pgfscope}%
\pgfsys@transformshift{3.108908in}{1.033138in}%
\pgfsys@useobject{currentmarker}{}%
\end{pgfscope}%
\begin{pgfscope}%
\pgfsys@transformshift{3.126502in}{0.884544in}%
\pgfsys@useobject{currentmarker}{}%
\end{pgfscope}%
\begin{pgfscope}%
\pgfsys@transformshift{3.144096in}{1.091444in}%
\pgfsys@useobject{currentmarker}{}%
\end{pgfscope}%
\begin{pgfscope}%
\pgfsys@transformshift{3.161690in}{0.945236in}%
\pgfsys@useobject{currentmarker}{}%
\end{pgfscope}%
\begin{pgfscope}%
\pgfsys@transformshift{3.179284in}{1.141979in}%
\pgfsys@useobject{currentmarker}{}%
\end{pgfscope}%
\begin{pgfscope}%
\pgfsys@transformshift{3.196878in}{1.196877in}%
\pgfsys@useobject{currentmarker}{}%
\end{pgfscope}%
\begin{pgfscope}%
\pgfsys@transformshift{3.214472in}{1.012441in}%
\pgfsys@useobject{currentmarker}{}%
\end{pgfscope}%
\begin{pgfscope}%
\pgfsys@transformshift{3.232065in}{1.207973in}%
\pgfsys@useobject{currentmarker}{}%
\end{pgfscope}%
\begin{pgfscope}%
\pgfsys@transformshift{3.249659in}{1.165507in}%
\pgfsys@useobject{currentmarker}{}%
\end{pgfscope}%
\begin{pgfscope}%
\pgfsys@transformshift{3.267253in}{1.218705in}%
\pgfsys@useobject{currentmarker}{}%
\end{pgfscope}%
\begin{pgfscope}%
\pgfsys@transformshift{3.284847in}{1.337626in}%
\pgfsys@useobject{currentmarker}{}%
\end{pgfscope}%
\begin{pgfscope}%
\pgfsys@transformshift{3.302441in}{1.128833in}%
\pgfsys@useobject{currentmarker}{}%
\end{pgfscope}%
\begin{pgfscope}%
\pgfsys@transformshift{3.320035in}{1.083711in}%
\pgfsys@useobject{currentmarker}{}%
\end{pgfscope}%
\begin{pgfscope}%
\pgfsys@transformshift{3.337628in}{1.074419in}%
\pgfsys@useobject{currentmarker}{}%
\end{pgfscope}%
\begin{pgfscope}%
\pgfsys@transformshift{3.355222in}{1.084402in}%
\pgfsys@useobject{currentmarker}{}%
\end{pgfscope}%
\begin{pgfscope}%
\pgfsys@transformshift{3.372816in}{1.159796in}%
\pgfsys@useobject{currentmarker}{}%
\end{pgfscope}%
\begin{pgfscope}%
\pgfsys@transformshift{3.390410in}{1.200769in}%
\pgfsys@useobject{currentmarker}{}%
\end{pgfscope}%
\begin{pgfscope}%
\pgfsys@transformshift{3.408004in}{1.190907in}%
\pgfsys@useobject{currentmarker}{}%
\end{pgfscope}%
\begin{pgfscope}%
\pgfsys@transformshift{3.425598in}{1.241472in}%
\pgfsys@useobject{currentmarker}{}%
\end{pgfscope}%
\begin{pgfscope}%
\pgfsys@transformshift{3.443192in}{1.151963in}%
\pgfsys@useobject{currentmarker}{}%
\end{pgfscope}%
\begin{pgfscope}%
\pgfsys@transformshift{3.460785in}{1.289169in}%
\pgfsys@useobject{currentmarker}{}%
\end{pgfscope}%
\begin{pgfscope}%
\pgfsys@transformshift{3.478379in}{1.104729in}%
\pgfsys@useobject{currentmarker}{}%
\end{pgfscope}%
\begin{pgfscope}%
\pgfsys@transformshift{3.495973in}{1.395285in}%
\pgfsys@useobject{currentmarker}{}%
\end{pgfscope}%
\begin{pgfscope}%
\pgfsys@transformshift{3.513567in}{1.169868in}%
\pgfsys@useobject{currentmarker}{}%
\end{pgfscope}%
\begin{pgfscope}%
\pgfsys@transformshift{3.531161in}{1.005221in}%
\pgfsys@useobject{currentmarker}{}%
\end{pgfscope}%
\begin{pgfscope}%
\pgfsys@transformshift{3.548755in}{0.968116in}%
\pgfsys@useobject{currentmarker}{}%
\end{pgfscope}%
\begin{pgfscope}%
\pgfsys@transformshift{3.566349in}{1.109187in}%
\pgfsys@useobject{currentmarker}{}%
\end{pgfscope}%
\begin{pgfscope}%
\pgfsys@transformshift{3.583942in}{1.020683in}%
\pgfsys@useobject{currentmarker}{}%
\end{pgfscope}%
\begin{pgfscope}%
\pgfsys@transformshift{3.601536in}{1.098203in}%
\pgfsys@useobject{currentmarker}{}%
\end{pgfscope}%
\begin{pgfscope}%
\pgfsys@transformshift{3.619130in}{1.056072in}%
\pgfsys@useobject{currentmarker}{}%
\end{pgfscope}%
\begin{pgfscope}%
\pgfsys@transformshift{3.636724in}{0.982935in}%
\pgfsys@useobject{currentmarker}{}%
\end{pgfscope}%
\begin{pgfscope}%
\pgfsys@transformshift{3.654318in}{0.886840in}%
\pgfsys@useobject{currentmarker}{}%
\end{pgfscope}%
\begin{pgfscope}%
\pgfsys@transformshift{3.671912in}{0.801812in}%
\pgfsys@useobject{currentmarker}{}%
\end{pgfscope}%
\begin{pgfscope}%
\pgfsys@transformshift{3.689506in}{0.893246in}%
\pgfsys@useobject{currentmarker}{}%
\end{pgfscope}%
\begin{pgfscope}%
\pgfsys@transformshift{3.707099in}{1.009182in}%
\pgfsys@useobject{currentmarker}{}%
\end{pgfscope}%
\begin{pgfscope}%
\pgfsys@transformshift{3.724693in}{0.928965in}%
\pgfsys@useobject{currentmarker}{}%
\end{pgfscope}%
\begin{pgfscope}%
\pgfsys@transformshift{3.742287in}{0.767030in}%
\pgfsys@useobject{currentmarker}{}%
\end{pgfscope}%
\begin{pgfscope}%
\pgfsys@transformshift{3.759881in}{0.898039in}%
\pgfsys@useobject{currentmarker}{}%
\end{pgfscope}%
\begin{pgfscope}%
\pgfsys@transformshift{3.777475in}{0.908225in}%
\pgfsys@useobject{currentmarker}{}%
\end{pgfscope}%
\begin{pgfscope}%
\pgfsys@transformshift{3.795069in}{0.769918in}%
\pgfsys@useobject{currentmarker}{}%
\end{pgfscope}%
\begin{pgfscope}%
\pgfsys@transformshift{3.812662in}{0.867043in}%
\pgfsys@useobject{currentmarker}{}%
\end{pgfscope}%
\begin{pgfscope}%
\pgfsys@transformshift{3.830256in}{0.851201in}%
\pgfsys@useobject{currentmarker}{}%
\end{pgfscope}%
\begin{pgfscope}%
\pgfsys@transformshift{3.847850in}{0.725269in}%
\pgfsys@useobject{currentmarker}{}%
\end{pgfscope}%
\begin{pgfscope}%
\pgfsys@transformshift{3.865444in}{0.875095in}%
\pgfsys@useobject{currentmarker}{}%
\end{pgfscope}%
\begin{pgfscope}%
\pgfsys@transformshift{3.883038in}{0.895506in}%
\pgfsys@useobject{currentmarker}{}%
\end{pgfscope}%
\begin{pgfscope}%
\pgfsys@transformshift{3.900632in}{0.950371in}%
\pgfsys@useobject{currentmarker}{}%
\end{pgfscope}%
\begin{pgfscope}%
\pgfsys@transformshift{3.918226in}{0.951484in}%
\pgfsys@useobject{currentmarker}{}%
\end{pgfscope}%
\begin{pgfscope}%
\pgfsys@transformshift{3.935819in}{0.711337in}%
\pgfsys@useobject{currentmarker}{}%
\end{pgfscope}%
\begin{pgfscope}%
\pgfsys@transformshift{3.953413in}{0.764200in}%
\pgfsys@useobject{currentmarker}{}%
\end{pgfscope}%
\begin{pgfscope}%
\pgfsys@transformshift{3.971007in}{0.921754in}%
\pgfsys@useobject{currentmarker}{}%
\end{pgfscope}%
\begin{pgfscope}%
\pgfsys@transformshift{3.988601in}{0.933977in}%
\pgfsys@useobject{currentmarker}{}%
\end{pgfscope}%
\begin{pgfscope}%
\pgfsys@transformshift{4.006195in}{0.948366in}%
\pgfsys@useobject{currentmarker}{}%
\end{pgfscope}%
\begin{pgfscope}%
\pgfsys@transformshift{4.023789in}{1.302598in}%
\pgfsys@useobject{currentmarker}{}%
\end{pgfscope}%
\begin{pgfscope}%
\pgfsys@transformshift{4.041383in}{0.987858in}%
\pgfsys@useobject{currentmarker}{}%
\end{pgfscope}%
\begin{pgfscope}%
\pgfsys@transformshift{4.058976in}{1.064389in}%
\pgfsys@useobject{currentmarker}{}%
\end{pgfscope}%
\begin{pgfscope}%
\pgfsys@transformshift{4.076570in}{1.066721in}%
\pgfsys@useobject{currentmarker}{}%
\end{pgfscope}%
\begin{pgfscope}%
\pgfsys@transformshift{4.094164in}{1.058035in}%
\pgfsys@useobject{currentmarker}{}%
\end{pgfscope}%
\begin{pgfscope}%
\pgfsys@transformshift{4.111758in}{0.983139in}%
\pgfsys@useobject{currentmarker}{}%
\end{pgfscope}%
\begin{pgfscope}%
\pgfsys@transformshift{4.129352in}{1.115904in}%
\pgfsys@useobject{currentmarker}{}%
\end{pgfscope}%
\begin{pgfscope}%
\pgfsys@transformshift{4.146946in}{0.985329in}%
\pgfsys@useobject{currentmarker}{}%
\end{pgfscope}%
\begin{pgfscope}%
\pgfsys@transformshift{4.164540in}{1.064749in}%
\pgfsys@useobject{currentmarker}{}%
\end{pgfscope}%
\begin{pgfscope}%
\pgfsys@transformshift{4.182133in}{1.064973in}%
\pgfsys@useobject{currentmarker}{}%
\end{pgfscope}%
\begin{pgfscope}%
\pgfsys@transformshift{4.199727in}{1.147933in}%
\pgfsys@useobject{currentmarker}{}%
\end{pgfscope}%
\begin{pgfscope}%
\pgfsys@transformshift{4.217321in}{1.399520in}%
\pgfsys@useobject{currentmarker}{}%
\end{pgfscope}%
\begin{pgfscope}%
\pgfsys@transformshift{4.234915in}{1.000873in}%
\pgfsys@useobject{currentmarker}{}%
\end{pgfscope}%
\begin{pgfscope}%
\pgfsys@transformshift{4.252509in}{1.284111in}%
\pgfsys@useobject{currentmarker}{}%
\end{pgfscope}%
\begin{pgfscope}%
\pgfsys@transformshift{4.270103in}{1.074999in}%
\pgfsys@useobject{currentmarker}{}%
\end{pgfscope}%
\begin{pgfscope}%
\pgfsys@transformshift{4.287696in}{1.213490in}%
\pgfsys@useobject{currentmarker}{}%
\end{pgfscope}%
\begin{pgfscope}%
\pgfsys@transformshift{4.305290in}{1.393468in}%
\pgfsys@useobject{currentmarker}{}%
\end{pgfscope}%
\begin{pgfscope}%
\pgfsys@transformshift{4.322884in}{1.310250in}%
\pgfsys@useobject{currentmarker}{}%
\end{pgfscope}%
\begin{pgfscope}%
\pgfsys@transformshift{4.340478in}{1.213755in}%
\pgfsys@useobject{currentmarker}{}%
\end{pgfscope}%
\begin{pgfscope}%
\pgfsys@transformshift{4.358072in}{1.268161in}%
\pgfsys@useobject{currentmarker}{}%
\end{pgfscope}%
\begin{pgfscope}%
\pgfsys@transformshift{4.375666in}{1.425530in}%
\pgfsys@useobject{currentmarker}{}%
\end{pgfscope}%
\begin{pgfscope}%
\pgfsys@transformshift{4.393260in}{1.296978in}%
\pgfsys@useobject{currentmarker}{}%
\end{pgfscope}%
\begin{pgfscope}%
\pgfsys@transformshift{4.410853in}{1.405378in}%
\pgfsys@useobject{currentmarker}{}%
\end{pgfscope}%
\begin{pgfscope}%
\pgfsys@transformshift{4.428447in}{1.398646in}%
\pgfsys@useobject{currentmarker}{}%
\end{pgfscope}%
\begin{pgfscope}%
\pgfsys@transformshift{4.446041in}{1.336657in}%
\pgfsys@useobject{currentmarker}{}%
\end{pgfscope}%
\begin{pgfscope}%
\pgfsys@transformshift{4.463635in}{1.626576in}%
\pgfsys@useobject{currentmarker}{}%
\end{pgfscope}%
\begin{pgfscope}%
\pgfsys@transformshift{4.481229in}{1.478312in}%
\pgfsys@useobject{currentmarker}{}%
\end{pgfscope}%
\begin{pgfscope}%
\pgfsys@transformshift{4.498823in}{1.211687in}%
\pgfsys@useobject{currentmarker}{}%
\end{pgfscope}%
\begin{pgfscope}%
\pgfsys@transformshift{4.516417in}{1.436632in}%
\pgfsys@useobject{currentmarker}{}%
\end{pgfscope}%
\begin{pgfscope}%
\pgfsys@transformshift{4.534010in}{1.349718in}%
\pgfsys@useobject{currentmarker}{}%
\end{pgfscope}%
\begin{pgfscope}%
\pgfsys@transformshift{4.551604in}{1.500388in}%
\pgfsys@useobject{currentmarker}{}%
\end{pgfscope}%
\begin{pgfscope}%
\pgfsys@transformshift{4.569198in}{1.329309in}%
\pgfsys@useobject{currentmarker}{}%
\end{pgfscope}%
\begin{pgfscope}%
\pgfsys@transformshift{4.586792in}{1.392001in}%
\pgfsys@useobject{currentmarker}{}%
\end{pgfscope}%
\begin{pgfscope}%
\pgfsys@transformshift{4.604386in}{1.447380in}%
\pgfsys@useobject{currentmarker}{}%
\end{pgfscope}%
\begin{pgfscope}%
\pgfsys@transformshift{4.621980in}{1.475239in}%
\pgfsys@useobject{currentmarker}{}%
\end{pgfscope}%
\begin{pgfscope}%
\pgfsys@transformshift{4.639573in}{1.256099in}%
\pgfsys@useobject{currentmarker}{}%
\end{pgfscope}%
\begin{pgfscope}%
\pgfsys@transformshift{4.657167in}{1.333048in}%
\pgfsys@useobject{currentmarker}{}%
\end{pgfscope}%
\begin{pgfscope}%
\pgfsys@transformshift{4.674761in}{1.307280in}%
\pgfsys@useobject{currentmarker}{}%
\end{pgfscope}%
\begin{pgfscope}%
\pgfsys@transformshift{4.692355in}{1.277088in}%
\pgfsys@useobject{currentmarker}{}%
\end{pgfscope}%
\begin{pgfscope}%
\pgfsys@transformshift{4.709949in}{1.509648in}%
\pgfsys@useobject{currentmarker}{}%
\end{pgfscope}%
\begin{pgfscope}%
\pgfsys@transformshift{4.727543in}{1.359130in}%
\pgfsys@useobject{currentmarker}{}%
\end{pgfscope}%
\begin{pgfscope}%
\pgfsys@transformshift{4.745137in}{1.177696in}%
\pgfsys@useobject{currentmarker}{}%
\end{pgfscope}%
\begin{pgfscope}%
\pgfsys@transformshift{4.762730in}{1.386012in}%
\pgfsys@useobject{currentmarker}{}%
\end{pgfscope}%
\begin{pgfscope}%
\pgfsys@transformshift{4.780324in}{1.496035in}%
\pgfsys@useobject{currentmarker}{}%
\end{pgfscope}%
\begin{pgfscope}%
\pgfsys@transformshift{4.797918in}{1.374276in}%
\pgfsys@useobject{currentmarker}{}%
\end{pgfscope}%
\begin{pgfscope}%
\pgfsys@transformshift{4.815512in}{1.105208in}%
\pgfsys@useobject{currentmarker}{}%
\end{pgfscope}%
\begin{pgfscope}%
\pgfsys@transformshift{4.833106in}{1.200574in}%
\pgfsys@useobject{currentmarker}{}%
\end{pgfscope}%
\end{pgfscope}%
\begin{pgfscope}%
\pgfsetbuttcap%
\pgfsetroundjoin%
\definecolor{currentfill}{rgb}{0.000000,0.000000,0.000000}%
\pgfsetfillcolor{currentfill}%
\pgfsetlinewidth{0.803000pt}%
\definecolor{currentstroke}{rgb}{0.000000,0.000000,0.000000}%
\pgfsetstrokecolor{currentstroke}%
\pgfsetdash{}{0pt}%
\pgfsys@defobject{currentmarker}{\pgfqpoint{0.000000in}{-0.048611in}}{\pgfqpoint{0.000000in}{0.000000in}}{%
\pgfpathmoveto{\pgfqpoint{0.000000in}{0.000000in}}%
\pgfpathlineto{\pgfqpoint{0.000000in}{-0.048611in}}%
\pgfusepath{stroke,fill}%
}%
\begin{pgfscope}%
\pgfsys@transformshift{0.456635in}{0.589870in}%
\pgfsys@useobject{currentmarker}{}%
\end{pgfscope}%
\end{pgfscope}%
\begin{pgfscope}%
\pgftext[x=0.456635in,y=0.492648in,,top]{\rmfamily\fontsize{10.000000}{12.000000}\selectfont \(\displaystyle -1.5\)}%
\end{pgfscope}%
\begin{pgfscope}%
\pgfsetbuttcap%
\pgfsetroundjoin%
\definecolor{currentfill}{rgb}{0.000000,0.000000,0.000000}%
\pgfsetfillcolor{currentfill}%
\pgfsetlinewidth{0.803000pt}%
\definecolor{currentstroke}{rgb}{0.000000,0.000000,0.000000}%
\pgfsetstrokecolor{currentstroke}%
\pgfsetdash{}{0pt}%
\pgfsys@defobject{currentmarker}{\pgfqpoint{0.000000in}{-0.048611in}}{\pgfqpoint{0.000000in}{0.000000in}}{%
\pgfpathmoveto{\pgfqpoint{0.000000in}{0.000000in}}%
\pgfpathlineto{\pgfqpoint{0.000000in}{-0.048611in}}%
\pgfusepath{stroke,fill}%
}%
\begin{pgfscope}%
\pgfsys@transformshift{1.331929in}{0.589870in}%
\pgfsys@useobject{currentmarker}{}%
\end{pgfscope}%
\end{pgfscope}%
\begin{pgfscope}%
\pgftext[x=1.331929in,y=0.492648in,,top]{\rmfamily\fontsize{10.000000}{12.000000}\selectfont \(\displaystyle -1.0\)}%
\end{pgfscope}%
\begin{pgfscope}%
\pgfsetbuttcap%
\pgfsetroundjoin%
\definecolor{currentfill}{rgb}{0.000000,0.000000,0.000000}%
\pgfsetfillcolor{currentfill}%
\pgfsetlinewidth{0.803000pt}%
\definecolor{currentstroke}{rgb}{0.000000,0.000000,0.000000}%
\pgfsetstrokecolor{currentstroke}%
\pgfsetdash{}{0pt}%
\pgfsys@defobject{currentmarker}{\pgfqpoint{0.000000in}{-0.048611in}}{\pgfqpoint{0.000000in}{0.000000in}}{%
\pgfpathmoveto{\pgfqpoint{0.000000in}{0.000000in}}%
\pgfpathlineto{\pgfqpoint{0.000000in}{-0.048611in}}%
\pgfusepath{stroke,fill}%
}%
\begin{pgfscope}%
\pgfsys@transformshift{2.207224in}{0.589870in}%
\pgfsys@useobject{currentmarker}{}%
\end{pgfscope}%
\end{pgfscope}%
\begin{pgfscope}%
\pgftext[x=2.207224in,y=0.492648in,,top]{\rmfamily\fontsize{10.000000}{12.000000}\selectfont \(\displaystyle -0.5\)}%
\end{pgfscope}%
\begin{pgfscope}%
\pgfsetbuttcap%
\pgfsetroundjoin%
\definecolor{currentfill}{rgb}{0.000000,0.000000,0.000000}%
\pgfsetfillcolor{currentfill}%
\pgfsetlinewidth{0.803000pt}%
\definecolor{currentstroke}{rgb}{0.000000,0.000000,0.000000}%
\pgfsetstrokecolor{currentstroke}%
\pgfsetdash{}{0pt}%
\pgfsys@defobject{currentmarker}{\pgfqpoint{0.000000in}{-0.048611in}}{\pgfqpoint{0.000000in}{0.000000in}}{%
\pgfpathmoveto{\pgfqpoint{0.000000in}{0.000000in}}%
\pgfpathlineto{\pgfqpoint{0.000000in}{-0.048611in}}%
\pgfusepath{stroke,fill}%
}%
\begin{pgfscope}%
\pgfsys@transformshift{3.082518in}{0.589870in}%
\pgfsys@useobject{currentmarker}{}%
\end{pgfscope}%
\end{pgfscope}%
\begin{pgfscope}%
\pgftext[x=3.082518in,y=0.492648in,,top]{\rmfamily\fontsize{10.000000}{12.000000}\selectfont \(\displaystyle 0.0\)}%
\end{pgfscope}%
\begin{pgfscope}%
\pgfsetbuttcap%
\pgfsetroundjoin%
\definecolor{currentfill}{rgb}{0.000000,0.000000,0.000000}%
\pgfsetfillcolor{currentfill}%
\pgfsetlinewidth{0.803000pt}%
\definecolor{currentstroke}{rgb}{0.000000,0.000000,0.000000}%
\pgfsetstrokecolor{currentstroke}%
\pgfsetdash{}{0pt}%
\pgfsys@defobject{currentmarker}{\pgfqpoint{0.000000in}{-0.048611in}}{\pgfqpoint{0.000000in}{0.000000in}}{%
\pgfpathmoveto{\pgfqpoint{0.000000in}{0.000000in}}%
\pgfpathlineto{\pgfqpoint{0.000000in}{-0.048611in}}%
\pgfusepath{stroke,fill}%
}%
\begin{pgfscope}%
\pgfsys@transformshift{3.957812in}{0.589870in}%
\pgfsys@useobject{currentmarker}{}%
\end{pgfscope}%
\end{pgfscope}%
\begin{pgfscope}%
\pgftext[x=3.957812in,y=0.492648in,,top]{\rmfamily\fontsize{10.000000}{12.000000}\selectfont \(\displaystyle 0.5\)}%
\end{pgfscope}%
\begin{pgfscope}%
\pgfsetbuttcap%
\pgfsetroundjoin%
\definecolor{currentfill}{rgb}{0.000000,0.000000,0.000000}%
\pgfsetfillcolor{currentfill}%
\pgfsetlinewidth{0.803000pt}%
\definecolor{currentstroke}{rgb}{0.000000,0.000000,0.000000}%
\pgfsetstrokecolor{currentstroke}%
\pgfsetdash{}{0pt}%
\pgfsys@defobject{currentmarker}{\pgfqpoint{0.000000in}{-0.048611in}}{\pgfqpoint{0.000000in}{0.000000in}}{%
\pgfpathmoveto{\pgfqpoint{0.000000in}{0.000000in}}%
\pgfpathlineto{\pgfqpoint{0.000000in}{-0.048611in}}%
\pgfusepath{stroke,fill}%
}%
\begin{pgfscope}%
\pgfsys@transformshift{4.833106in}{0.589870in}%
\pgfsys@useobject{currentmarker}{}%
\end{pgfscope}%
\end{pgfscope}%
\begin{pgfscope}%
\pgftext[x=4.833106in,y=0.492648in,,top]{\rmfamily\fontsize{10.000000}{12.000000}\selectfont \(\displaystyle 1.0\)}%
\end{pgfscope}%
\begin{pgfscope}%
\pgftext[x=2.644871in,y=0.302680in,,top]{\rmfamily\fontsize{10.000000}{12.000000}\selectfont x}%
\end{pgfscope}%
\begin{pgfscope}%
\pgfsetbuttcap%
\pgfsetroundjoin%
\definecolor{currentfill}{rgb}{0.000000,0.000000,0.000000}%
\pgfsetfillcolor{currentfill}%
\pgfsetlinewidth{0.803000pt}%
\definecolor{currentstroke}{rgb}{0.000000,0.000000,0.000000}%
\pgfsetstrokecolor{currentstroke}%
\pgfsetdash{}{0pt}%
\pgfsys@defobject{currentmarker}{\pgfqpoint{-0.048611in}{0.000000in}}{\pgfqpoint{0.000000in}{0.000000in}}{%
\pgfpathmoveto{\pgfqpoint{0.000000in}{0.000000in}}%
\pgfpathlineto{\pgfqpoint{-0.048611in}{0.000000in}}%
\pgfusepath{stroke,fill}%
}%
\begin{pgfscope}%
\pgfsys@transformshift{0.456635in}{0.954607in}%
\pgfsys@useobject{currentmarker}{}%
\end{pgfscope}%
\end{pgfscope}%
\begin{pgfscope}%
\pgftext[x=0.289968in,y=0.901846in,left,base]{\rmfamily\fontsize{10.000000}{12.000000}\selectfont \(\displaystyle 0\)}%
\end{pgfscope}%
\begin{pgfscope}%
\pgfsetbuttcap%
\pgfsetroundjoin%
\definecolor{currentfill}{rgb}{0.000000,0.000000,0.000000}%
\pgfsetfillcolor{currentfill}%
\pgfsetlinewidth{0.803000pt}%
\definecolor{currentstroke}{rgb}{0.000000,0.000000,0.000000}%
\pgfsetstrokecolor{currentstroke}%
\pgfsetdash{}{0pt}%
\pgfsys@defobject{currentmarker}{\pgfqpoint{-0.048611in}{0.000000in}}{\pgfqpoint{0.000000in}{0.000000in}}{%
\pgfpathmoveto{\pgfqpoint{0.000000in}{0.000000in}}%
\pgfpathlineto{\pgfqpoint{-0.048611in}{0.000000in}}%
\pgfusepath{stroke,fill}%
}%
\begin{pgfscope}%
\pgfsys@transformshift{0.456635in}{1.359870in}%
\pgfsys@useobject{currentmarker}{}%
\end{pgfscope}%
\end{pgfscope}%
\begin{pgfscope}%
\pgftext[x=0.289968in,y=1.307109in,left,base]{\rmfamily\fontsize{10.000000}{12.000000}\selectfont \(\displaystyle 2\)}%
\end{pgfscope}%
\begin{pgfscope}%
\pgftext[x=0.234413in,y=1.076186in,,bottom,rotate=90.000000]{\rmfamily\fontsize{10.000000}{12.000000}\selectfont y}%
\end{pgfscope}%
\begin{pgfscope}%
\pgfpathrectangle{\pgfqpoint{0.456635in}{0.589870in}}{\pgfqpoint{4.376471in}{0.972632in}}%
\pgfusepath{clip}%
\pgfsetrectcap%
\pgfsetroundjoin%
\pgfsetlinewidth{1.505625pt}%
\definecolor{currentstroke}{rgb}{0.121569,0.466667,0.705882}%
\pgfsetstrokecolor{currentstroke}%
\pgfsetdash{}{0pt}%
\pgfpathmoveto{\pgfqpoint{1.401376in}{0.579870in}}%
\pgfpathlineto{\pgfqpoint{1.402305in}{0.959444in}}%
\pgfpathlineto{\pgfqpoint{1.406214in}{1.572502in}}%
\pgfpathmoveto{\pgfqpoint{1.488123in}{1.572502in}}%
\pgfpathlineto{\pgfqpoint{1.490274in}{1.471555in}}%
\pgfpathlineto{\pgfqpoint{1.507868in}{1.029463in}}%
\pgfpathlineto{\pgfqpoint{1.525462in}{1.000762in}}%
\pgfpathlineto{\pgfqpoint{1.543056in}{1.317262in}}%
\pgfpathlineto{\pgfqpoint{1.551273in}{1.572502in}}%
\pgfpathmoveto{\pgfqpoint{1.795977in}{1.572502in}}%
\pgfpathlineto{\pgfqpoint{1.806963in}{1.391135in}}%
\pgfpathlineto{\pgfqpoint{1.824557in}{1.181191in}}%
\pgfpathlineto{\pgfqpoint{1.842151in}{1.049807in}}%
\pgfpathlineto{\pgfqpoint{1.859745in}{0.988048in}}%
\pgfpathlineto{\pgfqpoint{1.877339in}{0.982079in}}%
\pgfpathlineto{\pgfqpoint{1.894933in}{1.015384in}}%
\pgfpathlineto{\pgfqpoint{1.912526in}{1.070868in}}%
\pgfpathlineto{\pgfqpoint{1.930120in}{1.132685in}}%
\pgfpathlineto{\pgfqpoint{1.947714in}{1.187666in}}%
\pgfpathlineto{\pgfqpoint{1.965308in}{1.226255in}}%
\pgfpathlineto{\pgfqpoint{1.982902in}{1.242969in}}%
\pgfpathlineto{\pgfqpoint{2.000496in}{1.236374in}}%
\pgfpathlineto{\pgfqpoint{2.018090in}{1.208660in}}%
\pgfpathlineto{\pgfqpoint{2.035683in}{1.164904in}}%
\pgfpathlineto{\pgfqpoint{2.070871in}{1.058215in}}%
\pgfpathlineto{\pgfqpoint{2.088465in}{1.010967in}}%
\pgfpathlineto{\pgfqpoint{2.106059in}{0.977105in}}%
\pgfpathlineto{\pgfqpoint{2.123653in}{0.961575in}}%
\pgfpathlineto{\pgfqpoint{2.141247in}{0.967056in}}%
\pgfpathlineto{\pgfqpoint{2.158840in}{0.993731in}}%
\pgfpathlineto{\pgfqpoint{2.176434in}{1.039338in}}%
\pgfpathlineto{\pgfqpoint{2.194028in}{1.099449in}}%
\pgfpathlineto{\pgfqpoint{2.246810in}{1.301413in}}%
\pgfpathlineto{\pgfqpoint{2.264404in}{1.351826in}}%
\pgfpathlineto{\pgfqpoint{2.281997in}{1.383022in}}%
\pgfpathlineto{\pgfqpoint{2.299591in}{1.390615in}}%
\pgfpathlineto{\pgfqpoint{2.317185in}{1.372211in}}%
\pgfpathlineto{\pgfqpoint{2.334779in}{1.327609in}}%
\pgfpathlineto{\pgfqpoint{2.352373in}{1.258797in}}%
\pgfpathlineto{\pgfqpoint{2.369967in}{1.169757in}}%
\pgfpathlineto{\pgfqpoint{2.387560in}{1.066100in}}%
\pgfpathlineto{\pgfqpoint{2.440342in}{0.736988in}}%
\pgfpathlineto{\pgfqpoint{2.457936in}{0.644694in}}%
\pgfpathlineto{\pgfqpoint{2.473396in}{0.579870in}}%
\pgfpathmoveto{\pgfqpoint{2.576493in}{0.579870in}}%
\pgfpathlineto{\pgfqpoint{2.581093in}{0.592814in}}%
\pgfpathlineto{\pgfqpoint{2.633874in}{0.759359in}}%
\pgfpathlineto{\pgfqpoint{2.651468in}{0.801518in}}%
\pgfpathlineto{\pgfqpoint{2.669062in}{0.829462in}}%
\pgfpathlineto{\pgfqpoint{2.686656in}{0.840814in}}%
\pgfpathlineto{\pgfqpoint{2.704250in}{0.834969in}}%
\pgfpathlineto{\pgfqpoint{2.721844in}{0.813138in}}%
\pgfpathlineto{\pgfqpoint{2.739438in}{0.778229in}}%
\pgfpathlineto{\pgfqpoint{2.757031in}{0.734581in}}%
\pgfpathlineto{\pgfqpoint{2.792219in}{0.643121in}}%
\pgfpathlineto{\pgfqpoint{2.809813in}{0.607179in}}%
\pgfpathlineto{\pgfqpoint{2.827407in}{0.585149in}}%
\pgfpathlineto{\pgfqpoint{2.845001in}{0.581405in}}%
\pgfpathlineto{\pgfqpoint{2.862594in}{0.598854in}}%
\pgfpathlineto{\pgfqpoint{2.880188in}{0.638643in}}%
\pgfpathlineto{\pgfqpoint{2.897782in}{0.699993in}}%
\pgfpathlineto{\pgfqpoint{2.915376in}{0.780205in}}%
\pgfpathlineto{\pgfqpoint{2.932970in}{0.874829in}}%
\pgfpathlineto{\pgfqpoint{2.985751in}{1.181951in}}%
\pgfpathlineto{\pgfqpoint{3.003345in}{1.268288in}}%
\pgfpathlineto{\pgfqpoint{3.020939in}{1.335384in}}%
\pgfpathlineto{\pgfqpoint{3.038533in}{1.378108in}}%
\pgfpathlineto{\pgfqpoint{3.056127in}{1.393080in}}%
\pgfpathlineto{\pgfqpoint{3.073721in}{1.378994in}}%
\pgfpathlineto{\pgfqpoint{3.091315in}{1.336782in}}%
\pgfpathlineto{\pgfqpoint{3.108908in}{1.269598in}}%
\pgfpathlineto{\pgfqpoint{3.126502in}{1.182619in}}%
\pgfpathlineto{\pgfqpoint{3.161690in}{0.977740in}}%
\pgfpathlineto{\pgfqpoint{3.179284in}{0.876292in}}%
\pgfpathlineto{\pgfqpoint{3.196878in}{0.786639in}}%
\pgfpathlineto{\pgfqpoint{3.214472in}{0.716218in}}%
\pgfpathlineto{\pgfqpoint{3.232065in}{0.670950in}}%
\pgfpathlineto{\pgfqpoint{3.249659in}{0.654689in}}%
\pgfpathlineto{\pgfqpoint{3.267253in}{0.668823in}}%
\pgfpathlineto{\pgfqpoint{3.284847in}{0.712065in}}%
\pgfpathlineto{\pgfqpoint{3.302441in}{0.780442in}}%
\pgfpathlineto{\pgfqpoint{3.320035in}{0.867506in}}%
\pgfpathlineto{\pgfqpoint{3.355222in}{1.062237in}}%
\pgfpathlineto{\pgfqpoint{3.372816in}{1.149282in}}%
\pgfpathlineto{\pgfqpoint{3.390410in}{1.215342in}}%
\pgfpathlineto{\pgfqpoint{3.408004in}{1.250851in}}%
\pgfpathlineto{\pgfqpoint{3.425598in}{1.248037in}}%
\pgfpathlineto{\pgfqpoint{3.443192in}{1.201638in}}%
\pgfpathlineto{\pgfqpoint{3.460785in}{1.109454in}}%
\pgfpathlineto{\pgfqpoint{3.478379in}{0.972676in}}%
\pgfpathlineto{\pgfqpoint{3.495973in}{0.795971in}}%
\pgfpathlineto{\pgfqpoint{3.514136in}{0.579870in}}%
\pgfpathmoveto{\pgfqpoint{3.791628in}{0.579870in}}%
\pgfpathlineto{\pgfqpoint{3.795069in}{0.617106in}}%
\pgfpathlineto{\pgfqpoint{3.812662in}{0.767644in}}%
\pgfpathlineto{\pgfqpoint{3.830256in}{0.867377in}}%
\pgfpathlineto{\pgfqpoint{3.847850in}{0.909829in}}%
\pgfpathlineto{\pgfqpoint{3.865444in}{0.893765in}}%
\pgfpathlineto{\pgfqpoint{3.883038in}{0.823554in}}%
\pgfpathlineto{\pgfqpoint{3.900632in}{0.709056in}}%
\pgfpathlineto{\pgfqpoint{3.916412in}{0.579870in}}%
\pgfpathmoveto{\pgfqpoint{4.059577in}{0.579870in}}%
\pgfpathlineto{\pgfqpoint{4.076570in}{0.865753in}}%
\pgfpathlineto{\pgfqpoint{4.113647in}{1.572502in}}%
\pgfpathmoveto{\pgfqpoint{4.227871in}{1.572502in}}%
\pgfpathlineto{\pgfqpoint{4.234915in}{1.402084in}}%
\pgfpathlineto{\pgfqpoint{4.261949in}{0.579870in}}%
\pgfpathmoveto{\pgfqpoint{4.355230in}{0.579870in}}%
\pgfpathlineto{\pgfqpoint{4.358072in}{0.730595in}}%
\pgfpathlineto{\pgfqpoint{4.367245in}{1.572502in}}%
\pgfpathmoveto{\pgfqpoint{4.660499in}{1.572502in}}%
\pgfpathlineto{\pgfqpoint{4.662024in}{0.579870in}}%
\pgfpathmoveto{\pgfqpoint{4.727273in}{0.579870in}}%
\pgfpathlineto{\pgfqpoint{4.727893in}{1.572502in}}%
\pgfpathlineto{\pgfqpoint{4.727893in}{1.572502in}}%
\pgfusepath{stroke}%
\end{pgfscope}%
\begin{pgfscope}%
\pgfpathrectangle{\pgfqpoint{0.456635in}{0.589870in}}{\pgfqpoint{4.376471in}{0.972632in}}%
\pgfusepath{clip}%
\pgfsetbuttcap%
\pgfsetroundjoin%
\pgfsetlinewidth{1.505625pt}%
\definecolor{currentstroke}{rgb}{1.000000,0.498039,0.054902}%
\pgfsetstrokecolor{currentstroke}%
\pgfsetdash{{9.600000pt}{2.400000pt}{1.500000pt}{2.400000pt}}{0.000000pt}%
\pgfpathmoveto{\pgfqpoint{1.386138in}{0.579870in}}%
\pgfpathlineto{\pgfqpoint{1.402305in}{1.351559in}}%
\pgfpathlineto{\pgfqpoint{1.413929in}{1.572502in}}%
\pgfpathmoveto{\pgfqpoint{1.463870in}{1.572502in}}%
\pgfpathlineto{\pgfqpoint{1.490274in}{1.369764in}}%
\pgfpathlineto{\pgfqpoint{1.507868in}{1.268053in}}%
\pgfpathlineto{\pgfqpoint{1.525462in}{1.205522in}}%
\pgfpathlineto{\pgfqpoint{1.543056in}{1.177423in}}%
\pgfpathlineto{\pgfqpoint{1.560649in}{1.180193in}}%
\pgfpathlineto{\pgfqpoint{1.578243in}{1.197211in}}%
\pgfpathlineto{\pgfqpoint{1.595837in}{1.226498in}}%
\pgfpathlineto{\pgfqpoint{1.613431in}{1.251233in}}%
\pgfpathlineto{\pgfqpoint{1.631025in}{1.271813in}}%
\pgfpathlineto{\pgfqpoint{1.648619in}{1.278343in}}%
\pgfpathlineto{\pgfqpoint{1.666213in}{1.277848in}}%
\pgfpathlineto{\pgfqpoint{1.683806in}{1.268844in}}%
\pgfpathlineto{\pgfqpoint{1.701400in}{1.248759in}}%
\pgfpathlineto{\pgfqpoint{1.754182in}{1.175048in}}%
\pgfpathlineto{\pgfqpoint{1.771776in}{1.158822in}}%
\pgfpathlineto{\pgfqpoint{1.789370in}{1.137846in}}%
\pgfpathlineto{\pgfqpoint{1.806963in}{1.127161in}}%
\pgfpathlineto{\pgfqpoint{1.824557in}{1.113705in}}%
\pgfpathlineto{\pgfqpoint{1.842151in}{1.110538in}}%
\pgfpathlineto{\pgfqpoint{1.877339in}{1.112122in}}%
\pgfpathlineto{\pgfqpoint{1.894933in}{1.116079in}}%
\pgfpathlineto{\pgfqpoint{1.912526in}{1.116079in}}%
\pgfpathlineto{\pgfqpoint{1.930120in}{1.123994in}}%
\pgfpathlineto{\pgfqpoint{1.947714in}{1.125578in}}%
\pgfpathlineto{\pgfqpoint{1.965308in}{1.138242in}}%
\pgfpathlineto{\pgfqpoint{1.982902in}{1.138242in}}%
\pgfpathlineto{\pgfqpoint{2.000496in}{1.146157in}}%
\pgfpathlineto{\pgfqpoint{2.018090in}{1.147740in}}%
\pgfpathlineto{\pgfqpoint{2.035683in}{1.152490in}}%
\pgfpathlineto{\pgfqpoint{2.053277in}{1.169903in}}%
\pgfpathlineto{\pgfqpoint{2.088465in}{1.182568in}}%
\pgfpathlineto{\pgfqpoint{2.106059in}{1.195232in}}%
\pgfpathlineto{\pgfqpoint{2.123653in}{1.211063in}}%
\pgfpathlineto{\pgfqpoint{2.141247in}{1.217395in}}%
\pgfpathlineto{\pgfqpoint{2.158840in}{1.242724in}}%
\pgfpathlineto{\pgfqpoint{2.176434in}{1.249056in}}%
\pgfpathlineto{\pgfqpoint{2.194028in}{1.252222in}}%
\pgfpathlineto{\pgfqpoint{2.211622in}{1.264887in}}%
\pgfpathlineto{\pgfqpoint{2.229216in}{1.268053in}}%
\pgfpathlineto{\pgfqpoint{2.246810in}{1.268053in}}%
\pgfpathlineto{\pgfqpoint{2.264404in}{1.271219in}}%
\pgfpathlineto{\pgfqpoint{2.281997in}{1.264887in}}%
\pgfpathlineto{\pgfqpoint{2.317185in}{1.226893in}}%
\pgfpathlineto{\pgfqpoint{2.334779in}{1.195232in}}%
\pgfpathlineto{\pgfqpoint{2.352373in}{1.182568in}}%
\pgfpathlineto{\pgfqpoint{2.369967in}{1.138242in}}%
\pgfpathlineto{\pgfqpoint{2.387560in}{1.100249in}}%
\pgfpathlineto{\pgfqpoint{2.405154in}{1.093916in}}%
\pgfpathlineto{\pgfqpoint{2.422748in}{1.062255in}}%
\pgfpathlineto{\pgfqpoint{2.440342in}{1.011597in}}%
\pgfpathlineto{\pgfqpoint{2.457936in}{0.998933in}}%
\pgfpathlineto{\pgfqpoint{2.475530in}{0.967272in}}%
\pgfpathlineto{\pgfqpoint{2.493124in}{0.922946in}}%
\pgfpathlineto{\pgfqpoint{2.510717in}{0.929278in}}%
\pgfpathlineto{\pgfqpoint{2.528311in}{0.884953in}}%
\pgfpathlineto{\pgfqpoint{2.545905in}{0.859624in}}%
\pgfpathlineto{\pgfqpoint{2.563499in}{0.821630in}}%
\pgfpathlineto{\pgfqpoint{2.581093in}{0.770972in}}%
\pgfpathlineto{\pgfqpoint{2.598687in}{0.758308in}}%
\pgfpathlineto{\pgfqpoint{2.616281in}{0.739311in}}%
\pgfpathlineto{\pgfqpoint{2.633874in}{0.675989in}}%
\pgfpathlineto{\pgfqpoint{2.651468in}{0.669656in}}%
\pgfpathlineto{\pgfqpoint{2.669062in}{0.612666in}}%
\pgfpathlineto{\pgfqpoint{2.686656in}{0.612666in}}%
\pgfpathlineto{\pgfqpoint{2.701843in}{0.579870in}}%
\pgfpathmoveto{\pgfqpoint{2.837332in}{0.579870in}}%
\pgfpathlineto{\pgfqpoint{2.862594in}{0.625331in}}%
\pgfpathlineto{\pgfqpoint{2.880188in}{0.688653in}}%
\pgfpathlineto{\pgfqpoint{2.915376in}{0.777304in}}%
\pgfpathlineto{\pgfqpoint{2.932970in}{0.840627in}}%
\pgfpathlineto{\pgfqpoint{2.950564in}{0.884953in}}%
\pgfpathlineto{\pgfqpoint{2.985751in}{0.976770in}}%
\pgfpathlineto{\pgfqpoint{3.020939in}{1.027428in}}%
\pgfpathlineto{\pgfqpoint{3.038533in}{1.059089in}}%
\pgfpathlineto{\pgfqpoint{3.056127in}{1.049591in}}%
\pgfpathlineto{\pgfqpoint{3.073721in}{1.071754in}}%
\pgfpathlineto{\pgfqpoint{3.091315in}{1.062255in}}%
\pgfpathlineto{\pgfqpoint{3.108908in}{1.068587in}}%
\pgfpathlineto{\pgfqpoint{3.126502in}{1.040092in}}%
\pgfpathlineto{\pgfqpoint{3.144096in}{1.043258in}}%
\pgfpathlineto{\pgfqpoint{3.161690in}{1.033760in}}%
\pgfpathlineto{\pgfqpoint{3.196878in}{1.033760in}}%
\pgfpathlineto{\pgfqpoint{3.214472in}{1.040092in}}%
\pgfpathlineto{\pgfqpoint{3.232065in}{1.057506in}}%
\pgfpathlineto{\pgfqpoint{3.249659in}{1.081252in}}%
\pgfpathlineto{\pgfqpoint{3.267253in}{1.097082in}}%
\pgfpathlineto{\pgfqpoint{3.284847in}{1.122411in}}%
\pgfpathlineto{\pgfqpoint{3.302441in}{1.157239in}}%
\pgfpathlineto{\pgfqpoint{3.320035in}{1.180193in}}%
\pgfpathlineto{\pgfqpoint{3.337628in}{1.215812in}}%
\pgfpathlineto{\pgfqpoint{3.355222in}{1.243515in}}%
\pgfpathlineto{\pgfqpoint{3.372816in}{1.241932in}}%
\pgfpathlineto{\pgfqpoint{3.390410in}{1.251431in}}%
\pgfpathlineto{\pgfqpoint{3.408004in}{1.226102in}}%
\pgfpathlineto{\pgfqpoint{3.425598in}{1.204335in}}%
\pgfpathlineto{\pgfqpoint{3.443192in}{1.180589in}}%
\pgfpathlineto{\pgfqpoint{3.460785in}{1.136263in}}%
\pgfpathlineto{\pgfqpoint{3.478379in}{1.074524in}}%
\pgfpathlineto{\pgfqpoint{3.495973in}{0.998735in}}%
\pgfpathlineto{\pgfqpoint{3.513567in}{0.939172in}}%
\pgfpathlineto{\pgfqpoint{3.548755in}{0.777206in}}%
\pgfpathlineto{\pgfqpoint{3.566349in}{0.700625in}}%
\pgfpathlineto{\pgfqpoint{3.583942in}{0.636857in}}%
\pgfpathlineto{\pgfqpoint{3.601429in}{0.579870in}}%
\pgfpathmoveto{\pgfqpoint{3.739130in}{0.579870in}}%
\pgfpathlineto{\pgfqpoint{3.742287in}{0.589316in}}%
\pgfpathlineto{\pgfqpoint{3.759881in}{0.637600in}}%
\pgfpathlineto{\pgfqpoint{3.777475in}{0.666886in}}%
\pgfpathlineto{\pgfqpoint{3.812662in}{0.736936in}}%
\pgfpathlineto{\pgfqpoint{3.830256in}{0.761078in}}%
\pgfpathlineto{\pgfqpoint{3.847850in}{0.776513in}}%
\pgfpathlineto{\pgfqpoint{3.865444in}{0.788782in}}%
\pgfpathlineto{\pgfqpoint{3.883038in}{0.780273in}}%
\pgfpathlineto{\pgfqpoint{3.900632in}{0.783835in}}%
\pgfpathlineto{\pgfqpoint{3.918226in}{0.764046in}}%
\pgfpathlineto{\pgfqpoint{3.935819in}{0.755142in}}%
\pgfpathlineto{\pgfqpoint{3.971007in}{0.742675in}}%
\pgfpathlineto{\pgfqpoint{3.988601in}{0.742378in}}%
\pgfpathlineto{\pgfqpoint{4.006195in}{0.754251in}}%
\pgfpathlineto{\pgfqpoint{4.041383in}{0.815298in}}%
\pgfpathlineto{\pgfqpoint{4.058976in}{0.866648in}}%
\pgfpathlineto{\pgfqpoint{4.094164in}{0.983597in}}%
\pgfpathlineto{\pgfqpoint{4.111758in}{1.044347in}}%
\pgfpathlineto{\pgfqpoint{4.129352in}{1.098814in}}%
\pgfpathlineto{\pgfqpoint{4.146946in}{1.142991in}}%
\pgfpathlineto{\pgfqpoint{4.164540in}{1.179402in}}%
\pgfpathlineto{\pgfqpoint{4.182133in}{1.193253in}}%
\pgfpathlineto{\pgfqpoint{4.199727in}{1.189494in}}%
\pgfpathlineto{\pgfqpoint{4.217321in}{1.161592in}}%
\pgfpathlineto{\pgfqpoint{4.234915in}{1.113309in}}%
\pgfpathlineto{\pgfqpoint{4.252509in}{1.058297in}}%
\pgfpathlineto{\pgfqpoint{4.270103in}{0.996954in}}%
\pgfpathlineto{\pgfqpoint{4.287696in}{0.952233in}}%
\pgfpathlineto{\pgfqpoint{4.305290in}{0.922946in}}%
\pgfpathlineto{\pgfqpoint{4.322884in}{0.937985in}}%
\pgfpathlineto{\pgfqpoint{4.340478in}{1.010806in}}%
\pgfpathlineto{\pgfqpoint{4.358072in}{1.156447in}}%
\pgfpathlineto{\pgfqpoint{4.375666in}{1.393906in}}%
\pgfpathlineto{\pgfqpoint{4.384963in}{1.572502in}}%
\pgfpathmoveto{\pgfqpoint{4.658503in}{1.572502in}}%
\pgfpathlineto{\pgfqpoint{4.670520in}{0.579870in}}%
\pgfpathmoveto{\pgfqpoint{4.722607in}{0.579870in}}%
\pgfpathlineto{\pgfqpoint{4.727543in}{1.084418in}}%
\pgfpathlineto{\pgfqpoint{4.729437in}{1.572502in}}%
\pgfpathlineto{\pgfqpoint{4.729437in}{1.572502in}}%
\pgfusepath{stroke}%
\end{pgfscope}%
\begin{pgfscope}%
\pgfsetrectcap%
\pgfsetmiterjoin%
\pgfsetlinewidth{0.803000pt}%
\definecolor{currentstroke}{rgb}{0.000000,0.000000,0.000000}%
\pgfsetstrokecolor{currentstroke}%
\pgfsetdash{}{0pt}%
\pgfpathmoveto{\pgfqpoint{0.456635in}{0.589870in}}%
\pgfpathlineto{\pgfqpoint{0.456635in}{1.562502in}}%
\pgfusepath{stroke}%
\end{pgfscope}%
\begin{pgfscope}%
\pgfsetrectcap%
\pgfsetmiterjoin%
\pgfsetlinewidth{0.803000pt}%
\definecolor{currentstroke}{rgb}{0.000000,0.000000,0.000000}%
\pgfsetstrokecolor{currentstroke}%
\pgfsetdash{}{0pt}%
\pgfpathmoveto{\pgfqpoint{4.833106in}{0.589870in}}%
\pgfpathlineto{\pgfqpoint{4.833106in}{1.562502in}}%
\pgfusepath{stroke}%
\end{pgfscope}%
\begin{pgfscope}%
\pgfsetrectcap%
\pgfsetmiterjoin%
\pgfsetlinewidth{0.803000pt}%
\definecolor{currentstroke}{rgb}{0.000000,0.000000,0.000000}%
\pgfsetstrokecolor{currentstroke}%
\pgfsetdash{}{0pt}%
\pgfpathmoveto{\pgfqpoint{0.456635in}{0.589870in}}%
\pgfpathlineto{\pgfqpoint{4.833106in}{0.589870in}}%
\pgfusepath{stroke}%
\end{pgfscope}%
\begin{pgfscope}%
\pgfsetrectcap%
\pgfsetmiterjoin%
\pgfsetlinewidth{0.803000pt}%
\definecolor{currentstroke}{rgb}{0.000000,0.000000,0.000000}%
\pgfsetstrokecolor{currentstroke}%
\pgfsetdash{}{0pt}%
\pgfpathmoveto{\pgfqpoint{0.456635in}{1.562502in}}%
\pgfpathlineto{\pgfqpoint{4.833106in}{1.562502in}}%
\pgfusepath{stroke}%
\end{pgfscope}%
\begin{pgfscope}%
\pgfsetbuttcap%
\pgfsetmiterjoin%
\definecolor{currentfill}{rgb}{1.000000,1.000000,1.000000}%
\pgfsetfillcolor{currentfill}%
\pgfsetfillopacity{0.800000}%
\pgfsetlinewidth{1.003750pt}%
\definecolor{currentstroke}{rgb}{0.800000,0.800000,0.800000}%
\pgfsetstrokecolor{currentstroke}%
\pgfsetstrokeopacity{0.800000}%
\pgfsetdash{}{0pt}%
\pgfpathmoveto{\pgfqpoint{0.553858in}{0.659315in}}%
\pgfpathlineto{\pgfqpoint{1.337183in}{0.659315in}}%
\pgfpathquadraticcurveto{\pgfqpoint{1.364960in}{0.659315in}}{\pgfqpoint{1.364960in}{0.687092in}}%
\pgfpathlineto{\pgfqpoint{1.364960in}{1.490640in}}%
\pgfpathquadraticcurveto{\pgfqpoint{1.364960in}{1.518418in}}{\pgfqpoint{1.337183in}{1.518418in}}%
\pgfpathlineto{\pgfqpoint{0.553858in}{1.518418in}}%
\pgfpathquadraticcurveto{\pgfqpoint{0.526080in}{1.518418in}}{\pgfqpoint{0.526080in}{1.490640in}}%
\pgfpathlineto{\pgfqpoint{0.526080in}{0.687092in}}%
\pgfpathquadraticcurveto{\pgfqpoint{0.526080in}{0.659315in}}{\pgfqpoint{0.553858in}{0.659315in}}%
\pgfpathclose%
\pgfusepath{stroke,fill}%
\end{pgfscope}%
\begin{pgfscope}%
\pgfsetrectcap%
\pgfsetroundjoin%
\pgfsetlinewidth{1.505625pt}%
\definecolor{currentstroke}{rgb}{0.121569,0.466667,0.705882}%
\pgfsetstrokecolor{currentstroke}%
\pgfsetdash{}{0pt}%
\pgfpathmoveto{\pgfqpoint{0.581635in}{1.404947in}}%
\pgfpathlineto{\pgfqpoint{0.859413in}{1.404947in}}%
\pgfusepath{stroke}%
\end{pgfscope}%
\begin{pgfscope}%
\pgftext[x=0.970524in,y=1.356336in,left,base]{\rmfamily\fontsize{10.000000}{12.000000}\selectfont \(\displaystyle \widetilde{\Phi}^* \theta\)}%
\end{pgfscope}%
\begin{pgfscope}%
\pgfsetbuttcap%
\pgfsetroundjoin%
\pgfsetlinewidth{1.505625pt}%
\definecolor{currentstroke}{rgb}{1.000000,0.498039,0.054902}%
\pgfsetstrokecolor{currentstroke}%
\pgfsetdash{{9.600000pt}{2.400000pt}{1.500000pt}{2.400000pt}}{0.000000pt}%
\pgfpathmoveto{\pgfqpoint{0.581635in}{1.200086in}}%
\pgfpathlineto{\pgfqpoint{0.859413in}{1.200086in}}%
\pgfusepath{stroke}%
\end{pgfscope}%
\begin{pgfscope}%
\pgftext[x=0.970524in,y=1.151474in,left,base]{\rmfamily\fontsize{10.000000}{12.000000}\selectfont \(\displaystyle \widetilde{K}u\)}%
\end{pgfscope}%
\begin{pgfscope}%
\pgfsetbuttcap%
\pgfsetroundjoin%
\definecolor{currentfill}{rgb}{1.000000,0.000000,0.000000}%
\pgfsetfillcolor{currentfill}%
\pgfsetlinewidth{2.007500pt}%
\definecolor{currentstroke}{rgb}{1.000000,0.000000,0.000000}%
\pgfsetstrokecolor{currentstroke}%
\pgfsetdash{}{0pt}%
\pgfpathmoveto{\pgfqpoint{0.678857in}{0.984076in}}%
\pgfpathlineto{\pgfqpoint{0.762191in}{0.984076in}}%
\pgfpathmoveto{\pgfqpoint{0.720524in}{0.942409in}}%
\pgfpathlineto{\pgfqpoint{0.720524in}{1.025742in}}%
\pgfusepath{stroke,fill}%
\end{pgfscope}%
\begin{pgfscope}%
\pgftext[x=0.970524in,y=0.947617in,left,base]{\rmfamily\fontsize{10.000000}{12.000000}\selectfont train}%
\end{pgfscope}%
\begin{pgfscope}%
\pgfsetbuttcap%
\pgfsetroundjoin%
\definecolor{currentfill}{rgb}{0.000000,0.000000,0.000000}%
\pgfsetfillcolor{currentfill}%
\pgfsetlinewidth{1.003750pt}%
\definecolor{currentstroke}{rgb}{0.000000,0.000000,0.000000}%
\pgfsetstrokecolor{currentstroke}%
\pgfsetdash{}{0pt}%
\pgfsys@defobject{currentmarker}{\pgfqpoint{-0.020833in}{-0.020833in}}{\pgfqpoint{0.020833in}{0.020833in}}{%
\pgfpathmoveto{\pgfqpoint{0.000000in}{-0.020833in}}%
\pgfpathcurveto{\pgfqpoint{0.005525in}{-0.020833in}}{\pgfqpoint{0.010825in}{-0.018638in}}{\pgfqpoint{0.014731in}{-0.014731in}}%
\pgfpathcurveto{\pgfqpoint{0.018638in}{-0.010825in}}{\pgfqpoint{0.020833in}{-0.005525in}}{\pgfqpoint{0.020833in}{0.000000in}}%
\pgfpathcurveto{\pgfqpoint{0.020833in}{0.005525in}}{\pgfqpoint{0.018638in}{0.010825in}}{\pgfqpoint{0.014731in}{0.014731in}}%
\pgfpathcurveto{\pgfqpoint{0.010825in}{0.018638in}}{\pgfqpoint{0.005525in}{0.020833in}}{\pgfqpoint{0.000000in}{0.020833in}}%
\pgfpathcurveto{\pgfqpoint{-0.005525in}{0.020833in}}{\pgfqpoint{-0.010825in}{0.018638in}}{\pgfqpoint{-0.014731in}{0.014731in}}%
\pgfpathcurveto{\pgfqpoint{-0.018638in}{0.010825in}}{\pgfqpoint{-0.020833in}{0.005525in}}{\pgfqpoint{-0.020833in}{0.000000in}}%
\pgfpathcurveto{\pgfqpoint{-0.020833in}{-0.005525in}}{\pgfqpoint{-0.018638in}{-0.010825in}}{\pgfqpoint{-0.014731in}{-0.014731in}}%
\pgfpathcurveto{\pgfqpoint{-0.010825in}{-0.018638in}}{\pgfqpoint{-0.005525in}{-0.020833in}}{\pgfqpoint{0.000000in}{-0.020833in}}%
\pgfpathclose%
\pgfusepath{stroke,fill}%
}%
\begin{pgfscope}%
\pgfsys@transformshift{0.720524in}{0.780218in}%
\pgfsys@useobject{currentmarker}{}%
\end{pgfscope}%
\end{pgfscope}%
\begin{pgfscope}%
\pgftext[x=0.970524in,y=0.743760in,left,base]{\rmfamily\fontsize{10.000000}{12.000000}\selectfont test}%
\end{pgfscope}%
\begin{pgfscope}%
\pgfsetbuttcap%
\pgfsetmiterjoin%
\definecolor{currentfill}{rgb}{1.000000,1.000000,1.000000}%
\pgfsetfillcolor{currentfill}%
\pgfsetlinewidth{0.000000pt}%
\definecolor{currentstroke}{rgb}{0.000000,0.000000,0.000000}%
\pgfsetstrokecolor{currentstroke}%
\pgfsetstrokeopacity{0.000000}%
\pgfsetdash{}{0pt}%
\pgfpathmoveto{\pgfqpoint{5.562518in}{0.589870in}}%
\pgfpathlineto{\pgfqpoint{9.938988in}{0.589870in}}%
\pgfpathlineto{\pgfqpoint{9.938988in}{1.562502in}}%
\pgfpathlineto{\pgfqpoint{5.562518in}{1.562502in}}%
\pgfpathclose%
\pgfusepath{fill}%
\end{pgfscope}%
\begin{pgfscope}%
\pgfpathrectangle{\pgfqpoint{5.562518in}{0.589870in}}{\pgfqpoint{4.376471in}{0.972632in}}%
\pgfusepath{clip}%
\pgfsetbuttcap%
\pgfsetroundjoin%
\definecolor{currentfill}{rgb}{1.000000,0.000000,0.000000}%
\pgfsetfillcolor{currentfill}%
\pgfsetlinewidth{2.007500pt}%
\definecolor{currentstroke}{rgb}{1.000000,0.000000,0.000000}%
\pgfsetstrokecolor{currentstroke}%
\pgfsetdash{}{0pt}%
\pgfpathmoveto{\pgfqpoint{7.707476in}{0.743738in}}%
\pgfpathlineto{\pgfqpoint{7.790809in}{0.743738in}}%
\pgfpathmoveto{\pgfqpoint{7.749143in}{0.702071in}}%
\pgfpathlineto{\pgfqpoint{7.749143in}{0.785404in}}%
\pgfusepath{stroke,fill}%
\end{pgfscope}%
\begin{pgfscope}%
\pgfpathrectangle{\pgfqpoint{5.562518in}{0.589870in}}{\pgfqpoint{4.376471in}{0.972632in}}%
\pgfusepath{clip}%
\pgfsetbuttcap%
\pgfsetroundjoin%
\definecolor{currentfill}{rgb}{1.000000,0.000000,0.000000}%
\pgfsetfillcolor{currentfill}%
\pgfsetlinewidth{2.007500pt}%
\definecolor{currentstroke}{rgb}{1.000000,0.000000,0.000000}%
\pgfsetstrokecolor{currentstroke}%
\pgfsetdash{}{0pt}%
\pgfpathmoveto{\pgfqpoint{9.724764in}{1.402840in}}%
\pgfpathlineto{\pgfqpoint{9.808097in}{1.402840in}}%
\pgfpathmoveto{\pgfqpoint{9.766430in}{1.361173in}}%
\pgfpathlineto{\pgfqpoint{9.766430in}{1.444506in}}%
\pgfusepath{stroke,fill}%
\end{pgfscope}%
\begin{pgfscope}%
\pgfpathrectangle{\pgfqpoint{5.562518in}{0.589870in}}{\pgfqpoint{4.376471in}{0.972632in}}%
\pgfusepath{clip}%
\pgfsetbuttcap%
\pgfsetroundjoin%
\definecolor{currentfill}{rgb}{1.000000,0.000000,0.000000}%
\pgfsetfillcolor{currentfill}%
\pgfsetlinewidth{2.007500pt}%
\definecolor{currentstroke}{rgb}{1.000000,0.000000,0.000000}%
\pgfsetstrokecolor{currentstroke}%
\pgfsetdash{}{0pt}%
\pgfpathmoveto{\pgfqpoint{8.958985in}{0.778901in}}%
\pgfpathlineto{\pgfqpoint{9.042318in}{0.778901in}}%
\pgfpathmoveto{\pgfqpoint{9.000652in}{0.737234in}}%
\pgfpathlineto{\pgfqpoint{9.000652in}{0.820567in}}%
\pgfusepath{stroke,fill}%
\end{pgfscope}%
\begin{pgfscope}%
\pgfpathrectangle{\pgfqpoint{5.562518in}{0.589870in}}{\pgfqpoint{4.376471in}{0.972632in}}%
\pgfusepath{clip}%
\pgfsetbuttcap%
\pgfsetroundjoin%
\definecolor{currentfill}{rgb}{1.000000,0.000000,0.000000}%
\pgfsetfillcolor{currentfill}%
\pgfsetlinewidth{2.007500pt}%
\definecolor{currentstroke}{rgb}{1.000000,0.000000,0.000000}%
\pgfsetstrokecolor{currentstroke}%
\pgfsetdash{}{0pt}%
\pgfpathmoveto{\pgfqpoint{8.492154in}{1.127281in}}%
\pgfpathlineto{\pgfqpoint{8.575487in}{1.127281in}}%
\pgfpathmoveto{\pgfqpoint{8.533821in}{1.085614in}}%
\pgfpathlineto{\pgfqpoint{8.533821in}{1.168947in}}%
\pgfusepath{stroke,fill}%
\end{pgfscope}%
\begin{pgfscope}%
\pgfpathrectangle{\pgfqpoint{5.562518in}{0.589870in}}{\pgfqpoint{4.376471in}{0.972632in}}%
\pgfusepath{clip}%
\pgfsetbuttcap%
\pgfsetroundjoin%
\definecolor{currentfill}{rgb}{1.000000,0.000000,0.000000}%
\pgfsetfillcolor{currentfill}%
\pgfsetlinewidth{2.007500pt}%
\definecolor{currentstroke}{rgb}{1.000000,0.000000,0.000000}%
\pgfsetstrokecolor{currentstroke}%
\pgfsetdash{}{0pt}%
\pgfpathmoveto{\pgfqpoint{6.942394in}{0.971634in}}%
\pgfpathlineto{\pgfqpoint{7.025727in}{0.971634in}}%
\pgfpathmoveto{\pgfqpoint{6.984061in}{0.929968in}}%
\pgfpathlineto{\pgfqpoint{6.984061in}{1.013301in}}%
\pgfusepath{stroke,fill}%
\end{pgfscope}%
\begin{pgfscope}%
\pgfpathrectangle{\pgfqpoint{5.562518in}{0.589870in}}{\pgfqpoint{4.376471in}{0.972632in}}%
\pgfusepath{clip}%
\pgfsetbuttcap%
\pgfsetroundjoin%
\definecolor{currentfill}{rgb}{1.000000,0.000000,0.000000}%
\pgfsetfillcolor{currentfill}%
\pgfsetlinewidth{2.007500pt}%
\definecolor{currentstroke}{rgb}{1.000000,0.000000,0.000000}%
\pgfsetstrokecolor{currentstroke}%
\pgfsetdash{}{0pt}%
\pgfpathmoveto{\pgfqpoint{6.942309in}{1.220208in}}%
\pgfpathlineto{\pgfqpoint{7.025643in}{1.220208in}}%
\pgfpathmoveto{\pgfqpoint{6.983976in}{1.178541in}}%
\pgfpathlineto{\pgfqpoint{6.983976in}{1.261874in}}%
\pgfusepath{stroke,fill}%
\end{pgfscope}%
\begin{pgfscope}%
\pgfpathrectangle{\pgfqpoint{5.562518in}{0.589870in}}{\pgfqpoint{4.376471in}{0.972632in}}%
\pgfusepath{clip}%
\pgfsetbuttcap%
\pgfsetroundjoin%
\definecolor{currentfill}{rgb}{1.000000,0.000000,0.000000}%
\pgfsetfillcolor{currentfill}%
\pgfsetlinewidth{2.007500pt}%
\definecolor{currentstroke}{rgb}{1.000000,0.000000,0.000000}%
\pgfsetstrokecolor{currentstroke}%
\pgfsetdash{}{0pt}%
\pgfpathmoveto{\pgfqpoint{6.599506in}{1.157677in}}%
\pgfpathlineto{\pgfqpoint{6.682839in}{1.157677in}}%
\pgfpathmoveto{\pgfqpoint{6.641173in}{1.116011in}}%
\pgfpathlineto{\pgfqpoint{6.641173in}{1.199344in}}%
\pgfusepath{stroke,fill}%
\end{pgfscope}%
\begin{pgfscope}%
\pgfpathrectangle{\pgfqpoint{5.562518in}{0.589870in}}{\pgfqpoint{4.376471in}{0.972632in}}%
\pgfusepath{clip}%
\pgfsetbuttcap%
\pgfsetroundjoin%
\definecolor{currentfill}{rgb}{1.000000,0.000000,0.000000}%
\pgfsetfillcolor{currentfill}%
\pgfsetlinewidth{2.007500pt}%
\definecolor{currentstroke}{rgb}{1.000000,0.000000,0.000000}%
\pgfsetstrokecolor{currentstroke}%
\pgfsetdash{}{0pt}%
\pgfpathmoveto{\pgfqpoint{9.428781in}{1.239588in}}%
\pgfpathlineto{\pgfqpoint{9.512114in}{1.239588in}}%
\pgfpathmoveto{\pgfqpoint{9.470447in}{1.197921in}}%
\pgfpathlineto{\pgfqpoint{9.470447in}{1.281255in}}%
\pgfusepath{stroke,fill}%
\end{pgfscope}%
\begin{pgfscope}%
\pgfpathrectangle{\pgfqpoint{5.562518in}{0.589870in}}{\pgfqpoint{4.376471in}{0.972632in}}%
\pgfusepath{clip}%
\pgfsetbuttcap%
\pgfsetroundjoin%
\definecolor{currentfill}{rgb}{1.000000,0.000000,0.000000}%
\pgfsetfillcolor{currentfill}%
\pgfsetlinewidth{2.007500pt}%
\definecolor{currentstroke}{rgb}{1.000000,0.000000,0.000000}%
\pgfsetstrokecolor{currentstroke}%
\pgfsetdash{}{0pt}%
\pgfpathmoveto{\pgfqpoint{8.500755in}{1.236844in}}%
\pgfpathlineto{\pgfqpoint{8.584088in}{1.236844in}}%
\pgfpathmoveto{\pgfqpoint{8.542421in}{1.195178in}}%
\pgfpathlineto{\pgfqpoint{8.542421in}{1.278511in}}%
\pgfusepath{stroke,fill}%
\end{pgfscope}%
\begin{pgfscope}%
\pgfpathrectangle{\pgfqpoint{5.562518in}{0.589870in}}{\pgfqpoint{4.376471in}{0.972632in}}%
\pgfusepath{clip}%
\pgfsetbuttcap%
\pgfsetroundjoin%
\definecolor{currentfill}{rgb}{1.000000,0.000000,0.000000}%
\pgfsetfillcolor{currentfill}%
\pgfsetlinewidth{2.007500pt}%
\definecolor{currentstroke}{rgb}{1.000000,0.000000,0.000000}%
\pgfsetstrokecolor{currentstroke}%
\pgfsetdash{}{0pt}%
\pgfpathmoveto{\pgfqpoint{8.875232in}{0.728450in}}%
\pgfpathlineto{\pgfqpoint{8.958565in}{0.728450in}}%
\pgfpathmoveto{\pgfqpoint{8.916899in}{0.686783in}}%
\pgfpathlineto{\pgfqpoint{8.916899in}{0.770116in}}%
\pgfusepath{stroke,fill}%
\end{pgfscope}%
\begin{pgfscope}%
\pgfpathrectangle{\pgfqpoint{5.562518in}{0.589870in}}{\pgfqpoint{4.376471in}{0.972632in}}%
\pgfusepath{clip}%
\pgfsetbuttcap%
\pgfsetroundjoin%
\definecolor{currentfill}{rgb}{1.000000,0.000000,0.000000}%
\pgfsetfillcolor{currentfill}%
\pgfsetlinewidth{2.007500pt}%
\definecolor{currentstroke}{rgb}{1.000000,0.000000,0.000000}%
\pgfsetstrokecolor{currentstroke}%
\pgfsetdash{}{0pt}%
\pgfpathmoveto{\pgfqpoint{6.468215in}{1.381339in}}%
\pgfpathlineto{\pgfqpoint{6.551548in}{1.381339in}}%
\pgfpathmoveto{\pgfqpoint{6.509882in}{1.339672in}}%
\pgfpathlineto{\pgfqpoint{6.509882in}{1.423006in}}%
\pgfusepath{stroke,fill}%
\end{pgfscope}%
\begin{pgfscope}%
\pgfpathrectangle{\pgfqpoint{5.562518in}{0.589870in}}{\pgfqpoint{4.376471in}{0.972632in}}%
\pgfusepath{clip}%
\pgfsetbuttcap%
\pgfsetroundjoin%
\definecolor{currentfill}{rgb}{1.000000,0.000000,0.000000}%
\pgfsetfillcolor{currentfill}%
\pgfsetlinewidth{2.007500pt}%
\definecolor{currentstroke}{rgb}{1.000000,0.000000,0.000000}%
\pgfsetstrokecolor{currentstroke}%
\pgfsetdash{}{0pt}%
\pgfpathmoveto{\pgfqpoint{9.791971in}{1.119400in}}%
\pgfpathlineto{\pgfqpoint{9.875304in}{1.119400in}}%
\pgfpathmoveto{\pgfqpoint{9.833637in}{1.077734in}}%
\pgfpathlineto{\pgfqpoint{9.833637in}{1.161067in}}%
\pgfusepath{stroke,fill}%
\end{pgfscope}%
\begin{pgfscope}%
\pgfpathrectangle{\pgfqpoint{5.562518in}{0.589870in}}{\pgfqpoint{4.376471in}{0.972632in}}%
\pgfusepath{clip}%
\pgfsetbuttcap%
\pgfsetroundjoin%
\definecolor{currentfill}{rgb}{1.000000,0.000000,0.000000}%
\pgfsetfillcolor{currentfill}%
\pgfsetlinewidth{2.007500pt}%
\definecolor{currentstroke}{rgb}{1.000000,0.000000,0.000000}%
\pgfsetstrokecolor{currentstroke}%
\pgfsetdash{}{0pt}%
\pgfpathmoveto{\pgfqpoint{9.310674in}{1.071652in}}%
\pgfpathlineto{\pgfqpoint{9.394007in}{1.071652in}}%
\pgfpathmoveto{\pgfqpoint{9.352340in}{1.029985in}}%
\pgfpathlineto{\pgfqpoint{9.352340in}{1.113319in}}%
\pgfusepath{stroke,fill}%
\end{pgfscope}%
\begin{pgfscope}%
\pgfpathrectangle{\pgfqpoint{5.562518in}{0.589870in}}{\pgfqpoint{4.376471in}{0.972632in}}%
\pgfusepath{clip}%
\pgfsetbuttcap%
\pgfsetroundjoin%
\definecolor{currentfill}{rgb}{1.000000,0.000000,0.000000}%
\pgfsetfillcolor{currentfill}%
\pgfsetlinewidth{2.007500pt}%
\definecolor{currentstroke}{rgb}{1.000000,0.000000,0.000000}%
\pgfsetstrokecolor{currentstroke}%
\pgfsetdash{}{0pt}%
\pgfpathmoveto{\pgfqpoint{7.139582in}{1.211124in}}%
\pgfpathlineto{\pgfqpoint{7.222915in}{1.211124in}}%
\pgfpathmoveto{\pgfqpoint{7.181248in}{1.169457in}}%
\pgfpathlineto{\pgfqpoint{7.181248in}{1.252790in}}%
\pgfusepath{stroke,fill}%
\end{pgfscope}%
\begin{pgfscope}%
\pgfpathrectangle{\pgfqpoint{5.562518in}{0.589870in}}{\pgfqpoint{4.376471in}{0.972632in}}%
\pgfusepath{clip}%
\pgfsetbuttcap%
\pgfsetroundjoin%
\definecolor{currentfill}{rgb}{1.000000,0.000000,0.000000}%
\pgfsetfillcolor{currentfill}%
\pgfsetlinewidth{2.007500pt}%
\definecolor{currentstroke}{rgb}{1.000000,0.000000,0.000000}%
\pgfsetstrokecolor{currentstroke}%
\pgfsetdash{}{0pt}%
\pgfpathmoveto{\pgfqpoint{7.032746in}{1.177276in}}%
\pgfpathlineto{\pgfqpoint{7.116080in}{1.177276in}}%
\pgfpathmoveto{\pgfqpoint{7.074413in}{1.135610in}}%
\pgfpathlineto{\pgfqpoint{7.074413in}{1.218943in}}%
\pgfusepath{stroke,fill}%
\end{pgfscope}%
\begin{pgfscope}%
\pgfpathrectangle{\pgfqpoint{5.562518in}{0.589870in}}{\pgfqpoint{4.376471in}{0.972632in}}%
\pgfusepath{clip}%
\pgfsetbuttcap%
\pgfsetroundjoin%
\definecolor{currentfill}{rgb}{1.000000,0.000000,0.000000}%
\pgfsetfillcolor{currentfill}%
\pgfsetlinewidth{2.007500pt}%
\definecolor{currentstroke}{rgb}{1.000000,0.000000,0.000000}%
\pgfsetstrokecolor{currentstroke}%
\pgfsetdash{}{0pt}%
\pgfpathmoveto{\pgfqpoint{7.038277in}{1.124583in}}%
\pgfpathlineto{\pgfqpoint{7.121610in}{1.124583in}}%
\pgfpathmoveto{\pgfqpoint{7.079943in}{1.082916in}}%
\pgfpathlineto{\pgfqpoint{7.079943in}{1.166249in}}%
\pgfusepath{stroke,fill}%
\end{pgfscope}%
\begin{pgfscope}%
\pgfpathrectangle{\pgfqpoint{5.562518in}{0.589870in}}{\pgfqpoint{4.376471in}{0.972632in}}%
\pgfusepath{clip}%
\pgfsetbuttcap%
\pgfsetroundjoin%
\definecolor{currentfill}{rgb}{1.000000,0.000000,0.000000}%
\pgfsetfillcolor{currentfill}%
\pgfsetlinewidth{2.007500pt}%
\definecolor{currentstroke}{rgb}{1.000000,0.000000,0.000000}%
\pgfsetstrokecolor{currentstroke}%
\pgfsetdash{}{0pt}%
\pgfpathmoveto{\pgfqpoint{7.461351in}{1.146587in}}%
\pgfpathlineto{\pgfqpoint{7.544684in}{1.146587in}}%
\pgfpathmoveto{\pgfqpoint{7.503018in}{1.104920in}}%
\pgfpathlineto{\pgfqpoint{7.503018in}{1.188254in}}%
\pgfusepath{stroke,fill}%
\end{pgfscope}%
\begin{pgfscope}%
\pgfpathrectangle{\pgfqpoint{5.562518in}{0.589870in}}{\pgfqpoint{4.376471in}{0.972632in}}%
\pgfusepath{clip}%
\pgfsetbuttcap%
\pgfsetroundjoin%
\definecolor{currentfill}{rgb}{1.000000,0.000000,0.000000}%
\pgfsetfillcolor{currentfill}%
\pgfsetlinewidth{2.007500pt}%
\definecolor{currentstroke}{rgb}{1.000000,0.000000,0.000000}%
\pgfsetstrokecolor{currentstroke}%
\pgfsetdash{}{0pt}%
\pgfpathmoveto{\pgfqpoint{8.233410in}{1.021373in}}%
\pgfpathlineto{\pgfqpoint{8.316743in}{1.021373in}}%
\pgfpathmoveto{\pgfqpoint{8.275077in}{0.979707in}}%
\pgfpathlineto{\pgfqpoint{8.275077in}{1.063040in}}%
\pgfusepath{stroke,fill}%
\end{pgfscope}%
\begin{pgfscope}%
\pgfpathrectangle{\pgfqpoint{5.562518in}{0.589870in}}{\pgfqpoint{4.376471in}{0.972632in}}%
\pgfusepath{clip}%
\pgfsetbuttcap%
\pgfsetroundjoin%
\definecolor{currentfill}{rgb}{1.000000,0.000000,0.000000}%
\pgfsetfillcolor{currentfill}%
\pgfsetlinewidth{2.007500pt}%
\definecolor{currentstroke}{rgb}{1.000000,0.000000,0.000000}%
\pgfsetstrokecolor{currentstroke}%
\pgfsetdash{}{0pt}%
\pgfpathmoveto{\pgfqpoint{7.908461in}{0.614125in}}%
\pgfpathlineto{\pgfqpoint{7.991794in}{0.614125in}}%
\pgfpathmoveto{\pgfqpoint{7.950127in}{0.572458in}}%
\pgfpathlineto{\pgfqpoint{7.950127in}{0.655791in}}%
\pgfusepath{stroke,fill}%
\end{pgfscope}%
\begin{pgfscope}%
\pgfpathrectangle{\pgfqpoint{5.562518in}{0.589870in}}{\pgfqpoint{4.376471in}{0.972632in}}%
\pgfusepath{clip}%
\pgfsetbuttcap%
\pgfsetroundjoin%
\definecolor{currentfill}{rgb}{1.000000,0.000000,0.000000}%
\pgfsetfillcolor{currentfill}%
\pgfsetlinewidth{2.007500pt}%
\definecolor{currentstroke}{rgb}{1.000000,0.000000,0.000000}%
\pgfsetstrokecolor{currentstroke}%
\pgfsetdash{}{0pt}%
\pgfpathmoveto{\pgfqpoint{7.415790in}{1.126443in}}%
\pgfpathlineto{\pgfqpoint{7.499123in}{1.126443in}}%
\pgfpathmoveto{\pgfqpoint{7.457456in}{1.084776in}}%
\pgfpathlineto{\pgfqpoint{7.457456in}{1.168110in}}%
\pgfusepath{stroke,fill}%
\end{pgfscope}%
\begin{pgfscope}%
\pgfpathrectangle{\pgfqpoint{5.562518in}{0.589870in}}{\pgfqpoint{4.376471in}{0.972632in}}%
\pgfusepath{clip}%
\pgfsetbuttcap%
\pgfsetroundjoin%
\definecolor{currentfill}{rgb}{1.000000,0.000000,0.000000}%
\pgfsetfillcolor{currentfill}%
\pgfsetlinewidth{2.007500pt}%
\definecolor{currentstroke}{rgb}{1.000000,0.000000,0.000000}%
\pgfsetstrokecolor{currentstroke}%
\pgfsetdash{}{0pt}%
\pgfpathmoveto{\pgfqpoint{8.538350in}{1.087515in}}%
\pgfpathlineto{\pgfqpoint{8.621683in}{1.087515in}}%
\pgfpathmoveto{\pgfqpoint{8.580017in}{1.045848in}}%
\pgfpathlineto{\pgfqpoint{8.580017in}{1.129181in}}%
\pgfusepath{stroke,fill}%
\end{pgfscope}%
\begin{pgfscope}%
\pgfpathrectangle{\pgfqpoint{5.562518in}{0.589870in}}{\pgfqpoint{4.376471in}{0.972632in}}%
\pgfusepath{clip}%
\pgfsetbuttcap%
\pgfsetroundjoin%
\definecolor{currentfill}{rgb}{1.000000,0.000000,0.000000}%
\pgfsetfillcolor{currentfill}%
\pgfsetlinewidth{2.007500pt}%
\definecolor{currentstroke}{rgb}{1.000000,0.000000,0.000000}%
\pgfsetstrokecolor{currentstroke}%
\pgfsetdash{}{0pt}%
\pgfpathmoveto{\pgfqpoint{6.884538in}{1.110328in}}%
\pgfpathlineto{\pgfqpoint{6.967871in}{1.110328in}}%
\pgfpathmoveto{\pgfqpoint{6.926204in}{1.068661in}}%
\pgfpathlineto{\pgfqpoint{6.926204in}{1.151995in}}%
\pgfusepath{stroke,fill}%
\end{pgfscope}%
\begin{pgfscope}%
\pgfpathrectangle{\pgfqpoint{5.562518in}{0.589870in}}{\pgfqpoint{4.376471in}{0.972632in}}%
\pgfusepath{clip}%
\pgfsetbuttcap%
\pgfsetroundjoin%
\definecolor{currentfill}{rgb}{1.000000,0.000000,0.000000}%
\pgfsetfillcolor{currentfill}%
\pgfsetlinewidth{2.007500pt}%
\definecolor{currentstroke}{rgb}{1.000000,0.000000,0.000000}%
\pgfsetstrokecolor{currentstroke}%
\pgfsetdash{}{0pt}%
\pgfpathmoveto{\pgfqpoint{7.418995in}{1.231643in}}%
\pgfpathlineto{\pgfqpoint{7.502328in}{1.231643in}}%
\pgfpathmoveto{\pgfqpoint{7.460662in}{1.189977in}}%
\pgfpathlineto{\pgfqpoint{7.460662in}{1.273310in}}%
\pgfusepath{stroke,fill}%
\end{pgfscope}%
\begin{pgfscope}%
\pgfpathrectangle{\pgfqpoint{5.562518in}{0.589870in}}{\pgfqpoint{4.376471in}{0.972632in}}%
\pgfusepath{clip}%
\pgfsetbuttcap%
\pgfsetroundjoin%
\definecolor{currentfill}{rgb}{1.000000,0.000000,0.000000}%
\pgfsetfillcolor{currentfill}%
\pgfsetlinewidth{2.007500pt}%
\definecolor{currentstroke}{rgb}{1.000000,0.000000,0.000000}%
\pgfsetstrokecolor{currentstroke}%
\pgfsetdash{}{0pt}%
\pgfpathmoveto{\pgfqpoint{7.678843in}{0.713376in}}%
\pgfpathlineto{\pgfqpoint{7.762176in}{0.713376in}}%
\pgfpathmoveto{\pgfqpoint{7.720509in}{0.671710in}}%
\pgfpathlineto{\pgfqpoint{7.720509in}{0.755043in}}%
\pgfusepath{stroke,fill}%
\end{pgfscope}%
\begin{pgfscope}%
\pgfpathrectangle{\pgfqpoint{5.562518in}{0.589870in}}{\pgfqpoint{4.376471in}{0.972632in}}%
\pgfusepath{clip}%
\pgfsetbuttcap%
\pgfsetroundjoin%
\definecolor{currentfill}{rgb}{1.000000,0.000000,0.000000}%
\pgfsetfillcolor{currentfill}%
\pgfsetlinewidth{2.007500pt}%
\definecolor{currentstroke}{rgb}{1.000000,0.000000,0.000000}%
\pgfsetstrokecolor{currentstroke}%
\pgfsetdash{}{0pt}%
\pgfpathmoveto{\pgfqpoint{7.992927in}{0.834575in}}%
\pgfpathlineto{\pgfqpoint{8.076260in}{0.834575in}}%
\pgfpathmoveto{\pgfqpoint{8.034593in}{0.792908in}}%
\pgfpathlineto{\pgfqpoint{8.034593in}{0.876242in}}%
\pgfusepath{stroke,fill}%
\end{pgfscope}%
\begin{pgfscope}%
\pgfpathrectangle{\pgfqpoint{5.562518in}{0.589870in}}{\pgfqpoint{4.376471in}{0.972632in}}%
\pgfusepath{clip}%
\pgfsetbuttcap%
\pgfsetroundjoin%
\definecolor{currentfill}{rgb}{1.000000,0.000000,0.000000}%
\pgfsetfillcolor{currentfill}%
\pgfsetlinewidth{2.007500pt}%
\definecolor{currentstroke}{rgb}{1.000000,0.000000,0.000000}%
\pgfsetstrokecolor{currentstroke}%
\pgfsetdash{}{0pt}%
\pgfpathmoveto{\pgfqpoint{9.145185in}{0.936434in}}%
\pgfpathlineto{\pgfqpoint{9.228518in}{0.936434in}}%
\pgfpathmoveto{\pgfqpoint{9.186851in}{0.894768in}}%
\pgfpathlineto{\pgfqpoint{9.186851in}{0.978101in}}%
\pgfusepath{stroke,fill}%
\end{pgfscope}%
\begin{pgfscope}%
\pgfpathrectangle{\pgfqpoint{5.562518in}{0.589870in}}{\pgfqpoint{4.376471in}{0.972632in}}%
\pgfusepath{clip}%
\pgfsetbuttcap%
\pgfsetroundjoin%
\definecolor{currentfill}{rgb}{1.000000,0.000000,0.000000}%
\pgfsetfillcolor{currentfill}%
\pgfsetlinewidth{2.007500pt}%
\definecolor{currentstroke}{rgb}{1.000000,0.000000,0.000000}%
\pgfsetstrokecolor{currentstroke}%
\pgfsetdash{}{0pt}%
\pgfpathmoveto{\pgfqpoint{7.095238in}{1.087584in}}%
\pgfpathlineto{\pgfqpoint{7.178572in}{1.087584in}}%
\pgfpathmoveto{\pgfqpoint{7.136905in}{1.045918in}}%
\pgfpathlineto{\pgfqpoint{7.136905in}{1.129251in}}%
\pgfusepath{stroke,fill}%
\end{pgfscope}%
\begin{pgfscope}%
\pgfpathrectangle{\pgfqpoint{5.562518in}{0.589870in}}{\pgfqpoint{4.376471in}{0.972632in}}%
\pgfusepath{clip}%
\pgfsetbuttcap%
\pgfsetroundjoin%
\definecolor{currentfill}{rgb}{1.000000,0.000000,0.000000}%
\pgfsetfillcolor{currentfill}%
\pgfsetlinewidth{2.007500pt}%
\definecolor{currentstroke}{rgb}{1.000000,0.000000,0.000000}%
\pgfsetstrokecolor{currentstroke}%
\pgfsetdash{}{0pt}%
\pgfpathmoveto{\pgfqpoint{8.196571in}{1.073819in}}%
\pgfpathlineto{\pgfqpoint{8.279904in}{1.073819in}}%
\pgfpathmoveto{\pgfqpoint{8.238237in}{1.032152in}}%
\pgfpathlineto{\pgfqpoint{8.238237in}{1.115486in}}%
\pgfusepath{stroke,fill}%
\end{pgfscope}%
\begin{pgfscope}%
\pgfpathrectangle{\pgfqpoint{5.562518in}{0.589870in}}{\pgfqpoint{4.376471in}{0.972632in}}%
\pgfusepath{clip}%
\pgfsetbuttcap%
\pgfsetroundjoin%
\definecolor{currentfill}{rgb}{1.000000,0.000000,0.000000}%
\pgfsetfillcolor{currentfill}%
\pgfsetlinewidth{2.007500pt}%
\definecolor{currentstroke}{rgb}{1.000000,0.000000,0.000000}%
\pgfsetstrokecolor{currentstroke}%
\pgfsetdash{}{0pt}%
\pgfpathmoveto{\pgfqpoint{8.470293in}{1.267788in}}%
\pgfpathlineto{\pgfqpoint{8.553626in}{1.267788in}}%
\pgfpathmoveto{\pgfqpoint{8.511960in}{1.226121in}}%
\pgfpathlineto{\pgfqpoint{8.511960in}{1.309455in}}%
\pgfusepath{stroke,fill}%
\end{pgfscope}%
\begin{pgfscope}%
\pgfpathrectangle{\pgfqpoint{5.562518in}{0.589870in}}{\pgfqpoint{4.376471in}{0.972632in}}%
\pgfusepath{clip}%
\pgfsetbuttcap%
\pgfsetroundjoin%
\definecolor{currentfill}{rgb}{1.000000,0.000000,0.000000}%
\pgfsetfillcolor{currentfill}%
\pgfsetlinewidth{2.007500pt}%
\definecolor{currentstroke}{rgb}{1.000000,0.000000,0.000000}%
\pgfsetstrokecolor{currentstroke}%
\pgfsetdash{}{0pt}%
\pgfpathmoveto{\pgfqpoint{6.558776in}{1.312221in}}%
\pgfpathlineto{\pgfqpoint{6.642110in}{1.312221in}}%
\pgfpathmoveto{\pgfqpoint{6.600443in}{1.270554in}}%
\pgfpathlineto{\pgfqpoint{6.600443in}{1.353887in}}%
\pgfusepath{stroke,fill}%
\end{pgfscope}%
\begin{pgfscope}%
\pgfpathrectangle{\pgfqpoint{5.562518in}{0.589870in}}{\pgfqpoint{4.376471in}{0.972632in}}%
\pgfusepath{clip}%
\pgfsetbuttcap%
\pgfsetroundjoin%
\definecolor{currentfill}{rgb}{0.000000,0.000000,0.000000}%
\pgfsetfillcolor{currentfill}%
\pgfsetlinewidth{1.003750pt}%
\definecolor{currentstroke}{rgb}{0.000000,0.000000,0.000000}%
\pgfsetstrokecolor{currentstroke}%
\pgfsetdash{}{0pt}%
\pgfsys@defobject{currentmarker}{\pgfqpoint{-0.020833in}{-0.020833in}}{\pgfqpoint{0.020833in}{0.020833in}}{%
\pgfpathmoveto{\pgfqpoint{0.000000in}{-0.020833in}}%
\pgfpathcurveto{\pgfqpoint{0.005525in}{-0.020833in}}{\pgfqpoint{0.010825in}{-0.018638in}}{\pgfqpoint{0.014731in}{-0.014731in}}%
\pgfpathcurveto{\pgfqpoint{0.018638in}{-0.010825in}}{\pgfqpoint{0.020833in}{-0.005525in}}{\pgfqpoint{0.020833in}{0.000000in}}%
\pgfpathcurveto{\pgfqpoint{0.020833in}{0.005525in}}{\pgfqpoint{0.018638in}{0.010825in}}{\pgfqpoint{0.014731in}{0.014731in}}%
\pgfpathcurveto{\pgfqpoint{0.010825in}{0.018638in}}{\pgfqpoint{0.005525in}{0.020833in}}{\pgfqpoint{0.000000in}{0.020833in}}%
\pgfpathcurveto{\pgfqpoint{-0.005525in}{0.020833in}}{\pgfqpoint{-0.010825in}{0.018638in}}{\pgfqpoint{-0.014731in}{0.014731in}}%
\pgfpathcurveto{\pgfqpoint{-0.018638in}{0.010825in}}{\pgfqpoint{-0.020833in}{0.005525in}}{\pgfqpoint{-0.020833in}{0.000000in}}%
\pgfpathcurveto{\pgfqpoint{-0.020833in}{-0.005525in}}{\pgfqpoint{-0.018638in}{-0.010825in}}{\pgfqpoint{-0.014731in}{-0.014731in}}%
\pgfpathcurveto{\pgfqpoint{-0.010825in}{-0.018638in}}{\pgfqpoint{-0.005525in}{-0.020833in}}{\pgfqpoint{0.000000in}{-0.020833in}}%
\pgfpathclose%
\pgfusepath{stroke,fill}%
}%
\begin{pgfscope}%
\pgfsys@transformshift{6.437812in}{1.385080in}%
\pgfsys@useobject{currentmarker}{}%
\end{pgfscope}%
\begin{pgfscope}%
\pgfsys@transformshift{6.455406in}{1.413058in}%
\pgfsys@useobject{currentmarker}{}%
\end{pgfscope}%
\begin{pgfscope}%
\pgfsys@transformshift{6.472999in}{1.451375in}%
\pgfsys@useobject{currentmarker}{}%
\end{pgfscope}%
\begin{pgfscope}%
\pgfsys@transformshift{6.490593in}{1.489412in}%
\pgfsys@useobject{currentmarker}{}%
\end{pgfscope}%
\begin{pgfscope}%
\pgfsys@transformshift{6.508187in}{1.314316in}%
\pgfsys@useobject{currentmarker}{}%
\end{pgfscope}%
\begin{pgfscope}%
\pgfsys@transformshift{6.525781in}{1.316074in}%
\pgfsys@useobject{currentmarker}{}%
\end{pgfscope}%
\begin{pgfscope}%
\pgfsys@transformshift{6.543375in}{1.194748in}%
\pgfsys@useobject{currentmarker}{}%
\end{pgfscope}%
\begin{pgfscope}%
\pgfsys@transformshift{6.560969in}{1.157734in}%
\pgfsys@useobject{currentmarker}{}%
\end{pgfscope}%
\begin{pgfscope}%
\pgfsys@transformshift{6.578563in}{1.333686in}%
\pgfsys@useobject{currentmarker}{}%
\end{pgfscope}%
\begin{pgfscope}%
\pgfsys@transformshift{6.596156in}{1.361753in}%
\pgfsys@useobject{currentmarker}{}%
\end{pgfscope}%
\begin{pgfscope}%
\pgfsys@transformshift{6.613750in}{1.190773in}%
\pgfsys@useobject{currentmarker}{}%
\end{pgfscope}%
\begin{pgfscope}%
\pgfsys@transformshift{6.631344in}{1.274401in}%
\pgfsys@useobject{currentmarker}{}%
\end{pgfscope}%
\begin{pgfscope}%
\pgfsys@transformshift{6.648938in}{1.185141in}%
\pgfsys@useobject{currentmarker}{}%
\end{pgfscope}%
\begin{pgfscope}%
\pgfsys@transformshift{6.666532in}{1.060203in}%
\pgfsys@useobject{currentmarker}{}%
\end{pgfscope}%
\begin{pgfscope}%
\pgfsys@transformshift{6.684126in}{1.140689in}%
\pgfsys@useobject{currentmarker}{}%
\end{pgfscope}%
\begin{pgfscope}%
\pgfsys@transformshift{6.701720in}{1.240004in}%
\pgfsys@useobject{currentmarker}{}%
\end{pgfscope}%
\begin{pgfscope}%
\pgfsys@transformshift{6.719313in}{1.062375in}%
\pgfsys@useobject{currentmarker}{}%
\end{pgfscope}%
\begin{pgfscope}%
\pgfsys@transformshift{6.736907in}{1.208193in}%
\pgfsys@useobject{currentmarker}{}%
\end{pgfscope}%
\begin{pgfscope}%
\pgfsys@transformshift{6.754501in}{0.769848in}%
\pgfsys@useobject{currentmarker}{}%
\end{pgfscope}%
\begin{pgfscope}%
\pgfsys@transformshift{6.772095in}{1.106153in}%
\pgfsys@useobject{currentmarker}{}%
\end{pgfscope}%
\begin{pgfscope}%
\pgfsys@transformshift{6.789689in}{1.021382in}%
\pgfsys@useobject{currentmarker}{}%
\end{pgfscope}%
\begin{pgfscope}%
\pgfsys@transformshift{6.807283in}{0.974058in}%
\pgfsys@useobject{currentmarker}{}%
\end{pgfscope}%
\begin{pgfscope}%
\pgfsys@transformshift{6.824876in}{1.007564in}%
\pgfsys@useobject{currentmarker}{}%
\end{pgfscope}%
\begin{pgfscope}%
\pgfsys@transformshift{6.842470in}{0.792933in}%
\pgfsys@useobject{currentmarker}{}%
\end{pgfscope}%
\begin{pgfscope}%
\pgfsys@transformshift{6.860064in}{0.970186in}%
\pgfsys@useobject{currentmarker}{}%
\end{pgfscope}%
\begin{pgfscope}%
\pgfsys@transformshift{6.877658in}{1.028815in}%
\pgfsys@useobject{currentmarker}{}%
\end{pgfscope}%
\begin{pgfscope}%
\pgfsys@transformshift{6.895252in}{1.144550in}%
\pgfsys@useobject{currentmarker}{}%
\end{pgfscope}%
\begin{pgfscope}%
\pgfsys@transformshift{6.912846in}{0.946398in}%
\pgfsys@useobject{currentmarker}{}%
\end{pgfscope}%
\begin{pgfscope}%
\pgfsys@transformshift{6.930440in}{0.922890in}%
\pgfsys@useobject{currentmarker}{}%
\end{pgfscope}%
\begin{pgfscope}%
\pgfsys@transformshift{6.948033in}{0.961551in}%
\pgfsys@useobject{currentmarker}{}%
\end{pgfscope}%
\begin{pgfscope}%
\pgfsys@transformshift{6.965627in}{1.114268in}%
\pgfsys@useobject{currentmarker}{}%
\end{pgfscope}%
\begin{pgfscope}%
\pgfsys@transformshift{6.983221in}{1.065369in}%
\pgfsys@useobject{currentmarker}{}%
\end{pgfscope}%
\begin{pgfscope}%
\pgfsys@transformshift{7.000815in}{0.990165in}%
\pgfsys@useobject{currentmarker}{}%
\end{pgfscope}%
\begin{pgfscope}%
\pgfsys@transformshift{7.018409in}{1.108681in}%
\pgfsys@useobject{currentmarker}{}%
\end{pgfscope}%
\begin{pgfscope}%
\pgfsys@transformshift{7.036003in}{1.080233in}%
\pgfsys@useobject{currentmarker}{}%
\end{pgfscope}%
\begin{pgfscope}%
\pgfsys@transformshift{7.053597in}{1.182941in}%
\pgfsys@useobject{currentmarker}{}%
\end{pgfscope}%
\begin{pgfscope}%
\pgfsys@transformshift{7.071190in}{1.028565in}%
\pgfsys@useobject{currentmarker}{}%
\end{pgfscope}%
\begin{pgfscope}%
\pgfsys@transformshift{7.088784in}{1.081671in}%
\pgfsys@useobject{currentmarker}{}%
\end{pgfscope}%
\begin{pgfscope}%
\pgfsys@transformshift{7.106378in}{1.090395in}%
\pgfsys@useobject{currentmarker}{}%
\end{pgfscope}%
\begin{pgfscope}%
\pgfsys@transformshift{7.123972in}{0.996970in}%
\pgfsys@useobject{currentmarker}{}%
\end{pgfscope}%
\begin{pgfscope}%
\pgfsys@transformshift{7.141566in}{1.190045in}%
\pgfsys@useobject{currentmarker}{}%
\end{pgfscope}%
\begin{pgfscope}%
\pgfsys@transformshift{7.159160in}{1.200760in}%
\pgfsys@useobject{currentmarker}{}%
\end{pgfscope}%
\begin{pgfscope}%
\pgfsys@transformshift{7.176754in}{1.188374in}%
\pgfsys@useobject{currentmarker}{}%
\end{pgfscope}%
\begin{pgfscope}%
\pgfsys@transformshift{7.194347in}{1.176728in}%
\pgfsys@useobject{currentmarker}{}%
\end{pgfscope}%
\begin{pgfscope}%
\pgfsys@transformshift{7.211941in}{1.068654in}%
\pgfsys@useobject{currentmarker}{}%
\end{pgfscope}%
\begin{pgfscope}%
\pgfsys@transformshift{7.229535in}{1.179750in}%
\pgfsys@useobject{currentmarker}{}%
\end{pgfscope}%
\begin{pgfscope}%
\pgfsys@transformshift{7.247129in}{1.196567in}%
\pgfsys@useobject{currentmarker}{}%
\end{pgfscope}%
\begin{pgfscope}%
\pgfsys@transformshift{7.264723in}{1.157399in}%
\pgfsys@useobject{currentmarker}{}%
\end{pgfscope}%
\begin{pgfscope}%
\pgfsys@transformshift{7.282317in}{1.228089in}%
\pgfsys@useobject{currentmarker}{}%
\end{pgfscope}%
\begin{pgfscope}%
\pgfsys@transformshift{7.299910in}{1.289366in}%
\pgfsys@useobject{currentmarker}{}%
\end{pgfscope}%
\begin{pgfscope}%
\pgfsys@transformshift{7.317504in}{1.441699in}%
\pgfsys@useobject{currentmarker}{}%
\end{pgfscope}%
\begin{pgfscope}%
\pgfsys@transformshift{7.335098in}{1.268562in}%
\pgfsys@useobject{currentmarker}{}%
\end{pgfscope}%
\begin{pgfscope}%
\pgfsys@transformshift{7.352692in}{1.275309in}%
\pgfsys@useobject{currentmarker}{}%
\end{pgfscope}%
\begin{pgfscope}%
\pgfsys@transformshift{7.370286in}{1.238053in}%
\pgfsys@useobject{currentmarker}{}%
\end{pgfscope}%
\begin{pgfscope}%
\pgfsys@transformshift{7.387880in}{1.045613in}%
\pgfsys@useobject{currentmarker}{}%
\end{pgfscope}%
\begin{pgfscope}%
\pgfsys@transformshift{7.405474in}{1.229804in}%
\pgfsys@useobject{currentmarker}{}%
\end{pgfscope}%
\begin{pgfscope}%
\pgfsys@transformshift{7.423067in}{1.229164in}%
\pgfsys@useobject{currentmarker}{}%
\end{pgfscope}%
\begin{pgfscope}%
\pgfsys@transformshift{7.440661in}{1.461353in}%
\pgfsys@useobject{currentmarker}{}%
\end{pgfscope}%
\begin{pgfscope}%
\pgfsys@transformshift{7.458255in}{1.179256in}%
\pgfsys@useobject{currentmarker}{}%
\end{pgfscope}%
\begin{pgfscope}%
\pgfsys@transformshift{7.475849in}{1.214585in}%
\pgfsys@useobject{currentmarker}{}%
\end{pgfscope}%
\begin{pgfscope}%
\pgfsys@transformshift{7.493443in}{1.164248in}%
\pgfsys@useobject{currentmarker}{}%
\end{pgfscope}%
\begin{pgfscope}%
\pgfsys@transformshift{7.511037in}{1.031666in}%
\pgfsys@useobject{currentmarker}{}%
\end{pgfscope}%
\begin{pgfscope}%
\pgfsys@transformshift{7.528631in}{1.246886in}%
\pgfsys@useobject{currentmarker}{}%
\end{pgfscope}%
\begin{pgfscope}%
\pgfsys@transformshift{7.546224in}{1.187193in}%
\pgfsys@useobject{currentmarker}{}%
\end{pgfscope}%
\begin{pgfscope}%
\pgfsys@transformshift{7.563818in}{1.170118in}%
\pgfsys@useobject{currentmarker}{}%
\end{pgfscope}%
\begin{pgfscope}%
\pgfsys@transformshift{7.581412in}{0.976039in}%
\pgfsys@useobject{currentmarker}{}%
\end{pgfscope}%
\begin{pgfscope}%
\pgfsys@transformshift{7.599006in}{1.187927in}%
\pgfsys@useobject{currentmarker}{}%
\end{pgfscope}%
\begin{pgfscope}%
\pgfsys@transformshift{7.616600in}{0.881025in}%
\pgfsys@useobject{currentmarker}{}%
\end{pgfscope}%
\begin{pgfscope}%
\pgfsys@transformshift{7.634194in}{1.059593in}%
\pgfsys@useobject{currentmarker}{}%
\end{pgfscope}%
\begin{pgfscope}%
\pgfsys@transformshift{7.651788in}{1.199175in}%
\pgfsys@useobject{currentmarker}{}%
\end{pgfscope}%
\begin{pgfscope}%
\pgfsys@transformshift{7.669381in}{0.854241in}%
\pgfsys@useobject{currentmarker}{}%
\end{pgfscope}%
\begin{pgfscope}%
\pgfsys@transformshift{7.686975in}{0.875015in}%
\pgfsys@useobject{currentmarker}{}%
\end{pgfscope}%
\begin{pgfscope}%
\pgfsys@transformshift{7.704569in}{0.920919in}%
\pgfsys@useobject{currentmarker}{}%
\end{pgfscope}%
\begin{pgfscope}%
\pgfsys@transformshift{7.722163in}{0.839082in}%
\pgfsys@useobject{currentmarker}{}%
\end{pgfscope}%
\begin{pgfscope}%
\pgfsys@transformshift{7.739757in}{0.713278in}%
\pgfsys@useobject{currentmarker}{}%
\end{pgfscope}%
\begin{pgfscope}%
\pgfsys@transformshift{7.757351in}{0.858825in}%
\pgfsys@useobject{currentmarker}{}%
\end{pgfscope}%
\begin{pgfscope}%
\pgfsys@transformshift{7.774944in}{0.727110in}%
\pgfsys@useobject{currentmarker}{}%
\end{pgfscope}%
\begin{pgfscope}%
\pgfsys@transformshift{7.792538in}{0.867100in}%
\pgfsys@useobject{currentmarker}{}%
\end{pgfscope}%
\begin{pgfscope}%
\pgfsys@transformshift{7.810132in}{0.712003in}%
\pgfsys@useobject{currentmarker}{}%
\end{pgfscope}%
\begin{pgfscope}%
\pgfsys@transformshift{7.827726in}{0.950007in}%
\pgfsys@useobject{currentmarker}{}%
\end{pgfscope}%
\begin{pgfscope}%
\pgfsys@transformshift{7.845320in}{0.703324in}%
\pgfsys@useobject{currentmarker}{}%
\end{pgfscope}%
\begin{pgfscope}%
\pgfsys@transformshift{7.862914in}{0.741730in}%
\pgfsys@useobject{currentmarker}{}%
\end{pgfscope}%
\begin{pgfscope}%
\pgfsys@transformshift{7.880508in}{0.850504in}%
\pgfsys@useobject{currentmarker}{}%
\end{pgfscope}%
\begin{pgfscope}%
\pgfsys@transformshift{7.898101in}{0.639189in}%
\pgfsys@useobject{currentmarker}{}%
\end{pgfscope}%
\begin{pgfscope}%
\pgfsys@transformshift{7.915695in}{0.784873in}%
\pgfsys@useobject{currentmarker}{}%
\end{pgfscope}%
\begin{pgfscope}%
\pgfsys@transformshift{7.933289in}{0.894320in}%
\pgfsys@useobject{currentmarker}{}%
\end{pgfscope}%
\begin{pgfscope}%
\pgfsys@transformshift{7.950883in}{0.601195in}%
\pgfsys@useobject{currentmarker}{}%
\end{pgfscope}%
\begin{pgfscope}%
\pgfsys@transformshift{7.968477in}{0.787018in}%
\pgfsys@useobject{currentmarker}{}%
\end{pgfscope}%
\begin{pgfscope}%
\pgfsys@transformshift{7.986071in}{0.800921in}%
\pgfsys@useobject{currentmarker}{}%
\end{pgfscope}%
\begin{pgfscope}%
\pgfsys@transformshift{8.003665in}{0.862034in}%
\pgfsys@useobject{currentmarker}{}%
\end{pgfscope}%
\begin{pgfscope}%
\pgfsys@transformshift{8.021258in}{0.667594in}%
\pgfsys@useobject{currentmarker}{}%
\end{pgfscope}%
\begin{pgfscope}%
\pgfsys@transformshift{8.038852in}{0.670979in}%
\pgfsys@useobject{currentmarker}{}%
\end{pgfscope}%
\begin{pgfscope}%
\pgfsys@transformshift{8.056446in}{0.871111in}%
\pgfsys@useobject{currentmarker}{}%
\end{pgfscope}%
\begin{pgfscope}%
\pgfsys@transformshift{8.074040in}{0.863270in}%
\pgfsys@useobject{currentmarker}{}%
\end{pgfscope}%
\begin{pgfscope}%
\pgfsys@transformshift{8.091634in}{0.874834in}%
\pgfsys@useobject{currentmarker}{}%
\end{pgfscope}%
\begin{pgfscope}%
\pgfsys@transformshift{8.109228in}{0.901987in}%
\pgfsys@useobject{currentmarker}{}%
\end{pgfscope}%
\begin{pgfscope}%
\pgfsys@transformshift{8.126822in}{0.816395in}%
\pgfsys@useobject{currentmarker}{}%
\end{pgfscope}%
\begin{pgfscope}%
\pgfsys@transformshift{8.144415in}{0.928015in}%
\pgfsys@useobject{currentmarker}{}%
\end{pgfscope}%
\begin{pgfscope}%
\pgfsys@transformshift{8.162009in}{0.953960in}%
\pgfsys@useobject{currentmarker}{}%
\end{pgfscope}%
\begin{pgfscope}%
\pgfsys@transformshift{8.179603in}{0.872064in}%
\pgfsys@useobject{currentmarker}{}%
\end{pgfscope}%
\begin{pgfscope}%
\pgfsys@transformshift{8.197197in}{1.153828in}%
\pgfsys@useobject{currentmarker}{}%
\end{pgfscope}%
\begin{pgfscope}%
\pgfsys@transformshift{8.214791in}{1.033138in}%
\pgfsys@useobject{currentmarker}{}%
\end{pgfscope}%
\begin{pgfscope}%
\pgfsys@transformshift{8.232385in}{0.884544in}%
\pgfsys@useobject{currentmarker}{}%
\end{pgfscope}%
\begin{pgfscope}%
\pgfsys@transformshift{8.249978in}{1.091444in}%
\pgfsys@useobject{currentmarker}{}%
\end{pgfscope}%
\begin{pgfscope}%
\pgfsys@transformshift{8.267572in}{0.945236in}%
\pgfsys@useobject{currentmarker}{}%
\end{pgfscope}%
\begin{pgfscope}%
\pgfsys@transformshift{8.285166in}{1.141979in}%
\pgfsys@useobject{currentmarker}{}%
\end{pgfscope}%
\begin{pgfscope}%
\pgfsys@transformshift{8.302760in}{1.196877in}%
\pgfsys@useobject{currentmarker}{}%
\end{pgfscope}%
\begin{pgfscope}%
\pgfsys@transformshift{8.320354in}{1.012441in}%
\pgfsys@useobject{currentmarker}{}%
\end{pgfscope}%
\begin{pgfscope}%
\pgfsys@transformshift{8.337948in}{1.207973in}%
\pgfsys@useobject{currentmarker}{}%
\end{pgfscope}%
\begin{pgfscope}%
\pgfsys@transformshift{8.355542in}{1.165507in}%
\pgfsys@useobject{currentmarker}{}%
\end{pgfscope}%
\begin{pgfscope}%
\pgfsys@transformshift{8.373135in}{1.218705in}%
\pgfsys@useobject{currentmarker}{}%
\end{pgfscope}%
\begin{pgfscope}%
\pgfsys@transformshift{8.390729in}{1.337626in}%
\pgfsys@useobject{currentmarker}{}%
\end{pgfscope}%
\begin{pgfscope}%
\pgfsys@transformshift{8.408323in}{1.128833in}%
\pgfsys@useobject{currentmarker}{}%
\end{pgfscope}%
\begin{pgfscope}%
\pgfsys@transformshift{8.425917in}{1.083711in}%
\pgfsys@useobject{currentmarker}{}%
\end{pgfscope}%
\begin{pgfscope}%
\pgfsys@transformshift{8.443511in}{1.074419in}%
\pgfsys@useobject{currentmarker}{}%
\end{pgfscope}%
\begin{pgfscope}%
\pgfsys@transformshift{8.461105in}{1.084402in}%
\pgfsys@useobject{currentmarker}{}%
\end{pgfscope}%
\begin{pgfscope}%
\pgfsys@transformshift{8.478699in}{1.159796in}%
\pgfsys@useobject{currentmarker}{}%
\end{pgfscope}%
\begin{pgfscope}%
\pgfsys@transformshift{8.496292in}{1.200769in}%
\pgfsys@useobject{currentmarker}{}%
\end{pgfscope}%
\begin{pgfscope}%
\pgfsys@transformshift{8.513886in}{1.190907in}%
\pgfsys@useobject{currentmarker}{}%
\end{pgfscope}%
\begin{pgfscope}%
\pgfsys@transformshift{8.531480in}{1.241472in}%
\pgfsys@useobject{currentmarker}{}%
\end{pgfscope}%
\begin{pgfscope}%
\pgfsys@transformshift{8.549074in}{1.151963in}%
\pgfsys@useobject{currentmarker}{}%
\end{pgfscope}%
\begin{pgfscope}%
\pgfsys@transformshift{8.566668in}{1.289169in}%
\pgfsys@useobject{currentmarker}{}%
\end{pgfscope}%
\begin{pgfscope}%
\pgfsys@transformshift{8.584262in}{1.104729in}%
\pgfsys@useobject{currentmarker}{}%
\end{pgfscope}%
\begin{pgfscope}%
\pgfsys@transformshift{8.601855in}{1.395285in}%
\pgfsys@useobject{currentmarker}{}%
\end{pgfscope}%
\begin{pgfscope}%
\pgfsys@transformshift{8.619449in}{1.169868in}%
\pgfsys@useobject{currentmarker}{}%
\end{pgfscope}%
\begin{pgfscope}%
\pgfsys@transformshift{8.637043in}{1.005221in}%
\pgfsys@useobject{currentmarker}{}%
\end{pgfscope}%
\begin{pgfscope}%
\pgfsys@transformshift{8.654637in}{0.968116in}%
\pgfsys@useobject{currentmarker}{}%
\end{pgfscope}%
\begin{pgfscope}%
\pgfsys@transformshift{8.672231in}{1.109187in}%
\pgfsys@useobject{currentmarker}{}%
\end{pgfscope}%
\begin{pgfscope}%
\pgfsys@transformshift{8.689825in}{1.020683in}%
\pgfsys@useobject{currentmarker}{}%
\end{pgfscope}%
\begin{pgfscope}%
\pgfsys@transformshift{8.707419in}{1.098203in}%
\pgfsys@useobject{currentmarker}{}%
\end{pgfscope}%
\begin{pgfscope}%
\pgfsys@transformshift{8.725012in}{1.056072in}%
\pgfsys@useobject{currentmarker}{}%
\end{pgfscope}%
\begin{pgfscope}%
\pgfsys@transformshift{8.742606in}{0.982935in}%
\pgfsys@useobject{currentmarker}{}%
\end{pgfscope}%
\begin{pgfscope}%
\pgfsys@transformshift{8.760200in}{0.886840in}%
\pgfsys@useobject{currentmarker}{}%
\end{pgfscope}%
\begin{pgfscope}%
\pgfsys@transformshift{8.777794in}{0.801812in}%
\pgfsys@useobject{currentmarker}{}%
\end{pgfscope}%
\begin{pgfscope}%
\pgfsys@transformshift{8.795388in}{0.893246in}%
\pgfsys@useobject{currentmarker}{}%
\end{pgfscope}%
\begin{pgfscope}%
\pgfsys@transformshift{8.812982in}{1.009182in}%
\pgfsys@useobject{currentmarker}{}%
\end{pgfscope}%
\begin{pgfscope}%
\pgfsys@transformshift{8.830576in}{0.928965in}%
\pgfsys@useobject{currentmarker}{}%
\end{pgfscope}%
\begin{pgfscope}%
\pgfsys@transformshift{8.848169in}{0.767030in}%
\pgfsys@useobject{currentmarker}{}%
\end{pgfscope}%
\begin{pgfscope}%
\pgfsys@transformshift{8.865763in}{0.898039in}%
\pgfsys@useobject{currentmarker}{}%
\end{pgfscope}%
\begin{pgfscope}%
\pgfsys@transformshift{8.883357in}{0.908225in}%
\pgfsys@useobject{currentmarker}{}%
\end{pgfscope}%
\begin{pgfscope}%
\pgfsys@transformshift{8.900951in}{0.769918in}%
\pgfsys@useobject{currentmarker}{}%
\end{pgfscope}%
\begin{pgfscope}%
\pgfsys@transformshift{8.918545in}{0.867043in}%
\pgfsys@useobject{currentmarker}{}%
\end{pgfscope}%
\begin{pgfscope}%
\pgfsys@transformshift{8.936139in}{0.851201in}%
\pgfsys@useobject{currentmarker}{}%
\end{pgfscope}%
\begin{pgfscope}%
\pgfsys@transformshift{8.953733in}{0.725269in}%
\pgfsys@useobject{currentmarker}{}%
\end{pgfscope}%
\begin{pgfscope}%
\pgfsys@transformshift{8.971326in}{0.875095in}%
\pgfsys@useobject{currentmarker}{}%
\end{pgfscope}%
\begin{pgfscope}%
\pgfsys@transformshift{8.988920in}{0.895506in}%
\pgfsys@useobject{currentmarker}{}%
\end{pgfscope}%
\begin{pgfscope}%
\pgfsys@transformshift{9.006514in}{0.950371in}%
\pgfsys@useobject{currentmarker}{}%
\end{pgfscope}%
\begin{pgfscope}%
\pgfsys@transformshift{9.024108in}{0.951484in}%
\pgfsys@useobject{currentmarker}{}%
\end{pgfscope}%
\begin{pgfscope}%
\pgfsys@transformshift{9.041702in}{0.711337in}%
\pgfsys@useobject{currentmarker}{}%
\end{pgfscope}%
\begin{pgfscope}%
\pgfsys@transformshift{9.059296in}{0.764200in}%
\pgfsys@useobject{currentmarker}{}%
\end{pgfscope}%
\begin{pgfscope}%
\pgfsys@transformshift{9.076889in}{0.921754in}%
\pgfsys@useobject{currentmarker}{}%
\end{pgfscope}%
\begin{pgfscope}%
\pgfsys@transformshift{9.094483in}{0.933977in}%
\pgfsys@useobject{currentmarker}{}%
\end{pgfscope}%
\begin{pgfscope}%
\pgfsys@transformshift{9.112077in}{0.948366in}%
\pgfsys@useobject{currentmarker}{}%
\end{pgfscope}%
\begin{pgfscope}%
\pgfsys@transformshift{9.129671in}{1.302598in}%
\pgfsys@useobject{currentmarker}{}%
\end{pgfscope}%
\begin{pgfscope}%
\pgfsys@transformshift{9.147265in}{0.987858in}%
\pgfsys@useobject{currentmarker}{}%
\end{pgfscope}%
\begin{pgfscope}%
\pgfsys@transformshift{9.164859in}{1.064389in}%
\pgfsys@useobject{currentmarker}{}%
\end{pgfscope}%
\begin{pgfscope}%
\pgfsys@transformshift{9.182453in}{1.066721in}%
\pgfsys@useobject{currentmarker}{}%
\end{pgfscope}%
\begin{pgfscope}%
\pgfsys@transformshift{9.200046in}{1.058035in}%
\pgfsys@useobject{currentmarker}{}%
\end{pgfscope}%
\begin{pgfscope}%
\pgfsys@transformshift{9.217640in}{0.983139in}%
\pgfsys@useobject{currentmarker}{}%
\end{pgfscope}%
\begin{pgfscope}%
\pgfsys@transformshift{9.235234in}{1.115904in}%
\pgfsys@useobject{currentmarker}{}%
\end{pgfscope}%
\begin{pgfscope}%
\pgfsys@transformshift{9.252828in}{0.985329in}%
\pgfsys@useobject{currentmarker}{}%
\end{pgfscope}%
\begin{pgfscope}%
\pgfsys@transformshift{9.270422in}{1.064749in}%
\pgfsys@useobject{currentmarker}{}%
\end{pgfscope}%
\begin{pgfscope}%
\pgfsys@transformshift{9.288016in}{1.064973in}%
\pgfsys@useobject{currentmarker}{}%
\end{pgfscope}%
\begin{pgfscope}%
\pgfsys@transformshift{9.305610in}{1.147933in}%
\pgfsys@useobject{currentmarker}{}%
\end{pgfscope}%
\begin{pgfscope}%
\pgfsys@transformshift{9.323203in}{1.399520in}%
\pgfsys@useobject{currentmarker}{}%
\end{pgfscope}%
\begin{pgfscope}%
\pgfsys@transformshift{9.340797in}{1.000873in}%
\pgfsys@useobject{currentmarker}{}%
\end{pgfscope}%
\begin{pgfscope}%
\pgfsys@transformshift{9.358391in}{1.284111in}%
\pgfsys@useobject{currentmarker}{}%
\end{pgfscope}%
\begin{pgfscope}%
\pgfsys@transformshift{9.375985in}{1.074999in}%
\pgfsys@useobject{currentmarker}{}%
\end{pgfscope}%
\begin{pgfscope}%
\pgfsys@transformshift{9.393579in}{1.213490in}%
\pgfsys@useobject{currentmarker}{}%
\end{pgfscope}%
\begin{pgfscope}%
\pgfsys@transformshift{9.411173in}{1.393468in}%
\pgfsys@useobject{currentmarker}{}%
\end{pgfscope}%
\begin{pgfscope}%
\pgfsys@transformshift{9.428767in}{1.310250in}%
\pgfsys@useobject{currentmarker}{}%
\end{pgfscope}%
\begin{pgfscope}%
\pgfsys@transformshift{9.446360in}{1.213755in}%
\pgfsys@useobject{currentmarker}{}%
\end{pgfscope}%
\begin{pgfscope}%
\pgfsys@transformshift{9.463954in}{1.268161in}%
\pgfsys@useobject{currentmarker}{}%
\end{pgfscope}%
\begin{pgfscope}%
\pgfsys@transformshift{9.481548in}{1.425530in}%
\pgfsys@useobject{currentmarker}{}%
\end{pgfscope}%
\begin{pgfscope}%
\pgfsys@transformshift{9.499142in}{1.296978in}%
\pgfsys@useobject{currentmarker}{}%
\end{pgfscope}%
\begin{pgfscope}%
\pgfsys@transformshift{9.516736in}{1.405378in}%
\pgfsys@useobject{currentmarker}{}%
\end{pgfscope}%
\begin{pgfscope}%
\pgfsys@transformshift{9.534330in}{1.398646in}%
\pgfsys@useobject{currentmarker}{}%
\end{pgfscope}%
\begin{pgfscope}%
\pgfsys@transformshift{9.551923in}{1.336657in}%
\pgfsys@useobject{currentmarker}{}%
\end{pgfscope}%
\begin{pgfscope}%
\pgfsys@transformshift{9.569517in}{1.626576in}%
\pgfsys@useobject{currentmarker}{}%
\end{pgfscope}%
\begin{pgfscope}%
\pgfsys@transformshift{9.587111in}{1.478312in}%
\pgfsys@useobject{currentmarker}{}%
\end{pgfscope}%
\begin{pgfscope}%
\pgfsys@transformshift{9.604705in}{1.211687in}%
\pgfsys@useobject{currentmarker}{}%
\end{pgfscope}%
\begin{pgfscope}%
\pgfsys@transformshift{9.622299in}{1.436632in}%
\pgfsys@useobject{currentmarker}{}%
\end{pgfscope}%
\begin{pgfscope}%
\pgfsys@transformshift{9.639893in}{1.349718in}%
\pgfsys@useobject{currentmarker}{}%
\end{pgfscope}%
\begin{pgfscope}%
\pgfsys@transformshift{9.657487in}{1.500388in}%
\pgfsys@useobject{currentmarker}{}%
\end{pgfscope}%
\begin{pgfscope}%
\pgfsys@transformshift{9.675080in}{1.329309in}%
\pgfsys@useobject{currentmarker}{}%
\end{pgfscope}%
\begin{pgfscope}%
\pgfsys@transformshift{9.692674in}{1.392001in}%
\pgfsys@useobject{currentmarker}{}%
\end{pgfscope}%
\begin{pgfscope}%
\pgfsys@transformshift{9.710268in}{1.447380in}%
\pgfsys@useobject{currentmarker}{}%
\end{pgfscope}%
\begin{pgfscope}%
\pgfsys@transformshift{9.727862in}{1.475239in}%
\pgfsys@useobject{currentmarker}{}%
\end{pgfscope}%
\begin{pgfscope}%
\pgfsys@transformshift{9.745456in}{1.256099in}%
\pgfsys@useobject{currentmarker}{}%
\end{pgfscope}%
\begin{pgfscope}%
\pgfsys@transformshift{9.763050in}{1.333048in}%
\pgfsys@useobject{currentmarker}{}%
\end{pgfscope}%
\begin{pgfscope}%
\pgfsys@transformshift{9.780644in}{1.307280in}%
\pgfsys@useobject{currentmarker}{}%
\end{pgfscope}%
\begin{pgfscope}%
\pgfsys@transformshift{9.798237in}{1.277088in}%
\pgfsys@useobject{currentmarker}{}%
\end{pgfscope}%
\begin{pgfscope}%
\pgfsys@transformshift{9.815831in}{1.509648in}%
\pgfsys@useobject{currentmarker}{}%
\end{pgfscope}%
\begin{pgfscope}%
\pgfsys@transformshift{9.833425in}{1.359130in}%
\pgfsys@useobject{currentmarker}{}%
\end{pgfscope}%
\begin{pgfscope}%
\pgfsys@transformshift{9.851019in}{1.177696in}%
\pgfsys@useobject{currentmarker}{}%
\end{pgfscope}%
\begin{pgfscope}%
\pgfsys@transformshift{9.868613in}{1.386012in}%
\pgfsys@useobject{currentmarker}{}%
\end{pgfscope}%
\begin{pgfscope}%
\pgfsys@transformshift{9.886207in}{1.496035in}%
\pgfsys@useobject{currentmarker}{}%
\end{pgfscope}%
\begin{pgfscope}%
\pgfsys@transformshift{9.903801in}{1.374276in}%
\pgfsys@useobject{currentmarker}{}%
\end{pgfscope}%
\begin{pgfscope}%
\pgfsys@transformshift{9.921394in}{1.105208in}%
\pgfsys@useobject{currentmarker}{}%
\end{pgfscope}%
\begin{pgfscope}%
\pgfsys@transformshift{9.938988in}{1.200574in}%
\pgfsys@useobject{currentmarker}{}%
\end{pgfscope}%
\end{pgfscope}%
\begin{pgfscope}%
\pgfsetbuttcap%
\pgfsetroundjoin%
\definecolor{currentfill}{rgb}{0.000000,0.000000,0.000000}%
\pgfsetfillcolor{currentfill}%
\pgfsetlinewidth{0.803000pt}%
\definecolor{currentstroke}{rgb}{0.000000,0.000000,0.000000}%
\pgfsetstrokecolor{currentstroke}%
\pgfsetdash{}{0pt}%
\pgfsys@defobject{currentmarker}{\pgfqpoint{0.000000in}{-0.048611in}}{\pgfqpoint{0.000000in}{0.000000in}}{%
\pgfpathmoveto{\pgfqpoint{0.000000in}{0.000000in}}%
\pgfpathlineto{\pgfqpoint{0.000000in}{-0.048611in}}%
\pgfusepath{stroke,fill}%
}%
\begin{pgfscope}%
\pgfsys@transformshift{5.562518in}{0.589870in}%
\pgfsys@useobject{currentmarker}{}%
\end{pgfscope}%
\end{pgfscope}%
\begin{pgfscope}%
\pgftext[x=5.562518in,y=0.492648in,,top]{\rmfamily\fontsize{10.000000}{12.000000}\selectfont \(\displaystyle -1.5\)}%
\end{pgfscope}%
\begin{pgfscope}%
\pgfsetbuttcap%
\pgfsetroundjoin%
\definecolor{currentfill}{rgb}{0.000000,0.000000,0.000000}%
\pgfsetfillcolor{currentfill}%
\pgfsetlinewidth{0.803000pt}%
\definecolor{currentstroke}{rgb}{0.000000,0.000000,0.000000}%
\pgfsetstrokecolor{currentstroke}%
\pgfsetdash{}{0pt}%
\pgfsys@defobject{currentmarker}{\pgfqpoint{0.000000in}{-0.048611in}}{\pgfqpoint{0.000000in}{0.000000in}}{%
\pgfpathmoveto{\pgfqpoint{0.000000in}{0.000000in}}%
\pgfpathlineto{\pgfqpoint{0.000000in}{-0.048611in}}%
\pgfusepath{stroke,fill}%
}%
\begin{pgfscope}%
\pgfsys@transformshift{6.437812in}{0.589870in}%
\pgfsys@useobject{currentmarker}{}%
\end{pgfscope}%
\end{pgfscope}%
\begin{pgfscope}%
\pgftext[x=6.437812in,y=0.492648in,,top]{\rmfamily\fontsize{10.000000}{12.000000}\selectfont \(\displaystyle -1.0\)}%
\end{pgfscope}%
\begin{pgfscope}%
\pgfsetbuttcap%
\pgfsetroundjoin%
\definecolor{currentfill}{rgb}{0.000000,0.000000,0.000000}%
\pgfsetfillcolor{currentfill}%
\pgfsetlinewidth{0.803000pt}%
\definecolor{currentstroke}{rgb}{0.000000,0.000000,0.000000}%
\pgfsetstrokecolor{currentstroke}%
\pgfsetdash{}{0pt}%
\pgfsys@defobject{currentmarker}{\pgfqpoint{0.000000in}{-0.048611in}}{\pgfqpoint{0.000000in}{0.000000in}}{%
\pgfpathmoveto{\pgfqpoint{0.000000in}{0.000000in}}%
\pgfpathlineto{\pgfqpoint{0.000000in}{-0.048611in}}%
\pgfusepath{stroke,fill}%
}%
\begin{pgfscope}%
\pgfsys@transformshift{7.313106in}{0.589870in}%
\pgfsys@useobject{currentmarker}{}%
\end{pgfscope}%
\end{pgfscope}%
\begin{pgfscope}%
\pgftext[x=7.313106in,y=0.492648in,,top]{\rmfamily\fontsize{10.000000}{12.000000}\selectfont \(\displaystyle -0.5\)}%
\end{pgfscope}%
\begin{pgfscope}%
\pgfsetbuttcap%
\pgfsetroundjoin%
\definecolor{currentfill}{rgb}{0.000000,0.000000,0.000000}%
\pgfsetfillcolor{currentfill}%
\pgfsetlinewidth{0.803000pt}%
\definecolor{currentstroke}{rgb}{0.000000,0.000000,0.000000}%
\pgfsetstrokecolor{currentstroke}%
\pgfsetdash{}{0pt}%
\pgfsys@defobject{currentmarker}{\pgfqpoint{0.000000in}{-0.048611in}}{\pgfqpoint{0.000000in}{0.000000in}}{%
\pgfpathmoveto{\pgfqpoint{0.000000in}{0.000000in}}%
\pgfpathlineto{\pgfqpoint{0.000000in}{-0.048611in}}%
\pgfusepath{stroke,fill}%
}%
\begin{pgfscope}%
\pgfsys@transformshift{8.188400in}{0.589870in}%
\pgfsys@useobject{currentmarker}{}%
\end{pgfscope}%
\end{pgfscope}%
\begin{pgfscope}%
\pgftext[x=8.188400in,y=0.492648in,,top]{\rmfamily\fontsize{10.000000}{12.000000}\selectfont \(\displaystyle 0.0\)}%
\end{pgfscope}%
\begin{pgfscope}%
\pgfsetbuttcap%
\pgfsetroundjoin%
\definecolor{currentfill}{rgb}{0.000000,0.000000,0.000000}%
\pgfsetfillcolor{currentfill}%
\pgfsetlinewidth{0.803000pt}%
\definecolor{currentstroke}{rgb}{0.000000,0.000000,0.000000}%
\pgfsetstrokecolor{currentstroke}%
\pgfsetdash{}{0pt}%
\pgfsys@defobject{currentmarker}{\pgfqpoint{0.000000in}{-0.048611in}}{\pgfqpoint{0.000000in}{0.000000in}}{%
\pgfpathmoveto{\pgfqpoint{0.000000in}{0.000000in}}%
\pgfpathlineto{\pgfqpoint{0.000000in}{-0.048611in}}%
\pgfusepath{stroke,fill}%
}%
\begin{pgfscope}%
\pgfsys@transformshift{9.063694in}{0.589870in}%
\pgfsys@useobject{currentmarker}{}%
\end{pgfscope}%
\end{pgfscope}%
\begin{pgfscope}%
\pgftext[x=9.063694in,y=0.492648in,,top]{\rmfamily\fontsize{10.000000}{12.000000}\selectfont \(\displaystyle 0.5\)}%
\end{pgfscope}%
\begin{pgfscope}%
\pgfsetbuttcap%
\pgfsetroundjoin%
\definecolor{currentfill}{rgb}{0.000000,0.000000,0.000000}%
\pgfsetfillcolor{currentfill}%
\pgfsetlinewidth{0.803000pt}%
\definecolor{currentstroke}{rgb}{0.000000,0.000000,0.000000}%
\pgfsetstrokecolor{currentstroke}%
\pgfsetdash{}{0pt}%
\pgfsys@defobject{currentmarker}{\pgfqpoint{0.000000in}{-0.048611in}}{\pgfqpoint{0.000000in}{0.000000in}}{%
\pgfpathmoveto{\pgfqpoint{0.000000in}{0.000000in}}%
\pgfpathlineto{\pgfqpoint{0.000000in}{-0.048611in}}%
\pgfusepath{stroke,fill}%
}%
\begin{pgfscope}%
\pgfsys@transformshift{9.938988in}{0.589870in}%
\pgfsys@useobject{currentmarker}{}%
\end{pgfscope}%
\end{pgfscope}%
\begin{pgfscope}%
\pgftext[x=9.938988in,y=0.492648in,,top]{\rmfamily\fontsize{10.000000}{12.000000}\selectfont \(\displaystyle 1.0\)}%
\end{pgfscope}%
\begin{pgfscope}%
\pgftext[x=7.750753in,y=0.302680in,,top]{\rmfamily\fontsize{10.000000}{12.000000}\selectfont x}%
\end{pgfscope}%
\begin{pgfscope}%
\pgfsetbuttcap%
\pgfsetroundjoin%
\definecolor{currentfill}{rgb}{0.000000,0.000000,0.000000}%
\pgfsetfillcolor{currentfill}%
\pgfsetlinewidth{0.803000pt}%
\definecolor{currentstroke}{rgb}{0.000000,0.000000,0.000000}%
\pgfsetstrokecolor{currentstroke}%
\pgfsetdash{}{0pt}%
\pgfsys@defobject{currentmarker}{\pgfqpoint{-0.048611in}{0.000000in}}{\pgfqpoint{0.000000in}{0.000000in}}{%
\pgfpathmoveto{\pgfqpoint{0.000000in}{0.000000in}}%
\pgfpathlineto{\pgfqpoint{-0.048611in}{0.000000in}}%
\pgfusepath{stroke,fill}%
}%
\begin{pgfscope}%
\pgfsys@transformshift{5.562518in}{0.954607in}%
\pgfsys@useobject{currentmarker}{}%
\end{pgfscope}%
\end{pgfscope}%
\begin{pgfscope}%
\pgftext[x=5.395851in,y=0.901846in,left,base]{\rmfamily\fontsize{10.000000}{12.000000}\selectfont \(\displaystyle 0\)}%
\end{pgfscope}%
\begin{pgfscope}%
\pgfsetbuttcap%
\pgfsetroundjoin%
\definecolor{currentfill}{rgb}{0.000000,0.000000,0.000000}%
\pgfsetfillcolor{currentfill}%
\pgfsetlinewidth{0.803000pt}%
\definecolor{currentstroke}{rgb}{0.000000,0.000000,0.000000}%
\pgfsetstrokecolor{currentstroke}%
\pgfsetdash{}{0pt}%
\pgfsys@defobject{currentmarker}{\pgfqpoint{-0.048611in}{0.000000in}}{\pgfqpoint{0.000000in}{0.000000in}}{%
\pgfpathmoveto{\pgfqpoint{0.000000in}{0.000000in}}%
\pgfpathlineto{\pgfqpoint{-0.048611in}{0.000000in}}%
\pgfusepath{stroke,fill}%
}%
\begin{pgfscope}%
\pgfsys@transformshift{5.562518in}{1.359870in}%
\pgfsys@useobject{currentmarker}{}%
\end{pgfscope}%
\end{pgfscope}%
\begin{pgfscope}%
\pgftext[x=5.395851in,y=1.307109in,left,base]{\rmfamily\fontsize{10.000000}{12.000000}\selectfont \(\displaystyle 2\)}%
\end{pgfscope}%
\begin{pgfscope}%
\pgfpathrectangle{\pgfqpoint{5.562518in}{0.589870in}}{\pgfqpoint{4.376471in}{0.972632in}}%
\pgfusepath{clip}%
\pgfsetrectcap%
\pgfsetroundjoin%
\pgfsetlinewidth{1.505625pt}%
\definecolor{currentstroke}{rgb}{0.121569,0.466667,0.705882}%
\pgfsetstrokecolor{currentstroke}%
\pgfsetdash{}{0pt}%
\pgfpathmoveto{\pgfqpoint{6.437812in}{0.955121in}}%
\pgfpathlineto{\pgfqpoint{6.543375in}{0.954592in}}%
\pgfpathlineto{\pgfqpoint{9.851019in}{0.954394in}}%
\pgfpathlineto{\pgfqpoint{9.903801in}{0.952296in}}%
\pgfpathlineto{\pgfqpoint{9.938988in}{0.948689in}}%
\pgfpathlineto{\pgfqpoint{9.938988in}{0.948689in}}%
\pgfusepath{stroke}%
\end{pgfscope}%
\begin{pgfscope}%
\pgfsetrectcap%
\pgfsetmiterjoin%
\pgfsetlinewidth{0.803000pt}%
\definecolor{currentstroke}{rgb}{0.000000,0.000000,0.000000}%
\pgfsetstrokecolor{currentstroke}%
\pgfsetdash{}{0pt}%
\pgfpathmoveto{\pgfqpoint{5.562518in}{0.589870in}}%
\pgfpathlineto{\pgfqpoint{5.562518in}{1.562502in}}%
\pgfusepath{stroke}%
\end{pgfscope}%
\begin{pgfscope}%
\pgfsetrectcap%
\pgfsetmiterjoin%
\pgfsetlinewidth{0.803000pt}%
\definecolor{currentstroke}{rgb}{0.000000,0.000000,0.000000}%
\pgfsetstrokecolor{currentstroke}%
\pgfsetdash{}{0pt}%
\pgfpathmoveto{\pgfqpoint{9.938988in}{0.589870in}}%
\pgfpathlineto{\pgfqpoint{9.938988in}{1.562502in}}%
\pgfusepath{stroke}%
\end{pgfscope}%
\begin{pgfscope}%
\pgfsetrectcap%
\pgfsetmiterjoin%
\pgfsetlinewidth{0.803000pt}%
\definecolor{currentstroke}{rgb}{0.000000,0.000000,0.000000}%
\pgfsetstrokecolor{currentstroke}%
\pgfsetdash{}{0pt}%
\pgfpathmoveto{\pgfqpoint{5.562518in}{0.589870in}}%
\pgfpathlineto{\pgfqpoint{9.938988in}{0.589870in}}%
\pgfusepath{stroke}%
\end{pgfscope}%
\begin{pgfscope}%
\pgfsetrectcap%
\pgfsetmiterjoin%
\pgfsetlinewidth{0.803000pt}%
\definecolor{currentstroke}{rgb}{0.000000,0.000000,0.000000}%
\pgfsetstrokecolor{currentstroke}%
\pgfsetdash{}{0pt}%
\pgfpathmoveto{\pgfqpoint{5.562518in}{1.562502in}}%
\pgfpathlineto{\pgfqpoint{9.938988in}{1.562502in}}%
\pgfusepath{stroke}%
\end{pgfscope}%
\begin{pgfscope}%
\pgfsetbuttcap%
\pgfsetmiterjoin%
\definecolor{currentfill}{rgb}{1.000000,1.000000,1.000000}%
\pgfsetfillcolor{currentfill}%
\pgfsetfillopacity{0.800000}%
\pgfsetlinewidth{1.003750pt}%
\definecolor{currentstroke}{rgb}{0.800000,0.800000,0.800000}%
\pgfsetstrokecolor{currentstroke}%
\pgfsetstrokeopacity{0.800000}%
\pgfsetdash{}{0pt}%
\pgfpathmoveto{\pgfqpoint{5.659740in}{0.659315in}}%
\pgfpathlineto{\pgfqpoint{6.443065in}{0.659315in}}%
\pgfpathquadraticcurveto{\pgfqpoint{6.470843in}{0.659315in}}{\pgfqpoint{6.470843in}{0.687092in}}%
\pgfpathlineto{\pgfqpoint{6.470843in}{1.285779in}}%
\pgfpathquadraticcurveto{\pgfqpoint{6.470843in}{1.313557in}}{\pgfqpoint{6.443065in}{1.313557in}}%
\pgfpathlineto{\pgfqpoint{5.659740in}{1.313557in}}%
\pgfpathquadraticcurveto{\pgfqpoint{5.631962in}{1.313557in}}{\pgfqpoint{5.631962in}{1.285779in}}%
\pgfpathlineto{\pgfqpoint{5.631962in}{0.687092in}}%
\pgfpathquadraticcurveto{\pgfqpoint{5.631962in}{0.659315in}}{\pgfqpoint{5.659740in}{0.659315in}}%
\pgfpathclose%
\pgfusepath{stroke,fill}%
\end{pgfscope}%
\begin{pgfscope}%
\pgfsetrectcap%
\pgfsetroundjoin%
\pgfsetlinewidth{1.505625pt}%
\definecolor{currentstroke}{rgb}{0.121569,0.466667,0.705882}%
\pgfsetstrokecolor{currentstroke}%
\pgfsetdash{}{0pt}%
\pgfpathmoveto{\pgfqpoint{5.687518in}{1.200086in}}%
\pgfpathlineto{\pgfqpoint{5.965295in}{1.200086in}}%
\pgfusepath{stroke}%
\end{pgfscope}%
\begin{pgfscope}%
\pgftext[x=6.076407in,y=1.151474in,left,base]{\rmfamily\fontsize{10.000000}{12.000000}\selectfont \(\displaystyle \widetilde{\Phi}^* \theta^{\parallel}\)}%
\end{pgfscope}%
\begin{pgfscope}%
\pgfsetbuttcap%
\pgfsetroundjoin%
\definecolor{currentfill}{rgb}{1.000000,0.000000,0.000000}%
\pgfsetfillcolor{currentfill}%
\pgfsetlinewidth{2.007500pt}%
\definecolor{currentstroke}{rgb}{1.000000,0.000000,0.000000}%
\pgfsetstrokecolor{currentstroke}%
\pgfsetdash{}{0pt}%
\pgfpathmoveto{\pgfqpoint{5.784740in}{0.984076in}}%
\pgfpathlineto{\pgfqpoint{5.868073in}{0.984076in}}%
\pgfpathmoveto{\pgfqpoint{5.826407in}{0.942409in}}%
\pgfpathlineto{\pgfqpoint{5.826407in}{1.025742in}}%
\pgfusepath{stroke,fill}%
\end{pgfscope}%
\begin{pgfscope}%
\pgftext[x=6.076407in,y=0.947617in,left,base]{\rmfamily\fontsize{10.000000}{12.000000}\selectfont train}%
\end{pgfscope}%
\begin{pgfscope}%
\pgfsetbuttcap%
\pgfsetroundjoin%
\definecolor{currentfill}{rgb}{0.000000,0.000000,0.000000}%
\pgfsetfillcolor{currentfill}%
\pgfsetlinewidth{1.003750pt}%
\definecolor{currentstroke}{rgb}{0.000000,0.000000,0.000000}%
\pgfsetstrokecolor{currentstroke}%
\pgfsetdash{}{0pt}%
\pgfsys@defobject{currentmarker}{\pgfqpoint{-0.020833in}{-0.020833in}}{\pgfqpoint{0.020833in}{0.020833in}}{%
\pgfpathmoveto{\pgfqpoint{0.000000in}{-0.020833in}}%
\pgfpathcurveto{\pgfqpoint{0.005525in}{-0.020833in}}{\pgfqpoint{0.010825in}{-0.018638in}}{\pgfqpoint{0.014731in}{-0.014731in}}%
\pgfpathcurveto{\pgfqpoint{0.018638in}{-0.010825in}}{\pgfqpoint{0.020833in}{-0.005525in}}{\pgfqpoint{0.020833in}{0.000000in}}%
\pgfpathcurveto{\pgfqpoint{0.020833in}{0.005525in}}{\pgfqpoint{0.018638in}{0.010825in}}{\pgfqpoint{0.014731in}{0.014731in}}%
\pgfpathcurveto{\pgfqpoint{0.010825in}{0.018638in}}{\pgfqpoint{0.005525in}{0.020833in}}{\pgfqpoint{0.000000in}{0.020833in}}%
\pgfpathcurveto{\pgfqpoint{-0.005525in}{0.020833in}}{\pgfqpoint{-0.010825in}{0.018638in}}{\pgfqpoint{-0.014731in}{0.014731in}}%
\pgfpathcurveto{\pgfqpoint{-0.018638in}{0.010825in}}{\pgfqpoint{-0.020833in}{0.005525in}}{\pgfqpoint{-0.020833in}{0.000000in}}%
\pgfpathcurveto{\pgfqpoint{-0.020833in}{-0.005525in}}{\pgfqpoint{-0.018638in}{-0.010825in}}{\pgfqpoint{-0.014731in}{-0.014731in}}%
\pgfpathcurveto{\pgfqpoint{-0.010825in}{-0.018638in}}{\pgfqpoint{-0.005525in}{-0.020833in}}{\pgfqpoint{0.000000in}{-0.020833in}}%
\pgfpathclose%
\pgfusepath{stroke,fill}%
}%
\begin{pgfscope}%
\pgfsys@transformshift{5.826407in}{0.780218in}%
\pgfsys@useobject{currentmarker}{}%
\end{pgfscope}%
\end{pgfscope}%
\begin{pgfscope}%
\pgftext[x=6.076407in,y=0.743760in,left,base]{\rmfamily\fontsize{10.000000}{12.000000}\selectfont test}%
\end{pgfscope}%
\begin{pgfscope}%
\pgfsetbuttcap%
\pgfsetmiterjoin%
\definecolor{currentfill}{rgb}{1.000000,1.000000,1.000000}%
\pgfsetfillcolor{currentfill}%
\pgfsetlinewidth{0.000000pt}%
\definecolor{currentstroke}{rgb}{0.000000,0.000000,0.000000}%
\pgfsetstrokecolor{currentstroke}%
\pgfsetstrokeopacity{0.000000}%
\pgfsetdash{}{0pt}%
\pgfpathmoveto{\pgfqpoint{10.668400in}{0.589870in}}%
\pgfpathlineto{\pgfqpoint{12.856635in}{0.589870in}}%
\pgfpathlineto{\pgfqpoint{12.856635in}{1.562502in}}%
\pgfpathlineto{\pgfqpoint{10.668400in}{1.562502in}}%
\pgfpathclose%
\pgfusepath{fill}%
\end{pgfscope}%
\begin{pgfscope}%
\pgfpathrectangle{\pgfqpoint{10.668400in}{0.589870in}}{\pgfqpoint{2.188235in}{0.972632in}}%
\pgfusepath{clip}%
\pgfsetbuttcap%
\pgfsetmiterjoin%
\definecolor{currentfill}{rgb}{0.121569,0.466667,0.705882}%
\pgfsetfillcolor{currentfill}%
\pgfsetlinewidth{0.000000pt}%
\definecolor{currentstroke}{rgb}{0.000000,0.000000,0.000000}%
\pgfsetstrokecolor{currentstroke}%
\pgfsetstrokeopacity{0.000000}%
\pgfsetdash{}{0pt}%
\pgfpathmoveto{\pgfqpoint{-319.224843in}{0.634081in}}%
\pgfpathlineto{\pgfqpoint{12.399654in}{0.634081in}}%
\pgfpathlineto{\pgfqpoint{12.399654in}{0.641169in}}%
\pgfpathlineto{\pgfqpoint{-319.224843in}{0.641169in}}%
\pgfpathclose%
\pgfusepath{fill}%
\end{pgfscope}%
\begin{pgfscope}%
\pgfpathrectangle{\pgfqpoint{10.668400in}{0.589870in}}{\pgfqpoint{2.188235in}{0.972632in}}%
\pgfusepath{clip}%
\pgfsetbuttcap%
\pgfsetmiterjoin%
\definecolor{currentfill}{rgb}{0.121569,0.466667,0.705882}%
\pgfsetfillcolor{currentfill}%
\pgfsetlinewidth{0.000000pt}%
\definecolor{currentstroke}{rgb}{0.000000,0.000000,0.000000}%
\pgfsetstrokecolor{currentstroke}%
\pgfsetstrokeopacity{0.000000}%
\pgfsetdash{}{0pt}%
\pgfpathmoveto{\pgfqpoint{-319.224843in}{0.642941in}}%
\pgfpathlineto{\pgfqpoint{12.273848in}{0.642941in}}%
\pgfpathlineto{\pgfqpoint{12.273848in}{0.650028in}}%
\pgfpathlineto{\pgfqpoint{-319.224843in}{0.650028in}}%
\pgfpathclose%
\pgfusepath{fill}%
\end{pgfscope}%
\begin{pgfscope}%
\pgfpathrectangle{\pgfqpoint{10.668400in}{0.589870in}}{\pgfqpoint{2.188235in}{0.972632in}}%
\pgfusepath{clip}%
\pgfsetbuttcap%
\pgfsetmiterjoin%
\definecolor{currentfill}{rgb}{0.121569,0.466667,0.705882}%
\pgfsetfillcolor{currentfill}%
\pgfsetlinewidth{0.000000pt}%
\definecolor{currentstroke}{rgb}{0.000000,0.000000,0.000000}%
\pgfsetstrokecolor{currentstroke}%
\pgfsetstrokeopacity{0.000000}%
\pgfsetdash{}{0pt}%
\pgfpathmoveto{\pgfqpoint{-319.224843in}{0.651800in}}%
\pgfpathlineto{\pgfqpoint{12.391234in}{0.651800in}}%
\pgfpathlineto{\pgfqpoint{12.391234in}{0.658888in}}%
\pgfpathlineto{\pgfqpoint{-319.224843in}{0.658888in}}%
\pgfpathclose%
\pgfusepath{fill}%
\end{pgfscope}%
\begin{pgfscope}%
\pgfpathrectangle{\pgfqpoint{10.668400in}{0.589870in}}{\pgfqpoint{2.188235in}{0.972632in}}%
\pgfusepath{clip}%
\pgfsetbuttcap%
\pgfsetmiterjoin%
\definecolor{currentfill}{rgb}{0.121569,0.466667,0.705882}%
\pgfsetfillcolor{currentfill}%
\pgfsetlinewidth{0.000000pt}%
\definecolor{currentstroke}{rgb}{0.000000,0.000000,0.000000}%
\pgfsetstrokecolor{currentstroke}%
\pgfsetstrokeopacity{0.000000}%
\pgfsetdash{}{0pt}%
\pgfpathmoveto{\pgfqpoint{-319.224843in}{0.660660in}}%
\pgfpathlineto{\pgfqpoint{12.113619in}{0.660660in}}%
\pgfpathlineto{\pgfqpoint{12.113619in}{0.667748in}}%
\pgfpathlineto{\pgfqpoint{-319.224843in}{0.667748in}}%
\pgfpathclose%
\pgfusepath{fill}%
\end{pgfscope}%
\begin{pgfscope}%
\pgfpathrectangle{\pgfqpoint{10.668400in}{0.589870in}}{\pgfqpoint{2.188235in}{0.972632in}}%
\pgfusepath{clip}%
\pgfsetbuttcap%
\pgfsetmiterjoin%
\definecolor{currentfill}{rgb}{0.121569,0.466667,0.705882}%
\pgfsetfillcolor{currentfill}%
\pgfsetlinewidth{0.000000pt}%
\definecolor{currentstroke}{rgb}{0.000000,0.000000,0.000000}%
\pgfsetstrokecolor{currentstroke}%
\pgfsetstrokeopacity{0.000000}%
\pgfsetdash{}{0pt}%
\pgfpathmoveto{\pgfqpoint{-319.224843in}{0.669520in}}%
\pgfpathlineto{\pgfqpoint{12.371217in}{0.669520in}}%
\pgfpathlineto{\pgfqpoint{12.371217in}{0.676608in}}%
\pgfpathlineto{\pgfqpoint{-319.224843in}{0.676608in}}%
\pgfpathclose%
\pgfusepath{fill}%
\end{pgfscope}%
\begin{pgfscope}%
\pgfpathrectangle{\pgfqpoint{10.668400in}{0.589870in}}{\pgfqpoint{2.188235in}{0.972632in}}%
\pgfusepath{clip}%
\pgfsetbuttcap%
\pgfsetmiterjoin%
\definecolor{currentfill}{rgb}{0.121569,0.466667,0.705882}%
\pgfsetfillcolor{currentfill}%
\pgfsetlinewidth{0.000000pt}%
\definecolor{currentstroke}{rgb}{0.000000,0.000000,0.000000}%
\pgfsetstrokecolor{currentstroke}%
\pgfsetstrokeopacity{0.000000}%
\pgfsetdash{}{0pt}%
\pgfpathmoveto{\pgfqpoint{-319.224843in}{0.678380in}}%
\pgfpathlineto{\pgfqpoint{12.251396in}{0.678380in}}%
\pgfpathlineto{\pgfqpoint{12.251396in}{0.685468in}}%
\pgfpathlineto{\pgfqpoint{-319.224843in}{0.685468in}}%
\pgfpathclose%
\pgfusepath{fill}%
\end{pgfscope}%
\begin{pgfscope}%
\pgfpathrectangle{\pgfqpoint{10.668400in}{0.589870in}}{\pgfqpoint{2.188235in}{0.972632in}}%
\pgfusepath{clip}%
\pgfsetbuttcap%
\pgfsetmiterjoin%
\definecolor{currentfill}{rgb}{0.121569,0.466667,0.705882}%
\pgfsetfillcolor{currentfill}%
\pgfsetlinewidth{0.000000pt}%
\definecolor{currentstroke}{rgb}{0.000000,0.000000,0.000000}%
\pgfsetstrokecolor{currentstroke}%
\pgfsetstrokeopacity{0.000000}%
\pgfsetdash{}{0pt}%
\pgfpathmoveto{\pgfqpoint{-319.224843in}{0.687240in}}%
\pgfpathlineto{\pgfqpoint{11.003690in}{0.687240in}}%
\pgfpathlineto{\pgfqpoint{11.003690in}{0.694328in}}%
\pgfpathlineto{\pgfqpoint{-319.224843in}{0.694328in}}%
\pgfpathclose%
\pgfusepath{fill}%
\end{pgfscope}%
\begin{pgfscope}%
\pgfpathrectangle{\pgfqpoint{10.668400in}{0.589870in}}{\pgfqpoint{2.188235in}{0.972632in}}%
\pgfusepath{clip}%
\pgfsetbuttcap%
\pgfsetmiterjoin%
\definecolor{currentfill}{rgb}{0.121569,0.466667,0.705882}%
\pgfsetfillcolor{currentfill}%
\pgfsetlinewidth{0.000000pt}%
\definecolor{currentstroke}{rgb}{0.000000,0.000000,0.000000}%
\pgfsetstrokecolor{currentstroke}%
\pgfsetstrokeopacity{0.000000}%
\pgfsetdash{}{0pt}%
\pgfpathmoveto{\pgfqpoint{-319.224843in}{0.696100in}}%
\pgfpathlineto{\pgfqpoint{12.377368in}{0.696100in}}%
\pgfpathlineto{\pgfqpoint{12.377368in}{0.703187in}}%
\pgfpathlineto{\pgfqpoint{-319.224843in}{0.703187in}}%
\pgfpathclose%
\pgfusepath{fill}%
\end{pgfscope}%
\begin{pgfscope}%
\pgfpathrectangle{\pgfqpoint{10.668400in}{0.589870in}}{\pgfqpoint{2.188235in}{0.972632in}}%
\pgfusepath{clip}%
\pgfsetbuttcap%
\pgfsetmiterjoin%
\definecolor{currentfill}{rgb}{0.121569,0.466667,0.705882}%
\pgfsetfillcolor{currentfill}%
\pgfsetlinewidth{0.000000pt}%
\definecolor{currentstroke}{rgb}{0.000000,0.000000,0.000000}%
\pgfsetstrokecolor{currentstroke}%
\pgfsetstrokeopacity{0.000000}%
\pgfsetdash{}{0pt}%
\pgfpathmoveto{\pgfqpoint{-319.224843in}{0.704959in}}%
\pgfpathlineto{\pgfqpoint{12.159786in}{0.704959in}}%
\pgfpathlineto{\pgfqpoint{12.159786in}{0.712047in}}%
\pgfpathlineto{\pgfqpoint{-319.224843in}{0.712047in}}%
\pgfpathclose%
\pgfusepath{fill}%
\end{pgfscope}%
\begin{pgfscope}%
\pgfpathrectangle{\pgfqpoint{10.668400in}{0.589870in}}{\pgfqpoint{2.188235in}{0.972632in}}%
\pgfusepath{clip}%
\pgfsetbuttcap%
\pgfsetmiterjoin%
\definecolor{currentfill}{rgb}{0.121569,0.466667,0.705882}%
\pgfsetfillcolor{currentfill}%
\pgfsetlinewidth{0.000000pt}%
\definecolor{currentstroke}{rgb}{0.000000,0.000000,0.000000}%
\pgfsetstrokecolor{currentstroke}%
\pgfsetstrokeopacity{0.000000}%
\pgfsetdash{}{0pt}%
\pgfpathmoveto{\pgfqpoint{-319.224843in}{0.713819in}}%
\pgfpathlineto{\pgfqpoint{12.210821in}{0.713819in}}%
\pgfpathlineto{\pgfqpoint{12.210821in}{0.720907in}}%
\pgfpathlineto{\pgfqpoint{-319.224843in}{0.720907in}}%
\pgfpathclose%
\pgfusepath{fill}%
\end{pgfscope}%
\begin{pgfscope}%
\pgfpathrectangle{\pgfqpoint{10.668400in}{0.589870in}}{\pgfqpoint{2.188235in}{0.972632in}}%
\pgfusepath{clip}%
\pgfsetbuttcap%
\pgfsetmiterjoin%
\definecolor{currentfill}{rgb}{0.121569,0.466667,0.705882}%
\pgfsetfillcolor{currentfill}%
\pgfsetlinewidth{0.000000pt}%
\definecolor{currentstroke}{rgb}{0.000000,0.000000,0.000000}%
\pgfsetstrokecolor{currentstroke}%
\pgfsetstrokeopacity{0.000000}%
\pgfsetdash{}{0pt}%
\pgfpathmoveto{\pgfqpoint{-319.224843in}{0.722679in}}%
\pgfpathlineto{\pgfqpoint{12.648296in}{0.722679in}}%
\pgfpathlineto{\pgfqpoint{12.648296in}{0.729767in}}%
\pgfpathlineto{\pgfqpoint{-319.224843in}{0.729767in}}%
\pgfpathclose%
\pgfusepath{fill}%
\end{pgfscope}%
\begin{pgfscope}%
\pgfpathrectangle{\pgfqpoint{10.668400in}{0.589870in}}{\pgfqpoint{2.188235in}{0.972632in}}%
\pgfusepath{clip}%
\pgfsetbuttcap%
\pgfsetmiterjoin%
\definecolor{currentfill}{rgb}{0.121569,0.466667,0.705882}%
\pgfsetfillcolor{currentfill}%
\pgfsetlinewidth{0.000000pt}%
\definecolor{currentstroke}{rgb}{0.000000,0.000000,0.000000}%
\pgfsetstrokecolor{currentstroke}%
\pgfsetstrokeopacity{0.000000}%
\pgfsetdash{}{0pt}%
\pgfpathmoveto{\pgfqpoint{-319.224843in}{0.731539in}}%
\pgfpathlineto{\pgfqpoint{12.455109in}{0.731539in}}%
\pgfpathlineto{\pgfqpoint{12.455109in}{0.738627in}}%
\pgfpathlineto{\pgfqpoint{-319.224843in}{0.738627in}}%
\pgfpathclose%
\pgfusepath{fill}%
\end{pgfscope}%
\begin{pgfscope}%
\pgfpathrectangle{\pgfqpoint{10.668400in}{0.589870in}}{\pgfqpoint{2.188235in}{0.972632in}}%
\pgfusepath{clip}%
\pgfsetbuttcap%
\pgfsetmiterjoin%
\definecolor{currentfill}{rgb}{0.121569,0.466667,0.705882}%
\pgfsetfillcolor{currentfill}%
\pgfsetlinewidth{0.000000pt}%
\definecolor{currentstroke}{rgb}{0.000000,0.000000,0.000000}%
\pgfsetstrokecolor{currentstroke}%
\pgfsetstrokeopacity{0.000000}%
\pgfsetdash{}{0pt}%
\pgfpathmoveto{\pgfqpoint{-319.224843in}{0.740399in}}%
\pgfpathlineto{\pgfqpoint{12.277417in}{0.740399in}}%
\pgfpathlineto{\pgfqpoint{12.277417in}{0.747487in}}%
\pgfpathlineto{\pgfqpoint{-319.224843in}{0.747487in}}%
\pgfpathclose%
\pgfusepath{fill}%
\end{pgfscope}%
\begin{pgfscope}%
\pgfpathrectangle{\pgfqpoint{10.668400in}{0.589870in}}{\pgfqpoint{2.188235in}{0.972632in}}%
\pgfusepath{clip}%
\pgfsetbuttcap%
\pgfsetmiterjoin%
\definecolor{currentfill}{rgb}{0.121569,0.466667,0.705882}%
\pgfsetfillcolor{currentfill}%
\pgfsetlinewidth{0.000000pt}%
\definecolor{currentstroke}{rgb}{0.000000,0.000000,0.000000}%
\pgfsetstrokecolor{currentstroke}%
\pgfsetstrokeopacity{0.000000}%
\pgfsetdash{}{0pt}%
\pgfpathmoveto{\pgfqpoint{-319.224843in}{0.749259in}}%
\pgfpathlineto{\pgfqpoint{12.167107in}{0.749259in}}%
\pgfpathlineto{\pgfqpoint{12.167107in}{0.756346in}}%
\pgfpathlineto{\pgfqpoint{-319.224843in}{0.756346in}}%
\pgfpathclose%
\pgfusepath{fill}%
\end{pgfscope}%
\begin{pgfscope}%
\pgfpathrectangle{\pgfqpoint{10.668400in}{0.589870in}}{\pgfqpoint{2.188235in}{0.972632in}}%
\pgfusepath{clip}%
\pgfsetbuttcap%
\pgfsetmiterjoin%
\definecolor{currentfill}{rgb}{0.121569,0.466667,0.705882}%
\pgfsetfillcolor{currentfill}%
\pgfsetlinewidth{0.000000pt}%
\definecolor{currentstroke}{rgb}{0.000000,0.000000,0.000000}%
\pgfsetstrokecolor{currentstroke}%
\pgfsetstrokeopacity{0.000000}%
\pgfsetdash{}{0pt}%
\pgfpathmoveto{\pgfqpoint{-319.224843in}{0.758118in}}%
\pgfpathlineto{\pgfqpoint{12.294428in}{0.758118in}}%
\pgfpathlineto{\pgfqpoint{12.294428in}{0.765206in}}%
\pgfpathlineto{\pgfqpoint{-319.224843in}{0.765206in}}%
\pgfpathclose%
\pgfusepath{fill}%
\end{pgfscope}%
\begin{pgfscope}%
\pgfpathrectangle{\pgfqpoint{10.668400in}{0.589870in}}{\pgfqpoint{2.188235in}{0.972632in}}%
\pgfusepath{clip}%
\pgfsetbuttcap%
\pgfsetmiterjoin%
\definecolor{currentfill}{rgb}{0.121569,0.466667,0.705882}%
\pgfsetfillcolor{currentfill}%
\pgfsetlinewidth{0.000000pt}%
\definecolor{currentstroke}{rgb}{0.000000,0.000000,0.000000}%
\pgfsetstrokecolor{currentstroke}%
\pgfsetstrokeopacity{0.000000}%
\pgfsetdash{}{0pt}%
\pgfpathmoveto{\pgfqpoint{-319.224843in}{0.766978in}}%
\pgfpathlineto{\pgfqpoint{12.428359in}{0.766978in}}%
\pgfpathlineto{\pgfqpoint{12.428359in}{0.774066in}}%
\pgfpathlineto{\pgfqpoint{-319.224843in}{0.774066in}}%
\pgfpathclose%
\pgfusepath{fill}%
\end{pgfscope}%
\begin{pgfscope}%
\pgfpathrectangle{\pgfqpoint{10.668400in}{0.589870in}}{\pgfqpoint{2.188235in}{0.972632in}}%
\pgfusepath{clip}%
\pgfsetbuttcap%
\pgfsetmiterjoin%
\definecolor{currentfill}{rgb}{0.121569,0.466667,0.705882}%
\pgfsetfillcolor{currentfill}%
\pgfsetlinewidth{0.000000pt}%
\definecolor{currentstroke}{rgb}{0.000000,0.000000,0.000000}%
\pgfsetstrokecolor{currentstroke}%
\pgfsetstrokeopacity{0.000000}%
\pgfsetdash{}{0pt}%
\pgfpathmoveto{\pgfqpoint{-319.224843in}{0.775838in}}%
\pgfpathlineto{\pgfqpoint{12.541904in}{0.775838in}}%
\pgfpathlineto{\pgfqpoint{12.541904in}{0.782926in}}%
\pgfpathlineto{\pgfqpoint{-319.224843in}{0.782926in}}%
\pgfpathclose%
\pgfusepath{fill}%
\end{pgfscope}%
\begin{pgfscope}%
\pgfpathrectangle{\pgfqpoint{10.668400in}{0.589870in}}{\pgfqpoint{2.188235in}{0.972632in}}%
\pgfusepath{clip}%
\pgfsetbuttcap%
\pgfsetmiterjoin%
\definecolor{currentfill}{rgb}{0.121569,0.466667,0.705882}%
\pgfsetfillcolor{currentfill}%
\pgfsetlinewidth{0.000000pt}%
\definecolor{currentstroke}{rgb}{0.000000,0.000000,0.000000}%
\pgfsetstrokecolor{currentstroke}%
\pgfsetstrokeopacity{0.000000}%
\pgfsetdash{}{0pt}%
\pgfpathmoveto{\pgfqpoint{-319.224843in}{0.784698in}}%
\pgfpathlineto{\pgfqpoint{12.198220in}{0.784698in}}%
\pgfpathlineto{\pgfqpoint{12.198220in}{0.791786in}}%
\pgfpathlineto{\pgfqpoint{-319.224843in}{0.791786in}}%
\pgfpathclose%
\pgfusepath{fill}%
\end{pgfscope}%
\begin{pgfscope}%
\pgfpathrectangle{\pgfqpoint{10.668400in}{0.589870in}}{\pgfqpoint{2.188235in}{0.972632in}}%
\pgfusepath{clip}%
\pgfsetbuttcap%
\pgfsetmiterjoin%
\definecolor{currentfill}{rgb}{0.121569,0.466667,0.705882}%
\pgfsetfillcolor{currentfill}%
\pgfsetlinewidth{0.000000pt}%
\definecolor{currentstroke}{rgb}{0.000000,0.000000,0.000000}%
\pgfsetstrokecolor{currentstroke}%
\pgfsetstrokeopacity{0.000000}%
\pgfsetdash{}{0pt}%
\pgfpathmoveto{\pgfqpoint{-319.224843in}{0.793558in}}%
\pgfpathlineto{\pgfqpoint{12.107531in}{0.793558in}}%
\pgfpathlineto{\pgfqpoint{12.107531in}{0.800646in}}%
\pgfpathlineto{\pgfqpoint{-319.224843in}{0.800646in}}%
\pgfpathclose%
\pgfusepath{fill}%
\end{pgfscope}%
\begin{pgfscope}%
\pgfpathrectangle{\pgfqpoint{10.668400in}{0.589870in}}{\pgfqpoint{2.188235in}{0.972632in}}%
\pgfusepath{clip}%
\pgfsetbuttcap%
\pgfsetmiterjoin%
\definecolor{currentfill}{rgb}{0.121569,0.466667,0.705882}%
\pgfsetfillcolor{currentfill}%
\pgfsetlinewidth{0.000000pt}%
\definecolor{currentstroke}{rgb}{0.000000,0.000000,0.000000}%
\pgfsetstrokecolor{currentstroke}%
\pgfsetstrokeopacity{0.000000}%
\pgfsetdash{}{0pt}%
\pgfpathmoveto{\pgfqpoint{-319.224843in}{0.802417in}}%
\pgfpathlineto{\pgfqpoint{12.316070in}{0.802417in}}%
\pgfpathlineto{\pgfqpoint{12.316070in}{0.809505in}}%
\pgfpathlineto{\pgfqpoint{-319.224843in}{0.809505in}}%
\pgfpathclose%
\pgfusepath{fill}%
\end{pgfscope}%
\begin{pgfscope}%
\pgfpathrectangle{\pgfqpoint{10.668400in}{0.589870in}}{\pgfqpoint{2.188235in}{0.972632in}}%
\pgfusepath{clip}%
\pgfsetbuttcap%
\pgfsetmiterjoin%
\definecolor{currentfill}{rgb}{0.121569,0.466667,0.705882}%
\pgfsetfillcolor{currentfill}%
\pgfsetlinewidth{0.000000pt}%
\definecolor{currentstroke}{rgb}{0.000000,0.000000,0.000000}%
\pgfsetstrokecolor{currentstroke}%
\pgfsetstrokeopacity{0.000000}%
\pgfsetdash{}{0pt}%
\pgfpathmoveto{\pgfqpoint{-319.224843in}{0.811277in}}%
\pgfpathlineto{\pgfqpoint{12.266824in}{0.811277in}}%
\pgfpathlineto{\pgfqpoint{12.266824in}{0.818365in}}%
\pgfpathlineto{\pgfqpoint{-319.224843in}{0.818365in}}%
\pgfpathclose%
\pgfusepath{fill}%
\end{pgfscope}%
\begin{pgfscope}%
\pgfpathrectangle{\pgfqpoint{10.668400in}{0.589870in}}{\pgfqpoint{2.188235in}{0.972632in}}%
\pgfusepath{clip}%
\pgfsetbuttcap%
\pgfsetmiterjoin%
\definecolor{currentfill}{rgb}{0.121569,0.466667,0.705882}%
\pgfsetfillcolor{currentfill}%
\pgfsetlinewidth{0.000000pt}%
\definecolor{currentstroke}{rgb}{0.000000,0.000000,0.000000}%
\pgfsetstrokecolor{currentstroke}%
\pgfsetstrokeopacity{0.000000}%
\pgfsetdash{}{0pt}%
\pgfpathmoveto{\pgfqpoint{-319.224843in}{0.820137in}}%
\pgfpathlineto{\pgfqpoint{12.241206in}{0.820137in}}%
\pgfpathlineto{\pgfqpoint{12.241206in}{0.827225in}}%
\pgfpathlineto{\pgfqpoint{-319.224843in}{0.827225in}}%
\pgfpathclose%
\pgfusepath{fill}%
\end{pgfscope}%
\begin{pgfscope}%
\pgfpathrectangle{\pgfqpoint{10.668400in}{0.589870in}}{\pgfqpoint{2.188235in}{0.972632in}}%
\pgfusepath{clip}%
\pgfsetbuttcap%
\pgfsetmiterjoin%
\definecolor{currentfill}{rgb}{0.121569,0.466667,0.705882}%
\pgfsetfillcolor{currentfill}%
\pgfsetlinewidth{0.000000pt}%
\definecolor{currentstroke}{rgb}{0.000000,0.000000,0.000000}%
\pgfsetstrokecolor{currentstroke}%
\pgfsetstrokeopacity{0.000000}%
\pgfsetdash{}{0pt}%
\pgfpathmoveto{\pgfqpoint{-319.224843in}{0.828997in}}%
\pgfpathlineto{\pgfqpoint{11.894575in}{0.828997in}}%
\pgfpathlineto{\pgfqpoint{11.894575in}{0.836085in}}%
\pgfpathlineto{\pgfqpoint{-319.224843in}{0.836085in}}%
\pgfpathclose%
\pgfusepath{fill}%
\end{pgfscope}%
\begin{pgfscope}%
\pgfpathrectangle{\pgfqpoint{10.668400in}{0.589870in}}{\pgfqpoint{2.188235in}{0.972632in}}%
\pgfusepath{clip}%
\pgfsetbuttcap%
\pgfsetmiterjoin%
\definecolor{currentfill}{rgb}{0.121569,0.466667,0.705882}%
\pgfsetfillcolor{currentfill}%
\pgfsetlinewidth{0.000000pt}%
\definecolor{currentstroke}{rgb}{0.000000,0.000000,0.000000}%
\pgfsetstrokecolor{currentstroke}%
\pgfsetstrokeopacity{0.000000}%
\pgfsetdash{}{0pt}%
\pgfpathmoveto{\pgfqpoint{-319.224843in}{0.837857in}}%
\pgfpathlineto{\pgfqpoint{12.256322in}{0.837857in}}%
\pgfpathlineto{\pgfqpoint{12.256322in}{0.844945in}}%
\pgfpathlineto{\pgfqpoint{-319.224843in}{0.844945in}}%
\pgfpathclose%
\pgfusepath{fill}%
\end{pgfscope}%
\begin{pgfscope}%
\pgfpathrectangle{\pgfqpoint{10.668400in}{0.589870in}}{\pgfqpoint{2.188235in}{0.972632in}}%
\pgfusepath{clip}%
\pgfsetbuttcap%
\pgfsetmiterjoin%
\definecolor{currentfill}{rgb}{0.121569,0.466667,0.705882}%
\pgfsetfillcolor{currentfill}%
\pgfsetlinewidth{0.000000pt}%
\definecolor{currentstroke}{rgb}{0.000000,0.000000,0.000000}%
\pgfsetstrokecolor{currentstroke}%
\pgfsetstrokeopacity{0.000000}%
\pgfsetdash{}{0pt}%
\pgfpathmoveto{\pgfqpoint{-319.224843in}{0.846717in}}%
\pgfpathlineto{\pgfqpoint{12.116017in}{0.846717in}}%
\pgfpathlineto{\pgfqpoint{12.116017in}{0.853804in}}%
\pgfpathlineto{\pgfqpoint{-319.224843in}{0.853804in}}%
\pgfpathclose%
\pgfusepath{fill}%
\end{pgfscope}%
\begin{pgfscope}%
\pgfpathrectangle{\pgfqpoint{10.668400in}{0.589870in}}{\pgfqpoint{2.188235in}{0.972632in}}%
\pgfusepath{clip}%
\pgfsetbuttcap%
\pgfsetmiterjoin%
\definecolor{currentfill}{rgb}{0.121569,0.466667,0.705882}%
\pgfsetfillcolor{currentfill}%
\pgfsetlinewidth{0.000000pt}%
\definecolor{currentstroke}{rgb}{0.000000,0.000000,0.000000}%
\pgfsetstrokecolor{currentstroke}%
\pgfsetstrokeopacity{0.000000}%
\pgfsetdash{}{0pt}%
\pgfpathmoveto{\pgfqpoint{-319.224843in}{0.855576in}}%
\pgfpathlineto{\pgfqpoint{12.323964in}{0.855576in}}%
\pgfpathlineto{\pgfqpoint{12.323964in}{0.862664in}}%
\pgfpathlineto{\pgfqpoint{-319.224843in}{0.862664in}}%
\pgfpathclose%
\pgfusepath{fill}%
\end{pgfscope}%
\begin{pgfscope}%
\pgfpathrectangle{\pgfqpoint{10.668400in}{0.589870in}}{\pgfqpoint{2.188235in}{0.972632in}}%
\pgfusepath{clip}%
\pgfsetbuttcap%
\pgfsetmiterjoin%
\definecolor{currentfill}{rgb}{0.121569,0.466667,0.705882}%
\pgfsetfillcolor{currentfill}%
\pgfsetlinewidth{0.000000pt}%
\definecolor{currentstroke}{rgb}{0.000000,0.000000,0.000000}%
\pgfsetstrokecolor{currentstroke}%
\pgfsetstrokeopacity{0.000000}%
\pgfsetdash{}{0pt}%
\pgfpathmoveto{\pgfqpoint{-319.224843in}{0.864436in}}%
\pgfpathlineto{\pgfqpoint{12.697387in}{0.864436in}}%
\pgfpathlineto{\pgfqpoint{12.697387in}{0.871524in}}%
\pgfpathlineto{\pgfqpoint{-319.224843in}{0.871524in}}%
\pgfpathclose%
\pgfusepath{fill}%
\end{pgfscope}%
\begin{pgfscope}%
\pgfpathrectangle{\pgfqpoint{10.668400in}{0.589870in}}{\pgfqpoint{2.188235in}{0.972632in}}%
\pgfusepath{clip}%
\pgfsetbuttcap%
\pgfsetmiterjoin%
\definecolor{currentfill}{rgb}{0.121569,0.466667,0.705882}%
\pgfsetfillcolor{currentfill}%
\pgfsetlinewidth{0.000000pt}%
\definecolor{currentstroke}{rgb}{0.000000,0.000000,0.000000}%
\pgfsetstrokecolor{currentstroke}%
\pgfsetstrokeopacity{0.000000}%
\pgfsetdash{}{0pt}%
\pgfpathmoveto{\pgfqpoint{-319.224843in}{0.873296in}}%
\pgfpathlineto{\pgfqpoint{12.211532in}{0.873296in}}%
\pgfpathlineto{\pgfqpoint{12.211532in}{0.880384in}}%
\pgfpathlineto{\pgfqpoint{-319.224843in}{0.880384in}}%
\pgfpathclose%
\pgfusepath{fill}%
\end{pgfscope}%
\begin{pgfscope}%
\pgfpathrectangle{\pgfqpoint{10.668400in}{0.589870in}}{\pgfqpoint{2.188235in}{0.972632in}}%
\pgfusepath{clip}%
\pgfsetbuttcap%
\pgfsetmiterjoin%
\definecolor{currentfill}{rgb}{0.121569,0.466667,0.705882}%
\pgfsetfillcolor{currentfill}%
\pgfsetlinewidth{0.000000pt}%
\definecolor{currentstroke}{rgb}{0.000000,0.000000,0.000000}%
\pgfsetstrokecolor{currentstroke}%
\pgfsetstrokeopacity{0.000000}%
\pgfsetdash{}{0pt}%
\pgfpathmoveto{\pgfqpoint{-319.224843in}{0.882156in}}%
\pgfpathlineto{\pgfqpoint{11.890721in}{0.882156in}}%
\pgfpathlineto{\pgfqpoint{11.890721in}{0.889244in}}%
\pgfpathlineto{\pgfqpoint{-319.224843in}{0.889244in}}%
\pgfpathclose%
\pgfusepath{fill}%
\end{pgfscope}%
\begin{pgfscope}%
\pgfpathrectangle{\pgfqpoint{10.668400in}{0.589870in}}{\pgfqpoint{2.188235in}{0.972632in}}%
\pgfusepath{clip}%
\pgfsetbuttcap%
\pgfsetmiterjoin%
\definecolor{currentfill}{rgb}{0.121569,0.466667,0.705882}%
\pgfsetfillcolor{currentfill}%
\pgfsetlinewidth{0.000000pt}%
\definecolor{currentstroke}{rgb}{0.000000,0.000000,0.000000}%
\pgfsetstrokecolor{currentstroke}%
\pgfsetstrokeopacity{0.000000}%
\pgfsetdash{}{0pt}%
\pgfpathmoveto{\pgfqpoint{-319.224843in}{0.891016in}}%
\pgfpathlineto{\pgfqpoint{12.136203in}{0.891016in}}%
\pgfpathlineto{\pgfqpoint{12.136203in}{0.898104in}}%
\pgfpathlineto{\pgfqpoint{-319.224843in}{0.898104in}}%
\pgfpathclose%
\pgfusepath{fill}%
\end{pgfscope}%
\begin{pgfscope}%
\pgfpathrectangle{\pgfqpoint{10.668400in}{0.589870in}}{\pgfqpoint{2.188235in}{0.972632in}}%
\pgfusepath{clip}%
\pgfsetbuttcap%
\pgfsetmiterjoin%
\definecolor{currentfill}{rgb}{0.121569,0.466667,0.705882}%
\pgfsetfillcolor{currentfill}%
\pgfsetlinewidth{0.000000pt}%
\definecolor{currentstroke}{rgb}{0.000000,0.000000,0.000000}%
\pgfsetstrokecolor{currentstroke}%
\pgfsetstrokeopacity{0.000000}%
\pgfsetdash{}{0pt}%
\pgfpathmoveto{\pgfqpoint{-319.224843in}{0.899876in}}%
\pgfpathlineto{\pgfqpoint{12.187617in}{0.899876in}}%
\pgfpathlineto{\pgfqpoint{12.187617in}{0.906963in}}%
\pgfpathlineto{\pgfqpoint{-319.224843in}{0.906963in}}%
\pgfpathclose%
\pgfusepath{fill}%
\end{pgfscope}%
\begin{pgfscope}%
\pgfpathrectangle{\pgfqpoint{10.668400in}{0.589870in}}{\pgfqpoint{2.188235in}{0.972632in}}%
\pgfusepath{clip}%
\pgfsetbuttcap%
\pgfsetmiterjoin%
\definecolor{currentfill}{rgb}{0.121569,0.466667,0.705882}%
\pgfsetfillcolor{currentfill}%
\pgfsetlinewidth{0.000000pt}%
\definecolor{currentstroke}{rgb}{0.000000,0.000000,0.000000}%
\pgfsetstrokecolor{currentstroke}%
\pgfsetstrokeopacity{0.000000}%
\pgfsetdash{}{0pt}%
\pgfpathmoveto{\pgfqpoint{-319.224843in}{0.908735in}}%
\pgfpathlineto{\pgfqpoint{12.198269in}{0.908735in}}%
\pgfpathlineto{\pgfqpoint{12.198269in}{0.915823in}}%
\pgfpathlineto{\pgfqpoint{-319.224843in}{0.915823in}}%
\pgfpathclose%
\pgfusepath{fill}%
\end{pgfscope}%
\begin{pgfscope}%
\pgfpathrectangle{\pgfqpoint{10.668400in}{0.589870in}}{\pgfqpoint{2.188235in}{0.972632in}}%
\pgfusepath{clip}%
\pgfsetbuttcap%
\pgfsetmiterjoin%
\definecolor{currentfill}{rgb}{0.121569,0.466667,0.705882}%
\pgfsetfillcolor{currentfill}%
\pgfsetlinewidth{0.000000pt}%
\definecolor{currentstroke}{rgb}{0.000000,0.000000,0.000000}%
\pgfsetstrokecolor{currentstroke}%
\pgfsetstrokeopacity{0.000000}%
\pgfsetdash{}{0pt}%
\pgfpathmoveto{\pgfqpoint{-319.224843in}{0.917595in}}%
\pgfpathlineto{\pgfqpoint{12.161557in}{0.917595in}}%
\pgfpathlineto{\pgfqpoint{12.161557in}{0.924683in}}%
\pgfpathlineto{\pgfqpoint{-319.224843in}{0.924683in}}%
\pgfpathclose%
\pgfusepath{fill}%
\end{pgfscope}%
\begin{pgfscope}%
\pgfpathrectangle{\pgfqpoint{10.668400in}{0.589870in}}{\pgfqpoint{2.188235in}{0.972632in}}%
\pgfusepath{clip}%
\pgfsetbuttcap%
\pgfsetmiterjoin%
\definecolor{currentfill}{rgb}{0.121569,0.466667,0.705882}%
\pgfsetfillcolor{currentfill}%
\pgfsetlinewidth{0.000000pt}%
\definecolor{currentstroke}{rgb}{0.000000,0.000000,0.000000}%
\pgfsetstrokecolor{currentstroke}%
\pgfsetstrokeopacity{0.000000}%
\pgfsetdash{}{0pt}%
\pgfpathmoveto{\pgfqpoint{-319.224843in}{0.926455in}}%
\pgfpathlineto{\pgfqpoint{12.299833in}{0.926455in}}%
\pgfpathlineto{\pgfqpoint{12.299833in}{0.933543in}}%
\pgfpathlineto{\pgfqpoint{-319.224843in}{0.933543in}}%
\pgfpathclose%
\pgfusepath{fill}%
\end{pgfscope}%
\begin{pgfscope}%
\pgfpathrectangle{\pgfqpoint{10.668400in}{0.589870in}}{\pgfqpoint{2.188235in}{0.972632in}}%
\pgfusepath{clip}%
\pgfsetbuttcap%
\pgfsetmiterjoin%
\definecolor{currentfill}{rgb}{0.121569,0.466667,0.705882}%
\pgfsetfillcolor{currentfill}%
\pgfsetlinewidth{0.000000pt}%
\definecolor{currentstroke}{rgb}{0.000000,0.000000,0.000000}%
\pgfsetstrokecolor{currentstroke}%
\pgfsetstrokeopacity{0.000000}%
\pgfsetdash{}{0pt}%
\pgfpathmoveto{\pgfqpoint{-319.224843in}{0.935315in}}%
\pgfpathlineto{\pgfqpoint{12.258198in}{0.935315in}}%
\pgfpathlineto{\pgfqpoint{12.258198in}{0.942403in}}%
\pgfpathlineto{\pgfqpoint{-319.224843in}{0.942403in}}%
\pgfpathclose%
\pgfusepath{fill}%
\end{pgfscope}%
\begin{pgfscope}%
\pgfpathrectangle{\pgfqpoint{10.668400in}{0.589870in}}{\pgfqpoint{2.188235in}{0.972632in}}%
\pgfusepath{clip}%
\pgfsetbuttcap%
\pgfsetmiterjoin%
\definecolor{currentfill}{rgb}{0.121569,0.466667,0.705882}%
\pgfsetfillcolor{currentfill}%
\pgfsetlinewidth{0.000000pt}%
\definecolor{currentstroke}{rgb}{0.000000,0.000000,0.000000}%
\pgfsetstrokecolor{currentstroke}%
\pgfsetstrokeopacity{0.000000}%
\pgfsetdash{}{0pt}%
\pgfpathmoveto{\pgfqpoint{-319.224843in}{0.944175in}}%
\pgfpathlineto{\pgfqpoint{12.285314in}{0.944175in}}%
\pgfpathlineto{\pgfqpoint{12.285314in}{0.951263in}}%
\pgfpathlineto{\pgfqpoint{-319.224843in}{0.951263in}}%
\pgfpathclose%
\pgfusepath{fill}%
\end{pgfscope}%
\begin{pgfscope}%
\pgfpathrectangle{\pgfqpoint{10.668400in}{0.589870in}}{\pgfqpoint{2.188235in}{0.972632in}}%
\pgfusepath{clip}%
\pgfsetbuttcap%
\pgfsetmiterjoin%
\definecolor{currentfill}{rgb}{0.121569,0.466667,0.705882}%
\pgfsetfillcolor{currentfill}%
\pgfsetlinewidth{0.000000pt}%
\definecolor{currentstroke}{rgb}{0.000000,0.000000,0.000000}%
\pgfsetstrokecolor{currentstroke}%
\pgfsetstrokeopacity{0.000000}%
\pgfsetdash{}{0pt}%
\pgfpathmoveto{\pgfqpoint{-319.224843in}{0.953035in}}%
\pgfpathlineto{\pgfqpoint{12.144569in}{0.953035in}}%
\pgfpathlineto{\pgfqpoint{12.144569in}{0.960122in}}%
\pgfpathlineto{\pgfqpoint{-319.224843in}{0.960122in}}%
\pgfpathclose%
\pgfusepath{fill}%
\end{pgfscope}%
\begin{pgfscope}%
\pgfpathrectangle{\pgfqpoint{10.668400in}{0.589870in}}{\pgfqpoint{2.188235in}{0.972632in}}%
\pgfusepath{clip}%
\pgfsetbuttcap%
\pgfsetmiterjoin%
\definecolor{currentfill}{rgb}{0.121569,0.466667,0.705882}%
\pgfsetfillcolor{currentfill}%
\pgfsetlinewidth{0.000000pt}%
\definecolor{currentstroke}{rgb}{0.000000,0.000000,0.000000}%
\pgfsetstrokecolor{currentstroke}%
\pgfsetstrokeopacity{0.000000}%
\pgfsetdash{}{0pt}%
\pgfpathmoveto{\pgfqpoint{-319.224843in}{0.961894in}}%
\pgfpathlineto{\pgfqpoint{12.353653in}{0.961894in}}%
\pgfpathlineto{\pgfqpoint{12.353653in}{0.968982in}}%
\pgfpathlineto{\pgfqpoint{-319.224843in}{0.968982in}}%
\pgfpathclose%
\pgfusepath{fill}%
\end{pgfscope}%
\begin{pgfscope}%
\pgfpathrectangle{\pgfqpoint{10.668400in}{0.589870in}}{\pgfqpoint{2.188235in}{0.972632in}}%
\pgfusepath{clip}%
\pgfsetbuttcap%
\pgfsetmiterjoin%
\definecolor{currentfill}{rgb}{0.121569,0.466667,0.705882}%
\pgfsetfillcolor{currentfill}%
\pgfsetlinewidth{0.000000pt}%
\definecolor{currentstroke}{rgb}{0.000000,0.000000,0.000000}%
\pgfsetstrokecolor{currentstroke}%
\pgfsetstrokeopacity{0.000000}%
\pgfsetdash{}{0pt}%
\pgfpathmoveto{\pgfqpoint{-319.224843in}{0.970754in}}%
\pgfpathlineto{\pgfqpoint{12.202403in}{0.970754in}}%
\pgfpathlineto{\pgfqpoint{12.202403in}{0.977842in}}%
\pgfpathlineto{\pgfqpoint{-319.224843in}{0.977842in}}%
\pgfpathclose%
\pgfusepath{fill}%
\end{pgfscope}%
\begin{pgfscope}%
\pgfpathrectangle{\pgfqpoint{10.668400in}{0.589870in}}{\pgfqpoint{2.188235in}{0.972632in}}%
\pgfusepath{clip}%
\pgfsetbuttcap%
\pgfsetmiterjoin%
\definecolor{currentfill}{rgb}{0.121569,0.466667,0.705882}%
\pgfsetfillcolor{currentfill}%
\pgfsetlinewidth{0.000000pt}%
\definecolor{currentstroke}{rgb}{0.000000,0.000000,0.000000}%
\pgfsetstrokecolor{currentstroke}%
\pgfsetstrokeopacity{0.000000}%
\pgfsetdash{}{0pt}%
\pgfpathmoveto{\pgfqpoint{-319.224843in}{0.979614in}}%
\pgfpathlineto{\pgfqpoint{12.337011in}{0.979614in}}%
\pgfpathlineto{\pgfqpoint{12.337011in}{0.986702in}}%
\pgfpathlineto{\pgfqpoint{-319.224843in}{0.986702in}}%
\pgfpathclose%
\pgfusepath{fill}%
\end{pgfscope}%
\begin{pgfscope}%
\pgfpathrectangle{\pgfqpoint{10.668400in}{0.589870in}}{\pgfqpoint{2.188235in}{0.972632in}}%
\pgfusepath{clip}%
\pgfsetbuttcap%
\pgfsetmiterjoin%
\definecolor{currentfill}{rgb}{0.121569,0.466667,0.705882}%
\pgfsetfillcolor{currentfill}%
\pgfsetlinewidth{0.000000pt}%
\definecolor{currentstroke}{rgb}{0.000000,0.000000,0.000000}%
\pgfsetstrokecolor{currentstroke}%
\pgfsetstrokeopacity{0.000000}%
\pgfsetdash{}{0pt}%
\pgfpathmoveto{\pgfqpoint{-319.224843in}{0.988474in}}%
\pgfpathlineto{\pgfqpoint{12.401171in}{0.988474in}}%
\pgfpathlineto{\pgfqpoint{12.401171in}{0.995562in}}%
\pgfpathlineto{\pgfqpoint{-319.224843in}{0.995562in}}%
\pgfpathclose%
\pgfusepath{fill}%
\end{pgfscope}%
\begin{pgfscope}%
\pgfpathrectangle{\pgfqpoint{10.668400in}{0.589870in}}{\pgfqpoint{2.188235in}{0.972632in}}%
\pgfusepath{clip}%
\pgfsetbuttcap%
\pgfsetmiterjoin%
\definecolor{currentfill}{rgb}{0.121569,0.466667,0.705882}%
\pgfsetfillcolor{currentfill}%
\pgfsetlinewidth{0.000000pt}%
\definecolor{currentstroke}{rgb}{0.000000,0.000000,0.000000}%
\pgfsetstrokecolor{currentstroke}%
\pgfsetstrokeopacity{0.000000}%
\pgfsetdash{}{0pt}%
\pgfpathmoveto{\pgfqpoint{-319.224843in}{0.997334in}}%
\pgfpathlineto{\pgfqpoint{11.922539in}{0.997334in}}%
\pgfpathlineto{\pgfqpoint{11.922539in}{1.004421in}}%
\pgfpathlineto{\pgfqpoint{-319.224843in}{1.004421in}}%
\pgfpathclose%
\pgfusepath{fill}%
\end{pgfscope}%
\begin{pgfscope}%
\pgfpathrectangle{\pgfqpoint{10.668400in}{0.589870in}}{\pgfqpoint{2.188235in}{0.972632in}}%
\pgfusepath{clip}%
\pgfsetbuttcap%
\pgfsetmiterjoin%
\definecolor{currentfill}{rgb}{0.121569,0.466667,0.705882}%
\pgfsetfillcolor{currentfill}%
\pgfsetlinewidth{0.000000pt}%
\definecolor{currentstroke}{rgb}{0.000000,0.000000,0.000000}%
\pgfsetstrokecolor{currentstroke}%
\pgfsetstrokeopacity{0.000000}%
\pgfsetdash{}{0pt}%
\pgfpathmoveto{\pgfqpoint{-319.224843in}{1.006193in}}%
\pgfpathlineto{\pgfqpoint{12.284266in}{1.006193in}}%
\pgfpathlineto{\pgfqpoint{12.284266in}{1.013281in}}%
\pgfpathlineto{\pgfqpoint{-319.224843in}{1.013281in}}%
\pgfpathclose%
\pgfusepath{fill}%
\end{pgfscope}%
\begin{pgfscope}%
\pgfpathrectangle{\pgfqpoint{10.668400in}{0.589870in}}{\pgfqpoint{2.188235in}{0.972632in}}%
\pgfusepath{clip}%
\pgfsetbuttcap%
\pgfsetmiterjoin%
\definecolor{currentfill}{rgb}{0.121569,0.466667,0.705882}%
\pgfsetfillcolor{currentfill}%
\pgfsetlinewidth{0.000000pt}%
\definecolor{currentstroke}{rgb}{0.000000,0.000000,0.000000}%
\pgfsetstrokecolor{currentstroke}%
\pgfsetstrokeopacity{0.000000}%
\pgfsetdash{}{0pt}%
\pgfpathmoveto{\pgfqpoint{-319.224843in}{1.015053in}}%
\pgfpathlineto{\pgfqpoint{12.323823in}{1.015053in}}%
\pgfpathlineto{\pgfqpoint{12.323823in}{1.022141in}}%
\pgfpathlineto{\pgfqpoint{-319.224843in}{1.022141in}}%
\pgfpathclose%
\pgfusepath{fill}%
\end{pgfscope}%
\begin{pgfscope}%
\pgfpathrectangle{\pgfqpoint{10.668400in}{0.589870in}}{\pgfqpoint{2.188235in}{0.972632in}}%
\pgfusepath{clip}%
\pgfsetbuttcap%
\pgfsetmiterjoin%
\definecolor{currentfill}{rgb}{0.121569,0.466667,0.705882}%
\pgfsetfillcolor{currentfill}%
\pgfsetlinewidth{0.000000pt}%
\definecolor{currentstroke}{rgb}{0.000000,0.000000,0.000000}%
\pgfsetstrokecolor{currentstroke}%
\pgfsetstrokeopacity{0.000000}%
\pgfsetdash{}{0pt}%
\pgfpathmoveto{\pgfqpoint{-319.224843in}{1.023913in}}%
\pgfpathlineto{\pgfqpoint{12.353407in}{1.023913in}}%
\pgfpathlineto{\pgfqpoint{12.353407in}{1.031001in}}%
\pgfpathlineto{\pgfqpoint{-319.224843in}{1.031001in}}%
\pgfpathclose%
\pgfusepath{fill}%
\end{pgfscope}%
\begin{pgfscope}%
\pgfpathrectangle{\pgfqpoint{10.668400in}{0.589870in}}{\pgfqpoint{2.188235in}{0.972632in}}%
\pgfusepath{clip}%
\pgfsetbuttcap%
\pgfsetmiterjoin%
\definecolor{currentfill}{rgb}{0.121569,0.466667,0.705882}%
\pgfsetfillcolor{currentfill}%
\pgfsetlinewidth{0.000000pt}%
\definecolor{currentstroke}{rgb}{0.000000,0.000000,0.000000}%
\pgfsetstrokecolor{currentstroke}%
\pgfsetstrokeopacity{0.000000}%
\pgfsetdash{}{0pt}%
\pgfpathmoveto{\pgfqpoint{-319.224843in}{1.032773in}}%
\pgfpathlineto{\pgfqpoint{11.894715in}{1.032773in}}%
\pgfpathlineto{\pgfqpoint{11.894715in}{1.039861in}}%
\pgfpathlineto{\pgfqpoint{-319.224843in}{1.039861in}}%
\pgfpathclose%
\pgfusepath{fill}%
\end{pgfscope}%
\begin{pgfscope}%
\pgfpathrectangle{\pgfqpoint{10.668400in}{0.589870in}}{\pgfqpoint{2.188235in}{0.972632in}}%
\pgfusepath{clip}%
\pgfsetbuttcap%
\pgfsetmiterjoin%
\definecolor{currentfill}{rgb}{0.121569,0.466667,0.705882}%
\pgfsetfillcolor{currentfill}%
\pgfsetlinewidth{0.000000pt}%
\definecolor{currentstroke}{rgb}{0.000000,0.000000,0.000000}%
\pgfsetstrokecolor{currentstroke}%
\pgfsetstrokeopacity{0.000000}%
\pgfsetdash{}{0pt}%
\pgfpathmoveto{\pgfqpoint{-319.224843in}{1.041633in}}%
\pgfpathlineto{\pgfqpoint{12.227201in}{1.041633in}}%
\pgfpathlineto{\pgfqpoint{12.227201in}{1.048721in}}%
\pgfpathlineto{\pgfqpoint{-319.224843in}{1.048721in}}%
\pgfpathclose%
\pgfusepath{fill}%
\end{pgfscope}%
\begin{pgfscope}%
\pgfpathrectangle{\pgfqpoint{10.668400in}{0.589870in}}{\pgfqpoint{2.188235in}{0.972632in}}%
\pgfusepath{clip}%
\pgfsetbuttcap%
\pgfsetmiterjoin%
\definecolor{currentfill}{rgb}{0.121569,0.466667,0.705882}%
\pgfsetfillcolor{currentfill}%
\pgfsetlinewidth{0.000000pt}%
\definecolor{currentstroke}{rgb}{0.000000,0.000000,0.000000}%
\pgfsetstrokecolor{currentstroke}%
\pgfsetstrokeopacity{0.000000}%
\pgfsetdash{}{0pt}%
\pgfpathmoveto{\pgfqpoint{-319.224843in}{1.050493in}}%
\pgfpathlineto{\pgfqpoint{12.240635in}{1.050493in}}%
\pgfpathlineto{\pgfqpoint{12.240635in}{1.057580in}}%
\pgfpathlineto{\pgfqpoint{-319.224843in}{1.057580in}}%
\pgfpathclose%
\pgfusepath{fill}%
\end{pgfscope}%
\begin{pgfscope}%
\pgfpathrectangle{\pgfqpoint{10.668400in}{0.589870in}}{\pgfqpoint{2.188235in}{0.972632in}}%
\pgfusepath{clip}%
\pgfsetbuttcap%
\pgfsetmiterjoin%
\definecolor{currentfill}{rgb}{0.121569,0.466667,0.705882}%
\pgfsetfillcolor{currentfill}%
\pgfsetlinewidth{0.000000pt}%
\definecolor{currentstroke}{rgb}{0.000000,0.000000,0.000000}%
\pgfsetstrokecolor{currentstroke}%
\pgfsetstrokeopacity{0.000000}%
\pgfsetdash{}{0pt}%
\pgfpathmoveto{\pgfqpoint{-319.224843in}{1.059352in}}%
\pgfpathlineto{\pgfqpoint{12.163769in}{1.059352in}}%
\pgfpathlineto{\pgfqpoint{12.163769in}{1.066440in}}%
\pgfpathlineto{\pgfqpoint{-319.224843in}{1.066440in}}%
\pgfpathclose%
\pgfusepath{fill}%
\end{pgfscope}%
\begin{pgfscope}%
\pgfpathrectangle{\pgfqpoint{10.668400in}{0.589870in}}{\pgfqpoint{2.188235in}{0.972632in}}%
\pgfusepath{clip}%
\pgfsetbuttcap%
\pgfsetmiterjoin%
\definecolor{currentfill}{rgb}{0.121569,0.466667,0.705882}%
\pgfsetfillcolor{currentfill}%
\pgfsetlinewidth{0.000000pt}%
\definecolor{currentstroke}{rgb}{0.000000,0.000000,0.000000}%
\pgfsetstrokecolor{currentstroke}%
\pgfsetstrokeopacity{0.000000}%
\pgfsetdash{}{0pt}%
\pgfpathmoveto{\pgfqpoint{-319.224843in}{1.068212in}}%
\pgfpathlineto{\pgfqpoint{12.016860in}{1.068212in}}%
\pgfpathlineto{\pgfqpoint{12.016860in}{1.075300in}}%
\pgfpathlineto{\pgfqpoint{-319.224843in}{1.075300in}}%
\pgfpathclose%
\pgfusepath{fill}%
\end{pgfscope}%
\begin{pgfscope}%
\pgfpathrectangle{\pgfqpoint{10.668400in}{0.589870in}}{\pgfqpoint{2.188235in}{0.972632in}}%
\pgfusepath{clip}%
\pgfsetbuttcap%
\pgfsetmiterjoin%
\definecolor{currentfill}{rgb}{0.121569,0.466667,0.705882}%
\pgfsetfillcolor{currentfill}%
\pgfsetlinewidth{0.000000pt}%
\definecolor{currentstroke}{rgb}{0.000000,0.000000,0.000000}%
\pgfsetstrokecolor{currentstroke}%
\pgfsetstrokeopacity{0.000000}%
\pgfsetdash{}{0pt}%
\pgfpathmoveto{\pgfqpoint{-319.224843in}{1.077072in}}%
\pgfpathlineto{\pgfqpoint{12.566462in}{1.077072in}}%
\pgfpathlineto{\pgfqpoint{12.566462in}{1.084160in}}%
\pgfpathlineto{\pgfqpoint{-319.224843in}{1.084160in}}%
\pgfpathclose%
\pgfusepath{fill}%
\end{pgfscope}%
\begin{pgfscope}%
\pgfpathrectangle{\pgfqpoint{10.668400in}{0.589870in}}{\pgfqpoint{2.188235in}{0.972632in}}%
\pgfusepath{clip}%
\pgfsetbuttcap%
\pgfsetmiterjoin%
\definecolor{currentfill}{rgb}{0.121569,0.466667,0.705882}%
\pgfsetfillcolor{currentfill}%
\pgfsetlinewidth{0.000000pt}%
\definecolor{currentstroke}{rgb}{0.000000,0.000000,0.000000}%
\pgfsetstrokecolor{currentstroke}%
\pgfsetstrokeopacity{0.000000}%
\pgfsetdash{}{0pt}%
\pgfpathmoveto{\pgfqpoint{-319.224843in}{1.085932in}}%
\pgfpathlineto{\pgfqpoint{12.456415in}{1.085932in}}%
\pgfpathlineto{\pgfqpoint{12.456415in}{1.093020in}}%
\pgfpathlineto{\pgfqpoint{-319.224843in}{1.093020in}}%
\pgfpathclose%
\pgfusepath{fill}%
\end{pgfscope}%
\begin{pgfscope}%
\pgfpathrectangle{\pgfqpoint{10.668400in}{0.589870in}}{\pgfqpoint{2.188235in}{0.972632in}}%
\pgfusepath{clip}%
\pgfsetbuttcap%
\pgfsetmiterjoin%
\definecolor{currentfill}{rgb}{0.121569,0.466667,0.705882}%
\pgfsetfillcolor{currentfill}%
\pgfsetlinewidth{0.000000pt}%
\definecolor{currentstroke}{rgb}{0.000000,0.000000,0.000000}%
\pgfsetstrokecolor{currentstroke}%
\pgfsetstrokeopacity{0.000000}%
\pgfsetdash{}{0pt}%
\pgfpathmoveto{\pgfqpoint{-319.224843in}{1.094792in}}%
\pgfpathlineto{\pgfqpoint{12.555837in}{1.094792in}}%
\pgfpathlineto{\pgfqpoint{12.555837in}{1.101880in}}%
\pgfpathlineto{\pgfqpoint{-319.224843in}{1.101880in}}%
\pgfpathclose%
\pgfusepath{fill}%
\end{pgfscope}%
\begin{pgfscope}%
\pgfpathrectangle{\pgfqpoint{10.668400in}{0.589870in}}{\pgfqpoint{2.188235in}{0.972632in}}%
\pgfusepath{clip}%
\pgfsetbuttcap%
\pgfsetmiterjoin%
\definecolor{currentfill}{rgb}{0.121569,0.466667,0.705882}%
\pgfsetfillcolor{currentfill}%
\pgfsetlinewidth{0.000000pt}%
\definecolor{currentstroke}{rgb}{0.000000,0.000000,0.000000}%
\pgfsetstrokecolor{currentstroke}%
\pgfsetstrokeopacity{0.000000}%
\pgfsetdash{}{0pt}%
\pgfpathmoveto{\pgfqpoint{-319.224843in}{1.103652in}}%
\pgfpathlineto{\pgfqpoint{12.396334in}{1.103652in}}%
\pgfpathlineto{\pgfqpoint{12.396334in}{1.110739in}}%
\pgfpathlineto{\pgfqpoint{-319.224843in}{1.110739in}}%
\pgfpathclose%
\pgfusepath{fill}%
\end{pgfscope}%
\begin{pgfscope}%
\pgfpathrectangle{\pgfqpoint{10.668400in}{0.589870in}}{\pgfqpoint{2.188235in}{0.972632in}}%
\pgfusepath{clip}%
\pgfsetbuttcap%
\pgfsetmiterjoin%
\definecolor{currentfill}{rgb}{0.121569,0.466667,0.705882}%
\pgfsetfillcolor{currentfill}%
\pgfsetlinewidth{0.000000pt}%
\definecolor{currentstroke}{rgb}{0.000000,0.000000,0.000000}%
\pgfsetstrokecolor{currentstroke}%
\pgfsetstrokeopacity{0.000000}%
\pgfsetdash{}{0pt}%
\pgfpathmoveto{\pgfqpoint{-319.224843in}{1.112511in}}%
\pgfpathlineto{\pgfqpoint{12.518446in}{1.112511in}}%
\pgfpathlineto{\pgfqpoint{12.518446in}{1.119599in}}%
\pgfpathlineto{\pgfqpoint{-319.224843in}{1.119599in}}%
\pgfpathclose%
\pgfusepath{fill}%
\end{pgfscope}%
\begin{pgfscope}%
\pgfpathrectangle{\pgfqpoint{10.668400in}{0.589870in}}{\pgfqpoint{2.188235in}{0.972632in}}%
\pgfusepath{clip}%
\pgfsetbuttcap%
\pgfsetmiterjoin%
\definecolor{currentfill}{rgb}{0.121569,0.466667,0.705882}%
\pgfsetfillcolor{currentfill}%
\pgfsetlinewidth{0.000000pt}%
\definecolor{currentstroke}{rgb}{0.000000,0.000000,0.000000}%
\pgfsetstrokecolor{currentstroke}%
\pgfsetstrokeopacity{0.000000}%
\pgfsetdash{}{0pt}%
\pgfpathmoveto{\pgfqpoint{-319.224843in}{1.121371in}}%
\pgfpathlineto{\pgfqpoint{12.491957in}{1.121371in}}%
\pgfpathlineto{\pgfqpoint{12.491957in}{1.128459in}}%
\pgfpathlineto{\pgfqpoint{-319.224843in}{1.128459in}}%
\pgfpathclose%
\pgfusepath{fill}%
\end{pgfscope}%
\begin{pgfscope}%
\pgfpathrectangle{\pgfqpoint{10.668400in}{0.589870in}}{\pgfqpoint{2.188235in}{0.972632in}}%
\pgfusepath{clip}%
\pgfsetbuttcap%
\pgfsetmiterjoin%
\definecolor{currentfill}{rgb}{0.121569,0.466667,0.705882}%
\pgfsetfillcolor{currentfill}%
\pgfsetlinewidth{0.000000pt}%
\definecolor{currentstroke}{rgb}{0.000000,0.000000,0.000000}%
\pgfsetstrokecolor{currentstroke}%
\pgfsetstrokeopacity{0.000000}%
\pgfsetdash{}{0pt}%
\pgfpathmoveto{\pgfqpoint{-319.224843in}{1.130231in}}%
\pgfpathlineto{\pgfqpoint{10.767865in}{1.130231in}}%
\pgfpathlineto{\pgfqpoint{10.767865in}{1.137319in}}%
\pgfpathlineto{\pgfqpoint{-319.224843in}{1.137319in}}%
\pgfpathclose%
\pgfusepath{fill}%
\end{pgfscope}%
\begin{pgfscope}%
\pgfpathrectangle{\pgfqpoint{10.668400in}{0.589870in}}{\pgfqpoint{2.188235in}{0.972632in}}%
\pgfusepath{clip}%
\pgfsetbuttcap%
\pgfsetmiterjoin%
\definecolor{currentfill}{rgb}{0.121569,0.466667,0.705882}%
\pgfsetfillcolor{currentfill}%
\pgfsetlinewidth{0.000000pt}%
\definecolor{currentstroke}{rgb}{0.000000,0.000000,0.000000}%
\pgfsetstrokecolor{currentstroke}%
\pgfsetstrokeopacity{0.000000}%
\pgfsetdash{}{0pt}%
\pgfpathmoveto{\pgfqpoint{-319.224843in}{1.139091in}}%
\pgfpathlineto{\pgfqpoint{12.587271in}{1.139091in}}%
\pgfpathlineto{\pgfqpoint{12.587271in}{1.146179in}}%
\pgfpathlineto{\pgfqpoint{-319.224843in}{1.146179in}}%
\pgfpathclose%
\pgfusepath{fill}%
\end{pgfscope}%
\begin{pgfscope}%
\pgfpathrectangle{\pgfqpoint{10.668400in}{0.589870in}}{\pgfqpoint{2.188235in}{0.972632in}}%
\pgfusepath{clip}%
\pgfsetbuttcap%
\pgfsetmiterjoin%
\definecolor{currentfill}{rgb}{0.121569,0.466667,0.705882}%
\pgfsetfillcolor{currentfill}%
\pgfsetlinewidth{0.000000pt}%
\definecolor{currentstroke}{rgb}{0.000000,0.000000,0.000000}%
\pgfsetstrokecolor{currentstroke}%
\pgfsetstrokeopacity{0.000000}%
\pgfsetdash{}{0pt}%
\pgfpathmoveto{\pgfqpoint{-319.224843in}{1.147951in}}%
\pgfpathlineto{\pgfqpoint{11.912463in}{1.147951in}}%
\pgfpathlineto{\pgfqpoint{11.912463in}{1.155039in}}%
\pgfpathlineto{\pgfqpoint{-319.224843in}{1.155039in}}%
\pgfpathclose%
\pgfusepath{fill}%
\end{pgfscope}%
\begin{pgfscope}%
\pgfpathrectangle{\pgfqpoint{10.668400in}{0.589870in}}{\pgfqpoint{2.188235in}{0.972632in}}%
\pgfusepath{clip}%
\pgfsetbuttcap%
\pgfsetmiterjoin%
\definecolor{currentfill}{rgb}{0.121569,0.466667,0.705882}%
\pgfsetfillcolor{currentfill}%
\pgfsetlinewidth{0.000000pt}%
\definecolor{currentstroke}{rgb}{0.000000,0.000000,0.000000}%
\pgfsetstrokecolor{currentstroke}%
\pgfsetstrokeopacity{0.000000}%
\pgfsetdash{}{0pt}%
\pgfpathmoveto{\pgfqpoint{-319.224843in}{1.156810in}}%
\pgfpathlineto{\pgfqpoint{12.392225in}{1.156810in}}%
\pgfpathlineto{\pgfqpoint{12.392225in}{1.163898in}}%
\pgfpathlineto{\pgfqpoint{-319.224843in}{1.163898in}}%
\pgfpathclose%
\pgfusepath{fill}%
\end{pgfscope}%
\begin{pgfscope}%
\pgfpathrectangle{\pgfqpoint{10.668400in}{0.589870in}}{\pgfqpoint{2.188235in}{0.972632in}}%
\pgfusepath{clip}%
\pgfsetbuttcap%
\pgfsetmiterjoin%
\definecolor{currentfill}{rgb}{0.121569,0.466667,0.705882}%
\pgfsetfillcolor{currentfill}%
\pgfsetlinewidth{0.000000pt}%
\definecolor{currentstroke}{rgb}{0.000000,0.000000,0.000000}%
\pgfsetstrokecolor{currentstroke}%
\pgfsetstrokeopacity{0.000000}%
\pgfsetdash{}{0pt}%
\pgfpathmoveto{\pgfqpoint{-319.224843in}{1.165670in}}%
\pgfpathlineto{\pgfqpoint{12.705906in}{1.165670in}}%
\pgfpathlineto{\pgfqpoint{12.705906in}{1.172758in}}%
\pgfpathlineto{\pgfqpoint{-319.224843in}{1.172758in}}%
\pgfpathclose%
\pgfusepath{fill}%
\end{pgfscope}%
\begin{pgfscope}%
\pgfpathrectangle{\pgfqpoint{10.668400in}{0.589870in}}{\pgfqpoint{2.188235in}{0.972632in}}%
\pgfusepath{clip}%
\pgfsetbuttcap%
\pgfsetmiterjoin%
\definecolor{currentfill}{rgb}{0.121569,0.466667,0.705882}%
\pgfsetfillcolor{currentfill}%
\pgfsetlinewidth{0.000000pt}%
\definecolor{currentstroke}{rgb}{0.000000,0.000000,0.000000}%
\pgfsetstrokecolor{currentstroke}%
\pgfsetstrokeopacity{0.000000}%
\pgfsetdash{}{0pt}%
\pgfpathmoveto{\pgfqpoint{-319.224843in}{1.174530in}}%
\pgfpathlineto{\pgfqpoint{12.532625in}{1.174530in}}%
\pgfpathlineto{\pgfqpoint{12.532625in}{1.181618in}}%
\pgfpathlineto{\pgfqpoint{-319.224843in}{1.181618in}}%
\pgfpathclose%
\pgfusepath{fill}%
\end{pgfscope}%
\begin{pgfscope}%
\pgfpathrectangle{\pgfqpoint{10.668400in}{0.589870in}}{\pgfqpoint{2.188235in}{0.972632in}}%
\pgfusepath{clip}%
\pgfsetbuttcap%
\pgfsetmiterjoin%
\definecolor{currentfill}{rgb}{0.121569,0.466667,0.705882}%
\pgfsetfillcolor{currentfill}%
\pgfsetlinewidth{0.000000pt}%
\definecolor{currentstroke}{rgb}{0.000000,0.000000,0.000000}%
\pgfsetstrokecolor{currentstroke}%
\pgfsetstrokeopacity{0.000000}%
\pgfsetdash{}{0pt}%
\pgfpathmoveto{\pgfqpoint{-319.224843in}{1.183390in}}%
\pgfpathlineto{\pgfqpoint{12.558217in}{1.183390in}}%
\pgfpathlineto{\pgfqpoint{12.558217in}{1.190478in}}%
\pgfpathlineto{\pgfqpoint{-319.224843in}{1.190478in}}%
\pgfpathclose%
\pgfusepath{fill}%
\end{pgfscope}%
\begin{pgfscope}%
\pgfpathrectangle{\pgfqpoint{10.668400in}{0.589870in}}{\pgfqpoint{2.188235in}{0.972632in}}%
\pgfusepath{clip}%
\pgfsetbuttcap%
\pgfsetmiterjoin%
\definecolor{currentfill}{rgb}{0.121569,0.466667,0.705882}%
\pgfsetfillcolor{currentfill}%
\pgfsetlinewidth{0.000000pt}%
\definecolor{currentstroke}{rgb}{0.000000,0.000000,0.000000}%
\pgfsetstrokecolor{currentstroke}%
\pgfsetstrokeopacity{0.000000}%
\pgfsetdash{}{0pt}%
\pgfpathmoveto{\pgfqpoint{-319.224843in}{1.192250in}}%
\pgfpathlineto{\pgfqpoint{12.548896in}{1.192250in}}%
\pgfpathlineto{\pgfqpoint{12.548896in}{1.199338in}}%
\pgfpathlineto{\pgfqpoint{-319.224843in}{1.199338in}}%
\pgfpathclose%
\pgfusepath{fill}%
\end{pgfscope}%
\begin{pgfscope}%
\pgfpathrectangle{\pgfqpoint{10.668400in}{0.589870in}}{\pgfqpoint{2.188235in}{0.972632in}}%
\pgfusepath{clip}%
\pgfsetbuttcap%
\pgfsetmiterjoin%
\definecolor{currentfill}{rgb}{0.121569,0.466667,0.705882}%
\pgfsetfillcolor{currentfill}%
\pgfsetlinewidth{0.000000pt}%
\definecolor{currentstroke}{rgb}{0.000000,0.000000,0.000000}%
\pgfsetstrokecolor{currentstroke}%
\pgfsetstrokeopacity{0.000000}%
\pgfsetdash{}{0pt}%
\pgfpathmoveto{\pgfqpoint{-319.224843in}{1.201110in}}%
\pgfpathlineto{\pgfqpoint{12.322351in}{1.201110in}}%
\pgfpathlineto{\pgfqpoint{12.322351in}{1.208197in}}%
\pgfpathlineto{\pgfqpoint{-319.224843in}{1.208197in}}%
\pgfpathclose%
\pgfusepath{fill}%
\end{pgfscope}%
\begin{pgfscope}%
\pgfpathrectangle{\pgfqpoint{10.668400in}{0.589870in}}{\pgfqpoint{2.188235in}{0.972632in}}%
\pgfusepath{clip}%
\pgfsetbuttcap%
\pgfsetmiterjoin%
\definecolor{currentfill}{rgb}{0.121569,0.466667,0.705882}%
\pgfsetfillcolor{currentfill}%
\pgfsetlinewidth{0.000000pt}%
\definecolor{currentstroke}{rgb}{0.000000,0.000000,0.000000}%
\pgfsetstrokecolor{currentstroke}%
\pgfsetstrokeopacity{0.000000}%
\pgfsetdash{}{0pt}%
\pgfpathmoveto{\pgfqpoint{-319.224843in}{1.209969in}}%
\pgfpathlineto{\pgfqpoint{12.460045in}{1.209969in}}%
\pgfpathlineto{\pgfqpoint{12.460045in}{1.217057in}}%
\pgfpathlineto{\pgfqpoint{-319.224843in}{1.217057in}}%
\pgfpathclose%
\pgfusepath{fill}%
\end{pgfscope}%
\begin{pgfscope}%
\pgfpathrectangle{\pgfqpoint{10.668400in}{0.589870in}}{\pgfqpoint{2.188235in}{0.972632in}}%
\pgfusepath{clip}%
\pgfsetbuttcap%
\pgfsetmiterjoin%
\definecolor{currentfill}{rgb}{0.121569,0.466667,0.705882}%
\pgfsetfillcolor{currentfill}%
\pgfsetlinewidth{0.000000pt}%
\definecolor{currentstroke}{rgb}{0.000000,0.000000,0.000000}%
\pgfsetstrokecolor{currentstroke}%
\pgfsetstrokeopacity{0.000000}%
\pgfsetdash{}{0pt}%
\pgfpathmoveto{\pgfqpoint{-319.224843in}{1.218829in}}%
\pgfpathlineto{\pgfqpoint{12.670459in}{1.218829in}}%
\pgfpathlineto{\pgfqpoint{12.670459in}{1.225917in}}%
\pgfpathlineto{\pgfqpoint{-319.224843in}{1.225917in}}%
\pgfpathclose%
\pgfusepath{fill}%
\end{pgfscope}%
\begin{pgfscope}%
\pgfpathrectangle{\pgfqpoint{10.668400in}{0.589870in}}{\pgfqpoint{2.188235in}{0.972632in}}%
\pgfusepath{clip}%
\pgfsetbuttcap%
\pgfsetmiterjoin%
\definecolor{currentfill}{rgb}{0.121569,0.466667,0.705882}%
\pgfsetfillcolor{currentfill}%
\pgfsetlinewidth{0.000000pt}%
\definecolor{currentstroke}{rgb}{0.000000,0.000000,0.000000}%
\pgfsetstrokecolor{currentstroke}%
\pgfsetstrokeopacity{0.000000}%
\pgfsetdash{}{0pt}%
\pgfpathmoveto{\pgfqpoint{-319.224843in}{1.227689in}}%
\pgfpathlineto{\pgfqpoint{12.250596in}{1.227689in}}%
\pgfpathlineto{\pgfqpoint{12.250596in}{1.234777in}}%
\pgfpathlineto{\pgfqpoint{-319.224843in}{1.234777in}}%
\pgfpathclose%
\pgfusepath{fill}%
\end{pgfscope}%
\begin{pgfscope}%
\pgfpathrectangle{\pgfqpoint{10.668400in}{0.589870in}}{\pgfqpoint{2.188235in}{0.972632in}}%
\pgfusepath{clip}%
\pgfsetbuttcap%
\pgfsetmiterjoin%
\definecolor{currentfill}{rgb}{0.121569,0.466667,0.705882}%
\pgfsetfillcolor{currentfill}%
\pgfsetlinewidth{0.000000pt}%
\definecolor{currentstroke}{rgb}{0.000000,0.000000,0.000000}%
\pgfsetstrokecolor{currentstroke}%
\pgfsetstrokeopacity{0.000000}%
\pgfsetdash{}{0pt}%
\pgfpathmoveto{\pgfqpoint{-319.224843in}{1.236549in}}%
\pgfpathlineto{\pgfqpoint{12.368142in}{1.236549in}}%
\pgfpathlineto{\pgfqpoint{12.368142in}{1.243637in}}%
\pgfpathlineto{\pgfqpoint{-319.224843in}{1.243637in}}%
\pgfpathclose%
\pgfusepath{fill}%
\end{pgfscope}%
\begin{pgfscope}%
\pgfpathrectangle{\pgfqpoint{10.668400in}{0.589870in}}{\pgfqpoint{2.188235in}{0.972632in}}%
\pgfusepath{clip}%
\pgfsetbuttcap%
\pgfsetmiterjoin%
\definecolor{currentfill}{rgb}{0.121569,0.466667,0.705882}%
\pgfsetfillcolor{currentfill}%
\pgfsetlinewidth{0.000000pt}%
\definecolor{currentstroke}{rgb}{0.000000,0.000000,0.000000}%
\pgfsetstrokecolor{currentstroke}%
\pgfsetstrokeopacity{0.000000}%
\pgfsetdash{}{0pt}%
\pgfpathmoveto{\pgfqpoint{-319.224843in}{1.245409in}}%
\pgfpathlineto{\pgfqpoint{12.529395in}{1.245409in}}%
\pgfpathlineto{\pgfqpoint{12.529395in}{1.252497in}}%
\pgfpathlineto{\pgfqpoint{-319.224843in}{1.252497in}}%
\pgfpathclose%
\pgfusepath{fill}%
\end{pgfscope}%
\begin{pgfscope}%
\pgfpathrectangle{\pgfqpoint{10.668400in}{0.589870in}}{\pgfqpoint{2.188235in}{0.972632in}}%
\pgfusepath{clip}%
\pgfsetbuttcap%
\pgfsetmiterjoin%
\definecolor{currentfill}{rgb}{0.121569,0.466667,0.705882}%
\pgfsetfillcolor{currentfill}%
\pgfsetlinewidth{0.000000pt}%
\definecolor{currentstroke}{rgb}{0.000000,0.000000,0.000000}%
\pgfsetstrokecolor{currentstroke}%
\pgfsetstrokeopacity{0.000000}%
\pgfsetdash{}{0pt}%
\pgfpathmoveto{\pgfqpoint{-319.224843in}{1.254269in}}%
\pgfpathlineto{\pgfqpoint{12.475587in}{1.254269in}}%
\pgfpathlineto{\pgfqpoint{12.475587in}{1.261356in}}%
\pgfpathlineto{\pgfqpoint{-319.224843in}{1.261356in}}%
\pgfpathclose%
\pgfusepath{fill}%
\end{pgfscope}%
\begin{pgfscope}%
\pgfpathrectangle{\pgfqpoint{10.668400in}{0.589870in}}{\pgfqpoint{2.188235in}{0.972632in}}%
\pgfusepath{clip}%
\pgfsetbuttcap%
\pgfsetmiterjoin%
\definecolor{currentfill}{rgb}{0.121569,0.466667,0.705882}%
\pgfsetfillcolor{currentfill}%
\pgfsetlinewidth{0.000000pt}%
\definecolor{currentstroke}{rgb}{0.000000,0.000000,0.000000}%
\pgfsetstrokecolor{currentstroke}%
\pgfsetstrokeopacity{0.000000}%
\pgfsetdash{}{0pt}%
\pgfpathmoveto{\pgfqpoint{-319.224843in}{1.263128in}}%
\pgfpathlineto{\pgfqpoint{12.500476in}{1.263128in}}%
\pgfpathlineto{\pgfqpoint{12.500476in}{1.270216in}}%
\pgfpathlineto{\pgfqpoint{-319.224843in}{1.270216in}}%
\pgfpathclose%
\pgfusepath{fill}%
\end{pgfscope}%
\begin{pgfscope}%
\pgfpathrectangle{\pgfqpoint{10.668400in}{0.589870in}}{\pgfqpoint{2.188235in}{0.972632in}}%
\pgfusepath{clip}%
\pgfsetbuttcap%
\pgfsetmiterjoin%
\definecolor{currentfill}{rgb}{0.121569,0.466667,0.705882}%
\pgfsetfillcolor{currentfill}%
\pgfsetlinewidth{0.000000pt}%
\definecolor{currentstroke}{rgb}{0.000000,0.000000,0.000000}%
\pgfsetstrokecolor{currentstroke}%
\pgfsetstrokeopacity{0.000000}%
\pgfsetdash{}{0pt}%
\pgfpathmoveto{\pgfqpoint{-319.224843in}{1.271988in}}%
\pgfpathlineto{\pgfqpoint{12.509000in}{1.271988in}}%
\pgfpathlineto{\pgfqpoint{12.509000in}{1.279076in}}%
\pgfpathlineto{\pgfqpoint{-319.224843in}{1.279076in}}%
\pgfpathclose%
\pgfusepath{fill}%
\end{pgfscope}%
\begin{pgfscope}%
\pgfpathrectangle{\pgfqpoint{10.668400in}{0.589870in}}{\pgfqpoint{2.188235in}{0.972632in}}%
\pgfusepath{clip}%
\pgfsetbuttcap%
\pgfsetmiterjoin%
\definecolor{currentfill}{rgb}{0.121569,0.466667,0.705882}%
\pgfsetfillcolor{currentfill}%
\pgfsetlinewidth{0.000000pt}%
\definecolor{currentstroke}{rgb}{0.000000,0.000000,0.000000}%
\pgfsetstrokecolor{currentstroke}%
\pgfsetstrokeopacity{0.000000}%
\pgfsetdash{}{0pt}%
\pgfpathmoveto{\pgfqpoint{-319.224843in}{1.280848in}}%
\pgfpathlineto{\pgfqpoint{12.579357in}{1.280848in}}%
\pgfpathlineto{\pgfqpoint{12.579357in}{1.287936in}}%
\pgfpathlineto{\pgfqpoint{-319.224843in}{1.287936in}}%
\pgfpathclose%
\pgfusepath{fill}%
\end{pgfscope}%
\begin{pgfscope}%
\pgfpathrectangle{\pgfqpoint{10.668400in}{0.589870in}}{\pgfqpoint{2.188235in}{0.972632in}}%
\pgfusepath{clip}%
\pgfsetbuttcap%
\pgfsetmiterjoin%
\definecolor{currentfill}{rgb}{0.121569,0.466667,0.705882}%
\pgfsetfillcolor{currentfill}%
\pgfsetlinewidth{0.000000pt}%
\definecolor{currentstroke}{rgb}{0.000000,0.000000,0.000000}%
\pgfsetstrokecolor{currentstroke}%
\pgfsetstrokeopacity{0.000000}%
\pgfsetdash{}{0pt}%
\pgfpathmoveto{\pgfqpoint{-319.224843in}{1.289708in}}%
\pgfpathlineto{\pgfqpoint{12.547109in}{1.289708in}}%
\pgfpathlineto{\pgfqpoint{12.547109in}{1.296796in}}%
\pgfpathlineto{\pgfqpoint{-319.224843in}{1.296796in}}%
\pgfpathclose%
\pgfusepath{fill}%
\end{pgfscope}%
\begin{pgfscope}%
\pgfpathrectangle{\pgfqpoint{10.668400in}{0.589870in}}{\pgfqpoint{2.188235in}{0.972632in}}%
\pgfusepath{clip}%
\pgfsetbuttcap%
\pgfsetmiterjoin%
\definecolor{currentfill}{rgb}{0.121569,0.466667,0.705882}%
\pgfsetfillcolor{currentfill}%
\pgfsetlinewidth{0.000000pt}%
\definecolor{currentstroke}{rgb}{0.000000,0.000000,0.000000}%
\pgfsetstrokecolor{currentstroke}%
\pgfsetstrokeopacity{0.000000}%
\pgfsetdash{}{0pt}%
\pgfpathmoveto{\pgfqpoint{-319.224843in}{1.298568in}}%
\pgfpathlineto{\pgfqpoint{12.534056in}{1.298568in}}%
\pgfpathlineto{\pgfqpoint{12.534056in}{1.305656in}}%
\pgfpathlineto{\pgfqpoint{-319.224843in}{1.305656in}}%
\pgfpathclose%
\pgfusepath{fill}%
\end{pgfscope}%
\begin{pgfscope}%
\pgfpathrectangle{\pgfqpoint{10.668400in}{0.589870in}}{\pgfqpoint{2.188235in}{0.972632in}}%
\pgfusepath{clip}%
\pgfsetbuttcap%
\pgfsetmiterjoin%
\definecolor{currentfill}{rgb}{0.121569,0.466667,0.705882}%
\pgfsetfillcolor{currentfill}%
\pgfsetlinewidth{0.000000pt}%
\definecolor{currentstroke}{rgb}{0.000000,0.000000,0.000000}%
\pgfsetstrokecolor{currentstroke}%
\pgfsetstrokeopacity{0.000000}%
\pgfsetdash{}{0pt}%
\pgfpathmoveto{\pgfqpoint{-319.224843in}{1.307427in}}%
\pgfpathlineto{\pgfqpoint{12.757170in}{1.307427in}}%
\pgfpathlineto{\pgfqpoint{12.757170in}{1.314515in}}%
\pgfpathlineto{\pgfqpoint{-319.224843in}{1.314515in}}%
\pgfpathclose%
\pgfusepath{fill}%
\end{pgfscope}%
\begin{pgfscope}%
\pgfpathrectangle{\pgfqpoint{10.668400in}{0.589870in}}{\pgfqpoint{2.188235in}{0.972632in}}%
\pgfusepath{clip}%
\pgfsetbuttcap%
\pgfsetmiterjoin%
\definecolor{currentfill}{rgb}{0.121569,0.466667,0.705882}%
\pgfsetfillcolor{currentfill}%
\pgfsetlinewidth{0.000000pt}%
\definecolor{currentstroke}{rgb}{0.000000,0.000000,0.000000}%
\pgfsetstrokecolor{currentstroke}%
\pgfsetstrokeopacity{0.000000}%
\pgfsetdash{}{0pt}%
\pgfpathmoveto{\pgfqpoint{-319.224843in}{1.316287in}}%
\pgfpathlineto{\pgfqpoint{12.167601in}{1.316287in}}%
\pgfpathlineto{\pgfqpoint{12.167601in}{1.323375in}}%
\pgfpathlineto{\pgfqpoint{-319.224843in}{1.323375in}}%
\pgfpathclose%
\pgfusepath{fill}%
\end{pgfscope}%
\begin{pgfscope}%
\pgfpathrectangle{\pgfqpoint{10.668400in}{0.589870in}}{\pgfqpoint{2.188235in}{0.972632in}}%
\pgfusepath{clip}%
\pgfsetbuttcap%
\pgfsetmiterjoin%
\definecolor{currentfill}{rgb}{0.121569,0.466667,0.705882}%
\pgfsetfillcolor{currentfill}%
\pgfsetlinewidth{0.000000pt}%
\definecolor{currentstroke}{rgb}{0.000000,0.000000,0.000000}%
\pgfsetstrokecolor{currentstroke}%
\pgfsetstrokeopacity{0.000000}%
\pgfsetdash{}{0pt}%
\pgfpathmoveto{\pgfqpoint{-319.224843in}{1.325147in}}%
\pgfpathlineto{\pgfqpoint{11.903503in}{1.325147in}}%
\pgfpathlineto{\pgfqpoint{11.903503in}{1.332235in}}%
\pgfpathlineto{\pgfqpoint{-319.224843in}{1.332235in}}%
\pgfpathclose%
\pgfusepath{fill}%
\end{pgfscope}%
\begin{pgfscope}%
\pgfpathrectangle{\pgfqpoint{10.668400in}{0.589870in}}{\pgfqpoint{2.188235in}{0.972632in}}%
\pgfusepath{clip}%
\pgfsetbuttcap%
\pgfsetmiterjoin%
\definecolor{currentfill}{rgb}{0.121569,0.466667,0.705882}%
\pgfsetfillcolor{currentfill}%
\pgfsetlinewidth{0.000000pt}%
\definecolor{currentstroke}{rgb}{0.000000,0.000000,0.000000}%
\pgfsetstrokecolor{currentstroke}%
\pgfsetstrokeopacity{0.000000}%
\pgfsetdash{}{0pt}%
\pgfpathmoveto{\pgfqpoint{-319.224843in}{1.334007in}}%
\pgfpathlineto{\pgfqpoint{12.167527in}{1.334007in}}%
\pgfpathlineto{\pgfqpoint{12.167527in}{1.341095in}}%
\pgfpathlineto{\pgfqpoint{-319.224843in}{1.341095in}}%
\pgfpathclose%
\pgfusepath{fill}%
\end{pgfscope}%
\begin{pgfscope}%
\pgfpathrectangle{\pgfqpoint{10.668400in}{0.589870in}}{\pgfqpoint{2.188235in}{0.972632in}}%
\pgfusepath{clip}%
\pgfsetbuttcap%
\pgfsetmiterjoin%
\definecolor{currentfill}{rgb}{0.121569,0.466667,0.705882}%
\pgfsetfillcolor{currentfill}%
\pgfsetlinewidth{0.000000pt}%
\definecolor{currentstroke}{rgb}{0.000000,0.000000,0.000000}%
\pgfsetstrokecolor{currentstroke}%
\pgfsetstrokeopacity{0.000000}%
\pgfsetdash{}{0pt}%
\pgfpathmoveto{\pgfqpoint{-319.224843in}{1.342867in}}%
\pgfpathlineto{\pgfqpoint{12.541357in}{1.342867in}}%
\pgfpathlineto{\pgfqpoint{12.541357in}{1.349955in}}%
\pgfpathlineto{\pgfqpoint{-319.224843in}{1.349955in}}%
\pgfpathclose%
\pgfusepath{fill}%
\end{pgfscope}%
\begin{pgfscope}%
\pgfpathrectangle{\pgfqpoint{10.668400in}{0.589870in}}{\pgfqpoint{2.188235in}{0.972632in}}%
\pgfusepath{clip}%
\pgfsetbuttcap%
\pgfsetmiterjoin%
\definecolor{currentfill}{rgb}{0.121569,0.466667,0.705882}%
\pgfsetfillcolor{currentfill}%
\pgfsetlinewidth{0.000000pt}%
\definecolor{currentstroke}{rgb}{0.000000,0.000000,0.000000}%
\pgfsetstrokecolor{currentstroke}%
\pgfsetstrokeopacity{0.000000}%
\pgfsetdash{}{0pt}%
\pgfpathmoveto{\pgfqpoint{-319.224843in}{1.351727in}}%
\pgfpathlineto{\pgfqpoint{12.480622in}{1.351727in}}%
\pgfpathlineto{\pgfqpoint{12.480622in}{1.358814in}}%
\pgfpathlineto{\pgfqpoint{-319.224843in}{1.358814in}}%
\pgfpathclose%
\pgfusepath{fill}%
\end{pgfscope}%
\begin{pgfscope}%
\pgfpathrectangle{\pgfqpoint{10.668400in}{0.589870in}}{\pgfqpoint{2.188235in}{0.972632in}}%
\pgfusepath{clip}%
\pgfsetbuttcap%
\pgfsetmiterjoin%
\definecolor{currentfill}{rgb}{0.121569,0.466667,0.705882}%
\pgfsetfillcolor{currentfill}%
\pgfsetlinewidth{0.000000pt}%
\definecolor{currentstroke}{rgb}{0.000000,0.000000,0.000000}%
\pgfsetstrokecolor{currentstroke}%
\pgfsetstrokeopacity{0.000000}%
\pgfsetdash{}{0pt}%
\pgfpathmoveto{\pgfqpoint{-319.224843in}{1.360586in}}%
\pgfpathlineto{\pgfqpoint{12.410176in}{1.360586in}}%
\pgfpathlineto{\pgfqpoint{12.410176in}{1.367674in}}%
\pgfpathlineto{\pgfqpoint{-319.224843in}{1.367674in}}%
\pgfpathclose%
\pgfusepath{fill}%
\end{pgfscope}%
\begin{pgfscope}%
\pgfpathrectangle{\pgfqpoint{10.668400in}{0.589870in}}{\pgfqpoint{2.188235in}{0.972632in}}%
\pgfusepath{clip}%
\pgfsetbuttcap%
\pgfsetmiterjoin%
\definecolor{currentfill}{rgb}{0.121569,0.466667,0.705882}%
\pgfsetfillcolor{currentfill}%
\pgfsetlinewidth{0.000000pt}%
\definecolor{currentstroke}{rgb}{0.000000,0.000000,0.000000}%
\pgfsetstrokecolor{currentstroke}%
\pgfsetstrokeopacity{0.000000}%
\pgfsetdash{}{0pt}%
\pgfpathmoveto{\pgfqpoint{-319.224843in}{1.369446in}}%
\pgfpathlineto{\pgfqpoint{12.523367in}{1.369446in}}%
\pgfpathlineto{\pgfqpoint{12.523367in}{1.376534in}}%
\pgfpathlineto{\pgfqpoint{-319.224843in}{1.376534in}}%
\pgfpathclose%
\pgfusepath{fill}%
\end{pgfscope}%
\begin{pgfscope}%
\pgfpathrectangle{\pgfqpoint{10.668400in}{0.589870in}}{\pgfqpoint{2.188235in}{0.972632in}}%
\pgfusepath{clip}%
\pgfsetbuttcap%
\pgfsetmiterjoin%
\definecolor{currentfill}{rgb}{0.121569,0.466667,0.705882}%
\pgfsetfillcolor{currentfill}%
\pgfsetlinewidth{0.000000pt}%
\definecolor{currentstroke}{rgb}{0.000000,0.000000,0.000000}%
\pgfsetstrokecolor{currentstroke}%
\pgfsetstrokeopacity{0.000000}%
\pgfsetdash{}{0pt}%
\pgfpathmoveto{\pgfqpoint{-319.224843in}{1.378306in}}%
\pgfpathlineto{\pgfqpoint{12.456490in}{1.378306in}}%
\pgfpathlineto{\pgfqpoint{12.456490in}{1.385394in}}%
\pgfpathlineto{\pgfqpoint{-319.224843in}{1.385394in}}%
\pgfpathclose%
\pgfusepath{fill}%
\end{pgfscope}%
\begin{pgfscope}%
\pgfpathrectangle{\pgfqpoint{10.668400in}{0.589870in}}{\pgfqpoint{2.188235in}{0.972632in}}%
\pgfusepath{clip}%
\pgfsetbuttcap%
\pgfsetmiterjoin%
\definecolor{currentfill}{rgb}{0.121569,0.466667,0.705882}%
\pgfsetfillcolor{currentfill}%
\pgfsetlinewidth{0.000000pt}%
\definecolor{currentstroke}{rgb}{0.000000,0.000000,0.000000}%
\pgfsetstrokecolor{currentstroke}%
\pgfsetstrokeopacity{0.000000}%
\pgfsetdash{}{0pt}%
\pgfpathmoveto{\pgfqpoint{-319.224843in}{1.387166in}}%
\pgfpathlineto{\pgfqpoint{12.449356in}{1.387166in}}%
\pgfpathlineto{\pgfqpoint{12.449356in}{1.394254in}}%
\pgfpathlineto{\pgfqpoint{-319.224843in}{1.394254in}}%
\pgfpathclose%
\pgfusepath{fill}%
\end{pgfscope}%
\begin{pgfscope}%
\pgfpathrectangle{\pgfqpoint{10.668400in}{0.589870in}}{\pgfqpoint{2.188235in}{0.972632in}}%
\pgfusepath{clip}%
\pgfsetbuttcap%
\pgfsetmiterjoin%
\definecolor{currentfill}{rgb}{0.121569,0.466667,0.705882}%
\pgfsetfillcolor{currentfill}%
\pgfsetlinewidth{0.000000pt}%
\definecolor{currentstroke}{rgb}{0.000000,0.000000,0.000000}%
\pgfsetstrokecolor{currentstroke}%
\pgfsetstrokeopacity{0.000000}%
\pgfsetdash{}{0pt}%
\pgfpathmoveto{\pgfqpoint{-319.224843in}{1.396026in}}%
\pgfpathlineto{\pgfqpoint{12.492043in}{1.396026in}}%
\pgfpathlineto{\pgfqpoint{12.492043in}{1.403114in}}%
\pgfpathlineto{\pgfqpoint{-319.224843in}{1.403114in}}%
\pgfpathclose%
\pgfusepath{fill}%
\end{pgfscope}%
\begin{pgfscope}%
\pgfpathrectangle{\pgfqpoint{10.668400in}{0.589870in}}{\pgfqpoint{2.188235in}{0.972632in}}%
\pgfusepath{clip}%
\pgfsetbuttcap%
\pgfsetmiterjoin%
\definecolor{currentfill}{rgb}{0.121569,0.466667,0.705882}%
\pgfsetfillcolor{currentfill}%
\pgfsetlinewidth{0.000000pt}%
\definecolor{currentstroke}{rgb}{0.000000,0.000000,0.000000}%
\pgfsetstrokecolor{currentstroke}%
\pgfsetstrokeopacity{0.000000}%
\pgfsetdash{}{0pt}%
\pgfpathmoveto{\pgfqpoint{-319.224843in}{1.404886in}}%
\pgfpathlineto{\pgfqpoint{12.569105in}{1.404886in}}%
\pgfpathlineto{\pgfqpoint{12.569105in}{1.411973in}}%
\pgfpathlineto{\pgfqpoint{-319.224843in}{1.411973in}}%
\pgfpathclose%
\pgfusepath{fill}%
\end{pgfscope}%
\begin{pgfscope}%
\pgfpathrectangle{\pgfqpoint{10.668400in}{0.589870in}}{\pgfqpoint{2.188235in}{0.972632in}}%
\pgfusepath{clip}%
\pgfsetbuttcap%
\pgfsetmiterjoin%
\definecolor{currentfill}{rgb}{0.121569,0.466667,0.705882}%
\pgfsetfillcolor{currentfill}%
\pgfsetlinewidth{0.000000pt}%
\definecolor{currentstroke}{rgb}{0.000000,0.000000,0.000000}%
\pgfsetstrokecolor{currentstroke}%
\pgfsetstrokeopacity{0.000000}%
\pgfsetdash{}{0pt}%
\pgfpathmoveto{\pgfqpoint{-319.224843in}{1.413745in}}%
\pgfpathlineto{\pgfqpoint{12.572533in}{1.413745in}}%
\pgfpathlineto{\pgfqpoint{12.572533in}{1.420833in}}%
\pgfpathlineto{\pgfqpoint{-319.224843in}{1.420833in}}%
\pgfpathclose%
\pgfusepath{fill}%
\end{pgfscope}%
\begin{pgfscope}%
\pgfpathrectangle{\pgfqpoint{10.668400in}{0.589870in}}{\pgfqpoint{2.188235in}{0.972632in}}%
\pgfusepath{clip}%
\pgfsetbuttcap%
\pgfsetmiterjoin%
\definecolor{currentfill}{rgb}{0.121569,0.466667,0.705882}%
\pgfsetfillcolor{currentfill}%
\pgfsetlinewidth{0.000000pt}%
\definecolor{currentstroke}{rgb}{0.000000,0.000000,0.000000}%
\pgfsetstrokecolor{currentstroke}%
\pgfsetstrokeopacity{0.000000}%
\pgfsetdash{}{0pt}%
\pgfpathmoveto{\pgfqpoint{-319.224843in}{1.422605in}}%
\pgfpathlineto{\pgfqpoint{12.482571in}{1.422605in}}%
\pgfpathlineto{\pgfqpoint{12.482571in}{1.429693in}}%
\pgfpathlineto{\pgfqpoint{-319.224843in}{1.429693in}}%
\pgfpathclose%
\pgfusepath{fill}%
\end{pgfscope}%
\begin{pgfscope}%
\pgfpathrectangle{\pgfqpoint{10.668400in}{0.589870in}}{\pgfqpoint{2.188235in}{0.972632in}}%
\pgfusepath{clip}%
\pgfsetbuttcap%
\pgfsetmiterjoin%
\definecolor{currentfill}{rgb}{0.121569,0.466667,0.705882}%
\pgfsetfillcolor{currentfill}%
\pgfsetlinewidth{0.000000pt}%
\definecolor{currentstroke}{rgb}{0.000000,0.000000,0.000000}%
\pgfsetstrokecolor{currentstroke}%
\pgfsetstrokeopacity{0.000000}%
\pgfsetdash{}{0pt}%
\pgfpathmoveto{\pgfqpoint{-319.224843in}{1.431465in}}%
\pgfpathlineto{\pgfqpoint{12.565955in}{1.431465in}}%
\pgfpathlineto{\pgfqpoint{12.565955in}{1.438553in}}%
\pgfpathlineto{\pgfqpoint{-319.224843in}{1.438553in}}%
\pgfpathclose%
\pgfusepath{fill}%
\end{pgfscope}%
\begin{pgfscope}%
\pgfpathrectangle{\pgfqpoint{10.668400in}{0.589870in}}{\pgfqpoint{2.188235in}{0.972632in}}%
\pgfusepath{clip}%
\pgfsetbuttcap%
\pgfsetmiterjoin%
\definecolor{currentfill}{rgb}{0.121569,0.466667,0.705882}%
\pgfsetfillcolor{currentfill}%
\pgfsetlinewidth{0.000000pt}%
\definecolor{currentstroke}{rgb}{0.000000,0.000000,0.000000}%
\pgfsetstrokecolor{currentstroke}%
\pgfsetstrokeopacity{0.000000}%
\pgfsetdash{}{0pt}%
\pgfpathmoveto{\pgfqpoint{-319.224843in}{1.440325in}}%
\pgfpathlineto{\pgfqpoint{12.499409in}{1.440325in}}%
\pgfpathlineto{\pgfqpoint{12.499409in}{1.447413in}}%
\pgfpathlineto{\pgfqpoint{-319.224843in}{1.447413in}}%
\pgfpathclose%
\pgfusepath{fill}%
\end{pgfscope}%
\begin{pgfscope}%
\pgfpathrectangle{\pgfqpoint{10.668400in}{0.589870in}}{\pgfqpoint{2.188235in}{0.972632in}}%
\pgfusepath{clip}%
\pgfsetbuttcap%
\pgfsetmiterjoin%
\definecolor{currentfill}{rgb}{0.121569,0.466667,0.705882}%
\pgfsetfillcolor{currentfill}%
\pgfsetlinewidth{0.000000pt}%
\definecolor{currentstroke}{rgb}{0.000000,0.000000,0.000000}%
\pgfsetstrokecolor{currentstroke}%
\pgfsetstrokeopacity{0.000000}%
\pgfsetdash{}{0pt}%
\pgfpathmoveto{\pgfqpoint{-319.224843in}{1.449185in}}%
\pgfpathlineto{\pgfqpoint{12.530022in}{1.449185in}}%
\pgfpathlineto{\pgfqpoint{12.530022in}{1.456273in}}%
\pgfpathlineto{\pgfqpoint{-319.224843in}{1.456273in}}%
\pgfpathclose%
\pgfusepath{fill}%
\end{pgfscope}%
\begin{pgfscope}%
\pgfpathrectangle{\pgfqpoint{10.668400in}{0.589870in}}{\pgfqpoint{2.188235in}{0.972632in}}%
\pgfusepath{clip}%
\pgfsetbuttcap%
\pgfsetmiterjoin%
\definecolor{currentfill}{rgb}{0.121569,0.466667,0.705882}%
\pgfsetfillcolor{currentfill}%
\pgfsetlinewidth{0.000000pt}%
\definecolor{currentstroke}{rgb}{0.000000,0.000000,0.000000}%
\pgfsetstrokecolor{currentstroke}%
\pgfsetstrokeopacity{0.000000}%
\pgfsetdash{}{0pt}%
\pgfpathmoveto{\pgfqpoint{-319.224843in}{1.458045in}}%
\pgfpathlineto{\pgfqpoint{12.525716in}{1.458045in}}%
\pgfpathlineto{\pgfqpoint{12.525716in}{1.465132in}}%
\pgfpathlineto{\pgfqpoint{-319.224843in}{1.465132in}}%
\pgfpathclose%
\pgfusepath{fill}%
\end{pgfscope}%
\begin{pgfscope}%
\pgfpathrectangle{\pgfqpoint{10.668400in}{0.589870in}}{\pgfqpoint{2.188235in}{0.972632in}}%
\pgfusepath{clip}%
\pgfsetbuttcap%
\pgfsetmiterjoin%
\definecolor{currentfill}{rgb}{0.121569,0.466667,0.705882}%
\pgfsetfillcolor{currentfill}%
\pgfsetlinewidth{0.000000pt}%
\definecolor{currentstroke}{rgb}{0.000000,0.000000,0.000000}%
\pgfsetstrokecolor{currentstroke}%
\pgfsetstrokeopacity{0.000000}%
\pgfsetdash{}{0pt}%
\pgfpathmoveto{\pgfqpoint{-319.224843in}{1.466904in}}%
\pgfpathlineto{\pgfqpoint{12.552767in}{1.466904in}}%
\pgfpathlineto{\pgfqpoint{12.552767in}{1.473992in}}%
\pgfpathlineto{\pgfqpoint{-319.224843in}{1.473992in}}%
\pgfpathclose%
\pgfusepath{fill}%
\end{pgfscope}%
\begin{pgfscope}%
\pgfpathrectangle{\pgfqpoint{10.668400in}{0.589870in}}{\pgfqpoint{2.188235in}{0.972632in}}%
\pgfusepath{clip}%
\pgfsetbuttcap%
\pgfsetmiterjoin%
\definecolor{currentfill}{rgb}{0.121569,0.466667,0.705882}%
\pgfsetfillcolor{currentfill}%
\pgfsetlinewidth{0.000000pt}%
\definecolor{currentstroke}{rgb}{0.000000,0.000000,0.000000}%
\pgfsetstrokecolor{currentstroke}%
\pgfsetstrokeopacity{0.000000}%
\pgfsetdash{}{0pt}%
\pgfpathmoveto{\pgfqpoint{-319.224843in}{1.475764in}}%
\pgfpathlineto{\pgfqpoint{12.392402in}{1.475764in}}%
\pgfpathlineto{\pgfqpoint{12.392402in}{1.482852in}}%
\pgfpathlineto{\pgfqpoint{-319.224843in}{1.482852in}}%
\pgfpathclose%
\pgfusepath{fill}%
\end{pgfscope}%
\begin{pgfscope}%
\pgfpathrectangle{\pgfqpoint{10.668400in}{0.589870in}}{\pgfqpoint{2.188235in}{0.972632in}}%
\pgfusepath{clip}%
\pgfsetbuttcap%
\pgfsetmiterjoin%
\definecolor{currentfill}{rgb}{0.121569,0.466667,0.705882}%
\pgfsetfillcolor{currentfill}%
\pgfsetlinewidth{0.000000pt}%
\definecolor{currentstroke}{rgb}{0.000000,0.000000,0.000000}%
\pgfsetstrokecolor{currentstroke}%
\pgfsetstrokeopacity{0.000000}%
\pgfsetdash{}{0pt}%
\pgfpathmoveto{\pgfqpoint{-319.224843in}{1.484624in}}%
\pgfpathlineto{\pgfqpoint{11.867571in}{1.484624in}}%
\pgfpathlineto{\pgfqpoint{11.867571in}{1.491712in}}%
\pgfpathlineto{\pgfqpoint{-319.224843in}{1.491712in}}%
\pgfpathclose%
\pgfusepath{fill}%
\end{pgfscope}%
\begin{pgfscope}%
\pgfpathrectangle{\pgfqpoint{10.668400in}{0.589870in}}{\pgfqpoint{2.188235in}{0.972632in}}%
\pgfusepath{clip}%
\pgfsetbuttcap%
\pgfsetmiterjoin%
\definecolor{currentfill}{rgb}{0.121569,0.466667,0.705882}%
\pgfsetfillcolor{currentfill}%
\pgfsetlinewidth{0.000000pt}%
\definecolor{currentstroke}{rgb}{0.000000,0.000000,0.000000}%
\pgfsetstrokecolor{currentstroke}%
\pgfsetstrokeopacity{0.000000}%
\pgfsetdash{}{0pt}%
\pgfpathmoveto{\pgfqpoint{-319.224843in}{1.493484in}}%
\pgfpathlineto{\pgfqpoint{12.519193in}{1.493484in}}%
\pgfpathlineto{\pgfqpoint{12.519193in}{1.500572in}}%
\pgfpathlineto{\pgfqpoint{-319.224843in}{1.500572in}}%
\pgfpathclose%
\pgfusepath{fill}%
\end{pgfscope}%
\begin{pgfscope}%
\pgfpathrectangle{\pgfqpoint{10.668400in}{0.589870in}}{\pgfqpoint{2.188235in}{0.972632in}}%
\pgfusepath{clip}%
\pgfsetbuttcap%
\pgfsetmiterjoin%
\definecolor{currentfill}{rgb}{0.121569,0.466667,0.705882}%
\pgfsetfillcolor{currentfill}%
\pgfsetlinewidth{0.000000pt}%
\definecolor{currentstroke}{rgb}{0.000000,0.000000,0.000000}%
\pgfsetstrokecolor{currentstroke}%
\pgfsetstrokeopacity{0.000000}%
\pgfsetdash{}{0pt}%
\pgfpathmoveto{\pgfqpoint{-319.224843in}{1.502344in}}%
\pgfpathlineto{\pgfqpoint{12.351408in}{1.502344in}}%
\pgfpathlineto{\pgfqpoint{12.351408in}{1.509432in}}%
\pgfpathlineto{\pgfqpoint{-319.224843in}{1.509432in}}%
\pgfpathclose%
\pgfusepath{fill}%
\end{pgfscope}%
\begin{pgfscope}%
\pgfpathrectangle{\pgfqpoint{10.668400in}{0.589870in}}{\pgfqpoint{2.188235in}{0.972632in}}%
\pgfusepath{clip}%
\pgfsetbuttcap%
\pgfsetmiterjoin%
\definecolor{currentfill}{rgb}{0.121569,0.466667,0.705882}%
\pgfsetfillcolor{currentfill}%
\pgfsetlinewidth{0.000000pt}%
\definecolor{currentstroke}{rgb}{0.000000,0.000000,0.000000}%
\pgfsetstrokecolor{currentstroke}%
\pgfsetstrokeopacity{0.000000}%
\pgfsetdash{}{0pt}%
\pgfpathmoveto{\pgfqpoint{-319.224843in}{1.511203in}}%
\pgfpathlineto{\pgfqpoint{12.547406in}{1.511203in}}%
\pgfpathlineto{\pgfqpoint{12.547406in}{1.518291in}}%
\pgfpathlineto{\pgfqpoint{-319.224843in}{1.518291in}}%
\pgfpathclose%
\pgfusepath{fill}%
\end{pgfscope}%
\begin{pgfscope}%
\pgfsetbuttcap%
\pgfsetroundjoin%
\definecolor{currentfill}{rgb}{0.000000,0.000000,0.000000}%
\pgfsetfillcolor{currentfill}%
\pgfsetlinewidth{0.803000pt}%
\definecolor{currentstroke}{rgb}{0.000000,0.000000,0.000000}%
\pgfsetstrokecolor{currentstroke}%
\pgfsetdash{}{0pt}%
\pgfsys@defobject{currentmarker}{\pgfqpoint{0.000000in}{-0.048611in}}{\pgfqpoint{0.000000in}{0.000000in}}{%
\pgfpathmoveto{\pgfqpoint{0.000000in}{0.000000in}}%
\pgfpathlineto{\pgfqpoint{0.000000in}{-0.048611in}}%
\pgfusepath{stroke,fill}%
}%
\begin{pgfscope}%
\pgfsys@transformshift{11.218880in}{0.589870in}%
\pgfsys@useobject{currentmarker}{}%
\end{pgfscope}%
\end{pgfscope}%
\begin{pgfscope}%
\pgftext[x=11.218880in,y=0.492648in,,top]{\rmfamily\fontsize{10.000000}{12.000000}\selectfont \(\displaystyle 10^{-3}\)}%
\end{pgfscope}%
\begin{pgfscope}%
\pgfsetbuttcap%
\pgfsetroundjoin%
\definecolor{currentfill}{rgb}{0.000000,0.000000,0.000000}%
\pgfsetfillcolor{currentfill}%
\pgfsetlinewidth{0.803000pt}%
\definecolor{currentstroke}{rgb}{0.000000,0.000000,0.000000}%
\pgfsetstrokecolor{currentstroke}%
\pgfsetdash{}{0pt}%
\pgfsys@defobject{currentmarker}{\pgfqpoint{0.000000in}{-0.048611in}}{\pgfqpoint{0.000000in}{0.000000in}}{%
\pgfpathmoveto{\pgfqpoint{0.000000in}{0.000000in}}%
\pgfpathlineto{\pgfqpoint{0.000000in}{-0.048611in}}%
\pgfusepath{stroke,fill}%
}%
\begin{pgfscope}%
\pgfsys@transformshift{11.881756in}{0.589870in}%
\pgfsys@useobject{currentmarker}{}%
\end{pgfscope}%
\end{pgfscope}%
\begin{pgfscope}%
\pgftext[x=11.881756in,y=0.492648in,,top]{\rmfamily\fontsize{10.000000}{12.000000}\selectfont \(\displaystyle 10^{-1}\)}%
\end{pgfscope}%
\begin{pgfscope}%
\pgfsetbuttcap%
\pgfsetroundjoin%
\definecolor{currentfill}{rgb}{0.000000,0.000000,0.000000}%
\pgfsetfillcolor{currentfill}%
\pgfsetlinewidth{0.803000pt}%
\definecolor{currentstroke}{rgb}{0.000000,0.000000,0.000000}%
\pgfsetstrokecolor{currentstroke}%
\pgfsetdash{}{0pt}%
\pgfsys@defobject{currentmarker}{\pgfqpoint{0.000000in}{-0.048611in}}{\pgfqpoint{0.000000in}{0.000000in}}{%
\pgfpathmoveto{\pgfqpoint{0.000000in}{0.000000in}}%
\pgfpathlineto{\pgfqpoint{0.000000in}{-0.048611in}}%
\pgfusepath{stroke,fill}%
}%
\begin{pgfscope}%
\pgfsys@transformshift{12.544632in}{0.589870in}%
\pgfsys@useobject{currentmarker}{}%
\end{pgfscope}%
\end{pgfscope}%
\begin{pgfscope}%
\pgftext[x=12.544632in,y=0.492648in,,top]{\rmfamily\fontsize{10.000000}{12.000000}\selectfont \(\displaystyle 10^{1}\)}%
\end{pgfscope}%
\begin{pgfscope}%
\pgftext[x=11.762518in,y=0.302680in,,top]{\rmfamily\fontsize{10.000000}{12.000000}\selectfont \(\displaystyle |\theta^{\parallel}_j|\), \% of relative error}%
\end{pgfscope}%
\begin{pgfscope}%
\pgfsetbuttcap%
\pgfsetroundjoin%
\definecolor{currentfill}{rgb}{0.000000,0.000000,0.000000}%
\pgfsetfillcolor{currentfill}%
\pgfsetlinewidth{0.803000pt}%
\definecolor{currentstroke}{rgb}{0.000000,0.000000,0.000000}%
\pgfsetstrokecolor{currentstroke}%
\pgfsetdash{}{0pt}%
\pgfsys@defobject{currentmarker}{\pgfqpoint{-0.048611in}{0.000000in}}{\pgfqpoint{0.000000in}{0.000000in}}{%
\pgfpathmoveto{\pgfqpoint{0.000000in}{0.000000in}}%
\pgfpathlineto{\pgfqpoint{-0.048611in}{0.000000in}}%
\pgfusepath{stroke,fill}%
}%
\begin{pgfscope}%
\pgfsys@transformshift{10.668400in}{0.637625in}%
\pgfsys@useobject{currentmarker}{}%
\end{pgfscope}%
\end{pgfscope}%
\begin{pgfscope}%
\pgftext[x=10.501733in,y=0.584863in,left,base]{\rmfamily\fontsize{10.000000}{12.000000}\selectfont \(\displaystyle 0\)}%
\end{pgfscope}%
\begin{pgfscope}%
\pgfsetbuttcap%
\pgfsetroundjoin%
\definecolor{currentfill}{rgb}{0.000000,0.000000,0.000000}%
\pgfsetfillcolor{currentfill}%
\pgfsetlinewidth{0.803000pt}%
\definecolor{currentstroke}{rgb}{0.000000,0.000000,0.000000}%
\pgfsetstrokecolor{currentstroke}%
\pgfsetdash{}{0pt}%
\pgfsys@defobject{currentmarker}{\pgfqpoint{-0.048611in}{0.000000in}}{\pgfqpoint{0.000000in}{0.000000in}}{%
\pgfpathmoveto{\pgfqpoint{0.000000in}{0.000000in}}%
\pgfpathlineto{\pgfqpoint{-0.048611in}{0.000000in}}%
\pgfusepath{stroke,fill}%
}%
\begin{pgfscope}%
\pgfsys@transformshift{10.668400in}{1.080616in}%
\pgfsys@useobject{currentmarker}{}%
\end{pgfscope}%
\end{pgfscope}%
\begin{pgfscope}%
\pgftext[x=10.432288in,y=1.027854in,left,base]{\rmfamily\fontsize{10.000000}{12.000000}\selectfont \(\displaystyle 50\)}%
\end{pgfscope}%
\begin{pgfscope}%
\pgfsetbuttcap%
\pgfsetroundjoin%
\definecolor{currentfill}{rgb}{0.000000,0.000000,0.000000}%
\pgfsetfillcolor{currentfill}%
\pgfsetlinewidth{0.803000pt}%
\definecolor{currentstroke}{rgb}{0.000000,0.000000,0.000000}%
\pgfsetstrokecolor{currentstroke}%
\pgfsetdash{}{0pt}%
\pgfsys@defobject{currentmarker}{\pgfqpoint{-0.048611in}{0.000000in}}{\pgfqpoint{0.000000in}{0.000000in}}{%
\pgfpathmoveto{\pgfqpoint{0.000000in}{0.000000in}}%
\pgfpathlineto{\pgfqpoint{-0.048611in}{0.000000in}}%
\pgfusepath{stroke,fill}%
}%
\begin{pgfscope}%
\pgfsys@transformshift{10.668400in}{1.523607in}%
\pgfsys@useobject{currentmarker}{}%
\end{pgfscope}%
\end{pgfscope}%
\begin{pgfscope}%
\pgftext[x=10.362844in,y=1.470846in,left,base]{\rmfamily\fontsize{10.000000}{12.000000}\selectfont \(\displaystyle 100\)}%
\end{pgfscope}%
\begin{pgfscope}%
\pgftext[x=10.307288in,y=1.076186in,,bottom,rotate=90.000000]{\rmfamily\fontsize{10.000000}{12.000000}\selectfont \(\displaystyle j\)}%
\end{pgfscope}%
\begin{pgfscope}%
\pgfsetrectcap%
\pgfsetmiterjoin%
\pgfsetlinewidth{0.803000pt}%
\definecolor{currentstroke}{rgb}{0.000000,0.000000,0.000000}%
\pgfsetstrokecolor{currentstroke}%
\pgfsetdash{}{0pt}%
\pgfpathmoveto{\pgfqpoint{10.668400in}{0.589870in}}%
\pgfpathlineto{\pgfqpoint{10.668400in}{1.562502in}}%
\pgfusepath{stroke}%
\end{pgfscope}%
\begin{pgfscope}%
\pgfsetrectcap%
\pgfsetmiterjoin%
\pgfsetlinewidth{0.803000pt}%
\definecolor{currentstroke}{rgb}{0.000000,0.000000,0.000000}%
\pgfsetstrokecolor{currentstroke}%
\pgfsetdash{}{0pt}%
\pgfpathmoveto{\pgfqpoint{12.856635in}{0.589870in}}%
\pgfpathlineto{\pgfqpoint{12.856635in}{1.562502in}}%
\pgfusepath{stroke}%
\end{pgfscope}%
\begin{pgfscope}%
\pgfsetrectcap%
\pgfsetmiterjoin%
\pgfsetlinewidth{0.803000pt}%
\definecolor{currentstroke}{rgb}{0.000000,0.000000,0.000000}%
\pgfsetstrokecolor{currentstroke}%
\pgfsetdash{}{0pt}%
\pgfpathmoveto{\pgfqpoint{10.668400in}{0.589870in}}%
\pgfpathlineto{\pgfqpoint{12.856635in}{0.589870in}}%
\pgfusepath{stroke}%
\end{pgfscope}%
\begin{pgfscope}%
\pgfsetrectcap%
\pgfsetmiterjoin%
\pgfsetlinewidth{0.803000pt}%
\definecolor{currentstroke}{rgb}{0.000000,0.000000,0.000000}%
\pgfsetstrokecolor{currentstroke}%
\pgfsetdash{}{0pt}%
\pgfpathmoveto{\pgfqpoint{10.668400in}{1.562502in}}%
\pgfpathlineto{\pgfqpoint{12.856635in}{1.562502in}}%
\pgfusepath{stroke}%
\end{pgfscope}%
\end{pgfpicture}%
\makeatother%
\endgroup%
}')}
        \caption[\acs{ORFF} equivalence theorem.]{\acs{ORFF} equivalence
        theorem. \label{fig:representer}}
    \end{figure}
    %\clearpage
    %\begin{figure}
        %\pyc{print(r'\centering\resizebox{1.5\textwidth}{!}{\input{./representer2.pgf}}')}
        %\caption[\acs{ORFF} equivalence theorem with overfitting.]{\acs{ORFF}
        %equivalence theorem with overfitting. \label{fig:representer2}}
    %\end{figure}
\end{landscape}}

\subsection{Solving ORFF-based regression}\label{subsec:gradient_methods}
% We illustrate the ORFF representer theorem (\cref{cr:orff_representer}) on
% two experiment involving scalar valued kernels.
In order to find a solution to \cref{eq:argmin_applied}, we turn our attention
to gradient descent methods. We define an algorithm (\cref{alg:close_form}) to
find efficiently a solution to \cref{eq:argmin_applied} when
$c(y,y')=\norm{y=y'}_{\mathcal{Y}}^2$ and study its complexity.
\subsubsection{Gradient methods}\label{subsec:gradient_methods}
Since the solution of \cref{eq:argmin_applied} is unique when $\lambda>0$, a
sufficient and necessary condition is that the gradient of
$\mathfrak{R}_{\lambda}$ at the minimizer $\theta_{\seq{s}}$ is zero. We use
the Frechet derivative, the strongest notion of derivative in Banach
spaces~\citep{conway2013course, kurdila2006convex} which directly generalizes
the notion of gradient to Banach spaces.
%A function $f:\mathcal{H}_0\to\mathcal{H}_1$ is call Frechet
%differentiable at $\theta_0\in \mathcal{H}_0$ if there exist a bounded linear
%operator $A\in\mathcal{L}(\mathcal{H}_0,\mathcal{H}_1)$ such that
%\begin{dmath*}
    %\lim_{\norm{h}_{\mathcal{H}_0}\to 0} \frac{\norm{f(\theta_0 + h) -
    %f(\theta_0) - Ah}_{\mathcal{H}_1}}{\norm{h}_{\mathcal{H}_0}} = 0
%\end{dmath*}
%We write
%\begin{dmath*}
    %(D_Ff)(\theta_0)
    %\hiderel{=}\derivativeat{f(\theta)}{\theta}{\theta_0}
    %\hiderel{=}A
%\end{dmath*}
%and call it Frechet derivative of $f$ with respect to $\theta$ at $\theta_0$.
%With mild abuse of notation we write
%\begin{dmath*}
    %\derivativeat{f(\theta)}{\theta}{\theta_0}
    %=\derivative{f(\theta_0)}{\theta_0}.
%\end{dmath*}
The chain rule is valid in this context \cite[theorem 4.1.1 page
140]{kurdila2006convex}. Hence
%Namely, let $\mathcal{H}_0$, $\mathcal{H}_1$ and
%$\mathcal{H}_2$ be three Hilbert spaces. If a function
%$f:\mathcal{H}_0\to\mathcal{H}_1$ is Frechet differentiable at $\theta$ and
%$g:\mathcal{H}_1\to \mathcal{H}_2$ is Frechet differentiable at $f(\theta)$
%then $g\circ f$ is Frechet differentiable at $\theta$ and for all
%$h\in\mathcal{H}_0$
%\begin{dmath*}
    %\lderivative{(g\circ f)(\theta)}{\theta}\circ h
    %=\derivative{g(f(\theta))}{f(\theta)} \circ
    %\derivative{f(\theta)}{\theta}\circ h,
%\end{dmath*}
%or equivalently,
%\begin{dmath*}
    %D_F(g\circ f)(\theta)\circ h
    %= (D_Fg)(f(\theta)) \circ (D_Ff)(\theta)\circ h.
%\end{dmath*}
%If $f:\mathcal{H}\to\mathbb{R}$ then $(D_F f)(\theta_0)
%\in\mathcal{H}^\adjoint$ for all $\theta_0\in\mathcal{H}$, and by Riesz's
%representation theorem we define the gradient of $f$ noted $\nabla_{\theta}
%f(\theta)\in\mathcal{H}$ as the the vector in $\mathcal{H}$ such that
%\begin{dmath*}
    %\inner{\nabla_{\theta} f(\theta), h}_{\mathcal{H}} = (D_Ff)(\theta)\circ h
    %\hiderel{=} \derivative{f(\theta)}{\theta} \circ h.
%\end{dmath*}
%For a function $f:\mathcal{H}_0\to\mathcal{H}_1$ we note the jacobian of $f$ as
%$\jacobian_{\theta} f(\theta) = \derivative{f(\theta)}{\theta}$. In this
%context if $f:\mathcal{H}_0\to\mathcal{H}_1$ and $g:\mathcal{H}_1\to\mathbb{R}$
%the chain rule reads for all $h\in\mathcal{H}_0$
%\begin{dmath*}
    %\lderivative{(g\circ f)(\theta)}{\theta} \circ h
    %= \derivative{g(f(\theta))}{f(\theta)} \circ \jacobian_{\theta}f(\theta)
    %\circ h.
%\end{dmath*}
%By Riesz's representation theorem,
%\begin{dmath*}
    %\inner{\nabla_\theta(g\circ f)(\theta), h}_{\mathcal{H}_0}
    %= \inner{\nabla_{f(\theta)}g(f(\theta)) ,
    %\jacobian_{\theta}f(\theta)h}_{\mathcal{H}_0}
    %= \inner{\left( \jacobian_{\theta} f(\theta) \right)^\adjoint
    %\nabla_{f(\theta)} g(f(\theta)), h}_{\mathcal{H}_0}
%\end{dmath*}
%Hence
%\begin{dmath*}
    %\nabla_{\theta}(g\circ f)(\theta) =
    %\left(\jacobian_{\theta}f(\theta)\right)^\adjoint
    %\nabla_{f(\theta)}g(f(\theta)).
%\end{dmath*}
\begin{dmath*}
    \nabla_{\theta}c\left(\tildePhi{\omega}(x_i)^\adjoint \theta,
    y_i\right)= \tildePhi{\omega}(x_i)
    \left(\lderivativeat{c\left(y,
    y_i\right)}{y}{\tildePhi{\omega}(x_i)^\adjoint
    \theta}\right)^\adjoint, \text{~and }
    \nabla_{\theta}\norm{\theta}^2_{\tildeH{\omega}}\hiderel{=}2\theta.
\end{dmath*}
Provided that $c(y,y_i)$ is Frechet differentiable \acs{wrt}~$y$, for all $y$
and $y_i\in\mathcal{Y}$ we have $\nabla_{\theta} \mathfrak{R}_{\lambda}(\theta,
\seq{s}) \in \tildeH{\omega}$ and
\begin{dmath}
    \label{eq:grad_final}
    \nabla_{\theta} \mathfrak{R}_{\lambda}(\theta, \seq{s}) =
    \frac{1}{N}\sum_{i=1}^N \tildePhi{\omega}(x_i)
    \left(\lderivativeat{c\left(y,
    y_i\right)}{y}{\tildePhi{\omega}(x_i)^\adjoint \theta}\right)^\adjoint +
    \lambda\theta
\end{dmath}
\begin{example}[Naive closed form for the squared error cost]
    Consider the cost function defined for all $y$, $y'\in\mathcal{Y}$ by
    $c(y,y')=\frac{1}{2}\norm{y-y}_{\mathcal{Y}}^2$. Then
    $\left(\lderivativeat{c\left(y,
    y_i\right)}{y}{\tildePhi{\omega}(x_i)^\adjoint \theta}\right)^\adjoint =
    \left(\tildePhi{\omega}(x_i)^\adjoint \theta-y_i\right)$.  Thus, since the
    optimal solution $\theta_{\seq{s}}$ verifies $\nabla_{\theta_{\seq{s}}}
    \mathfrak{R}_{\lambda}(\theta_{\seq{s}}, \seq{s}) = 0$ we have
    $\frac{1}{N}\sum_{i=1}^N
    \tildePhi{\omega}(x_i)\left(\tildePhi{\omega}(x_i)^\adjoint
    \theta_{\seq{s}}-y_i\right) + \lambda \theta_{\seq{s}} = 0$.  Therefore,
    \begin{dmath}
        \label{eq:iff_solution} \left(\frac{1}{N}\sum_{i=1}^N
        \tildePhi{\omega}(x_i) \tildePhi{\omega}(x_i)^\adjoint +
        \lambda I_{\tildeH{\omega}}\right) \theta_{\seq{s}}
        = \frac{1}{N}\sum_{i=1}^N \tildePhi{\omega}(x_i) y_i.
    \end{dmath}
    Suppose that $\mathcal{Y}\subseteq\mathbb{R}^p$, and for all
    $x\in\mathcal{X}$, $\tildePhi{\omega}(x): \mathbb{R}^{r}\to\mathbb{R}^p$
    where all spaces are endowed with the Euclidean inner product. From this we
    can derive \cref{alg:close_form} which returns the closed form solution of
    \cref{eq:argmin_applied} for $c(y,y')=\frac{1}{2}\norm{y-y'}_2^2$.
\end{example}
\subsubsection{Complexity analysis}
\label{subsec:complexity}
\Cref{alg:close_form} constitutes our first step toward large-scale learning
with \aclp{OVK}. We can easily compute the time complexity of
\cref{alg:close_form} when all the operators act on finite dimensional Hilbert
spaces. Suppose that $p=\dim(\mathcal{Y})<\infty$ and for all
$x\in\mathcal{X}$, $\tildePhi{\omega}(x):\mathcal{Y}\to\tildeH{\omega}$ where
$r=\dim(\tildeH{\omega})<\infty$ is the dimension of the redescription space
$\tildeH{\omega}=\mathbb{R}^{r}$. Since $p$ and $r<\infty$, we view the
operators $\tildePhi{\omega}(x)$ and $I_{\tildeH{\omega}}$ as matrices.  Step 1
costs $O_t(Nr^2p)$. Steps 2 costs $O_t(Nrp)$. For step 3, the naive inversion
of the operator costs $O_t(r^3)$. Eventually the overall complexity of
\cref{alg:close_form} is $O_t\left(r^2(Np + r)\right)$, while the space
complexity is $O_s(r^2)$.
\begin{center}
    \begin{algorithm2e}[t!]
        \label{alg:close_form}
        \SetAlgoLined
        \Input{\begin{itemize}
            \item $\seq{s}=(x_i,
            y_i)_{i=1}^N\in\left(\mathcal{X}\times\mathbb{R}^p\right)^N$ a
            sequence of supervised training points,
            \item $\tildePhi{\omega}(x_i) \in \mathcal{L}\left(\mathbb{R}^p,
            \mathbb{R}^{r}\right)$ a feature map defined for all
            $x_i\in\mathcal{X}$,
            \item $\lambda \in\mathbb{R}_{>0}$ the
            Tychonov regularization term,
        \end{itemize}}
        \Output{A model $h:\mathcal{X} \to \mathbb{R}^p$,
        $h(x)=\tildePhi{\omega}(x)^\transpose \theta_{\seq{s}}$.  such that
        $\theta_{\seq{s}}$ minimize \cref{eq:argmin_applied}, where
        $c(y,y')=\norm{y-y'}_2^2$ and $\mathbb{R}^r$ and $\mathbb{R}^p$} 
        $\mathbf{P} \gets \frac{1}{N}\sum_{i=1}^N
        \tildePhi{\omega}(x_i) \tildePhi{\omega}(x_i)^\transpose
        \in\mathcal{L}(\mathbb{R}^{r}, \mathbb{R}^{r})  $\;
        $\mathbf{Y} \gets \frac{1}{N}\sum_{i=1}^N
        \tildePhi{\omega}(x_i)  y_i \in \mathbb{R}^{r} $\;
        $\theta_{\seq{s}} \gets \text{solve}_{\theta}\left((\mathbf{P} +
        \lambda I_r)\theta = \mathbf{Y} \right)$ \;
        \Return $h: x \mapsto \tildePhi{\omega}(x)^\transpose
        \theta_{\seq{s}}$\;
        \caption{Naive closed form for the squared error cost.}
    \end{algorithm2e}
\end{center}
This complexity is to compare with the kernelized solution. Let
\begin{dmath*}
    \mathbf{K}:
    \begin{cases}
        \mathcal{Y}^{N} \to \mathcal{Y}^{N} \\
        u\mapsto\Vect_{i=1}^{N+U}\sum_{j=1}^{N+U}K(x_i, x_j)u_j
    \end{cases}
\end{dmath*}
When $\mathcal{Y}=\mathbb{R}$,
\begin{dmath*}
    \mathbf{K}=
    \begin{pmatrix} K(x_1, x_1) & \hdots & K(x_1, x_{N+U}) \\ \vdots
        & \ddots & \vdots \\  K(x_{N+U}, x_1) & \hdots & K(x_{N+U}, x_{N+U})
    \end{pmatrix}
\end{dmath*}
is called the Gram matrix of $K$. When $\mathcal{Y}=\mathbb{R}^p$, $\mathbf{K}$
is a matrix-valued Gram matrix of size $pN\times pN$ where each entry
$\mathbf{K}_{ij}\in\mathcal{M}_{p,p}(\mathbb{R})$. Then the equivalent
kernelized solution $u_{\seq{s}}$ of \cref{th:representer} is
\begin{dmath*}
    \left(\frac{1}{N} \mathbf{K}  + \lambda
    I_{\Vect_{i=1}^{N}\mathcal{Y}}\right)u_{\seq{s}}=\Vect_{i=1}^N y_i.
\end{dmath*}
which has time complexity $O_t\left(N^3p^3\right)$ and space complexity
$O_s\left(N^2p^2\right)$. Suppose we are given a generic \acs{ORFF} map (see
\cref{subsec:examples_ORFF}). Then $r=2Dp$, where $D$ is the number of samples.
Hence \cref{alg:close_form} is better that its kernelized counterpart when
$r=2Dp$ is small compared to $Np$. Thus, roughly speaking it is better to use
\cref{alg:close_form} when the number of features, $r$, required is small
compared to the number of training points. Notice that \cref{alg:close_form}
has a linear complexity with respect to the number of supervised training
points $N$ so it is better suited to large scale learning provided that $D$
does not grows linearly with $N$.  Yet naive learning with
\cref{alg:close_form} by viewing all the operators as matrices is still
problematic. Indeed learning $p$ independent models with scalar Random Fourier
Features would cost $O_t\left(D^2p^3(N + D)\right)$ since $r=2Dp$. This Means
that learning vector-valued function has increased the (expected) complexity
from $p$ to $p^3$. However in some cases we can drastically reduce the
complexity by viewing the feature-maps as linear operators rather than
matrices.

%%%%%%%%%%%%%%%%%%%%%%%%%%%%%%%%%%%%%%%%%%%%%%%%%%%%%%%%%%%%%%%%%%%%%%%%%%%%%%%

\subsection{Efficient learning with ORFF}
\label{subsec:efficient_learning}
When developping \cref{alg:close_form} we considered that the feature map
$\tildePhi{\omega}(x)$ was a matrix from $\mathbb{R}^p$ to $\mathbb{R}^{r}$ for
all $x\in\mathcal{X}$, and therefore that computing
$\tildePhi{\omega}(x)\tildePhi{\omega}(z)^\transpose$ has a time complexity of
$O(r^2p)$.  While this holds true in the most generic senario, in many cases
the feature maps present some structure or sparsity allowing to reduce the
computational cost of evaluating the feature map. We focus on the \acl{ORFF}
given by \cref{alg:ORFF_construction}, developped in \cref{sec:building_ORFF}
and \cref{subsec:examples_ORFF} and treat the decomposable kernel, the
curl-free kernel and the divergence-free kernel as an example. We recall that
if $\mathcal{Y}'=\mathbb{R}^{p'}$ and $\mathcal{Y}=\mathbb{R}^p$, then
$\tildeH{\omega}=\mathbb{R}^{2Dp'}$ thus the \aclp{ORFF} given in
\cref{sec:ORFF_construction} have the form
\begin{dmath*}
    \begin{cases}
        \tildePhi{\omega}(x) \in\mathcal{L}\left(\mathbb{R}^p,
        \mathbb{R}^{2Dp'}\right) &: y \mapsto
        \frac{1}{\sqrt{D}}\Vect_{j=1}^D\pairing{x,
        \omega_j}B(\omega_j)^\transpose  y \\ \tildePhi{\omega}(x)^\transpose
        \in\mathcal{L}\left(\mathbb{R}^{2Dp'}, \mathbb{R}^p\right) &: \theta
        \mapsto \frac{1}{\sqrt{D}} \sum_{j=1}^D \pairing{x,
        \omega_j}B(\omega_j)\theta_j
    \end{cases},
\end{dmath*}
where $\omega_j\sim\probability_{\dual{\Haar}, \rho}$ \ac{iid}~and
$B(\omega_j)\in\mathcal{L}\left(\mathbb{R}^p,\mathbb{R}^{p'}\right)$ for all
$\omega_j\in\dual{\mathcal{X}}$. Hence the \acl{ORFF} can be seen as the block
matrix $\hiderel{\in}\mathcal{M}_{2Dp',p}\left(\mathbb{R}\right)$
\begin{dmath}
    \label{eq:matrix_orff}
    \tildePhi{\omega}(x) =
    \begin{pmatrix}
        \cos\inner{x,\omega_1}B(\omega_1)  &
        \sin\inner{x,\omega_1}B(\omega_1)  &
        \hdots &
        \cos\inner{x,\omega_D}B(\omega_D)  &
        \sin\inner{x,\omega_D}B(\omega_D)
    \end{pmatrix}^\transpose
\end{dmath}

\subsubsection{Case of study: the decomposable kernel}
\label{subsec:fast_decomposable}
Throughout this section we show how the mathematical formulation relates to a
concrete (Python) implementation. We propose a Python implementation based on
NumPy~\citep{oliphant2006guide}, SciPy~\citep{jones2014scipy} and
Scikit-learn~\citep{pedregosa2011scikit}. Following \cref{eq:matrix_orff}, the
feature map associated to the decomposable kernel would be
\begin{pycode}[efficient_linop][fontsize=\scriptsize]
r"""Example of efficient implementation of Gaussian decomposable ORFF."""

from time import time

from numpy.linalg import svd
from numpy.random import rand, seed
from numpy import (dot, diag, sqrt, kron, zeros,
                   logspace, log10, matrix, eye, int)
from scipy.sparse.linalg import LinearOperator
from sklearn.kernel_approximation import RBFSampler
from matplotlib.pyplot import savefig, subplots
\end{pycode}

\begin{dmath*}
    \label{eq:matrix_decomposable_orff}
    \tildePhi{\omega}(x) =
    \frac{1}{\sqrt{D}}
    \begin{pmatrix}
        \cos\inner{x,\omega_1} B  &
        \sin\inner{x,\omega_1} B  & \dots &
        \cos\inner{x,\omega_D} B  &
        \sin\inner{x,\omega_D} B
    \end{pmatrix}^\transpose
    = \underbrace{\frac{1}{\sqrt{D}}
    \begin{pmatrix}
        \cos\inner{x,\omega_1} & \sin\inner{x,\omega_1} & \dots &
        \cos\inner{x,\omega_D} & \sin\inner{x,\omega_D}
    \end{pmatrix}}_{\tildephi{\omega}(x)}^\transpose \otimes B^\transpose ,
    % \hiderel{\in}\mathcal{M}_{2Du'u}\left(\mathbb{R}\right),
\end{dmath*}
$\omega_j\sim\probability_{\dual{\Haar}, \rho}$ \ac{iid}, which would lead to
the following naive python implementation for the Gaussian (RBF) kernel of
parameter $\gamma$, whose associated spectral distribution is
$\probability_{\rho}=\mathcal{N}(0, 2\gamma)$.  Let
$\theta\in\mathbb{R}^{2Dp'}$ and $y\in\mathbb{R^p}$. With such implementation
evaluating a matrix vector product such as $\tildePhi{\omega}(x)^\transpose
\theta$ or $\tildePhi{\omega}(x)y$ have $O_t(2Dp'p)$ time complexity and
$O_s(2Dp'p)$ of space complexity, which is utterly inefficient. Indeed, recall
that if $B\in\mathcal{M}_{p,p'}\left(\mathbb{R}^{p'}\right)$ is matrix, the
operator $\tildePhi{\omega}(x)$ corresponding to the decomposable kernel is
\begin{dmath}
    \label{eq:phi_efficient}
    \tildePhi{\omega}(x)y =
    \frac{1}{\sqrt{D}}\Vect_{j=1}^D\begin{pmatrix}\cos\inner{x, \omega_j}
    B^\transpose y \\ \sin\inner{x, \omega_j} B^\transpose y \end{pmatrix}
    \hiderel{=}
    \left(\frac{1}{\sqrt{D}}\Vect_{j=1}^D\begin{pmatrix}\cos\inner{x, \omega_j}
    \\ \sin\inner{x, \omega_j} \end{pmatrix}\right)\otimes (B^\transpose y)
\end{dmath}
and
\begin{dmath}
    \label{eq:phi_transpose_efficient} \tildePhi{\omega}(x)^\transpose \theta =
    \frac{1}{\sqrt{D}} \sum_{j=1}^D \cos\inner{x, \omega_j}B\theta_j +
    \sin\inner{x, \omega_j}B\theta_j \hiderel{=} B\left(\frac{1}{\sqrt{D}}
    \sum_{j=1}^D \left(\cos\inner{x, \omega_j} + \sin\inner{x,
    \omega_j}\right)\theta_j\right).
\end{dmath}
Which requires only evaluation of $B$ on $y$ and can be implemented easily in
Python thanks to SciPy's LinearOperator. Note that the computation of these
expressions can be fully vectorized\footnote{See~\citet{walt2011numpy}.} using
the vectorization property of the Kronecker product. In the following we
consider $\Theta \in \mathcal{M}_{2D,u'}(\mathbb{R})$ and the operator
$\vectorize: \mathcal{M}_{p',2D}(\mathbb{R}) \to \mathbb{R}^{2Dp'}$ which turns
a matrix into a vector (\acs{ie}~$\theta_{p'i+j} = \vectorize(\Theta_{ij})$,
$i\in\mathbb{N}_{(2D-1)}$ and $j\in\mathbb{N}^*_{p'}$). Then
$\left(\tildephi{\omega}(x) \otimes B^\transpose \right)^\transpose  \theta
\hiderel{=} \left(\tildephi{\omega}(x)^\transpose  \otimes B\right)
\vectorize(\Theta) \hiderel{=} \vectorize\left(B \Theta \tildephi{\omega}(x)
\right)$.  with this trick, many authors \citep{Sindhwani2013,
brault2016random, rosasco2010learning, Carmeli2010} notice that the
decomposable kernel usually yields a Stein equation \citep{penzl1998numerical}.
Indeed rewriting step 3 of \cref{alg:close_form} gives a system to solve of the
form
\begin{dmath*}
    \tildephi{\omega}(X)\tildephi{\omega}(X)^\transpose \Theta B^\transpose B +
    \lambda \Theta - Y \hiderel{=} 0
    \Leftrightarrow
    \left(\tildephi{\omega}(X) \tildephi{\omega}(X)^\transpose
    \hiderel{\otimes} B^\transpose B \hiderel{+} \lambda I_{2Dp'}\right) \theta
    \hiderel{-} Y \hiderel{=} 0
\end{dmath*}
Many solvers exists to solve efficiently this kind of systems\footnote{For
instance \citet{sleijpen2010bi}.}, but most of them share the particularity
that they are not just restricted to handle Stein equations. Broadly speaking,
iterative solvers (or matrix free solvers) are designed to solve any systems of
equation of ther form $PX=C$, where $P$ is a linear operator (not a matrix).
This is exacly our case where $\tildephi{\omega}(x)\otimes B^T$ is the matrix
form of the operator $\Theta \mapsto \vectorize(B\Theta\tildephi{\omega}X)$.
\paragraph{}
This leads us to the following (more efficient) Python implementation of the
Decomposable \acs{ORFF} \say{operator} to be feed to a matrix-free solvers.
\begin{pyblock}[efficient_linop][fontsize=\scriptsize]
def EfficientDecomposableGaussianORFF(X, A, gamma=1.,
                                      D=100, eps=1e-5, random_state=0):
    r"""Return the efficient ORFF map associated with the data X.

    Parameters
    ----------
    X : {array-like}, shape = [n_samples, n_features]
        Samples.
    A : {array-like}, shape = [n_targets, n_targets]
        Operator of the Decomposable kernel (positive semi-definite)
    gamma : {float},
        Gamma parameter of the RBF kernel.
    D : {integer}
        Number of random features.
    eps : {float}
        Cutoff threshold for the singular values of A.
    random_state : {integer}
        Seed of the generator.

    Returns
    -------
    \tilde{\Phi}(X) : Linear Operator, callable
    """
    # Decompose A=BB^\transpose
    u, s, v = svd(A, full_matrices=False, compute_uv=True)
    B = dot(diag(sqrt(s[s > eps])), v[s > eps, :])

    # Sample a RFF from the scalar Gaussian kernel
    phi_s = RBFSampler(gamma=gamma, n_components=D, random_state=random_state)
    phiX = phi_s.fit_transform(X)

    # Create the ORFF linear operator
    return LinearOperator((phiX.shape[0] * B.shape[1], D * B.shape[0]),
                          matvec=lambda b: dot(phiX, dot(b.reshape((D, B.shape[0])), B)),
                          rmatvec=lambda r: dot(phiX.T, dot(r.reshape((X.shape[0], B.shape[1])), B.T)))
\end{pyblock}
%\subsection{Linear operators in matrix form}
%\label{subsec:efficient_linop}
%For convenience we give the operators corresponding to the decomposable,
%curl-free and divergence-free kernels in matrix form. Let $(x_i)_{i=1}^N$,
%$N\in\mathbb{N}^*$, $x_i$'s in $\mathbb{R}^d$, $d\le\infty$ be a sequence of
%points in $\mathbb{R}^d$. We note
%\begin{dmath*}
    %X=\begin{pmatrix}x_1 & \hdots &
    %x_N\end{pmatrix}\hiderel{\in}\mathcal{M}_{d,N}
%\end{dmath*}
%the data matrix where each column represents a data point\footnote{In many
%programming language, such as Python, C, C{}\verb!++! or Java each data point
%is traditionally represented by a row in the data matrix (row major
%formulation).  While this is more natural when parsing a data file, it is less
%common in mathematical formulations. In this document we adopt the \emph{column
%major} formulation used by Matlab, Fortran or Julia. Moreover although
%C{}\verb!++! is commonly row major, some libraries such as Eigen are column
%major. When dealing with row major formulation, one should \say{transpose} all
%the equations given in \cref{table:efficient-op}.}. Naturally if
%$\tildePhi{\omega}(x):\mathbb{R}^p\to\mathbb{R}^{r_1}$ and
%$\tildephi{\omega}(x):\mathbb{R}\to\mathbb{R}^{r_2}$, for all
%$x\in\mathbb{R}^d$ we define
%\begin{dmath*}
    %\tildePhi{\omega}(X)=\begin{pmatrix}\tildePhi{\omega}(x_1) & \hdots &
    %\tildePhi{\omega}(x_N)\end{pmatrix}\hiderel{\in}\mathcal{M}_{r_1,Np}
%\end{dmath*}
%and
%\begin{dmath*}
    %\tildephi{\omega}(X)=\begin{pmatrix}\tildephi{\omega}(x_1) & \hdots &
    %\tildephi{\omega}(x_N)\end{pmatrix}\hiderel{\in}\mathcal{M}_{r_2, N}
%\end{dmath*}
%and
%\begin{dmath*}
    %Y=\begin{pmatrix} y_1 & \hdots&  y_N
    %\end{pmatrix}\hiderel{\in}\mathcal{M}_{p, N}.
%\end{dmath*}
%Given a matrix $X\in\mathcal{M}_{m,n}(\mathbb{R})$, we note $X_{\bullet i}$ the
%\emph{column} vector corresponding to the $i$-th column of the matrix $X$ and
%$X_{i \bullet}$ the \emph{row} vector (covector) corresponding to the $i$-th
%line of the matrix $X$. With these notations, if $X\in\mathcal{M}_{m,n}$ and
%$Z\in\mathcal{M}_{n,m'}$, $X_{i\bullet}Z_{\bullet j}\in\mathbb{R}$ is the inner
%product between the $i$-th row of $X$ and the $j$-th column of $Z$ and
%$X_{\bullet i} Z_{j \bullet}\in\mathcal{M}_{m,m'}(\mathbb{R})$ is the outer
%product between the $i$-th column of $X$ and $j$-th row of $X$.
%\paragraph{}
%For the curl-free and divergence-free kernel given in
%\cref{subsec:examples_ORFF} we recall the unbounded \acs{ORFF} maps are
%respectively for all $y\in\mathcal{Y}$
%\begin{dmath*}
    %\tildePhi{\omega}(x) y =\frac{1}{\sqrt{D}}\Vect_{j=1}^D
    %\begin{pmatrix}
        %\cos{\inner{x,\omega_j}_2}\omega_j^\transpose y \\
        %\sin{\inner{x,\omega_j}_2}\omega_j^\transpose y
    %\end{pmatrix},
%\end{dmath*}
%and
%\begin{dmath*}
    %\tildePhi{\omega}(x) y = \frac{1}{\sqrt{D}}\Vect_{j=1}^D
    %\begin{pmatrix}
        %\cos{\inner{x,\omega_j}_2}\left(\norm{\omega_j }_2I_d -
        %\frac{\omega_j\omega_j^\transpose }{\norm{\omega_j}_2}\right) y\\
        %\sin{\inner{x,\omega_j}_2}\left(\norm{\omega_j}_2I_d-
        %\frac{\omega_j\omega_j^\transpose }{\norm{\omega_j}_2}\right) y
    %\end{pmatrix},
%\end{dmath*}
%where $\omega_j\sim \probability_{\mathcal{N}(0,\sigma^{-2}I_d)}$. To avoid
%complex index notations we decompose the feature maps $\tildePhi{\omega}(X)$
%into two sub feature maps $\tildePhi{\omega}^c$ and $\tildePhi{\omega}^s$
%corresponding to the cosine part and the sine part of each feature map. Namely,
%for the curl-free kernel, for all $y\in\mathcal{Y}$
%\begin{dmath*}
    %\tildePhi{\omega}(x) y =
    %\begin{cases}
        %\tildePhi{\omega}^c(x) y = \frac{1}{\sqrt{D}}\Vect_{j=1}^D
        %\begin{pmatrix}
            %\cos{\inner{x,\omega_j}_2}\omega_j^\transpose y
        %\end{pmatrix}, \\
        %\tildePhi{\omega}^s(x) y = \frac{1}{\sqrt{D}}\Vect_{j=1}^D
        %\begin{pmatrix}
            %\sin{\inner{x,\omega_j}_2}\omega_j^\transpose  y
        %\end{pmatrix}.
    %\end{cases}
%\end{dmath*}
%In the same way, for the divergence-free kernel,
%\begin{dmath*}
    %\tildePhi{\omega}(x) y =
    %\begin{cases}
        %\tildePhi{\omega}^c(x) y  = \frac{1}{\sqrt{D}}\Vect_{j=1}^D
        %\begin{pmatrix}
            %\cos{\inner{x,\omega_j}_2}\left(\norm{\omega_j}_2 I_d -
            %\frac{\omega_j\omega_j^\transpose}{\norm{\omega_j}_2} \right) y
        %\end{pmatrix}, \\
        %\tildePhi{\omega}^s(x) y  = \frac{1}{\sqrt{D}}\Vect_{j=1}^D
        %\begin{pmatrix}
            %\sin{\inner{x,\omega_j}_2} \left(\norm{\omega_j}_2 I_d -
            %\frac{\omega_j\omega_j^\transpose}{\norm{\omega_j}_2} \right) y
        %\end{pmatrix}.
    %\end{cases}
%\end{dmath*}
%We also introduce $\tildePhi{\omega}^e$, $e\in\Set{s,c}$ which denotes either
%$\tildePhi{\omega}^s$ or $\tildePhi{\omega}^c$. This equivalent formulation
%allows us to keep the notation \say{lighter} and closer to a proper
%Python/Matlab implementation with vectorization. With these notations, a
%summary of efficient linear operators in matrix form is given in
%\cref{table:efficient-op}. The complexity of evaluating all this operators is
%given in \cref{table:efficient-complexity}.
%\afterpage{%
%\begin{landscape}
    %\begin{table}[htb]{}
        %\centering
        %\begin{threeparttable}
            %\caption[Efficient linear-operators for different
            %\acs{ORFF}.]{Efficient linear-operator (in matrix form) for
            %different Feature maps. \label{table:efficient-op}}
            %\begin{tabularx}{1.4\textwidth}{Xcc}
                %\toprule
                    %Kernel & $\tildePhi{\omega}(X)^\adjoint$ &
                    %$\tildePhi{\omega}(X)$ \\
                %\midrule
                    %Decomposable\tnote{1} &$\Theta\mapsto B\left(\Theta
                    %\tildephi{\omega}(X)\right)$ & $Y\mapsto B^\transpose
                    %\left(Y\tildephi{\omega}(X)^\transpose \right)$ \\ Gaussian
                    %curl-free\tnote{2} & $\Theta^c, \Theta^s\mapsto
                    %\displaystyle\sum_{j=1}^D \omega_j \left(\Theta_{j}^c
                    %\tildephi{\omega}^{c}(X)_{j\bullet} +
                    %\Theta_{j}^s\tildephi{\omega}^{s}(X)_{j\bullet}\right)$ &
                    %$Y\mapsto \Theta_j^e=\omega_j^\transpose
                    %\left(Y\tildephi{\omega}^{e}(X)_{\bullet j}^\transpose
                    %\right)$ \\ Gaussian divergence-free\tnote{2,3} &
                    %$\Theta^c, \Theta^s \mapsto \displaystyle\sum_{j=1}^D
                    %\left(B(\omega_j) \Theta^{c}_{\bullet j}\right)
                    %\tildephi{\omega}^{c}(X)_{j\bullet} +
                    %\left(B(\omega_j)\Theta^{s}_{\bullet
                    %j}\right)\tildephi{\omega}^{s}(X)_{j\bullet}$ & $Y \mapsto
                    %\Theta^e_{\bullet
                    %j}=B(\omega_j)\left(Y\tildephi{\omega}^{e}(X)_{\bullet
                    %j}^\transpose \right)$ \\
                %\bottomrule
            %\end{tabularx}
            %\begin{tablenotes}
                %\item[1] Where $\tildephi{\omega}(X)=\begin{pmatrix}
                %\tildephi{\omega}(X_{\bullet 1}) & \hdots &
                %\tildephi{\omega}(X_{\bullet N})
                %\end{pmatrix}\in\mathcal{M}_{r, N}$ is any design matrix, with
                %scalar feature map
                %$\tildephi{\omega}:\mathbb{R}^d\to\mathbb{R}^r$ such that
                %$\tildephi{\omega}(x)^\adjoint
                %\tildephi{\omega}(z)=k(x,z)\in\mathbb{R}$ for all $x$,
                %$z\in\mathcal{X}$. The input data
                %$X\in\mathcal{M}_{d,N}(\mathbb{R})$, the output data
                %$U\in\mathcal{M}_{p,N}(\mathbb{R})$, the parameter matrices
                %$\Theta^c$ and $\Theta^s\in\mathcal{M}_{p', r}(\mathbb{R})$ and
                %the decomposable operator $B\in\mathcal{M}_{p,p'}(\mathbb{R})$.
                %\item[2] Where
                %$\tildephi{\omega}^{c}(X)_{ji}=\cos\inner{\omega_j, x_i}$ and
                %$\tildephi{\omega}^{s}(X)_{ji}=\sin\inner{\omega_j, x_i}$,
                %$j\in\mathbb{N}^*_D$ and $i\in\mathbb{N}^*_N$. Thus
                %$\tildephi{\omega}^{c}(X)\in\mathcal{M}_{D,N}(\mathbb{R})$ and
                %$\tildephi{\omega}^{s}(X)\in\mathcal{M}_{D,N}(\mathbb{R})$. The
                %input data $X\in\mathcal{M}_{d,N}(\mathbb{R})$, the output data
                %$U\in\mathcal{M}_{d,N}(\mathbb{R})$, the parameter matrices
                %$\Theta^c$ and $\Theta^s\in\mathbb{R}^D$, $\omega_j\sim
                %\probability_{\mathcal{N}(0,\sigma^{-2}I_d)}$ \ac{iid}~for all
                %$j\in\mathbb{N}^*_D$. Eventually $e\in\Set{s,c}$, namely
                %$\Theta^c=\begin{pmatrix} \Theta^{e=c}_1 & \hdots &
                %\Theta^{e=c}_D \end{pmatrix}^\transpose $ and
                %$\Theta^s=\begin{pmatrix} \Theta^{e=s}_1 & \hdots &
                %\Theta^{e=s}_D\end{pmatrix}^\transpose $.
                %\item[3] Here,
                %$\Theta^c$ and $\Theta^s\in\mathcal{M}_{d,D}(\mathbb{R})$ thus
                %$\Theta^c=\begin{pmatrix}\Theta^{e=c}_{\bullet 1} & \hdots &
                %\Theta^{e=c}_{\bullet D}\end{pmatrix}$,
                %$\Theta^s=\begin{pmatrix}\Theta^{e=s}_{\bullet 1} & \hdots &
                %\Theta^{e=s}_{\bullet D}\end{pmatrix}$ and
                %$B(\omega)=\left(\norm{\omega}_2I_d-\frac{\omega\omega^\transpose
                %}{\norm{\omega}_2}\right)\in\mathcal{M}_{d,d}$.
                %\end{tablenotes}
        %\end{threeparttable}
    %\end{table}
%\end{landscape}}
%\paragraph{}
It is worth mentioning that the same strategy can be applied in many different
language. For instance in C{}\verb!++!, the library Eigen~\citep{eigenweb}
allows to wrap a sparse matrix with a custom type, where the user overloads the
transpose and dot product operator (as in Python). Then the custom user
operator behaves as a (sparse) matrix --see
\url{https://eigen.tuxfamily.org/dox/group__MatrixfreeSolverExample.html}. With
this implementation the time complexity of $\tildePhi{\omega}(x)^\transpose
\theta$ and $\tildePhi{\omega}(x)y$ falls down to $O_t((D+p)p')$ and the same
holds for space complexity.
\paragraph{}
A quick experiment shows the advantage of seeing the decomposable kernel as a
linear operator rather than a matrix. We draw $N=100$ points $(x_i)_{i=1}^N$ in
the interval $(0,1)^{20}$ and use a decomposable kernel with matrix
$\Gamma=BB^\transpose \in\mathcal{M}_{p,p}(\mathbb{R})$ where
$B\in\mathcal{M}_{p,p}(\mathbb{R})$ is a random matrix with coefficients drawn
uniformly in $(0,1)$. We compute $\tildePhi{\omega}(x)^\transpose \theta$ for
all $x_i$'s, where $\theta\in\mathcal{M}_{2D,1}(\mathbb{R})$, $D=100$, with the
implementation \texttt{Ef\-fi\-cient\-De\-com\-po\-sa\-ble\-Gaus\-sian\-ORFF},
\cref{eq:phi_transpose_efficient}. The coefficients of $\theta$ were drawn at
random uniformly in $(0,1)$. We report the execution time in
\cref{fig:efficient_decomposable_gaussian} for different values of $p$, $1\le
p\le100$.
\begin{pycode}[efficient_linop]
sys.path.append('./src/')
import efficient_decomposable_gaussian

efficient_decomposable_gaussian.main()
\end{pycode}
\begin{figure}[t]
    \pyc{print(r'\centering\resizebox{\textwidth}{!}{\input{./efficient_decomposable_gaussian.pgf}}')}
    \caption[Efficient decomposable Gaussian \acs{ORFF}]{Efficient decomposable
    Gaussian ORFF (lower is better).}
    \label{fig:efficient_decomposable_gaussian}
\end{figure}
The left plot reports the execution time in seconds of the construction of the
feature. The middle plot reports the execution time of
$\tildePhi{\omega}(x)^\transpose \theta$, and the right plot the memory used in
bytes  to store $\tildePhi{\omega}(x)$ for all $x_i$'s. We averaged the results
over ten runs.
\paragraph{Curl-free kernel.}
We use the unbounded \acs{ORFF} map presented in
\cref{eq:unbounded_curl_free_orff}. We draw $N=1000$ points $(x_i)_{i=1}^N$ in
the interval $(0,1)^{p}$ and use a curl-free kernel. We compute
$\tildePhi{\omega}(x)^\transpose \theta$ for all $x_i$'s, where
$\theta\in\mathcal{M}_{2D,1}(\mathbb{R})$, $D=500$, with the matrix
implementation and the \texttt{LinearOperator} implementation. The coefficients
of $\theta$ were drawn at random uniformly in $(0,1)$. We report the execution
time in \cref{fig:efficient_curlfree_gaussian} for different values of $p$,
$1\le p\le100$.
\begin{pycode}[efficient_linop]
sys.path.append('./src/')
import efficient_curlfree_gaussian

efficient_curlfree_gaussian.main()
\end{pycode}
\begin{figure}[t]
    \pyc{print(r'\centering\resizebox{\textwidth}{!}{\input{./efficient_curlfree_gaussian.pgf}}')}
    \caption[Efficient curl-free Gaussian \acs{ORFF}]{Efficient curl-free
    Gaussian ORFF (lower is better).}
    \label{fig:efficient_curlfree_gaussian}
\end{figure}
The left plot reports the execution time in seconds of the construction of the
features. The middle plot reports the execution time of
$\tildePhi{\omega}(x)^\transpose \theta$, and the right plot the memory used in
bytes  to store $\tildePhi{\omega}(x)$ for all $x_i$'s. We averaged the results
over fifty runs. As we can see the linear-operator implementation is one order
of magnitude slower than its matrix counterpart. However it uses considerably
less memory.
\paragraph{Divergence-free kernel.}
We use the unbounded \acs{ORFF} map presented in
\cref{eq:unbounded_div_free_orff}. We draw $N=100$ points $(x_i)_{i=1}^N$ in
the interval $(0,1)^{p}$ and use a curl-free kernel. We compute
$\tildePhi{\omega}(x)^\transpose \theta$ for all $x_i$'s, where
$\theta\in\mathcal{M}_{2Dp,1}(\mathbb{R})$, $D=100$, with the matrix
implementation and the \texttt{LinearOperator} implementation. The coefficients
of $\theta$ were drawn at random uniformly in $(0,1)$. We report the execution
time in \cref{fig:efficient_curlfree_gaussian} for different values of $p$,
$1\le p\le100$.
\begin{pycode}[efficient_linop]
sys.path.append('./src/')
import efficient_divfree_gaussian

efficient_divfree_gaussian.main()
\end{pycode}
\begin{figure}[t]
    \pyc{print(r'\centering\resizebox{\textwidth}{!}{\input{./efficient_divfree_gaussian.pgf}}')}
    \caption[Efficient divergence-free Gaussian \acs{ORFF}]{Efficient
    divergence-free Gaussian ORFF (lower is better).}
    \label{fig:efficient_divfree_gaussian}
\end{figure}
The left plot reports the execution time in seconds of the construction of the
feature. The middle plot reports the execution time of
$\tildePhi{\omega}(x)^\transpose \theta$, and the right plot the memory used in
bytes  to store $\tildePhi{\omega}(x)$ for all $x_i$'s. We averaged the results
over ten runs. We draw the same conclusions as the curl-free kernel.
%\begin{table}[!htb]
    %\centering
    %\caption[Time complexity of efficient linear-operators for different
    %\acs{ORFF}.]{Complexity of efficient linear-operator (in matrix form) for
    %different Feature maps given in \cref{table:efficient-op}.
    %\label{table:efficient-complexity}}
    %\begin{tabularx}{\textwidth}{Xcc}
        %\toprule
            %Kernel & $\tildePhi{\omega}(X)^\adjoint$ & $\tildePhi{\omega}(X)$
            %\\
        %\midrule
            %Decomposable & $O_t\left((p'D+p'p)N\right)$ &
            %$O_t\left((pN+p'p)D\right)$ \\
            %Curl-free & $O_t\left(pND\right)$ & $O_t\left(pND\right)$ \\
            %Divergence-free & $O_t\left((p^2+pN)D\right)$ &
            %$O_t\left((p^2+pN)D\right)$ \\
        %\bottomrule
    %\end{tabularx}
%\end{table}

\section{Numerical experiments}
\label{sec:num_exp}
We present a set of experiments to complete the theoretical contribution and
illustrate the behavior of ORFF-regression. First we study how well the ORFF
regression recover the result of operator-valued kernel regression. Second we
show the advantages of ORFF regression over independent RFF regression. A code
implementing ORFF is available at \url{https://github.com/operalib/operalib} a
framework for OVK Learning.

\subsection{Learning with {ORFF} vs learning with {OVK}}
\subsubsection{Datasets}
The \emph{first dataset} considered is the handwritten digits recognition
dataset \textsc{MNIST}\footnote{available at
\url{http://yann.lecun.com/exdb/mnist}}. We select a training set of $12,000$
images and a test set of $10,000$ images. The inputs are images represented as
a vector $x_i\in[0,255]^{784}$ and the targets $y_i\in\mathbb{N}_9$ are
integers between $0$ and $9$.  First we scaled the inputs such that they take
values in $[-1,1]^{784}$. Then we binarize the targets such that each number is
represented by a unique binary vector of dimension $10$. The vector $y_i$ is
zero everywhere except on the dimension corresponding to the class where it is
one.  For instance the class $4$ is encoded $\begin{pmatrix} 0 & 0 & 0 & 0 & 1
& 0 & 0 & 0 & 0 & 0 \end{pmatrix}^\transpose$.  To predict classes, we use the
simplex coding method presented in \citet{mroueh2012multiclass}. The intuition
behind simplex coding is to project the binarized labels of dimension $p$ onto
the most separated vectors on the hypersphere of dimension $p-1$. For ORFF we
can encode directly this projection in the $B$ matrix of the decomposable
kernel $K_0(\delta)=B B^* k_0(\delta)$ where $k_0$ is a Gaussian kernel. The
matrix $B$ is computed via the recursion
\begin{dmath*}
    B_{p+1} \hiderel{=}
    \begin{pmatrix}
        1 & u^T \\
        0_{p-1} & \sqrt{1-p^{-2}}B_p
    \end{pmatrix},
    \qquad B_2 \hiderel{=}
    \begin{pmatrix}
        1 & -1
    \end{pmatrix},
\end{dmath*}
where $u=\begin{pmatrix} -p^{-2} & \hdots & -p^{-2}
\end{pmatrix}^T\in\mathbb{R}^{p-1}$ and $0_{p-1} = \begin{pmatrix} 0 & \hdots &
0 \end{pmatrix}^T \in\mathbb{R}^{p-1}$. For \aclp{OVK} we project the binarized
targets on the simplex as a preprocessing step, before learning with the
decomposable $K_0(\delta)=I_p k_0(\delta)$, where $k_0$ is a scalar Gaussian
kernel.
\paragraph{}
The \emph{second dataset} is a simulated five dimensional ($5D$) vector field
with structure. We generate a scalar field as a random function
$f:[-1,1]^5\to\mathbb{R}$, where
$\tildef{\omega}(x)=\tildephi{\omega}(x)^\adjoint \theta$ where $\theta$ is a
random matrix with each entry following a standard normal distribution,
$\tildephi{\omega}$ is a scalar Gaussian RFF with bandwidth $\sigma=0.4$. The
input data $x$ are generated from a uniform probability distribution. We take
the gradient of $\tildef{\omega}$ to generate the curl-free $5D$ vector field.
\paragraph{}
The \emph{third dataset} is a synthetic of data from
$\mathbb{R}^{20}\to\mathbb{R}^4$ as described in \citet{audiffren2013online}.
In this dataset, inputs ($x_1, \hdots, x_{20}$) are generated independently and
uniformly over $[0, 1]$ and the different outputs are computed as follows. Let
$\phi(x)=(x_1^2, x_4^2, x_1x_2, x_3x_5, x_2, x_4, 1)$ and $(w_i)$ denotes the
\acs{iid}~copies of a seven dimensional Gaussian distribution with zero mean
and covariance $\Sigma\in\mathcal{M}_{7,7}(\mathbb{R})$ such that
$\begin{pmatrix} 0.5 & 0.25 & 0.1 & 0.05 & 0.15 & 0.1 & 0.15 \end{pmatrix}$
Then, the outputs of the different tasks are generated as $y_i=w_i\phi(x)$. We
use this dataset with $p=4$, $10^5$ instances and for the train set and also
$10^5$ instances for the test set.

\subsubsection{Results}
\paragraph{Performance of ORFF regression on the first dataset.}
We trained both \acs{ORFF} and \acs{OVK} models on \textsc{MNIST} dataset with
a decomposable Gaussian kernel with signature
$K_0(\delta)=\exp\left(-\norm{\delta}/(2\sigma^2)\right)\Gamma$.  To apply
$\cref{alg:close_form}$ after noticing that in the case of the decomposable
kernel with $\lambda_M=0$, it boilds down to a Stein equation \citep[section
5.1]{brault2016random}, we use an off-the-shelf solver\footnote{Available at
\url{http://ta.twi.tudelft.nl/nw/users/gijzen/IDR.html}} able to handle Stein's
equation. For both methods we choose $\sigma=20$ and use a $2$-fold cross
validation on the training set to select the optimal $\lambda$. First,
\cref{fig:learning_accuracy} compares the running time between OVK and ORFF
models using $D=1000$ Fourier features against the number of data\-points $N$.
The log-log plot shows ORFF scaling better than the OVK \acs{wrt} the number of
points.  Second, \cref{fig:learning_accuracy} shows the test prediction error
versus the number of ORFFs $D$, when using $N=1000$ training points. As
expected, the ORFF model converges toward the OVK model when the number of
features increases.
\begin{figure}[t]
    \centering
    \begin{tabular}{cc}
        \resizebox{.45\textwidth}{!}{\input{./gfx/learning_accuracy_MNIST.tikz}} &
        \resizebox{.45\textwidth}{!}{\input{./gfx/learning_time_MNIST.tikz}}
    \end{tabular}
    % \includegraphics[\textwidth]{./gfx/learning_time_MNIST.tikz}
    \caption[Prediction Error in percent on the MNIST dataset versus $D$, the
    number of Fourier features]{Empirical comparison of ORFF and OVK regression
    on MNIST dataset and empirical behavior of ORFF regression versus $D$ and
    $N$.\label{fig:learning_accuracy}}
\end{figure}
\paragraph{Performance of ORFF regression on the second dataset.}
We perform a similar experiment on the second dataset ($5$D-vector field with
structure). We use a Gaussian curl-free kernel with bandwidth equal to the
median of the pairwise distances and tune the hyperparameter $\lambda$ on a
grid. Here we optimize \cref{eq:argmin_applied}, where $c$ is the squared error
cost, using Scipy's \acs{L-BFGS-B} \citep{byrd1995limited}
solver\footnote{Available at
\url{http://docs.scipy.org/doc/scipy/reference/optimize.html}} with the
gradients given in \cref{eq:grad_final} and the efficient linear operator
described in \cref{subsec:efficient_learning} (\acs{eg}
\cref{eq:phi_efficient,eq:phi_transpose_efficient}).
\Cref{fig:curl_experiment} (bottom row) reports the $R^2$ (coefficient of
determination) score on the test set versus the number of curl-\acs{ORFF} $D$
with a comparison with curl-\acs{OVK}.  In this experiment, we see that
curl-\acs{ORFF} can even be better than curl-\acs{OVK}, suggesting that
\acs{ORFF} might play an additional regularizing role. It also shows the
computation time of curl-\acs{ORFF} and curl-\acs{OVK}. We see that \acs{OVK}
regression does not scale with large datasets, while \acs{ORFF} regression
does. When $N>10^4$, \acs{OVK} regression exceeds memory capacity.
\begin{figure}[t]
    \centering
    \resizebox{.85\textwidth}{!}{\input{./gfx/Curl_ORFFvsOVK.pgf}}
    \caption{Empirical comparison between curl-free ORFF, curl-free OVK,
    independent ORFF, independent OVK on a synthetic vector field regression
    task. \label{fig:curl_experiment}}
\end{figure}
\paragraph{Structured prediction vs Independent (RFF) prediction.}
On the second dataset, \cref{fig:curl_experiment} (top row) compares $R^2$ score
and time of \acs{ORFF} regression using the trivial identity decomposable
kernel, \acs{eg} independent \acsp{RFF}, to curl-free \acs{ORFF} regression.
Curl-free \acs{ORFF} outperforms independent \acsp{RFF}, as expected, since the
dataset involves structured outputs.
\paragraph{Impact of the number of random features ($D$).}
In this setting we solved the optimisation problem for both \acs{ORFF} and
\acs{OVK} using a \acs{L-BFGS-B}. \Cref{fig:ORFFvsOVK_dec} top row shows that
for a fixed number of instance in the train set, \acs{OVK} performs better than
\acs{ORFF} in terms of accuracy ($R^2$). However \acs{ORFF} scales better than
\acs{OVK} \acs{wrt} the number of data. \acs{ORFF} is able to process more data
than \acs{OVK} in the same time and thus reach a better accuracy for a given
amount of time. Bottom row shows that \acs{ORFF} tends to reach \acs{OVK}'s
accuracy for a fixed number of data when the number of features increase.
\begin{figure}[t]
    \centering
    \resizebox{.85\textwidth}{!}{%
    \input{./gfx/ORFFvsOVK.pgf}}
    \caption{Decomposable kernel on the third dataset: $R^2$ score vs number of
    data in the train set ($N$) \label{fig:ORFFvsOVK_dec}}
\end{figure}
\begin{figure}[t]
    \centering
    \resizebox{.85\textwidth}{!}{%
    \input{./gfx/ORFFvsOVK_Dvariation.pgf}}
    \caption{Decomposable kernel on the third dataset: $R^2$ score vs number of
    data in the train set ($N$) for different number for different number of
    random samples ($D$). \label{fig:ORFFvsOVK}}
\end{figure}
\paragraph{Multitask learning.}
In this experiment we are interested in multitask learning with operator-valued
random Fourier features, and see whether the approximation of a joint \acs{OVK}
performs better than an independent \acs{OVK}. In this setting we assume that
for each entry $x_i\in\mathbb{R}^d$ we only have access to one observation
$y_i\in\mathbb{R}$ corresponding to a task $t_i$.  We used the SARCOS dataset,
taken from \url{http://www.gaussianprocess.org/gpml/data/} website. This is an
inverse dynamics problem, \acs{ie} we have to predict the $7$ joint torques
given the joint positions, velocities and accelerations.  Hence, we have to
solve a regression problem with $21$ inputs and $7$ outputs which is a very
nonlinear function. It has $45K$ inputs data.  Suppose that we are given a
collection of inputs data $x_1, \hdots, x_N\in\mathbb{R}^{21}$ and a collection
of output data $((y_1, t_1) \hdots, (y_N, t_N)) \in \left(\mathbb{R}\times
\mathbb{N}_T\right)^N$ where $T$ is the number of tasks.  We consider the
following multitask loss function $L(h(x), (y,t))=\frac{1}{2}\left(\inner{h(x),
e_t}_2-y\right)^2$, This loss function is adapted to datasets where the number
of data per tasks is unbalanced (\acs{ie}~for one input data we observe the
value of only one task and not all the tasks.). We optimise the regularized
risk
\begin{dmath*}
    \frac{1}{N}\sum_{i=1}^N  L\left(h(x_i), (y_i, t_i)\right) +
    \frac{\lambda}{2N}||{h}||_{\mathcal{H}}^2=\frac{1}{2N}\sum_{i=1}^N
    \left(\langle h(x_i), e_{t_i} \rangle-y_i\right)^2 +
    \frac{\lambda}{2N}||{h}||_{\mathcal{H}}^2
\end{dmath*}
We used a model $h$ based on the decomposable kernel $h(x)= (\phi(x)^T \otimes
B) \theta$ we chose $B$ such that $BB^T=A$, where $A$ is the inverse graph
Laplacian $L$ of the similarities between the tasks, parametrized by an
hyperparameter $\gamma \in \mathbb{R}_+$.
$L_{kl}=\exp\left(-\gamma\sqrt{\sum_{i=1}^N \left(y_i^k -
y_i^l\right)^2}\right)$.  We draw $N$ data randomly for each task, hence
creating a dataset of $N\times 7$ data and computed the nMSE on the proposed
test set ($4.5$K points). We repeated the experiments $80$ times to avoid
randomness. We choose $D=\frac{\max(N, 500)}{2}$ features, and optimized the
problem with a second order batch gradient.
\begin{table}
    \centering
    \begin{tabular}{c|cccc}
        \toprule
            N & Independant (\%) & Laplacian (\%)& p-value & T \\
        \midrule
            $50\times 7$ & $23.138 \pm 0.577$ & $22.254\pm 0.536$ & $2.68\%$ &
            $4$(s) \\
            $100\times 7$ & $16.191 \pm 0.221$ & $15.568 \pm 0.187$ & $<0.1\%$
            & $16$(s) \\
            %$150\times 7$ & $13.821 \pm 0.115$ & $13.459 \pm 0.106$ & $<0.1\%$
            %& $13$(s) \\
            $200\times 7$ & $12.713 \pm 0.0978$ & $12.554 \pm 0.0838$ &
            $1.52\%$ & $12$(s) \\
            $400\times 7$ & $10.785 \pm 0.0579$ & $10.651 \pm 0.0466$ & $<
            0.1\%$ & $10$(s) \\
            $800\times 7$ & $7.512\pm 0.0344$ & $7.512\pm 0.0344$ & $100\%$ &
            $15$(s) \\
            %$1600\times 7$ & $6.486 \pm 0.0242$ & $6.486 \pm 0.0242$ & $100\%$
            %& $20$(s) \\
            $3200\times 7$ & $5.658 \pm 0.0187$ & $5.658 \pm 0.0187$ & $100\%$
            & $20$(s) \\
        \bottomrule
    \end{tabular}
    \caption{Error (\% of nMSE) on SARCOS dataset.}
    \label{table:sarcos}
\end{table}
\Cref{table:sarcos} shows that using the \acs{ORFF} approximation of an
operator-valued kernel with a good prior on the data improves the performances
\acs{wrt} the independent \acs{ORFF}. However the advantage seems to be less
important the more data are available.

%%%%%%%%%%%%%%%%%%%%%%%%%%%%%%%%%%%%%%%%%%%%%%%%%%%%%%%%%%%%%%%%%%%%%%%%%%%%%%%

\section{Conclusion}
\label{sec:conclusion}
\acsp{OVK} naturally extend the celebrated kernel method used to learn
scalar-valued functions, to the case of learning vector-valued functions.
Although \acsp{OVK} are appealing from a theoretical aspect, these methods
scale poorly in terms of computation time when the number of data is high.
Indeed, to evaluate the value of function with an \acl{OVK}, it requires to
evaluate an \acl{OVK} on all the point in the given dataset.  Hence naive
learning with kernels usually scales cubicly in time with the number of data.
In the context of large-scale learning such scaling is not acceptable. Through
this work we propose a methodology to tackle this difficulty.
\paragraph{}
Enlightened by the literature on large-scale learning with
\emph{scalar}-valued kernel, in particular the work of Rahimi and Recht
\citep{Rahimi2007}, we propose to replace an \acs{OVK} by a random feature
map that we called \acl{ORFF}. Our contribution start with the formal
mathematical construction of this feature from an \acs{OVK}. Then we show
that it is also possible to obtain a kernel from an \acs{ORFF}. Eventually
we analyse the regularization properties in terms of \acl{FT} of
$\mathcal{Y}$-Mercer kernels. Then we moved on giving a bound on the error
due to the random approximation of the \acs{OVK} with high probability.
We showed that it is possible to bound the error even though the \acs{ORFF}
estimator of an \acs{OVK} is not a bounded random variable. Moreover we
also give a bound when the dimension of the output data infinite.
\paragraph{}
After ensuring that an \acs{ORFF} is a good approximation of a kernel, we
moved on giving a framework for supervised learnin with \aclp{OVK}. We showed
that learning with a feature map is equivalent to learn with the reconstructed
\acs{OVK} under some mild conditions. Then we focused on an efficient
implementation of \acs{ORFF} by viewing them as linear operators rather than
matrices and using matrix-free (iterative) solvers and concluded with some
numerical experiments.
\paragraph{}
Following Rahimi and Recht a generalization bound for \acs{ORFF} kernel ridge
would probably suggest that the number of feature to draw is proportional to
the number of data.  However new results of \citet{rudi2016generalization}
suggest that the number of feature should be proportional to the \emph{square
root} of the number of data. In a future work, we shall investigate this
results and extend it to \acs{ORFF}.
\paragraph{}
Since the contruction of \acs{ORFF} is valid for infinite dimensional Hilbert
spaces such as function spaces, we would also like to investigate learning
function valued functions in an efficient manner.

% Acknowledgements should go at the end, before appendices and references
\acks{The authors are grateful to Maxime Sangnier (UPMC, France) for the
insightful discussions and Markus Heinonen (Aalto University, Sweden) for the
preliminary experiments.}

% Manual newpage inserted to improve layout of sample file - not
% needed in general before appendices/bibliography.


\appendix
\section{Reminder on Abstract Harmonic Analysis}\label{app:theorem}
\label{sec:abstract_harmonic}
\subsection{Locally compact Abelian groups}
\begin{definition}[\acf{LCA} group.]
    A group $\mathcal{X}$ endowed with a binary operation $\groupop$ is said to
    be a Locally Compact Abelian group if $\mathcal{X}$ is a topological
    \emph{commutative} group \acs{wrt}~$\groupop$ for which every point has a
    compact neighborhood and is Hausdorff (T2).
\end{definition}
Moreover given a element $z$ of a \ac{LCA} group $\mathcal{X}$, we define the
set $z\groupop\mathcal{X}=\mathcal{X}\groupop z=\Set{z\groupop x|\forall
x\in\mathcal{X}}$ and the set $\mathcal{X}^{-1}=\Set{x^{-1}|\forall
x\in\mathcal{X}}$.  We also note $e$ the neutral element of $\mathcal{X}$ such
that $x\groupop e=e \groupop x= e$ for all $x\in\mathcal{X}$.  Throughout this
paper we focus on positive definite function. Let $\mathcal{Y}$ be a complex
separable Hilbert space. A function $f:\mathcal{X}\to\mathcal{Y}$ is positive
definite if for all $N\in\mathbb{N}$ and all $y\in\mathcal{Y}$,
\begin{dmath}
    \label{eq:positive_definite} \sum_{i,j=1}^N\inner*{y_i,
    f\left(x_j^{-1}\groupop x_i\right)y_j}_{\mathcal{Y}}\ge 0
\end{dmath}
for all sequences $(y_i)_{i\in\mathbb{N}_N^*}\in\mathcal{Y}^N$ and all sequences
$(x_i)_{i\in\mathbb{N}_N^*}\in\mathcal{X}^N$. If $\mathcal{Y}$ is real we add
the assumption that $f(x^{-1})=f(x)^*$ for all $x\in\mathcal{X}$

\subsection{Even and odd functions}
Let $\mathcal{X}$ be a \ac{LCA} group and $\mathbb{K}$ be a field viewed as an
additive group. We say that a function $f:\mathcal{X}\to\mathbb{K}$ is even if
for all $x\in\mathcal{X}$, $f(x)=f\left(\inv{x}\right)$ and odd if
$f(x)=-f\left(\inv{x}\right)$. The definition can be extended to
operator-valued functions.
\begin{definition}[Even and odd operator-valued function on a \ac{LCA} group]
    Let $\mathcal{X}$ be a measured \ac{LCA} group and $\mathcal{Y}$ be a
    Hilbert space, and $\mathcal{L}(\mathcal{Y})$ the space of bounded linear
    operators from $\mathcal{Y}$ to itself viewed as an additive group. A
    function $f:\mathcal{X}\to\mathcal{L}(\mathcal{Y})$ is (weakly) even if for
    all $x\in\mathcal{X}$ and all $y$, $y'\in\mathcal{Y}$,
    $\inner{y,f\left(\inv{x}\right)y'}_{\mathcal{Y}} =
    \inner{y,f(x)y'}_{\mathcal{Y}}$ and (weakly) odd if
    $\inner{y,f\left(\inv{x}\right)y'}_{\mathcal{Y}} =
    -\inner{y,f(x)y'}_{\mathcal{Y}}$.
\end{definition}
It is easy to check that if $f$ is odd then
$\int_{\mathcal{X}}\inner{y,f(x)y'}_{\mathcal{Y}}d\Haar(x)=0$.  Besides the
product of an even and an odd function is odd. Indeed for all $f$,
$g\in\mathcal{F}(\mathcal{X};\mathcal{L}(\mathcal{Y}))$, where $f$ is even and
$g$ odd. Define $h(x)=\inner{y,f(x)g(x)y'}$. Then we have
$h\left(\inv{x}\right) = \inner{y, f\left(\inv{x}\right)
g\left(\inv{x}\right)y'}_{\mathcal{Y}}
\hiderel{=}\inner{y,f(x)\left(-g(x)\right)y'}_{\mathcal{Y}} =-h(x)$.
\subsection{Characters}
\label{subsec:character} \acs{LCA} groups are central to the general definition
of Fourier Transform which is related to the concept of Pontryagin
duality~\citep{folland1994course}.  Let $(\mathcal{X}, \groupop)$ be a \ac{LCA}
group with $e$ its neutral element and the notation, $\inv{x}$, for the inverse
of $x \in \mathcal{X}$. A \emph{character} is a complex continuous homomorphism
$\omega:\mathcal{X}\to\mathbb{U}$ from $\mathcal{X}$ to the set of complex
numbers of unit module $\mathbb{U}$. The set of all characters of $\mathcal{X}$
forms the Pontryagin \emph{dual  group} $\dual{\mathcal{X}}$. The dual group of
an \ac{LCA} group is an \ac{LCA} group so that we can endow
$\dual{\mathcal{X}}$ with a \say{dual} Haar measure noted $\dual{\Haar}$. Then
the dual group operation is defined by $(\omega_1 \groupop'
\omega_2)(x)=\omega_1(x)\omega_2(x) \hiderel{\in} \mathbb{U}$.  The Pontryagin
duality theorem states that $\dual{\dual{\mathcal{X}}}\cong \mathcal{X}$.
\acs{ie}~there is a canonical isomorphism between any \ac{LCA} group and its
double dual. To emphasize this duality the following notation is usually
adopted: $\label{eq:paringdef} \omega(x) = \pairing{x, \omega} \hiderel{=}
\pairing{\omega, x} \hiderel{=} x(\omega)$, where
$x\in\mathcal{X}\cong\dual{\dual{\mathcal{X}}}$ and
$\omega\in\dual{\mathcal{X}}$. The form $\pairing{\cdot,\cdot}$ defined in
\cref{eq:paringdef} is called (duality) pairing. Another important property
involves the complex conjugate of the pairing which is defined as
$\conj{\pairing{x, \omega}} = \pairing*{\inv{x}, \omega} \hiderel{=}
\pairing*{x, \inv{\omega}}$.
\begin{table}[t!]
    \caption{Classification of \acl{FT}s in terms of their domain and transform
    domain.}
    \label{tab:dual_and_pairing}
    \centering
    \begin{tabularx}{\textwidth}{cccX}
        \toprule
            \multicolumn{1}{c}{$\mathcal{X}=$} &
            \multicolumn{1}{c}{$\dual{\mathcal{X}}\cong$} &
            \multicolumn{1}{c}{Operation} & \multicolumn{1}{l}{Pairing} \\
        \cmidrule{1-4}
            $\mathbb{R}^d$ & $\mathbb{R}^d$ & $+$ & $\pairing{x,\omega} =
            \exp\left(\iu \inner{x, \omega}_2\right)$ \\ $\mathbb{R}^d_{*,+}$ &
            $\mathbb{R}^d$ & $\cdot$ & $\pairing{x,\omega} =\exp\left( \iu
            \inner{\log(x), \omega}_2 \right)$ \\ $(-c;+\infty)^d$ &
            $\mathbb{R}^d$ & $\odot$ & $\pairing{x,\omega} =\exp\left( \iu
            \inner{\log(x+c), \omega}_2 \right)$ \\
        \bottomrule
    \end{tabularx}
\end{table}
We notice that for any pairing depending of $\omega$, there exists a function
$h_{\omega}: \mathcal{X} \to \mathbb{R}$ such that $(x,\omega)= \exp(\iu
h_{\omega}(x))$ since any pairing maps into $\mathbb{U}$. Moreover,
$\pairing*{x \groupop \inv{z},\omega} = \omega(x)\omega\left(\inv{z}\right)
\hiderel{=}\exp\left(+\iu h_{\omega}\left(x\right)\right)\exp\left(+\iu
h_{\omega}\left(\inv{z}\right)\right) =\exp\left(+\iu
h_{\omega}\left(x\right)\right)\exp\left(-\iu h_{\omega}\left(z\right)\right)$.
\Cref{tab:dual_and_pairing} provides an explicit list of pairings for various
groups based on $\mathbb{R}^d$ or its subsets. The interested reader can refer
to~\citet{folland1994course} for a more detailed construction of \ac{LCA},
Pontryagin duality and \acl{FT}s on \ac{LCA}.

\subsection[The Fourier Transform]{The \acl{FT}}
For a function with values in a separable Hilbert space, $f\in
L^1(\mathcal{X},\Haar;\mathcal{Y})$, we denote $\FT{f}$ its \acf{FT} which is
defined by
\begin{dmath*}
        \forall \omega \in \dual{\mathcal{X}},\enskip \FT{f}(\omega)
        \hiderel{=}\int_{\mathcal{X}} \conj{\pairing{x,\omega}}f(x)d\Haar(x).
\end{dmath*}
The \acf{IFT} of a function $g\in L^1(\dual{\mathcal{X}},\dual{\Haar};
\mathcal{Y})$ is noted $\IFT{g}$ defined by $\forall x \in \mathcal{X},\enskip
\IFT{g}(x) \hiderel{=}\int_{\dual{\mathcal{X}}} \pairing{x,\omega}g(\omega)
d\dual{\Haar}(\omega)$, We also define the flip operator $\mathcal{R}$ by
$(\mathcal{R}f)(x) \colonequals f\left(\inv{x}\right)$.
\begin{theorem}[Fourier inversion]
    \label{th:fourier_inversion} Given a measure $\Haar$ defined on
    $\mathcal{X}$, there exists a unique suitably normalized dual measure
    $\dual{\Haar}$ on $\dual{\mathcal{X}}$ such that for all $f \in
    L^1(\mathcal{X}, \Haar;\mathcal{Y})$ and if $\FT{f} \in
    L^1(\dual{\mathcal{X}}, \dual{\Haar}; \mathcal{Y})$ we have
    \begin{dmath}
        \label{fourier-l1} f(x) \hiderel{=} \int_{\dual{\mathcal{X}}}
        \pairing{x, \omega} \FT{f}(\omega) d\dual{\Haar}(\omega) \condition{for
        $\Haar$-almost all $x\in \mathcal{X}$.}
    \end{dmath}
    \acs{ie}~such that
    $(\mathcal{R}\mathcal{F}\FT{f})(x)=\mathcal{F}^{-1}\FT{f}(x)=f(x)$ for
    $\Haar$-almost all $x\in\mathcal{X}$. If $f$ is continuous this relation
    holds for all $x\in\mathcal{X}$.
\end{theorem}
Thus when a Haar measure $\Haar$ on $\mathcal{X}$ is given, the measure on
$\dual{\mathcal{X}}$ that makes \cref{th:fourier_inversion} true is called the
dual measure of $\Haar$, noted $\dual{\Haar}$. Let $c\in\mathbb{R}_*$ If
$c\Haar$ is the measure on $\mathcal{X}$, then $c^{-1}\dual{\Haar}$ is the dual
measure on $\dual{\mathcal{X}}$. Hence one must replace $\dual{\Haar}$ by
$c^{-1}\dual{\Haar}$ in the inversion formula to compensate. Whenever
$\dual{\Haar}=\Haar$ we say that the Haar measure is self-dual. For the
familiar case of a scalar-valued function $f$ on the \ac{LCA} group
$(\mathbb{R}^d, +)$, we have for all $\omega\in
\dual{\mathcal{X}}=\mathbb{R}^d$
\begin{dmath}
    \label{fourier-R-plus}
    \FT{f}(\omega)
    =\int_{\mathcal{X}} \conj{\pairing{x,\omega}}f(x)d\Haar(x)
    \hiderel{=}\int_{\mathbb{R}^d} \exp(-\iu \inner{x,\omega}_2)f(x) d\Leb(x),
\end{dmath}
the Haar measure being here the Lebesgue measure. Notice that the normalization
factor of $\dual{\Haar}$ on $\dual{\mathcal{X}}$ depends on the measure $\Haar$
on $\mathcal{X}$ \emph{and} the duality pairing. For instance let
$\mathcal{X}=(\mathbb{R}^d, +)$. If one endow $\mathcal{X}$ with the
Lebesgue measure as the Haar measure, the Haar measure on the dual is defined
for all $\mathcal{Z}\in\mathcal{B}(\mathbb{R}^d)$ by
\begin{dmath*}
    \Haar(\mathcal{Z})\hiderel{=}\Leb(\mathcal{Z}),
    \quad\text{and}\quad
    \dual{\Haar}(\mathcal{Z})\hiderel{=}\frac{1}{(2\pi)^d}\Leb(\mathcal{Z}),
\end{dmath*}
in order to have $\mathcal{F}^{-1}\FT{f}=f$. If one use the cleaner equivalent
pairing $\pairing{x,\omega}=\exp(2\iu\pi \inner{x, \omega}_2)$ rather than
$\pairing{x,\omega}=\exp(\iu \inner{x,\omega}_2)$, then
$\dual{\Haar}(\mathcal{Z})=\Leb(\mathcal{Z})$.  The pairing
$\pairing{x,\omega}=\exp(2\iu\pi \inner{x,\omega}_2)$ looks more attractive in
theory since it limits the messy factor outside the integral sign and make the
Haar measure self-dual. However it is of lesser use in practice since it yields
additional unnecessary computation when evaluating the pairing.  Hence for
symmetry reason on $(\mathbb{R}^d, +)$ and reduce computations we settle with
the Haar measure on $\mathbb{R}^d$ groups (additive and multiplicative) defined
as $\dual{\Haar}(\mathcal{Z}) = \Haar(\mathcal{Z}) \hiderel{=}
{\sqrt{2\pi}}^{-d} \Leb(\mathcal{Z})$.  We conclude this subsection by recalling
the injectevity property of the \acl{FT}.
\begin{corollary}[\acl{FT} injectivity]
    Given $\mu$ and $\nu$ two measures, if $\FT{\mu}=\FT{\nu}$ then $\mu=\nu$.
    Moreover given two functions $f$ and $g\in
    L^1(\mathcal{X},\Haar;\mathcal{Y})$ if $\FT{f}=\FT{g}$ then $f=g$
\end{corollary}

%%%%%%%%%%%%%%%%%%%%%%%%%%%%%%%%%%%%%%%%%%%%%%%%%%%%%%%%%%%%%%%%%%%%%%%%%%%%%%%

\section{Proofs}
In this section we give the proofs of our contributions stated in the main body
of the paper.
\subsection{Construction}
\subsubsection{Proof of \texorpdfstring{\cref{lm:C_characterization}}{Lemma %
\ref{lm:C_characterization}}}
\begin{proof}
    For any function $f$ on $(\mathcal{X},\groupop)$ define the flip operator
    $\mathcal{R}$ by $(\mathcal{R}f)(x) \colonequals f\left(\inv{x}\right)$.
    For any shift invariant $\mathcal{Y}$-Mercer kernel and for all
    $\delta\in\mathcal{X}$,
    $K_e(\delta)=K_e\left(\inv{\delta}\right)^\adjoint$. Indeed from the
    definition of a shift-invariant kernel,
    $K_e\left(\inv{\delta}\right)=K\left(\inv{\delta},e\right)
    \hiderel{=}K\left(e,\delta\right)
    \hiderel{=}K\left(\delta,e\right)^\adjoint
    \hiderel{=}K_e\left(\delta\right)^\adjoint$.
    \paragraph{}
    Item 1: taking the \acl{FT} yields,
    $\inner{y',C(\omega)y}_{\mathcal{Y}}=\IFT{\inner{y',
    K_e(\cdot)y}_{\mathcal{Y}}}(\omega)
        %=\IFT{\inner{y', (\mathcal{R}K_e(\cdot))^\adjoint
        %y}_{\mathcal{Y}}}(\omega)
        %=\IFT{\inner{\mathcal{R}K_e(\cdot)y', y}_{\mathcal{Y}}}(\omega)
        %=\IFT{\mathcal{R}\inner{K_e(\cdot)y', y}_{\mathcal{Y}}}(\omega)
    \hiderel{=}\mathcal{R}\IFT{\inner{K_e(\cdot)y', y}_{\mathcal{Y}}}(\omega)
    \hiderel{=}\mathcal{R}\inner{C(\cdot)y',y}_{\mathcal{Y}}(\omega)
    =\inner*{y',C\left(\inv{\omega}\right)^\adjoint y}_{\mathcal{Y}}$.  Hence
    $C(\omega)=C\left(\inv{\omega}\right)^\adjoint$. Suppose that $\mathcal{Y}$
    is a complex Hilbert space. Since for all $\omega\in\mathcal{\dual{X}}$,
    $C(\omega)$ is bounded and non-negative so $C(\omega)$ is self-adjoint.
    Besides we have $C(\omega)=C\left(\inv{\omega}\right)^\adjoint $ so $C$
    must be even.  Suppose that $\mathcal{Y}$ is a real Hilbert space. The
    \acl{FT} of a real valued function obeys
    $\FT{f}(\omega)=\conj{\FT{f}\left(\inv{\omega}\right)}$. Therefore since
    $C(\omega)$ is non-negative for all $\omega\in\dual{\mathcal{X}}$,
    $\inner{y', C(\omega)y} = \conj{\inner{y', C\left(\inv{\omega}\right)y}}
    \hiderel{=}\inner{y, C\left(\inv{\omega}\right)^* y'} \hiderel{=}\inner{y,
    C\left(\omega\right) y'}$.  Hence $C(\omega)$ is self-adjoint and thus $C$
    is even.
    \paragraph{}
    Item 2: simply, for all $y$, $y'\in\mathcal{Y}$, $\inner{y,
    C(\inv{\omega})y'}$ $=$ $\inner{y', C(\omega)y}$ thus $\IFT{\inner{y',
    K_e(\cdot)y}_{\mathcal{Y}}}(\omega)=\inner{y',
    C(\omega)y}\hiderel{=}\mathcal{R}\inner{y',
    C(\cdot)y}(\omega)=\mathcal{R}\IFT{\inner{y',
    K_e(\cdot)y}_{\mathcal{Y}}}(\omega) \hiderel{=} \FT{\inner{y',
    K_e(\cdot)y}_{\mathcal{Y}}}(\omega)$.
    \paragraph{}
    Item 3: from Item 2 we have
    $\IFT{\inner{y', K_e(\cdot)y}}$ $=$ $\mathcal{F}^{-1}\mathcal{R}{\inner{y',
    K_e(\cdot)y}}$. By injectivity of the \acl{FT}, $K_e$ is even. Since
    $K_e(\delta)=K_e(\inv{\delta})^\adjoint $, we must have
    $K_e(\delta)=K_e(\delta)^\adjoint $.
\end{proof}
\subsubsection{Proof of \texorpdfstring{\cref{pr:spectral}}{Proposition %
\ref{pr:spectral}}}
\begin{proof}
    This is a simple consequence of \cref{pr:inverse_ovk_Fourier_decomposition}
    and \cref{lm:C_characterization}. By taking $\inner{y',C(\omega)y} =
    \IFT{\inner{y', K_e(\cdot)y}}(\omega)=\FT{\inner{y', K_e(\cdot)y}}(\omega)$
    we can write the following equality concerning the \acs{OVK} signature
    $K_e$:
    % Suppose that $\mu$ is absolutely continuous \acs{wrt}~$d\omega$. Then for
    % all $\delta \in \mathcal{X}$ and for all $y,$ $y'$ in $\mathcal{Y}$
    $\inner{y', K_e(\delta)y}(\omega)=
    \int_{\dual{\mathcal{X}}}\conj{\pairing{\delta, \omega}}\inner{y',
    C(\omega)y}d\dual{\Haar}(\omega)
    \hiderel{=}\int_{\dual{\mathcal{X}}}\conj{\pairing{\delta,
    \omega}}\inner*{y',
    \frac{1}{\rho(\omega)}C(\omega)y}\rho(\omega)d\dual{\Haar}(\omega)$.  It is
    always possible to choose $\rho(\omega)$ such that
    $\int_{\dual{\mathcal{X}}}\rho(\omega)d\dual{\Haar}(\omega)=1$. For
    instance choose
    \begin{dmath*}
        \rho(\omega)=
        \frac{\norm{C(\omega)}_{\mathcal{Y},
        \mathcal{Y}}}{\int_{\dual{\mathcal{X}}} \norm{C(\omega)}_{\mathcal{Y},
        \mathcal{Y}} d\dual{\Haar}(\omega)}
    \end{dmath*}
    Since for all $y$, $y'\in\mathcal{Y}$, $\inner{y',C(\cdot)y}\in
    L^1(\dual{\mathcal{X}},\dual{\Haar})$ and $\mathcal{Y}$ is a separable
    Hilbert space, by Pettis measurability theorem, $\int_{\dual{\mathcal{X}}}
    \norm{C(\omega)}_{\mathcal{Y},\mathcal{Y}} d\dual{\Haar}(\omega)$ is finite
    and so is $\norm{C(\omega)}_{\mathcal{Y},\mathcal{Y}}$ for all
    $\omega\in\dual{\mathcal{X}}$.  Therefore $\rho(\omega)$ is the density of
    a probability measure $\probability_{\dual{\Haar},\rho}$, \acs{ie}~conclude
    by taking $\probability_{\dual{\Haar},\rho}(\mathcal{Z}) =
    \int_{\mathcal{Z}}\rho(\omega)d\dual{\Haar}(\omega)$, for all
    $\mathcal{Z}\in\mathcal{B}(\dual{\mathcal{X}})$.
\end{proof}
\subsubsection{Proof of \texorpdfstring{\cref{cr:ORFF-kernel}}{Proposition %
\ref{cr:ORFF-kernel}}}
\begin{proof}
    %Let us first notice that for a given $D$, $\tilde{K}$ satisfies the
    %properties of a shift-invariant $\mathcal{Y}$-Mercer kernel.  Second, F
    Suppose that for all $y$, $y'\in\mathcal{Y}$, $\inner{y',
    A(\omega)y}\rho(\omega)=\IFT{\inner{y', K_e(\cdot)y}}(\omega)$ where $\rho$
    is a probability distribution (see \cref{pr:spectral}). From the strong law
    of large numbers $\frac{1}{D} \sum_{j=1}^D
    \conj{\pairing{x\groupop\inv{z},\omega_j}} A(\omega_j)
    \converges{\acs{asurely}}{D \to \infty} \expectation_{\dual{\Haar},
    \rho}[\conj{\pairing{x \groupop z^{-1}, \omega_j} }A(\omega)]$ where the
    integral converges in the weak operator topology. Then by
    \cref{pr:spectral} we recover $K_e$ when $D\to\infty$ since,
    $\expectation_{\dual{\Haar}, \rho}[\conj{\pairing{x \groupop z^{-1},
    \omega_j}}A(\omega)] = K_e(x\groupop\inv{z})$.
\end{proof}
\subsubsection{Proof of \texorpdfstring{\cref{cr:ORFF-map-kernel}}{%
Proposition~\ref{cr:ORFF-map-kernel}}}
\begin{proof}
    Let $(\omega_j)_{j=1}^D$ be a sequence of $D\in\mathbb{N}^*$
    \ac{iid}~random variables following the law
    $\probability_{\dual{\Haar},\rho}$. For all $x$, $z \in \mathcal{X}$ and
    all $y$, $y' \in \mathcal{Y}$,
    \begin{dmath*}
        \inner*{\tildePhi{\omega}(x)y,\tildePhi{\omega}(z)y'}_{\Vect_{j=1}^D
        \mathcal{Y}'}
        =\frac{1}{D}\inner*{\Vect_{j=1}^D \left(\pairing{x,
        \omega_j}B(\omega_j)^\adjoint y\right), \Vect_{j=1}^D\left(\pairing{z,
        \omega_j}B(\omega_j)^\adjoint y'\right)}
    \end{dmath*}
    By definition of the inner product in direct sum of Hilbert spaces,
    \begin{dmath*}
        \frac{1}{D}\inner*{\Vect_{j=1}^D \left(\pairing{x,
        \omega_j}B(\omega_j)^\adjoint y\right), \Vect_{j=1}^D\left(\pairing{z,
        \omega_j}B(\omega_j)^\adjoint y'\right)}
        = \frac{1}{D} \sum_{j=1}^D \inner*{y, \conj{\pairing{x,
        \omega_j}}B(\omega_j)\pairing{z, \omega_j}B(\omega_j)^\adjoint
        y'}_{\mathcal{Y}}
        \hiderel{=} \inner*{y, \left(\frac{1}{D} \sum_{j=1}^D
        \conj{\pairing{x\groupop \inv{z},
        \omega_j}}A(\omega_j)\right)y'}_{\mathcal{Y}},
    \end{dmath*}
    %With similar reasoning about plug-in Monte-Carlo estimator, we get the
    %proof.
    Eventually apply \cref{cr:ORFF-kernel} to obtain the convergence of the
    Monte-Carlo plug-in estimator to the true kernel $K$.
\end{proof}
\subsubsection{Proof of~%
\texorpdfstring{\cref{pr:fourier_feature_map}}{Proposition~%
\ref{pr:fourier_feature_map}}}
\begin{proof}
    For all $y$, $y'\in \mathcal{Y}$ and $x$, $z\in\mathcal{X}$,
    \begin{dmath*}
        \inner{y, \Phi_x^\adjoint \Phi_z y'}_{\mathcal{Y}} \hiderel{=}
        \inner{\Phi_x y, \Phi_z
        y'}_{L^2(\dual{\mathcal{X}},\dual{\mu};\mathcal{Y}')}
        \hiderel{=} \int_{\dual{\mathcal{X}}}\conj{\pairing{x,\omega}}\inner{y,
        B(\omega)\pairing{z,\omega}B(\omega)^\adjoint y'}d\dual{\mu}(\omega)
        = \int_{\dual{\mathcal{X}}}\conj{\pairing{x \groupop
        \myinv{z},\omega}}\inner{y, B(\omega)B(\omega)^\adjoint
        y'}d\dual{\mu}(\omega)
        \hiderel{=} \int_{\dual{\mathcal{X}}}\conj{\pairing{x \groupop
        \inv{z},\omega}}\inner{y,A(\omega)y'}d\dual{\mu}(\omega),
    \end{dmath*}
    which defines a $\mathcal{Y}$-Mercer according to
    \cref{pr:mercer_kernel_bochner} of~\citet{Carmeli2010}.
\end{proof}
\subsubsection{Proof of \texorpdfstring{\cref{pr:ORFF-map}}{Proposition~%
\ref{pr:ORFF-map}}}
\begin{proof}
    %Let us first notice that for a given $D$, $\tilde{K}$ satisfies the
    %properties of a shift-invariant $\mathcal{Y}$-Mercer kernel.  Second, F
    From the strong law of large numbers $\frac{1}{D} \sum_{j=1}^D
    \conj{\pairing{x\groupop\inv{z},\omega_j}} A(\omega_j)
    \converges{\acs{asurely}}{D \to \infty} \expectation_{\dual{\Haar},
    \rho}[\conj{\pairing{x \groupop z^{-1}, \omega_j} }A(\omega)]$ where the
    integral converges in the weak operator topology. Then by
    \cref{pr:mercer_kernel_bochner}, $\expectation_{\dual{\Haar},
    \rho}[\conj{\pairing{x \groupop z^{-1}, \omega_j}}A(\omega)]
    =K_e(x\groupop\inv{z})$.
\end{proof}
\subsubsection{Proof of~%
\texorpdfstring{\cref{pr:orff_defines_kernel}}{Proposition~%
\ref{pr:orff_defines_kernel}}}
\begin{proof}
    Apply \cref{pr:feature_operator} to $\tildePhi{\omega}$ considering the
    Hilbert space $\tildeH{\omega}$ to show that $\tildeK{\omega}$ is an
    \acs{OVK}. Then \cref{pr:kernel_signature} shows that $\tildeK{\omega}$ is
    shift-invariant since
    $\tildeK{\omega}(x,z)=\tildeK{\omega}_e\left(x\groupop z^{-1}\right)$.
    Since $B(\omega)$ is a bounded operator, $\widetilde{K}$ is
    $\mathcal{Y}$-Mercer because all the functions in the sum are continuous.
\end{proof}
\subsubsection{Proof of \texorpdfstring{\cref{pr:phitilde_phi_rel}}{%
Proposition~\ref{pr:phitilde_phi_rel}}}
\begin{proof}[of \cref{pr:cv_feature_map_1}]
    Since $(\omega_j)_{j=1}^D$ are \ac{iid}~random vectors, for all $y\in
    \mathcal{Y}$ and for all $y'\in\mathcal{Y}'$, $\inner{y, B(\cdot)y'}\in
    L^2(\dual{\mathcal{X}},\probability_{\dual{\Haar},\rho})$ and $g\in
    L^2(\dual{\mathcal{X}},\probability_{\dual{\Haar},\rho};\mathcal{Y}')$,
    \begin{dmath*}
        (\tildeW{\omega} \theta)(x)=\tildePhi{\omega}(x)^\adjoint
        \theta\hiderel{=}\frac{1}{D}\sum_{j=1}^D
        \conj{\pairing{x,\omega_j}}B(\omega_j)g(\omega_j), \qquad \omega_j
        \hiderel{\sim} \probability_{\dual{\Haar},\rho} \enskip
        \text{\ac{iid}~} \\
        \converges{\acs{asurely}}{D\to\infty}
        \int_{\dual{\mathcal{X}}} \conj{\pairing{x,\omega}} B(\omega) g(\omega)
        d\probability_{\dual{\Haar}, \rho}(\omega)
        \hiderel{=} (Wg)(x)
        \hiderel{\colonequals} \Phi_x^\adjoint g.
    \end{dmath*}
    from the strong law of large numbers.
\end{proof}

\begin{proof}[of \cref{pr:cv_feature_map_2}]
    Again, since $(\omega_j)_{j-1}^D$ are \ac{iid}~random vectors and $g\in
    L^2(\dual{\mathcal{X}},\probability_{\dual{\Haar},\rho};\mathcal{Y}')$,
    \begin{dmath*}
        \norm{\theta}^2_{\tildeH{\omega}}
        = \frac{1}{D}\sum_{j=1}^D\norm{g(\omega_j)}^2_{\mathcal{Y}'},
        \qquad \omega_j \hiderel{\sim} \probability_{\dual{\Haar}, \rho}
        \enskip \text{\ac{iid}~} \\ \converges{\acs{asurely}}{D\to\infty}
        \int_{\dual{\mathcal{X}}}
        \norm{g(\omega)}_{\mathcal{Y}'}^2 d\probability_{\dual{\Haar},
        \rho}(\omega)
        \hiderel{=} \norm{g}_{L^2\left(\dual{\mathcal{X}},
        \probability_{\dual{\Haar},\rho};
        \mathcal{Y}'\right)}^2.
    \end{dmath*}
    from the strong law of large numbers.
\end{proof}
\subsubsection{Proof of \texorpdfstring{\cref{pr:fourier_reg_ovk}}{%
Proposition~\ref{pr:fourier_reg_ovk}}}
\begin{proof}
    We first show how the \acl{FT} relates to the feature operator. Since
    $\mathcal{H}_K$ is embedded into $\mathcal{H}=L^2(\dual{\mathcal{X}},
    \probability_{\dual{\Haar},\rho}; \mathcal{Y}')$ by means of the feature
    operator $W$, we have for all $f\in\mathcal{H}_k$, for all
    $f\in\mathcal{H}$ and for all $x\in\mathcal{X}$
    \begin{dgroup*}
        \begin{dmath*}
            \FT{\IFT{f}}(x)
            \hiderel{=} \int_{\dual{\mathcal{X}}} \conj{\pairing{x, \omega}}
            \IFT{f}(\omega) d\dual{\Haar}(\omega)
            = f(x)
        \end{dmath*}
        \begin{dmath*}
            (Wg)(x)
            \hiderel{=}\int_{\dual{\mathcal{X}}} \conj{\pairing{x,
            \omega}}\rho(\omega) B(\omega) g(\omega) d\dual{\Haar}(\omega)
            = f(x).
        \end{dmath*}
    \end{dgroup*}
    By injectivity of the \acl{FT},
    $\IFT{f}(\omega)=\rho(\omega)B(\omega)g(\omega)$. From
    \cref{pr:feature_operator} we have
    \begin{dmath*}
        \norm{f}^2_{K} = \inf \Set{\norm{g}^2_{\mathcal{H}} | \forall
        g\hiderel{\in}\mathcal{H}, \enskip Wg\hiderel{=}f}
        = \inf\Set{\int_{\dual{\mathcal{X}}}
        \norm{g(\omega)}^2_{\mathcal{Y}'}d\probability_{\dual{\Haar},\rho}
        (\omega) | \forall g\hiderel{\in}\mathcal{H},\enskip \IFT{f}
        \hiderel{=}\rho(\cdot)B(\cdot)g(\cdot)}.
    \end{dmath*}
    The pseudo inverse of the operator $B(\omega)$ -- noted $B(\omega)^\dagger$
    -- is the unique solution of the system
    $\IFT{f}(\omega)=\rho(\omega)B(\omega)g(\omega)$ \acs{wrt}~$g(\omega)$ with
    minimal norm\footnote{Note that since $B(\omega)$ is bounded the pseudo
    inverse of $B(\omega)$ is well defined for $\dual{\Haar}$-almost all
    $\omega$.}. Eventually, $\norm{f}^2_K = \int_{\dual{\mathcal{X}}}
    \frac{\norm{B(\omega)^\dagger
    \IFT{f}(\omega)}_{\mathcal{Y}}^2}{\rho(\omega)^2}
    d\probability_{\dual{\Haar}, \rho}(\omega)$ Using the fact that
    $\IFT{\cdot}=\mathcal{F}\mathcal{R}[\cdot]$ and
    $\mathcal{F}^2[\cdot]=\mathcal{R}[\cdot]$,
    \begin{dmath*}
        \norm{f}^2_K= \displaystyle\int_{\dual{\mathcal{X}}}
        \frac{\norm{\mathcal{R} \left[B(\cdot)^\dagger
        \rho(\cdot)\right](\omega)
        \FT{f}(\omega)}^2_{\mathcal{Y}}}{\rho(\omega)^2} d\dual{\Haar}(\omega)
        %= \displaystyle\int_{\dual{\mathcal{X}}}
        %\frac{\norm{B(\omega)^\dagger \rho(\omega)
        %\FT{f}(\omega)}^2_{\mathcal{Y}}}{\rho(\omega)^2} d\dual{\Haar}(\omega)
        = \displaystyle\int_{\dual{\mathcal{X}}}
        \frac{\inner{B(\omega)^\dagger \FT{f}(\omega),
        B(\omega)^\dagger\FT{f}(\omega)}_{\mathcal{Y}}}{\rho(\omega)}
        d\dual{\Haar}(\omega)
        = \displaystyle\int_{\dual{\mathcal{X}}}
        \frac{\inner{\FT{f}(\omega),
        A(\omega)^\dagger\FT{f}(\omega)}_{\mathcal{Y}}}{\rho(\omega)}
        d\dual{\Haar}(\omega).
    \end{dmath*}
\end{proof}
\subsection{Convergence with high probability of the ORFF estimator}
\label{subsec:concentration_proof}
We recall the notations $\delta=x\groupop z^{-1}$, for all $x$,
$z\in\mathcal{X}$,
$\tilde{K} (x,z) = {\tildePhi{\omega} (x)}^\adjoint \tildePhi{\omega} (z)$,
$\tilde{K}^j (x,z) = {\Phi_x (\omega_j)}^\adjoint \Phi_z (\omega_j)$, where
$\omega_j\sim\probability_{\dual{\Haar,\rho}}$ and $K_e (\delta)=K(x,z)$ and
$\tilde{K}_e(\delta)=\tilde{K}(x,z)$. For the sake of readabilty, we use
throughout the proof the quantities: $F (\delta)\colonequals\tilde{K} (x,z) - K
(x,z)$ and $F^j (\delta)\colonequals\frac{1}{D} \left(\tilde{K}^j (x,z) - K
(x,z)\right)$.  We also view $\mathcal{X}$ as a metric space endowed with the
distance $d_{\mathcal{X}}:\mathcal{X}\times\mathcal{X}\to\mathbb{R}_+$.
Compared to the scalar case, the proof follows the same scheme as the one
described in \citep{Rahimi2007, sutherland2015}, but we consider an operator
norm as measure of the error and therefore concentration inequality dealing
with these operator norm.  The main feature of \cref{pr:bound_approx_unbounded}
is that it covers the case of bounded \acs{ORFF} as well as unbounded
\acs{ORFF}. In the case of bounded \acs{ORFF}, a Bernstein inequality for
matrix concentration such that the one proved in \citet[Corollary
5.2]{Mackey2014} or the formulation of \citet{Tropp} recalled in
\citet{koltchinskii2013remark}~is suitable. However some kernels like the curl
and the divergence-free kernels do not have obvious bounded
$\norm{F^j}_{\mathcal{Y},\mathcal{Y}}$ but exhibit $F^j$ with subexponential
tails. Therefore, we use an operator Bernstein concentration inequality adapted
for random matrices with subexponential norms.
\subsubsection{Epsilon-net}
\label{subsec:epsilon-net}
Let $\mathcal{C}\subseteq\mathcal{X}$ be a compact subset of $\mathcal{X}$. Let
$\mathcal{D}_{\mathcal{C}} = \Set{x\groupop z^{-1} | x, z\in\mathcal{C} }$ with
diameter at most $2\abs{\mathcal{C}}$ where $\abs{\mathcal{C}}$ is the diameter
of $\mathcal{C}$. Since $\mathcal{C}$ is supposed compact, so is
$\mathcal{D}_{\mathcal{C}}$. Since $\mathcal{D}_{\mathcal{C}}$ is also a metric
space it is well known that a compact metric space is totally bounded. Thus it
is possible to find a finite $\epsilon$-net covering
$\mathcal{D}_{\mathcal{C}}$. We call $T=\mathcal{N}(\mathcal{D}_{\mathcal{C}},
r)$ the number of closed balls of radius $r$ required to cover
$\mathcal{D}_{\mathcal{C}}$. For instance if $\mathcal{D}_{\mathcal{C}}$ is a
subspace finite dimensional Banach space with diameter at most
$2\abs{\mathcal{C}}$ it is possible to cover the space with at most
$T={(4\abs{\mathcal{C}}/r)}^d$ balls of radius $r$
(see \citet[proposition 5]{cucker2001mathematical}).  Let us call
$\delta_i,i=1,\ldots,T$ the center of the $i$-th ball, also called anchor of
the $\epsilon$-net. Denote $L_{F}$ the Lipschitz constant of $F$. Let
$\norm{\cdot}_{\mathcal{Y},\mathcal{Y}}$ be the operator norm on
$\mathcal{L}(\mathcal{Y})$ (largest eigenvalue). We introduce the following
technical lemma.
\begin{lemma}\label{lm:error_decomposition}
    $\forall \delta \in \mathcal{D}_{\mathcal{C}}$, if
    \begin{dmath}
        L_{F}\le\frac{\epsilon}{2r}\label{condition1}
    \end{dmath}
    and
    \begin{dmath}
        \norm{F (\delta_i)}_{\mathcal{Y},\mathcal{Y}}
        \le\frac{\epsilon}{2}\condition{for all
        $i\in\mathbb{N}^*_T$}\label{condition2}
    \end{dmath}
    then $\norm{F(\delta)}_{\mathcal{Y},\mathcal{Y}} \leq \epsilon$.
\end{lemma}
\begin{proof}
    $\norm{F (\delta)}_{\mathcal{Y},\mathcal{Y}} = \norm{F (\delta) -
    F(\delta_i) + F(\delta_i)}_{\mathcal{Y},\mathcal{Y} }\le \norm{F(\delta) -
    F(\delta_i)}_{\mathcal{Y},\mathcal{Y}} +
    \norm{F(\delta_i)}_{\mathcal{Y},\mathcal{Y}}$ for all $0<i<T$. Using the
    Lipschitz continuity of $F$ we have $\norm{F(\delta) -
    F(\delta_i)}_{\mathcal{Y},\mathcal{Y}} \le d_{\mathcal{X}}(\delta,\delta_i)
    L_{F} \hiderel{\le} rL_{F}$ hence
    $\norm{F(\delta)}_{\mathcal{Y},\mathcal{Y}} \le rL_{F} +
    \norm{F(\delta_i)}_{\mathcal{Y},\mathcal{Y}} \hiderel{=} \frac{r\epsilon}{2
    r} + \frac{\epsilon}{2} \hiderel{=} \epsilon$.
\end{proof}
To apply the lemma, we must bound the Lipschitz constant of the operator-valued
function $F$ (\cref{condition1}) and
$\norm{F(\delta_i)}_{\mathcal{Y},\mathcal{Y}}$, for all $i=1, \ldots, T$ as
well (\cref{condition2}).
\subsubsection{Bounding the Lipschitz constant}
This proof is a slight generalization of \citet{minh2016operator} to arbitrary
metric spaces. It differ from our first approach \citep{brault2016random},
based on the proof of \citet{sutherland2015} which was only valid for a finite
dimensional input space $\mathcal{X}$ and imposed a twice differentiability
condition on the considered kernel.
\begin{lemma}
    \label{lm:LipschitzK}
    Let $H_\omega \in \mathbb{R}_+$ be the Lipschitz constant of
    $h_\omega(\cdot)$ and assume that $\int_{\dual{\mathcal{X}}} H_\omega
    \norm{A(\omega)}_{\mathcal{Y},\mathcal{Y}}d\probability_{\dual{\Haar},
    \rho}(\omega) < \infty$.  Then the operator-valued function
    $K_e:\mathcal{X}\to\mathcal{L}(\mathcal{Y})$ is Lipschitz with
    \begin{dmath}
        \norm{K_e(x) - K_e(z)}_{\mathcal{Y},\mathcal{Y}}\le
        d_{\mathcal{X}}(x,z) \int_{\dual{\mathcal{X}}} H_\omega
        \norm{A(\omega)}_{\mathcal{Y},\mathcal{Y}}d\probability_{\dual{\Haar},
        \rho}(\omega).
    \end{dmath}
\end{lemma}
\begin{proof}
    We use the fact that the cosine function is Lipschitz with constant $1$ and
    $h_{\omega}$ Lipschitz with constant $H_\omega$. For all $x$,
    $z\in\mathcal{X}$ we have
    \begin{dmath*}
        \norm{\tilde{K}_e(x) - K_e(z)}_{\mathcal{Y},\mathcal{Y}}
        = \norm{\int_{\dual{\mathcal{X}}} \left(\cos h_\omega(x) - \cos
        h_\omega(z)\right)A(\omega)d\probability_{\dual{\Haar},\rho}
        }_{\mathcal{Y},\mathcal{Y}}
        \le \int_{\dual{\mathcal{X}}} \abs{\cos h_\omega(x) - \cos
        h_\omega(z)}\norm{A(\omega)}_{\mathcal{Y},\mathcal{Y}}
        d\probability_{\dual{\Haar},\rho}
        \le \int_{\dual{\mathcal{X}}} \abs{h_\omega(x) -
        h_\omega(z)}\norm{A(\omega)}_{\mathcal{Y},\mathcal{Y}}
        d\probability_{\dual{\Haar},\rho}
        \le d_{\mathcal{X}}(x, z) \int_{\dual{\mathcal{X}}} H_{\omega}
        \norm{A(\omega)}_{\mathcal{Y},\mathcal{Y}}
        d\probability_{\dual{\Haar},\rho}
    \end{dmath*}
\end{proof}
In the same way, considering $\tilde{K}_e(\delta)=\frac{1}{D}\sum_{j=1}^D\cos
h_{\omega_j}(\delta)A(\omega_j)$, where
$\omega_j\sim\probability_{\dual{\Haar},\rho}$, we can show that $\tilde{K}_e$
is Lipschitz with $\norm{\tilde{K}_e(x) -
\tilde{K}_e(z)}_{\mathcal{Y},\mathcal{Y}} \le
d_{\mathcal{X}}(x,z)\frac{1}{D}\sum_{j=1}^DH_{\omega_j}
\norm{A(\omega_j)}_{\mathcal{Y},\mathcal{Y}}$.  Combining the Lipschitz
continuity of $\tilde{K}_e$ and $\tilde{K}$ (\cref{lm:LipschitzK}) we obtain
\begin{dmath*}
    \norm{F(x)-F(z)}_{\mathcal{Y},\mathcal{Y}}
    = \norm{\tilde{K}_e(x) - \tilde{K}_e(x) - \tilde{K}_e(z) +
    K_e(z)}_{\mathcal{Y}, \mathcal{Y}}
    \le \norm{\tilde{K}_e(x) -
    \tilde{K}_e(z)}_{\mathcal{Y}, \mathcal{Y}} + \norm{K_e(x) -
    K_e(z)}_{\mathcal{Y}, \mathcal{Y}}
    \le d_{\mathcal{X}}(x,z)\left(\int_{\dual{\mathcal{X}}} H_\omega
    \norm{A(\omega)}_{\mathcal{Y}, \mathcal{Y}}
    d\probability_{\dual{\Haar},\rho} +
    \frac{1}{D}\sum_{j=1}^DH_{\omega_j}\norm{A(\omega_j)}_{\mathcal{Y},
    \mathcal{Y}} \right)
\end{dmath*}
Taking the expectation yields $\expectation_{\dual{\Haar},\rho}\left[ L_F
\right] = 2 \int_{\dual{\mathcal{X}}} H_\omega
\norm{A(\omega)}_{\mathcal{Y},\mathcal{Y}}d\probability_{\dual{\Haar},\rho}
(\omega)$ Thus by Markov's inequality,
\begin{dmath}
    \probability_{\dual{\Haar},\rho}\set{(\omega_j)_{j=1}^D | L_F \ge \epsilon}
    \le \frac{\expectation_{\dual{\Haar},\rho}\left[ L_F \right]}{\epsilon}
    \hiderel{\le} \frac{2}{\epsilon} \int_{\dual{\mathcal{X}}} H_\omega
    \norm{A(\omega)}_{\mathcal{Y},\mathcal{Y}}
    d\probability_{\dual{\Haar},\rho}.  \label{eq:Lipschitz_constant}
\end{dmath}
\subsubsection[Bounding the error on a given anchor point]{Bounding $F$ on a
given anchor point $\delta_i$}
To bound $\norm{F(\delta_i)}_{\mathcal{Y}, \mathcal{Y}}$, Hoeffding inequality
devoted to matrix concentration \citep{Mackey2014}~can be applied. We prefer
here to turn to tighter and refined inequalities such as Matrix Bernstein
inequalities (\citet{sutherland2015}~also pointed that for the scalar case).
The first non-commutative (matrix) concentration inequalities are due to the
pioneer work of \citet{Ahls2002}, using bound on the moment generating
function. This gave rise to many applications
\citet{Tropp,oliveira2009concentration,koltchinskii2013remark}~ranging from
analysis of randomized optimization algorithm to analysis of random graphs and
generalization bounds usefull in machine learning.
%The following inequality has
%been proposed in \cite{koltchinskii2013remark}.
%\begin{theorem}[Bounded non-commutative Bernstein]\label{th:Bernstein1}
    %From Theorem 3 of \citet{koltchinskii2013remark}, consider a sequence
    %$(X_{j})_{j=1}^D$ of $D$ independent Hermitian $p \times p$ random matrices
    %acting on a finite dimensional Hilbert space $\mathcal{Y}$ that satisfy
    %$\expectation X_{j} = 0$, and suppose that there exist some constant $U \ge
    %\norm{X_{j}}_{\mathcal{Y},\mathcal{Y}}$ for each index $j$. Denote the
    %proxy bound on the matrix variance
    %\begin{dmath*}
        %V \succcurlyeq \sum_{j=1}^D \expectation X_j^2.
    %\end{dmath*}
    %Then, for all $\epsilon \geq 0$,
    %\begin{dmath*}
        %\probability\Set{\norm{\sum_{j=1}^D X_{j}}_{\mathcal{Y}, \mathcal{Y}}
        %\geq \epsilon } \leq p
        %\exp\left(-\frac{\epsilon^2}{2\norm{V}_{\mathcal{Y},\mathcal{Y}} +
        %2U\epsilon/3}\right)
    %\end{dmath*}
%\end{theorem}
The concentration inequatilty of \citet{koltchinskii2013remark} we used in our
original paper \citep{brault2016random}~has the default to grow linearly with
the dimension $p$ of the output space $\mathcal{Y}$. However if the evaluation
of the operator-valued kernel at two points yields a low-rank matrix, this
bound could be improved since only a few principal dimensions are relevant.
Moreover this bound cannot be used when dealing with operator-valued kernel
acting on infinite dimensional Hilbert spaces. Recent results of
\citet{minsker2011some} consider the notion of intrinsic dimension to avoid
this \say{curse of dimensionality} (see \cref{def:intdim} for the definition).
When $A$ is approximately low-rank (\acs{ie} many eigenvalues are small), or go
quickly to zero, the intrinsic dimension can be much lower than the
dimensionality.  Indeed, $1 \le\intdim(A) \hiderel{\le} \rank(A) \hiderel{\le}
\dim(A)$.
\begin{theorem}[Bounded non-commutative Bernstein with intrinsic dimension
\citep{minsker2011some, tropp2015introduction}]\label{th:Bernstein2}
    Consider a sequence ${(X_j)}_{j=1}^D$ of $D$ independent Hilbert-Schmidt
    self-adjoint random operators acting on a separable Hilbert $\mathcal{Y}$
    space that satisfy $\expectation X_j = 0$ for all $j\in\mathbb{N}^*_D$.
    Suppose that there exist some constant $U \ge
    2\norm{X_j}_{\mathcal{Y},\mathcal{Y}}$ almost surely for all
    $j\in\mathbb{N}^*_D$. Define a semi-definite upper bound for the the
    operator-valued variance $V \succcurlyeq \sum_{j=1}^D \expectation X_j^2$.
    Then for all $\epsilon \ge \sqrt{\norm{V}_{\mathcal{Y}, \mathcal{Y}}} +
    U/3$,
    \begin{dmath*}
        \probability\Set{\norm{\sum_{j=1}^D X_{j}}_{\mathcal{Y}, \mathcal{Y}}
        \ge \epsilon } \le 4\intdim(V)\exp\left(-\psi_{V,U}(\epsilon)\right)
    \end{dmath*}
    where $\psi_{V, U}(\epsilon)=\frac{\epsilon^2}{2\norm{V}_{\mathcal{Y},
    \mathcal{Y}} + 2 U \epsilon / 3}$
\end{theorem}
he concentration inequality is restricted to the case where $\epsilon \ge
\sqrt{\norm{V}_{\mathcal{Y}, \mathcal{Y}}} + U/3$ since the probability is
vacuous on the contrary. The assumption that $X_j$'s are Hilbert-Schmidt
operators comes from the fact that the product of two such operator yields a
trace-class operator, for which the intrinsic dimension is well defined.
\paragraph{}
However, to cover the general case including unbounded \acp{ORFF} like curl and
divergence-free \acp{ORFF}, we choose a version of Bernstein matrix
concentration inequality proposed in~\cite{koltchinskii2013remark} that allows
to consider matrices that are not uniformly bounded but have subexponential
tails.  In the following we use the notion of Orlicz norm to bound random
variable by their tail behavior rather than their value (see
\cref{def:orlicz}).  For the sake of simplicity, we now fix
$\psi(t)=\psi_1(t)=\exp(t)-1$. Although the Orlicz norm should be adapted to
the tail of the distribution of the random operator we want to quantify to
obtain the sharpest bounds.  We also introduce two technical lemmas related to
Orlicz norm. The first one relates the $\psi_1$-Orlicz norm to the moment
generating function ($\MGF$).
\begin{lemma}\label{lm:orlicz_mgf}
    Let $X$ be a random variable with a strictly monotonic moment-generating
    function. We have $\norm{X}_{\psi_1}^{-1}=\MGF_{\abs{X}}^{-1}(2)$.
\end{lemma}
\begin{proof}
    We have
    \begin{dmath*}
        \norm{X}_{\psi_1}=\inf \Set{C \hiderel{>} 0 \,\, | \,\,
        \expectation[\exp\left( \abs{X}/C \right)] \hiderel{\le} 2 }
        \hiderel{=} \frac{1}{\sup \Set{C \hiderel{>} 0 \,\, | \,\,
        \MGF_{\abs{X}}(C)\le 2 }}.
    \end{dmath*}
    $X$ has strictly monotonic moment-generating thus
    $C^{-1}=\MGF^{-1}_{\abs{X}}(2)$. Hence
    $\norm{X}_{\psi_1}^{-1}=\MGF^{-1}_{\abs{X}}(2)$.
\end{proof}
The second lemma gives the Orlicz norm of a positive constant.
\begin{lemma}
    If $a\in\mathbb{R}_+$ then $\norm{a}_{\psi_1} = \frac{a}{\ln(2)}<2a$.
    \label{lm:orlicz_cte}
\end{lemma}
\begin{proof}
    We consider $a$ as a positive constant random variable, whose \ac{MGF} is
    $\MGF_a(t)=\exp(at)$.  From \cref{lm:orlicz_mgf},
    $\norm{a}_{\psi_1}=\frac{1}{\MGF_X^{-1}(2)}$.  Then
    $\MGF^{-1}_{\abs{a}}(2)=\frac{\ln(2)}{\abs{a}}$, $a \neq 0$. If $a=0$ then
    $\norm{a}_{\psi_1}=0$ by definition of a norm. Thus $\norm{a}_{\psi_1} =
    \frac{a}{\ln(2)}$.
\end{proof}
We now turn our attention to \citet{minsker2011some}'s theorem to for unbounded
random variables.
\begin{theorem}[Unbounded non-commutative Bernstein with intrinsic dimension]
    \label{th:Bernstein3} Consider a sequence $(X_j)_{j=1}^D$ of $D$
    independent self-adjoint random operators acting on a finite dimensional
    Hilbert space $\mathcal{Y}$ of dimension $p$ that satisfy $\expectation X_j
    = 0$ for all $j\in\mathbb{N}^*_D$.  Suppose that there exist some constant
    $U \ge \norm{\norm{X_j}_{\mathcal{Y},\mathcal{Y}}}_{\psi}$ for all
    $j\in\mathbb{N}^*_D$. Define a semi-definite upper bound for the the
    operator-valued variance $V \succcurlyeq \sum_{j=1}^D \expectation X_j^2$.
    Then for all $\epsilon > 0$,
    \begin{dmath*}
        \probability\Set{\norm{\sum_{j=1}^D X_{j}}_{\mathcal{Y}, \mathcal{Y}}
        \ge \epsilon } \hiderel{\le}
        \begin{cases}
            2\intdim(V)\exp\left(-\frac{\epsilon^2}{2
            \norm{V}_{\mathcal{Y},\mathcal{Y}}\left(1 + \frac{1}{p}\right)}
            \right) r_{V}(\epsilon) \condition{$\epsilon \le
            \frac{\norm{V}_{\mathcal{Y},\mathcal{Y}}}{2U}\frac{1+1/p}{K(V,
            p)}$} \\
            2\intdim(V)\exp\left(-\frac{\epsilon}{4UK(V,
            p)}\right)r_{V}(\epsilon) \condition{otherwise.}
        \end{cases}
    \end{dmath*}
    where $K(V,p)=\log\left(16\sqrt{2}p\right)+\log\left(\frac{D
    U^2}{\norm{V}_{\mathcal{Y}, \mathcal{Y}}}\right)$ and $r_V(\epsilon)= 1 +
    \frac{3}{\epsilon^2\log^2(1 + \epsilon /
    \norm{V}_{\mathcal{Y},\mathcal{Y}})}$
\end{theorem}
Let $\psi=\psi_1$. To use \cref{th:Bernstein3}, we set $X_j=F^j(\delta_i)$. We
have indeed $\expectation_{\dual{\Haar},\rho}[F^j(\delta_i)] = 0$ since
$\tilde{K}(\delta_i)$ is the Monte-Carlo approximation of $K_e(\delta_i)$ and
the matrices $F^j(\delta_i)$ are self-adjoint. We assume we can bound all the
Orlicz norms of the $F^j(\delta_i)=\frac{1}{D}(\tilde{K}^j(\delta_i) -
K_e(\delta_i))$. In the following we use constants $u_i$ such that $u_i=D U$.
Using \cref{lm:orlicz_cte} and the sub-additivity of the
$\norm{\cdot}_{\mathcal{Y},\mathcal{Y}}$ and $\norm{\cdot}_{\psi_1}$ norm,
\begin{dmath*}
    u_i=2D\max_{1\le j\le
    D}\norm{\norm{F^j(\delta_i)}_{\mathcal{Y},\mathcal{Y}}}_{\psi_1}
    \hiderel{\le} 2\max_{1\le j\le
    D}\norm{\norm{\tilde{K}^j(\delta_i)}_{\mathcal{Y},\mathcal{Y}}}_{\psi_1} +
    2\norm{\norm{K_e(\delta_i)}_{\mathcal{Y},\mathcal{Y}}}_{\psi_1}
    <4\max_{1\le j\le
    D}\norm{\norm{A(\omega_j)}_{\mathcal{Y},\mathcal{Y}}}_{\psi_1} + 4
    \norm{K_e(\delta_i)}_{\mathcal{Y},\mathcal{Y}}
    \hiderel{=}4\left(\norm{\norm{A(\omega)}_{\mathcal{Y},\mathcal{Y}}}_{\psi_1}
    + \norm{K_e(\delta_i)}_{\mathcal{Y},\mathcal{Y}}\right)
\end{dmath*}
In the same way we defined the constants $v_i=DV$, $v_i=
D\sum_{j=1}^D\expectation_{\dual{\Haar,\rho}} F^j(\delta_i)^2
\hiderel{=}D\variance_{\dual{\Haar},\rho}\left[ \tilde{K}(\delta_i) \right]$
Then applying \cref{th:Bernstein3}, we get for all
$i\in\mathbb{N}^*_{\mathcal{N}(\mathcal{D}_{\mathcal{C}},r)}$ ($i$ is the index
of each anchor)
\begin{dmath*}
    \probability_{\dual{\Haar,\rho}}\Set{(\omega_j)_{j=1}^D |
    \norm{F(\delta_i)}_{\mathcal{Y}, \mathcal{Y}} \ge \epsilon } \le
    \begin{cases}
        4 \intdim(v_i)\exp\left(-D\frac{\epsilon^2}{2
        \norm{v_i}_{\mathcal{Y},\mathcal{Y}}\left(1 + \frac{1}{p}\right)}
        \right) r_{v_i/D}(\epsilon) \condition{$\epsilon \le
        \frac{\norm{v_i}_{\mathcal{Y},\mathcal{Y}}}{2u_i}\frac{1+1/p}{K(v_i,
        p)}$} \\
        4 \intdim(v_i)\exp\left(-D\frac{\epsilon}{4u_iK(v_i,
        p)}\right)r_{v_i/D}(\epsilon) \condition{otherwise.}
    \end{cases}
\end{dmath*}
with
\begin{dmath*}
    K(v_i,p)=\log\left(16 \sqrt{2}
    p\right)+\log\left(\frac{u_i^2}{\norm{v_i}_{\mathcal{Y},
    \mathcal{Y}}}\right)
\end{dmath*}
and
\begin{dmath*}
    r_{v_i/D}= 1 + \frac{3}{\epsilon^2\log^2(1 + D \epsilon /
    \norm{v_i}_{\mathcal{Y},\mathcal{Y}})}.
\end{dmath*}
To unify the bound on each anchor we
define two constant
\begin{dmath*}
    u = 4\left(\norm{\norm{A(\omega)}_{\mathcal{Y},\mathcal{Y}}}_{\psi_1} +
    \sup_{\delta\in\mathcal{D}_{\mathcal{C}}}
    \norm{K_e(\delta)}_{\mathcal{Y},\mathcal{Y}}\right)
    \hiderel{\ge} \max_{i=1,\hdots T} u_i
\end{dmath*}
and
\begin{dmath*}
    v = \sup_{\delta\in\mathcal{D}_{\mathcal{C}}}
    D\variance_{\dual{\Haar},\rho}\left[ \tilde{K}_e(\delta) \right]
    \hiderel{\ge} \max_{i=1,\hdots T} v_i.
\end{dmath*}
\subsubsection{Union Bound and examples}
Taking the union bound over the anchors yields
\begin{dmath}
    \probability_{\dual{\Haar,\rho}}\Set{(\omega_j)_{j=1}^D |
    \bigcup_{i=1}^{\mathcal{N}(\mathcal{D}_{\mathcal{C}}, r)}
    \norm{F(\delta_i)}_{\mathcal{Y}, \mathcal{Y}} \ge \epsilon
    } \le 4 \mathcal{N}(\mathcal{D}_{\mathcal{C}}, r) r_{v/D}(\epsilon)
    \intdim(v )
    \begin{cases}
        \exp\left(-D\frac{\epsilon^2}{2
        \norm{v}_{\mathcal{Y},\mathcal{Y}}\left(1 + \frac{1}{p}\right)}
        \right) \condition{$\epsilon \le
        \frac{\norm{v}_{\mathcal{Y},\mathcal{Y}}}{2u}\frac{1+1/p}{K(v,
        p)}$} \\
        \exp\left(-D\frac{\epsilon}{4uK(v,
        p)}\right)\condition{otherwise.}
    \end{cases}
    \label{eq:anchor_bound}
\end{dmath}
Hence combining \cref{eq:Lipschitz_constant} and \cref{eq:anchor_bound} gives
and summing up the hypothesis yields the following proposition
\begin{proposition}
    \label{pr:bound_approx_unbounded}
    Let $K:\mathcal{X}\times\mathcal{X}\to\mathcal{L}(\mathcal{Y})$ be a
    shift-invariant $\mathcal{Y}$-Mercer kernel, where $\mathcal{Y}$ is a
    finite dimensional Hilbert space of dimension $p$ and $\mathcal{X}$ a
    metric space. Moreover, let $\mathcal{C}$ be a compact subset of
    $\mathcal{X}$, $A:\dual{\mathcal{X}}\to\mathcal{L}(\mathcal{Y})$ and
    $\probability_{\dual{\Haar},\rho}$ a pair such that $\tilde{K}_e =
    \sum_{j=1}^D \cos{\pairing{\cdot,\omega_j}}A(\omega_j) \hiderel{\approx}
    K_e$ $\omega_j\sim\probability_{\dual{\Haar}, \rho}$ \acs{iid}.  Let
    $V(\delta) \succcurlyeq \variance_{\dual{\Haar},\rho} \tilde{K}_e(\delta)$,
    for all $\delta\in\mathcal{D}_{\mathcal{C}}$ and $H_\omega$ be the
    Lipschitz constant of the function $h: x\mapsto \pairing{x,\omega}$. If the
    three following constant exists
    \begin{dmath*}
        m \ge \int_{\dual{\mathcal{X}}} H_\omega
        \norm{A(\omega)}_{\mathcal{Y},\mathcal{Y}} d\probability_{\dual{\Haar},
        \rho} \hiderel{<} \infty
    \end{dmath*}
    and
    \begin{dmath*}
        u \ge 4\left(\norm{\norm{A(\omega)}_{\mathcal{Y},\mathcal{Y}}}_{\psi_1}
        + \sup_{\delta\in\mathcal{D}_{\mathcal{C}}}
        \norm{K_e(\delta)}_{\mathcal{Y},\mathcal{Y}}\right) \hiderel{<} \infty
    \end{dmath*}
    and
    \begin{dmath*}
        v \ge \sup_{\delta\in\mathcal{D}_{\mathcal{C}}} D
        \norm{V(\delta)}_{\mathcal{Y}, \mathcal{Y}} \hiderel{<} \infty.
    \end{dmath*}
    Define $p_{int}\ge \sup_{\delta\in\mathcal{D}_{\mathcal{C}}}
    \intdim(V(\delta))$ then for all $r\in\mathbb{R}_+^*$ and all
    $\epsilon\in\mathbb{R}_+^*$,
    \begin{dmath*}
        \label{eq:bound1}
        \probability_{\dual{\Haar,\rho}}\Set{(\omega_j)_{j=1}^D |
        \norm{\tilde{K}-K}_{\mathcal{C}\times\mathcal{C}} \ge \epsilon}
        \le 4\left(\frac{r m}{\epsilon} +
        p_{int} \mathcal{N}(\mathcal{D}_{\mathcal{C}},r) r_{v/D}(\epsilon)
        \\
        \begin{cases}
            \exp\left(-D\frac{\epsilon^2}{8
            v\left(1 + \frac{1}{p}\right)}
            \right) \condition{$\epsilon \le
            \frac{v}{u}\frac{1+1/p}{K(v,
            p)}$} \\
            \exp\left(-D\frac{\epsilon}{8uK(v,
            p)}\right)\condition{otherwise.}
        \end{cases}\right)
    \end{dmath*}
    where
    \begin{dmath*}
        K(v, p)=\log\left(16 \sqrt{2}
        p\right)+\log\left(\frac{u^2}{\norm{v}_{\mathcal{Y},
        \mathcal{Y}}}\right)
    \end{dmath*}
    and
    \begin{dmath*}
        r_{v/D}(\epsilon)=1 + \frac{3}{\epsilon^2\log^2(1 + D \epsilon /
        \norm{v}_{\mathcal{Y},\mathcal{Y}})}.
    \end{dmath*}
\end{proposition}
\begin{proof}
    Let $m=\int_{\dual{\mathcal{X}}} H_\omega
    \norm{A(\omega)}_{\mathcal{Y},\mathcal{Y}} d\probability_{\dual{\Haar},
    \rho}$. From \cref{lm:LipschitzK},
    $\probability_{\dual{\Haar},\rho}\Set{(\omega_j)_{j=1}^D | L_F \ge
    \frac{\epsilon}{2r}} \le \frac{4 r m}{\epsilon}$. Thus from
    \cref{lm:error_decomposition}, for all $r\in\mathbb{R}_+^*$,
    \begin{dmath*}
        \probability_{\dual{\Haar,\rho}}\Set{(\omega_j)_{j=1}^D |
        \sup_{\delta\in\mathcal{D}_{\mathcal{C}}}
        \norm{F(\delta)}_{\mathcal{Y}, \mathcal{Y}} \ge \epsilon
        } \hiderel{\le} \\
        \probability_{\dual{\Haar},\rho}\Set{(\omega_j)_{j=1}^D | L_F \ge
        \frac{\epsilon}{2r}} +
        \probability_{\dual{\Haar,\rho}}\Set{(\omega_j)_{j=1}^D |
        \bigcup_{i=1}^{\mathcal{N}(\mathcal{D}_{\mathcal{C}}, r)}
        \norm{F(\delta_i)}_{\mathcal{Y}, \mathcal{Y}} \ge \epsilon }
        = 4\frac{r m}{\epsilon} + 4 \mathcal{N}(\mathcal{D}_{\mathcal{C}}, r)
        r_{v/D}(\epsilon) \intdim(v )
        \begin{cases}
            \exp\left(-D\frac{\epsilon^2}{8
            \norm{v}_{\mathcal{Y},\mathcal{Y}}\left(1 + \frac{1}{p}\right)}
            \right) \condition{$\epsilon \le
            \frac{\norm{v}_{\mathcal{Y},\mathcal{Y}}}{u}\frac{1+1/p}{K(v,
            p)}$} \\
            \exp\left(-D\frac{\epsilon}{8uK(v,
            p)}\right)\condition{otherwise.}
        \end{cases}
    \end{dmath*}
\end{proof}
With minor modifications we can obtain a second inequality for the case where
the random operators $A(\omega_j)$ are bounded almost surely. This second bound
with more restrictions on $A$ has the advantage of working in infinite
dimension as long as $A(\omega_j)$ is a Hilbert-Schmidt operator.
\begin{proposition}
    \label{pr:bound_approx_bounded}
    Let $K:\mathcal{X}\times\mathcal{X}\to\mathcal{L}(\mathcal{Y})$ be a
    shift-invariant $\mathcal{Y}$-Mercer kernel, where $\mathcal{Y}$ is a
    Hilbert space and $\mathcal{X}$ a metric space. Moreover, let $\mathcal{C}$
    be a compact subset of $\mathcal{X}$,
    $A:\dual{\mathcal{X}}\to\mathcal{L}(\mathcal{Y})$ and
    $\probability_{\dual{\Haar},\rho}$ a pair such that $\tilde{K}_e =
    \sum_{j=1}^D \cos{\pairing{\cdot,\omega_j}}A (\omega_j) \hiderel{\approx}
    K_e$, $\omega_j\sim\probability_{\dual{\Haar}, \rho}$ \acs{iid}.  where
    $A(\omega_j)$ is a Hilbert-Schmidt operator for all $j \in \mathbb{N}^*_D$.
    Let $\mathcal{D}_{\mathcal{C}}=\mathcal{C} \groupop \mathcal{C}^{-1}$ and
    $V (\delta) \succcurlyeq\variance_{\dual{\Haar},\rho} \tilde{K}_e
    (\delta)$, for all $\delta\in\mathcal{D}_{\mathcal{C}}$ and $H_\omega$ be
    the Lipschitz constant of the function $h: x\mapsto \pairing{x,\omega}$. If
    the three following constant exists
    \begin{dmath*}
        m \ge\int_{\dual{\mathcal{X}}} H_{\omega}
        \norm{A (\omega)}_{\mathcal{Y},\mathcal{Y}}
        d\probability_{\dual{\Haar}, \rho} \hiderel{<} \infty{}
    \end{dmath*}
    and
    \begin{dmath*}
        u \ge\esssup_{\omega\in\dual{\mathcal{X}}}
        \norm{A (\omega)}_{\mathcal{Y}, \mathcal{Y}} +
        \sup_{\delta\in\mathcal{D}_{\mathcal{C}}}
        \norm{K_e (\delta)}_{\mathcal{Y}, \mathcal{Y}} \hiderel{<} \infty{}
    \end{dmath*}
    and
    \begin{dmath*}
        v \ge\sup_{\delta\in\mathcal{D}_{\mathcal{C}}} D
        \norm{V (\delta)}_{\mathcal{Y}, \mathcal{Y}} \hiderel{<} \infty.
    \end{dmath*}
    define $p_{int} \ge \sup_{\delta\in\mathcal{D}_{\mathcal{C}}}
    \intdim\left(V(\delta)\right)$ then for all $r\in\mathbb{R}_+^*$ and all
    $\epsilon>\sqrt{\frac{v}{D}} +
    \frac{1}{3D}u$,
    \begin{dmath*}
        \probability_{\dual{\Haar,\rho}}\Set{(\omega_j)_{j=1}^D |
        \sup_{\delta\in\mathcal{D}_{\mathcal{C}}}
        \norm{F (\delta)}_{\mathcal{Y}, \mathcal{Y}} \ge\epsilon} \le~4
        \left(\frac{r m}{\epsilon} + p_{int}
        \mathcal{N} (\mathcal{D}_{\mathcal{C}}, r)
        \exp\left(-D\psi_{v,u} (\epsilon) \right)\right)
    \end{dmath*}
    where $\psi_{v,u}(\epsilon)=\frac{\epsilon^2}{2(v + u
    \epsilon / 3)}$.
\end{proposition}
When the covering number $\mathcal{N}(\mathcal{D}_{\mathcal{C}}, r)$ of the
metric space $\mathcal{D}_{\mathcal{C}}$ has an analytical form, it is
possible to optimize the bound over the radius $r$ of the covering balls. As an
example, we refine \cref{pr:bound_approx_unbounded} and
\cref{pr:bound_approx_bounded} in the case where $\mathcal{C}$ is a finite
dimensional Banach space.
\begin{corollary}
    Let $K:\mathcal{X}\times\mathcal{X}\to\mathcal{L}(\mathcal{Y})$ be a
    shift-invariant $\mathcal{Y}$-Mercer kernel, where $\mathcal{Y}$ is a
    finite dimensional Hilbert space of dimension $p$ and $\mathcal{X}$ a
    finite dimensional Banach space of dimension $d$. Moreover, let
    $\mathcal{C}$ be a closed ball of $\mathcal{X}$ centered at the origin of
    diameter $\abs{\mathcal{C}}$,
    $A:\dual{\mathcal{X}}\to\mathcal{L}(\mathcal{Y})$ and
    $\probability_{\dual{\Haar},\rho}$ a pair such that $\tilde{K}_e =
    \sum_{j=1}^D \cos\pairing{\cdot,\omega_j}A(\omega_j) \approx K_e$,
    $\omega_j\sim\probability_{\dual{\Haar}, \rho}$ \acs{iid}.  Let
    $\mathcal{D}_{\mathcal{C}}=\mathcal{C}\groupop\mathcal{C}^{-1}$ and
    $V(\delta) \succcurlyeq \variance_{\dual{\Haar},\rho} \tilde{K}_e(\delta)$,
    for all $\delta\in\mathcal{D}_{\mathcal{C}}$ Let $H_\omega$ be the
    Lipschitz constant of $h_{\omega}:x\mapsto \pairing{x, \omega}$. If the
    three following constant exists
    \begin{dmath*}
        m \ge \int_{\dual{\mathcal{X}}} H_{\omega}
        \norm{A(\omega)}_{\mathcal{Y},\mathcal{Y}} d\probability_{\dual{\Haar},
        \rho} \hiderel{<} \infty
    \end{dmath*}
    and
    \begin{dmath*}
        u \ge 4\left(\norm{\norm{A(\omega)}_{\mathcal{Y},\mathcal{Y}}}_{\psi_1}
        + \sup_{\delta\in\mathcal{D}_{\mathcal{C}}}
        \norm{K_e(\delta)}_{\mathcal{Y},\mathcal{Y}}\right) \hiderel{<} \infty
    \end{dmath*}
    and
    \begin{dmath*}
        v \ge \sup_{\delta\in\mathcal{D}_{\mathcal{C}}} D
        \norm{V(\delta)}_{\mathcal{Y}, \mathcal{Y}} \hiderel{<} \infty.
    \end{dmath*}
    Define $p_{int}\ge \sup_{\delta\in\mathcal{D}_{\mathcal{C}}}
    \intdim(V(\delta))$, then for all $0 < \epsilon \le m \abs{C}$,
    \begin{dmath*}
        \probability_{\dual{\Haar,\rho}}\Set{(\omega_j)_{j=1}^D |
        \norm{\tilde{K}-K}_{\mathcal{C}\times\mathcal{C}} \ge \epsilon}
        \le 8\sqrt{2} \left( \frac{m\abs{\mathcal{C}}}{\epsilon}
        \right)
        {\left(p_{int}r_{v/D}(\epsilon)\right)}^{\frac{1}{d + 1}}
        \begin{cases}
            \exp\left(-D\frac{\epsilon^2}{8
            v(d+1)\left(1 + \frac{1}{p}\right)}
            \right) \condition{$\epsilon \le
            \frac{v}{u}\frac{1+1/p}{K(v,
            p)}$} \\
            \exp\left(-D\frac{\epsilon}{8u(d+1)K(v,
            p)}\right)\condition{otherwise,}
        \end{cases}
    \end{dmath*}
    where $K(v, p)=\log\left(16 \sqrt{2}
    p\right)+\log\left(\frac{u^2}{v}\right) $ and $r_{v/D}(\epsilon)=1 +
    \frac{3}{\epsilon^2\log^2(1 + D \epsilon / v)}$.
\end{corollary}
\begin{proof}
    As we have seen in~\cref{subsec:epsilon-net}, suppose that $\mathcal{X}$ is
    a finite dimensional Banach space. Let $\mathcal{C}\subset\mathcal{X}$ be
    a closed ball centered at the origin of diameter $\abs{\mathcal{C}}=C$ then
    the difference ball centered at the origin
    \begin{dmath*}
        \mathcal{D}_{\mathcal{C}}
        = \mathcal{C}\groupop\mathcal{C}^{-1}
        \hiderel{=} \Set{x \groupop~z^{-1} | \norm{x}_{\mathcal{X}}
        \hiderel{\le} C / 2, \norm{z}_{\mathcal{X}} \hiderel{\le} C / 2, (x,
        z)\in\mathcal{X}^2} \hiderel{\subset} \mathcal{X}
    \end{dmath*}
    is closed and bounded, so compact and has diameter $\abs{C}=2C$. It is
    possible to cover it with $\log(\mathcal{N} (\mathcal{D}_{\mathcal{C}}, r))
    = d\log\left(\frac{2\abs{C}}{r} \right)$ closed balls of radius $r$.
    Pluging back into \cref{eq:bound1} yields
    \begin{dmath*}
        \probability_{\dual{\Haar,\rho}}\Set{(\omega_j)_{j=1}^D |
        \norm{\tilde{K}-K}_{\mathcal{C}\times\mathcal{C}} \ge \epsilon}
        \le 4\left(\frac{r m}{\epsilon} + p_{int} \left(\frac{2\abs{C}}{r}
        \right)^d r_{v/D}(\epsilon) \\
        \begin{cases}
            \exp\left(-D\frac{\epsilon^2}{8
            v\left(1 + \frac{1}{p}\right)}
            \right) \condition{$\epsilon \le
            \frac{v}{u}\frac{1+1/p}{K(v,
            p)}$} \\
            \exp\left(-D\frac{\epsilon}{8uK(v,
            p)}\right)\condition{otherwise.}
        \end{cases}\right)
    \end{dmath*}
    The right hand side of the equation has the form $ar+br^{-d}$ with $a =
    \frac{m}{\epsilon}$ and
    \begin{dmath*}
        b =  p_{int} {\left(2 \abs{\mathcal{C}}\right)}^d r_{v/D}(\epsilon)
        \begin{cases}
            \exp\left(-D\frac{\epsilon^2}{8
            v\left(1 + \frac{1}{p}\right)}
            \right) \condition{$\epsilon \le
            \frac{v}{u}\frac{1+1/p}{K(v,
            p)}$} \\
            \exp\left(-D\frac{\epsilon}{8uK(v,
            p)}\right)\condition{otherwise.}
        \end{cases}
    \end{dmath*}
    Following \cite{Rahimi2007, sutherland2015, minh2016operator}, we optimize
    over $r$.  It is a convex continuous function on $\mathbb{R}_+$ and achieve
    minimum at $r=\left(\frac{bd}{a}\right)^{\frac{1}{d+1}}$ and the minimum
    value is $r_*=a^{\frac{d}{d + 1}}b^{\frac{1}{d + 1}}\left( d^{\frac{1}{d +
    1}} + d^{-\frac{d}{d+1}} \right)$, hence
    \begin{dmath*}
        \probability_{\dual{\Haar,\rho}}\Set{(\omega_j)_{j=1}^D |
        \norm{\tilde{K}-K}_{\mathcal{C}\times\mathcal{C}} \ge \epsilon}
        \le C_d {\left( \frac{2m\abs{\mathcal{C}}}{\epsilon}
        \right)}^{\frac{d}{d + 1}}
        {\left(p_{int}r_{v/D}(\epsilon)\right)}^{\frac{1}{d + 1}}
        \begin{cases}
            \exp\left(-D\frac{\epsilon^2}{8
            v(d+1)\left(1 + \frac{1}{p}\right)}
            \right) \condition{$\epsilon \le
            \frac{v}{u}\frac{1+1/p}{K(v,
            p)}$} \\
            \exp\left(-D\frac{\epsilon}{8u(d+1)K(v,
            p)}\right)\condition{otherwise,}
        \end{cases}
        \le 8\sqrt{2} \left( \frac{m\abs{\mathcal{C}}}{\epsilon}
        \right)
        {\left(p_{int}r_{v/D}(\epsilon)\right)}^{\frac{1}{d + 1}}
        \begin{cases}
            \exp\left(-D\frac{\epsilon^2}{8
            v(d+1)\left(1 + \frac{1}{p}\right)}
            \right) \condition{$\epsilon \le
            \frac{v}{u}\frac{1+1/p}{K(v,
            p)}$} \\
            \exp\left(-D\frac{\epsilon}{8u(d+1)K(v,
            p)}\right)\condition{otherwise,}
        \end{cases}
    \end{dmath*}
    where $C_d = 4 \left( d^{\frac{1}{d + 1}} + d^{-\frac{d}{d+1}} \right)$.
    Eventually when $\mathcal{X}$ is a Banach space, the Lipschitz constant of
    $h_{\omega}$ is the supremum of the gradient $H_{\omega} =
    \sup_{\delta\in\mathcal{D}_{\mathcal{C}}} \norm{(\nabla h_{\omega})
    (\delta)}_{\dual{\mathcal{X}}}$.
\end{proof}
Following the same proof technique we obtain the second bound for bounded
\ac{ORFF}.
\begin{corollary}
    Let $K:\mathcal{X}\times\mathcal{X}\to\mathcal{L}(\mathcal{Y})$ be a
    shift-invariant $\mathcal{Y}$-Mercer kernel, where $\mathcal{Y}$ is a
    Hilbert space and $\mathcal{X}$ a finite dimensional Banach space of
    dimension $D$. Moreover, let $\mathcal{C}$ be a closed ball of
    $\mathcal{X}$ centered at the origin of diameter $\abs{\mathcal{C}}$,
    subset of $\mathcal{X}$, $A:\dual{\mathcal{X}}\to\mathcal{L}(\mathcal{Y})$
    and $\probability_{\dual{\Haar},\rho}$ a pair such that $\tilde{K}_e =
    \sum_{j=1}^D \cos{\pairing{\cdot,\omega_j}}A (\omega_j) \hiderel{\approx}
    K_e$, $\omega_j\sim\probability_{\dual{\Haar}, \rho}$ \acs{iid}.  where
    $A(\omega_j)$ is a Hilbert-Schmidt operator for all $j \in \mathbb{N}^*_D$.
    Let $\mathcal{D}_{\mathcal{C}}=\mathcal{C} \groupop \mathcal{C}^{-1}$ and
    $V (\delta) \succcurlyeq\variance_{\dual{\Haar},\rho} \tilde{K}_e (\delta)$
    for all $\delta\in\mathcal{D}_{\mathcal{C}}$    and $H_\omega$ be the
    Lipschitz constant of the function $h: x\mapsto \pairing{x,\omega}$. If the
    three following constant exists
    \begin{dmath*}
        m \ge\int_{\dual{\mathcal{X}}} H_{\omega}
        \norm{A (\omega)}_{\mathcal{Y},\mathcal{Y}}
        d\probability_{\dual{\Haar}, \rho} \hiderel{<} \infty{}
    \end{dmath*}
    and
    \begin{dmath*}
        u \ge\esssup_{\omega\in\dual{\mathcal{X}}}
        \norm{A (\omega)}_{\mathcal{Y}, \mathcal{Y}} +
        \sup_{\delta\in\mathcal{D}_{\mathcal{C}}}
        \norm{K_e (\delta)}_{\mathcal{Y}, \mathcal{Y}} \hiderel{<} \infty{}
    \end{dmath*}
    and
    \begin{dmath*}
        v \ge\sup_{\delta\in\mathcal{D}_{\mathcal{C}}} D
        \norm{V (\delta)}_{\mathcal{Y}, \mathcal{Y}} \hiderel{<} \infty.
    \end{dmath*}
    define $p_{int} \ge \sup_{\delta\in\mathcal{D}_{\mathcal{C}}}
    \intdim\left(V(\delta)\right)$ then for all $\sqrt{\frac{v}{D}} +
    \frac{u}{3D} < \epsilon < m\abs{\mathcal{C}}$,
    \begin{dmath*}
        \probability_{\dual{\Haar,\rho}}\Set{(\omega_j)_{j=1}^D |
        \sup_{\delta\in\mathcal{D}_{\mathcal{C}}}
        \norm{F (\delta)}_{\mathcal{Y}, \mathcal{Y}} \ge\epsilon} \le~8\sqrt{2}
        \left(\frac{m\abs{\mathcal{C}}}{\epsilon}\right) p_{int}^{\frac{1}{d +
        1}} \exp\left(-D\psi_{v,d,u} (\epsilon) \right)
    \end{dmath*}
    where $\psi_{v,d,u}(\epsilon)=\frac{\epsilon^2}{2(d+1)(v + u
    \epsilon / 3)}$.
\end{corollary}
\subsubsection{Proof of the ORFF estimator variance bound
(\texorpdfstring{\cref{pr:variance_bound}}{Proposition~%
\ref{pr:variance_bound}}).}
We use the notations $\delta = x \groupop z^{-1}$ for all $x, z
\in\mathcal{X}$, $\tilde{K}(x,z) = {\tildePhi{\omega}(x)}^\adjoint
\tildePhi{\omega}(z)$, $\tilde{K}^j(x, z) = {\Phi_x(\omega_j)}^\adjoint
\Phi_z(\omega_j)$ and $K_e(\delta)=K_e(x, z)$.
\begin{proof}
    Let $\delta\in\mathcal{D}_{\mathcal{C}}$ be a constant. From the definition
    of the variance of a random variable and using the fact that the
    $(\omega_j)_{j=1}^D$ are \ac{iid} random variables,
    \begin{dmath*}
        \variance_{\dual{\Haar}, \rho} \left[ \tilde{K}_e(\delta) \right]
        = \expectation_{\dual{\Haar}, \rho}\left[ \frac{1}{D} \sum_{j=1}^D
        \tilde{K}^j_e(\delta) - K_e(\delta) \right]^2
        \hiderel{=} \frac{1}{D^2} \expectation_{\dual{\Haar}, \rho}\left[
        \sum_{j=1}^D \tilde{K}^j_e(\delta) - K_e(\delta) \right]^2
        = \frac{1}{D} \expectation_{\dual{\Haar}, \rho} \left[
        \tilde{K}_e^j(\delta)^2 - \tilde{K}_e^j(\delta)K_e(\delta) -
        K_e(\delta)\tilde{K}_e^j(\delta) + K_e(\delta)^2 \right]
    \end{dmath*}
    From the definition of $\tilde{K}^j_e$, $\expectation_{\dual{\Haar}, \rho}
    \tilde{K}^j_e(\delta) = K_e(\delta)$, which leads to
    \begin{dmath*}
        \variance_{\dual{\Haar}, \rho} \left[ \tilde{K}_e(\delta) \right]
        = \frac{1}{D} \expectation_{\dual{\Haar}, \rho} \left[
        \tilde{K}^j_e(\delta)^2 - K_e(\delta)^2 \right]
    \end{dmath*}
    A trigonometric identity gives us $(\cos\pairing{\delta,
    \omega})^2=\frac{1}{2}\left( \cos\pairing{2\delta, \omega} +
    \cos\pairing{e, \omega} \right)$. Thus
    \begin{dmath*}
        \variance_{\dual{\Haar}, \rho} \left[ \tilde{K}_e(\delta) \right]
        = \frac{1}{2D} \expectation_{\dual{\Haar}, \rho} \left[ \left(
        \cos\pairing{2\delta, \omega} + \cos\pairing{e, \omega} \right)
        A(\omega)^2 - 2 K_e(\delta)^2 \right].
    \end{dmath*}
    Also,
    \begin{dmath*}
        \expectation_{\dual{\Haar}, \rho} \left[ \cos\pairing{2\delta, \omega}
        A(\omega)^2 \right]
        = \expectation_{\dual{\Haar}, \rho}\left[ \cos\pairing{2\delta, \omega}
        A(\omega) \right] \expectation_{\dual{\Haar}, \rho}\left[ A(\omega)
        \right] + \covariances_{\dual{\Haar}, \rho}\left[ \cos\pairing{2\delta,
        \omega} A(\omega), A(\omega) \right]
        = K_e(2\delta) \expectation_{\dual{\Haar}, \rho}\left[ A(\omega)
        \right] + \covariances_{\dual{\Haar}, \rho}\left[ \cos\pairing{2\delta,
        \omega} A(\omega), A(\omega) \right]
    \end{dmath*}
    Similarly we obtain
    \begin{dmath*}
        \expectation_{\dual{\Haar}, \rho}\left[ \cos\pairing{e, \omega}
        A(\omega)^2 \right] = K_e(e)\expectation_{\dual{\Haar}, \rho}\left[
        A(\omega) \right] + \covariances_{\dual{\Haar}, \rho}\left[
        \cos\pairing{e, \omega} A(\omega), A(\omega) \right]
    \end{dmath*}
    Therefore
    \begin{dmath*}
        \variance_{\dual{\Haar}, \rho} \left[ \tilde{K}_e(\delta) \right]
        = \frac{1}{2D} \left( \left( K_e(2\delta) + K_e(e) \right)
        \expectation_{\dual{\Haar}, \rho}\left[ A(\omega) \right] -
        2K_e(\delta)^2 + \covariances_{\dual{\Haar}, \rho}\left[
        \left(\cos\pairing{2\delta, \omega} + \cos\pairing{e, \omega}\right)
        A(\omega), A(\omega) \right]\right)
        = \frac{1}{2D} \left( \left( K_e(2\delta) + K_e(e) \right)
        \expectation_{\dual{\Haar}, \rho}\left[ A(\omega) \right] -
        2K_e(\delta)^2 + \covariances_{\dual{\Haar}, \rho}\left[
        \left(\cos\pairing{\delta, \omega}\right)^2
        A(\omega), A(\omega) \right]\right) \\
        \preccurlyeq \frac{1}{2D} \left( \left( K_e(2\delta) + K_e(e) \right)
        \expectation_{\dual{\Haar}, \rho}\left[ A(\omega) \right] -
        2 K_e(\delta)^2 + \variance_{\dual{\Haar}, \rho}\left[
        A(\omega) \right]\right)
    \end{dmath*}
\end{proof}
\subsection{Learning}
\subsubsection{Proof of \texorpdfstring{\cref{th:representer}}{Theorem~%
\ref{th:representer}}}
\begin{proof}
    Since $f(x)=K_x^*f$, the optimization problem reads
    \begin{dmath*}
        f_{\seq{s}} = \argmin_{f\in\mathcal{H}_K}
        \frac{1}{N}\displaystyle\sum_{i=1}^N c(K_{x_i}^\adjoint f, y_i) +
        \frac{\lambda}{2}\norm{f}^2_{K}
    \end{dmath*}
    Let $W_{\seq{s}}:\mathcal{H}_K\to\Vect_{i=1}^N\mathcal{Y}$ be the
    restriction\footnote{$W_{\seq{s}}$ is sometimes called the sampling or
    evaluation operator as in \citet{minh2016unifying}. However we prefer
    calling it \say{restriction operator} as in \citet{rosasco2010learning}
    since $W_{\seq{s}}f$ is the restriction of $f$ to the points in $\seq{s}$.}
    linear operator defined as $W_{\seq{s}}f = \Vect_{i=1}^N K_{x_i}^\adjoint
    f$, with $K_{x_i}^\adjoint:\mathcal{H}_K\to\mathcal{Y}$ and
    $K_{x_i}:\mathcal{Y}\to\mathcal{H}_K$. Let
    $Y=\vect_{i=1}^Ny_i\in\mathcal{Y}^N$. We have
    $\inner{Y,W_{\seq{s}}f}_{\Vect_{i=1}^N\mathcal{Y}} =
    \sum_{i=1}^N\inner{y_i, K_{x_i}^\adjoint f}_{\mathcal{Y}}
    \hiderel{=}\sum_{i=1}^N\inner{K_{x_i} y_i, f}_{\mathcal{H}_K}$.  Thus the
    adjoint operator $W_{\seq{s}}^\adjoint :
    \Vect_{i=1}^N\mathcal{Y}\to\mathcal{H}_K$ is $W_{\seq{s}}^\adjoint
    Y=\sum_{i=1}^NK_{x_i} y_i$, and the operator $W_{\seq{s}}^* W_{\seq{s}} :
    \mathcal{H}_K \to \mathcal{H}_K$ is $W_{\seq{s}}^\adjoint W_{\seq{s}}f =
    \sum_{i=1}^NK_{x_i} K_{x_i}^\adjoint f$.  Let $\mathfrak{R}_{\lambda}(f,
    \seq{s}) = \underbrace{\frac{1}{N}\displaystyle\sum_{i=1}^N c(f(x_i),
    y_i)}_{=\mathfrak{R}_c} + \frac{\lambda}{2}\norm{f}^2_{K}$. To ensure that
    $\mathfrak{R}_{\lambda}$ has a global minimizer we need the following
    technical lemma (which is a consequence of the Hahn-Banach theorem for
    lower-semicontimuous functional, see~\citet{kurdila2006convex}).
    \begin{lemma}
        \label{lm:strongly_convex_is_coercive} Let $\mathfrak{R}$ be a proper,
        convex, lower semi-continuous functional, defined on a Hilbert space
        $\mathcal{H}$. If $\mathfrak{R}$ is strongly convex, then
        $\mathfrak{R}$ is coercive.
    \end{lemma}
    %\begin{proof}
        %Consider the convex function $G(f)\colonequals
        %\mathfrak{R}(f)-\lambda\norm{f}^2$, for some $\lambda>0$. Since
        %$\mathfrak{R}$ is by assumption proper, lower semi-continuous and
        %strongly convex with parameter $\lambda$, $G$ is proper, lower
        %semi-continuous and convex.  Thus Hahn-Banach theorem apply, stating
        %that $G$ is bounded by below by an affine functional. \acs{ie}~there
        %exists $f_0$ and $f_1\in\mathcal{H}$ such that
        %\begin{dmath*}
            %G(f)\ge G(f_0) + \inner{f - f_0, f_1} \condition{for all
            %$f\in\mathcal{H}$.}
        %\end{dmath*}
        %Then substitute the definition of $G$ to obtain
        %\begin{dmath*}
            %\mathfrak{R}(f)\ge \mathfrak{R}(f_0) +
            %\lambda\left(\norm{f}-\norm{f_0}\right) + \inner{f - f_0, f_1}.
        %\end{dmath*}
        %By the Cauchy-Schwartz inequality, $\inner{f, f_1}\ge -
        %\norm{f}\norm{f_1}$, thus
        %\begin{dmath*}
            %\mathfrak{R}(f)\ge \mathfrak{R}(f_0) +
            %\lambda\left(\norm{f}-\norm{f_0}\right) - \norm{f}\norm{f_1} -
            %\inner{f_0, f_1},
        %\end{dmath*}
        %which tends to infinity as $f$ tends to infinity. Hence $\mathfrak{R}$
        %is coercive
    %\end{proof}
    Since $c$ is proper, lower semi-continuous and convex by assumption, thus
    the term $\mathfrak{R}_c$ is also proper, lower semi-continuous and convex.
    Moreover the term $\frac{\lambda}{2}\norm{f}^2_{K}$ is strongly convex.
    Thus $\mathfrak{R}_{\lambda}$ is strongly convex. Apply
    \cref{lm:strongly_convex_is_coercive} to obtain the coercivity of
    $\mathfrak{R}_{\lambda}$, and then Mazur-Schauder's theorem (see
    \citet{gorniewicz1999topological, kurdila2006convex}) to show that
    $\mathfrak{R}_{\lambda}$ has a unique minimizer and is attained. Then let
    $\mathcal{H}_{K, \seq{s}}=\Set{\sum_{j=1}^{N}K_{x_j}u_j| \forall
    (u_i)_{i=1}^{N} \in\mathcal{Y}^{N}}$.  For $f\in\mathcal{H}_{K,
    \seq{s}}^\perp$\footnote{$\mathcal{H}_{K,
    \seq{s}}^\perp\oplus\mathcal{H}_{K, \seq{s}}=\mathcal{H}_K$ because
    $W_{\seq{s}}$ is bounded.}, the operator $W_{\seq{s}}$ satisfies $\inner{Y,
    W_{\seq{s}}f}_{\Vect_{i=1}^N\mathcal{Y}} =
    \inner{\underbrace{f}_{\in\mathcal{H}_{K, \seq{s}}^\perp},
    \underbrace{\sum_{i=1}^{N}K_{x_i}V^\adjoint y_i}_{\in\mathcal{H}_{K,
    \seq{s}}}}_{\mathcal{H}_K} \hiderel{=} 0$ for all sequences
    $(y_i)_{i=1}^N$, since $y_i\in\mathcal{Y}$.  Hence,
    \begin{dmath}
        \label{eq:null1} (f(x_i))_{i=1}^{N}=0
    \end{dmath}
    In the same way, $\sum_{i=1}^{N}\inner{K_{x_i}^* f, u_i}_{\mathcal{Y}}
    \hiderel{=} \inner{\underbrace{f}_{\in\mathcal{H}_{K, \seq{s}}^\perp},
    \underbrace{\sum_{j=1}^{N}K_{x_j}u_j}_{\in\mathcal{H}_{K,
    \seq{s}}}}_{\mathcal{H}_K} \hiderel{=} 0$.  for all sequences
    $(u_i)_{i=1}^{N}\in\mathcal{Y}^{N}$. As a result,
    \begin{dmath}
        \label{eq:null2} (f(x_i))_{i=1}^{N}=0.
    \end{dmath}
    Now for an arbitrary $f\in\mathcal{H_K}$, consider the orthogonal
    decomposition $f = f^{\perp} + f^{\parallel}$, where $f^{\perp} \in
    \mathcal{H}_{K, \seq{s}}^\perp$ and $f^{\parallel} \in \mathcal{H}_{K,
    \seq{s}}$. Then since $\norm{f^{\perp} + f^{\parallel}}_{\mathcal{H}_K}^2
    =\norm{f^{\perp}}_{\mathcal{H}_K}^2 +
    \norm{f^{\parallel}}_{\mathcal{H}_K}^2$, \cref{eq:null1} and
    \cref{eq:null2} shows that if $\lambda > 0$, clearly then
    $\mathfrak{R}_{\lambda}(f, \seq{s}) = \mathfrak{R}_{\lambda}\left(f^{\perp}
    + f^{\parallel}, \seq{s}\right) \hiderel{\ge}
    \mathfrak{R}_{\lambda}\left(f^{\parallel}, \seq{s}\right)$ The last
    inequality holds only when $\norm{f^{\perp}}_{\mathcal{H}_K}=0$, that is
    when $f^{\perp}=0$. As a result since the minimizer of
    $\mathfrak{R}_{\lambda}$is unique and attained, it must lies in
    $\mathcal{H}_{K, \seq{s}}$.
\end{proof}
\subsubsection{Proof of \texorpdfstring{\cref{th:orff_representer}}{Theorem~%
\ref{th:orff_representer}}}
\label{subsubsec:proof_feature_equiv}
\begin{proof}
    Since $\tildeK{\omega}$ is an operator-valued kernel, from
    \cref{th:representer}, \cref{eq:argmin_RKHS_rand} has a solution of the
    form
    \begin{dmath*}
        \widetilde{f}_{\seq{s}} = \sum_{i=1}^{N} \tildeK{\omega}(\cdot,
        x_i)u_i \hiderel{=} \sum_{i=1}^N
        \tildePhi{\omega}(\cdot)^\adjoint \tildePhi{\omega}(x_i)u_i \hiderel{=}
        \tildePhi{\omega}(\cdot)^\adjoint \underbrace{\left(\sum_{i=1}^{N}
        \tildePhi{\omega}(x_i) u_i \right)}_{= \theta \in \left( \Ker
        \tildeW{\omega}\right)^\perp \subset \tildeH{\omega}},
    \end{dmath*}
    where $u_i \hiderel{\in} \mathcal{Y}$ and $x_i\in\mathcal{X}$. Let
    $\theta_{\seq{s}}=\argmin_{\theta\in\left(\Ker
    \tildeW{\omega}\right)^\perp}
    \frac{1}{N}\sum_{i=1}^Nc\left(\tildePhi{\omega}(x_i)^\adjoint \theta,
    y_i\right) + \frac{\lambda}{2} \norm{\tildePhi{\omega}(\cdot)^\adjoint
    \theta}^2_{\tildeK{\omega}}$.  Since $\theta\in(\Ker
    \tildeW{\omega})^\perp$ and $W$ is an isometry from $(\Ker
    \tildeW{\omega})^\perp\subset \tildeH{\omega}$ onto
    $\mathcal{H}_{\tildeK{\omega}}$, we have
    $\norm{\tildePhi{\omega}(\cdot)^\adjoint\theta}^2_{\tildeK{\omega}} =
    \norm{\theta}^2_{\tildeH{\omega}}$. Hence
        $\theta_{\seq{s}}=\argmin_{\theta\in\left(\Ker
        \tildeW{\omega}\right)^\perp}
        \frac{1}{N}\sum_{i=1}^Nc\left(\tildePhi{\omega}(x_i)^\adjoint \theta,
        y_i\right) + \frac{\lambda}{2}\norm{\theta}^2_{\tildeH{\omega}}$
    Finding a minimizer $\theta_{\seq{s}}$ over $\left(\Ker
    \tildeW{\omega}\right)^\perp$ is not the same as finding a minimizer over
    $\tildeH{\omega}$. Although in both cases Mazur-Schauder's theorem
    guarantees that the respective minimizers are unique, they might not be the
    same. Since $\tildeW{\omega}$ is bounded, $\Ker \tildeW{\omega}$ is closed,
    so that we can perform the decomposition $\tildeH{\omega}=\left(\Ker
    \tildeW{\omega}\right)^\perp\oplus \left(\Ker \tildeW{\omega}\right)$. Then
    clearly by linearity of $W$ and the fact that for all
    $\theta^{\parallel}\in\Ker \tildeW{\omega}$,
    $\tildeW{\omega}\theta^{\parallel}=0$, if $\lambda > 0$ we have
    $\theta_{\seq{s}}=\argmin_{\theta\in\tildeH{\omega}}
    \frac{1}{N}\sum_{i=1}^Nc\left(\tildePhi{\omega}(x_i)^\adjoint \theta,
    y_i\right) + \frac{\lambda}{2}\norm{\theta}^2_{\tildeH{\omega}}$
    Thus $\theta_{\seq{s}} =\argmin_{\substack{\theta^{\perp}\in\left(\Ker
    \tildeW{\omega}\right)^\perp, \\ \theta^{\parallel}\in\Ker
    \tildeW{\omega}}} \frac{1}{N} \sum_{i=1}^N c\left(\left( \tildeW{\omega}
    \theta^{\perp} \right)(x) + \underbrace{\left( \tildeW{\omega}
    \theta^{\parallel} \right)(x)}_{=0 \enskip \text{for all}\enskip
    \theta^{\parallel} }, y_i\right) +
    \frac{\lambda}{2}\norm{\theta^\perp}^2_{\tildeH{\omega}} +
    \underbrace{\frac{\lambda}{2} \norm{\theta^{\parallel} }^2_{%
    \tildeH{\omega}} }_{=0 \enskip\text{only if}\enskip \theta^{\parallel}=0}$
    Thus $\theta_{\seq{s}}=\argmin_{\theta^{\perp}\in\left(\Ker
    \tildeW{\omega}\right)^\perp} \frac{1}{N}\sum_{i=1}^Nc\left( \left(
    \tildeW{\omega} \theta^{\perp} \right)(x), y_i \right) + \frac{\lambda}{2}
    \norm{\theta^\perp}^2_{\tildeH{\omega}}$ Hence minimizing over $\left(\Ker
    \tildeW{\omega}\right)^\perp$ or $\widetilde{\mathcal{H}}{\omega}$ is the
    same when $\lambda > 0$.  Eventually,
    % Eventually for any outcome of $\omega_j \sim
    % \probability_{\dual{\Haar},\rho}$ \ac{iid},
    $\theta_{\seq{s}}=\argmin_{\theta\in\tildeH{\omega}}
    \frac{1}{N}\sum_{i=1}^Nc\left(\tildePhi{\omega}(x_i)^\adjoint \theta,
    y_i\right) + \frac{\lambda}{2}\norm{\theta}^2_{\tildeH{\omega}}$.
\end{proof}



\bibliography{jmlr-orff}

\end{document}

